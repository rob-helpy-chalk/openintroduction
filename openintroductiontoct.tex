\documentclass[twoside, openright]{book}

%glossary and indexing stuff
\usepackage[toc, nopostdot, numberedsection]{glossaries}
\usepackage{datatool}
%\newglossary*{catstatements}{Chapter \ref{chap:catstatements} Key Terms}
%\newglossary*{catsyllogisms}{Key Terms}
\renewcommand{\glsnamefont}[1]{\makefirstuc{#1}}
\makeglossaries

%General packages
\usepackage{answers}
\usepackage{textcomp}
\usepackage{anyfontsize}	
\usepackage{geometry}
\usepackage{url}
\usepackage{changepage}
\usepackage{syntonly}
\usepackage{enumitem}
\usepackage{turnstile}
\usepackage[normalem]{ulem}
\usepackage{fixltx2e}	
\usepackage{wasysym}
\usepackage{tocloft}
\usepackage{fancyhdr}
\usepackage{fancyref}
\usepackage{etoolbox}
\usepackage[utf8]{inputenc}
%\usepackage{amsthm} for some reason, this conflicts with the fitch.sty part of openlogic.sty. 

%%% Bibstuff
\usepackage[authordate,autocite=inline,backend=biber, natbib]{biblatex-chicago}
\bibliography{tex/z-openlogic}
% To typeset the bibliography, you need to run "biber --output-safechars openintroductiontoct" from the command line. You can't use the function within TeXworks, because it doen't have the --output-safechars flag. Without that flag, biber is unable to handle many characters used for Sanskrit words, like the n with a dot under it.




%%%  graphics packages %%%
\usepackage{tikz}
\usetikzlibrary{shapes,backgrounds,matrix,arrows,decorations,positioning,arrows.new,decorations.markings}
\usepgflibrary{arrows.new}
\usepackage{graphicx}
\usepackage{xcolor}


%%%    Table and figure packages %%%
\usepackage{float}
\usepackage[singlelinecheck=false, skip=0pt]{caption} %left aligns captions for tables, moves them closer to table. 
\usepackage{tabularx}
\usepackage{longtable}
\usepackage{tabu}
\usepackage[framemethod=1]{mdframed} 
\usepackage{wrapfig} %I used this to put a frame around tables.
\tabulinesep=.75ex
\usepackage{colortbl}
\floatstyle{plain} 
\restylefloat{figure}
\usepackage[export]{adjustbox}
\usepackage{multirow}
\usepackage{rotating}
\usepackage{booktabs}
\usepackage{array}
\newcolumntype{?}{!{\vrule width 1pt}}
	

%linking and bookmarks in the pdf.
\usepackage{hyperref}
\hypersetup{pdftex,colorlinks=true,allcolors=blue}
\usepackage{hypcap}	
	

\usepackage{openintroduction}
	
\pdfinfo{
  /Title (An Open Introduction to Logic)
  /Author (J. Robert Loftis, Cathal Woods, and P.D. Magnus)
  /Subject (An open access introductory textbook in logic and critical thinking)
  )
}

\begin{document}

%\label{showanswers} %uncomment this tag and typeset twice to show answers
%\label{blank_prob_set} %uncomment this tag and typeset twice to create a blank problem set sheet. Don't use with \label{showanswers} uncommented

\raggedright
\setlength{\parindent}{1em}
\setlength{\parskip}{1em}	
\tikzset{myarrowhead/.style={decoration={markings,mark=at position 1 with %
    {\arrow[scale=2,>=stealth]{>}}},postaction={decorate}}}



\frontmatter
\pagestyle{plain} %Says there are no running heads, only page numbers centered at the bottom. 
\include{tex/01-cover-ct}
\include{tex/02-frontmatter}	


{
\setlength{\parskip}{0em}
\cftpagenumbersoff{part}
%\cftpagenumbersoff{chapter}

\renewcommand{\cftpartpresnum}{\sf\Large\partname\ }
\tableofcontents
}

\include{tex/03-aboutthisbook}	
\include{tex/04-acknowledgements}



\mainmatter
\setlength{\parindent}{1em}
\pagestyle{headings} % puts the running heads back.
\label{full_version} %Include this label to make cross references work right when typesetting full text
\label{CTVersion}

%\part{Basic Concepts} \label{part:basic_concepts}
%\chapter{What Is \iflabelexists{CTVersion}{Critical Thinking?}{Logic?}}
\label{Chap:what_is_logic}
\markright{Ch. \ref{Chap:what_is_logic}: What Is \iflabelexists{CTVersion}{Critical Thinking?}{Logic?}}

\label{noexplanation}

% *****************************
% *		Introduction                      *
% ****************************

\section{Introduction}

\iflabelexists{CTVersion}{Critical thinking is a part of a more general field called logic.}{} Logic is a part of the study of human reason, the ability we have to think abstractly, solve problems, explain the things that we know, and infer new knowledge on the basis of evidence. Traditionally, logic has focused on the last of these items, the ability to make inferences on the basis of evidence. \iflabelexists{CTVersion}{Sometimes we study our ability to make inferences for the purpose of improving that ability in practical sitations. When we are doing logic for practical purposes like this, we say we are studying critical thinking.
}{}

\iflabelexists{CTVersion}{Making inferences}{This} is an activity you engage in every day. Consider, for instance, the game of Clue. (For those of you who have never played, Clue is a murder mystery game where players have to decide who committed the murder, what weapon they used, and where they were.) A player in the game might decide that the murder weapon was the candlestick by ruling out the other weapons in the game: the knife, the revolver, the rope, the lead pipe, and the wrench. This evidence lets the player know something they did not know previously, namely, the identity of the murderer.

\iflabelexists{CTVersion}{In logic and critical thinking,}{In logic,} we use the word ``argument'' to refer to the attempt to show that certain evidence supports a conclusion. This is very different from the sort of argument you might have when you are mad at someone, which could involve screaming and throwing things. We are going to use the word ``argument'' a lot in this book, so you need to get used to thinking of it as a name for a rational process, and not a word that describes what happens when people disagree.

A logical argument is structured to give someone a reason to believe some conclusion. Here is the argument about a game of Clue written out in a way that shows its structure. 


\label{argClue}
\begin{earg}
\item[P$_1$:] In a game of Clue, the possible murder weapons are the knife, the candlestick, the revolver, the rope, the lead pipe, and the wrench.
\item[P$_2$:] The murder weapon was not the knife.
\item[P$_3$:] The murder weapon was also not the revolver, the rope, the lead pipe, or the wrench.
\vspace{-.5em}
\item [] \rule{0.9\linewidth}{.5pt} 
\item[C:] Therefore, the murder weapon was the candlestick.
\end{earg} 

In the argument above, statements P$_1$--P$_3$ are the evidence. We call these the \emph{premises}. The word ``therefore'' indicates that the final statement, marked with a C, is the \emph{conclusion} of the argument. If you believe the premises, then the argument provides you with a reason to believe the conclusion. You might use reasoning like this purely in your own head, without talking with anyone else. You might wonder what the murder weapon is, and then mentally rule out each item, leaving only the candlestick. On the other hand, you might use reasoning like this while talking to someone else, to convince them that the murder weapon is the candlestick. (Perhaps you are playing as a team.) Either way the structure of the reasoning is the same. 

\newglossaryentry{logic}
{
name=logic,
description={the part of the study of reasoning that focuses on argument.}
}

We can define \textsc{\Gls{logic}}\label{def:logic} then more precisely as the part of the study of reasoning that focuses on argument. In more casual situations, we will follow ordinary practice and use the word ``logic'' to either refer to the business of studying human reason or the thing being studied, that is, human reasoning itself. While logic focuses on argument, other disciplines, like decision theory and cognitive science, deal with other aspects of human reasoning, like abstract thinking and problem solving more generally. Logic, as the study of argument, has been pursued for thousands of years by people from civilizations all over the globe. The initial motivation for studying logic is generally practical. Given that we use arguments and make inferences all the time, it only makes sense that we would want to learn to do these things better. \iflabelexists{CTVersion}{So the earliest attempts at logic could be classified as critical thinking.}{} Once people begin to study logic, however, they quickly realize that it is a fascinating topic in its own right. Thus the study of logic quickly moves from being a practical business to a theoretical endeavor people pursue for its own sake. 

\newglossaryentry{metareasoning}
{
name=metareasoning,
description={Using reasoning to study reasoning. See also \emph{metacognition}.}
}

\newglossaryentry{metacognition}
{
name=metacognition,
description={Thought processes that are applied to other thought processes See also \emph{metareasoning}.}
}

In order to study reasoning, we have to apply our ability to reason to our reason itself. This reasoning about reasoning is called \textsc{\gls{metareasoning}}\label{def:Metareasoning}. It is part of a more general set of processes called \textsc{\gls{metacognition}}\label{def:Metacognition}, which is just any kind of thinking about thinking. When we are pursing logic as a practical discipline, one important part of metacognition will be awareness of your own thinking, especially its weakness and biases, as it is occurring. More theoretical metacognition will be about attempting to understand the structure of thought itself. 


\newglossaryentry{content neutrality}
{
name=content neutrality,
description={the feature of the study of logic that makes it indifferent to the topic being argued about. If a method of argument is considered rational in one domain, it should be considered rational in any other domain, all other things being equal.}
}

Whether we are pursuing logical for practical or theoretical reasons, our focus is on argument. The key to studying argument is to set aside the subject being argued about and to focus on the \emph{way} it is argued \emph{for}. The section opened with an example that was about a game of Clue. However, the kind of reasoning used in that example was just the process of elimination. Process of elimination can be applied to any subject. Suppose a group of friends is deciding which restaurant to eat at, and there are six restaurants in town. If you could rule out five of the possibilities, you would use an argument just like the one above to decide where to eat. Because logic sets aside what an argument is about, and just looks at how it works rationally, logic is said to have \textsc{\gls{content neutrality}}. \label{def:content_neutrality} If we say an argument is good, then the same kind of argument applied to a different topic will also be good.  If we say an argument is good for solving murders, we will also say that the same kind of argument is good for deciding where to eat, what kind of disease is destroying your crops, or who to vote for. 

\newglossaryentry{formal logic}
{
name=formal logic,
description={A way of studying logic that achieves content neutrality by replacing parts of the arguments being studied with abstract symbols. Often this will involve the construction of full formal languages.}
}


When logic is studied for theoretical reasons, it typically is pursued as \textsc{\gls{formal logic}}. \label{def:Formal_logic} In formal logic we get content neutrality by replacing parts of the argument we are studying with abstract symbols. For instance, we could turn the argument above into a formal argument like this:

\label{argClueformal}
\begin{earg}
\item[P$_1$:] There are six possibilities: A, B, C, D, E, and F.
\item[P$_2$:] A is false.
\item[P$_3$:] B, D, E, and F are also false.
\vspace{-.5em}
\item [] \rule{0.6\linewidth}{.5pt} 
\item[C:]  $\therefore$ The correct answer is C.
\end{earg} 

Here we have replaced the concrete possibilities in the first argument with abstract letters that could stand for anything. We have also replaced the English word ``therefore'' with the symbol ``$\therefore$,'' which means therefore. This lets us see the formal structure of the argument, which is why it works in any domain you can think of. In fact, we can think of formal logic as the method for studying argument that uses abstract notation to identify the formal structure of argument.  Formal logic is closely allied with mathematics, and studying formal logic often has the sort of puzzle-solving character one associates with mathematics. \iflabelexists{full_version}{You will see this when we get to Part \ref{part:formal_logic}, which covers formal logic.}
	{\iflabelexists{part:CT}{}
	{\iflabelexists{part:cat_logic}{You will see this when we get to Parts \ref{part:cat_logic} and \ref{part:sent_logic}, which cover formal logic.}
{\iflabelexists{part:quant_logic}{You will see this when we get to Parts \ref{part:sent_logic} and \ref{part:quant_logic}, which cover formal logic.}}	
	{}	}}
 
\newglossaryentry{critical thinking}
{
name=critical thinking,
description={The use of metareasoning to improve our reasoning in practical situations. Sometimes the term is also used to refer to the results of this effort at self improvement, that is, reasoning in practical situations that has been sharpened by reflection and metareasoning.}
}

\newglossaryentry{critical thinker}
{
name=critical thinker,
description={A person who has both sharpened their reasoning abilities using metareasoning and deploys those sharpened abilities in real world situations..}
}

\newglossaryentry{informal logic}
{
name=informal logic,
description={The study of arguments given in ordinary language.}
}


\iflabelexists{CTVersion}{As we said before, when}{When} logic is studied for practical reasons, it is typically called critical thinking. We will define \textsc{\gls{critical thinking}}\label{def:Critical_Thinking}  narrowly as the use of metareasoning to improve our reasoning in practical situations.  Sometimes we will use the term ``critical thinking'' more broadly to refer to the results of this effort at self-improvement.  You are ``thinking critically'' when you reason in a way that has been sharpened by reflection and metareasoning. A \textsc{\gls{critical thinker}}\label{def:critical_thinker} someone who has both sharpened their reasoning abilities using metareasoning and deploys those sharpened abilities in real world situations.

Critical thinking is generally pursued as \textsc{\gls{informal logic}}, rather than formal logic. This means that we will keep arguments in ordinary language and draw extensively on your knowledge of the world to evaluate them. In contrast to the clarity and rigor of formal logic, informal logic is suffused with ambiguity and vagueness. There are problems  with multiple correct answers, or where reasonable people can disagree with what the correct answer is. This is because you will be dealing with reasoning in the real world, which is messy. \label{messiness_warning} \label{ver_var} \iflabelexists{part:CT}{You will learn more about this in the chapters on critical thinking Part \ref{part:CT}}{}
  

You can think of the difference between formal logic and informal logic as the difference between a laboratory science and a field science. \label{lab_vs._field_science} If you are studying, say, mice, you could discover things about them by running experiments in a lab, or you can go out into the field where mice live and observe them in their natural habitat.  Informal logic is the field science for arguments: you go out and study arguments in their natural habitats, like newspapers, courtrooms, and scientific journal articles. Like studying mice scurrying around a meadow, the process takes patience, and often doesn't yield clear answers but it lets you see how things work in the real world. Formal logic takes arguments out of their natural habitat and performs experiments on them to see what they are capable of. The arguments here are like lab mice. They are pumped full of chemicals and asked to perform strange tasks, as it were. They live lives very different than their wild cousins. Some of the arguments will wind up looking like the ``ob/ob mouse'', a genetically engineered obese mouse scientists use to study type II diabetes (See Figure \ref{fig:ob_ob_mouse}). These arguments will be huge, awkward, and completely unable to survive in the wild. But they will tell us a lot about the limits of logic as a process.

\begin{figure}
\begin{mdframed}[style=mytableclearbox]
\begin{center}
\includegraphics*[scale=.8]{img/Fatmouse}
\end{center}
\end{mdframed}
\caption{The ob/ob mouse (left), a laboratory mouse which has been genetically engineered to be obese, and an ordinary mouse (right). Formal logic, which takes arguments out of their natural environment, often winds up studying arguments that look like the ob/ob mouse. They are huge, awkward, and unable to survive in the wild, but they tell us a lot about the limits of logic as a process. Photo from \cite{WikimediaCommons2006}.}
\label{fig:ob_ob_mouse}
\end{figure}


\newglossaryentry{rhetoric}
{
name=rhetoric,
description={The study of effective persuasion.}
}


Our main goal in studying arguments is to separate the good ones from the bad ones. The argument about Clue we saw earlier is a good one, based on the process of elimination.  It is good because it leads to truth. If I've got all the premises right, the conclusion will also be right. The textbook \textit{Logic: Techniques of Formal Reasoning} \citep{Kalish1980} had a nice way of capturing the meaning of logic: ``logic is the study of virtue in argument.'' \label{virtue_in_argument} This textbook will accept this definition, with the caveat that an argument is virtuous if it helps us get to the truth.

Logic is different from \textsc{\gls{rhetoric}}, which is the study of effective persuasion. Rhetoric does not look at virtue in argument. It only looks at the power of arguments, regardless of whether they lead to truth. An advertisement might convince you to buy a new truck by having a gravelly voiced announcer tell you it is ``ram tough'' and showing you a picture of the truck on top of a mountain, where it no doubt actually had to be airlifted. This sort of persuasion is often more effective at getting people to believe things than logical argument, but it has nothing to do with whether the truck is really the right thing to buy. In this textbook we will only be interested in rhetoric to the extent that we need to learn to defend ourselves against the misleading rhetoric of others. \iflabelexists{part:CT}{The sections of this text on critical thinking will emphasize becoming aware of our biases and how others might use misleading rhetoric to exploit them. }{}This will not, however, be anything close to a full treatment of the study of rhetoric.


% ******************************************
% *		Statement, Argument, Premise, Conclusion  *
% ******************************************

\section{Statement, Argument, Premise, Conclusion}
\label{sec:SAPC}

\newglossaryentry{statement}
{
name=statement,
description={A unit of language that can be true or false.}
}

\iflabelexists{CTVersion}{So far we have defined logic as the study of argument and critical thinking as logic done for practical purposes.}{So far we have defined logic as the study of argument and outlined its relationship to related fields.} To go any further, we are going to need a more precise definition of what exactly an argument is. We have said that an argument is not simply two people disagreeing; it is an attempt to prove something using evidence. More specifically, an argument is composed of statements. In \iflabelexists{CTVersion}{logic and critical thinking}{logic}, we define a \textsc{\gls{statement}} \label{def:statement} as a unit of language that can be true or false. To put it another way, it is some combination of words or symbols that have been put together in a way that lets someone agree or disagree with it. All of the items below are statements.

\begin{enumerate}[label=(\alph*)]
\item \label{itm:t.rex_true}\emph{Tyrannosaurus rex} went extinct 65 million years ago. 
\item \label{itm:t.rex_false}\emph{Tyrannosaurus rex} went extinct last week.
\item \label{itm:t.rex_unknown}On this exact spot, 100  million years ago, a \emph{T. rex} laid a clutch of eggs. 
\item \label{itm:silly}George W. Bush is the king of Jupiter. 
\item \label{itm:moral}Murder is wrong. 
\item \label{itm:opinion1}Abortion is murder. 
\item \label{itm:opinion2}Abortion is a woman's right. 
\item \label{itm:opinion3}Lady Gaga is pretty.
\item \label{itm:definition}Murder is the unjustified killing of a person.
\item \label{itm:nonsense}The slithy toves did gyre and gimble in the wabe.
\item \label{itm:history}The murder of logician Richard Montague was never solved. 
\end{enumerate}

Because a statement is something that can be true \emph{or} false, statements include truths like \ref{itm:t.rex_true} and falsehoods like \ref{itm:t.rex_false}. A statement can also be something that that must either be true or false, but we don't know which, like \ref{itm:t.rex_unknown}. A statement can be something that is completely silly, like \ref{itm:silly}. Statements in logic include statements about morality, like \ref{itm:moral}, and things that in other contexts might be called ``opinions,'' like \ref{itm:opinion1} and \ref{itm:opinion2}. People disagree strongly about whether \ref{itm:opinion1} or \ref{itm:opinion2} are true, but it is definitely possible for one of them to be true. The same is true about \ref{itm:opinion3}, although it is a less important issue than \ref{itm:opinion1} and \ref{itm:opinion2}. A statement in logic can also simply give a definition, like \ref{itm:definition}.  Statements can include nonsense words like \ref{itm:nonsense}, because we don't really need to know what the statement is about to see that it is the sort of thing that can be true or false. All of this relates back to the content neutrality of the study of argument. The statements we study can be about dinosaurs, abortion, Lady Gaga, and even the history of logic itself, as in statement \ref{itm:history}, which is true.

% You have now put the normative/descriptive distinction only in the CT version of the text}

\iflabelexists{CTVersion}{
\newglossaryentry{descriptive statement}
{
name=descriptive statement,
description={A statement which talks about the way the world is. These are contrasted with normative statements, which talk about the way the world should be.}
}


\newglossaryentry{normative statement}
{
name=normative statement,
description={A statement which talks about the way the world should be. These are contrasted with descriptive statements, which talk about the way the world is.}
}



So you see that statement is a broad category that can include all kinds of things that you might not normally lump together. One important division within the class of statements is the distinction between statements which talk about the way the world \textit{is} and statements which talk about the way the world should be. Statements that are about the way the world is are called \textsc{\glspl{descriptive statement}}\label{def:descriptive_statement}. These include true descriptions of the world, like statement \ref{itm:t.rex_true} and false descriptions like \label{itm:silly}. Statements that are about the way the world should be are called \textsc{\glspl{normative statement}}\label{def:normative_statement}. These include statements with an explicit ``should'' in them, like ``you shouldn't chew with your mouth open.'' They also include statements that contain an implicit ``should'' like statements  \ref{itm:opinion1} and \ref{itm:opinion2}. If you assert that abortion is murder, you are saying that people \textit{shouldn't} have abortions and that abortions \textit{should} be illegal. Conversely, you say that abortion is a woman's right, you are saying it \textit{should} be legal. All of this is different than describing what people actually do or what is actually legal. 

Some kinds of statements can be interpreted normatively or discriptively. Definitions like statement \ref{itm:definition} are like that. It could describe the way people actually use a word or announce that people should use a certain way. Sometimes the same definition can be used different ways. Dictionaries, for instance, are written to be descriptive: they simply summarize how a word has been used up to this point. However, when dictionaries are actually used, they are generally used normatively. We use them to correct people's use of words, to say that they \textit{should} use words differently.}{}

We are treating statements primarily as units of language or strings of symbols, and most of the time the statements you will be working with will just be words printed on a page. However, it is important to remember that statements are also what philosophers call ``speech acts.'' They are actions people take when they speak (or write). If someone makes a statement they are typically telling other people that they believe the statement to be true, and will back it up with evidence if asked to. When people make statements, they always do it in a context---they make statements at a place and a time with an audience. Often the context statements are made in will be important for us, so when we give examples, statements, or arguments we will sometimes include a description of the context. When we do that, we will give the context in \textit{italics.} See Figure \ref{fig:statements_and_context} for examples. \label{context_marker} For the most part, the context for a statement or argument will be important in the chapters on critical thinking, when we are pursing the study of logic for practical reasons. In the chapters on formal logic, context is less important, and we will be more likely to skip it. 

\begin{figure}
\begin{mdframed}[style=mytableclearbox]
\includegraphics*[width=\linewidth]{img/statement_and_contexts}
\end{mdframed}
\caption{A statement in different contexts, or no context.} \label{fig:statements_and_context}
\end{figure}


``Statements' in this text does \emph{not} include questions, commands, exclamations, or sentence fragments. Someone who asks a \emph{question} like ``Does the grass need to be mowed?'' is typically not claiming that anything is true or false. Generally, \emph{questions} will not count as statements, but \emph{answers} will. ``What is this course about?'' is not a statement. ``No one knows what this course is about,'' is a statement.

For the same reason \emph{commands} do not count as statements for us. If someone bellows ``Mow the grass, now!'' they are not saying whether the grass has been mowed or not. You might infer that they believe the lawn has not been mowed, but then again maybe they think the lawn is fine and just want to see you exercise. 

An exclamation like ``Ouch!'' is also neither true nor false. On its own, it is not a statement. We will treat ``Ouch, I hurt my toe!'' as meaning the same thing as ``I hurt my toe.'' The ``ouch'' does not add anything that could be true or false.

Finally, a lot of possible strings of words will fail to qualify as statements simply because they don't form a complete sentence. In your composition classes, these were probably referred to as sentence fragments. This includes strings of words that are parts of sentences, such as noun phrases like ``The tall man with the hat'' and verb phrases, like ``ran down the hall.'' Phrases like these are missing something they need to make a claim about the world. The class of sentence fragments also includes completely random combinations of words, like ``The up if blender route,'' which don't even have the form of a statement about the world.  

Other logic textbooks describe the components of argument as ``propositions,'' or ``assertions,'' and we will use these terms sometimes as well.  There is actually a great deal of disagreement about what the differences between all of these things are and which term is best used to describe parts of arguments. However, none of that makes a difference for this textbook. We could have used any of the other terms in this text, and it wouldn't change anything. Some textbooks will also use the term ``sentence'' here. We will not use the word ``sentence'' to mean the same thing as ``statement.'' Instead, we will use ``sentence'' the way it is used in ordinary grammar, to refer generally to statements, questions, and commands. 

Sometimes the outward form of a speech act does not match how it is actually being used. A rhetorical question, for instance, has the outward form of a question, but is really a statement or a command. If someone says ``don't you think the lawn needs to be mowed?'' they may actually mean a statement like ``the lawn needs to be mowed'' or a command like ``mow the lawn, now.'' Similarly one might disguise a command as a statement. ``You will respect my authority'' \emph{is} either true or false---either you will or you will not. But the speaker may intend this as an order---''Respect me!''---rather than a prediction of how you will behave.

When we study argument, we need to express things as statements, because arguments are composed of statements. Thus if we encounter a rhetorical question while examining an argument, we need to convert it into a statement. ``Don't you think the lawn needs to be mowed'' will become ``the lawn needs to be mowed.'' Similarly, commands will become should statements. ``Mow the lawn, now!'' will need to be transformed into ``You should mow the lawn.'' 

\newglossaryentry{practical argument}
{
name=practical argument,
description={An argument whose conclusion is a statement that someone should do something.}
}

The latter kind of change will be important in critical thinking, because critical thinking often studies arguments whose goal is to an get audience to do something. These are called \textsc{\glspl{practical argument}}\label{def:practical_argument}. Most advertising and political speech consists of practical arguments, and these are crucial topics for critical thinking.

\newglossaryentry{argument}
{
name=argument,
description={a connected series of statements designed to convince an audience of another statement.}
}

\newglossaryentry{premise}
{
name=premise,
description={a statement in an argument that provides evidence for the conclusion}
}

\newglossaryentry{conclusion}
{
name=conclusion,
description={the statement that an argument is trying to convince an audience of.}
}

 
Once we have a collection of statements, we can use them to build arguments. An \textsc{\gls{argument}} \label{def:Argument} is a connected series of statements designed to convince an audience of another statement. Here an audience might be a literal audience sitting in front of you at some public speaking engagement. Or it might be the readers of a book or article. The audience might even be yourself as you reason your way through a problem. Let's start with an example of an argument given to an external audience. This passage is from an essay by Peter Singer called ``Famine, Affluence, and Morality'' in which he tries to convince people in rich nations that they need to do more to help people in poor nations who are experiencing famine.

\begin{quotation}\noindent \textit{A contemporary philosopher writing in an academic journal} If it is in our power to prevent something bad from happening, without thereby sacrificing anything of comparable moral importance, we ought, morally, to do so. Famine is something bad, and it can be prevented without sacrificing anything of comparable moral importance. So, we ought to prevent famine. \citep{Singer1972} \label{singer_quote} \end{quotation} 

Singer wants his readers to work to prevent famine. This is represented by the last statement of the passage, ``we ought to prevent famine,'' which is called the conclusion of the passage. The \textsc{\gls{conclusion}} \label{def:conclusion} of an argument is the statement that the argument is trying to convince the audience of. The statements that do the convincing are called the \textsc{\glspl{premise}}. \label{def:premise}In this case, the argument has three premises: (1) ``If it is in our power to prevent something bad from happening, without thereby sacrificing anything of comparable moral importance, we ought, morally, to do so''; (2) ``Famine is something bad''; and (3) ``it can be prevented without sacrificing anything of comparable moral importance.''

Now let's look at an example of internal reasoning. 

\begin{quotation}\noindent\textit{Jack arrives at the track, in bad weather.} There is no one here. I guess the race is not happening. \label{racetrack}
\end{quotation}

In the passage above, the words in \textit{italics} explain the context for the reasoning, and the words in regular type represent what Jack is actually thinking to himself. \nix{(We will talk more about his way of representing reasoning in section \ref{sec:arguments_and_context}, below.)} This passage again has a premise and a conclusion. The premise is that no one is at the track, and the conclusion is that the race was canceled. The context gives another reason why Jack might believe the race has been canceled, the weather is bad. You could view this as another premise--it is very likely a reason Jack has come to believe that the race is canceled. In general, when you are looking at people's internal reasoning, it is often hard to determine what is actually working as a premise and what is just working in the background of their unconscious. %[We will talk more about this in section...]


\newglossaryentry{premise indicator}
{
name=premise indicator,
description={a word or phrase such as ``because'' used to indicate that what follows is the premise of an argument.}
}

\newglossaryentry{conclusion indicator}
{
name=conclusion indicator,
description={a word or phrase such as ``therefore'' used to indicate that what follows is the conclusion of an argument.}
}

When people give arguments to each other, they typically use words like ``therefore'' and ``because.'' These are meant to signal to the audience that what is coming is either a premise or a conclusion in an argument. Words and phrases like ``because'' signal that a premise is coming, so we call these \textsc{\glspl{premise indicator}}. Similarly, words and phrases like ``therefore'' signal a conclusion and are called \textsc{\glspl{conclusion indicator}}. The argument from Peter Singer (on page \pageref{singer_quote}) uses the conclusion indicator word, ``so.'' Table \ref{table:Indicators} is an incomplete list of indicator words and phrases in English.


\begin{table}
\begin{mdframed}[style=mytablebox]

\begin{longtabu}{X[1,p]X[2,p]}
\textbf{Premise Indicators:} & because, as, for, since, given that, for the reason that \\
\textbf{Conclusion Indicators:} & therefore, thus, hence, so, consequently, it follows that, in conclusion, as a result, then, must, accordingly, this implies that, this entails that, we may infer that \\
\end{longtabu}
\end{mdframed}
\caption{Premise and Conclusion Indicators.}
\label{table:Indicators}
\end{table}

\newglossaryentry{standard form}
{
name=standard form,
description={a method for representing arguments where each premise is written on a separate, numbered, line, followed by a horizontal bar and then the conclusion. Statements in the argument might be paraphrased for brevity and indicator words are removed.}
}


The two passages we have looked at in this section so far have been simply presented as quotations. But often it is extremely useful to rewrite arguments in a way that makes their logical structure clear. One way to do this is to use something called ``standard form.''   An argument written in \textsc{\gls{standard form}} \label{def:canonical_form}has each premise numbered and written on a separate line. Indicator words and other unnecessary material should be removed from the premises. Although you can shorten the premises and conclusion, you need to be sure to keep them all complete sentences with the same meaning, so that they can be true or false. The argument from Peter Singer, above, looks like this in standard form:

\begin{earg}
\item[P$_1$:] If we can stop something bad from happening, without sacrificing anything of comparable moral importance, we ought to do so. 
\item[P$_2$:] Famine is something bad.
\item[P$_3$:] Famine can be prevented without sacrificing anything of comparable moral importance.
\vspace{-.5em}
\item [] \rule{0.9\linewidth}{.5pt} 
\item[C:] We ought to prevent famine.
\end{earg} 

Each statement has been written on its own line and given a number. The statements have been paraphrased slightly, for brevity, and the indicator word ``so'' has been removed. Also notice that the ``it'' in the third premise has been replaced by the word ``famine,'' so that statements reads naturally on its own.  

Similarly, we can rewrite the argument Jack gives at the racetrack, on page \pageref{racetrack}, like this:

\begin{earg}
\item[P:] There is no one at the race track.
\vspace{-.5em}
\item [] \rule{0.4\linewidth}{.5pt} 
\item[C:] The race is not happening. 
\end{earg} 

Notice that we did not include anything from the part of the passage in italics. The italics represent the context, not the argument itself. Also, notice that the ``I guess'' has been removed. When we write things out in standard form, we write the content of the statements, ignore information about the speaker's mental state, like ``I believe'' or ``I guess.'' 

One of the first things you have to learn to do in logic is to identify arguments and rewrite them in standard form. This is a foundational skill for everything else we will be doing in this text, so we are going to run through a few examples now, and there will be more in the exercises. The passage below is paraphrased from the ancient Greek philosopher Aristotle. 

\begin{quotation}\noindent \textit{An ancient philosopher, writing for his students} Again, our observations of the stars make it evident that the earth is round. For quite a small change of position to south or north causes a manifest alteration in the stars which are overhead. (\cite{Aristotle:heavens}, 298a2-10)
\label{on_the_heavens} \end{quotation}

The first thing we need to do to put this argument in standard form is to identify the conclusion. The indicator words are the best way to do this. The phrase ``make it evident that'' is a conclusion indicator phrase. He is saying that everything else is \textit{evidence} for what follows. So we know that the conclusion is that the earth is round. ``For'' is a premise indicator word---it is sort of a weaker version of ``because.''  Thus the premise is that the stars in the sky change if you move north or south. In standard form, Aristotle's argument that the earth is round looks like this.\\


\begin{earg}
\item[P:] There are different stars overhead in the northern and southern parts of the earth.
\vspace{-.5em}
\item [] \rule{0.9\linewidth}{.5pt} 
\item[C:] The earth is spherical in shape. 
\end{earg} 

That one is fairly simple, because it just has one premise. Here's another example of an argument, this time from the book of Ecclesiastes in the Bible. The speaker in this part of the bible is generally referred to as The Preacher, or in Hebrew, Koheleth. In this verse, Koheleth uses both a premise indicator and a conclusion indicator to let you know he is giving reasons for enjoying life.

\begin{quotation}
\noindent \textit{The words of the Preacher, son of David, King of Jerusalem} There is something else meaningless that occurs on earth: the righteous who get what the wicked deserve, and the wicked who get what the righteous deserve. \ldots So I commend the enjoyment of life, because there is nothing better for a person under the sun than to eat and drink and be glad. (Ecclesiastes 8:14-15, New International Version)
\end{quotation}

Koheleth begins by pointing out that good things happen to bad people and bad things happen to good people. This is his first premise. (Most Bible teachers provide some context here by pointing that that the ways of God are mysterious and this is an important theme in Ecclesiastes.) Then Koheleth gives his conclusion, that we should enjoy life, which he marks with the word ``so.'' Finally he gives an extra premise, marked with a ``because,'' that there is nothing better for a person than to eat and drink and be glad. In standard form, the argument would look like this.


\begin{earg}
\item[P$_1$:] Good things happen to bad people and bad things happen to good people.
\item[P$_2$:] There is nothing better for people than to eat, to drink and to enjoy life.
\vspace{-.5em}
\item [] \rule{0.8\linewidth}{.5pt} 
\item[C:] You should enjoy life.
\end{earg} 

Notice that in the original passages, Aristotle put the conclusion in the first sentence, while Koheleth put it in the middle of the passage, between two premises. In ordinary English, people can put the conclusion of their argument where ever they want. However, when we write the argument in standard form, the conclusion goes last.

Unfortunately, indicator words aren't a perfect guide to when people are giving an argument. Look at this passage from a newspaper:

\begin{quotation}
\noindent \textit{From the general news section of a national newspaper} The new budget underscores the consistent and paramount importance of tax cuts in the Bush philosophy. His first term cuts affected more money than any other initiative undertaken in his presidency, including the costs thus far of the war in Iraq. All told, including tax incentives for health care programs and the extension of other tax breaks that are likely to be taken up by Congress, the White House budget calls for nearly \$300 billion in tax cuts over the next five years, and \$1.5 trillion over the next 10 years.  \citep{Toner2006}
\end{quotation}

Although there are no indicator words, this is in fact an argument. The writer wants you to believe something about George Bush: tax cuts are his number one priority. The next two sentences in the paragraph give you reasons to believe this. You can write the argument in standard form like this.

\begin{earg}
\item[P$_1$:] Bush's first term cuts affected more money than any other initiative undertaken in his presidency, including the costs thus far of the war in Iraq. 
\item[P$_2$:] The White House budget calls for nearly \$300 billion in tax cuts over the next five years, and \$1.5 trillion over the next 10 years. 
\vspace{-.5em}
\item [] \rule{0.9\linewidth}{.5pt} 
\item[C:] Tax cuts are of consistent and paramount importance of in the Bush philosophy.
\end{earg} 

The ultimate test of whether something is an argument is simply whether some of the statements provide reason to believe another one of the statements. If some statements support others, you are looking at an argument. The speakers in these two cases use indicator phrases to let you know they are trying to give an argument.

\newglossaryentry{inference}
{
name=inference,
description={the act of coming to believe a conclusion on the basis of some set of premises.}
}

A final bit of terminology for this section. An \textsc{\gls{inference}} \label{def:Inference} is the act of coming to believe a conclusion on the basis of some set of premises. When Jack in the example above saw that no one was at the track, and came to believe that the race was not on, he was making an inference. We also use the term inference to refer to the connection between the premises and the conclusion of an argument. If your mind moves from premises to conclusion, you make an inference, and the premises and the conclusion are said to be linked by an inference. In that way inferences are like argument glue: they hold the premises and conclusion together. 

%%%% Practice Problems


\practiceproblems
Throughout the book, you will find a series of practice problems that review and explore the material covered in the chapter. There is no substitute for actually working through some problems, because \iflabelexists{CTVersion}{critical thinking}{logic} is more about a way of thinking than it is about memorizing facts. %The answers to some of the problems are provided at the end of the book in Appendix \ref{app.solutions}; the problems that are solved in the appendix are marked with a star (\solutions.)

\noindent\problempart Decide whether the following passages are statements, as the term is used in this textbook, and give reasons for your answers.

\begin{longtabu}{p{.1\linewidth}p{.9\linewidth}}
\textbf{Example}: & Did you follow the instructions? \\
\textbf{Answer}: & Not a statement, a question. \\
\end{longtabu}


\begin{exercises}
\item England is smaller than China. \answerblank{\underline{Statement}}{\vspace{.25in}}
\item Greenland is south of Jerusalem. \answerblank{\underline{Statement}}{\vspace{.25in}}
\item Is New Jersey east of Wisconsin? \answerblank{\underline{A question, not a Statement.}}{\vspace{.25in}}
\item The atomic number of helium is 2. \answerblank{\underline{Statement}}{\vspace{.25in}}
\item The atomic number of helium is $\pi$. \answerblank{\underline{Statement}}{\vspace{.25in}}
\item I hate overcooked noodles. \answerblank{\underline{Statement}}{\vspace{.25in}}
\item Blech! Overcooked noodles! \answerblank{\underline{An exclamation, not a statement.}}{\vspace{.25in}}
\item Overcooked noodles are disgusting.\answerblank{\underline{Statement}}{\vspace{.25in}}
\item Take your time. \answerblank{\underline{A command, not a Statement}}{\vspace{.25in}}
\item This is the last question. \answerblank{\underline{Statement}}{\vspace{.25in}}
\end{exercises}


\noindent\problempart Decide whether the following passages are statements, as the term is used in this textbook, and give reasons for your answers.
\answer{Answers from Ben Sheredos.}
\begin{exercises}
\item Is this a question? \answer{\underline{Question, not a statement.}}
\item Nineteen out of the 20 known species of Eurasian elephants are extinct. \answer{\underline{Statement; has to be true or false (might be false bc 20 is the wrong number, or because they are not extinct, etc.)}}
\item The government of the United Kingdom has formally apologized for the way it treated the logician Alan Turing. \answer{\underline{ Statement: has to be true or false; they either have or have not apologized}} 

\item Texting while driving \answer{\underline{Not a statement, but a sentence fragment}}
\item Texting while driving is dangerous. \answer{\underline{Statement; has to be true or false.}}
\item Insanity ran in the family of logician Bertrand Russell, and he had a life-long fear of going mad. \answer{\underline{Complex, but a statement: both halves are true or false, so is the whole.}}
\item For the love of Pete, put that thing down before someone gets hurt!  \answer{\underline{Not a statement: First bit is an exclamation, second is a command.}}
\item Don't try to make too much sense of this. \answer{\underline{Not a statement, a command.}}
\item Never look a gift horse in the mouth.  \answer{\underline{Not a statement, a command.}}
\item The physical impossibility of death in the mind of someone living  \answer{\underline{ Not a statement, sentence fragment.}}
\end{exercises}

\noindent\problempart Rewrite each of the following arguments in standard form. Be sure to remove all indicator words and keep the premises and conclusion as complete sentences. Write the indicator words and phrases separately and state whether they are premise or conclusion indicators. 

%NTS: when writing these problems, be sure to include a mix of conclusion-first, conclusion-last and conclusion middle, as well as a mix of arguments with true and false premises and a variety of indicator words (or lack thereof).

\begin{longtabu}{p{.1\linewidth}p{.9\linewidth}}	
\textbf{Example}: & \textit{An ancient philosopher writes} We should not be distressed or concerned by the thought of our our own death in any way. Why? Look back on the time before you were born: It is a time you did not exist, but it does not trouble you in any way. The time after you die is also a time when you will not exist, so it shouldn't trouble you either. (Based on Lucretius \citetitle{Lucretius2001} 3.972--75)\\
\textbf{Answer}: & 
\vspace{-16pt}
\begin{earg}
\item[P$_1$:] The time before you were born is a time you did not exist.
\item[P$_2$:] You are not troubled by the time before you were born. 
\item[P$_3$:] The time after you die is also a time you will not exist.
\vspace{-.5em}
\item [] \rule{0.6\linewidth}{.5pt} 
\item[C:] We should not be distressed or concerned by the thought of our our own death. 
\end{earg} 
Premise indicator: So
\\
\end{longtabu}
	
\begin{exercises}

\item \textit{A detective is speaking: }Henry's finger-prints were found on the stolen computer. So, I infer that Henry stole the computer.  

\answerblank{
\begin{earg*}
\item Henry's finger-prints were found on the stolen computer
\itemc Henry stole the computer.  
\end{earg*}
Conclusion indicator word: So}{\vspace{1.5in}}


\item \textit{Monica is wondering about her co-workers political opinions} You cannot both oppose abortion and support the death penalty, unless you think there is a difference between fetuses and felons. Steve opposes abortion and supports the death penalty. Therefore Steve thinks there is a difference between fetuses and felons. 
		%Conclusion-last

\answerblank{
\begin{earg*}
\item You cannot both oppose abortion and support the death penalty, unless you think there is a difference between fetuses and felons. 
\item Steve opposes abortion and supports the death penalty. 
\itemc Steve thinks there is a difference between fetuses and felons. 
\end{earg*}
Conclusion Indicator: Therefore}{\vspace{1.5in}}


\item \textit{The Grand Moff of Earth defense considers strategy} We know that whenever people from one planet invade another, they always wind up being killed by the local diseases, because in 1938, when Martians invaded the Earth, they were defeated because they lacked immunity to Earth's diseases. Also, in 1942, when Hitler's forces landed on the Moon, they were killed by Moon diseases.
		%Conclusion-first

\answerblank{
\begin{earg} 
\item[1.] In 1938, when Martians invaded the Earth, they were defeated because they lacked immunity to Earth's diseases. 
\item[2.] In 1942, when Hitler's forces landed on the Moon, they were killed by Moon diseases.
\item [] \noindent\hrulefill 
\item[$\therefore$] Whenever people from one planet invade another, they always wind up being killed by the local diseases, 
\end{earg}
Premise indicator: Because }{\vspace{1.5in}}


\item If you have slain the Jabberwock, my son, it will be a frabjous day. The Jabberwock lies there dead, its head cleft with your vorpal sword. This is truly a fabjous day. 
%Conclusion-last
\answerblank{ 
\begin{earg*} 
\item  If you have slain the Jabberwock, my son, it will be a frabjous day. 
\item The Jabberwock lies there dead
 
\itemc This is truly a fabjous day 
\end{earg*}
Indicators: none		
}{\vspace{1.5in}}	

\item \textit{A detective trying to crack a case thinks to herself} Miss Scarlett was jealous that Professor Plum would not leave his wife to be with her. Therefore she must be the killer, because she is the only one with a motive. 
%Conclusion-middle
\answerblank{
\begin{earg*} 
\item Miss Scarlett was jealous that Professor Plum would not leave his wife to be with her. 
\item Miss Scarlett is the only one with a motive. 
 
\itemc Miss Scarlett must be the killer
\end{earg*}

Premise Indicator: Because \\
Conclusion Indicator: Therefore}{\vspace{1.5in}}
\end{exercises}



\noindent\problempart Rewrite each of the following arguments in standard form. Be sure to remove all indicator words and keep the premises and conclusion as complete sentences. Write the indicator words and phrases separately and state whether they are premise or conclusion indicators. 

\answer{Answers from Ben Sheredos.}

\begin{enumerate}[label=\arabic*), topsep=0pt, parsep=0pt, itemsep=6pt]
\item \textit{A pundit is speaking on a Sunday political talk show} Hillary Clinton should drop out of the race for Democratic Presidential nominee. For every day she stays in the race, McCain gets a day free from public scrutiny and the members of the Democratic party get to fight one another.  
			%Conclusion-first

\answer{ 
	\begin{earg*} 
		\item For every day Hillary stays in the race, McCain gets a day free from public scrutiny and the members of the Democratic party get to fight one another.
		\itemc Hillary Clinton should drop out of the race for Democratic Presidential Nominee.
	\end{earg*}
	"For" could be a premise-indicator, functioning like "since."
}


\item You have to be smart to understand the rules of Dungeons and Dragons. Most smart people are nerds. So, I bet most people who play D\&D are nerds.  
			%Conclusion-last

\answer{ 
			\begin{earg*} 
				\item You have to be smart to understand the rules of D\&D.
				\item Most smart people are nerds.
				\itemc $[I bet]$ most people who play D\&D are nerds.
			\end{earg*}
			"So" is definitely a conclusion-indicator; "I bet" is probably part of a conclusion-indicator as well, with the speaker indicating that they think this argument is a bit weak.
		}

\item Any time the public receives a tax rebate, consumer spending increases. Since the public just received a tax rebate, consumer spending will increase. 
		%Conclusion-last

\answer{ 
	\begin{earg*} 
		\item Any time the public receives a tax rebate, consumer spending increases. 
		\item The public just received a tax rebate.
		\itemc Consumer spending will increase.
	\end{earg*}
	"Since" is a premise-indicator, but the last sentence needs to be split up into premise and conclusion. This would be more clear if the speaker said "\underline{Since} the public just received a tax rebate, \underline{it follows that} consumer spending will increase." Our speaker is lazy.
}

\item Isabelle is taller than Jacob. Kate must also be taller than Jacob, because she is taller than Isabelle. 
%conclusion-middle

\answer{ 
	\begin{earg*} 
		\item Isabelle is taller than Jacob.
		\item Kate is taller than Isabelle.
		\itemc Kate is taller than Jacob.
	\end{earg*}
	"Must" is a conclusion indicator, "because" is a premise-indicator, and so the last sentence has to be split up to put this argument into standard form.
}
\end{enumerate}

% * **********************************
% *     Arguments and Nonarguments          *
% ************************************

\section{Arguments and Nonarguments}
\label{sec:arguments_and_nonarguments}

We just saw that arguments are made of statements. However, there are lots of other things you can do with statements. Part of learning what an argument is involves learning what an argument is not, so in this section and the next we are going to look at some other things you can do with statements besides make arguments. 

The list below of kinds of nonarguments is not meant to be exhaustive: there are all sorts of things you can do with statements that are not discussed. Nor are the items on this list meant to be exclusive. One passage may function as both, for instance, a narrative and a statement of belief. Right now we are looking at real world reasoning, so you should expect a lot of ambiguity and imperfection. If your class is continuing on into the critical thinking portions of this textbook, you will quickly get used to this. 

\subsection{Simple Statements of Belief}

\newglossaryentry{simple statement of belief}
{
name=simple statement of belief,
description={A kind of nonargumentative passage where the speaker simply asserts what they believe without giving reasons. }
}

An argument is an attempt to persuade an audience to believe something, using reasons. Often, though, when people try to persuade others to believe something, they skip the reasons, and give a \textsc{\gls{simple statement of belief}}. \label{def:simple_statement_of_belief} This is a kind of nonargumentative passage where the speaker simply asserts what they believe without giving reasons. Sometimes simple statements of belief are prefaced with the words ``I believe,'' and sometimes they are not. A simple statements of belief can be a profoundly inspiring way to change people's hearts and minds. Consider this passage from Dr. Martin Luther King's Nobel acceptance speech.

\begin{quotation} \noindent I believe that even amid today's mortar bursts and whining bullets, there is still hope for a brighter tomorrow. I believe that wounded justice, lying prostrate on the blood-flowing streets of our nations, can be lifted from this dust of shame to reign supreme among the children of men. I have the audacity to believe that peoples everywhere can have three meals a day for their bodies, education and culture for their minds, and dignity, equality and freedom for their spirits. \citep{King2001} \end{quotation}

This actually is a part of a longer passage that consists almost entirely of statements that begin with some variation of ``I believe.''It is incredibly powerful oration, because the audience, feeling the power of King's beliefs, comes to share in those beliefs. The language King uses to describe how he believes is important, too. He says his belief in freedom and equality requires audacity, making the audience feel his courage and want to share in this courage by believing the same things. 

These statements are moving, but they do not form an argument. None of these statements provide evidence for any of the other statements. In fact, they all say roughly the same thing, that good will triumph over evil. So the study of this kind of speech belongs to the discipline of rhetoric, not of logic.  
  
\subsection{Expository Passages}

Perhaps the most basic use of a statement is to convey information. Often if we have a lot of information to convey, we will sometimes organize our statements around a theme or a topic. Information organized in this fashion can often appear like an argument, because all of the statements in the passage relate back to some central statement. However, unless the other statements are given as reasons to believe the central statement, the passage you are looking at is not an argument. Consider this passage:

\begin{quotation}\noindent\textit{From a college psychology textbook.} Eysenck advocated three major behavior techniques that have been used successfully to treat a variety of phobias. These techniques are modeling, flooding, and systematic desensitization. In \textbf{modeling} phobic people watch nonphobics cope successfully with dreaded objects or situations.In \textbf{flooding} clients are exposed to dreaded objects or situations for prolonged periods of time in order to extinguish their fear. In contrast to flooding, \textbf{systematic desensitization} involves gradual, client-controlled exposure to the anxiety eliciting object or situation. (Adapted from Ryckman \cite*{Ryckman2007}) \end{quotation}

\newglossaryentry{expository passage}
{
name=expository passage,
description={A nonargumentative passage that organizes statements around a central theme or topic statement.}
}

We call this kind of passage an expository passage. In an \textsc{\gls{expository passage}}, \label{def:expository_passage} statements are organized around a central theme or topic statement. The topic statement might look like a conclusion, but the other statements are not meant to be evidence for the topic statement. Instead, they elaborate on the topic statement by providing more details or giving examples. In the passage above, the topic statement is ``Eysenck advocated three major behavioral techniques \ldots.'' The statements describing these techniques elaborate on the topic statement, but they are not evidence for it. Although the audience may not have known this fact about Eysenk before reading the passage, they will typically accept the truth of this statement instantly, based on the textbook's authority. Subsequent statements in the passage merely provide detail. 

Deciding whether a passage is an argument or an expository passage is complicated by the fact that sometimes people argue by example: 

\begin{adjustwidth}{2em}{0em}
\begin{longtabu}{p{.1\linewidth}p{.8\linewidth}}
\textbf{Steve:} & Kenyans are better distance runners than everyone else. \\
\textbf{Monica:} & Oh come on, that sounds like an exaggeration of a stereotype that isn't even true.\\
\textbf{Steve:} & What about Dennis Kimetto, the Kenyan who set the world record for running the marathon? And you know who the previous record holder was: Emmanuel Mutai, also Kenyan. \\
\end{longtabu}
\end{adjustwidth}
\vspace{-1.5cm}

Here Steve has made a general statement about all Kenyans. Monica clearly doubts this claim, so Steve backs it up with some examples that seem to match his generalization. This isn't a very strong way to argue: moving from two examples to statement about all Kenyans is probably going to be a kind of bad argument known as a hasty generalization. (This mistake is covered in the complete version of this text in the chapter on induction\nix{Chapter \ref{chap:induction} on induction.}\label{ver_var}) The point here however, is that Steve is just offering it as an argument. 

The key to telling the difference between expository passages and arguments by example is whether there is a conclusion that they audience needs to be convinced of. In the passage from the psychology textbook, ``Eysenck advocated three major behavioral techniques'' doesn't really work as a conclusion for an argument. The audience, students in an introductory psychology course, aren't likely to challenge this assertion, the way Monica  challenges Steve's overgeneralizing claim. 

Context is very important here, too. The Internet is a place where people argue in the ordinary sense of exchanging angry words and insults. In that context, people are likely to actually give some arguments in the logical sense of giving reasons to believe a conclusion. 

\subsection{Narratives} 

Statements can also be organized into descriptions of events and actions, as in this snippet from book V of \textit{Harry Potter}.

\begin{quotation} \noindent But she [Hermione] broke off; the morning post was arriving and, as usual, the \textit{Daily Prophet} was soaring toward her in the beak of a screech owl, which landed perilously close to the sugar bowl and held out a leg. Hermione pushed a Knut into its leather pouch, took the newspaper, and scanned the front page critically as the owl took off again. \citep{Rowling2003} \end{quotation} 

\newglossaryentry{narrative}
{
name=narrative,
description={A nonargumentative passage that describes a sequence of events or actions.}
}

We will use the term \textsc{\gls{narrative}} \label{def:narrative} loosely to refer to any passage that gives a sequence of events or actions. A narrative can be fictional or nonfictional. It can be told in regular temporal sequence or it can jump around, forcing the audience to try to reconstruct a temporal sequence. A narrative can describe a short sequence of actions, like Hermione taking a newspaper from an owl, or a grand sweep of events, like this passage about the  rise and fall of an empire in the ancient near east:

\begin{quotation}\noindent The Guti were finally expelled from Mesopotamia by the Sumerians of Erech (\textit{c}. 2100), but it was left to the kings of Ur's famous third dynasty to re-establish the Sargonoid frontiers and write the final chapter of the Sumerian History. The dynasty lasted through the twenty first century at the close of which the armies of Ur were overthrown by the Elamites and Amorites \citep{McEvedy1967}. \end{quotation} 

This passage does not feature individual people performing specific actions, but it is still united by character and action. Instead of Hermione at breakfast, we have the Sumarians in Mesopotamia. Instead of retrieving a message from an owl, the conquer the Guti, but then are conquered by the Elamites and Amorites. The important thing is that the statements in a narrative are not related as premises and conclusion. Instead, they are all events which are united common characters acting in specific times and places. 

%%%%%%% Practice Problems

\practiceproblems
\problempart Identify each passage below as an argument or a nonargument, and give reasons for your answers. If it is a nonargument, say what kind of nonargument you think it is. If it is an argument, write it out in standard form.

\begin{longtabu}{p{.1\linewidth}p{.9\linewidth}}
\textbf{Example}: & \textit{One student speaks to another student who has missed class:} The instructor passed out the syllabus at 9:30. Then he went over some basic points about reasoning, arguments and explanations. Then he said we should call it a day. \\
\textbf{Answer}: & Not an argument, because none of the statements provide any support for any of the others. This is probably better classified as a narration because the events are in temporal sequence. \\
\end{longtabu}

\begin{exercises}
%\item \textit{An anthropology teacher is speaking to her class }Different gangs use different colors to distinguish themselves. Here are some illustrations: biologists tend to wear some blue, while the philosophy gang wears black. 
%\answerblank{\\ Not an argument. Expository passage. The students probably will believe the teacher as soon as she makes an assertion. The word ``illustration'' is also a clue.}{\vspace{1.5in}}

\item \textit{From Bob Willis “Payrolls Jump Casts Doubt on Fed Rate Pledge” - Feb 3, 2012, in Bloomberg news.} The unemployment rate dropped to 8.3 percent, the lowest since February 2009, Labor Department figures showed today in Washington. The 243,000 increase in jobs was the biggest in nine months and exceeded the most optimistic forecast in a Bloomberg News survey. Service industries grew by the most in a year, according to a separate report.
\answerblank{\\ Not an argument. A report or narration.}{\vspace{1.5in}}

\item \textit{A caller on a radio call-in show has a theory.} The economy has been in trouble recently. And it's certainly true that cell phone use has been rising during that same period. So, I suspect increasing cell phone use is bad for the economy. 
\answerblank{\\  Argument. The indicator ``so'' is a clue. 
\begin{earg*}
\item The economy has been in trouble recently. 
\item Cell phone use has been rising during that same period. 
\itemc Cell phone use is bad for the economy. 
\end{earg*}
}{\vspace{1.5in}}


\item \textit{At Widget-World Corporate Headquarters:} We believe that our company must deliver a quality product to our customers. Our customers also expect first-class customer service. At the same time, we must make a profit. 

%\vspace{6pt}
\answerblank{ Not an argument. The speaker is not using any of the propositions as reasons to believe or explain any of the others; rather she is simply asserting various things.}{\vspace{1.5in}}
      
\item \textit{Jack is at the breakfast table and shows no sign of hurrying. Gill says:} You should leave now. It's almost nine a.m. and it takes three hours to get there.

\answerblank{Arguing. Jack's inaction suggests that he does believe that he needs to leave now and so Gill provides reasons that might convince him. Notice that there are no argument flag words or phrases.

This example also includes the word ``should'' in its conclusion. Words such as ``ought'' and ``should'' indicate that the speaker is trying to get the audience to do or believe something that they are not currently doing or believing.

\begin{earg*}
\item It's almost nine a.m. 
\item It takes three hours to get there.
\itemc  You should leave now.
\end{earg*}
}{\vspace{1.5in}}
      
\item \textit{In a text book on the brain:} Axons are distinguished from dendrites by several features, including shape (dendrites often taper while axons usually maintain a constant radius), length (dendrites are restricted to a small region around the cell body while axons can be much longer), and function (dendrites usually receive signals while axons usually transmit them).

\answerblank{Not an argument. Expository passage. The features named just fill in the first statement.}{\vspace{1.5in}}

\end{exercises}
%
\problempart Identify each passage below as an argument or a nonargument, and give reasons for your answers. If it is a nonargument, say what kind of nonargument you think it is. If it is an argument, write it out in standard form.

%
\begin{exercises}
\item \textit{Suzi doesn't believe she can quit smoking. Her friend Brenda says} Some people have been able to give up cigarettes by using their will-power. Everyone can draw on their will-power. So, anyone who wants to give up cigarettes can do so.

\item \textit{The words of the Preacher, son of David, King of Jerusalem} I have seen something else under the sun: The race is not to the swift or the battle to the strong, nor does food come to the wise or wealth to the brilliant or favor to the learned; but time and chance happen to them all. (Ecclesiastes 9:11, New International Version)

\item \textit{An economic development expert is speaking.} The introduction of cooperative marketing into Europe greatly increased the prosperity of the farmers, so we may be confident that a similar system in Africa will greatly increase the prosperity of our farmers.

\item \textit{From the CBS News website, US section.} Headline: ``FBI nabs 5 in alleged plot to blow up Ohio bridge.'' Five alleged anarchists have been arrested after a months-long sting operation, charged with plotting to blow up a bridge in the Cleveland area, the FBI announced Tuesday. CBS News senior correspondent John Miller reports the group had been involved in a series of escalating plots that ended with their arrest last night by FBI agents. The sting operation supplied the anarchists with what they thought were explosives and bomb-making materials. At no time during the case was the public in danger, the FBI said. \citep{CBSNews2012}


\item \textit{At a school board meeting.} Since creationism can be discussed effectively as a scientific model, and since evolutionism is fundamentally a religious philosophy rather than a science, it is unsound educational practice for evolution to be taught and promoted in the public schools to the exclusion or detriment of special creation. (Kitcher \cite*{Kitcher1982}, p. 177, citing Morris \cite*{Morris1975}.)

\end{exercises}

\iflabelexists{CTVersion}{text for CT version}{

% * **********************************
% *     Arguments and Explanations          *
% ************************************

\section{Arguments and Explanations}
\label{arguments_and_explanations}

Explanations are are not arguments, but they they share important characteristics with arguments, so we should devote a separate section to them. Both explanations and arguments are parts of reasoning, because both feature statements that act as reasons for other statements. The difference is that explanations are not used to convince an audience of a conclusion.  

Let's start with workplace example. Suppose you see your co-worker, Henry, removing a computer from his office. You think to yourself ``Gosh, is he stealing from work?'' But when you ask him about it later, Henry says, ``I took the computer because I believed that it was scheduled for repair.'' Henry's statement looks like an argument. It has the indicator word ``because'' in it, which would mean that the statement ``I believed it was scheduled for repairs'' would be a premise. If it was, we could put the argument in standard form, like this: 

\begin{earg}
\item[P:] I believed the computer was scheduled for repair
\vspace{-.5em}
\item [] \rule{0.6\linewidth}{.5pt} 
\item[C:] I took the computer from the office. 
\end{earg} 

But this would be awfully weird as an argument. If it were an argument, it would be trying to convince us of the conclusion, that Henry took the computer from the office. But you don't need to be convinced of this. You already know it---that's why you were talking to him in the first place. 
  
Henry is giving reasons here, but they aren't reasons that try to \textit{prove} something. They are reasons that \textit{explain} something. When you explain something with reasons, you increase your understanding of the world by placing something you already know in a new context. You already knew that Henry took the computer, but now you know \textit{why} Henry took the computer, and can see that his action was completely innocent (if his story checks out). 


\newglossaryentry{explanation}
{
name=explanation,
description={A kind of reasoning where reasons are used to provide a greater understanding of something that is already known.}
}

\newglossaryentry{explainer}
{
name=explainer,
description={The part of an explanation that provides greater understanding of the explainee.}
}

\newglossaryentry{explainee}
{
name=explainee,
description={The part of an explanation that one gains a greater understanding of as a result of the explainer.}
}

\newglossaryentry{reason}
{
name=reason,
description={The premise of an argument or the explainer in an explanation; the part of reasoning that provides logical support for the target proposition.}
}

\newglossaryentry{target proposition}
{
name=target proposition,
description={The conclusion of an argument or the explainee in an explanation; the part of reasoning that is logically supported by the reasons.}
}



Both arguments and explanations both involve giving reasons, but the reasons function differently in each case. An \textsc{\gls{explanation}} \label{def:explanation}is defined as a kind of reasoning where reasons are used to provide a greater understanding of something that is already known.  

Because both arguments and explanations are parts of reasoning, we will use parallel language to describe them. In the case of an argument, we called the reasons ``premises.'' In the case of an explanation, we will call them \textsc{\glspl{explainer}}. \label{def:explainer} Instead of a ``conclusion,'' we say that the explanation has an \textsc{\gls{explainee}}.  \label{def:explainee} We can use the generic term \textsc{\glspl{reason}} \label{def:reason} to refer to either premises or explainers and the generic term \textsc{\gls{target proposition}} \label{def:target_proposition} to refer to either conclusions or explainees. Figure \ref{fig:arguments_explanations} shows this relationship. 


\begin{figure}
\begin{mdframed}[style=mytableclearbox, userdefinedwidth=.65\textwidth]
\begin{tikzpicture}

\tikzstyle{mynode} = [rectangle, draw, fill=light-gray, rounded corners=3pt,] 

\path
	(0,0) node [mynode] (premises) {Premises}
	(0,-2) node [mynode] (conclusion) {Conclusion}
	(3,0) node  [mynode] (explainers) {Explainers}
	(3,-2) node [mynode] (explainee) {Explainee}
	(5,0) node  [anchor=west](reasons) {Reasons}
	(5,-2) node [anchor=west](target) {Target Proposition};

\tikzstyle{myblockarrow} = [thick, fill=light-gray,decoration={markings,mark=at position
   1 with {\arrow[semithick]{open triangle 60}}},
   double distance=2pt, shorten >= 5.5pt,
   preaction = {decorate},
   postaction = {draw,line width=2pt, white,shorten >= 4.5pt}]


\draw [myarrow2] (premises) to node [left] {\color{black}Prove} (conclusion);
\draw [myarrow2] (explainers) to node [right] {\color{black}Clarify} (explainee);
\draw [myarrow1, shorten >=.5cm] (reasons) to (explainers);
\draw [myarrow1, shorten >=.5cm] (target) to (explainee);


\end{tikzpicture}
\end{mdframed}
\caption{Arguments vs. Explanations.} \label{fig:arguments_explanations}
\end{figure}

We can put explanations in standard form, just like arguments, but to distinguish the two, we will simply number the statements, rather than writing Ps and Cs, and we will put an E next to the line that separates explainers and exaplainee, like this:

\begin{tikzpicture}
\path
	(0,0) node [anchor=west] {1. Henry believed the computer was scheduled for repair}
	(9,-8pt) node [anchor=west]{E}
	(0,-20pt) node[anchor=west] {2. Henry took the computer from the office.};
\draw (.5,-8pt) -- (9,-8pt);
\end{tikzpicture}

Cases where the target proposition is something that is completely common sense are clearcut cases of explanation. Consider the following passage.

\begin{quotation}
\noindent\textit{From Livescience, a science education website, under the headline “Why is grass green?”} Like many plants, most species of grass produce a bright pigment called chlorophyll. Chlorophyll absorbs blue light (high energy, short wavelengths) and red light (low energy, longer wavelengths) well, but mostly reflects green light, which accounts for your lawn's color. \citep{Mauk2013}
\end{quotation}

The passage contains reasoning. The nature of chlorophyll ``accounts for'' the color of grass. But in this case the audience does not need to be convinced that grass is green. Everyone knows that. The audience went to the Livescience website because they wanted an \emph{explanation} for why grass was green. 

Often the same piece of reasoning can work as either an argument or an explanation, depending on the situation where it is used. Consider this short dialogue

\begin{adjustwidth}{2em}{0em}
\begin{longtabu}{p{.1\linewidth}p{.8\linewidth}}
\multicolumn{2}{p{.9\linewidth}}{\textit{Monica visits Steve's cubical}.}\\
\textbf{Monica:} &All your plants are dead.\\
\textbf{Steve:} & It's because I never water them.
\end{longtabu}
\end{adjustwidth}
\vspace{-1cm}


In the passage above, Steve uses the word ``because,'' which we've seen in the past is a premise indicator word. But if it were a premise, the conclusion would be ``All Steve's plants are dead.'' But Steve can't possibly be trying to convince Monica that all his plants are dead. It is something that Monica herself says, and that they both can see. The ``because'' here indicates a reason, but here Steve is giving an explanation, not an argument. He takes something that Steve and Monica already know---that the plants are dead---and puts it in a new light by explaining how it came to be. In this case, the plants died because they didn't get water, rather than dying because they didn't get enough light or were poisoned by a malicious co-worker. The reasoning is best represented like this:

\begin{tikzpicture}
\path
	(0,0) node [anchor=west] {1. Steve never waters his plants.}
	(5.5,-8pt) node [anchor=west]{E}
	(0,-20pt) node[anchor=west] {2. All the plants are dead.};
\draw (.5,-8pt) -- (5.5,-8pt);
\end{tikzpicture}

But the same piece of reasoning can change form an explanation into an argument simply by putting it into a new situation:


\begin{adjustwidth}{2em}{0em}
\begin{longtabu}{p{.1\linewidth}p{.8\linewidth}}
\multicolumn{2}{p{.9\linewidth}}{\textit{Monica and Steve are away from the office}.}\\
\textbf{Monica:} &Did you have someone water your plants while you were away?\\
\textbf{Steve:}& No.\\
\textbf{Monica:}& I bet they are all dead.
\end{longtabu}
\end{adjustwidth}
\vspace{-1cm}

Here Steve and Monica do not know that Steve's plants are dead. Monica is inferring this idea based on the premise which she learns from Steve, that his plants are not being watered. This time ``Steve's plants are not being watered'' is a premise and ``The plants are dead'' is a conclusion. We represent the argument like this:

\begin{earg}
\item[P.] Steve never waters his plants. 
\vspace{-.5em}
\item [] \rule{0.3\linewidth}{.5pt} 
\item[C.] All the plants are dead. 
\end{earg}

In the example of Steve's plants, the same piece of reasoning can function either as an argument or an explanation, depending on the context where it is given. This is because the reasoning in the example of the plants is causal: the \textit{causes} of the plants dying are given as reasons for the death, and we can appeal to causes either to explain something that we know happened or to predict something that we think might have happened. 

Not all kinds of reasoning are flexible like that, however. Reasoning from authority can be used in some kinds of argument, but often makes a lousy explanation. Consider another conversation between Steve and Monica:

\begin{adjustwidth}{2em}{0em}
\begin{longtabu}{p{.1\linewidth}p{.8\linewidth}}
\textbf{Monica:} & I saw on a documentary last night that the universe is expanding and probably will keep expanding for ever. \\
\textbf{Steve:} & Really?\\
\textbf{Monica:} &Yeah, Steven Hawking said so. \\
\end{longtabu}
\end{adjustwidth}
\vspace{-1cm}

There aren't any indicator words here, but it looks like Monica is giving an argument. She states that the universe is expanding, and Steve gives a skeptical ``really?'' Monica then replies by saying that she got this information from the famous physicist Steven Hawking. It looks like Steve is supposed to believe that the universe will expand indefinitely because Hawking, an authority in the relevant field, said so. This makes for an ok argument: 

 \begin{earg}
\item[P:] Steven Hawking said that the universe is expanding and will continue to do so indefinitely.
\vspace{-.5em}
\item [] \rule{\linewidth}{.5pt} 
\item[C:] The universe is expanding and will continue to do so indefinitely.
\end{earg} 

Arguments from authority aren't very reliable, but for very many things they are all we have to go on. We can't all be experts on everything. But now try to imagine this argument as an explanation. What would it mean to say that the expansion of the universe can be \textit{explained} by the fact that Steven Hawking said that it should expand. It would be as if Hawking were a god, and the universe obeyed his commands! Arguments from authority are acceptable, but not ideal. Explanations from authority, on the other hand, are completely illegitimate. \label{no_exp_from_authority}

In general, arguments that appeal to how the world works are more satisfying than ones which appeals to the authority or expertise of others. Compare the following pair of arguments:

\begin{enumerate}[label=(\alph*)]
\item Jack says traffic will be bad this afternoon. So, traffic will be bad this afternoon. 
\item Oh no! Highway repairs begin downtown today. And a bridge lift is scheduled for the middle of rush hour. Traffic is going to be terrible \end{enumerate}

Even though the second passage is an argument, the reasons used to justify the conclusion could be used in an explanation. Someone who accepts this argument will also have an explanation ready to offer if someone should later ask ``Traffic was terrible today! I wonder why?''. This is not true of the first passage: bad traffic is not explained by saying ``Jack said it would be bad.'' The argument that refers to the drawbridge going up is appealing to a more powerful sort of reason, one that works in both explanations and arguments. This simply makes for a more satisfying argument, one that makes for a deeper understanding of the world, than one that merely appeals to authority. 

Although arguments based on explanatory premises are preferred, we must often rely on other people for our beliefs, because of constraints on our time and access to evidence. But the other people we rely on should hopefully hold the belief on the basis of an empirical understanding. And if \textit{those} people are just relying on authority, then we should hope that at some point the chain of testimony ends with someone who is relying on something more than mere authority. In [cross ref] we'll look more closely at sources and how much you should trust them.

We just have seen that they same set of statements can be used as an argument or an explanation depending on the context. This can cause confusion between speakers as to what is going on. Consider the following case:

\begin{adjustwidth}{2em}{0em}
\begin{longtabu}{p{.1\linewidth}p{.8\linewidth}}
\multicolumn{2}{p{.9\linewidth}}{\textit{Bill and Henry have just finished playing basketball.}}\\
\textbf{Bill:} & Man, I was terrible today. \\
\textbf{Henry:} & I thought you played fine. \\
\textbf{Bill:} & Nah. It's because I have a lot on my mind from work. \\
\end{longtabu}
\end{adjustwidth}
\vspace{-1cm}

Bill and Henry disagree about what is happening---arguing or explaining. Henry doubts Bill's initial statement, which should provoke Bill to argue. But instead, he appears to plough ahead with his explanation. What Henry can do in this case, however, is take the reason that Bill offers as an explanation (that Bill is preoccupied by issues at work) and use it as a premise in an argument for the conclusion ``Bill played terribly.'' Perhaps Henry will argue (to himself) something like this: ``It's true that Bill has a lot on his mind from work. And whenever a person is preoccupied, his basketball performance is likely to be degraded. So, perhaps he did play poorly today (even though I didn't notice).''

In other situations, people can switch back and forth between arguing and explaining. Imagine that Jones says ``The reservoir is at a low level because of several releases to protect the down-stream ecology.'' Jones might intend this as an explanation, but since Smith does not share the belief that the reservoir's water level is low, he will first have to be given reasons for believing that it is low. The conversation might go as follows:

\begin{adjustwidth}{2em}{0em}
\begin{longtabu}{p{.1\linewidth}p{.8\linewidth}}
\textbf{Jones:} & The reservoir is at a low level because of several releases to protect the down-stream ecology. \\
\textbf{Smith:} & Wait. The reservoir is low?\\
\textbf{Jones:} & Yeah. I just walked by there this morning. You haven't been up there in a while? \\
\textbf{Smith:} & I guess not. \\
\textbf{Jones:} & Yeah, it's because they've been releasing a lot of water to protect the ecology lately. \\
\end{longtabu}
\end{adjustwidth}
\vspace{-1cm}

When challenged, Smith offers evidence from his memory: he saw the reservoir that morning. Once Smith accepts that the water level is low, Jones can restate his explanation.

Some forms of explanation overlap with other kinds of nonargumentative passages. We are dealing right now with thinking in the real world, and as we mentioned on page \pageref{messiness_warning} the real world is full of messiness and ambiguity. One effect of this is that all the categories we are discussing will wind up overlapping. Narratives and expository passages, for instance, can also function as explanations. Consider this passage

\begin{quotation} \noindent\textit{From the sports section} Duke beat Butler 61-59 for the national championship Monday night. Gordon Hayward's half-court, 3-point heave for the win barely missed to leave tiny Butler one cruel basket short of the Hollywood ending. (Based on \cite{AP2010}) \end{quotation}

On the one hand, this is clearly a narrative---retelling a sequence of events united by time, place, and character. But it also can work as an explanation about how Duke won, if the audience immediately accepts the result. 'The last shot was a miss\textit{ }and then Duke won' can be understood as 'the last shot was a miss and so Duke won'.


%%%%%% Practice problems


\practiceproblems 
\problempart Identify each of the passages below as an argument, an explanation, or neither, and justify your answer. If the passage is an argument write it in standard form, with premises marked P$_1$ etc., then a line, and then the conclusion marked with a C. If the argument is an explanation, write it in the standard form for an explanation, with the explainers numbered and an ``E'' after the line that separates the explainers and the explainee. If the argument is neither an argument nor an explanation, state what kind of nonargument you think it is, such as a narrative or an expository passage.
 
\begin{longtabu}{p{.1\linewidth}p{.9\linewidth}}
\textbf{Example}: & \textit{Henry arrives at work late and wonders to himself: }Bill is not here. He very rarely arrives late. So, he is not coming in today. \\
\textbf{Answer}: & \textit{Argument} You can tell Henry is giving an argument to himself here because the conclusion is something that he did not already believe. \\
&\begin{earg}
\item[P$_1$:] Bill is not here. 
\item[P$_2$:] Bill very rarely arrives late. 
\vspace{-.5em}
\item [] \rule{0.6\linewidth}{.5pt} 
\item[C:] Bill is not coming in today
\end{earg} 
\end{longtabu}


\begin{exercises}

\item \textit{From a science education website run by NASA, also promoted by Google as the answer to the question “Why is the sky blue?”}  Blue light is scattered in all directions by the tiny molecules of air in Earth's atmosphere. Blue is scattered more than other colors because it travels as shorter, smaller waves. This is why we see a blue sky most of the time. \citep{NASA2015}

\answerblank{\vspace{6pt}This is an explanation, because the target proposition is common knowledge, as in the ``Grass is green'' example in your text. \vspace{6pt}

\begin{tikzpicture}
\path
	(0,0) node [anchor=west] {1. Blue travels in shorter wavelengths. }
	(0,-11pt) node [anchor=west] {2. Blue is scattered more than other colors.}
	(0,-22pt)  node [anchor=west] {3. Light is scattered by molecules in the atmosphere.}
	(9, -33pt) node [anchor=west] {E}
	(0, -44 pt) node[anchor=west] {4. We see a blue sky most of the time.};
\draw (.5,-33pt) -- (9,-33pt);
\end{tikzpicture}

\vspace{6pt}There are actually two separate levels of explanation in this passage, each marked with separate indicator words. The first ``because'' relates the sentence ``Blue travels in shorter wavelengths'' to the sentence ``Blue is scattered more than other colors.'' The short wavelengths explain why blue is scattered more. The fact that blue is scattered more and that light is scattered when it enters the atmosphere in turn explains why we see the sky as blue. If you wanted to be very precise, you would represent the explanations with two separate diagrams. First

\begin{tikzpicture}
\path
	(0,0) node [anchor=west] {1. Blue travels in shorter wavelengths. }
	(9, -11pt) node [anchor=west] {E}
	(0,-22pt) node [anchor=west] {2. Blue is scattered more than other colors.};
\draw (.5,-11pt) -- (9,-11pt);
\end{tikzpicture}

and then \vspace{6pt}

\begin{tikzpicture}
\path
	(0,0) node [anchor=west] {1. Blue is scattered more than other colors.}
	(0,-11pt) node [anchor=west] {2. Light is scattered by molecules in the atmosphere.}
	(9,-22pt)  node [anchor=west] {E}
	(0, -33 pt) node[anchor=west] {3. We see a blue sky most of the time.};
\draw (.5,-22pt) -- (9,-22pt);
\end{tikzpicture}

}{\vspace{1.5in}}



\item \textit{Jack is reading a popular science magazine. It reads: }Recent research has shown that people who rate themselves as ``very happy'' are less successful financially than those who rate themselves as ``moderately happy.'' \textit{He says,} ``Huh! It seems that a little unhappiness is good in life.''  


\answerblank{
\begin{earg*}
\item People who rate themselves as ``very happy'' are less successful financially than those who rate themselves as ``moderately happy.''
\itemc A little unhappiness is good in life.
\end{earg*}

Jack is arguing to himself. He makes an inference about happiness based on the premise he read in the magazine. Other people may have drawn a different conclusion from that premise. }{\vspace{1.5in}}

\item \textit{An anthropologist is speaking. }People get nicknames based on some distinctive feature they possess. And so, Mark, for example, who is 6'6'' is  (ironically) called ``Smalls'', while Matt, who looks young, is called ``Baby Face.'' John looks just like his dad, and is called ``Chip.'' 

\answerblank{\vspace{6pt}
Neither. Mark, Matt and John are examples of the general proposition stated in the first statement. However, the examples aren't being used to prove the first statement, nor do they explain why the first statement is true.}{\vspace{1.5in}} 




\item \textit{Two teenaged friends are talking. Analyze Saida's reasoning.}
\vspace{-6pt}
\begin{adjustwidth}{2em}{0em}
\begin{longtabu}{p{.1\linewidth}p{.8\linewidth}}
\textbf{Saida}: &I can't go to the show tonight. \\
\textbf{Jordan}:& Bummer. \\
\textbf{Saida}: &I know! My mother wouldn't let me go out when I asked. 
\end{longtabu}
\end{adjustwidth}
\vspace{-.9cm}

\answerblank{\vspace{6pt} Explanation or expository passage. It is not an argument because Jordan is going to believe Saida right away because they are friends. So Saida doesn't need to prove that she can't go to the show. Most likely this is an explanation. "My mother won't let me," \textit{explains} why she can't go. \\

\begin{tikzpicture}
\path
	(0,0) node [anchor=west] {1. My mother won't let me go to the show.}
	(9,-8pt) node [anchor=west]{E}
	(0,-19pt) node[anchor=west] {2. I can't go to the show.};
\draw (.5,-8pt) -- (9,-8pt);
\end{tikzpicture}}{\vspace{1.5in}}


\item \textit{A mother is speaking to her teenage son. }You should always listen 
to your mother. I say ``no\texttt.'' So, you have to stay in tonight. 

\answerblank{

\begin{earg*}
\item You should always listen to your mother. 
\item Your mother says ``no.'' 
\itemc You have to stay in tonight. 
\end{earg*}
Conclusion indicator word: So\\

Arguing. The mother is giving her son a reason to stay home--her authority. We will talk more about arguments from authority later.
}{\vspace{1.5in}}

\item \textit{An economist is speaking. }Any time the public receives a tax rebate, consumer spending increases. Since the public just received a tax rebate, consumer spending will increase. 


\answerblank{ 
\begin{earg*}
\item Any time the public receives a tax rebate, consumer spending increases. 
\item The public just received a tax rebate
\itemc Consumer spending will increase.
\end{earg*}

Premise indicator word: Since\\
\vspace{6pt}
Arguing. ``Since'' is a flag work here that shows some kind of inference is being made. But the increase in spending hasn't happened yet, so the audience needs to be convinced by the economist that it will happen. Therefore the passage is an argument. If the increase in spending had already happened, and the audience already believe it, then the economist would be explaining. In general, if the target proposition is a prediction, then the passage is likely to be an argument, because the audience doesn't already know the future. % (JRL)
}{\vspace{1.5in}}


\item \textit{In a letter to the editor. }Today's kids are all slackers. American 
society is doomed. 

\answerblank{
\begin{earg*}
\item Today's kids are all slackers. 
\itemc American society is doomed. 
\end{earg*}
Argument. This is an argument for the same reason as the last one. It is a prediction about the future.}{\vspace{1.5in}}

\item  \textit{On Monday, Jack is told that his unit ships to Iraq in two days: }I 
was hoping to go to Henry's birthday party next weekend. But I'm shipping out on Wednesday. So, I will miss it. 

\answerblank{
\begin{earg*}
\item I was hoping to go to Henry's birthday party next weekend. 
\item I'm shipping out on Wednesday. 
\itemc I will miss it.
\end{earg*}

Arguing. Jack learns a fact, in this case that he is shipping out, and then infers another fact, that he will miss the party. I have no idea why missing a party is the first thing on his mind when he is given this news. }{\vspace{1.5in}}%(JRL) 



\item \textit{A student is speaking to her instructor: }I was late for class because the battery in my mobile phone, which I was using as an alarm clock, ran out.
\answerblank{Explaining. The instructor already knows the student is late for class. \\

\begin{tikzpicture}
\path
	(0,0) node [anchor=west] {1. I use my mobile phone for an alarm clock.}
	(0,-11pt) node [anchor=west] {2. The battery on my phone ran out.}
	(9,-19pt) node [anchor=west]{E}
	(0,-30pt) node[anchor=west] {3. I was late to class.};
\draw (.5,-19pt) -- (9,-19pt);
\end{tikzpicture}
}{\vspace{1.5in}}

\item There is a lot of positive talk concerning parenthood because people tend to think about the positive effects that have a child brings and they tend to exclude the numerous negatives that it brings.
\answerblank{Explaining. The flag word ``because'' indicates reasoning and that the target comes first. The kind of reasoning here is likely explaining, because the target is a commonly held belief. \\

\begin{tikzpicture}
\path
	(0,0) node [anchor=west] {1. People tend to think about the positive effects that have a child brings.}
	(0,-11pt) node [anchor=west] {2. People tend to tend to exclude the numerous negatives that it brings.}
	(9,-19pt) node [anchor=west]{E}
	(0,-30pt) node[anchor=west] {3. There is a lot of positive talk concerning parenthood.};
\draw (.5,-19pt) -- (9,-19pt);
\end{tikzpicture}
}{\vspace{1.5in}}

\end{exercises}


\noindent\problempart Identify each of the passages below as an argument, an explanation, or neither, and justify your answer. If the passage is an argument write it in standard form, with premises marked P$_1$ etc., then a line, and then the conclusion marked with a C. If the argument is an explanation, write it in the standard form for an explanation, with the explainers numbered and an ``E'' after the line that separates the explainers and the explainee. If the argument is neither an argument nor an explanation, state what kind of nonargument you think it is, such as a narrative or an expository passage.

 
\begin{exercises}

\item You have to be smart to understand the rules of Dungeons and Dragons. Most smart people are nerds. So, I bet most people who play D\&D are nerds.

% use this passage as a basis for some problems
%Notice that knowledge of an explanation can be used (on a different occasion) to make an argument for the truth of a conclusion. For example, if extremely cold weather in Europe is explained by the movement of air from Siberia, on a future occasion the movement of air from Siberia can be used to argue that it is or will be extremely cold. 

%also this one
%
%\begin{quote} The IPCC, a panel of experts from various countries, has stated that human activity has an impact on climate. So, that's how it is.\end{quote}

%In this passage, a speaker provides a reason for believing \textit{that} human activity has an impact on climate, namely, that an international panel believes so. That is, the speaker provides a premise which might justify adopting the conclusion as a belief. This premise, however, it does not explain \textit{why }or \textit{how}human activity impacts climate. It might thus be a justification, but it could not be used as an explanation. If a speaker says something is so because some source says it, you are looking at an argument. 



\item \textit{A coach is emailing parents in a neighborhood youth soccer league.} The game is canceled since it is raining heavily.


\item  \textit{At the market. }You know, granola bars generally aren't healthy. The ingredients include lots of processed sugars.

\item  \textit{At the pet store.}
\vspace{-8pt}
\begin{adjustwidth}{2em}{0em}
\begin{longtabu}{p{.1\linewidth}p{.8\linewidth}}
\textbf{Salesman}:     &A small dog makes just as effective a guard dog for your 
home as a big dog.\\
\textbf{Henry}:        &   No way!\\
\textbf{Salesman}: &    It might seem strange. But smaller ``yappy'' dogs bark readily and they also generate distinctive higher-pitched sounds. Most of a dog's effectiveness as a guard is due to making a sound, not physical size. \end{longtabu}
\end{adjustwidth}
\vspace{-.9cm}

\item \textit{A child is thinking out loud. }I think my cat must be dead. It isn't in any of its usual places. And when I asked my mother if she had seen it, she couldn't look me in the eyes. 

\item {\color{white}flurm}
\vspace{-24pt}
\begin{adjustwidth}{2em}{0em}
\begin{longtabu}{p{.1\linewidth}p{.8\linewidth}}
\textbf{Smith:} & I can solve any puzzle more quickly than you.\\
\textbf{Jones:}& Get out of here. \\
\textbf{Smith:} & It's true! I'm a member of MENSA, and you're not. 
\end{longtabu}
\end{adjustwidth}
\vspace{-.9cm}

\item \textit{In the comments on a biology blog: }According to Darwin's theory, my ancestors were monkeys. But since that's ridiculous, Darwin's theory is false. 

\item If you believe in [the Christian] God and turn out to be incorrect, you have lost nothing. But if you don't believe in God and turn out to be incorrect, you will go to hell. Believing in God is better in both cases. One should therefore believe in God. (A formulation of ``Pascal's Wager'' by Blaise Pascal.) 

\item \textit{Bill and Henry are in Columbus.}
\vspace{-8pt}
\begin{adjustwidth}{2em}{0em}
\begin{longtabu}{p{.1\linewidth}p{.8\linewidth}}
\textbf{Bill:} & Good news---I just accepted a job offer in Omaha. \\
\textbf{Henry:} & That's great. Congratulations! I suppose this means you'll be leaving us, then?\\
\textbf{Bill:} & Yes, I'll need to move sometime before September.  \\
\end{longtabu}
\end{adjustwidth}
\vspace{-.9cm}

\item You already know that God kicked humanity out of Eden before they could eat of the tree of life but only after they had eaten of the tree of knowledge of good and evil. That was because Satan wanted to take over God's throne and was responsible for their eating from the tree. If humans had eaten of both trees they could have been a threat to God. 

\end{exercises}
}


% *****************************************
% *  		Recognizing Arguments in Wild		*
% *****************************************

% a section for working with newspapers and field projects.

%
%\section{Recognizing Arguments in Wild}
%%When faced with a passage or dialogue, you must first determine whether or not it contains reasoning, and in particular whether the reasons involved are reasons-to-believe or reasons-which-explain. 
%
%%reiterate flag words. 
% 
%       
%%      There are an infinitely large number of flag words and phrases.
%%
%%      These flag words and phrases indicate reasoning because they indicate premises or target, but they do not distinguish between arguing and explaining. Moreover, passages sometimes do not have any flag words. So, we need other ways of telling whether and what kind of reasoning is taking place.
%
%
%Discuss replacing pronouns, making each statement stand on its own, paraphrasing for length. Use lots of real world examples.
%
% 
%   Discuss reports of arguments here. 
%
%Arguments vs. explanations (again)
%
%``Should'' statements in the conclusion are generally a sign of arguing. Predictions are a sign of arguing.
%
%
%\begin{quote}Highway repairs begin downtown today. And a bridge lift is scheduled for the middle of rush hour. I predict that traffic is going to be terrible.\end{quote}
%
%\begin{quote}Yeah, I know traffic is going to be terrible. It's because repairs begin downtown today. And a bridge lift is scheduled for the middle of rush hour.\end{quote}
%
%The words ``I predict'' in the first passage suggest the conclusion is a novel belief, (in fact, it's novel even to the speaker). The second passage starts out with the speaker saying ``I know'' about what is clearly the target, because of the reasons offered subsequently. In the first, therefore, the speaker is making an inference and trying to convince someone (perhaps herself) that the proposition ``Traffic is going to be terrible.'' is true. The second, on the other hand, is an explanation. The speaker is not trying to increase her (or anyone else's) store of knowledge; she is trying to describe connections between states of affairs.
%
%      Sometimes you need to use your knowledge of what various specific people know, as well as your general knowledge of the knowledge that people have, including your knowledge of what you can reasonably expect people or different ages (children, teens, adults) or different backgrounds (people from your own country or region as opposed to foreigners) and so on. This is called epistemic score-keeping.
%      




\section*{Key Terms}
\begin{multicols}{2}
\begin{sortedlist}
\sortitem{Logic}{}
\sortitem{Metareasoning}{}
\sortitem{Metacognition}{} 	
\sortitem{Content neutrality}{}
\sortitem{Formal logic}{}
\sortitem{Critical thinking}{}
\sortitem{Informal logic}{}
\sortitem{Rhetoric}{}
\sortitem{Standard form}{}
\sortitem{Conclusion indicator}{} 
\sortitem{Premise indicator}{}
\sortitem{Statement}{}
\sortitem{Argument}{}
\sortitem{Conclusion}{}
\sortitem{Premise}{}
\sortitem{Inference}{}
\sortitem{Simple statement of belief}{}
\sortitem{Expository passage}{}
\sortitem{Narrative}{}
\sortitem{Explanation}{}
\sortitem{Explainer}{}
\sortitem{Explainee}{}
\sortitem{Reason}{}
\sortitem{Target proposition}{}
\sortitem{Critical thinker}{}
\sortitem{Practical argument}{}
\iflabelexists{CTVersion}{\sortitem{Descriptive statement}{}}{}
\iflabelexists{CTVersion}{\sortitem{Normative statement}{}}{}
\end{sortedlist}
\end{multicols}	%label in text to remove explanation section
%\chapter{The Basics of Evaluating Argument}
\markright{Ch. \ref{chap:basicevaluation}: The Basics of Evaluating Argument}
\label{chap:basicevaluation}
\setlength{\parindent}{1em}

% **************************************************** 	
% *			Two ways that arguments can go wrong			*
% ****************************************************


\section{Two Ways an Argument Can Go Wrong}
\label{sec:two_ways}

Arguments are supposed to lead us to the truth, but they don't always succeed. There are two ways they can fail in their mission. First, they can simply start out wrong, using false premises. Consider the following argument. 

\begin{earg*}
\item It is raining heavily.
\item If you do not take an umbrella, you will get soaked.
\itemc You should take an umbrella.
\end{earg*}

If premise (1) is false---if it is sunny outside---then the argument gives you no reason to carry an umbrella.The argument has failed its job. Premise (2) could also be false: Even if it is raining outside, you might not need an umbrella. You might wear a rain poncho or keep to covered walkways and still avoid getting soaked. Again, the argument fails because a premise is false.

Even if an argument has all true premises, there is still a second way it can fail. Suppose for a moment that both the premises in the argument above are true. You do not own a rain poncho. You need to go places where there are no covered walkways. Now does the argument show you that you should take an umbrella? Not necessarily. Perhaps you enjoy walking in the rain, and you would like to get soaked. In that case, even though the premises were true, the conclusion would be false. The premises, although true, do not \emph{support} the conclusion. Back on page \pageref{def:Inference} we defined an inference, and said  it was like argument glue: it holds the premises and conclusion together. When an argument goes wrong because the premises do not support the conclusion, we say there is something wrong with the inference. %When there is something wrong with the inference, that means there is something wrong with the \emph{logical form} of the argument: Premises of the kind given do not necessarily lead to a conclusion of the kind given. We will be interested primarily in the logical form of arguments. We will learn to identify bad inferences by identifying bad logical forms. 

Consider another example: 

\begin{earg*}
\item You are reading this book.
\item This is a logic book.
\itemc[.3] You are a logic student.
\end{earg*}

This is not a terrible argument. Most people who read this book are logic students. Yet, it is possible for someone besides a logic student to read this book. If your roommate picked up the book and thumbed through it, they would not immediately become a logic student. So the premises of this argument, even though they are true, do not guarantee the truth of the conclusion. Its inference is less than perfect.

Again, for any argument, there are two ways that it could fail. First, one or more of the premises might be false.  Second, the premises might fail to support the conclusion. Even if the premises were true, the form of the argument might be weak, meaning the inference is bad.


%  ********************************************************
% *				Valid, Sound									* 
% ********************************************************

\section{Valid, Sound}

\newglossaryentry{valid}
{
name=valid,
description={A property of arguments where it is impossible for the premises to be true and the conclusion false.}
}

In logic, we are mostly concerned with evaluating the quality of inferences, not the truth of the premises. The truth of various premises will be a matter of whatever specific topic we are arguing about, and, as we have said, logic is content neutral.

The strongest inference possible would be one where the premises, if true, would somehow force the conclusion to be true. This kind of inference is called valid. There are a number of different ways to make this idea of the premises forcing the truth of the conclusion more precise. Here are a few:
 
An argument is valid if and only if\ldots 
\begin{enumerate}[label=(\alph*)]
\item it is impossible to consistently both (i) accept the premises and (ii) reject the conclusion

\item \label{itm:our_def} it is impossible for the premises to be true and the conclusion false

\item \label{itm:necessary} the premises, if true, would necessarily make the conclusion true.

\item \label{itm:imagination} the conclusion is true in every imaginable scenario in which the premises are true

\item \label{itm:story} it is impossible to write a consistent story (even fictional) in which the premises are true and the conclusion is false

\end{enumerate} 
 
In the glossary, we formally adopt item \ref{itm:our_def} as the definition for this textbook: an argument is \textsc{\gls{valid}} \label{def:valid} if and only if it is impossible for the premises to be true and the conclusion false.  However, nothing will really ride on the differences between the definitions in the list above, and we can look at all of them in order to give us a sense of what logicians mean when they use the term ``valid''.  
 
The important thing to see is that all the definitions in the list above try to get at what \textit{would} happen if the premises were true. None of them assert that the premises actually \textit{are} true. This is why definitions \ref{itm:imagination} and \ref{itm:story} talk about what would happen if you somehow \textit{pretend} the premises are true, for instance by telling a story. The argument is valid if, when you pretend the premises are true, you also have to pretend the conclusion is true. Consider the argument in Figure \ref{fig:Gaga_valid}


\begin{figure}
\begin{mdframed}[style=mytablebox]
\begin{earg*}
\item Lady Gaga is from Mars. 
\itemc[.4] Lady Gaga is from the fourth planet from our sun.
\end{earg*}
\end{mdframed}
\caption{A \textbf{valid} argument.} \label{fig:Gaga_valid}
\end{figure}

The American pop star Lady Gaga is not from Mars. (She's from New York City.) Nevertheless, if you imagine she's from Mars, you simply have to imagine that she is from the fourth planet from our sun, because mars simply is the fourth planet form our sun. Therefore this argument is valid. 

This way of understanding validity is based on what you can imagine, but not everyone is convinced that the imagination is a reliable tool in logic. That is why definitions like \ref{itm:necessary} and \ref{itm:our_def} talk about what is necessary or impossible. If the premises are true, the conclusion necessarily must be true. Alternately, it is impossible for the premises to be true and the conclusion false. The idea here is that instead of talking about the imagination, we will just talk about what can or cannot happen at the same time. The fundamental notion of validity remains the same, however: the truth of the premises would simply guarantee the truth of conclusion. 

So, assessing validity means wondering about whether the conclusion would be true \textit{if} the premises were true. This means that valid arguments can have false conclusions. This is important to keep in mind because people naturally tend to think that any argument must be good if they agree with the conclusion. And the more passionately people believe in the conclusion, the more likely we are to think that any argument for it must be brilliant. Conversely, if the conclusion is something we don't believe in, we naturally tend to think the argument is poor. And the more we don't like the conclusion, the less likely we are to like the argument. 



%\newglossaryentry{myside fallacy}
%{
%name=myside fallacy,
%description={The common mistake of evaluating an argument based merely on whether one agrees or disagrees with the conclusion.}
%}


\newglossaryentry{cognitive bias}
{
name=cognitive bias,
description={a habit of reasoning that can become dysfunctional in certain circumstances. Often these biases are not a matter of explicit belief. See also \emph{fallacy}}
}

But this is not the correct way to evaluate inferences at all. The quality of the inference is entirely independent of the truth of the conclusion. You can have great arguments for false conclusions and horrible arguments for true conclusions. We have trouble seeing this because of biases built deep in the way we think called ``cognitive biases.'' A  \textsc{\gls{cognitive bias}}\label{def:cognitive_bias} is a habit of reasoning that can be dysfunctional in certain circumstances. Generally these biases developed for a reason, so they serve us well in many or most circumstances. But cognitive biases also systematically distort our reasoning in other circumstances, so we must be on guard against them.

\newglossaryentry{confirmation bias}
{
name=confirmation bias,
description={The tendency to discount or ignore evidence and arguments that contradict one's current beliefs.}
}

There is a particular cognitive bias that makes it hard for us to recognize when a poor argument is being given for a conclusion we agree with. It is called ``confirmation bias'' and it is in many ways the mother of all cognitive biases.  \textsc{\Gls{confirmation bias}} \label{def:confirmation_bias} is the tendency to discount or ignore evidence and arguments that contradict one's current beliefs. It really pervades all of our thinking, right down to our perceptions. \iflabelexists{part:CT}{We will learn more about cognitive biases in Chapter \ref{Chap:what_is_ct}.}{}

Because of confirmation bias, we need to train ourselves to recognize valid arguments for conclusions we think are false. Remember, an argument is valid if it is impossible for the premises to be true and the conclusion false. This means that you can have valid arguments with false conclusions, they just have to also have false premises. Consider the example in Figure \ref{fig:valid_oranges}


\begin{figure}[t]
\begin{mdframed}[style=mytablebox]
\begin{earg*}
\item Oranges are either fruits or musical instruments.
\item Oranges are not fruits.
\itemc Oranges are musical instruments.
\end{earg*}
\end{mdframed}
\caption{A \textbf{valid} argument} \label{fig:valid_oranges}
\end{figure}

\label{valid_arg_false_premises}

The conclusion of this argument is ridiculous. Nevertheless, it follows validly from the premises. This is a valid argument. \emph{If} both premises were true, \emph{then} the conclusion would necessarily be true.

This shows that a valid argument does not need to have true premises or a true conclusion. Conversely, having true premises and a true conclusion is not enough to make an argument valid. Consider the example in Figure \ref{fig:invalid_paris}


\begin{figure}[b]
\begin{mdframed}[style=mytablebox]
\begin{earg*}
\item London is in England.
\item Beijing is in China.
\itemc[.3] Paris is in France.
\end{earg*}
\end{mdframed}
\caption{An \textbf{invalid} argument.} \label{fig:invalid_paris}
\end{figure}
\label{invalid_true_premises_and_conclusion}


\newglossaryentry{invalid}
{
name=invalid,
description={A property of arguments that holds when the premises do not force the truth of the conclusion. The opposite of valid.}
}
 

The premises and conclusion of this argument are, as a matter of fact, all true. This is a terrible argument, however, because the premises have nothing to do with the conclusion. Imagine what would happen if Paris declared independence from the rest of France. Then the conclusion would be false, even though the premises would both still be true. Thus, it is \emph{logically possible} for the premises of this argument to be true and the conclusion false. The argument is not valid.  If an argument is not valid, it is called \textsc{\gls{invalid}}. \label{def:invalid} As we shall see, this term is a little misleading, because less than perfect arguments can be very useful. But before we do that, we need to look more at the concept of validity.

In general, then, the \textit{actual }truth or falsity of the premises, if known, do not tell you whether or not an inference is valid. There is one exception: when the premises are true and the conclusion is false, the inference cannot be valid, because valid reasoning can only yield a true conclusion when beginning from true premises. 
 
Figure \ref{fig:invalid_animals} has another invalid argument:

\begin{figure}
\begin{mdframed}[style=mytablehalfbox]
\begin{earg*}
\item All dogs are mammals
\item All dogs are animals
\itemc All animals are mammals.
\end{earg*}
\end{mdframed}
\caption{An \textbf{invalid} argument.} \label{fig:invalid_animals}
\end{figure}

In this case, we can see that the argument is invalid by looking at the truth of the premises and conclusion. We know the premises are true. We know that the conclusion is false. This is the one circumstance that a valid argument is supposed to make impossible. 

Some invalid arguments are hard to detect because they resemble valid arguments. Consider the one in Figure \ref{fig:invalid_stimulus}

\begin{figure}[b]
\begin{mdframed}[style=mytablebox]
\begin{earg*}
\item An economic stimulus package will allow the U.S. to avoid a depression. 
\item There is no economic stimulus package
\itemc[.3] The U.S. will go into a depression. 
\end{earg*}
\end{mdframed}
\caption{An \textbf{invalid} argument} \label{fig:invalid_stimulus}
\end{figure}


This reasoning is not valid since the premises do not \textit{definitively} support the conclusion. To see this, assume that the premises are true and then ask, "Is it possible that the conclusion could be false in such a situation?". There is no inconsistency in taking the premises to be true without taking the conclusion to be true. The first premise says that the stimulus package will allow the U.S. to avoid a depression, but it does not say that a stimulus package is the \textit{only }way to avoid a depression. Thus, the mere fact that there is no stimulus package does not necessarily mean that a depression will occur. 

\newglossaryentry{fallacy}
{
name=fallacy,
plural=fallacies,
description={A common mistake in reasoning. Fallacies are generally conceived of as mistake forms of inference and are generally explained by arguments represented in standard form. See also \emph{cognitive bias}.}
}

When an argument resembles a good argument but is actually a bad one, we say it is a .\textsc{\gls{fallacy}}\label{def:fallacy}. Fallacies are similar to cognitive biases, in that they are ways our reasoning can go wrong.  Fallacies, however, are always mistakes you can explicitly lay out as arguments in standard form, as above. \iflabelexists{part:CT}{We will learn more about fallacies in Chapter \ref{Chap:what_is_ct}.}{}

Here is another, trickier, example. I will give it first in ordinary language. 

\begin{quotation} \noindent\textit{A pundit is speaking on a cable news show} If the U.S. economy were in recession and inflation were running at more than 4\%, then the value of the U.S. dollar would be falling against other major currencies. But this is not happening --- the dollar continues to be strong. So, the U.S. is not in recession. \end{quotation}

The conclusion is "The U.S. economy is not in recession." If we put the argument in standard form, it looks like figure \ref{fig:invalid_recession}

\begin{figure}
\begin{mdframed}[style=mytablebox]
\begin{earg*}
\item If the U.S. were in a recession with more than 4\% inflation, then the dollar would be falling
\item The dollar is not falling
\itemc[.3] The U.S. is not in a recession. 
\end{earg*}
\end{mdframed}
\caption{An \textbf{invalid} argument} \label{fig:invalid_recession}
\end{figure}

The conclusion does not follow necessarily from the premises. It does follow necessarily from the premises that (i) the U.S. economy is not in recession or (ii) inflation is running at more than 4\%, but they do not guarantee (i) in particular, which is the conclusion. For all the premises say, it is possible that the U.S. economy is in recession but inflation is less than 4\%. So, the inference does not \textit{necessarily} establish that the U.S. is not in recession. A parallel inference would be "Jack needs eggs and milk to make an omelet. He can't make an omelet. So, he doesn't have eggs.". 

\newglossaryentry{sound}
{
name=sound,
description={A property of arguments that holds if the argument is valid and has all true premises.}
}

If an argument is not only valid, but also has true premises, we call it \textsc{\gls{sound}}. \label{def:sound} ``Sound'' is the highest compliment you can pay an argument. If logic is the study of virtue in argument, sound arguments are the most virtuous. We said in Section \ref{sec:two_ways} that there were two ways an argument could go wrong, either by having false premises or weak inferences. Sound arguments have true premises and undeniable inferences. If someone gives a sound argument in a conversation, you have to believe the conclusion, or else you are irrational.  

The argument on the left in Figure \ref{fig:valid_sound} is valid, but not sound. The argument on the right is both valid and sound.

\begin{figure}[b]
\begin{mdframed}[style=mytablebox]
\begin{longtabu}{X[l,c]X[l,c]}
\vspace{-16pt}
\begin{earg*}
\item Socrates is a person.
\item All people are carrots.
\itemc[.5] Therefore, Socrates is a carrot.
\end{earg*}
&
\vspace{-16pt}
\begin{earg*}
\item Socrates is a person.
\item All people are mortal.
\itemc[.5] Therefore, Socrates is mortal.
\end{earg*}
\\
\textbf{Valid, but not sound}&
\textbf{Valid and sound}
\end{longtabu}
\end{mdframed}
\caption{These two arguments are valid, but only the one on the right is sound} \label{fig:valid_sound}
\end{figure}

Both arguments have the exact same form. They say that a thing belongs to a general category and everything in that category has a certain property, so the thing has that property. Because the form is the same, it is the same valid inference each time. The difference in the arguments is not the validity of the inference, but the truth of the second premise. People are not carrots, therefore the argument on the left is not sound. People are mortal, so the argument on the right is sound. 

Often it is easy to tell the difference between validity and soundness if you are using completely silly examples. Things become more complicated with false premises that you might be tempted to believe, as in the argument in Figure \ref{fig:valid_unsound}.

\begin{figure}
\begin{mdframed}[style=mytablehalfbox]
\begin{earg*}
\item Every Irishman drinks Guinness
\item Smith is an Irishman
\itemc Smith drinks Guinness.
\end{earg*}
\end{mdframed}
\caption{An argument that is \textbf{valid} but not \textit{sound}} \label{fig:valid_unsound}
\end{figure}


You might have a general sense that the argument in Figure \ref{fig:valid_unsound} is bad---you shouldn't assume that someone drinks Guinness just because they are Irish. But the argument is completely valid (at least when it is expressed this way.) The inference here is the same as it was in the previous two arguments. The problem is the first premise. Not all Irishmen drink Guinness, but if they did, and Smith was an Irishman, he would drink Guinness. 

The important thing to remember is that validity is not about the actual truth or falsity of the statements in the argument. Instead, it is about the way the premises and conclusion are put together. It is really about the \emph{form} of the argument. A valid argument has perfect logical form. The premises and conclusion have been put together so that the truth of the premises is incompatible with the falsity of the conclusion. 

A general trick for determining whether an argument is valid is to try to come up with just one way in which the premises could be true but the conclusion false. If you can think of one (just one! anything at all! but no violating the laws of physics!), the reasoning is \textit {invalid.}    
 

% Practice Problems %%%%%%%%%%%%%%%

\practiceproblems

\noindent\problempart  For each passage, (i) put the argument in standard form and (ii) say whether it is valid or invalid.

\begin{longtabu}{X[1,l,p]X[.15,l,p]X[8.5,l,p]}

\textbf{Example}: & \multicolumn{2}{p{.9\linewidth}}{\textit{Monica is looking for her coworker} Jack is in his office. Jack's office is on the second floor. So, Jack is on the second floor.} \\
\\
\textbf{Answer}: & (i) & {\color{white}.} \vspace{-22pt} \begin{earg*}
\item Jack is in his office. 
\item Jack's office is on the second floor.
\itemc Jack is on the second floor.
\end{earg*} \\
& (ii) & Valid 
\end{longtabu}

\begin{exercises}
\item All dinosaurs are people, and all people are fruit. Therefore all dinosaurs are fruit. 

\answerblank{
\begin{enumerate}[label=(\roman*)]
\item {\color{white}.} \vspace{-13pt} \begin{earg*}
\item All dinosaurs are people
\item All people are fruit. 
\itemc[.4] All dinosaurs are fruit. 
\end{earg*}
\item Valid
\end{enumerate}
}{\vspace{1.5in}}
%% F, F, conclusion last


\item All people are mortal. Socrates is mortal. Therefore all people are Socrates. 
\answerblank{
\begin{enumerate}[label=(\roman*)]
\item {\color{white}.} \vspace{-13pt} \begin{earg*}
\item  All people are mortal.
\item Socrates is mortal. 
\itemc[.4]  All people are Socrates. 
\end{earg*}
\item Invalid
\end{enumerate}}{\vspace{1.5in}}
%%Formal fallacy



\item All dogs are mammals. Therefore, Fido is a mammal, because Fido is a dog.  

\answerblank{
\begin{enumerate}[label=(\roman*)]
\item {\color{white}.} \vspace{-13pt} \begin{earg*}
\item All dogs are mammals
\item Fido is a dog.   
\itemc[.4] Fido is a mammal, 
\end{earg*}
\item Valid
\end{enumerate}
}{\vspace{1.5in}}
%%Made up, conclusion middle

\item Abe Lincoln must have been from France, because he was either from France or from Luxemborg, and we know was not from Luxemborg. 
\answerblank{
\begin{enumerate}[label=(\roman*)]
\item {\color{white}.} \vspace{-13pt} \begin{earg*}
\item Abe Lincoln was either from France or from Luxemborg
\item Abe Lincoln was not from Luxemborg. 
\itemc[.4] Abe Lincoln was from France. 
\end{earg*}
\item Valid
\end{enumerate}
}{\vspace{1.5in}}
%%F, F, conclusion first


\item If the world were to end today, then I would not need to get up tomorrow morning. I will need to get up tomorrow morning. Therefore, the world will not end today.

\answerblank{
\begin{enumerate}[label=(\roman*)]
\item {\color{white}.} \vspace{-13pt} \begin{earg*}
\item If the world were to end today, then I would not need to get up tomorrow morning.
\item I will need to get up tomorrow morning.
\itemc[.4]  The world will not end today.
\end{earg*}
\item Valid
\end{enumerate}
}{\vspace{1.5in}}
%%Made up

\item If the triceratops were a dinosaur, it would be extinct. Therefore, the triceratops is extinct, because the triceratops was a dinosaur. 
\answerblank{
\begin{enumerate}[label=(\roman*)]
\item {\color{white}.} \vspace{-13pt} \begin{earg*}
\item  If the triceratops were a dinosaur, it would be extinct.
\item The triceratops was a dinosaur.
\itemc[.4] The triceratops is extinct  
\end{earg*}
\item Valid
\end{enumerate}
}{\vspace{1.5in}}
%%T, T, conclusion middle

\item If George Washington was assassinated, he is dead. George Washington is dead. Therefore George Washington was assassinated.
\answerblank{
\begin{enumerate}[label=(\roman*)]
\item {\color{white}.} \vspace{-13pt} \begin{earg*}
\item  If George Washington was assassinated, he is dead.
\item George Washington is dead.
\itemc[.4] George Washington was assassinated.
\end{earg*}
\item Invalid
\end{enumerate}
}{\vspace{1.5in}}
%% Formal fallacy
%
\item Jack prefers Pepsi to Coke. After all, about 52\% of people prefer Pepsi to Coke, and Jack is a person. 

\answerblank{
\begin{enumerate}[label=(\roman*)]
\item {\color{white}.} \vspace{-13pt} \begin{earg*}
\item  About 52\% of people prefer Pepsi to Coke
\item Jack is a person. 
\itemc[.4] Jack prefers Pepsi to Coke.
\end{earg*}
\item invalid
\end{enumerate}
}{\vspace{1.5in}}
%%Inductive, conclusion first 

\item \textit{Steve thinks about the consequences of laziness.}  If I don't mow the lawn, it will become a haven for all kinds of exotic insect species. If the lawn becomes a haven for all kinds of exotic insect species, I will be protecting biodiversity. Therefore, if I don't mow the lawn, I'll be protecting biodiversity.

\answerblank{
\begin{enumerate}[label=(\roman*)]
\item {\color{white}.} \vspace{-15pt} \begin{earg*} 
\item  If I don't mow the lawn, it will become a haven for insects.
\item  If the lawn becomes a haven for insects, I will be protecting biodiversity 
\itemc[.6]  If I don't mow the lawn, I will be protecting biodiversity.
\end{earg*}
\item Valid. If the premises were true, the conclusion would have to be true.

In general, the argument has the form, 

\begin{earg*}
\item If $A$, then $B$
\item If $B$, then $C$
\itemc[.2] If $A$ then $C$.
\end{earg*}
which is valid. 

\end{enumerate}
}{\vspace{1.5in}}

%conclusion last

\item \textit{A forest ranger is surveying the park} I can tell that bears have been down by the river, because there are tracks in the mud. Tracks like these are made by bears in almost every case. 

\answerblank{
\begin{enumerate}[label=(\roman*)]
\item {\color{white}.} \vspace{-13pt} \begin{earg*}
\item  There are tracks in the mud.
\item Tracks like these are made by bears in almost every case. 
\itemc[.4]  Bears have been down by the river
\end{earg*}
\item Invalid
\end{enumerate}
}{\vspace{1.5in}}
%%Inductive, conclusion first 

\end{exercises}

%part B
\noindent\problempart For each passage, (i) put the argument in standard form and (ii) say whether it is valid or invalid.
\answer{Answers by Ben Sheredos}
\begin{exercises}
\item Cindy Lou Who lives in Whoville. You can tell because Cindy Lou Who is a Who, and all Whos live in Whoville.  

\answer{ 
	\begin{earg*} 
		\item Cindy Lou Who is a Who.
		\item All Whos live in Whoville.
		\itemc Cindy Lou Who lives in Whoville.
	\end{earg*}
Valid}
%Made up, conclusion first

\item If Frog and Toad like each other, they are friends. Frog and Toad like each other. Therefore, Frog and Toad are friends. 
\answer{
	\begin{earg*} 
		\item If Frog and Toad like each other, they are friends.
		\item Frog and Toad like each other.
		\itemc Frog and Toad are friends.
	\end{earg*}
Valid}
%Made up

\item If Cindy Lou Who is no more than two, then she is not five years old. Cindy Lou Who is not five. Therefore Cindy Lou Who is two or more.
\answer{
	\begin{earg*} 
		\item If Cindy Lou Who is no more than two, then she is not five years old.
		\item Cindy Lou Who is not five.
		\itemc Cindy Lou Who is two or more.
	\end{earg*}
Invalid. This starts out as a formal fallacy, affirming the consequent. Then it goes even further wrong by swapping "no more than two" for "two or more."}
% Formal fallacy

\item \textit{Jack's suspicious house mate is in the kitchen} Jack has moved my leftover slice of pizza. Jack must have moved it, because Jack is the only person who has been in the house, and the pizza is no longer in the fridge.
\answer{ 
	\begin{earg*} 
		\item Jack is the only person who has been in the house, and the pizza is no longer in the fridge,
		\itemc Jack has moved my leftover slice of pizza.
	\end{earg*}
Alternatively:
	\begin{earg*} 
	\item Jack is the only person who has been in the house.
	\item The pizza is no longer in the fridge.
	\itemc Jack has moved my leftover slice of pizza.
	\end{earg*}
Invalid. Maybe pets or robots can open the fridge? Or possibly, Jack opened the fridge, but did so with his cell phone is his hand and right at that moment received a funny video from a friend and left the fridge door open while he watched it, allowing the dog to steal the pizza. Not likely, admittedly, but if it could happen, the argument is \underline{not valid}.}
% Inductive, conclusion first
 
\item Jack is Smith's work colleague. So, Jack and Smith are friends.
\answer{ 
	\begin{earg*} 
		\item Jack is Smith's work colleague.
		\itemc Jack and Smith are friends.
	\end{earg*}
Invalid. Not all coworkers are friends.}

%Inductive 

\item Abe Lincoln was either born in Illinois or he was once president. Therefore Abe Lincoln was born in Illinois, because he was never president. 
\answer{ 
	\begin{earg*} 
		\item Lincoln was either born in Illinois or he was once president.
		\item Lincoln was never president.
		\itemc Lincoln was born in Illinois.
	\end{earg*}
Valid. Probably every statement here is false, but what matters is that IF the premises were true, the conclusion would have to be true.}
%F, T, conclusion middle

\item Politicians get a generous allowance for transportation costs. Enda Kenny is a politician. Therefore Kenny gets a generous transportation allowance.
\answer{ 
	\begin{earg*} 
		\item Politicians get a generous allowance for transportation costs.
		\item Enda Kenny is a politician.
		\itemc Kenny gets a generous transportation allowance.
	\end{earg*}
Valid. If the plural "politicians" is understood to mean "all politicians" rather than "most", the inference is valid. If you wrote "invalid" and explained that you thought "politicians" only meant "most politicians," that would be OK, as long as you made it clear. English is ambiguous like that.}

% Valid. If the plural "politicians" is understood to mean "all politicians" rather than "most", the inference is valid.
%might as well be made up

\item Jones is taller than Bill, because Smith is taller than Jones and Bill is shorter than Smith. 
\answer{
	\begin{earg*} 
		\item Smith is taller than Jones.
		\item Bill is shorter than Smith.
		\itemc Jones is taller than Bill.
	\end{earg*}
Invalid. Jones and Bill could be the same height.}

%Formal fallacy, conclusion first

\item If grass is green, then I am the pope. Grass is green. So, I am the pope.
\answer{
	\begin{earg*} 
		\item If grass is green, then I am the pope.
		\item Grass is green.
		\itemc I am the pope.
	\end{earg*}
Valid. IF premises are true, conclusion has to be.}

%F, F

\item Smith is paid more than Jack. They are both paid weekly. So, Smith has more money than Jack.
\answer{
	\begin{earg*} 
		\item Smith is paid more than Jack.
		\item Both Smith and Jones are paid weekly.
		\itemc Smith has more money than Jack.
	\end{earg*}
Invalid. There are sources of wealth other than what one is paid.}% Weak
\end{exercises}

%Part C
\noindent\problempart For each passage, (i) put the argument in standard form and (ii) say whether it is valid or invalid.

\begin{exercises}
\item Jack is close to the pond. The pond is close to the playground. So, Jack is close to the playground.
\answerblank{
\begin{enumerate}[label=(\roman*)]
\item {\color{white}.} \vspace{-13pt} \begin{earg*}
\item  Jack is close to the pond. 
\item The pond is close to the playground. 
\itemc[.4] Jack is close to the playground.
\end{earg*}
\item Invalid
\end{enumerate}
}{\vspace{1.5in}}
%% General fallacy

\item \textit{Jack is at work, and is unable to leave early} I have up to half an hour to get to the bank, because work ends at 5:00 and the bank closes at 5:30. 
\answerblank{
\begin{enumerate}[label=(\roman*)]
\item {\color{white}.} \vspace{-13pt} \begin{earg*}
\item  Work ends at 5:00
\item The bank closes at 5:30. 
\itemc[.4] I have up to half an hour to get to the bank
\end{earg*}
\item Valid
\end{enumerate}
}{\vspace{1.5in}}
%% Made up, conclusion first.

\item Jack and Gill ate at Guadalajara restaurant earlier and both of them feel nauseated now. So, something they ate there is making them sick.
\answerblank{
\begin{enumerate}[label=(\roman*)]
\item {\color{white}.} \vspace{-13pt} \begin{earg*}
\item  Jack and Gill ate at Guadalajara restaurant earlier
\item  Jack and Gill feel nauseated now. 
\itemc[.4] Something they ate there is making them sick.
\end{earg*}
\item Invalid
\end{enumerate}
}{\vspace{1.5in}}
%% Inductive

\item Zhaoquing must be west of Huizhou, because Zhaoquing is west of Guangzhou, which is west of Huizhou. 
\answerblank{
\begin{enumerate}[label=(\roman*)]
\item {\color{white}.} \vspace{-13pt} \begin{earg*}
\item   Zhaoquing is west of Guangzhou
\item  Guangzhou is west of Huizhou.  
\itemc[.4] Zhaoquing is west of Huizhou
\end{earg*}
\item Valid
\end{enumerate}
}{\vspace{1.5in}}
%%T, T (?), conclusion first
%
\item \textit{Henry can't find his glasses. }I remember I had them when I came in from the car. So, they are in the house somewhere.
\answerblank{
\begin{enumerate}[label=(\roman*)]
\item {\color{white}.} \vspace{-13pt} \begin{earg*}
\item   Henry had his glasses when he came in from the car
\itemc[.4]  His glasses are in the house somewhere.
\end{earg*} 
\item Invalid
\end{enumerate}}{\vspace{1.5in}}
% False dilemma
%% General fallacy

\item I was talking about tall John---the one who is over 6'4''---but Jack was talking about short John, who is at most 5'2''. So, we were talking about two different Johns.
\answerblank{
\begin{enumerate}[label=(\roman*)]
\item {\color{white}.} \vspace{-13pt} \begin{earg*}
\item I was talking about tall John  
\item   Jack was talking about short John
\itemc[.4]  We were talking about two different Johns.
\end{earg*}\item Valid
\end{enumerate}
}{\vspace{1.5in}}
%% Made up

\item Tomorrow's trip to Ensenada will take about 10 hours, because the last time I drove there from here it took 10 hours. 
\answerblank{
\begin{enumerate}[label=(\roman*)]
\item {\color{white}.} \vspace{-13pt} \begin{earg*}
\item   The last time I drove to Ensenada from here it took 10 hours. 
\itemc[.4]  Tomorrow's trip to Ensenada will take about 10 hours
\end{earg*}\item Invalid
\end{enumerate}
}{\vspace{1.5in}}
%%Induction, conclusion first.
\end{exercises}

\noindent\problempart For each passage, (i) put the argument in standard form and (ii) say whether it is valid or invalid.
\answer{Answers by Ben Sheredos}
\begin{exercises}
\item \textit{Monica is surveying the crowd that showed up for her talk} There must be at least 150 people here. That's how many people the auditorium holds, and every seat is full and people are beginning to sit on the stairs at the side. 
\answer{
	\begin{earg*} 
		\item The auditor	ium holds 150 people.
		\item Every seat in the auditorium is full and people are beginning to sit on the stairs at the side.
		\itemc There are at least 150 people here.
	\end{earg*}
Valid}
% Made up, conclusion first

\item The fire bell in the building is ringing. There is sometimes a fire in the building when the alarm goes off. So, there is a fire.
\answer{
	\begin{earg*} 
		\item The fire bell in the building is ringing.
		\item There is sometimes a fire in the building when the alarm goes off.
		\itemc There is a fire (in the building).
	\end{earg*}
Invalid. "Sometimes" isn't "always," so the conclusion is not necessarily true.}
% Inductive

\item I cannot drive on the motorways yet, because I just passed my driving test and anyone who passes can drive on the roads but not on the motorway for six months.  
\answer{
	\begin{earg*} 
		\item I just passed my driving test.
		\item Anyone who passes can drive on the roads but not on the motorway for 6 months.
		\itemc I cannot drive on the motorways yet.
	\end{earg*}
Valid; it's implied pretty strongly that "just passing" means "passed within the past 6 months." There is room for equivocation here, but it looks pretty solid.}

% Made up

\item Yesterday's the temperature reached 91 degrees Fahrenheit. Today it is 94. So, today is warmer than yesterday.
\answer{
	\begin{earg*} 
		\item Yesterday the temp. reached 91F.
		\item Today the tepm. is 94F.
		\itemc Today is warmer than yesterday.
	\end{earg*}
Valid, unless the speaker inexplicably changes to a Celsius scale or something, but more likely the idea is that they just told you what scale they were using, and so they don't repeat it.}
% Made up

\item  My car is functioning well at the moment. So, all of the parts in my car are functioning well.
\answer{
	\begin{earg*} 
		\item My car is functioning well at the moment.
		\itemc All the parts of my car are functioning well.
	\end{earg*}
Probably invalid -- probably a fallacy of "Composition and Division." Suppose the speaker added, as P2: ''I mean, the antenna fell off, so I can't listen to Jazz 98.3, but I'll fix that later'' We wouldn't jump on them and say ''\textit{A-HA!} so your car \textit{isn't} functioning well!''}
\item It has been sunny every day for the last five days. So, it will be sunny today.

\answer{
	\begin{earg*} 
		\item It has been sunny every day for the past 5 days.
		\itemc It will be sunny today.
	\end{earg*}
Not valid in the logical sense defined here. Five days in a row is no guarantee that the sixth day will be the same.}

% Inductive

\item Jack is in front of Gill. So, Gill is behind Jack.
\answer{
	\begin{earg*} 
		\item Jack is front of Gill.
		\itemc Gill is behind Jack.
	\end{earg*}
Valid}

% made up

\item \textit{Gill is returning home}: The door to my house is still locked. So, my possessions are still inside.
\answer{
	\begin{earg*} 
		\item The door to my house is still locked.
		\itemc My possessions are still inside.
	\end{earg*}
Invalid. Oh simple, naive Gil. A pro would definitely pick the lock, rob you blind, and lock the door on the way out so as not to arouse suspicions. By now your possessions have been pawned, and the thief is halfway to Vegas.}

%Inductive.

\end{exercises}




% *******************************************
% *			Strong, Cogent, Deductive, Inductive	    *	
% *******************************************


\section{Strong, Cogent, Deductive, Inductive}

We have just seen that sound arguments are the very best arguments. Unfortunately, sound arguments are really hard to come by, and when you do find them, they often only prove things that were already quite obvious, like that Socrates (a dead man) is mortal. Fortunately, arguments can still be worthwhile, even if they are not sound. Consider this one:

\begin{earg*}
\item In January 1997, it rained in San Diego.
\item In January 1998, it rained in San Diego.
\item In January 1999, it rained in San Diego.
\itemc[.6] It rains every January in San Diego.
\end{earg*}


This argument is not valid, because the conclusion could be false even though the premises are true. It is possible, although unlikely, that it will fail to rain next January in San Diego. Moreover, we know that the weather can be fickle. No amount of evidence should convince us that it rains there \emph{every} January. Who is to say that some year will not be a freakish year in which there is no rain in January in San Diego? Even a single counterexample is enough to make the conclusion of the argument false.

\newglossaryentry{strong}
{
name=strong,
description={A property of arguments which holds when the premises, if true, mean the conclusion must be likely to be true.}
}


\newglossaryentry{cogent}
{
name=cogent,
description={A property of arguments that holds when the argument is strong and the premises are true.}
}



\newglossaryentry{weak}
{
name=weak,
description={A property of arguments that are neither valid nor strong. In a weak argument, the premises would not even make the conclusion likely, even if they were true.}
}



Still, this argument is pretty good. Certainly, the argument could be made stronger by adding additional premises: In January 2000, it rained in San Diego. In January 2001$\ldots$ and so on. Regardless of how many premises we add, however, the argument will still not be deductively valid. Instead of being valid, this argument is strong. An argument is \textsc{\gls{strong}} \label{def:strong} if the premises would make the conclusion more likely, were they true. In a strong argument, the premises don't guarantee the truth of the conclusion, but they do make it a good bet. If an argument is strong, and it has true premises, we say that it is \textsc{\gls{cogent}} \label{def:cogent} Cogency is the equivalent of soundness in strong arguments. If an inference is neither valid, nor strong, we say it is \textsc{\gls{weak}}. \label{def:weak}In a weak argument, the premises would not even make the conclusion likely, even if they were true.

You may have noticed that the word ``likely'' is a little vague. How likely do the premises have to make the conclusion before we can count the argument as strong? The answer is a very unsatisfying ``it depends.'' It depends on what is at stake in the decision to believe the conclusion. What happens if you are wrong? What happens if you are right? The phrase ``make the conclusion a good bet'' is really quite apt. Whether something is a good bet depends a lot on how much money is at stake and how much you are willing to lose. Sometimes people feel comfortable taking a bet that has a 50\% chance of doubling their money, sometimes they don't. 

The vagueness of the word ``likely'' brings out an interesting feature of strong arguments: some strong arguments are stronger than others. The argument about rain in San Diego, above, has three premises referring to three previous Januaries. The argument is pretty strong, but it can become stronger if we go back farther into the past, and find more years where it rains in January. The more evidence we have, the better a bet the conclusion is. Validity is not like this. Validity is a black-or-white matter. You either have it, and you're perfect, or you don't, and you're nothing. There is no point in adding premises to an argument that is already valid. 

\newglossaryentry{deductive}
{
name=deductive,
description={A style of arguing where one attempts to use valid arguments.}
}

\newglossaryentry{inductive}
{
name=inductive,
description={A style of arguing where one attempts to use strong arguments.}
}

Arguments that are valid, or at least try to be, are called \textsc{\gls{deductive}} \label{def:deductive}, and people who attempt to argue using valid arguments are said to be arguing \textit{deductively.} The notion of validity we are using here is, in fact, sometimes called \textit{deductive validity}. Deductive argument is difficult, because, as we said, in the real world sound arguments are hard to come by, and people don't always recognize them as sound when they find them. Arguments that purport to merely be strong rather than valid are called \textsc{\gls{inductive}}. \label{def:inductive} The most common kind of inductive argument includes arguments like the one above about rain in San Diego, which generalize from many cases to a conclusion about all cases.

Deduction is possible in only a few contexts. You need to have clear, fixed meanings for all of your terms and rules that are universal and have no exceptions.   One can find situations like this if you are dealing with things like legal codes, mathematical systems or logical puzzles. One can also create, as it were, a context where deduction is possible by imagining a universal, exceptionless rule, even if you know that no such rule exists in reality. In the example above about rain in San Diego, we can change the argument from inductive to deductive by adding a universal, exceptionless premise like ``It always rains in January in San Diego.'' This premise is unlikely to be true, but it can make the inference valid. (For more about trade offs between the validity of the inference and the truth of the premise, see the chapter on incomplete arguments in the complete version of this text. \label{ver_var} \nix{Chapter \ref{chap:incomplete_arguments}}

Here is an example in which the context is an artificial code --- the tax code: 

\begin{quotation} \noindent\textit{From a the legal code posted on a government website} A tax credit for energy-efficient home improvement is available at 30\% of the cost, up to \$1,500 total, in 2009 \& 2010, ONLY for existing homes, NOT new construction, that are your "principal residence" for Windows and Doors (including sliding glass doors, garage doors,~storm doors and storm windows), Insulation, Roofs (Metal and Asphalt), HVAC: Central Air Conditioners, Air Source Heat Pumps, Furnaces and Boilers, Water Heaters: Gas, Oil, \& Propane Water Heaters, Electric Heat Pump Water Heaters, Biomass Stoves. \end{quotation}

This rule describes the conditions under which a person can or cannot take a certain tax credit. Such a rule can be used to reach a valid conclusion that the tax credit can or cannot be taken.

As another example of an inference in an artificial situation with limited and clearly defined options, consider a Sudoku puzzle. The rules of Sudoku are that each cell contains a single number from 1 to 9, and each row, each column and each 9-cell square contain one occurrence of each number from 1 to 9. Consider the following partially completed board:

\begin{center}
\noindent \includegraphics*[width=2.45in, height=2.45in, keepaspectratio=false]{img/sudoku}
\end{center}

The following inference shows that, in the first column, a 9 must be entered below the 7:

\begin{quotation} The 9 in the first column must go in one of the open cells in the column. It cannot go in the third cell in the column, because there is already a 9 in that 9-cell square. It cannot go in the eighth or ninth cell because each of these rows already contains a 9, and a row cannot contain two occurrences of the same number. Therefore, since there must be a 9 somewhere in this column, it must be entered in the seventh cell, below the 7.\end{quotation}

The reasoning in this inference is valid: if the premises are true, then the conclusion must be true. Logic puzzles of all sorts operate by artificially restricting the available options in various ways. This then means that each conclusion arrived at (assuming the reasoning is correct) is necessarily true. 

One can also create a context where deduction is possible by imagining a rule that holds without exception. This can be done with respect to any subject matter at all. Speakers often exaggerate the connecting premise in order to ensure that the justificatory or explanatory power of the inference is as strong as possible. Consider Smith's words in the following passage: 


\begin{adjustwidth}{2em}{0em}
\begin{longtabu}{p{.1\linewidth}p{.8\linewidth}}
\textbf{Smith:} & I'm going to have some excellent pizza this evening. \\
\textbf{Jones:} & I'm glad to hear it. How do you know?\\
\textbf{Smith:} & I'm going to Adriatico's. They always make a great pizza. \\
\end{longtabu}
\end{adjustwidth}
\vspace{-1cm}

Here, Smith justifies his belief that the pizza will be excellent --- it comes from Adriatico's, where the pizza, he claims, is \textit{always }great: in the past, present and future. 

As stated by Smith, the inference that the pizza will be great this evening is valid. However, making the inference valid in this way often means making the general premise false: it's not likely that the pizza is great \textit{every single }time; Smith is overstating the case for emphasis. Note that Smith does not need to use a universal proposition in order to convince Jones that the pizza will \textit{very likely} be good. The inference to the conclusion would be strong (though not valid) if he had said that the pizza is "almost always" great, or that the pizza has been great on all of the many occasions he has been at that restaurant in the past. The strength of the inference would fall to some extent---it would not be guaranteed to be great this evening---but a slightly weaker inference seems appropriate, given that sometimes things go contrary to expectation. 

Sometimes the laws of nature make constructing contexts for valid arguments more reasonable. Now consider the following passage, which involves a scientific law:

\begin{quotation}\noindent Jack is about to let go of Jim's leash. The operation of gravity makes all unsupported objects near the Earth's surface fall toward the center of the Earth. Nothing stands in the way. Therefore, Jim's leash will fall. \end{quotation}

(Or, as Spock said in a Star Trek episode, "If I let go of a hammer on a planet that has a positive gravity, I need not see it fall to know that it has in fact fallen.") The inference above is represented in standard form as follows:

\begin{adjustwidth}{2em}{2em}
\begin{earg*}
\item  Jack is about to let go of Jim's leash. 
\item  The operation of gravity makes all unsupported objects near the Earth's surface fall toward the center of the Earth. 
\item  Nothing stands in the way of the leash falling. 
\itemc  Jim's leash will fall toward the center of the Earth.
\end{earg*}
\end{adjustwidth}

As stated, this argument is valid. That is, if you pretend that they are true or accept them "for the sake of argument", you would \textit{necessarily }also accept the conclusion. Or, to put it another way, there is no way in which you could hold the premises to be true and the conclusion false.

Although this argument is valid, it involves idealizing assumptions similar to the ones we saw in the pizza example. P$_2$ states a physical law which is about as well confirmed as any statement about the world around us you care to name. However, physical laws make assumptions about the situations they apply to---they typically neglect things like wind resistance. In this case, the idealizing assumption is just that nothing stands in the way of the leash falling. This can be checked just by looking, but this check can go wrong. Perhaps there is an invisible pillar underneath Jack's hand? Perhaps a huge gust of wind will come? These events are much less likely than Adriatico's making a lousy pizza, but they are still possible. 

Thus we see that using scientific laws to create a context where deductive validity is possible is a much safer bet than simply asserting whatever exceptionless rule pops into your head. However, it still involves improving the quality of the inference by introducing premises that are less likely to be true. 

So deduction is possible in artificial contexts like logical puzzles and legal codes. It is also possible in cases where we make idealizing assumptions or imagine exceptionless rules. The rest of the time we are dealing with induction. When we do induction, we try for strong inferences, where the premises, assuming they are true, would make the truth of the conclusion very likely, though not necessary. Consider the two arguments in Figure \ref{fig:strong_weak}


\begin{figure}
\begin{mdframed}[style=mytablebox]
\begin{tabu}{X[1,c]X[1,c]}
\begin{earg*}
\item  92\% of Republicans from Texas voted for Bush in 2000. 
\item  Jack is a Republican from Texas. 
\itemc  Jack voted for Bush. 
\end{earg*}
&
\begin{earg*}
\item  Just over half of drivers are female. 
\item  There's a person driving the car that just cut me off. 
\itemc  The person driving the car that just cut me off is female.
\end{earg*}
\\
A \textbf{strong} argument &
A \textbf{weak} argument
\end{tabu}
\end{mdframed}
\caption{Neither argument is valid, but one is strong and one is weak} \label{fig:strong_weak}
\end{figure}


Note that the premises in neither inference \textit{guarantee }the truth of the conclusion. For all the premises in the first one say, Jack could be one of the 8\% of Republicans from Texas who did not vote for Bush; perhaps, for example, Jack soured on Bush, but not on Republicans in general, when Bush served as governor. Likewise for the second; the driver could be one of the 49\%. 

So, neither inference is valid. But there is a big difference between how much support the premises, if true, would give to the conclusion in the first and how much they would in the second. The premises in the first, assuming they are true, would provide very strong reasons to accept the conclusion. This, however, is not the case with the second: if the premises in it were true then they would give only weak reasons for believing the conclusion. thus, the first is strong while the second is weak.

As we said earlier, there there are only two options with respect to validity---valid or not valid. On the other hand, strength comes in degrees, and sometimes arguments will have percentages that will enable you to exactly quantify their strength, as in the two examples in Figure \ref{fig:strong_weak}. 

However, even where the degree of support is made explicit by a percentage there is no firm way to say at what degree of support an inference can be classified as strong and below which it is weak. In other words, it is difficult to say whether or not a conclusion is \textit{very likely} to be true. For example, In the inference about whether Jack, a Texas Republican, voted for Bush. If 92\% of Texas Republicans voted for Bush, the conclusion, if the premises are granted, would very probably be true. But what if the number were 85\%? Or 75\%? Or 65\%? Would the conclusion very likely be true? Similarly, the second inference involves a percentage greater than 50\%, but this does not seem sufficient. At what point, however, would it be sufficient? 

In order to answer this question, go back to basics and ask yourself: "If I accept the truth of the premises, would I then have sufficient reason to believe the conclusion?". If you would not feel safe in adopting the conclusion as a belief as a result of the inference, then you think it is weak, that is, you do not think the premises give sufficient support to the conclusion. 

Note that the same inference might be weak in one context but strong in another, because the degree of support needed changes. For example, if you merely have a deposit to make, you might accept that the bank is open on Saturday based on your memory of having gone to the bank on Saturday at some time in the past. If, on the other hand, you have a vital mortgage payment to make, you might not consider your memory sufficient justification. Instead, you will want to call up the bank and increase your level of confidence in the belief that it will be open on Saturday.

Most inferences (if successful) are strong rather than valid. This is because they deal with situations which are in some way open-ended or where our knowledge is not precise. In the example of Jack voting for Bush, we know only that 92\% of Republicans voted for Bush, and so there is no definitive connection between being a Texas Republican and voting for Bush. Further, we have only statistical information to go on. This statistical information was based on polling or surveying a sample of Texas voters and so is itself subject to error (as is discussed in the chapter on induction in the complete version of this text.\label{ver_var} \nix{Chapter \ref{chap:induction} on induction).} A more precise version of the premise might be "92\% $\pm$ 3\% of Texas Republicans voted for Bush.".

At the risk of redundancy, let's consider a variety of examples of valid, strong and weak inferences, presented in standard form. 

\begin{earg*}
\item  David Duchovny weighs more than 200 pounds. 
\itemc  David Duchovny weighs more than 150 pounds.
\end{earg*}

The inference here is valid. It is valid because of the number system (here applied to weight): 200 is more than 150. It might be false, as a matter of fact, that David Duchovny weighs more than 200 pounds, and false, as a matter of fact, that David Duchovny weighs more than 150 pounds. But if you \textit{suppose }or \textit{grant }or \textit{imagine }that David Duchovny weighs more than 200 pounds, it would then \textit{have }to be true that David Duchovny weighs more than 150 pounds. Next:

\begin{earg*}
\item  Armistice Day is November 11th, each year. 
\item  Halloween is October 31st, each year.
\itemc  Armistice Day is later than Halloween, each year. 
\end{earg*}

This inference is valid. It is valid because of order of the months in the Gregorian calendar and the placement of the New Year in this system. Next:

\begin{earg*}
\item  All people are mortal. 
\item  Professor Pappas is a person. 
\itemc  Professor Pappas is mortal. 
\end{earg*}

As written, this inference is valid. If you accept for the sake of argument that all men are mortal (as the first premise says) and likewise that Professor Pappas is a man (as the second premise says), then you would have to accept also that Professor Pappas is mortal (as the conclusion says). You could not consistently both (i) affirm that all men are mortal and that Professor Pappas is a man and (ii) deny that Professor Pappas is mortal. If a person accepted these premises but denied the conclusion, that person would be making a mistake in logic.

This inference's validity is due to the fact that the first premise uses the word "all". You might, however, wonder whether or not this premise is true, given that we believe it to be true only on our experience of men \textit{in the past}. This might be a case of over-stating a premise, which we mentioned earlier. Next:

\begin{earg*}
\item  In 1933, it rained in Columbus, Ohio on 175 days.
\item  In 1934, it rained in Columbus, Ohio on 177 days.
\item  In 1935, it rained in Columbus, Ohio on 171 days.
\itemc  In 1936, it rained in Columbus, Ohio on at least 150 days.
\end{earg*}

This inference is strong. The premises establish a record of days of rainfall that is well above 150. It is possible, however, that 1936 was exceptionally dry, and this possibility means that the inference does not achieve validity. Next:

\begin{earg*}
\item  The Bible says that homosexuality is an abomination.
\itemc  Homosexuality is an abomination.
\end{earg*}

This inference is an appeal to a source. Appeals to sources are discussed in the sections on arguments from authority in the complete version of this text. \label{ver_var} \nix{Section \ref{sec:sources} of Part \ref{part:CT_and_informal_logic}} of this book. In brief, you should think about whether the source is reliable, is biased, and whether the claim is consistent with what other authorities on the subject say. You should apply all these criteria to this argument for yourself. You should ask what issues, if any, the Bible is reliable on. If you believe humans had any role in writing the Bible, you can ask about what biases and agendas they might have had. And you can think about what other sources---religious texts or moral experts---say on this issue. You can certainly find many who disagree. Given the controversial nature of this issue, we will not give our evaluation. We will only encourage you to think it through systematically.


\begin{earg*}
\item  Some professional philosophers published books in 2007.
\item  Some books published in 2007 sold more than 100,000 copies. 
\itemc  Some professional philosophers published books in 2007 that sold more than 100,000 copies. 
\end{earg*}

This reasoning is weak. Both premises use the word "some" which doesn't tell you a lot about many professional philosophers published books and how many books sold more than 100,000 copies in 2007. This means that you cannot be confident that even one professional philosopher sold more than 100,000 copies. Next:

\begin{earg*}
\item  Lots of Russians prefer vodka to bourbon. 
\itemc  George Bush was the President of the United States in 2006.
\end{earg*}

No one (in her right mind) would make an inference like this. It is presented here as an example only: it is clearly weak. It's hard to see how the premise justifies the conclusion to any extent at all.  
      
To sum up this section, we have seen that there are two styles of reasoning, deductive and inductive. The former tries to use valid arguments, while the latter contents itself to give arguments that are merely strong. The section of this book on formal logic will deal entirely with deductive reasoning. Historically, most of formal logic has been devoted to the study of deductive arguments, although many great systems have been developed for the formal treatment of inductive logic. On the other hand, the sections of this book on informal logic and critical thinking will focus mostly on inductive logic, because these arguments are more readily available in the real world. 



\practiceproblems

\noindent\problempart For each inference, (i) say whether it is valid, strong, or weak and (ii) explain your answer.

\begin{longtabu}{p{.1\linewidth}p{.8\linewidth}}
\textbf{Example}: & The patient has a red rash covering the extremities and head, but not the torso. The only cause of such a rash is a deficiency in vitamin K. So, the patient must have a vitamin K deficiency. \\
\textbf{Answer}: & \noindent (i) Valid. \newline
\noindent (ii) The word "only" means it must be vitamin K deficiency.
\\
\end{longtabu}

\begin{exercises}

\item On 2003-06-19 in Norfolk, VA, a violent storm blew through and the power went out over much of the city. So, the storm caused the power to go out.

\answerblank{\begin{enumerate}[label=(\roman*)]
\item Strong
\item Storms often cause power outages, but other things can cause them, too. In general, causal inferences are part of inductive reasoning, and are therefore at best strong. 
\end{enumerate}
}{\vspace{1.5in}}

\item  All human beings are things with purple hair, and all things with purple hair have nine legs. Therefore, all human beings have nine legs.

\answerblank{\begin{enumerate}[label=(\roman*)]
\item Valid.
\item You can see this by substituting in different things for ``purple hair'' and ``has nine legs.'' For instance, you could use ``mammals'' and ``has hair.'' There is no way to do this that will make the premises true and the conclusion false. So the argument is valid. 
\end{enumerate}
}{\vspace{1.5in}}


\item  Elvis Presley was known as The King. Elvis had 18 songs reach \#1 in the Billboard charts. So, The King had 18 \#1 hits.

\answerblank{\begin{enumerate}[label=(\roman*)]
\item Valid
\item This is an example of substituting different names for the same thing to create a valid argument. 
\end{enumerate}}{\vspace{1.5in}}

\item Most philosophers are right-handed. Terence Irwin is a philosopher. So, he is right-handed.

\answerblank{\begin{enumerate}[label=(\roman*)]
\item Either strong or weak depending on how much confidence you need. Definitely not valid.
\item The conclusion isn't necessarily true, so the inference is not valid. Is it strong or weak? If you read "most" as "very many" or something like that, it would be strong; if you read "most" as "a majority" (in the sense of 'somewhere between 51\% and 99\%' ), it would probably be weak. Advertisers sometimes use the vagueness of "most" to get you to feel that lots of people are buying a certain product or service, when in fact only a small majority is.
\end{enumerate}}{\vspace{1.5in}}

\item  Jack has purple hair, and purple toe nails. Hence, he has toe nails.

\answerblank{\begin{enumerate}[label=(\roman*)]
\item Valid
\item If the color exists, the colored object has to exist
\end{enumerate}
}{\vspace{1.5in}}

\item  The Ohio State football team beat the Miami football team on 2003-01-03 for the college national championship. So, the Ohio State football team was the best team in college football in the 2002-2003 season.

\answerblank{\begin{enumerate}[label=(\roman*)]
\item Strong or weak, depending on your background beliefs.
\item What are the chances that the non-best team would win the championship? If you think that a non-best time wins somewhat frequently, this inference would be weak. If, on the other hand, you think winning the championship is  a good way to judge the best team, you think the inference is strong.
\end{enumerate}
}{\vspace{1.5in}}

\item Willie Mosconi made almost all of the pool shots he took from 1940-1945. He took a bunch of shots in 1941. So, he made almost every shot he took in 1941.

\answerblank{\begin{enumerate}[label=(\roman*)]
\item Strong
\item ``Almost all'' is fairly vague. So we are not sure how many missed shots we are talking about in the five year period. If there were all clustered in 1941, it is possible that the success rate for 1941 would no longer qualify as ``almost all,'' but this is unlikely. 
\end{enumerate}
}{\vspace{1.5in}}

\item Some philosophers are people who are right-handed. Therefore, some people who are right-handed are philosophers.

\answerblank{\begin{enumerate}[label=(\roman*)]
\item Valid.
\item The conclusion can't be false if the premise is true.
\end{enumerate}
}{\vspace{1.5in}}

\item U.S. President Obama firmly believed that Iran is planning a nuclear attack against Israel. We can conclude that Iran is planning a nuclear attack on Israel.

\answerblank{\begin{enumerate}[label=(\roman*)]
\item Weak
\item Even if you think Obama's judgment is generally reliable here, a nuclear attack on Israel is an incredibly unlikely event. After all, this land is holy to Muslims, too. Extraordinary claims require extraordinary evidence, and I don't think any one's person's judgment is enough to go on here.
\end{enumerate} 
}{\vspace{1.5in}}

\item Since the Spanish American War occurred before the American Civil War, and since the American Civil War occurred after the Korean War, it follows that the Spanish American War occurred before the Korean War.

\answerblank{\begin{enumerate}[label=(\roman*)]
\item Weak.
\item If A is before B and C is before B, we know nothing about the relationship between B and C. A and C could be at the same time or either one before the other.
\end{enumerate}
}{\vspace{1.5in}}

\item There are exactly 10 humans in Carnegie Hall right now. Every human in Carnegie Hall right now has exactly ten legs. And, of course, no human in Carnegie Hall shares any legs with another human. Thus, there are at least 100 legs in Carnegie Hall right now.

\answerblank{\begin{enumerate}[label=(\roman*)]
\item Valid
\item The conclusion follows from the fact that $10 \times 10 = 100$
\end{enumerate} 
}{\vspace{1.5in}}

%\item Amy Bishop is an evolutionary biologist (who shot a number of her colleagues to death in 2010). Evolutionary biology is incompatible with [Christian] scriptural teaching. Scriptural teaching is the only grounding for morality. Thus, evolutionary biologists are immoral.
%
%\answerblank{\begin{enumerate}[label=(\roman*)]
%\item Weak
%\item Many beliefs are incompatible with scriptural teaching on some point or other, but it's not clear that the people who hold those beliefs are immoral, unless morality is defined as following every single scriptural edict.
%\end{enumerate}
%}{\vspace{1.5in}}

%\item Corrupt people do harm to those around them, and no one intentionally wants to be done harm. Therefore, I [Socrates] did not corrupt my associates intentionally.
%
%\answer{\underline{Valid} \\ The argument works if you think the premises are true always and everywhere. They seem like natural enough statements to make, but are they really perfectly and unequivocally true?}

\item Taxation means paying some of your earned income to the government. Some of this income is distributed to others. Paying so that someone else can benefit is slavery. Therefore, taxation is slavery.

\answerblank{\begin{enumerate}[label=(\roman*)]
\item Valid.
\item Chain argument. However, the definition of slavery here is contentious, to say the least.
\end{enumerate}}{\vspace{1.5in}}

%\item Attempts have been made recently to carry bombs or bomb-making materials onto planes in the underwear and in other personal areas. These types of procedure provide a large measure of security against such attempts. Thus, flyers are required to submit to either a full-body scan or a thorough pat-down.
%
%%{\color{red}This problem shouldn't have been here, because it is really an explanation and not an argument.}
\end{exercises}

\noindent\problempart For each inference, (i) say whether it is valid, strong, or weak and (ii) explain your answer.
\answer{Answers by Ben Sheredos}
\begin{exercises} 
\item The sun has come up in the east every day in the past. So, the sun will come up in the east tomorrow.

\answer{Invalid, but strong. That's a huge number of cases to generalize from, so the conclusion is very likely to be true, even if it is not \textit{certain}}

\item Jack's dog Jim will die before the age of 73 (in human years). After all, you are familiar with lots of dogs, and lots of different kinds of dogs, and any dog that is now dead died before the age of 73 (in human years).

\answer{Invalid, but strong. That's a huge number of cases to generalize from, so the conclusion is very likely to be true. Don't get confused because you yourself are \textit{absolutely certain} that no dog will live to 73 in human years. The question is how well \textit{this argument} supports that claim.}

\item Any time the public receives a tax rebate, consumer spending increases, and the economy is stimulated. Since the public just received a tax rebate, consumer spending will increase.

\answer{Valid. One could continue on to infer that the economy will be stimulated. The key is that the premise is that \textit{every time} there is a tax rebate, spending increases. This might be false, but \textit{if} it is true, the conclusion follows.}

\item  90\% of the marbles in the box are blue. So, about 90\% of the 20 I pick at random will be blue.

\answer{Invalid, and pretty weak. This is a common error in statistical reasoning. The 20 marbles you pick out are not connected in any way, so if you pick out a non-blue marble, that doesn't increase the odds of you picking out a blue one next time. You might happen to pick nothing but marbles from the 10\% of marbles that are not blue.}

\item  According to the world-renowned physicist Stephen Hawking, quarks are one of the fundamental particles of matter. So, quarks are one of the fundamental particles of matter.

\answer{This is a simple appeal to authority, which is a fallacy. The argument is weak. It might be supplemented to be made stronger (''Hawking is \textit{the world's foremost authority} on this topic, and he has put forth a convincing argument that quarks are a fundamental particle''). But as it is stated here, it's garbage. }

\item Sean Penn, Susan Sarandon and Tim Robbins are actors, and Democrats. So, most actors are Democrats.

\answer{Invalid and weak. This is a very hasty generalization.}


\item  The President's approval rating has now fallen to 53\%, employment is at a 10 year high, and he is in charge of two foreign wars. He would not win another term in two years' time, if he were to run.

\answer{Weak. There are some suppressed premises here, concerning how voters are likely to respond to the claims presumed in the premises. Only by filling them in could the argument be made strong.}


\item If Bill Gates owns a lot of gold then Bill Gates is rich, and Bill Gates doesn't own a lot of gold. So, Bill Gates isn't rich.

\answer{Weak, since (\textit{a}) owning gold is not necessary for being rich, and (\textit{b}) Bill Gates is demonstrably rich even though (suppose) he owns little gold.}

\item All birds have wings, and all vertebrates have wings. So, all birds are vertebrates.

\answer{ Weaksauce, even if the premises are true. Compare: ''All students in PHIL 10 are enrolled at UCSD, and all students in PHIL 163 are enrolled at UCSD. So all students in Phil 10 are in PHIL 163.'' Clearly wrongheaded.}

\item U.S. President Obama gave a speech in Berlin shortly after his inauguration. Berlin, of course, is where Hitler gave many speeches. Thus, Obama intends to establish a socialist system in the U.S.

\answer{ Obviously Weak. Doing something as general as ``giving a speech'' in a place where Hitler gave a speech does not make one relevantly like Hitler to draw this conclusion.}

\item Einstein said that he believed in a god only in the sense of a pantheistic god equivalent with nature. Thus, there is no god in the Judeo-Christian sense.

\answer{Weak. Appeal to authority. The argument concludes that something is true just because one person believed it; why trust Einstein on this? No support is provided. What, are we just supposed to be impressed because it was Einstein?}

\item The United States Congress has more members than there are days in the year. Thus, at least two members of the United States Congress celebrate their birthdays on the same day of the year.

\answer{Valid. There are 365 days in a year. In any group of 366 people, at least 2 people have to share birthdays. (And don't try weaseling in that leap-year nonsense.)}

\item The base at Guantanamo ought to be closed. The continued incarceration of prisoners without any move to try or release them provides terrorist organizations with an effective recruiting tool, perhaps leading to attacks against Americans overseas.

\answer{Pretty strong? The first sentence is the conclusion, and reasons are provided for thinking it is true. Maybe there are countervailing reasons that tell against...? Informal reasoning is tricky.}

\item Smith and Jones surveyed teenagers (13-19 years old) at a local mall and found that 94\% of this group owned a mobile phone. Therefore, they concluded, about 94\% of all teenagers own mobile phone.

\answer{Weak. Why suppose an un-specified number of teenagers in one place are representative of that entire group of people, worldwide?}

\item Janice Brooks is an unfit mother. Her Facebook and Twitter records show that in the hour prior to the youngest son's accident she had sent 50 messages --- any parent who spends this much time on social media when they have kids is not giving them proper attention. 

\answer{Weak. Does Janice Brooks even live with her youngest son? Was there any reason she should've known his accident was impending? What, are people with children never allowed to have a day chatting with friends?}

\end{exercises}


% *********************************************************
% *	Fallacies, Cognitive Biases and Dysfunctional Dialogues		  *	
% *********************************************************


\section{Fallacies, Cognitive Biases and Dysfunctional Dialogues}

Our project here is to learn to identify good and bad forms of reasoning in the real world. To do this well, we need to give names to these forms. Names are power. If we have a name for something, we are more likely to notice it when we encounter it. \nix{insert example here} Names also shape our associations and assumptions about things. A series of studies by the Erika Hall, Katherine Phillips and Sarah Townsend showed that Americans have significantly worse associations with the term ``black'' than they do the term ``African American.'' \citep{ErikaV.Hall2015} Hall and colleagues wrote up pairs of things like job applications and crime reports that differed only in that in one case the person was described as ``Black'' and in the other as ``African-American,'' and found that people were much more likely to have negative feelings about the person described as ``Black'' than the person described as ``African-American.'' All this shows that we need to choose our names carefully, whether we are naming kinds of people, or forms of reasoning. 

In this text we will be drawing our terms from two main traditions, a modern scientific tradition based largely in cognitive psychology and an older philosophical tradition with roots all over the world. We got a hint of these traditions on page \ref{def:cognitive_bias} when we looked at cognitive biases and fallacies. ``Fallacy'' is the term most associated with the philosophical tradition. Fallacies are generally thought of as mistaken forms of inference from one set of beliefs to another belief and can they generally be explained by representing arguments in canonical form. \label{fallacy_detail} For instance in Chapter \ref{chap:emotionalreasons} and again in Chapter \ref{chap:sources} we will look at the \emph{ad populum} fallacy, which happens when people infer that a belief is true because it is popular. Here is an argument in canonical form that at least seems to commit the ad populum fallacy.

\begin{earg*}
\item Everyone believes the Sun goes around the Earth.
\itemc The Sun goes around the Earth.
\end{earg*}

The study of fallacies like this crops up early in philosophical traditions around the world. In the classical Indian tradition, in the first few centuries \textsc{BCE} a work known as the \textit{Ny\={a}ya S\={u}tra} identified five common fallacies in reasoning (\cite{Gautama1982}). The ancient Greek tradition began identifying fallacies in the fourth century \textsc{BCE} when the Greek philosopher Aristotle wrote a book called the \cite*{Aristotle:refutations} outlining 13 fallacies he thought were commonplace. Aristotle had a lot of followers in the ancient world, first in pagan Greece and Rome and then later in the Islamic, Christian, and Jewish traditions. These thinkers continued the practice of naming and listing fallacies. 

\newglossaryentry{argumentation scheme}
{
name=argumentation scheme,
description={A standardized format for representing a common form of reasoning from either an everyday or a technical context, consisting of the argument in canonical form together with warrants particular to that type of reasoning.}
}

One issue that quickly comes up in the study of fallacies is that anything that gets proposed as a fallacy can actually be a good argument some of the time. Some of the time, it actually is a good idea to believe something because other people say it is true. Indeed, as we will see on page \pageref{def:ad_populum}, the argument above might have been pretty reasonable for someone living in the middle ages. This is why more modern scholars will talk about ``argumentation schemes'' rather than fallacies. An \textsc{\gls{argumentation scheme}}\label{def:argumentation_scheme} is a standardized format for representing a common form of reasoning from either an everyday or a technical context, consisting of the argument in canonical form together with special premises called ``warrants' particular to that type of reasoning (\cite{Walton2008b}). Whether an argument is a fallacy or not will depend on whether the warrants can be provided. In the case of the ad populum fallacy, the relevant argument scheme will be the argument from authority, which we will cover in Chapter \ref{chap:sources}. The bottom line is that the fallacy is essentially the defective form of the argumentation scheme. 

Thinking about bad reasoning in terms of fallacies and argumentation schemes makes it look like mistakes in reasoning are conscious moves from one explicitly held belief to another. This has the advantage of laying all the major elements of the reasoning out in the open so we can evaluate the strength of the reasoning involved. The disadvantage is that it doesn't really capture what is going on in people's heads when they make a mistake in reasoning. Typically, people don't consciously think to themselves, ``everyone believes the Sun goes around the Earth'' and then think ``the sun must really go around the Earth.'' People just go along with the crowd in a much more unconscious fashion. The modern psychological approach does a better job of capturing this fact.

More recently psychologists have begun studying mistakes in reasoning from a more scientific perspective. Typically, rather than talking about fallacies, though, psychologists will talk about cognitive bias. As we saw in section \ref{def:cognitive_bias} \label{cognitive_bias_detail} a cognitive bias is a habit of reasoning that can become dysfunctional in certain circumstances. These biases are often not a matter of explicit belief, and are thus are sometimes called ``implicit biases.'' The study by Hall and colleagues mentioned above is an example of research into implicit biases. More work on implicit bias against different groups of people has been conducted using  the Implicit Associations Test (AIT), which measures the speed in which subjects to sort words or pictures into categories. (See Banaji and Greenwald \cite*{Banaji2013}) Because the test measures the speed of your response rather than just the association you give, it is able to measure attitudes you have that you may not be aware of yourself. The test consistently reveals that people have more prejudices than they are aware of, and often have biases against groups they are members of---so some women are found to have biases against women. 

This is a kind of failure of reasoning that is not adequately understood in therms of fallacies. These biases are not a product of moving from some explicitly held belief to another. They are unconscious habits that distort the way we form beliefs. What is more interesting is that these habits are often products of features of our brains that are useful much of the time. We need to be able to make generalizations without much evidence. Our ancestors needed to judge quickly whether an animal was dangerous, for instance, and having an overactive fear response was a much safe bet than having too little of a fear response. So we develop a propensity to make fearful associations easily. But in a modern context this same habit of mind, rather than keeping us safe, perpetuates injustice. 

There is a lot of overlap between the modern study of cognitive bias and the traditional study of the fallacies. Often the same kind of mistake can be looked at from either perspective. The ad populum fallacy is a case in point. You can consider it as an explicit fallacious argument with premises and conclusions. But we noted that you can also think about it in terms of unconsciously  going with the flow. One cognitive bias that works similarly to the ad populum fallacy is called ``affinity bias.'' Affinity bias is our bias in favor of people we think are similar to us. We are more likely to do things like believe what they say and copy their behavior. We are also more likely to want to be around them, and this can lead to unfairness in things like housing and employment. Affinity bias is also what enables affinity fraud, when a con artist uses a shared background with their victim to earn their trust. The ad populum fallacy and affinity bias are both ways that the human tendency to form tribes distorts rational behavior, but when we talk about the ad populum fallacy, we are just talking about it's impact on argument. Affinity bias is a more pervasive, unconscious phenomenon. 


\newglossaryentry{dysfunctional dialogue}
{
name=dysfunctional dialogue,
description={a failure of reasoning that is the collective responsibility of two or more people in conversation.}
}

In this text, we will be looking at mistakes in reasoning as both fallacies and cognitive biases. We will also be considering a separate kind of mistake we will call a dysfunctional dialogue. A \textsc{\gls{dysfunctional dialogue}} \label{def:dysfunctional_dialogue} is a failure of reasoning that is the collective responsibility of two or more people in conversation. If two people are talking past each other, because, for instance, they are working from different definitions, or because they disagree about standards or proof or unacknowledged premises, There is a failure of reasoning there that is both of their responsibilities. We will see one example of this in Chapter \ref{chap:substitutes} when we look at the notion of burden of proof. The burden of proof is the obligation of one side of a debate to prove its case to a certain standard, and if they don't, things will revert to some default situation. Court cases provide common examples of burden of proof. So, for instance, in US law, the burden is on the prosecution of a murder trial to prove their case beyond a reasonable doubt. If they don't, the defendant goes free. In legal contexts the burden of proof is well defined, but in ordinary conversation, people can disagree about where the burden of proof lies. As we shall see, there is a kind of dysfunctional dialogue commonly known as ``burden tennis'' where each side takes turns insisting that the burden of proof is on the other side, and that they have failed to meet it. Conversations like this are generally frustrating and unproductive and represent a failure of both parties to cooperate in rational thought.  





\section*{Key Terms}
\begin{sortedlist}
\sortitem{Valid}{} 	
\sortitem{Invalid}{} 	
\sortitem{Sound}{} 
\sortitem{Strong}{} 
\sortitem{Cogent}{} 
\sortitem{Deductive}{} 
\sortitem{Inductive}{} 
\sortitem{Fallacy}{}
\sortitem{Weak}{}
\sortitem{Confirmation bias}{} 	
\sortitem{Cognitive Bias}{} 	
\end{sortedlist}



\include{tex/ch19-maps}
%\include{tex/ch15-emotions}
%\include{tex/ch18-sources}
%\include{tex/ch17-analogy}

\label{part:CT}	
%
%\part{Critical Thinking} 
%\include{tex/ch12-whatiscriticalthinking}
%\include{tex/ch13-substitutes}
%\include{tex/ch14-incompletearguments}
%
%\include{tex/ch16-generalizations}
%
%
%\include{tex/ch20-practicalarguments}

%\part{Inductive and Scientific Reasoning}  \label{part:inductive_scientific}
%\include{tex/ch21-whatareinductionandscientificreasoning}
%\include{tex/ch22-inductioninscience}
%\include{tex/ch23-mills-methods}
%\include{tex/ch24-causation-explanation}
%\include{tex/ch25-Analogy-in-Science}
%\include{tex/ch26-experimental-methods}
%\include{tex/ch27-Association-Diagrams-Cross-Tabulations}
%\include{tex/ch28-Explanation-Building}
%\include{tex/ch29-Problems-In-Induction}

\appendix
\iflabelexists{part:formal_logic}{\include{tex/z-app-notation}}{}
%\include{app-solutions}

%Bibstuff
%If the {part:CT} label is found, LaTeX will typeset separate bibliographies for sample passages and logical sources.

\iflabelexists{part:CT}{%text for CT version}

\defbibnote{sample}{\textit{ \large  This bibliography includes all sources except for those that were used as examples for logical analysis, either in the main text or problem sets}}

\printbibliography [keyword=samplepassage, title=Bibliography of Sample Passages, prenote=sample, heading=bibnumbered] %for separate bibs sample passages and general citations

\printbibliography [notkeyword=samplepassage, title=General Bibliography, heading=bibnumbered] %for separate bibs sample passages and general citations

}% End CT version
{\printbibliography[heading=bibnumbered]} %single bib for non-CT version


%%The way I’ve set this up now is that there is one bib for sample passages and one bib for everything else. This means that it would not be possible to put one entry in both bibliographies. (This might be needed for Aristotle.) To do that, you will need to define a separate logicsource category


\setglossarysection{chapter}
\printglossaries

\iflabelexists{part:formal_logic}{\include{tex/zz-quickreference}}{}
\include{tex/zzz-backmatter}	







\end{document}
