\documentclass{article}
%\documentclass[twoside, openright]{book}

%%%%%%%%%%%%%%%%%%%%%%%%%%%%%%%%%%%%%%%%%%%%%%%%%%%%%%%%%%%%%
%                                                                                                                                                                                                 %
%		This is a standalone reading covering the basic ideas in argument for classes that aren't logic or CT                                %
%                                                                                                                                                                                                 %
%		It is based on the January 2019 version of For All X LC Remix                                                                                        % 
%                                                                                                                                                                                                 %
%%%%%%%%%%%%%%%%%%%%%%%%%%%%%%%%%%%%%%%%%%%%%%%%%%%%%%%%%%%%%

%glossary and indexing stuff
\usepackage[toc, nopostdot, numberedsection]{glossaries}
\usepackage{datatool}
%\newglossary*{catstatements}{Chapter \ref{chap:catstatements} Key Terms}
%\newglossary*{catsyllogisms}{Key Terms}
\renewcommand{\glsnamefont}[1]{\makefirstuc{#1}}
\makeglossaries

%General packages
\usepackage{answers}
\usepackage{textcomp}
\usepackage{anyfontsize}	
\usepackage{geometry}
\usepackage{url}
\usepackage{changepage}
\usepackage{syntonly}
\usepackage{enumitem}
\usepackage{turnstile}
\usepackage[normalem]{ulem}
\usepackage{fixltx2e}	
\usepackage{wasysym}
\usepackage{tocloft}
\usepackage{fancyhdr}
\usepackage{fancyref}
\usepackage{etoolbox}
\usepackage[utf8]{inputenc}
%\usepackage{amsthm} for some reason, this conflicts with the fitch.sty part of openlogic.sty. 

%%% Bibstuff
\usepackage[authordate,autocite=inline,backend=biber, natbib]{biblatex-chicago}
\bibliography{tex/z-openlogic}
% To typeset the bibliography, you need to run "biber --output-safechars chap1excerpts" from the command line. You can't use the function within TeXworks, because it doen't have the --output-safechars flag. Without that flag, biber is unable to handle many characters used for Sanskrit words, like the n with a dot under it.




%%%  graphics packages %%%
\usepackage{tikz}
\usetikzlibrary{shapes,backgrounds,matrix,arrows,decorations,positioning}
\usepgflibrary{arrows.new}
\usepackage{graphicx}
\usepackage{xcolor}


%%%    Table and figure packages %%%
\usepackage{float}
\usepackage[singlelinecheck=false, skip=0pt]{caption} %left aligns captions for tables, moves them closer to table. 
\usepackage{tabularx}
\usepackage{longtable}
\usepackage{tabu}
\usepackage[framemethod=1]{mdframed} 
\usepackage{wrapfig} %I used this to put a frame around tables.
\tabulinesep=.75ex
\usepackage{colortbl}
\floatstyle{plain} 
\restylefloat{figure}
\usepackage[export]{adjustbox}
\usepackage{multirow}
\usepackage{rotating}
\usepackage{booktabs}
	

%linking and bookmarks in the pdf.
\usepackage{hyperref}
\hypersetup{pdftex,colorlinks=true,allcolors=blue}
\usepackage{hypcap}	
	

\usepackage{openintroduction-exp}
	
\pdfinfo{
  /Title (An Open Introduction to Logic)
  /Author (J. Robert Loftis, Cathal Woods, and P.D. Magnus)
  /Subject (An open access introductory textbook in logic and critical thinking)
  )
}

\begin{document}

%\label{showanswers} %uncomment this tag and typeset twice to show answers
%\label{blank_prob_set} %uncomment this tag and typeset twice to create a blank problem set sheet. Don't use with \label{showanswers} uncommented

\raggedright
\setlength{\parindent}{1em}
\setlength{\parskip}{1em}	



{
\setlength{\parskip}{0em}
\cftpagenumbersoff{part}
%\cftpagenumbersoff{chapter}

\renewcommand{\cftpartpresnum}{\sf\Large\partname\ }
}





\setlength{\parindent}{1em}
\pagestyle{headings} % puts the running heads back.
\label{full_version} %Include this label to make cross references work right when typesetting full text

\title{A Quick Introduction to Argument}
\date{}
\maketitle

% *****************************
% *		Introduction                      *
% ****************************

\section{Introduction}

This is a philosophy course and that means that there is going to be a substantial emphasis placed on understanding why we believe what we believe and on being able to justify those beliefs to others. Our goal is not just to explore big philosophical ideas, but to develop theories about those ideas that we can argue for rationally. In order to do this well, we need to have a solid understanding of what an argument is and how it works. In philosophy, we use the word ``argument'' to refer to the attempt to show that certain evidence supports a conclusion. This is very different from the sort of argument you might have when you are mad at someone, which could involve screaming and throwing things. We are going to use the word ``argument'' a lot in this course, so you need to get used to thinking of it as a name for a rational process, and not a word that describes what happens when people disagree.

Argument, as we are using the term, is a rational process you engage in every day. Consider, for instance, the game of Clue. (For those of you who have never played, Clue is a murder mystery game where players have to decide who committed the murder, what weapon they used, and where they were.) A player in the game might decide that the murder weapon was the candlestick by ruling out the other weapons in the game: the knife, the revolver, the rope, the lead pipe, and the wrench. This evidence lets the player know something they did not know previously, namely, the identity of the murderer.

A logical argument is structured to give someone a reason to believe some conclusion. Here is the argument about a game of Clue written out in a way that shows its structure. 


\label{argClue}
\begin{earg}
\item[P$_1$:] In a game of Clue, the possible murder weapons are the knife, the candlestick, the revolver, the rope, the lead pipe, and the wrench.
\item[P$_2$:] The murder weapon was not the knife.
\item[P$_3$:] The murder weapon was also not the revolver, the rope, the lead pipe, or the wrench.
\vspace{-.5em}
\item [] \rule{0.9\linewidth}{.5pt} 
\item[C:] Therefore, the murder weapon was the candlestick.
\end{earg} 

In the argument above, statements P$_1$--P$_3$ are the evidence. We call these the \emph{premises}. The word ``therefore'' indicates that the final statement, marked with a C, is the \emph{conclusion} of the argument. If you believe the premises, then the argument provides you with a reason to believe the conclusion. You might use reasoning like this purely in your own head, without talking with anyone else. You might wonder what the murder weapon is, and then mentally rule out each item, leaving only the candlestick. On the other hand, you might use reasoning like this while talking to someone else, to convince them that the murder weapon is the candlestick. (Perhaps you are playing as a team.) Either way the structure of the reasoning is the same. 



\newglossaryentry{metacognition}
{
name=metacognition,
description={Thought processes that are applied to other thought processes.}
}

In order to study reasoning, we have to apply our ability to reason to our reason itself. This kind of thinking about thinking is called \textsc{\gls{metacognition}}\label{def:Metacognition}. In this reading will provide you with tools to improve your metacognition by giving you terminology to understand rational thinking.


% ******************************************
% *		Statement, Argument, Premise, Conclusion  *
% ******************************************

\section{Statement, Argument, Premise, Conclusion}
\label{sec:SAPC}

\newglossaryentry{statement}
{
name=statement,
description={A unit of language that can be true or false.}
}

We have said that an argument is not simply two people disagreeing; it is an attempt to prove something using evidence. To go any further, we are going to need a more precise definition of argument. An argument is composed of statements. In philosophy, we define a \textsc{\gls{statement}} \label{def:statement} as a unit of language that can be true or false. To put it another way, it is some combination of words or symbols that have been put together in a way that lets someone agree or disagree with it. All of the items below are statements.

\begin{enumerate}[label=(\alph*)]
\item \label{itm:t.rex_true}\emph{Tyrannosaurus rex} went extinct 65 million years ago. 
\item \label{itm:t.rex_false}\emph{Tyrannosaurus rex} went extinct last week.
\item \label{itm:t.rex_unknown}On this exact spot, 100  million years ago, a \emph{T. rex} laid a clutch of eggs. 
\item \label{itm:silly}George W. Bush is the king of Jupiter. 
\item \label{itm:moral}Murder is wrong. 
\item \label{itm:opinion1}Abortion is murder. 
\item \label{itm:opinion2}Abortion is a woman's right. 
\item \label{itm:opinion3}Lady Gaga is pretty.
\item \label{itm:definition}Murder is the unjustified killing of a person.
\item \label{itm:nonsense}The slithy toves did gyre and gimble in the wabe.
\item \label{itm:history}The murder of logician Richard Montague was never solved. 
\end{enumerate}

Because a statement is something that can be true \emph{or} false, statements include truths like \ref{itm:t.rex_true} and falsehoods like \ref{itm:t.rex_false}. A statement can also be something that that must either be true or false, but we don't know which, like \ref{itm:t.rex_unknown}. A statement can be something that is completely silly, like \ref{itm:silly}. Statements in philosophy include statements about morality, like \ref{itm:moral}, and things that in other contexts might be called ``opinions,'' like \ref{itm:opinion1} and \ref{itm:opinion2}. People disagree strongly about whether \ref{itm:opinion1} or \ref{itm:opinion2} are true, but it is definitely possible for one of them to be true. The same is true about \ref{itm:opinion3}, although it is a less important issue than \ref{itm:opinion1} and \ref{itm:opinion2}. A statement in philosophy can also simply give a definition, like \ref{itm:definition}.  Statements can include nonsense words like \ref{itm:nonsense}, because we don't really need to know what the statement is about to see that it is the sort of thing that can be true or false. The study of argument is content neutral, meaning that we set what is being argumed about and just look at the way it is argued. The statements we study can be about dinosaurs, abortion, Lady Gaga, and even the history of philosophy, as in statement \ref{itm:history}, which is true.

\newglossaryentry{descriptive statement}
{
name=descriptive statement,
description={A statement which talks about the way the world is. These are contrasted with normative statements, which talk about the way the world should be.}
}


\newglossaryentry{normative statement}
{
name=normative statement,
description={A statement which talks about the way the world should be. These are contrasted with descriptive statements, which talk about the way the world is.}
}



So you see that statement is a broad category that can include all kinds of things that you might not normally lump together. One important division within the class of statements is the distinction between statements which talk about the way the world \textit{is} and statements which talk about the way the world should be. Statements that are about the way the world is are called \textsc{\glspl{descriptive statement}}\label{def:descriptive_statement}. These include true descriptions of the world, like statement \ref{itm:t.rex_true} and false descriptions like \label{itm:silly}. Statements that are about the way the world should be are called \textsc{\glspl{normative statement}}\label{def:normative_statement}. These include statements with an explicit ``should'' in them, like ``you shouldn't chew with your mouth open.'' They also include statements that contain an implicit ``should'' like statements  \ref{itm:opinion1} and \ref{itm:opinion2}. If you assert that abortion is murder, you are saying that people \textit{shouldn't} have abortions and that abortions \textit{should} be illegal. Conversely, you say that abortion is a woman's right, you are saying it \textit{should} be legal. All of this is different than describing what people actually do or what is actually legal. 

Some kinds of statements can be interpreted normatively or discriptively. Definitions like statement \ref{itm:definition} are like that. It could describe the way people actually use a word or announce that people should use a certain way. Sometimes the same definition can be used different ways. Dictionaries, for instance, are written to be descriptive: they simply summarize how a word has been used up to this point. However, when dictionaries are actually used, they are generally used normatively. We use them to correct people's use of words, to say that they \textit{should} use words differently.

We are treating statements primarily as units of language or strings of symbols, and most of the time the statements you will be working with will just be words printed on a page. However, it is important to remember that statements are also what philosophers call ``speech acts.'' They are actions people take when they speak (or write). If someone makes a statement they are typically telling other people that they believe the statement to be true, and will back it up with evidence if asked to. When people make statements, they always do it in a context---they make statements at a place and a time with an audience. Often the context statements are made in will be important for us, so when we give examples, statements, or arguments we will sometimes include a description of the context. When we do that, we will give the context in \textit{italics.} See Figure \ref{fig:statements_and_context} for examples. \label{context_marker} 

\begin{figure}
\begin{mdframed}[style=mytableclearbox]
\includegraphics*[width=\linewidth]{img/statement_and_contexts}
\end{mdframed}
\caption{A statement in different contexts, or no context.} \label{fig:statements_and_context}
\end{figure}


``Statements' in this text does \emph{not} include questions, commands, exclamations, or sentence fragments. Someone who asks a \emph{question} like ``Does the grass need to be mowed?'' is typically not claiming that anything is true or false. Generally, \emph{questions} will not count as statements, but \emph{answers} will. ``What is this course about?'' is not a statement. ``No one knows what this course is about,'' is a statement.

For the same reason \emph{commands} do not count as statements for us. If someone bellows ``Mow the grass, now!'' they are not saying whether the grass has been mowed or not. You might infer that they believe the lawn has not been mowed, but then again maybe they think the lawn is fine and just want to see you exercise. 

An exclamation like ``Ouch!'' is also neither true nor false. On its own, it is not a statement. We will treat ``Ouch, I hurt my toe!'' as meaning the same thing as ``I hurt my toe.'' The ``ouch'' does not add anything that could be true or false.

Finally, a lot of possible strings of words will fail to qualify as statements simply because they don't form a complete sentence. In your composition classes, these were probably referred to as sentence fragments. This includes strings of words that are parts of sentences, such as noun phrases like ``The tall man with the hat'' and verb phrases, like ``ran down the hall.'' Phrases like these are missing something they need to make a claim about the world. The class of sentence fragments also includes completely random combinations of words, like ``The up if blender route,'' which don't even have the form of a statement about the world.  

Sometimes philosophers describe the components of argument as ``propositions,'' or ``assertions,'' and we will use these terms periodically as well.  There is actually a great deal of disagreement about what the differences between all of these things are and which term is best used to describe parts of arguments. However, none of that makes a difference for this class. We could have used any of the other terms in this text, and it wouldn't change anything. 

Sometimes the outward form of a speech act does not match how it is actually being used. A rhetorical question, for instance, has the outward form of a question, but is really a statement or a command. If someone says ``don't you think the lawn needs to be mowed?'' they may actually mean a statement like ``the lawn needs to be mowed'' or a command like ``mow the lawn, now.'' Similarly one might disguise a command as a statement. ``You will respect my authority'' \emph{is} either true or false---either you will or you will not. But the speaker may intend this as an order---''Respect me!''---rather than a prediction of how you will behave.

When we study argument, we need to express things as statements, because arguments are composed of statements. Thus if we encounter a rhetorical question while examining an argument, we need to convert it into a statement. ``Don't you think the lawn needs to be mowed'' will become ``the lawn needs to be mowed.'' Similarly, commands will become should statements. ``Mow the lawn, now!'' will need to be transformed into ``You should mow the lawn.'' 

\newglossaryentry{argument}
{
name=argument,
description={a connected series of statements designed to convince an audience of another statement.}
}

\newglossaryentry{premise}
{
name=premise,
description={a statement in an argument that provides evidence for the conclusion}
}

\newglossaryentry{conclusion}
{
name=conclusion,
description={the statement that an argument is trying to convince an audience of.}
}

 
Once we have a collection of statements, we can use them to build arguments. An \textsc{\gls{argument}} \label{def:Argument} is a connected series of statements designed to convince an audience of another statement. Here an audience might be a literal audience sitting in front of you at some public speaking engagement. Or it might be the readers of a book or article. The audience might even be yourself as you reason your way through a problem. Let's start with an example of an argument given to an external audience. This passage is from an essay by Peter Singer called ``Famine, Affluence, and Morality'' in which he tries to convince people in rich nations that they need to do more to help people in poor nations who are experiencing famine.

\begin{quotation}\noindent \textit{A contemporary philosopher writing in an academic journal} If it is in our power to prevent something bad from happening, without thereby sacrificing anything of comparable moral importance, we ought, morally, to do so. Famine is something bad, and it can be prevented without sacrificing anything of comparable moral importance. So, we ought to prevent famine. \citep{Singer1972} \label{singer_quote} \end{quotation} 

Singer wants his readers to work to prevent famine. This is represented by the last statement of the passage, ``we ought to prevent famine,'' which is called the conclusion of the passage. The \textsc{\gls{conclusion}} \label{def:conclusion} of an argument is the statement that the argument is trying to convince the audience of. The statements that do the convincing are called the \textsc{\glspl{premise}}. \label{def:premise}In this case, the argument has three premises: (1) ``If it is in our power to prevent something bad from happening, without thereby sacrificing anything of comparable moral importance, we ought, morally, to do so''; (2) ``Famine is something bad''; and (3) ``it can be prevented without sacrificing anything of comparable moral importance.''

Now let's look at an example of internal reasoning. 

\begin{quotation}\noindent\textit{Jack arrives at the track, in bad weather.} There is no one here. I guess the race is not happening. \label{racetrack}
\end{quotation}

In the passage above, the words in \textit{italics} explain the context for the reasoning, and the words in regular type represent what Jack is actually thinking to himself. \nix{(We will talk more about his way of representing reasoning in section \ref{sec:arguments_and_context}, below.)} This passage again has a premise and a conclusion. The premise is that no one is at the track, and the conclusion is that the race was canceled. The context gives another reason why Jack might believe the race has been canceled, the weather is bad. You could view this as another premise--it is very likely a reason Jack has come to believe that the race is canceled. In general, when you are looking at people's internal reasoning, it is often hard to determine what is actually working as a premise and what is just working in the background of their unconscious. %[We will talk more about this in section...]


\newglossaryentry{premise indicator}
{
name=premise indicator,
description={a word or phrase such as ``because'' used to indicate that what follows is the premise of an argument.}
}

\newglossaryentry{conclusion indicator}
{
name=conclusion indicator,
description={a word or phrase such as ``therefore'' used to indicate that what follows is the conclusion of an argument.}
}

When people give arguments to each other, they typically use words like ``therefore'' and ``because.'' These are meant to signal to the audience that what is coming is either a premise or a conclusion in an argument. Words and phrases like ``because'' signal that a premise is coming, so we call these \textsc{\glspl{premise indicator}}. Similarly, words and phrases like ``therefore'' signal a conclusion and are called \textsc{\glspl{conclusion indicator}}. The argument from Peter Singer (on page \pageref{singer_quote}) uses the conclusion indicator word, ``so.'' Table \ref{table:Indicators} is an incomplete list of indicator words and phrases in English.


\begin{table}
\begin{mdframed}[style=mytablebox]

\begin{longtabu}{X[1,p]X[2,p]}
\textbf{Premise Indicators:} & because, as, for, since, given that, for the reason that \\
\textbf{Conclusion Indicators:} & therefore, thus, hence, so, consequently, it follows that, in conclusion, as a result, then, must, accordingly, this implies that, this entails that, we may infer that \\
\end{longtabu}
\end{mdframed}
\caption{Premise and Conclusion Indicators.}
\label{table:Indicators}
\end{table}

\newglossaryentry{canonical form}
{
name=canonical form,
description={a method for representing arguments where each premise is written on a separate, numbered, line, followed by a horizontal bar and then the conclusion. Statements in the argument might be paraphrased for brevity and indicator words are removed.}
}


The two passages we have looked at in this section so far have been simply presented as quotations. But often it is extremely useful to rewrite arguments in a way that makes their logical structure clear. One way to do this is to use something called ``canonical form.''   An argument written in \textsc{\gls{canonical form}} \label{def:canonical_form}has each premise numbered and written on a separate line. Indicator words and other unnecessary material should be removed from the premises. Although you can shorten the premises and conclusion, you need to be sure to keep them all complete sentences with the same meaning, so that they can be true or false. The argument from Peter Singer, above, looks like this in canonical form:

\begin{earg}
\item[P$_1$:] If we can stop something bad from happening, without sacrificing anything of comparable moral importance, we ought to do so. 
\item[P$_2$:] Famine is something bad.
\item[P$_3$:] Famine can be prevented without sacrificing anything of comparable moral importance.
\vspace{-.5em}
\item [] \rule{0.9\linewidth}{.5pt} 
\item[C:] We ought to prevent famine.
\end{earg} 

Each statement has been written on its own line and given a number. The statements have been paraphrased slightly, for brevity, and the indicator word ``so'' has been removed. Also notice that the ``it'' in the third premise has been replaced by the word ``famine,'' so that statements reads naturally on its own.  

Similarly, we can rewrite the argument Jack gives at the racetrack, on page \pageref{racetrack}, like this:

\begin{earg}
\item[P:] There is no one at the race track.
\vspace{-.5em}
\item [] \rule{0.4\linewidth}{.5pt} 
\item[C:] The race is not happening. 
\end{earg} 

Notice that we did not include anything from the part of the passage in italics. The italics represent the context, not the argument itself. Also, notice that the ``I guess'' has been removed. When we write things out in canonical form, we write the content of the statements, ignore information about the speaker's mental state, like ``I believe'' or ``I guess.'' 

One of the first things you have to learn to do in philosophy is to identify arguments and rewrite them in canonical form. This is a foundational skill for everything else we will be doing in this course, so we are going to run through a few examples now, and there will be more in the exercises. The passage below is paraphrased from the ancient Greek philosopher Aristotle. 

\begin{quotation}\noindent \textit{An ancient philosopher, writing for his students} Again, our observations of the stars make it evident that the earth is round. For quite a small change of position to south or north causes a manifest alteration in the stars which are overhead. (\cite{Aristotle:heavens}, 298a2-10)
\label{on_the_heavens} \end{quotation}

The first thing we need to do to put this argument in canonical form is to identify the conclusion. The indicator words are the best way to do this. The phrase ``make it evident that'' is a conclusion indicator phrase. He is saying that everything else is \textit{evidence} for what follows. So we know that the conclusion is that the earth is round. ``For'' is a premise indicator word---it is sort of a weaker version of ``because.''  Thus the premise is that the stars in the sky change if you move north or south. In canonical form, Aristotle's argument that the earth is round looks like this.\\


\begin{earg}
\item[P:] There are different stars overhead in the northern and southern parts of the earth.
\vspace{-.5em}
\item [] \rule{0.9\linewidth}{.5pt} 
\item[C:] The earth is spherical in shape. 
\end{earg} 

That one is fairly simple, because it just has one premise. Here's another example of an argument, this time from the book of Ecclesiastes in the Bible. The speaker in this part of the bible is generally referred to as The Preacher, or in Hebrew, Koheleth. In this verse, Koheleth uses both a premise indicator and a conclusion indicator to let you know he is giving reasons for enjoying life.

\begin{quotation}
\noindent \textit{The words of the Preacher, son of David, King of Jerusalem} There is something else meaningless that occurs on earth: the righteous who get what the wicked deserve, and the wicked who get what the righteous deserve. \ldots So I commend the enjoyment of life, because there is nothing better for a person under the sun than to eat and drink and be glad. (Ecclesiastes 8:14-15, New International Version)
\end{quotation}

Koheleth begins by pointing out that good things happen to bad people and bad things happen to good people. This is his first premise. (Most Bible teachers provide some context here by pointing that that the ways of God are mysterious and this is an important theme in Ecclesiastes.) Then Koheleth gives his conclusion, that we should enjoy life, which he marks with the word ``so.'' Finally he gives an extra premise, marked with a ``because,'' that there is nothing better for a person than to eat and drink and be glad. In canonical form, the argument would look like this.


\begin{earg}
\item[P$_1$:] Good things happen to bad people and bad things happen to good people.
\item[P$_2$:] There is nothing better for people than to eat, to drink and to enjoy life.
\vspace{-.5em}
\item [] \rule{0.8\linewidth}{.5pt} 
\item[C:] You should enjoy life.
\end{earg} 

Notice that in the original passages, Aristotle put the conclusion in the first sentence, while Koheleth put it in the middle of the passage, between two premises. In ordinary English, people can put the conclusion of their argument where ever they want. However, when we write the argument in canonical form, the conclusion goes last.

Unfortunately, indicator words aren't a perfect guide to when people are giving an argument. Look at this passage from a newspaper:

\begin{quotation}
\noindent \textit{From the general news section of a national newspaper} The new budget underscores the consistent and paramount importance of tax cuts in the Bush philosophy. His first term cuts affected more money than any other initiative undertaken in his presidency, including the costs thus far of the war in Iraq. All told, including tax incentives for health care programs and the extension of other tax breaks that are likely to be taken up by Congress, the White House budget calls for nearly \$300 billion in tax cuts over the next five years, and \$1.5 trillion over the next 10 years.  \citep{Toner2006}
\end{quotation}

Although there are no indicator words, this is in fact an argument. The writer wants you to believe something about George Bush: tax cuts are his number one priority. The next two sentences in the paragraph give you reasons to believe this. You can write the argument in canonical form like this.

\begin{earg}
\item[P$_1$:] Bush's first term cuts affected more money than any other initiative undertaken in his presidency, including the costs thus far of the war in Iraq. 
\item[P$_2$:] The White House budget calls for nearly \$300 billion in tax cuts over the next five years, and \$1.5 trillion over the next 10 years. 
\vspace{-.5em}
\item [] \rule{0.9\linewidth}{.5pt} 
\item[C:] Tax cuts are of consistent and paramount importance of in the Bush philosophy.
\end{earg} 

The ultimate test of whether something is an argument is simply whether some of the statements provide reason to believe another one of the statements. If some statements support others, you are looking at an argument. The speakers in these two cases use indicator phrases to let you know they are trying to give an argument.

\newglossaryentry{inference}
{
name=inference,
description={the act of coming to believe a conclusion on the basis of some set of premises.}
}

A final bit of terminology for this section. An \textsc{\gls{inference}} \label{def:Inference} is the act of coming to believe a conclusion on the basis of some set of premises. When Jack in the example above saw that no one was at the track, and came to believe that the race was not on, he was making an inference. We also use the term inference to refer to the connection between the premises and the conclusion of an argument. If your mind moves from premises to conclusion, you make an inference, and the premises and the conclusion are said to be linked by an inference. In that way inferences are like argument glue: they hold the premises and conclusion together. 

%%%% Practice Problems


\practiceproblems

\noindent\problempart Decide whether the following passages are statements and give reasons for your answers.

\begin{longtabu}{p{.1\linewidth}p{.9\linewidth}}
\textbf{Example}: & Did you follow the instructions? \\
\textbf{Answer}: & Not a statement, a question. \\
\end{longtabu}


\begin{exercises}
\item England is smaller than China. \answerblank{\underline{Statement}}{\vspace{.25in}}
\item Greenland is south of Jerusalem. \answerblank{\underline{Statement}}{\vspace{.25in}}
\item Is New Jersey east of Wisconsin? \answerblank{\underline{A question, not a Statement.}}{\vspace{.25in}}
\item The atomic number of helium is 2. \answerblank{\underline{Statement}}{\vspace{.25in}}
\item The atomic number of helium is $\pi$. \answerblank{\underline{Statement}}{\vspace{.25in}}
\item I hate overcooked noodles. \answerblank{\underline{Statement}}{\vspace{.25in}}
\item Blech! Overcooked noodles! \answerblank{\underline{An exclamation, not a statement.}}{\vspace{.25in}}
\item Overcooked noodles are disgusting.\answerblank{\underline{Statement}}{\vspace{.25in}}
\item Take your time. \answerblank{\underline{A command, not a Statement}}{\vspace{.25in}}
\item This is the last question. \answerblank{\underline{Statement}}{\vspace{.25in}}
\end{exercises}


\noindent\problempart Decide whether the following passages are statements and give reasons for your answers.
\answer{Answers from Ben Sheredos.}
\begin{exercises}
\item Is this a question? \answer{\underline{Question, not a statement.}}
\item Nineteen out of the 20 known species of Eurasian elephants are extinct. \answer{\underline{Statement; has to be true or false (might be false bc 20 is the wrong number, or because they are not extinct, etc.)}}
\item The government of the United Kingdom has formally apologized for the way it treated the logician Alan Turing. \answer{\underline{ Statement: has to be true or false; they either have or have not apologized}} 

\item Texting while driving \answer{\underline{Not a statement, but a sentence fragment}}
\item Texting while driving is dangerous. \answer{\underline{Statement; has to be true or false.}}
\item Insanity ran in the family of logician Bertrand Russell, and he had a life-long fear of going mad. \answer{\underline{Complex, but a statement: both halves are true or false, so is the whole.}}
\item For the love of Pete, put that thing down before someone gets hurt!  \answer{\underline{Not a statement: First bit is an exclamation, second is a command.}}
\item Don't try to make too much sense of this. \answer{\underline{Not a statement, a command.}}
\item Never look a gift horse in the mouth.  \answer{\underline{Not a statement, a command.}}
\item The physical impossibility of death in the mind of someone living  \answer{\underline{ Not a statement, sentence fragment.}}
\end{exercises}

\noindent\problempart Rewrite each of the following arguments in canonical form. Be sure to remove all indicator words and keep the premises and conclusion as complete sentences. Write the indicator words and phrases separately and state whether they are premise or conclusion indicators. 

%NTS: when writing these problems, be sure to include a mix of conclusion-first, conclusion-last and conclusion middle, as well as a mix of arguments with true and false premises and a variety of indicator words (or lack thereof).

\begin{longtabu}{p{.1\linewidth}p{.9\linewidth}}	
\textbf{Example}: & \textit{An ancient philosopher writes} We should not be distressed or concerned by the thought of our our own death in any way. Why? Look back on the time before you were born: It is a time you did not exist, but it does not trouble you in any way. The time after you die is also a time when you will not exist, so it shouldn't trouble you either. (Based on Lucretius \citetitle{Lucretius2001} 3.972--75)\\
\textbf{Answer}: & 
\vspace{-16pt}
\begin{earg}
\item[P$_1$:] The time before you were born is a time you did not exist.
\item[P$_2$:] You are not troubled by the time before you were born. 
\item[P$_3$:] The time after you die is also a time you will not exist.
\vspace{-.5em}
\item [] \rule{0.6\linewidth}{.5pt} 
\item[C:] We should not be distressed or concerned by the thought of our our own death. 
\end{earg} 
Premise indicator: So
\\
\end{longtabu}
	
\begin{exercises}

\item \textit{A detective is speaking: }Henry's finger-prints were found on the stolen computer. So, I infer that Henry stole the computer.  

\answerblank{
\begin{earg*}
\item Henry's finger-prints were found on the stolen computer
\itemc Henry stole the computer.  
\end{earg*}
Conclusion indicator word: So}{\vspace{1.5in}}


\item \textit{Monica is wondering about her co-workers political opinions} You cannot both oppose abortion and support the death penalty, unless you think there is a difference between fetuses and felons. Steve opposes abortion and supports the death penalty. Therefore Steve thinks there is a difference between fetuses and felons. 
		%Conclusion-last

\answerblank{
\begin{earg*}
\item You cannot both oppose abortion and support the death penalty, unless you think there is a difference between fetuses and felons. 
\item Steve opposes abortion and supports the death penalty. 
\itemc Steve thinks there is a difference between fetuses and felons. 
\end{earg*}
Conclusion Indicator: Therefore}{\vspace{1.5in}}


\item \textit{The Grand Moff of Earth defense considers strategy} We know that whenever people from one planet invade another, they always wind up being killed by the local diseases, because in 1938, when Martians invaded the Earth, they were defeated because they lacked immunity to Earth's diseases. Also, in 1942, when Hitler's forces landed on the Moon, they were killed by Moon diseases.
		%Conclusion-first

\answerblank{
\begin{earg} 
\item[1.] In 1938, when Martians invaded the Earth, they were defeated because they lacked immunity to Earth's diseases. 
\item[2.] In 1942, when Hitler's forces landed on the Moon, they were killed by Moon diseases.
\item [] \noindent\hrulefill 
\item[$\therefore$] Whenever people from one planet invade another, they always wind up being killed by the local diseases, 
\end{earg}
Premise indicator: Because }{\vspace{1.5in}}


\item If you have slain the Jabberwock, my son, it will be a frabjous day. The Jabberwock lies there dead, its head cleft with your vorpal sword. This is truly a fabjous day. 
%Conclusion-last
\answerblank{ 
\begin{earg*} 
\item  If you have slain the Jabberwock, my son, it will be a frabjous day. 
\item The Jabberwock lies there dead
 
\itemc This is truly a fabjous day 
\end{earg*}
Indicators: none		
}{\vspace{1.5in}}	

\item \textit{A detective trying to crack a case thinks to herself} Miss Scarlett was jealous that Professor Plum would not leave his wife to be with her. Therefore she must be the killer, because she is the only one with a motive. 
%Conclusion-middle
\answerblank{
\begin{earg*} 
\item Miss Scarlett was jealous that Professor Plum would not leave his wife to be with her. 
\item Miss Scarlett is the only one with a motive. 
 
\itemc Miss Scarlett must be the killer
\end{earg*}

Premise Indicator: Because \\
Conclusion Indicator: Therefore}{\vspace{1.5in}}
\end{exercises}



\noindent\problempart Rewrite each of the following arguments in canonical form. Be sure to remove all indicator words and keep the premises and conclusion as complete sentences. Write the indicator words and phrases separately and state whether they are premise or conclusion indicators. 

\answer{Answers from Ben Sheredos.}

\begin{enumerate}[label=\arabic*), topsep=0pt, parsep=0pt, itemsep=6pt]
\item \textit{A pundit is speaking on a Sunday political talk show} Hillary Clinton should drop out of the race for Democratic Presidential nominee. For every day she stays in the race, McCain gets a day free from public scrutiny and the members of the Democratic party get to fight one another.  
			%Conclusion-first

\answer{ 
	\begin{earg*} 
		\item For every day Hillary stays in the race, McCain gets a day free from public scrutiny and the members of the Democratic party get to fight one another.
		\itemc Hillary Clinton should drop out of the race for Democratic Presidential Nominee.
	\end{earg*}
	"For" could be a premise-indicator, functioning like "since."
}


\item You have to be smart to understand the rules of Dungeons and Dragons. Most smart people are nerds. So, I bet most people who play D\&D are nerds.  
			%Conclusion-last

\answer{ 
			\begin{earg*} 
				\item You have to be smart to understand the rules of D\&D.
				\item Most smart people are nerds.
				\itemc $[I bet]$ most people who play D\&D are nerds.
			\end{earg*}
			"So" is definitely a conclusion-indicator; "I bet" is probably part of a conclusion-indicator as well, with the speaker indicating that they think this argument is a bit weak.
		}

\item Any time the public receives a tax rebate, consumer spending increases. Since the public just received a tax rebate, consumer spending will increase. 
		%Conclusion-last

\answer{ 
	\begin{earg*} 
		\item Any time the public receives a tax rebate, consumer spending increases. 
		\item The public just received a tax rebate.
		\itemc Consumer spending will increase.
	\end{earg*}
	"Since" is a premise-indicator, but the last sentence needs to be split up into premise and conclusion. This would be more clear if the speaker said "\underline{Since} the public just received a tax rebate, \underline{it follows that} consumer spending will increase." Our speaker is lazy.
}

\item Isabelle is taller than Jacob. Kate must also be taller than Jacob, because she is taller than Isabelle. 
%conclusion-middle

\answer{ 
	\begin{earg*} 
		\item Isabelle is taller than Jacob.
		\item Kate is taller than Isabelle.
		\itemc Kate is taller than Jacob.
	\end{earg*}
	"Must" is a conclusion indicator, "because" is a premise-indicator, and so the last sentence has to be split up to put this argument into canonical form.
}
\end{enumerate}

% * **********************************
% *     Arguments and Nonarguments          *
% ************************************

\section{Arguments and Nonarguments}
\label{sec:arguments_and_nonarguments}

We just saw that arguments are made of statements. However, there are lots of other things you can do with statements. Part of learning what an argument is involves learning what an argument is not, so in this section and the next we are going to look at some other things you can do with statements besides make arguments. 

The list below of kinds of nonarguments is not meant to be exhaustive: there are all sorts of things you can do with statements that are not discussed. Nor are the items on this list meant to be exclusive. One passage may function as both, for instance, a narrative and a statement of belief. Right now we are looking at real world reasoning, so you should expect a lot of ambiguity and imperfection. If your class is continuing on into the critical thinking portions of this textbook, you will quickly get used to this. 

\subsection{Simple Statements of Belief}

\newglossaryentry{simple statement of belief}
{
name=simple statement of belief,
description={A kind of nonargumentative passage where the speaker simply asserts what they believe without giving reasons. }
}

An argument is an attempt to persuade an audience to believe something, using reasons. Often, though, when people try to persuade others to believe something, they skip the reasons, and give a \textsc{\gls{simple statement of belief}}. \label{def:simple_statement_of_belief} This is a kind of nonargumentative passage where the speaker simply asserts what they believe without giving reasons. Sometimes simple statements of belief are prefaced with the words ``I believe,'' and sometimes they are not. A simple statements of belief can be a profoundly inspiring way to change people's hearts and minds. Consider this passage from Dr. Martin Luther King's Nobel acceptance speech.

\begin{quotation} \noindent I believe that even amid today's mortar bursts and whining bullets, there is still hope for a brighter tomorrow. I believe that wounded justice, lying prostrate on the blood-flowing streets of our nations, can be lifted from this dust of shame to reign supreme among the children of men. I have the audacity to believe that peoples everywhere can have three meals a day for their bodies, education and culture for their minds, and dignity, equality and freedom for their spirits. \citep{King2001} \end{quotation}

This actually is a part of a longer passage that consists almost entirely of statements that begin with some variation of ``I believe.''It is incredibly powerful oration, because the audience, feeling the power of King's beliefs, comes to share in those beliefs. The language King uses to describe how he believes is important, too. He says his belief in freedom and equality requires audacity, making the audience feel his courage and want to share in this courage by believing the same things. 

These statements are moving, but they do not form an argument. None of these statements provide evidence for any of the other statements. In fact, they all say roughly the same thing, that good will triumph over evil. %So the study of this kind of speech belongs to the discipline of rhetoric, not of logic.  
  
\subsection{Expository Passages}

Perhaps the most basic use of a statement is to convey information. Often if we have a lot of information to convey, we will sometimes organize our statements around a theme or a topic. Information organized in this fashion can often appear like an argument, because all of the statements in the passage relate back to some central statement. However, unless the other statements are given as reasons to believe the central statement, the passage you are looking at is not an argument. Consider this passage:

\begin{quotation}\noindent\textit{From a college psychology textbook.} Eysenck advocated three major behavior techniques that have been used successfully to treat a variety of phobias. These techniques are modeling, flooding, and systematic desensitization. In \textbf{modeling} phobic people watch nonphobics cope successfully with dreaded objects or situations.In \textbf{flooding} clients are exposed to dreaded objects or situations for prolonged periods of time in order to extinguish their fear. In contrast to flooding, \textbf{systematic desensitization} involves gradual, client-controlled exposure to the anxiety eliciting object or situation. (Adapted from Ryckman \cite*{Ryckman2007}) \end{quotation}

\newglossaryentry{expository passage}
{
name=expository passage,
description={A nonargumentative passage that organizes statements around a central theme or topic statement.}
}

We call this kind of passage an expository passage. In an \textsc{\gls{expository passage}}, \label{def:expository_passage} statements are organized around a central theme or topic statement. The topic statement might look like a conclusion, but the other statements are not meant to be evidence for the topic statement. Instead, they elaborate on the topic statement by providing more details or giving examples. In the passage above, the topic statement is ``Eysenck advocated three major behavioral techniques \ldots.'' The statements describing these techniques elaborate on the topic statement, but they are not evidence for it. Although the audience may not have known this fact about Eysenk before reading the passage, they will typically accept the truth of this statement instantly, based on the textbook's authority. Subsequent statements in the passage merely provide detail. 

Deciding whether a passage is an argument or an expository passage is complicated by the fact that sometimes people argue by example: 

\begin{adjustwidth}{2em}{0em}
\begin{longtabu}{p{.1\linewidth}p{.8\linewidth}}
\textbf{Steve:} & Kenyans are better distance runners than everyone else. \\
\textbf{Monica:} & Oh come on, that sounds like an exaggeration of a stereotype that isn't even true.\\
\textbf{Steve:} & What about Dennis Kimetto, the Kenyan who set the world record for running the marathon? And you know who the previous record holder was: Emmanuel Mutai, also Kenyan. \\
\end{longtabu}
\end{adjustwidth}
\vspace{-1.5cm}

Here Steve has made a general statement about all Kenyans. Monica clearly doubts this claim, so Steve backs it up with some examples that seem to match his generalization. This isn't a very strong way to argue: moving from two examples to statement about all Kenyans is probably going to be a kind of bad argument known as a hasty generalization. (This mistake is covered in the complete version of this text in the chapter on induction\nix{Chapter \ref{chap:induction} on induction.}\label{ver_var}) The point here however, is that Steve is just offering it as an argument. 

The key to telling the difference between expository passages and arguments by example is whether there is a conclusion that they audience needs to be convinced of. In the passage from the psychology textbook, ``Eysenck advocated three major behavioral techniques'' doesn't really work as a conclusion for an argument. The audience, students in an introductory psychology course, aren't likely to challenge this assertion, the way Monica  challenges Steve's overgeneralizing claim. 

Context is very important here, too. The Internet is a place where people argue in the ordinary sense of exchanging angry words and insults. In that context, people are likely to actually give some arguments in the philosophical sense of giving reasons to believe a conclusion. 

\subsection{Narratives} 

Statements can also be organized into descriptions of events and actions, as in this snippet from book V of \textit{Harry Potter}.

\begin{quotation} \noindent But she [Hermione] broke off; the morning post was arriving and, as usual, the \textit{Daily Prophet} was soaring toward her in the beak of a screech owl, which landed perilously close to the sugar bowl and held out a leg. Hermione pushed a Knut into its leather pouch, took the newspaper, and scanned the front page critically as the owl took off again. \citep{Rowling2003} \end{quotation} 

\newglossaryentry{narrative}
{
name=narrative,
description={A nonargumentative passage that describes a sequence of events or actions.}
}

We will use the term \textsc{\gls{narrative}} \label{def:narrative} loosely to refer to any passage that gives a sequence of events or actions. A narrative can be fictional or nonfictional. It can be told in regular temporal sequence or it can jump around, forcing the audience to try to reconstruct a temporal sequence. A narrative can describe a short sequence of actions, like Hermione taking a newspaper from an owl, or a grand sweep of events, like this passage about the  rise and fall of an empire in the ancient near east:

\begin{quotation}\noindent The Guti were finally expelled from Mesopotamia by the Sumerians of Erech (\textit{c}. 2100), but it was left to the kings of Ur's famous third dynasty to re-establish the Sargonoid frontiers and write the final chapter of the Sumerian History. The dynasty lasted through the twenty first century at the close of which the armies of Ur were overthrown by the Elamites and Amorites \citep{McEvedy1967}. \end{quotation} 

This passage does not feature individual people performing specific actions, but it is still united by character and action. Instead of Hermione at breakfast, we have the Sumarians in Mesopotamia. Instead of retrieving a message from an owl, the conquer the Guti, but then are conquered by the Elamites and Amorites. The important thing is that the statements in a narrative are not related as premises and conclusion. Instead, they are all events which are united common characters acting in specific times and places. 

%%%%%%% Practice Problems

\practiceproblems
\problempart Identify each passage below as an argument or a nonargument, and give reasons for your answers. If it is a nonargument, say what kind of nonargument you think it is. If it is an argument, write it out in canonical form.

\begin{longtabu}{p{.1\linewidth}p{.9\linewidth}}
\textbf{Example}: & \textit{One student speaks to another student who has missed class:} The instructor passed out the syllabus at 9:30. Then he went over some basic points about reasoning, arguments and explanations. Then he said we should call it a day. \\
\textbf{Answer}: & Not an argument, because none of the statements provide any support for any of the others. This is probably better classified as a narration because the events are in temporal sequence. \\
\end{longtabu}

\begin{exercises}
\item \textit{An anthropology teacher is speaking to her class }Different gangs use different colors to distinguish themselves. Here are some illustrations: biologists tend to wear some blue, while the philosophy gang wears black. 
\answerblank{\\ Not an argument. Expository passage. The students probably will believe the teacher as soon as she makes an assertion. The word ``illustration'' is also a clue.}{\vspace{1.5in}}

\item The economy has been in trouble recently. And it's certainly true that cell phone use has been rising during that same period. So, I suspect increasing cell phone use is bad for the economy. 
\answer{\\  Argument. The indicator ``so'' is a clue. 

\begin{earg*}
\item The economy has been in trouble recently. 
\item Cell phone use has been rising during that same period. 
\itemc Cell phone use is bad for the economy. 
\end{earg*}
}


\item \textit{At Widget-World Corporate Headquarters:} We believe that our company must deliver a quality product to our customers. Our customers also expect first-class customer service. At the same time, we must make a profit. 

%\vspace{6pt}
\answerblank{ Not an argument. The speaker is not using any of the propositions as reasons to believe or explain any of the others; rather she is simply asserting various things.}{\vspace{1.5in}}
      
\item \textit{Jack is at the breakfast table and shows no sign of hurrying. Gill says:} You should leave now. It's almost nine a.m. and it takes three hours to get there.

\answerblank{Arguing. Jack's inaction suggests that he does believe that he needs to leave now and so Gill provides reasons that might convince him. Notice that there are no argument flag words or phrases.

This example also includes the word ``should'' in its conclusion. Words such as ``ought'' and ``should'' indicate that the speaker is trying to get the audience to do or believe something that they are not currently doing or believing.

\begin{earg*}
\item It's almost nine a.m. 
\item It takes three hours to get there.
\itemc  You should leave now.
\end{earg*}
}{\vspace{1.5in}}
      
\item \textit{In a text book on the brain:} Axons are distinguished from dendrites by several features, including shape (dendrites often taper while axons usually maintain a constant radius), length (dendrites are restricted to a small region around the cell body while axons can be much longer), and function (dendrites usually receive signals while axons usually transmit them).

\answerblank{Not an argument. Expository passage. The features named just fill in the first statement.}{\vspace{1.5in}}

\end{exercises}
%
\problempart Identify each passage below as an argument or a nonargument, and give reasons for your answers. If it is a nonargument, say what kind of nonargument you think it is. If it is an argument, write it out in canonical form.

%
\begin{exercises}
\item \textit{Suzi doesn't believe she can quit smoking. Her friend Brenda says} Some people have been able to give up cigarettes by using their will-power. Everyone can draw on their will-power. So, anyone who wants to give up cigarettes can do so.

\item \textit{The words of the Preacher, son of David, King of Jerusalem} I have seen something else under the sun: The race is not to the swift or the battle to the strong, nor does food come to the wise or wealth to the brilliant or favor to the learned; but time and chance happen to them all. (Ecclesiastes 9:11, New International Version)

\item \textit{An economic development expert is speaking.} The introduction of cooperative marketing into Europe greatly increased the prosperity of the farmers, so we may be confident that a similar system in Africa will greatly increase the prosperity of our farmers.

\item \textit{From the CBS News website, US section.} Headline: ``FBI nabs 5 in alleged plot to blow up Ohio bridge.'' Five alleged anarchists have been arrested after a months-long sting operation, charged with plotting to blow up a bridge in the Cleveland area, the FBI announced Tuesday. CBS News senior correspondent John Miller reports the group had been involved in a series of escalating plots that ended with their arrest last night by FBI agents. The sting operation supplied the anarchists with what they thought were explosives and bomb-making materials. At no time during the case was the public in danger, the FBI said. \citep{CBSNews2012}


\item \textit{At a school board meeting.} Since creationism can be discussed effectively as a scientific model, and since evolutionism is fundamentally a religious philosophy rather than a science, it is unsound educational practice for evolution to be taught and promoted in the public schools to the exclusion or detriment of special creation. (Kitcher \cite*{Kitcher1982}, p. 177, citing Morris \cite*{Morris1975}.)

\end{exercises}

\pagebreak
\section*{Key Terms}
\begin{multicols}{2}
\begin{sortedlist}
\sortitem{Metacognition}{} 	
\sortitem{Canonical form}{}
\sortitem{Conclusion indicator}{} 
\sortitem{Premise indicator}{}
\sortitem{Statement}{}
\sortitem{Argument}{}
\sortitem{Conclusion}{}
\sortitem{Premise}{}
\sortitem{Inference}{}
\sortitem{Simple statement of belief}{}
\sortitem{Expository passage}{}
\sortitem{Narrative}{}
\sortitem{Descriptive statement}{}
\sortitem{Normative statement}{}
\end{sortedlist}
\end{multicols}	

%Bibstuff
%If the {part:CT} label is found, LaTeX will typeset separate bibliographies for sample passages and logical sources.

\iflabelexists{part:CT}{%text for CT version}

\defbibnote{sample}{\textit{ \large  This bibliography includes all sources except for those that were used as examples for logical analysis, either in the main text or problem sets}}

\printbibliography [keyword=samplepassage, title=Bibliography of Sample Passages, prenote=sample, heading=bibnumbered] %for separate bibs sample passages and general citations

\printbibliography [notkeyword=samplepassage, title=General Bibliography, heading=bibnumbered] %for separate bibs sample passages and general citations

}% End CT version
{\printbibliography[heading=bibnumbered]} %single bib for non-CT version


%%The way I’ve set this up now is that there is one bib for sample passages and one bib for everything else. This means that it would not be possible to put one entry in both bibliographies. (This might be needed for Aristotle.) To do that, you will need to define a separate logicsource category


\setglossarysection{chapter}
\printglossaries








\end{document}
