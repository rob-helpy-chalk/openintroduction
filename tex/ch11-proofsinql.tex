\chapter{Proofs in Quantified Logic}
\markright{Chap \ref{chap:proofsinQL}: Proofs in QL}
\label{chap:proofsinQL}
\setlength{\parindent}{1em}

% ******************************************
%  *                       6.1 Rules for Quantifiers             *
%******************************************

\section{Rules for Quantifiers}

For proofs in QL, we use all of the basic rules of SL plus four new basic rules: both introduction and elimination rules for each of the quantifiers.

Since all of the derived rules of SL are derived from the basic rules, they will also hold in QL. We will add another derived rule, a replacement rule called quantifier negation.

\subsection{Universal elimination}

If you have $\forall x Ax$, it is legitimate to infer that anything is an $A$. You can infer $Aa$, $Ab$, $Az$, $Ad_3$--- in short, you can infer $A\script{c}$ for any constant \script{c}. This is the general form of the universal elimination rule ($\forall$E):

\begin{proof}
	\have[m]{a}{\forall \script{x}\script{A}}
	\have[\ ]{c}{\script{A}[\script{c}|\script{x}]} \Ae{a}
\end{proof}

$\script{A}[\script{c}|\script{x}]$ is a substitution instance of $\forall\script{x}\script{A}$. The symbols for a substitution instance are not symbols of QL, so you cannot write them in a proof. Instead, you write the subsituted sentence with the constant \script{c} replacing all occurrences of the variable \script{x} in \script{A}. For example:

\begin{proof}
	\hypo{a}{\forall x(Mx \eif Rxd)}
	\have{c}{Ma \eif Rad} \Ae{a}
	\have{d}{Md \eif Rdd} \Ae{a}
\end{proof}


\subsection{Existential introduction}

When is it legitimate to infer $\exists x Ax$? If you know that something is an $A$--- for instance, if you have $Aa$ available in the proof.

This is the existential introduction rule ($\exists$I):

\begin{proof}
	\have[m]{a}{\script{A}}
	\have[\ ]{c}{\exists \script{x}\script{A}[\script{x}||\script{c}]} \Ei{a}
\end{proof}

It is important to notice that $\script{A}[\script{x}||\script{c}]$ is not the same as a substitution instance. We write it with two bars to show that the variable \script{x} does not need to replace all occurrences of the constant \script{c}. You can decide which occurrences to replace and which to leave in place. For example:

\begin{proof}
	\hypo{a}{Ma \eif Rad}
	\have{b}{\exists x(Ma \eif Rax)} \Ei{a}
	\have{c}{\exists x(Mx \eif Rxd)} \Ei{a}
	\have{d}{\exists x(Mx \eif Rad)} \Ei{a}
	\have{e}{\exists y\exists x(Mx \eif Ryd)} \Ei{d}
	\have{f}{\exists z\exists y\exists x(Mx \eif Ryz)} \Ei{e}
\end{proof}


\subsection{Universal introduction}
A universal claim like $\forall x Px$ would be proven if {every} substitution instance of it had been proven, if every sentence $Pa$, $Pb$, $\ldots$ were available in a proof. Alas, there is no hope of proving \emph{every} substitution instance. That would require proving $Pa$, $Pb$, $\ldots$, $Pj_2$, $\ldots$, $Ps_7$, $\ldots$, and so on to infinity. There are infinitely many constants in QL, and so this process would never come to an end.

Consider a simple argument: $\forall x Mx$, \therefore\ $\forall y My$

It makes no difference to the meaning of the sentence whether we use the variable $x$ or the variable $y$, so this argument is obviously valid. Suppose we begin in this way:

\begin{proof}
	\hypo{x}{\forall x Mx} \by{want $\forall y My$}{}
	\have{a}{Ma} \Ae{x}
\end{proof}

We have derived $Ma$. Nothing stops us from using the same justification to derive $Mb$, $\ldots$, $Mj_2$, $\ldots$, $Ms_7$, $\ldots$, and so on until we run out of space or patience. We have effectively shown the way to prove $M\script{c}$ for any constant \script{c}. From this, $\forall x Mx$ follows.

\begin{proof}
	\hypo{x}{\forall x Mx}
	\have{a}{Ma} \Ae{x}
	\have{y}{\forall y My} \Ai{a}
\end{proof}

It is important here that $a$ was just some arbitrary constant. We had not made any special assumptions about it. If $Ma$ were a premise of the argument, then this would not show anything about \emph{all} $y$. For example:

\begin{proof}
	\hypo{x}{\forall x Rxa}
	\have{a}{Raa} \Ae{x}
	\have{y}{\forall y Ryy} \by{not allowed!}{}
\end{proof}


This is the schematic form of the universal introduction rule ($\forall$I):

\begin{proof}
	\have[m]{a}{\script{A}}
	\have[\ ]{c}{\forall \script{x}\script{A}[\script{x}|\script{c}]^\ast} \Ai{a}
\end{proof}
$^\ast$ \script{c} must not occur in any undischarged assumptions.

Note that we can do this for any constant that does not occur in an undischarged assumption and for any variable.

Note also that the constant may not occur in any \emph{undischarged} assumption, but it may occur as the assumption of a subproof that we have already closed. For example, we can prove $\forall z(Dz \eif Dz)$ without any premises.

\begin{proof}
	\open
		\hypo{f1}{Df}\by{want $Df$}{}
		\have{f2}{Df}\by{R}{f1}
	\close
	\have{ff}{Df \eif Df}\ci{f1-f2}
	\have{zz}{\forall z(Dz \eif Dz)}\Ai{ff}
\end{proof}


\subsection{Existential elimination}
A sentence with an existential quantifier tells us that there is \emph{some} member of the UD that satisfies a formula. For example, $\exists x Sx$ tells us (roughly) that there is at least one $S$. It does not tell us \emph{which} member of the UD satisfies $S$, however. We cannot immediately conclude $Sa$, $Sf_{23}$, or any other substitution instance of the sentence. What can we do?

Suppose that we knew both $\exists x Sx$ and $\forall x(Sx \eif Tx)$. We could reason in this way:
\begin{quote}
Since $\exists x Sx$, there is something that is an $S$. We do not know which constants refer to this thing, if any do, so call this thing $\Omega$. From $\forall x(Sx \eif Tx)$, it follows that if $\Omega$ is an $S$, then it is a $T$. Therefore $\Omega$ is a $T$.  Because $\Omega$ is a $T$, we know that $\exists x Tx$.
\end{quote}
In this paragraph, we introduced a name for the thing that is an $S$. We called it $\Omega$, so that we could reason about it and derive some consequences from there being an $S$. Since $\Omega$ is just a bogus name introduced for the purpose of the proof and not a genuine constant, we could not mention it in the conclusion. Yet we could derive a sentence that does not mention $\Omega$; namely, $\exists x Tx$. This sentence does follow from the two premises.

We want the existential elimination rule to work in a similar way. Yet since Greek letters like $\Omega$ are not symbols of QL, we cannot use them in formal proofs. Instead, we will use constants of QL which do not otherwise appear in the proof. A constant that is used to stand in for whatever it is that satisfies an existential claim is called a \define{proxy}. Reasoning with the proxy must all occur inside a subproof, and the proxy cannot be a constant that is doing work elsewhere in the proof.

This is the schematic form of the existential elimination rule ($\exists$E): 

\begin{proof}
	\have[m]{a}{\exists \script{x}\script{A}}
	\open	
		\hypo[n]{b}{\script{A}[\script{c}|\script{x}]^\ast}
		\have[p]{c}{\script{B}}
	\close
	\have[\ ]{d}{\script{B}} \Ee{a,b-c}
\end{proof}
$^\ast$ The constant \script{c} must not appear in $\exists\script{x}\script{A}$, in \script{B}, or in any undischarged assumption.

With this rule, we can give a formal proof that $\exists x Sx$ and $\forall x(Sx \eif Tx)$ together entail $\exists x Tx$. The structure of the proof is effectively the same as the English-language argument with which we began, except that the subproof uses the constant ``$a$'' rather than the bogus name $\Omega$.

\begin{proof}
	\hypo{es}{\exists x Sx}
	\hypo{ast}{\forall x(Sx \eif Tx)}\by{want $\exists x Tx$}{}
	\open
		\hypo{s}{Sa}
		\have{st}{Sa \eif Ta}\Ae{ast}
		\have{t}{Ta} \ce{s,st}
		\have{et1}{\exists x Tx}\Ei{t}
	\close
	\have{et2}{\exists x Tx}\Ee{es,s-et1}
\end{proof}

\subsection{Quantifier negation}

When translating from English to QL, we noted that $\enot\exists x\enot\script{A}$ is logically equivalent to $\forall x\script{A}$. In QL, they are provably equivalent. We can prove one half of the equivalence with a rather gruesome proof:

\begin{proof}
	\hypo{Aa}{\forall x Ax} \by{want $\enot\exists x\enot Ax$}{}
	\open
		\hypo{Ena}{\exists\enot Ax}\by{for reductio}{}
		\open
			\hypo{nc}{\enot Ac}\by{for $\exists$E}{}
			\open
				\hypo{Aa2}{\forall x Ax}\by{for reductio}{}
				\have{c2}{Ac}\Ae{Aa}
				\have{nc2}{\enot Ac}\by{R}{nc}
			\close
			\have{nAa}{\enot\forall x Ax}\ni{Aa2-nc2}
		\close
		\have{Aa3}{\forall x Ax}\by{R}{Aa}
		\have{nAa3}{\enot\forall x Ax}\Ee{nc-nAa}
	\close
	\have{nEna}{\enot\exists\enot Ax}\ni{Ena-nAa3}
\end{proof}

In order to show that the two sentences are genuinely equivalent, we need a second proof that assumes $\enot\exists x\enot\script{A}$ and derives $\forall x\script{A}$. We leave that proof as an exercise for the reader.

It will often be useful to translate between quantifiers by adding or subtracting negations in this way, so we add two derived rules for this purpose. These rules are called quantifier negation (QN):
\begin{center}
\begin{tabular}{rl}
$\enot\forall\script{x}\script{A} \Longleftrightarrow \exists\script{x}\enot\script{A}$\\
$\enot\exists\script{x}\script{A} \Longleftrightarrow \forall\script{x}\enot\script{A}$
& QN
\end{tabular}
\end{center}
Since QN is a replacement rule, it can be used on whole sentences or on subformulae.

%%%%%%%%%   		Practice problems for Section 6.1                %%%%

% rob: I put all the quantification problems from the original chapter 6 that don't involve identity or models in this section. 

\practiceproblems

\setlength{\parindent}{0pt}

\problempart
\label{pr.justifyQLproof}
Provide a justification (rule and line numbers) for each line of proof that requires one.

\begin{enumerate}[label=\arabic*)]
\begin{multicols}{2}
\begin{minipage}{\linewidth}
\item \textcolor{white}{.} \\  %$\vdash \exists x Mx \eor \forall x\enot Mx$
\vspace{-24pt}
\begin{proof}
	\open
		\hypo{p1}{\enot (\exists x Mx \eor \forall x\enot Mx)}
		\have{p2}{\enot \exists x Mx \eand \enot \forall x\enot Mx}{}
		\have{p3}{\enot \exists x Mx}{}
		\have{p4}{\forall x\enot Mx}{}
		\have{p5}{\enot \forall x\enot Mx}{}
	\close
\have{n}{\exists x Mx \eor \forall x\enot Mx} {}
\end{proof}
\end{minipage}

\pagebreak[4]
\item \textcolor{white}{.} \\ %$\{\forall x(\exists y)(Rxy \eor Ryx),\forall x\enot Rmx\}\vdash\exists xRxm$
\vspace{-16pt}
\begin{proof}
\hypo{p1}{\forall x\exists y(Rxy \eor Ryx)}
\hypo{p2}{\forall x\enot Rmx}
\have{3}{\exists y(Rmy \eor Rym)}{}
	\open
		\hypo{a1}{Rma \eor Ram}
		\have{a2}{\enot Rma}{}
		\have{a3}{Ram}{}
		\have{a4}{\exists x Rxm}{}
	\close
\have{n}{\exists x Rxm} {}
\end{proof}

\vspace{2ex}

\item \textcolor{white}{.} \\ %$\{\forall x(\exists yLxy \eif \forall zLzx), Lab\} \vdash \forall xLxx$
\vspace{-16pt}
\begin{proof} 
\hypo{1}{\forall x(\exists yLxy \eif \forall zLzx)}
\hypo{2}{Lab}
\have{3}{\exists y Lay \eif \forall zLza}{}
\have{4}{\exists y Lay} {}
\have{5}{\forall z Lza} {}
\have{6}{Lca}{}
\have{7}{\exists y Lcy \eif \forall zLzc}{}
\have{8}{\exists y Lcy}{}
\have{9}{\forall z Lzc}{}
\have{10}{Lcc}{}
\have{11}{\forall x Lxx}{}
\end{proof}



\item \textcolor{white}{.} \\ % $\{\forall x(Jx \eif Kx), \exists x\forall y Lxy, \forall x Jx\} \vdash \exists x(Kx \eand Lxx)$
\vspace{-16pt}
\begin{proof}
\hypo{a}{\forall x(Jx \eif Kx)}
\hypo{b}{\exists x\forall y Lxy}
\hypo{c}{\forall x Jx}
\have{d}{Ja}{}
\have{e}{Ja \eif Ka}{}
\have{f}{Ka}{}
\open
	\hypo{2}{\forall y Lay}
	\have{3}{Laa}{}
	\have{4}{Ka \eand Laa}{}
	\have{5}{\exists x(Kx \eand Lxx)}{}
\close
\have{j}{\exists x(Kx \eand Lxx)}{}
\end{proof}
\end{multicols}
\end{enumerate}

\problempart Without using the QN rule, prove $\enot\exists x\enot\script{A} \sststile{}{} \forall x\script{A}$

\problempart
\label{pr.someQLproofs}
Provide a proof of each claim.
\begin{earg}
\item $\sststile{}{}  \forall x Fx \eor \enot \forall x Fx$
\item $\{\forall x(Mx \eiff Nx), Ma\eand\exists x Rxa\}\sststile{}{}  \exists x Nx$
\item $\{\forall x(\enot Mx \eor Ljx), \forall x(Bx\eif Ljx), \forall x(Mx\eor Bx)\}\sststile{}{}  \forall xLjx$
\item $\forall x(Cx \eand Dt)\sststile{}{}  \forall xCx \eand Dt$
\item $\exists x(Cx \eor Dt)\sststile{}{}  \exists x Cx \eor Dt$
\end{earg}

\problempart
%Provide a proof of the argument about Billy on p.~\pageref{surgeon2}.

In the previous chapter (p.~\pageref{surgeon2}), we gave the following example

	\begin{quote}
	
	
	The hospital will only hire a skilled surgeon. All surgeons are greedy. Billy is a surgeon, but is not skilled. Therefore, Billy is greedy, but the hospital will not hire him.
	
	\begin{ekey}
	\item[UD:] people
	\item[Gx:] $x$ is greedy.
	\item[Hx:] The hospital will hire $x$.
	\item[Rx:] $x$ is a surgeon.
	\item[Kx:] $x$ is skilled.
	\item[b:] Billy
	\end{ekey}
	
	\begin{earg}
	\label{surgeon2}
	\item[] $\forall x\bigl[\enot (Rx \eand Kx) \eif \enot Hx\bigr]$
	\item[] $\forall x(Rx \eif Gx)$
	\item[] $Rb \eand \enot Kb$
	\item[\therefore] $Gb \eand \enot Hb$
	\end{earg}
	
	\end{quote}

Prove the symbolized argument. 

\problempart \label{pr.BarbaraEtc.proof1} \iflabelexists{chap:catstatements}{On page \pageref{table:full_twentyfour} you were introduced to the twenty-four valid Aristotelian syllogisms, and on page \pageref{venn_proofs} you were able to show 15 of these valid using Venn diagrams.  Now that we have translated them into QL (see page \pageref{pr.BarbaraEtc}) we can actually prove all of them valid. In this section, you will prove the unconditional forms. I have omitted Datisi and Ferio because their proofs are trivial variations on Darii and Ferison.}{On page \pageref{pr.BarbaraEtc} you translated the basic categorical syllogisms studied by Aristotle and his followers into QL. Now you need to provide derivations for some of them.}

\begin{enumerate}[label=\arabic*), topsep=0pt, parsep=0pt, itemsep=3pt]
\item \textbf{Barbara:} All $B$s are $C$s. All $A$s are $B$s.
	\therefore\  All $A$s are $C$s.
\item \textbf{Baroco:} All $C$s are $B$s. Some $A$ is not $B$.
	\therefore\  Some $A$ is not $C$.
\item \textbf{Bocardo:} Some $B$ is not $C$. All $B$s are $A$s.
	\therefore\  Some $A$ is not $C$.
\item\textbf{Celantes:} No $B$s are $C$s. All $A$s are $B$s.
	\therefore\  No $C$s are $A$s.
\item\textbf{Celarent:} No $B$s are $C$s. All $A$s are $B$s.
	\therefore\  No $A$s are $C$s.
\item\textbf{Campestres:} All $C$s are $B$s. No $A$s are $B$s.
	\therefore\  No $A$s are $C$s.
\item\textbf{Cesare:} No $C$s are $B$s. All $A$s are $B$s.
	\therefore\  No $A$s are $C$s.
\item\textbf{Dabitis:} All $B$s are $C$s. Some $A$ is $B$.
	\therefore\  Some $C$ is $A$.
\item\textbf{Darii:} All $B$s are $C$s. Some $A$ is $B$.
	\therefore\  Some $A$ is $C$.
\item\textbf{Disamis:} Some $B$ is $C$. All $B$s are $A$s.
	\therefore\  Some $A$ is $C$.
\item\textbf{Ferison:} No $B$s are $C$s. Some $B$ is $A$.
	\therefore\  Some $A$ is not $C$.
\item\textbf{Festino:} No $C$s are $B$s. Some $A$ is $B$.
	\therefore\  Some $A$ is not $C$.
\item\textbf{Frisesomorum:} Some $B$ is $C$. No $A$s are $B$s.
	\therefore\  Some $C$ is not $A$.
\end{enumerate}


\problempart
\label{pr.BarbaraEtc.proof2}
Now prove the conditionally valid syllogisms using QL. Symbolize each of the following and add the additional assumptions ``There is an $A$'' and ``There is a $B$.'' Then prove that the supplemented arguments forms are valid in QL. Calemos and Cesaro have been skipped because they are trivial variations of Camestros and Celaront. 

\begin{enumerate}[label=\arabic*), topsep=0pt, parsep=0pt, itemsep=3pt]

\item\textbf{Barbari:} All $B$s are $C$s. All $A$s are $B$s.
	\therefore\  Some $A$ is $C$.
\item\textbf{Celaront:} No $B$s are $C$s. All $A$s are $B$s.
	\therefore\  Some $A$ is not $C$.
\item\textbf{Camestros:} All $C$s are $B$s. No $A$s are $B$s.
	\therefore\  Some $A$ is not $C$.
\item\textbf{Darapti:} All $A$s are $B$s. All $A$s are $C$s.
	\therefore\  Some $B$ is $C$.
\item\textbf{Felapton:} No $B$s are $C$s. All $A$s are $B$s.
	\therefore\  Some $A$ is not $C$.
\item\textbf{Baralipton:} All $B$s are $C$s. All $A$s are $B$s.
	\therefore\  Some $C$ is $A$.
\item\textbf{Fapesmo:} All $B$s are $C$s. No $A$s are $B$s.
	\therefore\  Some $C$ is not $A$.
\end{enumerate}



\problempart
Provide a proof of each claim.
\begin{enumerate}[label=\arabic*), topsep=0pt, parsep=0pt, itemsep=3pt]
\item $\forall x \forall y Gxy \sststile{}{} \exists x Gxx$
\item $\forall x \forall y (Gxy \eif Gyx) \sststile{}{} \forall x\forall y (Gxy \eiff Gyx)$
\item $\{\forall x(Ax\eif Bx), \exists x Ax\} \sststile{}{} \exists x Bx$
\item $\{Na \eif \forall x(Mx \eiff Ma), Ma, \enot Mb\}\sststile{}{} \enot Na$
\item $\sststile{}{}\forall z (Pz \eor \enot Pz)$
\item $\sststile{}{}\forall x Rxx\eif \exists x \exists y Rxy$
\item $\sststile{}{}\forall y \exists x (Qy \eif Qx)$
\end{enumerate}

\problempart
Show that each pair of sentences is provably equivalent.
\begin{enumerate}[label=\arabic*), topsep=0pt, parsep=0pt, itemsep=3pt]
\item $\forall x (Ax\eif \enot Bx) \nsststile{}{} \hspace{.5em} \sststile{}{}\enot\exists x(Ax \eand Bx)$
\item $\forall x (\enot Ax\eif Bd) \nsststile{}{} \hspace{.5em} \sststile{}{} \forall x Ax \eor Bd$
\item $\exists x Px \eif Qc \nsststile{}{} \hspace{.5em} \sststile{}{}\forall x (Px \eif Qc)$
%\item $Rca \eiff \forall x Rxa$, $\forall x(Rca \eiff Rxa)$  rob: I'm embarassed to say I can't solve this one. 
\end{enumerate}



\problempart
Show that each of the following is provably inconsistent.
\begin{enumerate}[label=\arabic*), topsep=0pt, parsep=0pt, itemsep=3pt]
\item \{$Sa\eif Tm$, $Tm \eif Sa$, $Tm \eand \enot Sa$\}
\item \{$\enot\exists x \exists y Lxy$, $Laa$\}
\item \{$\forall x(Px \eif Qx)$, $\forall z(Pz \eif Rz)$, $\forall y Py$, $\enot Qa \eand \enot Rb$\}
\end{enumerate}




\problempart
\label{pr.likes}
Write a symbolization key for the following argument, translate it, and prove it:
\begin{quote}
There is someone who likes everyone who likes everyone that he likes. Therefore, there is someone who likes himself.
\end{quote}



%\problempart
%Look back at Part \ref{pr.QLarguments} on p.~\pageref{pr.QLarguments}. For each argument: If it is valid in QL, give a proof. If it is invalid, construct a model to show that it is invalid.


\problempart
\label{pr.QLequivornot}
For each of the following pairs of sentences: If they are logically equivalent in QL, give proofs to show this. If they are not, construct a model to show this.
\begin{enumerate}[label=\arabic*), topsep=0pt, parsep=0pt, itemsep=3pt]
\item $\forall x Px \eif Qc \nsststile{}{} \hspace{.5em} \sststile{}{}\forall x (Px \eif Qc)$
\item $\forall x Px \eand Qc\nsststile{}{} \hspace{.5em} \sststile{}{}\forall x (Px \eand Qc)$
\item $Qc \eor \exists x Qx\nsststile{}{} \hspace{.5em} \sststile{}{}\exists x (Qc \eor Qx)$
\item $\forall x\forall y \forall z Bxyz\nsststile{}{} \hspace{.5em} \sststile{}{}\forall x Bxxx$
\item $\forall x\forall y Dxy\nsststile{}{} \hspace{.5em} \sststile{}{}\forall y\forall x Dxy$
\item $\exists x\forall y Dxy\nsststile{}{} \hspace{.5em} \sststile{}{}\forall y\exists x Dxy$
\end{enumerate}

\problempart
\label{pr.QLvalidornot}
For each of the following arguments: If it is valid in QL, give a proof. If it is invalid, construct a model to show that it is invalid.
\begin{enumerate}[label=\arabic*), topsep=0pt, parsep=0pt, itemsep=3pt]
\item $\forall x\exists y Rxy \sststile{}{} \exists y\forall x Rxy$
\item $\exists y\forall x Rxy \sststile{}{} \forall x\exists y Rxy$
\item $\exists x(Px \eand \enot Qx) \sststile{}{} \forall x(Px \eif \enot Qx)$
\item $\{\forall x(Sx \eif Ta)$, $Sd\} \sststile{}{}Ta$
\item $\{\forall x(Ax\eif Bx)$, $\forall x(Bx \eif Cx)\} \sststile{}{} \forall x(Ax \eif Cx)$
\item $\{\exists x(Dx \eor Ex)$, $\forall x(Dx \eif Fx)\} \sststile{}{} \exists x(Dx \eand Fx)$
\item $\forall x\forall y(Rxy \eor Ryx)\sststile{}{} Rjj$
\item $\exists x\exists y(Rxy \eor Ryx)\sststile{}{}Rjj$
\item $\{\forall x Px \eif \forall x Qx$, $\exists x \enot Px\}\sststile{}{}\exists x \enot Qx$
\item $\{\exists x Mx \eif \exists x Nx$, $\enot \exists x Nx\}\sststile{}{} \forall x \enot Mx$
\end{enumerate}


%%%%%%%%%%%%%%%%%%%%%%%%%%%%%%%%                         6.2 Rules for Identity %%%%%%%%%%%%%%%%%%%%%%%%%%%%%%%%

\section{Rules for Identity}
The identity predicate is not part of QL, but we add it when we need to symbolize certain sentences. For proofs involving identity, we add two rules of proof.

Suppose you know that many things that are true of $a$ are also true of $b$. For example: $Aa\eand Ab$, $Ba\eand Bb$, $\enot Ca\eand\enot Cb$, $Da\eand Db$, $\enot Ea\eand\enot Eb$, and so on. This would not be enough to justify the conclusion $a=b$. (See p.~\pageref{model.nonidentity}.) In general, there are no sentences that do not already contain the identity predicate that could justify the conclusion $a=b$. This means that the identity introduction rule will not justify $a=b$ or any other identity claim containing two different constants.

However, it is always true that $a=a$. In general, no premises are required in order to conclude that something is identical to itself. So this will be the identity introduction rule, abbreviated {=}I:

\begin{proof}
	\have[\ \,\,\,]{x}{\script{c}=\script{c}} \by{=I}{}
\end{proof}

Notice that the {=}I rule does not require referring to any prior lines of the proof. For any constant \script{c}, you can write $\script{c}=\script{c}$ on any point with only the {=}I rule as justification.

If you have shown that $a=b$, then anything that is true of $a$ must also be true of $b$. For any sentence with $a$ in it, you can replace some or all of the occurrences of $a$ with $b$ and produce an equivalent sentence. For example, if you already know $Raa$, then you are justified in concluding $Rab$, $Rba$, $Rbb$. Recall that $\script{A}[\script{a}||\script{b}]$ is the sentence produced by replacing \script{a} in \script{A} with \script{b}. This is not the same as a substitution instance, because \script{b} may replace some or all occurrences of \script{a}. The identity elimination rule ({=}E) justifies replacing terms with other terms that are identical to it:
\begin{proof}
	\have[m]{e}{\script{a}=\script{b}}
	\have[n]{a}{\script{A}}
	\have[\ ]{ea1}{\script{A}[\script{a}||\script{b}]} \by{=E}{e,a}
	\have[\ ]{ea2}{\script{A}[\script{b}||\script{a}]} \by{=E}{e,a}
\end{proof}

%The basic rules for conjunction can be valuable in a proof even if there are no conjunctions in any of the assumptions; the basic rules for disjunction can be used even if there are no disjunctions in any assumptions; and similarly for the other basic rules. The rules for identity are different, in that there must be an identity claim in some assumption in order for the rules to do any work. Other than the trivial identity that we can introduce with the {=}I rule


%do not apply we can now prove that identity is \emph{transitive}: If $a=b$ and $b=c$, then $a=c$. The proof proceeds in this way:
%\begin{proof}
%	\open
%		\hypo{p}{a=b \eand b=c}\by{want $a=c$}{}
%		\have{ab}{a=b}\ae{p}
%		\have{bc}{b=c}\ae{p}
%		\have{ac}{a=c}\by{{=}E}{ab,bc}
%	\close
%	\have{conc}{(a=b \eand b=c)\eif a=c} \ci{p-ac}
%\end{proof}


%As an example, consider this argument:
%\begin{quote}
%There is only one button in my pocket. There is a blue button in my pocket. Therefore, there is no button in my pocket that is not blue.
%\end{quote}
%We begin by defining a symbolization key:
%\begin{ekey}
%\item{UD:} buttons in my pocket
%\item{Bx:} $x$ is blue.
%\end{ekey}
%\begin{proof}
%	\hypo{one}{\forall x\forall y\ x=y}
%	\hypo{eb}{\exists x Bx} \by{want $\enot\exists x \enot Bx$}{}
%	\open
%		\hypo{be1}{Be}
%		\have{ef1}{e=f}\Ae{one}
%		\have{bf1}{Bf}\by{{=}E}{ef1,be1}
%	\close
%	\have{bf}{Bf}\Ee{eb,be1-bf1}
%	\have{ab}{\forall x Bx}\Ai{bf}
%	\have{nnab}{\enot\enot\forall x Bx}\by{DN}{ab}
%	\have{nenb}{\enot\exists x\enot Bx}\by{QN}{nnab}
%\end{proof}

To see the rules in action, consider this proof:
\begin{proof}
	\hypo{one}{\forall x\forall y\ x=y}
	\hypo{eb}{\exists x Bx}
	\hypo{Abnc}{\forall x(Bx \eif \enot Cx)}
		\by{want $\enot\exists x Cx$}{}
	\open
		\hypo{be1}{Be}
		\have{ef1}{\forall y\ e=y}\Ae{one}
		\have{ef2}{e=f}\Ae{ef1}
		\have{bf1}{Bf}\by{{=}E}{ef2,be1}
		\have{bnc1}{Bf\eif\enot Cf}\Ae{Abnc}
		\have{ncf1}{\enot Cf}\ce{bnc1,bf1}
	\close
	\have{cf}{\enot Cf}\Ee{eb,be1-ncf1}
	\have{Anc}{\forall x \enot Cx}\Ai{cf}
	\have{nEc}{\enot\exists x Cx}\by{QN}{Anc}
\end{proof}

%\section*{Summary of definitions}
%\begin{itemize}
%\item A sentence \script{A} is a \define{theorem} if and only if $\vdash\script{A}$.
%
%\item Two sentences \script{A} and \script{B} are \define{provably equivalent} if and only if $\script{A}\vdash\script{B}$ and $\script{B}\vdash\script{A}$.
%
%\item $\{\script{A}_1,\script{A}_2,\ldots\}$ is \define{provably inconsistent} if and only if, for some sentence \script{B}, $\{\script{A}_1,\script{A}_2,\ldots\}\vdash(\script{B} \eand \enot \script{B})$.
%\end{itemize}

\practiceproblems


\problempart
\label{pr.identity}
Provide a proof of each claim.
\begin{enumerate}[label=\arabic*), topsep=0pt, parsep=0pt, itemsep=3pt] 
\item $\{Pa \eor Qb, Qb \eif b=c, \enot Pa\}\sststile{}{} Qc$
\item $\{m=n \eor n=o, An\}\sststile{}{} Am \eor Ao$
\item $\{\forall x x=m, Rma\}\sststile{}{} \exists x Rxx$
\item $\enot \exists x x \neq m \sststile{}{} \forall x\forall y (Px \eif Py)$
\item $\forall x\forall y(Rxy \eif x=y)\sststile{}{} Rab \eif Rba$
\item $\{\exists x Jx, \exists x \enot Jx\}\sststile{}{} \exists x \exists y x\neq y$
%\item $\{\forall x(x=n \eiff Mx), \forall x(Ox \eand \eor Mx)\}\sststile{}{} On$
\item $\{\exists x Dx, \forall x(x=p \eiff Dx)\}\sststile{}{} Dp$
\item $\{\exists x [(Kx \eand Bx) \eand \forall y(Ky \eif x=y)], Kd\}\sststile{}{} Bd$
\item $\sststile{}{} Pa \eif \forall x(Px \eor x \neq a)$
\end{enumerate}









