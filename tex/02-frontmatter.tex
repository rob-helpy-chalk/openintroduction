\thispagestyle{empty}%
%\clearpage\setcounter{page}{1}
%\cftpagenumbersoff{chapter}
This is version 0.1 of An Open Introduction to Logic. It is current as of \today. %Complete version information is available at \textbookhomepage.

\vfill

{\copyright\ \ifthenelse{\year=2005}{\number\year}{2005--\number\year} by Cathal Woods, P.D. Magnus, and J. Robert Loftis. Some rights reserved. }

\begin{adjustwidth}{2em}{0em}
{\footnotesize Licensed under a Creative Commons license.\\
	(Attribution-NonComercial-ShareAlike 4.0 International )
	\url{https://creativecommons.org/licenses/by-nc-sa/4.0/}


\includegraphics[width=66pt, height=23pt, keepaspectratio=true]{img/cc-by-nc-sa.png}

}

\end{adjustwidth}

\vfill

This book incorporates material from \emph{An Introduction to Reasoning} by Cathal Woods, available at \url{sites.google.com/site/anintroductiontoreasoning/}
and \emph{For All X} by P.D. Magnus (version 1.27 [090604]), available at \url{www.fecundity.com/logic}.


\textit{Introduction to Reasoning} \copyright\ 2007--2014 by Cathal Woods. Some rights reserved.

\begin{adjustwidth}{2em}{0em}
{\footnotesize Licensed under a Creative Commons license: Attribution-NonCommercial-ShareAlike 3.0 Unported. \url{http://creativecommons.org/licenses/by-nc-sa/3.0/}}
\end{adjustwidth}

\textit{For All} X \copyright\  2005--2010 by P.D. Magnus. Some rights reserved.

\begin{adjustwidth}{2em}{0em}
{\footnotesize Licensed under a Creative Commons license: Attribution ShareAlike \url{http://creativecommons.org/licenses/by-sa/3.0/}}
\end{adjustwidth}

\vfill

J. Robert Loftis compiled this edition and wrote original material for it. He takes full responsibility for any mistakes remaining in this version of the text.


\vfill

Typesetting was carried out entirely in \LaTeX$2\varepsilon$. The style for typesetting proofs is based on fitch.sty (v0.4) by Peter Selinger, University of Ottawa.




\iflabelexists{part:formal_logic}{
\pagebreak

``When you come to any passage you don't understand, \emph{read it again}: if you \emph{still} don't understand it, \emph{read it again}: if you fail, even after \emph{three} readings, very likely your brain is getting a little tired. In that case, put the book away, and take to other occupations, and next day, when you come to it fresh, you will very likely find that it is \emph{quite} easy.'' \\
-- Charles Dodgson (Lewis Carroll) \emph{Symbolic Logic} \parencite*{Dodgson1896}
}{}


