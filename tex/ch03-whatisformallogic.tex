\chapter{What is Formal Logic?}
\label{chap:whatisformallogic}
\markright{Ch. \ref{chap:whatisformallogic}: What is Formal Logic?}
\setlength{\parindent}{1em}

% **************************************************
% *	3.1 Formal as in  Concerned with the Form of Things             *
% **************************************************

\section{Formal as in Concerned with the Form of Things}


The chapters in 
\iflabelexists{part:formal_logic}{Part \ref{part:formal_logic}} %This prints ``Part $N$ if there is a single section for all of formal logic
{Parts \iflabelexists{part:cat_logic}{\ref{part:cat_logic} and \ref{part:sent_logic}} %this prints ``Parts $N$ and $M'$' if there is a section for cat logic, where $N$ and $M$ are the part numbers for cat and sent logic.
{\ref{part:sent_logic} and \ref{part:quant_logic}}} %this prints ``Parts $N$ and $M'$' if there is not a section for cat logic, where $N$ and $M$ are the part numbers for sent and quant logic.
deal with formal logic. Formal logic is distinguished from other branches of logic by the way it achieves content neutrality. Back on page \pageref{def:content_neutrality}, we said that a distinctive feature of logic is that it is neutral about the content of the argument it evaluates. If a kind of argument is strong---say, a kind of statistical argument---it will be strong whether it is applied to sports, politics, science or whatever. Formal logic takes radical measures to ensure content neutrality: it removes the parts of a statement that tie it to particular objects in the world and replaces them with abstract symbols. 

Consider the two arguments from Figure \ref{fig:valid_sound} again:
\begin{multicols}{2}
\begin{earg*}
\item Socrates is a person.
\item All persons are mortal.
\itemc Socrates is mortal.
\end{earg*}

\begin{earg*}
\item Socrates is a person.
\item All people are carrots.
\itemc Socrates is a carrot.
\end{earg*}

\end{multicols}

These arguments are both valid. In each case, if the premises were true, the conclusion would have to be true. (In the case of the first argument, the premises are actually true, so the argument is sound, but that is not what we are concerned with right now.) What makes these arguments valid is that they are put together the right way. Another way of thinking about this is to say that they have the same logical form. Both arguments can be written like this:

\begin{earg*}
\item $S$ is $M$.
\item All $M$ are $P$.
\itemc[.2] $S$ is $P$.
\end{earg*}

In both arguments $S$ stands for Socrates and $M$ stands for person. In the first argument, $P$ stands for mortal; in the second, $P$ stands for carrot. \iflabelexists{chap:catstatements}{(The reason we chose these letters will become clear in Chapters \ref{chap:catstatements} and \ref{chap:cat_syllogisms}.)}{} The letters `S', `M', and `P' are variables. They are just like the variables you may have learned about in algebra class. In algebra, you had equations like $y = 2x + 3$, where $x$ and $y$ were variables that could stand for any number. Just as $x$ could stand for any number in algebra, `S' can stand for any name in logic. In fact, this is one of the original uses of variables. Long before variables were used to stand for numbers in algebra, they were used to stand for classes of things, like people or carrots, by Aristotle in his book the \cite*{Aristotle:prior}. At about the same time, over in India, the ancient grammarian and linguist P\={a}\d{n}ini was also using variables to represent possible sounds that could be used in different forms of a word \citep{Panini2015}. Both thinkers introduce their variables fairly causally, as if their readers would be familiar with the idea, so it may be that people prior to them actually invented the variable.

Whoever invented it, the variable was one of the most important conceptual innovations in human history, right up there with the invention of the zero, or alphabetic writing. The importance of the variable for the history of mathematics is obvious. But it was also incredibly important in one of its original fields of application, logic. For one thing, it allows logicians to be more content neutral. We can set aside any associations we have with people, or carrots, or whatever, when we are analyzing an argument. More importantly, once we set aside content in this way, we discover that something incredibly powerful is left over, the logical structure of the sentence itself. This is what we investigate when we study formal logic. In the case of the two arguments above, identifying the logical structure of statements reveals not only that the two arguments have the same logical form, but they have an impeccable logical form. Both arguments are valid, and any other arguments that have this form will be valid. 

When Aristotle introduced the variable to the study of logic he used it the way we did in the argument above. His variables stood for names and categories in simple two-premise arguments called syllogisms. The system of logic Aristotle outlined became the dominant logic in the Western world for more than two millennia. It was studied and elaborated on by philosophers and logicians from Baghdad to Paris. The thinkers that carried on Aristotelian tradition were divided by language and religion. They were pagans, Muslims, Jews, and Christians writing typically in Greek, Latin or Arabic. But they were all united by the sense that the tools Aristotle had given them allowed them to see something profound about the nature of reality. They were looking at abstract structures which somehow seemed to be at the foundation of things. As the philosopher and historian of logic Catarina Dutilh Novaes points out, the logic that the thinkers of all these religious traditions were pursuing was formal in that it concerned the \textit{forms} of things \citep{DutilhNovaes2011}. As formal logic evolved, however, the idea of being ``formal'' would take on an additional meaning. 

\vfill

% *********************************************************
% *				3.2 Formal meaning strictly following rules              *
% ********************************************************


\section{Formal as in Strictly Following Rules}


\newglossaryentry{artificial language}
{
name=artificial language,
description={A language that was consciously developed by identifiable individuals for some purpose.}
}

\newglossaryentry{natural language}
{
name=natural language,
description={A language that develops spontaneously and learned by infants as their first language.}
}

\newglossaryentry{formal language}
{
name=formal language,
description={An artificial language designed to bring out the logical structure of  ideas and remove all the ambiguity and vagueness that plague natural languages like English. Sometimes, formal languages are also said to be languages that can be implemented by a machine.}
}

Despite its historical importance, Aristotelean logic has largely been superseded. Starting in the 19th century people learned to do more than simply replace categories with variables. They learned to replicate the whole structure of sentences with a formal system that brought out all sorts of features of the logical form of arguments. The result was the creation of entire artificial languages. An \textsc{\gls{artificial language}} \label{def:artificial_language} is a language that was consciously developed by identifiable individuals for some purpose. Esperanto, for instance, is an artificial language developed by Ludwig Lazarus Zamenhof in the 19th century with the hope of promoting world peace by creating a common language for all. J.R.R. Tolkien invented several languages to flesh out the fictional world of his fantasy novels, and even created timelines for their evolution. For Tolkien, the creation of languages was an art form in itself, ``An art for which life is not long enough, indeed: the construction of imaginary languages in full or outline for amusement, for the pleasure of the constructor or even conceivably of any critic that might occur'' (\cite*{Tolkien1931}). And it is an art that is really beginning to catch on, especially with Hollywood commissioning languages to be constructed for blockbuster films. 

Artificial languages contrast with \textsc{\glspl{natural language}}, \label{def:natural_language} which develop spontaneously and are learned by infants as their first language. Natural languages include all the well-known languages spoken around the world, like English or Japanese or Arabic. It also includes more recently developed languages and evolved spontaneously amongst groups of people. For instance, whenever you put deaf children together, for instance in a boarding school, they will spontaneously develop their own sign language. This phenomenon was important for the development of American Sign Language (ASL) and is part of why ASL counts as a \textit{natural} language. For similar reasons Nicaraguan Sign Language counts as a natural language, even though it emerged very recently---in the late 1970s and 80s, when the new Sandinista government set up schools for the deaf for the first time. Natural languages can also develop by creolization, when languages merge and children grow up speaking the merged language as their first language. Haitian Creole is the most famous example of this.   

The languages developed by logicians are artificial, not natural. Their goal is not to promote global harmony, like Zamenhof's Esperanto. Nor are they creating art for art's sake, as Tolkein was, although logical languages can have a great deal of beauty. When the languages first started being developed in the late 19th and early 20th centuries, the goal was, in fact, to have a logically pure language, free of the irrationalities the plague natural languages. More specifically, they had two distinct goals: first, remove all ambiguity and vagueness, and second, to make the logical structure of the language immediately apparent, so that the language wore its logical structure on its face, as it were. If such a language could be developed, it would help us solve all kinds of problems. The logician and philosopher Rudolf Carnap, for instance, felt that the right artificial language could simply make philosophical problems disappear \citep{Carnap1928}.

The languages developed by logicians in the late 19th and early 20th centuries got labeled formal languages, in part because the logicians in question were working in the tradition of formal logic that was already established. A shift began to happen here with the meaning of formal, however, a change which is well documented by Dutilh Novaes  (\cite*{DutilhNovaes2011}). Logicians began to hope that the languages that were being developed were so logical that everything about them could be characterized by a machine. A machine could be used to create sentences in this language, and then again to identify all the valid arguments in this language. This brings out another sense of the word ``formal.'' As Dutilh Novaes puts it (\cite*{DutilhNovaes2011}) instead of being ``formal'' in the sense of concerning the forms of things, logic was formal in the sense that it followed rules perfectly precisely. You might compare this to the way a ``formal hearing'' in a court of law follows the rule of law to the letter. 

For the purposes of this textbook, we will say that the core idea of a  \textsc{\gls{formal language}} \label{def:formal_language} is that it is an artificial language designed to bring out the logical structure of ideas and remove all the ambiguity and vagueness that plague natural languages like English. We will further add that sometimes, formal languages are languages that can be implemented by a machine. Creating formal languages always involves all kinds of trade offs. On the one hand, we are trying to create a language that makes a logical structure clear and obvious. This will require simplifying things, removing excess baggage from the language. On the other hand, we want to make the language perfectly precise, free of vagueness and ambiguity. This will mean adding complexity to the language. The other thing was that it was very important for the people developing these languages that you be able to prove the all the truths of mathematics in them. This meant that the languages had to have a certain scope.

This was a trade off no logician was ever able to get perfectly correct, because, as it turns out, a logically pure language is impossible. No formal language can do everything that a natural language can do. Logicians became convinced of this, naturally enough, because of a pair of logical proofs. In 1931, the logician Kurt G\"{o}del showed that you couldn't do all of mathematics in a consistent logical system, which was enough to persuade most of the logicians engaged in the project to drop it. There is a more general problem with the idea of a purely logical language, though, which is that that many of the features logicians were trying to remove from language were actually necessary to make it function. Arika Okrent puts the point quite well. For Okrent, the failure of artificial languages is precisely what illuminates the virtues of natural language. 

\begin{quotation}\noindent [By studying artificial languages we] gain a deeper appreciation of natural language and the messy qualities that give it so much flexibility and power and that a simple communication device. The ambiguity and lack of precision allow to serve as a instrument of thought \textit{formation}, of experimentation and discovery. We don't know exactly what we mean before we speak; we can figure it out as we go along,. We can talk just to talk, to be social, to feel connected, to participate. At the same time natural language still works as an instrument of thought transmission, one that can be \textit{made} extremely precise and reliable when we need it to be, or left loose and sloppy when we can't spare the time or effort \citep{Okrent2009} \end{quotation} 

The languages developed in the late 19th and early 20th centuries had goals that were theoretical, rather than practical. They languages were meant to improve our understanding of the world for the sake of improving our understanding of the world. They failed at this theoretical goal, but they wound up having a practical spin-off of world-historical proportions, which is why formal logic is a thriving discipline to this day. Remember that in this period people started thinking of formal languages as languages that could be implemented mechanically. At first, the idea of a a mechanistic language was a metaphor. The rules that were being followed to the letter were to be followed by a human being actually writing down symbols. This human being was generally referred to as a ``computer,'' because they were computing things. The world changed when a logician named Alan Turing started using literal machines to be computers.

In the 1930s, Turing developed the idea of a reasoning machine that could compute any function. At first, this was just an abstract idea: it involved an infinite stretch of tape. But during World War II, Turing went to work the British code breaking effort at Bletchley Park. The Nazis encoded messages using a device called the Enigma Machine. The Allies had captured one, but since they settings on the machine were reshuffled for each message, it didn't do them much good. Turing, together with people like the mathematicians Gordon Welchman and Joan Clarke, managed to build another machine that could test Enigma settings rapidly to identify the configuration being used. People had made computing machines before, but now the science of logic was so much more advanced that they real power of mechanical computing could be exploited. The human computers became the fully programmable machines we know today, and the formal languages logicians created for theoretical reasons came the computer languages the world of the 21st century depends on. (All of this information, plus lots of fascinating pictures and diagrams, is available at www.turing.org.uk.)



%This version is for the complete text, where all formal sections are covered in a unified Part II.
\iflabelexists{part:formal_logic}{Part \ref{part:formal_logic} of this book begins by exploring the world of Aristotelian logic, where logic is ``formal'' in the sense of being about the forms of things. Chapter \ref{chap:catstatements} looks at the logical structure of the individual statements studied by the Aristotelian tradition. Chapter \ref{chap:cat_syllogisms} then builds these into valid arguments. After we study Aristotelian logic, we will develop two formal languages, called SL and QL.  Chapters \ref{chap:SL} through \ref{chap:proofsinSL} develop SL. In SL, the smallest units are individual statements. Simple statements are represented as letters and connected with {logical connectives} like \emph{and} and \emph{not} to make more complex statements. Chapters \ref{chap:QL} through \ref{chap:proofsinQL} develop QL. In QL, the basic units are objects, properties of objects, and relations between objects.}{
%% This version of the paragraph is for texts that just do cat and sent.
\iflabelexists{part:cat_logic}{Part \ref{part:cat_logic} of this book explores the world of Aristotelian logic. Chapter \ref{chap:catstatements} looks at the logical structure of the individual statements studied by the Aristotelian tradition. Chapter \ref{chap:cat_syllogisms} then builds these into valid arguments. Part \ref{part:sent_logic} develops a full-blown formal language, called Sentential logic, or SL. In SL Simple statements are represented as letters and connected with logical connectives like \emph{and} and \emph{not} to make more complex statements.}{
%This version of the paragraph is for texts that just do sent and quant
In this book we will be developing two formal languages, called SL and QL. Part \ref{part:sent_logic} develops SL.In SL, the smallest units are individual statements. Simple statements are represented as letters and connected with logical connectives like \emph{and} and \emph{not} to make more complex statements. Part \ref{part:quant_logic} develops QL. In QL, the basic units are objects, properties of objects, and relations between objects.} }




%\iflabelexists{part:formal_logic}{ and QL.  Chapters \ref{chap:SL} through \ref{chap:proofsinSL} develop SL. In SL, the smallest units are individual statements. Several chapters in the complete version of this text \label{ver_var}\nix{Chapters \ref{chap:QL} through \ref{chap:proofsinQL}} develop QL. In QL, the basic units are objects, properties of objects, and relations between objects.


%% **********************************************
%% *			On Learning a Formal Logical System        *
%% **********************************************
%\section{On Learning a Formal Logical System}
%\label{sec:On_learning_a_formal_logical_system}
%
%You may be reading this book because you have a keen interest in logic and are excited to learn more about it. You may also be reading this book because it was assigned in a class that you need to fulfill a distribution requirement. As the chapters on formal logic roll on, and the pages begin to fill up with unfamiliar squiggles, you may even begin to question whether the study of logic is for you. Rest assured, if you have a human brain capable of reading this sentence, you are also capable of doing formal logic---and you can benefit from doing so, too. In this section, we are going to talk about why you can be confident in your ability to do logic, even if you are new to it. We are also going to offer some strategies for studying formal logic, so even if you are already quite confident in your abilities, it will be worth reading the rest of this section. 
%
%All of the basic mental skills used in a formal logical system are just that: basic mental skills. They are things you do whenever you use language. A basic part of formal logic is using abstract symbols to refer to a group of things that aren't specified. So earlier we used ``$P$'' in place of the words ``mortal'' and ``carrot'' and a whole bunch of other words that might occupy that spot in an argument. This is the same thing you do when you use a word like ``dog'' to refer to Spot and Fido and a whole bunch of other dogs that you don't know about. We are also going to spend time transforming things in one logical form into another. Again, this is something you already do when you speak. You know that ``Jane gave the ball to Sally'' can be changes to ``Sally was given the ball by Jane'' without changing its meaning. The kinds of things we are doing in this text are no different. 
%


% **********************************************
% *			More Logical Notions for Formal Logic      *
% **********************************************
\section{More Logical Notions for Formal Logic}
\label{sec:other_logical_notions}
\setlength{\parindent}{1em}

Part \ref{part:basic_concepts} covered the basic concepts you need to study any kind of logic. When we study formal logic, we will be interested in some additional logical concepts, which we will explain here. 

%1.5.1 Truth values

\subsection{Truth values}

\newglossaryentry{truth value}
{
  name=truth value,
  description={The status of a statement with relationship to truth. For this textbook, this means the status of a statement as true or false.}
}

\newglossaryentry{bivalent}
{
  name=bivalent,
  description={A property of logical systems which is present when the system only has two truth values, generally ``true'' and ``false.''}
}



A truth value is the status of a statement as true or false. Thus the truth value of the sentence ``All dogs are mammals'' is ``True,'' while the truth value of ``All dogs are reptiles'' is false. More precisely, a \textsc{\gls{truth value}} \label{def:Truth_value} is the status of a statement with relationship to truth. We have to say this, because there are systems of logic that allow for truth values besides ``true'' and ``false,'' like ``maybe true,'' or ``approximately true,'' or ``kinda sorta true.'' For instance, some philosophers have claimed that the future is not yet determined. If they are right, then statements about \emph{what will be the case} are not yet true or false. Some systems of logic accommodate this by having an additional truth value. Other formal languages, so-called paraconsistent logics, allow for statements that are both true \emph{and} false. We won't be dealing with those in this textbook, however. For our purposes, there are two truth values, ``true'' and ``false,'' and every statement has exactly one of these. Logical systems like ours are called \textsc{\gls{bivalent}}. \label{def:Bivalent}







%1.5.2 Tautology, Contingent Statement, Contradiction

\subsection{Tautology, contingent statement, contradiction}

In considering arguments formally, we care about what would be true \emph{if} the premises were true. Generally, we are not concerned with the actual truth value of any particular statements--- whether they are \emph{actually} true or false. Yet there are some statements that must be true, just as a matter of logic.

Consider these statements:
\begin{enumerate}[label=(\alph*)]
\item \label{itm:ex_contingent} It is raining.
\item \label{itm:ex_tautology} Either it is raining, or it is not.
\item \label{itm:ex_contradiction} It is both raining and not raining.
\end{enumerate}
In order to know if statement \ref{itm:ex_contingent} is true, you would need to look outside or check the weather channel. Logically speaking, it might be either true or false. Statements like this are called \emph{contingent} statements.


\newglossaryentry{tautology}
{
name=tautology,
description={A statement that must be true, as a matter of logic.}
}

Statement \ref{itm:ex_tautology} is different. You do not need to look outside to know that it is true. Regardless of what the weather is like, it is either raining or not. If it is drizzling, you might describe it as partly raining or in a way raining and a way not raining. However, our assumption of bivalence means that we have to draw a line, and say at some point that it is raining. And if we have not crossed this line, it is not raining. Thus the statement ``either it is raining or it is not'' is always going to be true, no matter what is going on outside. A statement that has to be true, as a matter of logic is called a \textsc{\gls{tautology}} \label{def:tautology} or logical truth. 

\newglossaryentry{contradiction}
{
name=contradiction,
description={A statement that must be false, as a matter of logic.}
}

You do not need to check the weather to know about statement \ref{itm:ex_contradiction}, either. It must be false, simply as a matter of logic. It might be raining here and not raining across town, it might be raining now but stop raining even as you read this, but it is impossible for it to be both raining and not raining here at this moment. The third statement is \emph{logically false}; it is false regardless of what the world is like. A logically false statement is called a \textsc{\gls{contradiction}}. \label{def:contradiction}

\newglossaryentry{contingent statement}
{
name=contingent statement,
description={A statement that is neither a tautology nor a contradiction.}
}

We have already said that a contingent statement is one that could be true, or could be false, as far as logic is concerned. To be more precise, we should define a \textsc{\gls{contingent statement}}  \label{def:contingent_statement} as a statement that is neither a tautology nor a contradiction. This allows us to avoid worrying about what it means for something to be logically possible. We can just piggyback on the idea of being logically necessary or logically impossible. 

A statement might \emph{always} be true and still be contingent. For instance, it may be the case that in no time in the history of the universe was there ever an elephant with tiger stripes. Elephants only ever evolved on Earth, and there was never any reason for them to evolve tiger stripes. The statement ``Some elephants have tiger stripes,'' is therefore always false. It is, however, still a contingent statement. The fact that it is always false is not a matter of logic. There is no contradiction in considering a possible world in which elephants evolved tiger stripes, perhaps to hide in really tall grass. The important question is whether the statement \emph{must} be true, just on account of logic.

When you combine the idea of tautologies and contradictions with the notion of deductive validity, as we have defined it, you get some curious results. For one thing, any argument with a tautology in the conclusion will be valid, even if the premises are not relevant to the conclusion. This argument, for instance, is valid.

\begin{earg*}
\item There is coffee in the coffee pot.
\item There is a dragon playing bassoon on the armoire.
\itemc All bachelors are unmarried men.
\end{earg*}

The statement ``All bachelors are unmarried men'' is a tautology. No matter what happens in the world, all bachelors have to be unmarried men, because that is how the word ``bachelor'' is defined. But if the conclusion of the argument is a tautology, then there is no way that the premises could be true and the conclusion false. So the argument must be valid.

Even though it is valid, something seems really wrong with the argument above. The premises are not relevant to the conclusion. Each sentence is about something completely different. This notion of relevance, however, is something that we don't have the ability to capture in the kind of simple logical systems we will be studying. The logical notion of validity we are using here will not capture everything we like about arguments.

Another curious result of our definition of validity is that any argument with a contradiction in the premises will also be valid. In our kind of logic, once you assert a contradiction, you can say anything you want. This is weird, because you wouldn't ordinarily say someone who starts out with contradictory premises is arguing well. Nevertheless, an argument with contradictory premises is valid.

%1.5.3 Logical equivalence. 

\subsection{Logically Equivalent and Contradictory Pairs of Sentences}

We can also ask about the logical relations \emph{between} two statements. For example:

\begin{enumerate}[label=(\alph*)]
\item John went to the store after he washed the dishes.
\item John washed the dishes before he went to the store.
\end{enumerate}

\newglossaryentry{logical equivalence}
{
name={logical equivalence},
text={logically equivalent},
description={A property held by a pair of sentences that must always have the same truth value.}
}

These two statements are both contingent, since John might not have gone to the store or washed dishes at all. Yet they must have the same truth value. If either of the statements is true, then they both are; if either of the statements is false, then they both are. When two statements necessarily have the same truth value, we say that they are \textsc{\gls{logical equivalence}}. \label{def:logical_equivalence}

\newglossaryentry{contradictories}
{
name=contradictories,
description={Two statements that must have opposite truth values, so that one must true and the other false.}
}

On the other hand, if two sentences must have opposite truth values, we say that they are \textsc{\gls{contradictories}}. \label{def:contradictory}Consider these two sentences 

\begin{enumerate}[label=(\alph*)]
\item Susan is taller than Monica.
\item Susan is shorter or the same height as Monica.
\end{enumerate}

One of these sentences must be true, and if one of the sentences is true, the other one is false. It is important to remember the difference between a single sentence that is a \emph{contradiction} and a pair of sentences that are \emph{contradictory}. A single sentence that is a contradiction is in conflict with itself, so it is never true. When a pair of sentences is contradictory, one must always be true and the other false.

%%%%%%%%%%%%%%  consistency

\subsection{Consistency}
Consider these two statements:

\begin{enumerate}[label=(\alph*)]
\item \label{itm:taller} My only brother is taller than I am.
\item \label{itm:shorter} My only brother is shorter than I am.
\end{enumerate}

Logic alone cannot tell us which, if either, of these statements is true. Yet we can say that \emph{if} the first statement \ref{itm:taller} is true, \emph{then} the second statement \ref{itm:shorter} must be false. And if \ref{itm:shorter}  is true, then \ref{itm:taller} must be false. It cannot be the case that both of these statements are true. It is possible, however that both statements can be false. My only brother could be the same height as I am. 

\newglossaryentry{inconsistency}
{
name=inconsistency,
text={inconsistent},
description={A property possessed by a set of sentences when they cannot all be true at the same time, but they may all be false at the same time.}
}

\newglossaryentry{consistency}
{
name=consistency,
text={consistent},
description={A property possessed by a set of sentences when they can all be true at the same time, but are not necessarily so.}
}

If a set of statements could not all be true at the same time, they are said to be \textsc{\gls{inconsistency}}. \label{def:inconsistency} Otherwise, they are \textsc{\gls{consistency}}. \label{def:consistency} 

We can ask about the consistency of any number of statements. For example, consider the following list of statements:

\label{MartianGiraffes}
\begin{enumerate}[label=(\alph*)]
\item \label{itm:at_least_four}There are at least four giraffes at the wild animal park.
\item \label{itm:exactly_seven} There are exactly seven gorillas at the wild animal park.
\item \label{itm:not_more_than_two} There are not more than two Martians at the wild animal park.
\item \label{itm:martians} Every giraffe at the wild animal park is a Martian.
\end{enumerate}

Statements \ref{itm:at_least_four} and \ref{itm:martians} together imply that there are at least four Martian giraffes at the park. This conflicts with \ref{itm:not_more_than_two}, which implies that there are no more than two Martian giraffes there. So the set of statements \ref{itm:at_least_four}--\ref{itm:martians} is inconsistent. Notice that the inconsistency has nothing at all to do with \ref{itm:exactly_seven}. Statement \ref{itm:exactly_seven} just happens to be part of an inconsistent set.

Sometimes, people will say that an inconsistent set of statements ``contains a contradiction.'' By this, they mean that it would be logically impossible for all of the statements to be true at once. A set can be inconsistent even when all of the statements in it are either contingent or tautologous. When a single statement is a contradiction, then that statement alone cannot be true.

%%%%%%%%%%%  Practice Problems %%%%%%%%%%%


\practiceproblems
\noindent \problempart \label{pr.EnglishTautology} Label the following tautology, contradiction, or contingent statement.

\begin{longtabu}{p{.1\linewidth}p{.9\linewidth}}
\textbf{Example}: & Caesar crossed the Rubicon. \\
\textbf{Answer}: & Contingent statement. \\
&The Rubicon is a river in Italy. When General Julius Caesar took his army across it, he was committing to a revolution against the Roman Republic. Since that time, ``crossing the Rubicon'' has been a expression referring to making an irreversible decision. This kind of decision certainly seems to be contingent. Caesar could have decided otherwise.\\
\end{longtabu}

\begin{exercises}
\item Someone once crossed the Rubicon. \answerblank{\underline{Contingent statement}}{\vspace{.25in}}

\item No one has ever crossed the Rubicon. \answerblank{\underline{Contingent  statement}}{\vspace{.25in}}

\item If Caesar crossed the Rubicon, then someone has. \answerblank{\underline{Tautology}}{\vspace{.25in}}

\item Even though Caesar crossed the Rubicon, no one has ever crossed the Rubicon. \answerblank{\underline{Contradiction}}{\vspace{.25in}}

\item If anyone has ever crossed the Rubicon, it was Caesar. \answerblank{\underline{Contingent statement}}{\vspace{.25in}}
\end{exercises}


\noindent \problempart Label the following tautology, contradiction, or contingent statement.
\begin{exercises}
\item Elephants dissolve in water. \answer{\underline{Contingent}}
\item Wood is a light, durable substance useful for building things. \answer{\underline{Contingent}}
\item If wood were a good building material, it would be useful for building things. \answer{\underline{Tautology}}
\item I live in a three story building that is two stories tall. \answer{\underline{Contradiction}}
\item If gerbils were mammals they would nurse their young. \answer{\underline{Tautology}}
\end{exercises}

\noindent \problempart Label the following logically equivalent, contradictory, or neither. 

\begin{longtabu}{p{.1\linewidth}p{.9\linewidth}}
\textbf{Example}: &  All students who study will pass the test. \\
& If Jeremy studies, he will pass the test. \\
\textbf{Answer}: & Neither. \\
&If the first statement is true, then the second statement has to be true, but the reverse is not the case. It might be that Jeremy will pass the test if he studies, but some other students are going to fail no matter what.\\
\end{longtabu}

 
\begin{exercises}
\item Elephants dissolve in water.	\\
	If you put an elephant in water, it will dissolve.
\answerblank{\\\underline{Logically equivalent}}{\vspace{.25in}}	

\item All mammals dissolve in water.\\		
	If you put an elephant in water, it will dissolve. 
\answerblank{\\ \underline{Neither}}{\vspace{.25in}}

\item Elephants are bigger than lions. \\                                                                                        
Elephants are smaller or the same size as lions.
\answerblank{\\ \underline{Contradictory}}{\vspace{.25in}}

\item The Eurasian elephant is an herbivore \\
All the Eurasian elephant sometimes eats meat
\answerblank{\\ \underline{Contradictory}}{\vspace{.25in}}

\item Elephants dissolve in water. 	\\	
	All mammals dissolve in water. 
\answerblank{\\ \underline{Neither}}{\vspace{.25in}}

\end{exercises}

\noindent \problempart Label the following logically equivalent, contradictory, or neither. 

\begin{exercises}
\item  Thelonious Monk played piano.	\\
John Coltrane played tenor sax. 
\answer{\\ \underline{Neither}}

\item  Thelonious Monk played gigs with John Coltrane.	\\
	John Coltrane played gigs with Thelonious Monk.
\answer{\\ \underline{Logically equivalent}}

\item  All professional piano players have big hands.	\\
	Piano player Bud Powell had big hands.
	\answer{\\ \underline{Neither}}

\item  Bud Powell suffered from severe mental illness.	 \\
	All piano players suffer from severe mental illness.
	\answer{\\ \underline{Neither}}

\item John Coltrane was deeply religious.	 \\
John Coltrane was moderately or not at all religious 
\answer{\\ \underline{Contradictory}}
\end{exercises}


\noindent \problempart Consider again the statements on p.\pageref{MartianGiraffes}: 
\begin{enumerate}[label=(\alph*)]
\item \label{itm:at_least_four}There are at least four giraffes at the wild animal park.
\item \label{itm:exactly_seven} There are exactly seven gorillas at the wild animal park.
\item \label{itm:not_more_than_two} There are not more than two Martians at the wild animal park.
\item \label{itm:martians} Every giraffe at the wild animal park is a Martian.
\end{enumerate}
Now mark each of the following sets of statements consistent or inconsistent.
\begin{longtabu}{p{.1\linewidth}p{.9\linewidth}}
\textbf{Example}: & Statements \ref{itm:at_least_four}, \ref{itm:not_more_than_two}, and \ref{itm:martians}\\
\textbf{Answer}: & Inconsistent. If there are at least four giraffes, and every one of them is Martian, there can't be no more than two Martians in the park.\\
\end{longtabu}



\begin{exercises}
\item Statements \ref{itm:exactly_seven}, \ref{itm:not_more_than_two}, and \ref{itm:martians} \answerblank{\underline{consistent}}{\vspace{.25in}}
\item Statements \ref{itm:at_least_four}, \ref{itm:exactly_seven}, \ref{itm:not_more_than_two}, and \ref{itm:martians} \answerblank{\underline{inconsistent}}{\vspace{.25in}}
\item Statements \ref{itm:at_least_four}, \ref{itm:exactly_seven}, and \ref{itm:martians}\answerblank{\underline{consistent}}{\vspace{.25in}}
\item Statements \ref{itm:at_least_four}, \ref{itm:exactly_seven}, and \ref{itm:not_more_than_two} \answerblank{\underline{consistent}}{\vspace{.25in}}
\end{exercises}

\noindent \problempart Consider the following set of statements.
\begin{enumerate}[label=(\alph*)]
\item \label{itm:allmortal} All people are mortal.
\item \label{itm:socperson} Socrates is a person.
\item \label{itm:socnotdie} Socrates will never die.
\item \label{itm:socmortal} Socrates is mortal.
\end{enumerate}
Which combinations of statements form consistent sets? Mark each “consistent” or “inconsistent.”
\begin{exercises}
\item Statements \ref{itm:allmortal}, \ref{itm:socperson}, and \ref{itm:socnotdie}  \answer{\underline{Inconsistent}}
\item Statements \ref{itm:socperson}, \ref{itm:socnotdie}, and \ref{itm:socmortal} \answer{\underline{Inconsistent}}
\item Statements \ref{itm:socperson} and \ref{itm:socnotdie} \answer{\underline{Consistent}}
\item Statements \ref{itm:allmortal} and \ref{itm:socmortal} \answer{\underline{Consistent}}
\item Statements \ref{itm:allmortal}, \ref{itm:socperson}, \ref{itm:socnotdie}, and \ref{itm:socmortal} \answer{\underline{Inconsistent}} 
\end{exercises}

\noindent \problempart \label{pr.EnglishCombinations} Which of the following is possible? If it is possible, give an example. If it is not possible, explain why.


\begin{longtabu}{p{.1\linewidth}p{.9\linewidth}}
\textbf{Example}: & A valid argument that has one false premise and one true premise.\\
\textbf{Answer}: & Possible: Example: If Taylor Swift were a kangaroo, she would be a marsupial (true). Taylor Swift is a kangaroo. (False.) Therefore Taylor Swift is a marsupial (false.)\\ &Remember, if an argument is valid, the only thing that can't happen is for it to have all true premises and a false conclusion. So if you don't specify a false conclusion anything is possible.\\
\end{longtabu}



\begin{exercises}
\item A false tautology. 

\answerblank{Impossible. Tautologies, by definition, are always true.}{\vspace{1.5in}}

\item A valid argument that has a false conclusion

\answerblank{\underline{Possible}. Example: If grass is green, then I am the pope. (False) Grass is green. (True) \therefore  I am the pope. (False)}{\vspace{1.5in}}

\item A valid argument, the conclusion of which is a contradiction

\answerblank{\underline{Possible}. The conclusion is always false, but if the premises are also always false, you are fine. Example: If A, then not A. \therefore If B, then not B. \\}{\vspace{1.5in}}

\item An invalid argument, the conclusion of which is a tautology

\answerblank{\underline{Impossible}. If the conclusion is always true, then the there is no way for all the premises to be true and conclusion false.\\}{\vspace{1.5in}}

\item A tautology that is contingent

\answerblank{\underline{Impossible}. Contradictions, contingencies, and tautologies are exclusive categories. If you are one, you can't be either of the others. \\}{\vspace{1.5in}}


\item Two logically equivalent sentences, both of which are tautologies

\answerblank{\underline{Possible} In fact, all tautologies are logically equivalent. Logically equivalent sentences always have the same truth value, and all tautologies are always true. \\}{\vspace{1.5in}}


\item Two logically equivalent sentences, one of which is a tautology and one of which is contingent

\answerblank{\underline{Impossible}. A tautology is always true, but contingent sentences can be false. Therefore they can have different truth values. \\}{\vspace{1.5in}}


\item Two logically equivalent sentences that together are an inconsistent set

\answerblank{\underline{Possible} Two contradictions are logically equivalent, however it is impossible for them to both be true, because it is impossible for either one to be true. \\}{\vspace{1.5in}}


\item A consistent set of sentences that contains a contradiction

\answerblank{\underline{Impossible}. The contradiction can never be true, so the whole set cannot never all be true. \\}{\vspace{1.5in}}


\item An inconsistent set of sentences that contains a tautology
\answerblank{\underline{Possible}. Example: A, Not A, If A then A.}{\vspace{1.5in}} 
\end{exercises}

\noindent \problempart Which of the following is possible? If it is possible, give an example. If it is not possible, explain why.
\answer{All answers, except for the last question, are by Ben Sheredos}
\begin{exercises}
\item A valid argument, whose premises are all tautologies, and whose conclusion is contingent
\answer{Not Possible. If the argument is valid, then the conclusion must be true if the premises are true. If the premises are \textit{tautologies}, then the premises are \textit{always} true, and so the conclusion also must always be true.}

\item A valid argument with true premises and a false conclusion
\answer{ \textit{Absolutely not!} This contradicts the very definition of a valid argument.
}
\item A consistent set of sentences that contains two sentences that are not logically equivalent
\answer{ Most definitely. Here are two sentences that are consistent but not logically equivalent: ``Today is a Wednesday'' and ``I like pie.''
}
\item A consistent set of sentences, all of which are contingent
\answer{For sure. See the examples given in the previous answer. Both are contingent (sometimes it's not Wednesday today, and I might've hated pie.)
}
\item A false tautology
\answer{Not possible. By definition, a tautology is always true.
}
\item A valid argument with false premises
\answer{ Yup. Because validity only requires that \textit{if} the premises are true, \textit{then} the conclusion must be true. But all of them could be false, and the argument would remain valid. 
}
\item A logically equivalent pair of sentences that are not consistent
\answer{ Careful here. Our definition of consistency is that a set of statements are consistent if they could all be true at the same time. Well, consider the case of 2 statements which are logically equivalent, and which are both \textit{contradictions}. Neither can be true. So they cannot \textit{both} be true. So they are not consistent. 
}
\item A tautological contradiction
\answer{ Impossible. This is gibberish-nonsense.
}
\item A consistent set of sentences that are all contradictions
\answer{ Nope: see again \#7 above. If a set of statements contains nothing but contradictions, then none of them can be true. But if none of them can be true, then they cannot be true together, and so they cannot be consistent.
}

\item A valid argument, whose premises are all tautologies, and whose conclusion is contingent.
\answer{Impossible. If the conclusion is contingent, then it could be false, in which case you would have true premises and a false conclusion, which would make the argument invalid.}

\end{exercises}

\section*{Key Terms}
\begin{sortedlist}
\sortitem{Truth value}{} 	
\sortitem{Natural language}{}
\sortitem{Artificial language}{}
\sortitem{Formal language}{}
\sortitem{Tautology}{}
\sortitem{Contradiction}{}
\sortitem{Contingent statement}{}
\sortitem{Logically equivalent}{}
\sortitem{Contradictories}{}
\sortitem{Consistent}{}
\sortitem{Inconsistent}{}
\sortitem{Formal logic as concern for logical form}{}
\sortitem{Formal logic as strictly following rules}{}
\sortitem{Bivalent}{}
\end{sortedlist}
