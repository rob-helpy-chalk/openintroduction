\documentclass[a4paper,12pt]{article}
\usepackage[T1]{fontenc}
\usepackage{textcomp}
%\usepackage{endnotes}
\usepackage[latin1]{inputenc}
\usepackage[german,french,american]{babel}
\usepackage[autostyle]{csquotes}
%\usepackage[document]{ragged2e}
\usepackage[authordate,backend=biber,autolang=none,booklongxref=false,%
bibencoding=latin1,postnotepunct,compresspages,strict,%
annotation]{biblatex-chicago}
% \usepackage[style=chicago-authordate,backend=biber,usecompiler=true,%
% babel=hyphen,bibencoding=auto,sorting=nyt,cmslos,autocite=inline]{biblatex}
\usepackage{lmodern}
\usepackage{gentium}
%\renewcommand*{\rmdefault}{fgn}% The font (gentium) used for pdf
\usepackage{ifthen}
\usepackage{setspace}
\usepackage{vmargin} \setpapersize{A4}
\setmarginsrb{1in}{20pt}{1in}{.5in}{1pt}{2pt}{0pt}{13pt}
\usepackage{url}
\urlstyle{rm}
\appto\bibsetup{\sloppy}
\hyphenation{evans-ton clem-ens mc-hugh}
\setlength{\dimen\footins}{9.5in}
\setlength{\parindent}{0pt}
\setlength{\parskip}{5pt}
\providecommand{\theendnote}{}
\protected\def\onethird{{\scriptsize\raisebox{.7ex}{1}%
    \hspace{-0.1em}\raisebox{.2ex}{/}\hspace{-0.03em}3}}
\newcommand{\cmd}[1]{\texttt{\textbackslash #1}}
\newcommand{\mycolor}{}%[1]{\textcolor[HTML]{228B22}{#1}}
\usepackage{xr-hyper}
\externaldocument[cms-]{../../Docs/biblatex-chicago}%
\externaldocument[trad-]{cms-trad-appendix}
\usepackage[pdftex,hyperref,svgnames]{xcolor}
\usepackage[pdftex,colorlinks,urlcolor=DarkSlateGrey,citecolor=MidnightBlue,
plainpages=false,breaklinks=true,linkcolor=DarkSlateGrey,filecolor=Teal,
baseurl=biblatex-chicago.pdf\#]{hyperref}
\usepackage{cmsdocs}
\bibliography{dates-test}
%%\onehalfspacing
%\tracingstats=2
\begin{document}

\section*{The Chicago Author-Date Specification}
\label{sec:spec}

This file is intended as a brief introduction to the Chicago
author-date specification (16th ed.) \autocite{chicago:manual} as
implemented by \textsf{biblatex-chicago}, and falls somewhere in
between the \enquote{Quickstart} section of
\textsf{biblatex-chicago.pdf} and the full documentation as presented
in section~5 \cmssecref{cms-sec:authdate} of that same document.  I've
attempted to design this introduction for ease of cross-reference, so
clicking on citations should bring you to the reference list entry,
whence clicking on the entry key in the annotations should present you
with the entry as it appears in the .bib file, where clicking on the
entry type should return you to the reference list.  If you have
questions beyond the scope of this introduction, then the full
documentation is the place to look next --- marginal notes here refer
to section or page numbers there, and if you've installed the package
using the standard \TeX\ Live method then clicking on these marginal
notes should take you to the other document.  If you can't find
answers there, please write to me at the email address in
\textsf{biblatex-chicago.pdf}.

\subsection*{Important Note}
\label{bibernote}

Starting with \textsf{biblatex} version 1.5, in order to adhere to the
author-date specification you will need to use \textsf{Biber} to
process your .bib files, as \textsc{Bib}\TeX\ (and its more recent
variants) will no longer provide all the features the style requires.
For this release, you really need the current versions of
\textsf{Biber} (2.7) and \textsf{biblatex} (3.7), which contain
features and bug-fixes on which my own code relies.  The advice that
follows in this document assumes that you are using \textsf{Biber}; if
you wish to continue using \textsc{Bib}\TeX\ then you need
\textsf{biblatex} version 1.4c and \textsf{biblatex-chicago} 0.9.7a.

\subsection*{Editions}
\label{editions}

The 16th edition of \emph{The Chicago Manual of Style} implements
significant changes to what the author-date specification has,
historically, recommended, and there are certain to be users who
prefer the older format with titles capitalized sentence-style and
not, in the case of most un-book-like entries, enclosed in quotation
marks.  For such users, the \textsf{authordate-trad} style, as
envisaged by the \textcite[15.45]{chicago:manual}, grafts the
traditional Chicago author-date title formatting onto the current
recommendations for the remainder of the reference apparatus.  Please
consult
\href{file:cms-trad-appendix.pdf}{\textsf{cms-trad-appendix.pdf}}
for notes on the few .bib entries from this introduction that would
require modification for the \texttt{trad} style; for the remainder,
you'll notice a few extra sets of curly braces in various
\textsf{title} fields to make the entry usable in both author-date
styles.  The 15th-edition styles are still in the package, but they
have not been updated in some time, and are now officially obsolete.
I would strongly encourage all users to switch to one of the
16th-edition styles as soon as possible, as I am concentrating all of
my development time there.

\subsection*{Usage}
\label{usage}

As a general rule, you'll probably want to use the \cmd{autocite}
command for most citations.  For most sources, the result will be
exactly as you expect it to be.  A few examples:
\autocite{adorno:benj}; \autocite{ashbrook:brain};
\autocite{babb:peru}; \autocite{barcott:review}.  Any page references
should also appear as you expect: \autocite[338]{batson};
\autocite[79]{beattie:crime}; \autocite[36]{boxer:china}.

\subsection*{Repeated citations}
\label{sec:ibidem}

Repeated citations are somewhat complicated.  The Chicago author-date
style doesn't use \enquote{\emph{Ibid},} but in general a repeated
citation on the same page will print only the page reference:
\autocite{browning:aurora}; \autocite[45]{browning:aurora}.
Technically, this should only occur when a source is cited
\enquote{more than once in one paragraph}
\autocite[15.26]{chicago:manual}, so you can use the \cmd{citereset}
command from \textsf{biblatex} to achieve the greatest compliance, as
the package only offers automatic resetting on part, chapter, section,
and subsection boundaries, while \textsf{biblatex-chicago}
automatically resets the tracker at page breaks:

\citereset\cmd{citereset}\ \autocite[15.27]{chicago:manual}.  If you
are going to repeat a source, make sure that the cite command provides
a postnote --- when using \textsf{biblatex-chicago} you'll no longer
get any annoying empty parentheses, but you will get another standard
citation, which may add too much clutter: \autocite{chicago:manual}.
If you don't need to cite a specific page, then it may be better, or
at least more concise, only to use one citation command rather than
two.

\subsection*{Other citation commands}
\label{sec:other}

The other \cmssecref{cms-sec:cite:authordate} citation commands from
\textsf{biblatex} also work fine:

\cmd{textcite}: \textcite{conley:fifthgrade}; \cmd{autocite*}:
\autocite*{connell:chronic}; \cmd{cite}: \cite{conway:evolution};
\cmd{cite*}: \cite*{davenport:attention}; \cmd{foot\-note} with
\cmd{autocite};\footnote{\autocite{donne:var}.}\ \cmd{footcite}
(=\cmd{cite} inside a \cmd{footnote}).  \footcite{dunn:revolutions}

Multicites should work as you expect, too:

\cmd{autocites}: \autocites{dyna:browser}{eliot:pound};
\cmd{autocites} by the same author:
\autocites{pirumova}{pirumova:russian}; \cmd{autocites} by the same
author with postnotes: \autocites{pirumova}[14]{pirumova:russian};
\cmd{textcites} by the same author with postnotes:
\textcites[37]{pirumova}{pirumova:russian}.

\textsf{Biblatex-chicago} now also provides a \cmd{gentextcite}
command, which prints an \gentextcite{author:forthcoming} name in the
genitive case in what is otherwise a standard \cmd{textcite}.  If you
want to change the default -- \textbf{'s} -- printed there you can
specify whatever text you wish like so:
\cmd{gentextcite[<ending>][][]\{entry:key\}}.  There is also a
\cmd{gentextcites} command, modified thus:
\cmd{gentextcites[<ending>]()()[][]\{key1\}\{key2\}}.

\subsection*{Shorthands}
\label{sec:shorthands}

Chicago's author-date style
\cmssecref[shorthand]{cms-sec:ad:shorthand} only seems to recommend
the use of shorthands as abbreviations for long authors' names,
particularly institutional names \autocite[15.36]{chicago:manual}.  By
default, I have followed this recommendation: \cmd{autocites}:
\autocites{bsi:abbreviation}{iso:electrodoc}; \cmd{textcites}:
\textcites{bsi:abbreviation}{iso:electrodoc}.  This \textsf{shorthand}
will by default appear at the head of the entry in the list of
references, followed by the parenthesized expansion of the shorthand,
taken from the \textsf{author} field.  (This is a change from the 15th
edition.)  You will usually also need a \textsf{sortkey} field to make
sure that the entry is alphabetized by the \textsf{shorthand} rather
than by the \textsf{title}.  If you use a
\cmd{printbiblist\{shorthand\}} command, the list of shorthands will
still be printed, so you now have a variety of options available for
presenting the expansions depending on your specific requirements.
Please note, also, that you can get back something approaching the
\enquote{standard} behavior of shorthands if you give the
\texttt{cmslos=false} option to \textsf{biblatex-chicago} in your
document preamble.

\subsection*{Mildly problematic entries}
\label{sec:problematic}

In most \cmssecref[author]{cms-sec:ad:author} entries, the absence of
an author can be supplied by, e.g., an editor or a translator:
\autocite{chaucer:alt}; \autocite{silver:gawain}.  Sometimes an
anonymous work's author is known or can be guessed:
\autocite{horsley:prosodies}; \autocite{cook:sotweed}.  Alternatively,
in some cases the \textsf{title} may appear in place of the
\textsf{author}: \autocite{anon:stanze};
\autocite{virginia:plantation}.  The 16th edition is less than
enthusiastic about the use of \enquote{\texttt{Anon.}}\ as author.

By default, in most \cmssecref[date]{cms-sec:ad:date} entry types, an
absent \textsf{date} will automatically provoke \textsf{Biber} into
searching for other sorts of dates in the entry, in the order
\textsf{year, eventyear, origyear, urlyear}: e.g.,
\autocite{evanston:library}, which only has a \textsf{urlyear}.  In
three entry types --- \textsf{Music}, \textsf{Review}, and
\textsf{Video} --- this search order is \textsf{eventyear, origyear,
  year, urlyear}, as in these types the earliest year should take
precedence (cf.\ page~\pageref{sec:audiovisual}, below).  You can also
change the default search order, for all but the three types just
mentioned, by using the \texttt{cmsdate} option in the preamble of
your document, instead of (or in addition to) using it in the
\textsf{options} field of individual entries.  Setting that option in
the preamble either to \enquote{\texttt{both}} or
\enquote{\texttt{on}} makes the document-wide search order:
\textsf{origyear, year, eventyear, urlyear}.  This may be useful for
documents that contain many entries with multiple dates, and where you
want \emph{always} to present the earlier (i.e., \textsf{orig}) dates
at the head of reference list entries and in citations.  You can
eliminate some of these dates from the running, or change the search
order, using the \cmd{DeclareLabeldate} command in your preamble, but
please be aware that I have hard-coded the possibilities above into
the author-date style in order to cope with some tricky corners of the
specification.  If you reorder these dates, and your references enter
these tricky corners, the results might be surprising.  (Cf.\
section~4.5.8 in \textsf{biblatex.pdf}.)

In most entry types, the absence of all four possible dates will
automatically produce \mbox{\enquote{\texttt{n.d.}\hspace{-2pt}}}
instead: \autocite{bernstein:shostakovich}.  You can also give it
yourself in the form \cmd{bibstring\{nodate\}}:
\autocite{ross:thesis}.  A date that can be guessed should appear
within square brackets: \autocite{clark:mesopot}.  Forthcoming works
are straightforward, assuming you remember to use the \cmd{autocap}
macro and the \textsf{year} (instead of the \textsf{date}) field, so
that the word appears correctly in both citations and the list of
references: \autocite{author:forthcoming}; \autocite{contrib:contrib}.

The 16th edition of the \emph{Manual} has changed the rules for
entries with more than one date \autocite[15.38]{chicago:manual}.
First, \textsf{Music}, \textsf{Review}, and \textsf{Video} entries
have their own rules, which are applied automatically.  (Once again,
see page~\pageref{sec:audiovisual}, below.)  For other entry types,
there are two options, corresponding to two different states of the
\texttt{cmsdate} entry (or preamble) option.  The default is
\texttt{cmsdate=off}: \autocite{maitland:equity}.  Here, setting the
\textsf{pubstate} field to \texttt{reprint} ensures that a notice of
the original publication date will be printed at the end of the
reference list entry.  Alternatively, you can use
\texttt{cmsdate=both}: \autocite{emerson:nature};
\autocite{maitland:canon}.  \texttt{cmsdate=new} and
\texttt{cmsdate=old} are both now synonyms of \texttt{both}, while
\texttt{cmsdate=on} is still available even though it falls outside
the specification: \autocite{james:ambassadors}.  These options, in
combination with others available in your .bib files, can cover a wide
range of difficult cases.  Please see the next section below, and also
the following entries in \textsf{dates-test.bib}:
\autocites{schweitzer:bach}{white:russ}{white:ross:memo}.

\subsection*{Corners of the specification}
\label{sec:corners}

In some cases, the \emph{Manual} isn't altogether clear about how to
present entries in the author-date style.  By following up on
suggestions from the notes \&\ bibliography style, one can be
reasonably certain about most of what follows, but if you interpret
the specification differently please let me know.

\subsubsection*{InReference entries}
\label{sec:inref}

These present \cmssecref[inreference]{cms-sec:ad:inreference} several
peculiarities: the title of the work should always take the place of
any author, no \enquote{\texttt{n.d.}\hspace{-2pt}} will automatically
be provided, and any postnote field will be enclosed in quotation
marks preceded by \enquote{\texttt{s.v.}\hspace{-2pt}} for
\enquote{\emph{sub verbo}.}  This allows you to refer to alphabetized
articles in well-known reference works: \autocite[Hume,
David]{ency:britannica}; \autocite[Sibelius, Jean]{grove:sibelius};
\autocite[BibTeX]{wikiped:bibtex}.

\subsubsection*{Author-less Article, Review, and Manual entries}
\label{sec:authless:art}

In \textsf{Article} and \textsf{Review} entries
\cmssecref[article]{cms-sec:ad:article} with the \texttt{magazine}
entrysubtype, the absence of an author automatically places the
\textsf{journaltitle} of the periodical in citations and at the head
of the entry in the list of references: \autocite{gourmet:052006}.
(Without the entrysubtype, you'll get the \textsf{title} at the head
rather than the \textsf{journaltitle}.)  You can cite newspaper and
magazine articles entirely within the text, i.e., without them
appearing in the reference list \autocite[15.47]{chicago:manual}, if
you set the \texttt{cmsdate=full} entry option:
\autocite{lakeforester:pushcarts}; \autocite{nyt:trevorobit}.  In
\textsf{Manual} entries, the \textsf{organization} field does the
same: \autocite{dyna:browser}.  If you wish to present an abbreviated
form of the organization name in citations only, then the
\textsf{shortauthor} field --- or in other cases the
\textsf{shorthand} field --- is the place for it:
\autocite{bsi:abbreviation}.  For abbreviated \textsf{journaltitles},
you can use \textsf{shortjournal}, which also allows you, should you
wish, to provide a list of abbreviated journal names with their
expansions using \cmd{printbiblist\{shortjournal\}}:
\autocite{unsigned:ranke}.

\subsubsection*{Misc entries with an entrysubtype}
\label{sec:misc}

When \cmssecref[misc]{cms-sec:ad:misc} citing individual letter-like
pieces from an unpublished archive where only an \textsf{origdate} is
present, you no longer need to set the \texttt{cmsdate} option in your
.bib entry, as \textsf{Biber} and \textsf{biblatex-chicago} now handle
this automatically: \autocite{creel:house}.  Non-letters, e.g.,
interviews, use the \textsf{date} field, so you don't need
\texttt{cmsdate} there, either: \autocite{spock:interview}.  For
undated pieces you can put \cmd{bibstring\{nodate\}} in the
\textsf{year} field: \autocite{dinkel:agassiz}.  For citing whole
collections, see the next section.

\subsubsection*{entrysubtype = \{classical\}}
\label{sec:classical}

This option's \cmssecref[entrysubtype]{cms-sec:ad:entrysubtype} name
derives from its use for citing texts from classical antiquity, though
in the author-date style especially it can be put to use in several
other contexts.  In a nutshell, any entry with such an
\textsf{entrysubtype} will be treated, in citations only, not as
author-date but as author-title.  (Entries in the list of references,
e.g., a particular edition of Aristotle, will still appear in standard
author-date format.)  A \cmd{cite*} or \cmd{autocite*} command will,
in such a case, produce the title rather than the year.  Some examples
should make this clearer:

%\enlargethispage{-\baselineskip}

Classical works: without abbreviation:
\autocite{aristotle:metaphy:trans}; with abbreviation:
\autocite{aristotle:metaphy:gr}; \autocite{plato:republic:gr}; using
standard pagination: \autocite*[3.2.996b5--8]{aristotle:metaphy:gr};
\autocite*[420e]{plato:republic:gr}; work cited by page of a modern
edition, i.e., without \textsf{entrysubtype}:
\autocite[198]{euripides:orestes}.

Sacred works, e.g., the Bible and the Qur'an:
\autocite[25:19--36:43]{genesis}.

An unpublished archive, from which more than one work has been cited:
\autocite[file 12]{house:papers}.  (Both this and the previous example
use a Misc entry with \texttt{classical} \textsf{entrysubtype}.)

\subsubsection*{Comments inside citations}
\label{sec:comments}

If you wish to include a comment inside the parentheses of a citation,
it will need to be separated by a semicolon
\autocite[15.23]{chicago:manual}.  If you have a \textsf{postnote},
then you can manually provide the punctuation and comment in that
field, e.g., \autocite[4; the unrevised trans.]{stendhal:parma}.
Without a \textsf{postnote}, you have two choices.  You can enable the
\texttt{postnotepunct} package option, which allows you simply to type
\cmd{autocite[; the unrevised trans.]\{stendhal:\\parma\}}
\citereset\autocite[; the unrevised trans.]{stendhal:parma}, or you
can continue to use a separate \textsf{Misc} or \textsf{CustomC} entry
containing just the text of the comment in the \textsf{title} field,
\textsf{entrysubtype} \texttt{classical}, and \textsf{options}
\texttt{skipbib}.  An \cmd{autocites} command calling both the main
text and the comment will then do the trick, e.g.,
\autocites{chicago:manual}{chicago:comment}.

\subsubsection*{Multiple authors}
\label{sec:multiple}

The default settings in \textsf{biblatex-chicago} are
\texttt{maxnames=3,minnames=1} in citations and
\texttt{max\-bibnames=10,minbibnames=7} in the list of references
(these latter parameters set in \textsf{biblatex-chicago.sty}).  In
practice, this means that an entry like hlatky:hrt, with 5 authors,
will present all of them in the list of references but will truncate
to one in citations, like so: \autocite{hlatky:hrt}.  For the vast
majority of circumstances, these settings are exactly right for the
Chicago author-date specification.  However, if \enquote{a reference
  list includes another work \emph{of the same date} that would also
  be abbreviated as [\enquote{Hlatky et al.}] but whose coauthors are
  different persons or listed in a different order, the text citations
  must distinguish between them} \autocite[15.28]{chicago:manual}.
The (\textsf{Biber}-only) \textsf{biblatex} option
\texttt{uniquelist}, set for you in \textsf{biblatex-chicago.sty},
will automatically handle many of these situations for you, but it is
as well to understand that it does so by temporarily suspending the
limits, listed above, on how many names to print in a citation.
Without \texttt{uniquelist}, \textsf{biblatex} would present such a
work as, e.g., (Hlatky et al. 2002b), while hlatky:hrt would be
(Hlatky et al. 2002a).  This does distinguish between them, but
inaccurately, as it suggests that the two different author lists are
exactly the same.  With \texttt{uniquelist}, the two citations might
look like (Hlatky, Boothroyd et al.\ 2002) and (Hlatky, Smith et al.\
2002), which is what the specification requires.

If, however, the distinguishing name occurs further down the author
list --- in fourth or fifth position in our examples --- then the
default settings would produce citations with all 4 or 5 names
printed, which can become awkwardly long.  In such a situation, you
can provide \textsf{shortauthor} fields that look like this:
\{\{Hlatky et al., \textbackslash mkbibquote\{Quality of Life,\}\}\}
and \{\{Hlatky et al., \textbackslash mkbibquote\{Depressive
Symptoms,\}\}\}, using a shortened title to distinguish the
references.  This would produce (Hlatky et al., \enquote{Quality of
  Life,} 2002) and (Hlatky et al., \enquote{Depressive Symptoms,}
2002), as the spec recommends.  There is, unfortunately, no simpler
way that I know of to deal with this situation.

\subsubsection*{Audiovisual entries}
\label{sec:audiovisual}

According \cmssecref{cms-sec:ad:avdate} to the \emph{Manual},
\enquote{Chicago recommends a more comprehensive approach to dating
  audiovisual materials than in previous editions.}  This means, for
instance, that, even when consulting a digital copy, \enquote{it is
  generally useful to give information about the original source.}
Also, \enquote{the date of the original recording should be privileged
  in the citation} \autocite[15.53]{chicago:manual}.  The rather more
book-like entries are generally unaffected by these changes, so
published (\textsf{Audio}) and unpublished (\textsf{Misc}) scores are
no problem at all: \autocite{schubert:muellerin};
\autocite{verdi:corsaro}; \autocite{shapey:partita}.  The dating of
online materials has been enhanced: \autocite{coolidge:speech};
\autocite{horowitz:youtube}; \autocite{pollan:plant}.  The most
significant changes, however, appear in \textsf{Music} and
\textsf{Video} entries, where every effort should be made to find
date(s) for sources: \autocite{auden:reading};
\autocite{friends:leia}; \autocite{handel:messiah};
\autocite{holiday:fool}; \autocite{nytrumpet:art}.  Others perhaps
require further information in the entry or genuinely are better
suited to presentation in running text: \autocite{beethoven:sonata29}.
The standard \textsf{biblatex} tools for subdividing reference lists
are all available if you want to follow the \emph{Manual's}
recommendations on presenting this kind of material separately from
other sources.

\subsubsection*{Related entries}
\label{sec:related}

\textsf{Biblatex} provides \cmssecref{cms-sec:authrelated} a powerful
mechanism, using the \textsf{related} field, for grouping two (or
more) works together in a single entry in the list of references,
while \textsf{biblatex-chicago} offers both this functionality and
some Chicago-specific variants which employ different means.  You can
find a full discussion of this in \textsf{biblatex-chicago.pdf}, but a
few of the entries already cited in this introduction show some of the
possibilities: \autocite{aristotle:metaphy:trans};
\autocite{coolidge:speech}; \autocite{emerson:nature};
\autocite{schweitzer:bach}.

\subsection*{In conclusion}
\label{sec:conclude}

Allow me, finally, to emphasize just how multifarious are the sources
illustrated in the \emph{CMS}, only a small selection of which have
appeared in this introduction.  You will find significantly fuller
guidance in \textsf{biblatex.pdf} and \textsf{biblatex-chicago.pdf},
but the \emph{CMS} itself defines the specification and shall
arbitrate all disputes.  If you see something in
\textsf{biblatex-chicago} that looks wrong to you, or if the
documentation has left you perplexed, please let me know.


\printbibliography[title=References]
\setlength{\textheight}{10.5in}
\twocolumn[\Large \texttt{The Database File}]
\vspace*{-6pt}
\begin{lstlisting}[language=BibTeX,label=prologue]
%% Database entries used to produce
%% citations in this file, taken
%% from dates-test.bib. I have
%% removed the annotations to save
%% room -- you can click on
%% the entry type to return to the
%% reference list entry, where you'll
%% also find the annotations. You can
%% click on text with a grey back-
%% ground to switch to that entry
%% within this .bib listing. Through-
%% out this listing you'll see curly
%% braces around parts of titles and
%% subtitles, which allow the entry
%% to work equally well in authordate
%% and authordate-trad.

@String{cup = {Cambridge University Press}}
@String{hup = {Harvard University Press}}
@String{uchp = {University of Chicago Press}}
@String{oup = {Oxford University Press}}
\end{lstlisting}
\begin{lstlisting}[language=BibTeX,label=adorno:benj]
*\adlnbackref{Book}{adorno:benj}*,
  title = 	 {The Complete Correspondence, 1928--1940},
  publisher = 	 hup,
  year = 	 1999,
  author = 	 {Adorno, Theodor~W. and Benjamin, Walter},
  editor = 	 {Lonitz, Henri},
  translator = 	 {Nicholas Walker},
  location =  {Cambridge, MA}
}
\end{lstlisting}
\begin{lstlisting}[language=BibTeX,label=anon:stanze]
*\adlnbackref{Book}{anon:stanze}*,
  title = 	 {Stanze in lode della donna brutta},
  date = 	 1547,
  address = 	 {Florence},
  shorttitle = 	 {Stanze}
}
\end{lstlisting}
\begin{lstlisting}[language=BibTeX,label=aristotle:metaphy:gr]
*\adlnbackref{Book}{aristotle:metaphy:gr}[aristotle:metaphy:trans]*,
  shorttitle = 	 {Metaph\adddot},
  title = 	 {Metaphysics},
  options = 	 {skipbib},
  entrysubtype = {classical},
  origdate = 	 1924,
  date = 1997,
  author = 	 {Aristotle},
  editor = 	 {Ross, W.~D.},
  publisher = {Oxford Univ.\ Press and Sandpiper Books},
  pubstate = 	 {reprint},
  volumes = 	 2,
  location =  {Oxford}
}
\end{lstlisting}
\begin{lstlisting}[language=BibTeX,label=aristotle:metaphy:trans]
*\adlnbackref{Book}{aristotle:metaphy:trans}*,
  title = 	 {Metaphysica},
  entrysubtype = {classical},
  year = 	 1928,
  volume = 	 8,
  author = 	 {Aristotle},
  editor = 	 {Ross, W.~D.},
  nameb = 	 {Ross, W.~D.},
  origlanguage = {greek},
  userf = 	 {*\hyperlink{\getrefbykeydefault%
{aristotle:metaphy:gr}{anchor}{}}%
{\{\colorbox{Gainsboro}{aristotle:metaphy:gr}\}}*},
  maintitle = 	 {The Works of {Aristotle}, Translated into {English}},
  publisher = {Clarendon Press},
  edition = 	 2,
  location =  {Oxford}
}
\end{lstlisting}
\begin{lstlisting}[language=BibTeX,label=ashbrook:brain]
*\adlnbackref{InBook}{ashbrook:brain}*,
  author = 	 {Ashbrook, James~B. and Albright, Carol Rausch},
  title = 	 {The Frontal Lobes, Intending, and a Purposeful God},
  booktitle = 	 {The Humanizing Brain},
  publisher = {Pilgrim Press},
  year = 	 1997,
  chapter = 	 7,
  location =  {Cleveland, OH},
  shorttitle = {The Frontal Lobes}
}
\end{lstlisting}
\begin{lstlisting}[language=BibTeX,label=auden:reading]
*\adlnbackref{Music}{auden:reading}*,
  title = 	 {Selected Poems},
  author = 	 {Auden, W. H.},
  date = 	 {1991},
  number = 	 7137,
  series = 	 {Spoken Arts},
  type = 	 {audiocassette},
  note = 	 {read by the author}
}
\end{lstlisting}
\begin{lstlisting}[language=BibTeX,label=author:forthcoming]
*\adlnbackref{Article}{author:forthcoming}*,
  author = 	 {Author, Margaret~M.},
  title = 	 {Article Title},
  journaltitle = {Journal Name},
  year = 	 {\autocap{f}orthcoming},
  volume = 	 98
}
\end{lstlisting}
\begin{lstlisting}[language=BibTeX,label=babb:peru]
*\adlnbackref{Book}{babb:peru}*,
  title = 	 {Between Field and Cooking Pot},
  subtitle = 	 {The Political Economy of Marketwomen in {Peru}},
  year = 	 1989,
  author = 	 {Babb, Florence},
  publisher = {University of Texas Press},
  edition = 	 {\bibstring{revisededition}},
  location =  {Austin}
}
\end{lstlisting}
\begin{lstlisting}[language=BibTeX,label=barcott:review]
*\adlnbackref{Review}{barcott:review}*,
  journaltitle = {New York Times Book Review},
  author =	 {Barcott, Bruce},
  date = 	 {2000-04-16},
  entrysubtype = {magazine},
  title =	 {\bibstring{reviewof} \mkbibemph{The Last Marlin: The Story of a Family at Sea}, \bibstring{by} Fred Waitzkin},
  pages =	 7
}
\end{lstlisting}
\begin{lstlisting}[language=BibTeX,label=batson]
*\adlnbackref{Article}{batson}*,
  author =	 {Batson, C.~Daniel},
  title =	 {How Social Is the Animal? {The} Human Capacity for Caring},
  journaltitle = {American Psychologist},
  volume =	 45,
  date = 	 {1990-03},
  pages =	 {336--346}
}
\end{lstlisting}
\begin{lstlisting}[language=BibTeX,label=beattie:crime]
*\adlnbackref{Article}{beattie:crime}*,
  author = 	 {Beattie, J.~M.},
  title = 	 {The Pattern of Crime in {England}, 1660--1800},
  journaltitle = {Past and Present},
  year = 	 1974,
  number = 	 62,
  pages = 	 {47--95}
}
\end{lstlisting}
\begin{lstlisting}[language=BibTeX,label=beethoven:sonata29]
*\adlnbackref{Music}{beethoven:sonata29}*,
  title = 	 {Piano Sonata \bibstring{number} 29 \mkbibquote{Hammerklavier}},
  author = 	 {Beethoven},
  editor = 	 {Peter Serkin},
  editortype = 	 {none},
  number = 	 {CDD 270},
  series = 	 {Proarte Digital}
}
\end{lstlisting}
\begin{lstlisting}[language=BibTeX,label=bernstein:shostakovich]
*\adlnbackref{Music}{bernstein:shostakovich}*,
  title = 	 {Symphony \bibstring{number} 5},
  author = 	 {Shostakovich, Dmitri},
  editor = 	 {Bernstein, Leonard},
  editortype = 	 {conductor},
  editora = 	 {{New York Philharmonic}},
  editoratype =  {none},
  number = 	 {IM 35854},
  series = 	 {CBS},
  options = 	 {useauthor=false}
}
\end{lstlisting}
\begin{lstlisting}[language=BibTeX,label=boxer:china]
*\adlnbackref{Book}{boxer:china}*,
  title = 	 {South {China} in the Sixteenth Century},
  year = 	 1953,
  editor = 	 {Boxer, Charles~R.},
  number = 	 {2nd ser., 106},
  series = 	 {Hakluyt Society Publications},
  location =  {London}
}
\end{lstlisting}
\begin{lstlisting}[language=BibTeX,label=browning:aurora]
*\adlnbackref{Book}{browning:aurora}*,
  title =	 {{Aurora Leigh}},
  subtitle =	 {Authoritative Text, Backgrounds and Contexts, Criticism},
  year =	 1996,
  author =	 {Browning, Elizabeth Barrett},
  editor =	 {Reynolds, Margaret},
  publisher =	 {Norton},
  series =	 {Norton Critical Editions},
  location =	 {New York}
}
\end{lstlisting}
\begin{lstlisting}[language=BibTeX,label=bsi:abbreviation]
*\adlnbackref{Manual}{bsi:abbreviation}*,
  title = 	 {Specification for Abbreviation of Title Words and Titles of Publications},
  date = 	 1985,
  organization = {British Standards Institute},
  sortname = 	 {BSI},
  address = 	 {Linford Woods, Milton Keynes, UK},
  shorthand = 	 {BSI}
}
\end{lstlisting}
\begin{lstlisting}[language=BibTeX,label=chaucer:alt]
*\adlnbackref{Book}{chaucer:alt}*,
  title = 	 {Chaucer Life-Records},
  year = 	 1966,
  editor = 	 {Crow, Martin~M. and Olson, Clair~C.},
  namec = 	 {Manly, John~M. and Richert, Edith},
  publisher = oup,
  note = 	 {with the assistance of Lilian~J. Redstone and others},
  location =  {London}
}
\end{lstlisting}
\begin{lstlisting}[language=BibTeX,label=chicago:comment]
*\adlnbackref{CustomC}{chicago:comment}[chicago:manual]*,
  title = 	 {the most recent edition},
  entrysubtype = {classical},
  options = 	 {skipbib},
  annotation = 	 {An example of how to use a CustomC entry to insert a comment inside another parenthetical citation.}
}
\end{lstlisting}
\begin{lstlisting}[language=BibTeX,label=chicago:manual]
*\adlnbackref{Book}{chicago:manual}*,
  title = 	 {The {Chicago} Manual of Style},
  year = 	 2010,
  author = 	 {{University of Chicago Press}},
  publisher = uchp,
  edition = 	 16,
  location =  {Chicago}
}
\end{lstlisting}
\begin{lstlisting}[language=BibTeX,label=clark:mesopot]
*\adlnbackref{Booklet}{clark:mesopot}*,
  title = 	 {Mesopotamia},
  subtitle = 	 {Between Two Rivers},
  author = 	 {Hazel V. Clark},
  howpublished = {End of the Commons General Store},
  year = 	 {\mkbibbrackets{1957?}},
  location =  {Mesopotamia, OH}
}
\end{lstlisting}
\begin{lstlisting}[language=BibTeX,label=conley:fifthgrade]
*\adlnbackref{Article}{conley:fifthgrade}*,
  author =	 {Conley, Alice},
  title =	 {Fifth-Grade Boys' Decisions about Participation in Sports Activities},
  issuetitle =	 {Non-subject-matter Outcomes of Schooling},
  journaltitle = {Elementary School Journal},
  note = 	 {special issue},
  year =	 1999,
  volume =	 99,
  editor =       {Good, Thomas~L.},
  number =	 5,
  pages =	 {131--146}
}
\end{lstlisting}
\begin{lstlisting}[language=BibTeX,label=connell:chronic]
*\adlnbackref{Article}{connell:chronic}*,
  author = 	 {Connell, A.~D. and Airey, D.~D.},
  title = 	 {The Chronic Effects of Fluoride on the Estuarine Amphipods \mkbibemph{Grandidierella lutosa} and \mkbibemph{G. lignorum}},
  journaltitle = {Water Research},
  date = 	 1982,
  volume = 	 16,
  pages = 	 {1313--1317}
}
\end{lstlisting}
\begin{lstlisting}[language=BibTeX,label=contrib:contrib]
*\adlnbackref{InCollection}{contrib:contrib}*,
  author = 	 {Contributor, Anna},
  title = 	 {Contribution},
  booktitle = 	 {Edited Volume},
  publisher = {Publisher},
  year = 	 {\autocap{f}orthcoming},
  editor = 	 {Editor, Ellen},
  location =  {Place}
}
\end{lstlisting}
\begin{lstlisting}[language=BibTeX,label=conway:evolution]
*\adlnbackref{Article}{conway:evolution}*,
  author = 	 {Conway, M.~S.},
  title = 	 {The Evolution of Diversity in Ancient Ecosystems},
  subtitle = 	 {A Review},
  journaltitle = {Philosophical Transactions of the Royal Society},
  date = 	 1998,
  volume = 	 {B 353},
  pages = 	 {327--345}
}
\end{lstlisting}
\begin{lstlisting}[language=BibTeX,label=cook:sotweed]
*\adlnbackref{Book}{cook:sotweed}*,
  title = 	 {Sotweed Redivivus, or The Planter's Looking-Glass},
  year = 	 1730,
  author = 	 {Cook, Ebenezer},
  authortype = 	 {anon?},
  note = 	 {\bibstring{by} \mkbibquote{E.~C. Gent}},
  location =  {Annapolis}
}
\end{lstlisting}
\begin{lstlisting}[language=BibTeX,label=coolidge:speech]
*\adlnbackref{Online}{coolidge:speech}*,
  author = 	 {Coolidge, Calvin},
  title = 	 {Equal Rights},
  note = 	 {copy of an undated 78 rpm disc},
  options = 	 {ptitleaddon=space},
  titleaddon = 	 {(speech)},
  related = 	 {*\hyperlink{\getrefbykeydefault%
{loc:leaders}{anchor}{}}{\{\colorbox{Gainsboro}{loc:leaders}\}}*},
  year = 	 {[1920?]},
  relatedstring = {from}
}
\end{lstlisting}
\begin{lstlisting}[language=BibTeX,label=creel:house]
*\adlnbackref{Misc}{creel:house}*,
  author = 	 {Creel, George},
  entrysubtype = {letter},
  title = 	 {George Creel to Colonel House},
  note = 	 {Edward~M. House Papers},
  origdate = 	 {1918-09-25},
  organization =  {Yale University Library}
}
\end{lstlisting}
\begin{lstlisting}[language=BibTeX,label=davenport:attention]
*\adlnbackref{Book}{davenport:attention}*,
  title =	 {The Attention Economy},
  subtitle =	 {Understanding the New Currency of Business},
  year =	 2001,
  author =	 {Davenport, Thomas~H. and Beck, John~C.},
  publisher =	 {Harvard Business School Press},
  addendum =	 {TK3 Reader e-book},
  location =	 {Cambridge, MA}
}
\end{lstlisting}
\begin{lstlisting}[language=BibTeX,label=dinkel:agassiz]
*\adlnbackref{Misc}{dinkel:agassiz}*,
  author = 	 {Dinkel, Joseph},
  title = 	 {description of Louis Agassiz written at the request of Elizabeth Cary Agassiz},
  year = 	 {\bibstring{nodate}},
  entrysubtype = {yes},
  note = 	 {Agassiz Papers},
  location =  {Harvard University},
  organization =  {Houghton Library}
}
\end{lstlisting}
\begin{lstlisting}[language=BibTeX,label=donne:var]
*\adlnbackref{Book}{donne:var}*,
  author =	 {Donne, John},
  editor =	 {Stringer, Gary~A.},
  title =	 {The \mkbibquote{Anniversaries} and the \mkbibquote{Epicedes and Obsequies}},
  namea =	 {Stringer, Gary~A. and Pebworth, Ted-Larry},
  publisher =	 {Indiana Univ. Press},
  maintitle =	 {The Variorum Edition of the Poetry of {John Donne}},
  year =	 1995,
  volume =	 6,
  location =	 {Bloomington}
}
\end{lstlisting}
\begin{lstlisting}[language=BibTeX,label=dunn:revolutions]
*\adlnbackref{Book}{dunn:revolutions}*,
  title = 	 {Sister Revolutions},
  subtitle = 	 {French Lightning, {American} Light},
  year = 	 1999,
  author = 	 {Dunn, Susan},
  publisher = {Faber \& Faber and Farrar, Straus \& Giroux},
  location =  {New York}
}
\end{lstlisting}
\begin{lstlisting}[language=BibTeX,label=dyna:browser]
*\adlnbackref{Manual}{dyna:browser}*,
  title = 	 {Dynatext, Electronic Book Indexer/Browser},
  organization = {Electronic Book Technology Inc.},
  address = 	 {Providence, RI},
  year = 	 1991
}
\end{lstlisting}
\begin{lstlisting}[language=BibTeX,label=eliot:pound]
*\adlnbackref{Book}{eliot:pound}*,
  title = 	 {Literary Essays},
  options = {useauthor=false},
  year = 	 1953,
  author = 	 {Pound, Ezra},
  editor =  {Eliot, T.~S.},
  publisher = {New Directions},
  location =  {New York}
}
\end{lstlisting}
\begin{lstlisting}[language=BibTeX,label=emerson:nature]
*\adlnbackref{Book}{emerson:nature}*,
  title =	 {Nature},
  year =	 1985,
  origdate =	 1836,
  location = 	 {Boston},
  options = 	 {cmsdate=old},
  author =	 {Emerson, Ralph Waldo},
  publisher = 	 {Beacon},
  note = 	 {a facsimile of the first \bibstring{edition} with an \bibstring{introduction} by Jaroslav Pelikan}
}
\end{lstlisting}
\begin{lstlisting}[language=BibTeX,label=ency:britannica]
*\adlnbackref{InReference}{ency:britannica}*,
  title =        {Encyclopaedia Britannica},
  edition =      {15},
  shorttitle = 	 {Ency. {Brit}., \mkbibemph{15th ed}\adddot},
}
\end{lstlisting}
\begin{lstlisting}[language=BibTeX,label=euripides:orestes]
*\adlnbackref{BookInBook}{euripides:orestes}*,
  title = 	 {Orestes},
  year = 	 1958,
  booktitle = 	 {Euripides},
  maintitle = 	 {The Complete {Greek} Tragedies},
  nameb = 	 {Arrowsmith, William},
  volume = 	 4,
  author = 	 {Euripides},
  editor = 	 {Grene, David and Lattimore, Richmond},
  publisher = uchp,
  pages = 	 {185--288},
  location =  {Chicago},
}
\end{lstlisting}
\begin{lstlisting}[language=BibTeX,label=evanston:library]
*\adlnbackref{Online}{evanston:library}*,
  author = 	 {{Evanston Public Library Board of Trustees}},
  shortauthor = {{Evanston Public Library}},
  title = 	 {Evanston Public Library Strategic Plan, 2000--2010},
  subtitle = 	 {A Decade of Outreach},
  organization = {Evanston Public Library},
  url = 	 {http://www.epl.org/library/ strategic-plan-00.html},
  urldate = 	 {2002-07-18}
}
\end{lstlisting}
\begin{lstlisting}[language=BibTeX,label=friends:leia]
*\adlnbackref{Video}{friends:leia}*,
  title = 	 {The One with the {Princess Leia} Fantasy},
  date = 	 2003,
  booktitle = 	 {Friends},
  booktitleaddon = 	 {season~3, episode~1},
  author = 	 {Curtis, Michael and Malins, Gregory~S.},
  eventdate = 	 {1996-09-19},
  editor = 	 {Mancuso, Gail},
  editortype = 	 {director},
  publisher = {Warner Home Video},
  type = 	 {DVD},
  address = 	 {Burbank, CA}
}
\end{lstlisting}
\begin{lstlisting}[language=BibTeX,label=genesis]
*\adlnbackref{Misc}{genesis}*,
  shorttitle = 	 {Gen\adddot},
  entrysubtype = {classical},
  keywords = 	 {nosample},
  title = 	 {Genesis},
}
\end{lstlisting}
\begin{lstlisting}[language=BibTeX,label=gourmet:052006]
*\adlnbackref{Review}{gourmet:052006}*,
  journaltitle = {Gourmet},
  entrysubtype = {magazine},
  date = 	 {2000-05},
  title =        {Kitchen {Notebook}}
}
\end{lstlisting}
\begin{lstlisting}[language=BibTeX,label=grove:sibelius]
*\adlnbackref{InReference}{grove:sibelius}*,
  title = 	 {The New {Grove} Dictionary of Music and Musicians},
  author = 	 {Hepokoski, James},
  shorttitle = 	 {New {Grove} Dict\adddot},
  lista = {Sibelius, Jean},
  url = 	 {http://www.grovemusic.com/},
  urldate = 	 {2002-01-03},
  sortkey = 	 {New Grove}
}
\end{lstlisting}
\begin{lstlisting}[language=BibTeX,label=handel:messiah]
*\adlnbackref{Video}{handel:messiah}*,
  title = 	 {Messiah},
  date = 	 {1988},
  eventdate = 	 {1987-12-19},
  userd = 	 {performed},
  type = 	 {videocassette (VHS), 141 min\adddot},
  editor = 	 {{Atlanta Symphony Orchestra and Chamber Chorus}},
  editortype = 	 {none},
  editora = 	 {Shaw, Robert},
  editoratype =  {none},
  author = 	 {Handel, George Frederic},
  publisher = {Video Artists International},
  address = 	 {Ansonia Station, NY}
}
\end{lstlisting}
\begin{lstlisting}[language=BibTeX,label=hlatky:hrt]
*\adlnbackref{Article}{hlatky:hrt}*,
  author =	 {Hlatky, Mark~A. and Boothroyd, Derek and Vittinghoff, Eric and Sharp, Penny and Whooley, Mary~A.},
  title =	 {Quality-of-Life and Depressive Symptoms in Postmenopausal Women after Receiving Hormone Therapy},
  subtitle =	 {Results from the {Heart and Estrogen/Progestin Replacement Study (HERS)} Trial},
  journaltitle = {Journal of the American Medical Association},
  date = 	 {2002-02-06},
  volume =	 287,
  number =	 5,
  url =   {http://jama.ama-assn.org/issues/ v287n5/rfull/joc10108.html#aainfo},
  urldate =	 {2002-01-07}
}
\end{lstlisting}
\begin{lstlisting}[language=BibTeX,label=holiday:fool]
*\adlnbackref{Music}{holiday:fool}*,
  title = 	 {I'm a Fool to Want You},
  eventdate = 	 {1958-02-20},
  date = 	 {1960},
  booktitle = 	 {Lady in Satin},
  author = 	 {Herron, Joel and Sinatra, Frank and Wolf, Jack},
  editor = 	 {Holiday, Billie},
  editortype = 	 {none},
  number = 	 {CL 1157},
  publisher = {Columbia},
  type = 	 {33\onethird\ rpm},
  note = 	 {with Ray Ellis},
  options = 	 {useauthor=false}
}
\end{lstlisting}
\begin{lstlisting}[language=BibTeX,label=horowitz:youtube]
*\adlnbackref{Online}{horowitz:youtube}*,
  title = 	 {{HOROWITZ AT CARNEGIE HALL} 2-{Chopin Nocturne} in Fm Op.55},
  organization = {YouTube video, 5:53},
  sortkey = 	 {Horowitz},
  url = 	 {http://www.youtube.com/watch? v=cDVBtuWkMS8},
  urldate = 	 {2009-01-09},
  userd = 	 {posted by \mkbibquote{hubanj,}},
  note = 	 {from a performance televised by CBS on\nopunct},
  date = 	 {1968-09-22},
  shorttitle = {HOROWITZ}
}
\end{lstlisting}
\begin{lstlisting}[language=BibTeX,label=horsley:prosodies]
*\adlnbackref{Book}{horsley:prosodies}*,
  title =	 {On the Prosodies of the {Greek and Latin} Languages},
  year =	 1796,
  author =	 {Horsley, Samuel},
  authortype =	 {anon}
}
\end{lstlisting}
\begin{lstlisting}[language=BibTeX,label=house:papers]
*\adlnbackref{Misc}{house:papers}*,
  author = 	 {House, Edward~M\adddot\addcomma},
  title = 	 {Papers},
  note = 	 {Yale University Library},
  entrysubtype = {classical}
}
\end{lstlisting}
\begin{lstlisting}[language=BibTeX,label=iso:electrodoc]
*\adlnbackref{Book}{iso:electrodoc}*,
  title = 	 {Electronic Documents or Parts thereof. {Excerpts} from {International Standard ISO} 690-2},
  part = 	 {part 2},
  date = 	 2001,
  maintitle = 	 {Information and Documentation},
  mainsubtitle = {Bibliographic References},
  author = 	 {{International Organization for Standardization}},
  shorthand =  {ISO},
  publisher = {National Library of Canada},
  sortname = 	 {ISO},
  address = 	 {Ottawa},
  url = 	 {http://www.nlc-bnc.ca/iso/ tc46sc9/standard/690-2e.htm}
}
\end{lstlisting}
\begin{lstlisting}[language=BibTeX,label=james:ambassadors]
*\adlnbackref{Book}{james:ambassadors}*,
  title = 	 {The Ambassadors},
  year = 	 1996,
  origdate = 	 1909,
  options = 	 {cmsdate=on},
  author = 	 {James, Henry},
  publisher = {Project Gutenberg},
  url = 	 {ftp://ibiblio.org/pub/docs/ books/gutenberg/etext96/ambas10.txt}
}
\end{lstlisting}\clearpage
\begin{lstlisting}[language=BibTeX,label=lakeforester:pushcarts]
*\adlnbackref{Article}{lakeforester:pushcarts}*,
  journaltitle = {Lake Forester},
  date = 	 {2000-03-23},
  entrysubtype = {magazine},
  title = 	 {Pushcarts Evolve to Trendy Kiosks},
  options = 	 {cmsdate=full},
  location =  {Lake Forest, IL}
}
\end{lstlisting}
\begin{lstlisting}[language=BibTeX,label=loc:leaders]
*\adlnbackref{Online}{loc:leaders}[coolidge:speech]*,
  author = 	 {Library of Congress},
  title = 	 {American Leaders Speak},
  subtitle = 	 {Recordings from {World War I} and the 1920 Election, 1918--1920},
  url = 	 {http://memory.loc.gov/ammem/ nfhtml/nforSpeakers01.html},
  note = 	 {RealAudio and WAV formats}
}
\end{lstlisting}
\begin{lstlisting}[language=BibTeX,label=maitland:canon]
*\adlnbackref{Book}{maitland:canon}*,
  title = 	 {Roman canon law in the {Church of England}},
  date = 	 1998,
  origdate = 	 1898,
  author = 	 {Maitland, Frederic W.},
  publisher = {Lawbook Exchange},
  address = 	 {Union, NJ},
  options = 	 {cmsdate=new},
  pubstate = 	 {reprint}
}
\end{lstlisting}
\begin{lstlisting}[language=BibTeX,label=maitland:equity]
*\adlnbackref{Book}{maitland:equity}*,
  title = 	 {Equity, also the Forms of Action at Common Law},
  subtitle = 	 {Two Courses of Lectures},
  date = 	 1926,
  origdate = 	 1909,
  author = 	 {Maitland, Frederic W.},
  editor = 	 {Chaytor, A.~H. and others},
  publisher = cup,
  address = 	 {Cambridge},
  pubstate = 	 {reprint},
  sortyear = 	 {2010}
}
\end{lstlisting}
\begin{lstlisting}[language=BibTeX,label=nytrumpet:art]
*\adlnbackref{Music}{nytrumpet:art}*,
  title = 	 {Art of the Trumpet},
  date = 	 1982,
  origdate = 	 {1981-06-01/1981-06-02},
  author = 	 {{The New York Trumpet Ensemble, with Edward Carroll (trumpet) and Edward Brewer (organ)}},
  shortauthor =  {{New York Trumpet Ensemble}},
  number = 	 {PVT 7183},
  series = 	 {Vox/Turnabout},
  userd = 	 {recorded at the Madeira Festival,},
  sortkey = 	 {New York Trumpet},
  type = 	 {compact disc}
}
\end{lstlisting}
\begin{lstlisting}[language=BibTeX,label=nyt:trevorobit]
*\adlnbackref{Review}{nyt:trevorobit}*,
  journaltitle = {New York Times},
  entrysubtype = {magazine},
  date = 	 {2000-04-10},
  title = {obituary of {Claire Trevor}},
  options = 	 {cmsdate=full},
  pages = 	 {national edition}
}
\end{lstlisting}
\begin{lstlisting}[language=BibTeX,label=pirumova]
*\adlnbackref{Book}{pirumova}*,
  author =	 {Pirumova, N.~M.},
  title =	 {The Zemstvo Liberal Movement},
  subtitle =	 {Its Social Roots and Evolution to the Beginning of the Twentieth Century},
  publisher =	 {Izdatel'stvo \mkbibquote{Nauka}},
  year =	 1977,
  language =	 {russian},
  location =	 {Moscow}
}
\end{lstlisting}
\begin{lstlisting}[language=BibTeX,label=pirumova:russian]
*\adlnbackref{Book}{pirumova:russian}*,
  title = 	 {Zemskoe liberal'noe dvizhenie},
  subtitle = 	 {Sotsial'nye korni i evoliutsiia do nachala XX veka},
  date = 	 1977,
  usere = 	 {The zemstvo liberal movement: Its social roots and evolution to the beginning of the twentieth century},
  langid =       {russian},
  author = 	 {Pirumova, N.~M.},
  publisher = {Izdatel'stvo \mkbibquote{Nauka}},
  address = 	 {Moscow}
}
\end{lstlisting}
\begin{lstlisting}[language=BibTeX,label=plato:republic:gr]
*\adlnbackref{BookInBook}{plato:republic:gr}*,
  title = 	 {Republic},
  shorttitle = 	 {Resp\adddot},
  entrysubtype = {classical},
  year = 	 1902,
  volume = 	 4,
  author = 	 {Plato},
  editor = 	 {Burnet, J.},
  shortauthor =  {Pl\adddot},
  booktitle = 	 {{Clitophon, Republic, Timaeus, Critias}},
  maintitle = 	 {Opera},
  publisher = {Clarendon Press},
  series = {Oxford Classical Texts},
  pages = 	 {327--621},
  location =  {Oxford}
}
\end{lstlisting}
\begin{lstlisting}[language=BibTeX,label=pollan:plant]
*\adlnbackref{Online}{pollan:plant}*,
  author = 	 {Pollan, Michael},
  title = 	 {Michael {Pollan} Gives a Plant's-Eye View},
  organization = {TED video, 17:31},
  url = 	 {http://www.ted.com/index.php/ talks/michael_pollan_gives_a_ plant_s_eye_view.html},
  urldate = 	 {2008-02},
  date = 	 {2007-03},
  userd = 	 {posted}
}
\end{lstlisting}
\begin{lstlisting}[language=BibTeX,label=ross:thesis]
*\adlnbackref{MastersThesis}{ross:thesis}*,
  author = 	 {Ross, Dorothy},
  title = 	 {The {Irish-Catholic} Immigrant, 1880--1900},
  subtitle = 	 {A Study in Social Mobility},
  school = 	 {Columbia University},
  year = 	 {\bibstring{nodate}}
}
\end{lstlisting}
\begin{lstlisting}[language=BibTeX,label=schubert:muellerin]
*\adlnbackref{Audio}{schubert:muellerin}*,
  title = 	 {{Das Wandern (Wandering)}},
  date = 	 1895,
  booktitle = 	 {{Die sch\"one M\"ullerin} ({The} Maid of the Mill)},
  maintitleaddon = {(for high voice)},
  maintitle = 	 {First Vocal Album},
  options = 	 {ctitleaddon=space},
  author = 	 {Schubert, Franz},
  publisher = {G.~Schirmer},
  address = 	 {New York}
}
\end{lstlisting}
\begin{lstlisting}[language=BibTeX,label=schweitzer:bach]
*\adlnbackref{Book}{schweitzer:bach}*,
  title = 	 {{J. S. Bach}},
  origdate = 	 1966,
  date = 	 1911,
  author = 	 {Schweitzer, Albert},
  origlocation = {London},
  origpublisher = {Breitkopf \&\ H�rtel},
  addendum = 	 {Citations refer to the Dover edition},
  options = 	 {cmsdate=both},
  translator = 	 {Newman, Ernest},
  publisher = {Dover},
  pubstate = 	 {reprint},
  location =  {New York}
}
\end{lstlisting}
\begin{lstlisting}[language=BibTeX,label=shapey:partita]
*\adlnbackref{Misc}{shapey:partita}*,
  author = 	 {Shapey, Ralph},
  title = 	 {\mkbibquote{Partita for Violin and Thirteen Players}},
  titleaddon = 	 {score},
  entrysubtype = {music},
  date = 	 1966,
  note = 	 {Special Collections},
  organization = {Joseph Regenstein Library},
  institution =  {University of Chicago}
}
\end{lstlisting}
\begin{lstlisting}[language=BibTeX,label=silver:gawain]
*\adlnbackref{Book}{silver:gawain}*,
  title = 	 {Sir {Gawain} and the {Green Knight}},
  publisher = 	 uchp,
  year = 	 1974,
  translator = 	 {Silverstein, Theodore},
  location =  {Chicago}
}
\end{lstlisting}
\begin{lstlisting}[language=BibTeX,label=spock:interview]
*\adlnbackref{Misc}{spock:interview}*,
  author = 	 {Spock, Benjamin},
  entrysubtype = {letter},
  title = 	 {interview by Milton J. E. Senn},
  date = 	 {1974-11-20},
  note = 	 {interview 67A, transcript},
  organization = {Senn Oral History Collection},
  institution =  {National Library of Medicine},
  location =  {Bethesda, MD}
}
\end{lstlisting}
\begin{lstlisting}[language=BibTeX,label=stendhal:parma]
*\adlnbackref{Book}{stendhal:parma}*,
  title = 	 {The Charterhouse of {Parma}},
  date = 	 1925,
  author = 	 {Stendhal},
  nameaddon = 	 {Marie Henri Beyle},
  publisher = {Boni \& Liveright},
  address = 	 {New York},
  translator = 	 {Scott-Moncrieff, C.~K.}
}
\end{lstlisting}
\begin{lstlisting}[language=BibTeX,label=unsigned:ranke]
*\adlnbackref{Review}{unsigned:ranke}*,
  journaltitle = {Erg�nzungsbl�tter zur Allgemeinen Literatur-Zeitung},
  entrysubtype = {magazine},
  date = 	 {1828-02},
  title =	 {unsigned review of \mkbibemph{Geschichten der romanischen und germanischen V�lker}, by {Leopold von Ranke}},
  number =	 {23--24},
  sortkey = 	 {Erg},
  shortjournal =  {Erg\"anzungsbl\"atter z. Allg. Lit.-Ztg.}
}
\end{lstlisting}
\begin{lstlisting}[language=BibTeX,label=verdi:corsaro]
*\adlnbackref{Audio}{verdi:corsaro}*,
  title = 	 {Il corsaro (melodramma tragico \mkbibemph{in three acts})},
  titleaddon = 	 {libretto by Francesco Maria Piave},
  date = 	 1998,
  author = 	 {Verdi, Giuseppe},
  editor = 	 {Hudson, Elizabeth},
  number = 	 {\bibstring{jourser} 1, Operas},
  series = 	 {The Works of Giuseppe Verdi},
  publisher = {University of Chicago Press; Milan: G.\ Ricordi},
  volumes = 	 2,
  address = 	 {Chicago}
}
\end{lstlisting}\clearpage
\begin{lstlisting}[language=BibTeX,label=virginia:plantation]
*\adlnbackref{Book}{virginia:plantation}*,
  title =	 {A True and Sincere Declaration of the Purpose and Ends of the Plantation Begun in {Virginia}, of the Degrees Which It Hath Received, and Means by Which It Hath Been Advanced},
  location = 	 {London},
  sorttitle = 	 {True and Sincere},
  shorttitle = 	 {True and Sincere Declaration},
  year = 	 1610
}
\end{lstlisting}
\begin{lstlisting}[language=BibTeX,label=white:ross:memo]
*\adlnbackref{Letter}{white:ross:memo}*,
  author = 	 {White, E.~B.},
  title = 	 {EBW to Harold Ross},
  titleaddon = 	 {memorandum},
  xref = 	 {*\hyperlink{\getrefbykeydefault%
{white:total}{anchor}{}}{\{\colorbox{Gainsboro}{white:total}\}}*},
  pages = 	 273,
  origdate = 	 {1946-05-02}
}
\end{lstlisting}
\begin{lstlisting}[language=BibTeX,label=white:russ]
*\adlnbackref{Letter}{white:russ}*,
  author = 	 {White, E.~B.},
  title = 	 {EBW to B.~Russell},
  xref =  {*\hyperlink{\getrefbykeydefault%
{white:total}{anchor}{}}{\{\colorbox{Gainsboro}{white:total}\}}*},
  pages = 	 283,
  origdate = 	 {1946-09-02}
}
\end{lstlisting}
\begin{lstlisting}[language=BibTeX,label=white:total]
*\adlnbackref{Book}{white:total}*,
  title = 	 {{Letters of E.~B. White}},
  year = 	 1976,
  author = 	 {White, E.~B.},
  editor = 	 {Guth, Dorothy Lobrano},
  publisher = {Harper \&\ Row},
  location =  {New York}
}
\end{lstlisting}
\begin{lstlisting}[language=BibTeX,label=wikiped:bibtex]
*\adlnbackref{InReference}{wikiped:bibtex}*,
  title = 	 {Wikipedia},
  lista =        {BibTeX},
  userd = 	 {last modified},
  url = 	 {http://en.wikipedia.org/wiki/ BibTeX},
  urldate = 	 {2012-05-18}
}
\end{lstlisting}
\end{document}
%%% Local Variables: 
%%% mode: latex
%%% TeX-master: t
%%% End: 
