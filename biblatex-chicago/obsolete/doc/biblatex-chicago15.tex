%
% This file documents the biblatex-chicago package, which allows users
% of the biblatex package to format references according to the
% Chicago Manual of Style, 15th edition.
%
\documentclass[a4paper]{article}
\usepackage[T1]{fontenc}
\usepackage{textcomp}
\usepackage[latin1]{inputenc}
\usepackage[american]{babel}
\usepackage[autostyle]{csquotes}
\usepackage{vmargin}
\setpapersize{A4}
\setmarginsrb{1.65in}{.9in}{1.75in}{.6in}{0pt}{0pt}{12pt}{24pt}
\setlength{\marginparwidth}{1in}
\usepackage[colorlinks,urlcolor=blue,linkcolor=blue]{hyperref}
\usepackage[osf]{mathpazo}
\usepackage[scaled]{helvet}
\usepackage[pdftex]{xcolor}
%\usepackage[dvips]{xcolor}
\newcommand{\mycolor}[1]{\textcolor[HTML]{228B22}{#1}}
\usepackage{multicol}
% Some generic settings.
\newcommand{\cmd}[1]{\texttt{\textbackslash #1}}
\setlength{\parindent}{0pt}
\newcommand{\mymarginpar}[1]{\marginpar{\flushright#1}}
\newcommand{\colmarginpar}[1]{\mymarginpar{\mycolor{#1}}}
\newcommand{\mybigspace}{\vspace{\baselineskip}}
\newcommand{\mylittlespace}{\vspace{.5\baselineskip}}
\makeatletter
\renewcommand{\section}{\@startsection
  {section}%
  {1}%
  {0mm}%
  {\baselineskip}%
  {\baselineskip}%
  {\sffamily\normalsize\bfseries}}%
\renewcommand{\subsection}{\@startsection
  {subsection}%
  {1}%
  {0mm}%
  {\baselineskip}%
  {.5\baselineskip}%
  {\sffamily\normalsize\bfseries}}%
\renewcommand{\subsubsection}{\@startsection
  {subsubsection}%
  {1}%
  {0mm}%
  {\baselineskip}%
  {.5\baselineskip}%
  {\sffamily\normalsize\bfseries}}%
\makeatother
\begin{document}
\begin{center}
  \sffamily\large\bfseries The biblatex-chicago package: \\
  15th-edition style files for biblatex

\vspace{.3\baselineskip}
\sffamily\normalsize\bfseries David Fussner\qquad Version 0.9.9g (beta) \\
\href{mailto:djf027@googlemail.com}{djf027@googlemail.com}\\ \today

\end{center} 
\setcounter{tocdepth}{3}
\begin{multicols}{2}
\footnotesize
\tableofcontents
\end{multicols}
\normalsize
\vspace{-.5\baselineskip}
\section{Notice}
\label{sec:Notice}

\textbf{This file documents the \textsf{biblatex} style files
  implementing the 15th edition of \emph{The Chicago Manual of Style}.
  This edition has been superseded by the 16th edition, documented in
  \textsf{biblatex-chicago.pdf}.  There may still be some users,
  however, who wish to continue using the 15th-edition styles, so I am
  continuing to provide them, though I am now marking them as
  \enquote{strongly deprecated,} and will soon mark them as
  \enquote{obsolete.}  I have included very few fixes in this release
  cycle, enough so that my test files do still compile correctly.  The
  main reason to continue using these files is perhaps a desire not to
  interrupt a work-in-progress.  Given how little attention they are
  currently receiving, however, I encourage all users of both the
  \textsf{authordate15} and \textsf{notes15} styles to upgrade to the
  current release of the 16th-edition styles as soon as may be
  practicable.  If the changes to the author-date style's treatment of
  titles has thus far discouraged you from upgrading, please be aware
  that the new \mycolor{\textsf{authordate-trad}} style retains the
  traditional Chicago sentence-style capitalization (and absence of
  quotation marks) in titles while in all other respects adhering to
  the 16th-edition specification.  I can't promise that switching to a
  newer style will be entirely painless, but if you read through the
  RELEASE file you'll find that it won't, under ordinary
  circumstances, require Herculean efforts.  \mylittlespace \\ The
  package remains beta software.  If it seems like it could be of use
  to you, yet it doesn't do something you need/want it to do, please
  feel free to let me know, and of course any suggestions for solving
  problems more elegantly or accurately would be most welcome.}

\mylittlespace\textbf{Important Note:} If you have used
\textsf{biblatex-chicago} before, please make sure you have read the
RELEASE file that came with the package.  It details the changes
you'll need to make to your .bib database in order for it to work
properly with this release.  If you are new to these styles, please
read on.

\section{Quickstart}
\reversemarginpar

%\enlargethispage{-3\baselineskip}

The \textsf{biblatex-chicago} package is designed for writers who wish
to use \LaTeX\ and \textsf{biblatex}, and who either want or need to
format their references according to one of the specifications defined
by the \emph{Chicago Manual of Style}.  This package includes both the
\emph{Manual's} \enquote{author-date} system, favored by many
disciplines in the sciences and social sciences, and its
\enquote{notes \&\ bibliography} style, generally favored in the
humanities.  The latter code produces a full reference in a first
footnote, shorter references in subsequent notes, and a full reference
in the bibliography. Some authors prefer to use the shorter note form
even for the first occurrence, relying on the bibliography to provide
the full information.  This, too, is supported by the code.  The
author-date style produces a short, in-text citation inside
parentheses --- (Author Year) --- keyed to a list of references where
entries start with the same name and year.

\mylittlespace The documentation you are reading covers both of these
Chicago styles and their variants.  Much of what follows is relevant
to all users, but I have decided, after some experimentation, to keep
the instructions for the two styles separate, at least in
sections~\ref{sec:Spec} and \ref{sec:authdate}.  Information provided
under one style will often duplicate that found under the other, but
efficiency's loss should, I hope, be clarity's gain, and much of what
you learn using one style will be applicable without alteration to the
other.  Throughout the documentation, any \mycolor{green} text
\colmarginpar{\textsf{New!}} indicates something \mycolor{new} in this
release.

\mylittlespace Here's a list of things you will need in order to use
\textsf{biblatex-chicago}:

\begin{itemize}{}{}
\item Philipp Lehman's \textsf{biblatex} package, of course!  You
  should use the latest version(s) --- 1.7 or 2.9a --- as those
  versions have been tested more thoroughly than any other, meaning
  that these style files may well not function properly with earlier
  iterations of \textsf{biblatex}. Lehman's tools require several
  packages, and he strongly recommends several more:
  \begin{itemize}{}{}
  \item e-\TeX\ (required)
  \item \textsf{etoolbox} --- available from CTAN (required)
  \item \textsf{keyval} --- a standard package (required)
  \item \textsf{ifthen} --- a standard package (required)
  \item \textsf{url} --- a standard package (required)
  \item \textsf{babel} --- a standard package (\emph{strongly}
    recommended)
  \item \textsf{csquotes} --- available from CTAN (recommended).
    Please upgrade to the latest version of \textsf{csquotes} (5.1b).
  \item \textsf{bibtex8} --- a replacement for \textsc{Bib}\TeX, which
    can, with the right com\-mand-line switches, process very large
    .bib files.  It also does the right thing when alphabetizing
    non-ASCII entries.  It is available from CTAN, but please be aware
    that this database parser no longer suffices if you are using the
    Chicago author-date style with any version of \textsf{biblatex}
    from version 1.5 onwards.  For that style you must use the
    following:
  \item \textsf{Biber} --- the next-generation \textsc{Bib}\TeX\
    replacement, which is available from SourceForge.  You should use
    the latest version, 1.9, to work with \textsf{biblatex} 2.9a, and
    it is required for users who are either using the author-date
    style or processing a .bib file in Unicode.  See
    \textsf{cms15-dates-sample.pdf} for more details.
  \end{itemize}
\item The line:
  \begin{quote}
    \cmd{usepackage[\mycolor{notes15}]\{biblatex-chicago\}}
  \end{quote}
  in your document preamble to load the 15th-edition notes \&\
  bibliography style, or the line:
  \begin{quote}
    \cmd{usepackage[\mycolor{authordate15},backend=biber]\{biblatex-chicago\}}
  \end{quote}
  to load the 15th-edition author-date style.  Any other options you
  usually pass to \textsf{biblatex} can be given to
  \textsf{biblatex-chicago} instead, but loading it this way sets up a
  number of other parameters automatically.  You can also load the
  package via the traditional \cmd{usepackage\{biblatex\}}, adding
  \mycolor{\texttt{style=chicago-notes15}} or
  \mycolor{\texttt{style=chicago-authordate15}}, but these are mainly
  for those who wish to set much of the low-level formatting of their
  document themselves.  Please see sections~\ref{sec:loading} and
  \ref{sec:loading:auth} below for a fuller discussion of the issues
  involved here.
\item You can use
  \cmd{usepackage[\mycolor{notes15},short]\{biblatex-chicago\}} to get
  the short note format even in the first reference of a notes \&\
  bibliography document, letting the bibliography provide the full
  reference.
\item If you are accustomed to using the \textsf{natbib} compatibility
  option with \textsf{biblatex}, then you can continue to do so with
  \textsf{biblatex-chicago}.  If you are using
  \cmd{usepackage\{biblatex-chicago\}} to load the package, then the
  option must be the plain \texttt{natbib} rather than
  \texttt{natbib=true}.  If you use the latter, you'll get a
  \textsf{keyval} error.  Please see sections~\ref{sec:useropts} and
  \ref{sec:authuseropts}, below.
\item By far the simplest setup is to use \textsf{babel}, and to have
  \texttt{american} as the main text language.  As before,
  \textsf{babel}-less setups, and also those choosing \texttt{english}
  as the main text language, should work out of the box.
  \textsf{Biblatex-chicago} also now provides support for German and
  French.  Please see below (section~\ref{sec:international}) for a
  fuller explanation of all the options.
\item The \mycolor{\textsf{chicago-notes15.bbx}},
  \mycolor{\textsf{chicago-notes15.cbx}},
  \mycolor{\textsf{chicago-authordate15.cbx}},
  \mycolor{\textsf{chi\-cago-authordate15.bbx}},
  \textsf{biblatex-chicago.sty}, \textsf{cms-am\-erican.lbx},
  \textsf{cms-french.lbx}, \textsf{cms-german.lbx}, and
  \textsf{cms-ngerman.lbx} files from \textsf{biblatex-chicago},
  installed either in a system-wide \TeX\ directory, or in the working
  directory where you keep your *.tex files.  The .zip file from CTAN
  contains several subdirectories to help keep the growing number of
  files organized, so the files listed above can be found in the
  \texttt{latex/} subdirectory, itself further divided into the
  \texttt{bbx/}, \texttt{cbx/}, and \texttt{lbx/} subdirectories.  If
  you install in a system-wide directory, I recommend following
  standard advice and using
  \texttt{<TEXMFLOCAL>/tex/latex/biblatex-contrib/biblatex-chicago},
  where \texttt{<TEXMFLOCAL>} is the root of your local \TeX\
  installation --- for example, and depending on your system and
  preferences, \texttt{/usr/share/texmf-local},
  \texttt{/usr/local/share/texmf}, or \texttt{C:\textbackslash{}Local
    TeX Files\textbackslash}.  Then you can copy the contents of the
  \texttt{latex/} directory there, subdirectories and all.  (If you
  install into your working directory, then you'll need to copy the
  files directly there, without subdirectories.)  Of course, if you
  choose to place them anywhere in the \texttt{texmf} tree, you'll
  need to update the file name database to make sure \TeX\ can find
  them.
\item Philipp Lehman's very clear and detailed documentation of the
  \textsf{biblatex} system, available in his package as
  \textsf{biblatex.pdf}.  Here he explains why you might want to use
  the system, the rules for constructing .bib files for it, and the
  (numerous) methods at your disposal for modifying the formatted
  output.
\item The annotated bibliography files \textsf{notes-test.bib} and
  \textsf{dates-test.bib}, which will acquaint you with most of the
  details on how to get started constructing your own .bib files for
  use with the two \textsf{biblatex-chicago} styles.
\item The files \textsf{cms15-notes-sample.pdf} and
  \textsf{cms15-dates-sample.pdf}.  The first shows how my system
  processes \textsf{notes-test.bib} and
  \textsf{cms15-notes-sample.tex}, in both footnotes and bibliography,
  the second is the result of processing \textsf{dates-test.bib} and
  \textsf{cms15-dates-sample.tex}.  All of these files are in
  \texttt{doc/examples/}.
\item The file you are reading, \textsf{biblatex-chicago15.pdf}, which
  aims to be as complete a description as possible of the rules for
  creating a .bib file that will, when processed by \LaTeX\ and
  \textsc{Bib}\TeX, at least somewhat ease the burden when you try to
  implement the \emph{Chicago Manual of Style}'s specifications.
  These docs may seem frustratingly over-long, but remember that you
  only need to read the part(s) that apply to the style in which you
  are interested.  Much of the information in section~\ref{sec:Spec}
  is duplicated in section~\ref{sec:authdate}, so even if you have a
  need for both styles then using one will be excellent preparation
  for the other.  If you have used a previous version of this package,
  please pay particular attention to the sections on Obsolete and
  Deprecated Features, starting on page~\pageref{deprec:obsol}.  You
  will find the seven previous files in the \texttt{doc/} subdirectory
  once you've extracted \textsf{biblatex-chicago.zip}.  If you wish to
  place them in a system-wide directory, I would recommend
  \texttt{<TEXMFLOCAL>/doc/latex/biblatex-contrib/biblatex-chicago},
  all the while remembering, of course, to update the file name
  database afterward.  (Let me reiterate, also, that if you currently
  have quoted material in your .bib file, and are using \cmd{enquote}
  or the standard \LaTeX\ mechanisms there, then the simplest
  procedure is always to use \cmd{mkbibquote} instead in order to
  ensure that punctuation works out right.)
\item Access to a copy of \emph{The Chicago Manual of Style} itself,
  which naturally contains incomparably more information than I can
  hope to present here.  It should always be your first port of call
  when any doubts arise as to exactly what the specification requires.
\end{itemize}

\subsection{License}
\label{sec:lppl}

Copyright � 2008--2014 David Fussner.  This package is
author-maintained.  This work may be copied, distributed and/or
modified under the conditions of the \LaTeX\ Project Public License,
either version 1.3 of this license or (at your option) any later
version.  The latest version of this license is in
http://www.latex-project.org/lppl.txt and version 1.3 or later is part
of all distributions of \LaTeX\ version 2005/12/01 or later.  This
software is provided \enquote{as is,} without warranty of any kind,
either expressed or implied, including, but not limited to, the
implied warranties of merchantability and fitness for a particular
purpose.

%\enlargethispage{-\baselineskip}

\subsection{Acknowledgements}
\label{sec:acknowl}

Even a cursory glance at the cbx and bbx files in the package will
demonstrate how much of Lehman's code from \textsf{biblatex} I've
adapted and re-used, and I've also followed some of the advice he gave
to others in the \texttt{comp.text.tex} newsgroup.  He has been
instrumental in improving the contextual capitalization procedures of
which the style makes such frequent use, and his advice on
constructing \textsf{biblatex-chicago.sty} was invaluable.  The code
for formatting the footnote marks, and that for printing the
separating rule only after a run-on note, I've adapted from the
\textsf{footmisc} package by Robin Fairbairns, and I've borrowed ideas
for the \texttt{shorthandibid} option from Dominik Wa�enhoven's
\textsf{biblatex-dw} package.  I am very grateful to Baldur
Kristinsson for providing the Icelandic localization, and to H�kon
Malmedal for the Norwegian localizations.  Kazuo Teramoto and Gildas
Hamel both sent patches to improve the package, and there may be other
\LaTeX\ code I've appropriated and forgotten, in which case please
remind me.  Finally, Charles Schaum and Joseph Reagle Jr.\ were both
extremely generous with their help and advice during the development
of this package, and have both continued indefatigably to test it and
suggest needed improvements.  They were particularly instrumental in
encouraging the greatest possible degree of compatibility with other
\textsf{biblatex} styles.  Indeed, if the task of adapting .bib files
for use with the Chicago style seems onerous now, you should have
tried it before they got their hands on it.

\section{Detailed Introduction}
\label{sec:Intro}

The \emph{Chicago Manual of Style}, implemented here in its 15th
edition, has long, in America at least, been one of the most
influential style guides for writers and publishers.  While one's
choices are now perhaps more extensive than ever, the \emph{Manual} at
least still provides a widely-recognized, and widely-utilized,
standard.  Indeed, when you add to this the sheer completeness of the
specification, its detailed instructions for referencing an enormous
number of different kinds of source material, then your choice (or
your publisher's choice) of the \emph{Manual} as a style guide seems
set to be a happy one.

\mylittlespace These very strengths, however, also make the style
difficult to use.  Admittedly, the \emph{Manual} does leave room for
\enquote{inventive solutions} to particular problems (17.2), and it
also emphasizes consistency within a work, as opposed to rigid
adherence to the specification, at least when writer and publisher
agree (17.18).  Sometimes a publisher demands such adherence, however,
and anyone who has attempted to produce it may well come away with the
impression that the specification itself is somewhat idiosyncratic in
its complexity, and I can't help but agree.  In the notes \&\
bibliography style, the numerous differences in punctuation (and
strings identifying translators, editors, and the like) between
footnotes and bibliographies and the sometimes unusual location of
page numbers; in both styles the distinction between \enquote{journal}
and \enquote{magazine,} and the formatting differences between (e.g.)
a work from antiquity and one from the Renaissance, all of these tend
to overburden the writer who wants to comply with the standard.  Many
of these complexities, in truth, make the specification very nearly
impossible to implement straightforwardly in a system like
\textsc{Bib}\TeX\ --- options multiply, each requiring a particular
sort of formatting, until one almost reaches the point of believing
that every individual book or article should have its own entry type.
Completeness and usability tend each to exclude the other, so the code
you have before you is a first attempt to achieve the former without
utterly sacrificing the latter.

\subsection*{What \textsf{biblatex-chicago} can and can't do}
\label{sec:bltries}

In short, the \textsf{biblatex} style files in this package try to
simplify the task of following the two Chicago specifications.  In the
notes \&\ bibliography style, the two sorts of reference are treated
separately (as are the two different note forms, long and short), and
you can choose always to use the short note form, even at the first
citation.  In the author-date style, a series of options allows you to
choose which date (original printing, reprint, or both) appears in
citations and at the head of entries in the list of references.  In
both styles, punctuation is placed within quotation marks when needed,
and as a general rule as many parts of the style as possible are
implemented as transparently as possible.  Thanks to advice I received
from Joseph Reagle Jr.\ and Charles Schaum while these files were a
work in progress, I have attended as carefully as I can to backward
compatibility with the standard \textsf{biblatex} styles, and have
attempted to minimize both any changes you need to make to achieve
compliance with the Chicago specification, and indeed also any changes
necessary to switch between the two Chicago styles.  There is no doubt
room for improvement on this score, but even now, for a substantial
number of entries, any well-constructed .bib file that works for other
\textsf{biblatex} styles will \enquote{just work} under
\textsf{biblatex-chicago}.  By no means, however, will all entries in
such a .bib file produce equally satisfactory results.  Using this
documentation and the examples in \textsf{dates-test.bib} and/or
\textsf{notes-test.bib}, it should be possible to achieve compliance,
though the amount of revision necessary to do so will vary
significantly from .bib file to .bib file.  Conversely, once you have
created a database for \textsf{biblatex-chicago}, it won't necessarily
work well with other \textsf{biblatex} styles.  Indeed, most, quite
possibly all, users will find that they need to use special formatting
macros within the .bib file that would make such a file unusable in
any other context.  I strongly recommend, if you want to experiment
with this style, that you work on a copy of any .bib files that are
important to you, until you have determined that this package does
what you need/want it to do.

\mylittlespace When I first began working on this package, I made the
decision to alter as little as possible the main files from Lehman's
\textsf{biblatex}, so that my .bbx and .cbx files would use his
original \LaTeX\ .sty file and \textsc{Bib}\TeX\ .bst file.  As you
proceed, you will no doubt encounter some of the consequences of this
decision, with certain fields and entry types in the .bib file having
less-than-memorable names because I chose to use the supplementary
ones provided by \textsf{biblatex.bst} rather than alter that file.  I
intended then, if it turned out that anyone besides myself actually
used \textsf{biblatex-chicago}, to ask Mr.\ Lehman to include more
descriptive names for these few entry types and fields in
biblatex.bst, if he were willing.  As luck would have it, several new
types appeared in \textsf{biblatex} 0.8, many of which I have
incorporated as replacements for the custom entry types I defined
before.  If a consensus emerges about how best to assign the data to
various fields in such entries, then I shall adopt it.  In the
meantime, as you will see below, I have made two of the old custom
types obsolete, and recycled the third for an entirely new purpose.
Needless to say, I'm open to advice and suggestions on this score.

\section{The Specification:\ Notes\,\&\,Bibliography}
\label{sec:Spec}

In what follows, I attempt to explain all the parts of
\textsf{biblatex-chicago-notes} that might be considered somehow
\enquote{non standard,} at least with respect to the styles included
with \textsf{biblatex} itself, though in the section on entry fields I
have also duplicated a lot of the information in
\textsf{biblatex.pdf}, which I hope won't badly annoy expert users of
the system.  Headings in \mycolor{green} \colmarginpar{\textsf{New in
    this release}} indicate material new to this release, or
occasionally old material that has undergone significant revision.
Numbers in parentheses refer to sections of the \emph{Chicago Manual
  of Style}, 15th edition.  The file \textsf{notes-test.bib} contains
many examples from the \emph{Manual} which, when processed using
\textsf{biblatex-chicago-notes}, should produce the same output as you
see in the \emph{Manual} itself, or at least compliant output, where
the specifications are vague or open to interpretation, a state of
affairs which does sometimes occur.  I have provided
\textsf{cms-notes-sample.pdf}, which shows how my system processes
\textsf{notes-test.bib}, and I have also included the reference keys
from the latter file below in parentheses.

%\enlargethispage{\baselineskip}

\subsection{Entry Types}
\label{sec:entrytypes}

The complete list of entry types currently available in
\textsf{biblatex-chicago-notes}, minus the odd \textsf{biblatex}
alias, is as follows: \mycolor{\textbf{article}}, \textbf{artwork},
\textbf{audio}, \textbf{book}, \textbf{bookinbook}, \textbf{booklet},
\textbf{collection}, \textbf{customc}, \textbf{image},
\textbf{inbook}, \textbf{incollection}, \textbf{inproceedings},
\textbf{inreference}, \textbf{letter}, \textbf{manual},\textbf{misc},
\textbf{music}, \textbf{online} (with its alias \textbf{www}),
\textbf{patent}, \textbf{periodical}, \textbf{proceedings},
\textbf{reference}, \textbf{report} (with its alias
\textbf{techreport}), \mycolor{\textbf{review}}, \textbf{suppbook},
\textbf{suppcollection}, \textbf{suppperiodical}, \textbf{thesis}
(with its aliases \textbf{mastersthesis} and \textbf{phdthesis}),
\textbf{unpublished}, and \textbf{video}.

\mylittlespace What follows is an attempt to specify all the
differences between these types and the standard provided by
\textsf{biblatex}.  If an entry type isn't discussed here, then it is
safe to assume that it works as it does in the standard styles.  In
general, I have attempted not to discuss specific entry fields here,
unless such a field is crucial to the overall operation of a given
entry type.  As a general and important rule, most entry types require
very few fields when you use \textsf{biblatex-chicago-notes}, so it
seemed to me better to gather information pertaining to fields in the
next section.

\mybigspace The \colmarginpar{\textbf{article}} \emph{Chicago Manual of
  Style} (17.148) recognizes three different sorts of periodical
publication, \enquote{journals,} \enquote{magazines,} and
\enquote{newspapers.} The first (17.150) includes \enquote{scholarly
  or professional periodicals available mainly by subscription,} while
the second refers to \enquote{weekly or monthly} publications that are
\enquote{available either by subscription or in individual issues at
  bookstores or newsstands.}  \enquote{Magazines} will tend to be
\enquote{more accessible to general readers,} and typically won't have
a volume number.  Indeed, by fiat I declare that should you need to
refer to a journal that identifies its issues mainly by year, month,
or week, then for the purposes of \textsf{biblatex-chicago-notes} such
a publication is a \enquote{magazine,} and not a \enquote{journal.}

\mylittlespace Now, for articles in \enquote{journals} you can simply
use the traditional \textsc{Bib}\TeX\ --- and indeed \textsf{biblatex}
--- \textsf{article} entry type, which will work as expected and set
off the page numbers with a colon, as required by the \emph{Manual}.
If, however, you need to refer to a \enquote{magazine} or a
\enquote{newspaper,} then you need to add an \textsf{entrysubtype}
field containing the exact string \texttt{magazine}.  The main
formatting differences between a \texttt{magazine} (which includes
both \enquote{magazines} and \enquote{newspapers}) and a plain
\textsf{article} are that the year isn't placed within parentheses,
and that page numbers are set off by a comma rather than a colon.
Otherwise, the two sorts of reference have much in common.  (For
\textsf{article}, see \emph{Manual} 17.154--181; batson,
beattie:crime, friedman:learning, garaud:gatine, garrett, hlatky:hrt,
kern, lewis, loften:hamlet, mcmillen:antebellum, warr:ellison,
white:callimachus. For \textsf{entrysubtype} \texttt{magazine}, see
17.166, 17.182--198; assocpress:gun, morgenson:market, reaves:rosen,
rozner:liberation, stenger:privacy.)

\mylittlespace It gets worse.  The \emph{Manual} treats reviews (of
books, plays, performances, etc.) as a sort of recognizable subset of
\enquote{journals,} \enquote{magazines,} and \enquote{newspapers,}
distinguished mainly by the way one formats the title of the review
itself.  In \textsf{biblatex 0.7}, happily, Lehman provided a
\textsf{review} entry type which will handle a large subset of such
citations, though not all.  The key rule is this: if a review has a
separate, non-generic title (gibbard; osborne:poison) in addition to
something that reads like \enquote{review of \ldots,} then you need an
\textsf{article} entry, with or without the \texttt{magazine}
\textsf{entrysubtype}, depending on the sort of publication containing
the review.  If the only title is the generic \enquote{review of
  \ldots,} for example, then you'll need the \textsf{review} entry
type, with or without this same \textsf{entrysubtype} toggle using
\texttt{magazine}.  On \textsf{review} entries, see below.  (The
curious reader will no doubt notice that the code for formatting any
sort of review still exists in \textsf{article}, as it was initially
designed for \textsf{biblatex 0.6}, but this new arrangement is
somewhat simpler and therefore, I hope, better.)

%\enlargethispage{-\baselineskip}

\mylittlespace In the case of a review with a specific as well as a
generic title, the former goes in the \textsf{title} field, and the
latter in the \textsf{titleaddon} field.  Standard \textsf{biblatex}
intends this field for use with additions to titles that may need to
be formatted differently from the titles themselves, and
\textsf{biblatex-chicago-notes} uses it in just this way, with the
additional wrinkle that it can, if needed, replace the \textsf{title}
entirely, and this in, effectively, any entry type, providing a fairly
powerful, if somewhat complicated, tool for getting \textsc{Bib}\TeX\
to do what you want.  Here, however, if all you need is a
\textsf{titleaddon}, then you want to switch to the \textsf{review}
type, where you can simply use the \textsf{title} field instead.

\mylittlespace No less than six more things need explication here.
First, since the \emph{Manual} specifies that much of what goes into a
\textsf{titleaddon} field stays unformatted --- no italics, no
quotation marks --- this plain style is the default for such text,
which means that you'll have to format any titles within
\textsf{titleaddon} yourself, e.g., with \cmd{mkbibemph\{\}}.  Second,
the \emph{Manual} specifies a similar plain style for the titles of
other sorts of material found in \enquote{magazines} and
\enquote{newspapers,} e.g., obituaries, letters to the editor,
interviews, the names of regular columns, and the like.  References
may contain both the title of an individual article and the name of
the regular column, in which case the former should go, as usual, in a
\textsf{title} field, and the latter in \textsf{titleaddon}.  As with
reviews proper, if there is only the generic title, then you want the
\textsf{review} entry type.  (See 17.188, 17.190, 17.193;
morgenson:market, reaves:rosen.)

\mylittlespace Third, the \emph{Manual} suggests that, in the case of
\enquote{unsigned newspaper articles or features \ldots the name of
  the newspaper stands in place of the author} (17.192).  It doesn't
always carry through on this in its own presentation of newspaper
citations (see esp.\ 17.188), but I've implemented their
recommendation nonetheless, which means that in an \textsf{article}
entry, \textsf{entrysubtype} \texttt{magazine}, or in a
\textsf{review} entry, \textsf{entrysubtype} \texttt{magazine}, and
only in such entries, a missing \textsf{author} field results in the
name of the periodical (in the \textsf{journaltitle} field) being used
as the missing author.  If, for reasons of emphasis or merely because
of personal preference, you wish to keep the \textsf{title} in initial
position, then you need to define the \textsf{author} using,
effectively, anything at all, then set \texttt{useauthor=false} in the
\textsf{options} field.  (The \cmd{isdot} macro in the \textsf{author}
field no longer works on its own under \textsf{biblatex 1.6} and
later, so you may need to change your .bib files when you upgrade.)
Note that if you choose to use the name of the newspaper as an author,
then you'll need a \textsf{sortkey} field to ensure that the
bibliography entry is alphabetized by \textsf{journaltitle} rather
than by \textsf{title}.  The provision of a \textsf{shortauthor} field
is no longer necessary, as in its absence the package automatically
takes it from \textsf{journaltitle}.  (See lakeforester:pushcarts.)

\enlargethispage{\baselineskip}

\mylittlespace Fourth, if you've been using
\textsf{biblatex-chicago-notes} for a while, you may remember using
the single-letter \cmd{bibstring} mechanism in order to help
\textsf{biblatex} decide where to capitalize a wide variety of strings
in numerous entry fields.  This mechanism was particularly common in
all the periodical types, but if you've had a look in
\textsf{notes-test.bib} while following this documentation, you'll
have noticed that it no longer appears there.  The regular whole-word
bibstrings still work as normal, but the single-letter ones are now
obsolete, replaced by Lehman's macro \cmd{autocap}, which itself only
occurs twice in \textsf{notes-test.bib}.  Basically, in certain
fields, just beginning your data with a lowercase letter activates the
mechanism for capitalizing that letter depending on its context within
a note or bibliography entry.  Please see \textbf{\textbackslash
  autocap} below for the details, but both the \textsf{titleaddon} and
\textsf{note} fields are among those treating their data this way, and
since both appear regularly in \textsf{article} entries, I thought the
problem merited a preliminary mention here.

\mylittlespace Fifth, if you need to cite an entire issue of any sort
of periodical, rather than one article in an issue, then the
\textsf{periodical} entry type, once again with or without the
\texttt{magazine} toggle in \textsf{entrysubtype}, is what you'll
need.  (You can also use the \textsf{article} type, placing what would
normally be the \textsf{issuetitle} in the \textsf{title} field and
retaining the usual \textsf{journaltitle} field, but this arrangement
isn't compatible with standard \textsf{biblatex}.)  The \textsf{note}
field is where you place something like \enquote{special issue} (with
the small \enquote{s} enabling the automatic capitalization routines),
whether you are citing one article or the whole issue
(conley:fifthgrade, good:wholeissue).  Indeed, this is a somewhat
specialized use of \textsf{note}, and if you have other sorts of
information you need to include in an \textsf{article},
\textsf{periodical}, or \textsf{review} entry, then you shouldn't put
it in the \textsf{note} field, but rather in \textsf{titleaddon} or
perhaps \textsf{addendum} (brown:bremer).

\mylittlespace Finally, and in the interests of completeness, it may be
as well to suggest that if you wish to cite a television or radio
broadcast, the \textsf{article} type, \textsf{entrysubtype}
\texttt{magazine} is the place for it.  The name of the program would
go in \textsf{journaltitle}, with the name of the episode in
\textsf{title}.  The network's name now goes into the new
\textsf{usera} field, replacing the formatting kludge I suggested in
version 0.7.  Of course, if the piece you are citing has only a
generic name (an interview, for example), then the \textsf{review}
type would be the best place for it.  (8.196, 17.207; see
bundy:macneil for an example of how this all might look in a .bib
file.)

%\enlargethispage{\baselineskip}

\mylittlespace If you're still with me, allow me to recommend that you
browse through \textsf{notes-test.bib} to get a feel for just how many
of the \emph{Manual}'s complexities the \textsf{article} and
\textsf{review} (and, indeed, \textsf{periodical}) types attempt to
address.  It may be that in future releases of
\textsf{biblatex-chicago-notes} I'll be able to simplify these
procedures somewhat, but in the meantime it might be of some comfort
that I have found in my own research that the unusual and/or limit
cases are really rather rare, and that the vast majority of sources
won't require any knowledge of these onerous details.

\mybigspace Arne \mymarginpar{\textbf{artwork}} Kjell Vikhagen has
pointed out to me that none of the standard entry types were
straightforwardly adaptable when referring to visual artworks.  The
\emph{Manual} doesn't give any thorough specifications for such
references, and indeed it's unclear that it believes it necessary to
include them in the bibliographical apparatus at all.  Still, it's
easy to conceive of contexts in which a list of artworks studied might
be desirable, and \textsf{biblatex} includes entry types for just this
purpose, though the standard styles leave them undefined.  The two I
have included in this release are \textsf{artwork} and \textsf{image},
the former intended for paintings, sculptures, etchings, and the like,
the latter for photographs.  The two entry types work in exactly the
same way as far as constructing your .bib entry, and when printed the
only difference will be that the titles of \textsf{artworks} are
italicized, those of \textsf{images} placed within quotation marks.

\mylittlespace As one might expect, the artist goes in \textsf{author}
and the name of the work in \textsf{title}.  The \textsf{type} field
is intended for the medium --- e.g., oil on canvas, charcoal on paper
--- and the \textsf{version} field might contain the state of an
etching.  You can place the dimensions of the work in \textsf{note},
and the current location in \textsf{organization},
\textsf{institution}, and/or \textsf{location}, in ascending order of
generality.  The \textsf{type} field, as in several other entry types,
uses \textsf{biblatex's} automatic capitalization routines, so if the
first word only needs a capital letter at the beginning of a sentence,
use lowercase in the .bib file and let \textsf{biblatex} handle it for
you.  (See \emph{Manual} 12.33; leo:madonna, bedford:photo.)

\mylittlespace As a final complication, the \emph{Manual} (8.206) says
that \enquote{the names of works of antiquity \ldots\,are usually set
  in roman.}  If you should need to include such a work in the
reference apparatus, you can either define an \textsf{entrysubtype}
for an \textsf{artwork} entry --- anything will do --- or you could
use the \textsf{misc} entry type with an \textsf{entrysubtype}.
Fortunately, in this instance the other fields in a \textsf{misc}
entry function pretty much as in \textsf{artwork} or \textsf{image}.

\mybigspace For \mymarginpar{\textbf{audio}} this release of
\textsf{biblatex-chicago}, following the request of Johan Nordstrom, I
have included three new entry types, all undefined by the standard
styles, designed to allow users to present audiovisual sources in
accordance with the Chicago specifications.  The \emph{Manual's}
presentation of such sources (17.263--273), though admirably brief,
seems to me somewhat inconsistent.  I attempted to condense all the
requirements into two new entry types, but ended up relying on three,
the differences between which I shall attempt to delineate here.
There are likely to be occasions when your choice of entry type is not
obvious, but at the very least \textsf{biblatex-chicago} should help
you maintain consistency.

\mylittlespace The \textbf{music} type is intended for all musical
recordings that do not have a video component.  This means, for
example, digital media (whether on CD or hard drive), vinyl records,
and tapes.  The \textbf{video} type includes (nearly) all visual
media, whether it be films, TV shows, tapes and DVDs of the preceding
or of any sort of performance (including music), or online multimedia.
Finally, the \textbf{audio} type, our current concern, fills gaps in
the two others, and presents its sources in a more \enquote{book-like}
manner.  Published musical scores need this type --- unpublished ones
would use \textsf{misc} with an \textsf{entrysubtype} (shapey:partita)
--- as do such favorite educational formats as the slideshow and the
filmstrip (greek:filmstrip, schubert:muellerin, verdi:corsaro).  The
\emph{Manual} (17.269--270) sometimes uses a similar format for audio
books and even for films (twain:audio, weed:flatiron), though
elsewhere these sorts of material are presented as \textsf{music} and
\textsf{video}, respectively.  It would appear to depend on which
sorts of publication facts you wish to present --- cf.\ \emph{Manual}
17.269.

%\enlargethispage{\baselineskip}

\mylittlespace Once you've accepted the analogy of composer to
\textsf{author}, constructing an \textsf{audio} entry should be fairly
straightforward, since many of the fields function just as they do in
\textsf{book} or \textsf{inbook} entries.  Indeed, please note that I
compare it to both these other types as, in common with the other
audiovisual types, \textsf{audio} has to do double duty as an analogue
for both books and collections, so while there will normally be an
\textsf{author}, a \textsf{title}, a \textsf{publisher}, a
\textsf{date}, and a \textsf{location}, there may also be a
\textsf{booktitle} and/or a \textsf{maintitle} --- see
schubert:muellerin for an entry that uses all three in citing one song
from a cycle.  If the medium in question needs specifying, the
\textsf{type} field is the place for it.  (It is characteristic of
this entry type that such information is placed after the publisher
information, whereas in the other audiovisual types their order is
reversed.)  Finally, the \textsf{titleaddon} field can specify
functions for which \textsf{biblatex-chicago} provides no automated
handling, e.g., a librettist (verdi:corsaro).

\mybigspace This \mymarginpar{\textbf{bookinbook}} type provides the
means of referring to parts of books that are considered, in other
contexts, themselves to be books, rather than chapters, essays, or
articles.  (Older versions of \textsf{biblatex-chicago} used
\textbf{customb} for this purpose, but this is now obsolete.)  Such an
entry can have a \textsf{title} and a \textsf{maintitle}, but it can
also contain a \textsf{booktitle}, all three of which will be
italicized when printed.  In general usage it is, therefore, rather
like the traditional \textsf{inbook} type, only with its
\textsf{title} in italics rather than in quotation marks.  (See
\emph{Manual} 17.72, 17.89, 17.93; bernard:boris, euripides:orestes,
plato:republic:gr.)

\mylittlespace \textbf{NB}: The Euripides play receives slightly
different presentations in 17.89 and 17.93.  Although the
specification is very detailed, it doesn't eliminate all choice or
variation.  Using a system like \textsc{Bib}\TeX\ should help to
maintain consistency.  

\mybigspace This \mymarginpar{\textbf{booklet}} is the first of two
entry types --- the other being \textsf{manual}, on which see below
--- which are traditional in \textsc{Bib}\TeX\ styles, but which the
\emph{Manual} (17.241) suggests may well be treated basically as
books.  In the interests of backward compatibility,
\textsf{biblatex-chicago-notes} will so format such an entry, which
uses the \textsf{howpublished} field instead of a standard
\textsf{publisher}, though of course if you do decide just to use a
\textsf{book} entry then any information you might have given in a
\textsf{howpublished} field should instead go in \textsf{publisher}.
(See clark:mesopot.)

\mybigspace This \mymarginpar{\textbf{customa}} entry type is now
obsolete, and any such entries in your .bib file will trigger an
error.  Please use the standard \textsf{biblatex} \textbf{letter} type
instead.

%\enlargethispage{-\baselineskip}

\mybigspace This \mymarginpar{\textbf{customb}} entry type is now
obsolete, and any such entries in your .bib file will trigger an
error.  Please use the standard \textsf{biblatex} \textbf{bookinbook}
type instead.

\mybigspace This \mymarginpar{\textbf{customc}} entry type has
undergone a metamorphosis with this release, as I previously warned
both here and in the RELEASE file.  Rather than being a (deprecated)
alias of the standard \textsf{biblatex} \textbf{suppbook}, it now
allows you to include alphabetized cross-references to other, separate
entries in the bibliography, particularly to other names or
pseudonyms, as recommended by the \emph{Manual}.  (This is different
from the usual \textsf{crossref}, \textsf{xref}, and \textsf{userf}
mechanisms, all primarily designed to include cross-references to
other works.  Cf.\ 17.39--40).  The lecarre:cornwell entry, for
example, would allow your readers to find the more-commonly-used
pseudonym \enquote{John Le Carr�} even if they were, for some reason,
looking under his real name \enquote{David John Moore Cornwell.}

\mylittlespace In such a case, you would need merely to place the
author's real name in the \textsf{author} field, and the pseudonym(s),
under which his or her works are presented in the bibliography, in the
\textsf{title} field.  To make sure the cross-reference also appears
in the bibliography, you can either manually include the entry key in
a \cmd{nocite} command, or you can put that entry key in the
\textbf{userc} field in the main .bib entry, in which case
\textsf{biblatex-chicago} will print the expanded abbreviation if and
only if you cite the main entry.  (Cf.\ lecarre:cornwell,
lecarre:quest; \textsf{userc}, below.)

\mylittlespace Under ordinary circumstances, \textsf{biblatex-chicago}
will connect the two parts of the cross-reference with the word
\enquote{\emph{See}} --- or its equivalent in the document's language
--- in italics.  If you wish to present the cross-reference
differently, you can put the connecting word(s) into the
\textsf{nameaddon} field.

\mybigspace This \mymarginpar{\textbf{image}} entry type, left
undefined in the standard styles, is in \textsf{biblatex-chicago}
intended for referring to photographs.  Excluding the possible use of
the \textsf{entrysubtype} field, which in an \textsf{image} entry
would be ignored, this type is a clone of \textsf{artwork}, so you
should consult the latter's documentation above to see how to
construct your .bib entry.  (See \emph{Manual} 12.33; bedford:photo.)

\mybigspace These \mymarginpar{\textbf{inbook}\\\textbf{incollection}}
two standard \textsf{biblatex} types have very nearly identical
formatting requirements as far as the Chicago specification is
concerned, but I have retained both of them for compatibility.
\textsf{Biblatex.pdf} (�~2.1.1) intends the first for \enquote{a part
  of a book which forms a self-contained unit with its own title,}
while the second would hold \enquote{a contribution to a collection
  which forms a self-contained unit with a distinct author and its own
  title.}  The \textsf{title} of both sorts will be placed within
quotation marks, and in general you can use either type for most
material falling into these categories.  There is, however, an
important difference between them, as it is only in
\textsf{incollection} entries that I implement the \emph{Manual's}
recommendations for space-saving abbreviations in notes and
bibliography when you cite multiple pieces from the same
\textsf{collection}.  These abbreviations are activated when you use
the \textsf{crossref} or \textsf{xref} field in \textsf{incollection}
entries, and not in \textsf{inbook} entries, mainly because the
\emph{Manual} (17.70) here specifies a \enquote{multiauthor book.}
(For more on this mechanism see \textbf{crossref}, below, and note
that it is also active in \textsf{letter} and \textsf{inproceedings}
entries.  There is, of course, nothing to prevent you from using the
mechanism when referring to, e.g., chapters from a single-author book,
but you'll have to use \textsf{incollection} instead of
\textsf{inbook}.)  If the part of a book to which you are referring
has had a separate publishing history as a book in its own right, then
you may wish to use the \textsf{bookinbook} type, instead, on which
see above.  (See \emph{Manual} 17.68--72; \textsf{inbook}:
ashbrook:brain, phibbs:diary, will:cohere; \textsf{incollection}:
centinel:letters, contrib:contrib, sirosh:visualcortex; ellet:galena,
keating:dearborn, and lippincott:chicago [and the \textsf{collection}
entry prairie:state] demonstrate the use of the \textsf{crossref}
field with its attendant abbreviations in notes and bibliography.)

\mylittlespace \textbf{NB}: The \emph{Manual} suggests that, when
referring to a chapter, one use either a chapter number or the
inclusive page numbers, not both.  If, however, you wish to refer in a
footnote to a specific page within the chapter,
\textsf{biblatex-chicago-notes} will always print the optional,
postnote argument of a \cmd{cite} command --- the page number, say ---
instead of any inclusive page numbers given in the .bib file
\textsf{incollection} entry.  This mechanism is quite general, that
is, any specific page reference given in any sort of \cmd{cite}
command overrides the contents of a \textsf{pages} field in a .bib
file entry.

\mybigspace This \mymarginpar{\textbf{inproceedings}} entry type works
pretty much as in standard \textsf{biblatex}.  Indeed, the main
differences between it and \textsf{incollection} are the lack of an
\textsf{edition} field and the possibility that an
\textsf{organization} may be cited alongside the \textsf{publisher},
even though the \emph{Manual} doesn't specify its use (17.71).  Please
note, also, that the \textsf{crossref} and \textsf{xref} mechanism for
shortening citations of multiple pieces from the same
\textsf{proceedings} is operative here, just as it is in
\textsf{incollection} entries.  See \textbf{crossref}, below, for more
details.

\enlargethispage{\baselineskip}

\mybigspace This \mymarginpar{\textbf{inreference}} entry type is
aliased to \textsf{incollection} in the standard styles, but the
\emph{Manual} has particular requirements, so if you are citing
\enquote{[w]ell-known reference books, such as major dictionaries and
  encyclopedias,} then this type should simplify the task of
conforming to the specifications (17.238--239).  The main thing to
keep in mind is that I have designed this entry type for
\enquote{alphabetically arranged} works, which you shouldn't cite by
page, but rather by the name(s) of the article(s).  Because of the
formatting required by the \emph{Manual}, we need one of
\textsf{biblatex's} list fields for this purpose, and in order to keep
all this out of the way of the standard styles, I have chosen the
\textsf{lista} field.  You should present these article names just as
they appear in the work, separated by the keyword
\enquote{\texttt{and}} if there is more than one, and
\textsf{biblatex-chicago-notes} will provide the appropriate prefatory
string (\texttt{s.v.}, plural \texttt{s.vv.}), and enclose each in its
own set of quotation marks (ency:britannica).  In a typical
\textsf{inreference} entry, very few other fields are needed, as
\enquote{the facts of publication are often omitted, but the edition
  (if not the first) must be specified.}  In practice, this means a
\textsf{title} and possibly an \textsf{edition} field.

\mylittlespace There are quite a few other peculiarities to explain
here.  First of all, you should present any well-known works
\emph{only} in notes, not in a bibliography, as your readers are
assumed to know where to go for such a reference.  You can use the
\texttt{skipbib} option or the \textsf{keywords} mechanism I discuss
below under \textbf{crossref} and \textbf{keywords}.  For such works,
and given how little information will be present even in a full note,
you may wish to use \cmd{fullcite} or \cmd{footfullcite} in place of
the short form, especially if, for example, you are citing different
versions of an article appearing in different editions.

\mylittlespace If the work is slightly less well known, it may be that
full publication details are appropriate (times:guide), but this makes
things more complicated.  In previous releases of
\textsf{biblatex-chicago-notes}, you would have had to format the
\textsf{postnote} field of short notes appropriately, including the
prefatory string and quotation marks I mentioned above.  Now you can
put an article name in the \textsf{postnote} field of
\textsf{inreference} entries and have it formatted for you, and this
holds for both long and short notes, which could allow you to refer
separately to many different articles from the same reference work
using only one .bib entry.  (In a long note, any \textsf{postnote}
field stops the printing of the contents of \textsf{lista}.)  The only
limitation on this system is that the \textsf{postnote} field, unlike
\textsf{lista}, is not a list, and therefore for the formatting to
work correctly you can only put one article name in it.  Despite this
limitation, I hope that the current system might simplify things for
users who cite numerous works of reference.

\mylittlespace If it seems appropriate to include such a work in the
bibliography, be aware that the contents of the \textsf{lista} field
will also be presented there, which may not be what you want.  A
separate \textsf{reference} entry might solve this problem, but you
may also need a \textsf{sortkey} field to ensure proper
alphabetization, as \textsf{biblatex} will attempt to use an
\textsf{editor} or \textsf{author} name, if either is present.  (Cf.\
mla:style, a \textsf{reference} entry that uses section numbers
instead of alphabetized headings, and \texttt{useeditor=false} in the
\textsf{options} field instead of a \textsf{sortkey} to ensure the
correct alphabetization.)

\mylittlespace Speaking of the \textsf{author}, this field holds the
author of the specific entry (in \textsf{lista}), not the author of
the \textsf{title} as a whole.  This name will be printed in
parentheses after the entry's name (grove:sibelius).  If you wish to
refer to a reference work by author or indeed by editor, having either
appear at the head of the note (long or short) or bibliography entry,
then you'll need to use a \textsf{book} entry instead (cf.\
schellinger:novel), where the \textsf{lista} mechanism will also work
in the bibliography, but which in every other way will be treated as a
normal book, often a good choice for unfamiliar or non-standard
reference works.

\mylittlespace Finally, all of these rules apply to online reference
works, as well, for which you need to provide not only a \textsf{url}
but also, always, a \textsf{urldate}, as these sources are in constant
flux (wikiped:bibtex, grove:sibelius).

\mybigspace This \mymarginpar{\textbf{letter}} is the entry type to
use for citing letters, memoranda, or similar texts, but \emph{only}
when they appear in a published collection.  (Unpublished material of
this nature needs a \textsf{misc} entry, for which see below.)
Depending on what sort of information you need to present in a
citation, you may simply be able to get away with a standard
\textsf{book} entry, which may then be cited by page number (see
\emph{Manual} 17.31, 17.42; meredith:letters, adorno:benj).  If,
however, for whatever reason, you need to give full details of a
specific letter, then you'll need to use the \textsf{letter} entry
type, which attempts to simplify for you the \emph{Manual}'s rather
complicated rules for formatting such references.  (See 17.76--78;
jackson:paulina:letter, white:ross:memo, white:russ [a completely
fictitious entry to show the \textsf{xref} mechanism], white:total [a
\textsf{book} entry, for the bibliography]).

\mylittlespace To start, the name of the letter writer goes in the
\textsf{author} field, while the \textsf{title} field contains both
the name of the writer and that of the recipient, in the form
\texttt{Author to Recipient}.  The \textsf{titleaddon} field contains
the type of correspondence involved.  If it's a letter, this field may
be left blank, but if it's a memorandum or report or the like, then
this is the place to specify that fact.  Also, because the
\textsf{origdate} field only accepts numbers, if you want to use the
abbreviation \enquote{n.d.} (or \cmd{bibstring\{nodate\}}) for undated
letters, then this is where you should put it.  If you need to specify
where a letter was written, then you can also use this field, and, if
both are present, remember to separate the location from the type with
a comma, like so: \texttt{memorandum, London}.  Alternatively, you can
put the place of writing into the \mycolor{\textsf{origlocation}}
field.  Most importantly, the date of the letter itself goes in the
\textsf{origdate} field (\texttt{year-month-day}), which now allows a
full date specification, while the publishing date of the whole
collection goes in the \textsf{date} field, instead of in the obsolete
\textsf{origyear}.  As in other entry types, then, the \textsf{date}
field now has its ordinary meaning of \enquote{date of publication.}
(You may have noticed here that the presentation of the
\textsf{origdate} in this sort of reference is different from the date
format required elsewhere by the \emph{Manual}.  This appears to
result from some recent changes to the specification, and it may be
that we could get away with choosing one or the other format for all
occurrences [6.46], but for the moment I hope this mixed solution will
suffice.)  Another difficulty arises when producing the short footnote
form, which requires you to provide a \textsf{shorttitle} field of the
form \enquote{\texttt{to Recipient},} the latter name as short as
possible while avoiding ambiguity.  The remaining fields are fairly
self explanatory, but do remember that the title of the published
collection belongs in \textsf{booktitle} rather than in
\textsf{title}.

\mylittlespace Finally, the \emph{Manual} specifies that if you cite
more than one letter from a given published collection, then the
bibliography should contain only a reference to said collection,
rather than to each individual letter, while the form of footnotes
would remain the same.  This should be possible using
\textsc{Bib}\TeX's standard \textsf{crossref} field, with each
\textsf{letter} entry pointing to a \textsf{collection} or
\textsf{book} entry, for example.  I shall discuss cross references at
length later (\textbf{crossref} and \textbf{xref}, below), but I
should mention here that \textsf{letter} is one of the entry types in
which a \textsf{crossref} or an \textsf{xref} field automatically
results in special shortened forms in notes and bibliography if more
than one piece from a single collection is cited.  (The other entry
types are \textsf{incollection} and \textsf{inproceedings}; see 17.70
for the \emph{Manual}'s specification.)  This ordinarily won't be an
issue for \textsf{letter} entries in the bibliography, as individual
letters aren't included there, but it is operative in notes, where you
can disable it simply by not using a \textsf{crossref} or an
\textsf{xref} field.  In the \textsf{crossref} docs, below, I
recommend a way of keeping the individual letters from turning up in
the bibliography, involving the use of the \textsf{keywords} field.

%\enlargethispage{\baselineskip}

\mybigspace This \mymarginpar{\textbf{manual}} is the second of two
traditional \textsc{Bib}\TeX\ entry types that the \emph{Manual}
suggests formatting as books, the other being \textsf{booklet}. As
with this latter, I have retained it in
\textsf{biblatex-chicago-notes} for backward compatibility, its main
peculiarity being that, in the absence of a named author, the
\textsf{organization} producing the manual will be printed both as
author and as publisher.  In such a case, you'll need a
\textsf{sortkey} field to aid \textsf{biblatex's} alphabetization
routines, but you no longer need to provide a \textsf{shortauthor}
field, as the style will automatically use \textsf{organization} in
the absence of anything else.  Of course, if you were to use a
\textsf{book} entry for such a reference, then you would need to
define both \textsf{author} and \textsf{publisher} using the name you
here might have put in \textsf{organization}.  (See 17:47;
chicago:manual:15, dyna:browser, natrecoff:camera.)

\mybigspace As \colmarginpar{\textbf{misc}} its name suggests, the
\textsf{misc} entry type was designed as a hold-all for citations that
didn't quite fit into other categories.  In
\textsf{biblatex-chicago-notes}, I have somewhat extended its
applicability, while retaining its traditional use.  Put simply, with
no \textsf{entrysubtype} field, a \textsf{misc} entry will retain
backward compatibility with the standard styles, so the usual
\textsf{howpublished}, \textsf{version}, and \textsf{type} fields are
all available for specifying an otherwise unclassifiable text, and the
\textsf{title} will be italicized.  (The \emph{Manual}, you may wish
to note, doesn't give specific instructions on how such citations
should be formatted, so when using the Chicago style I would recommend
you have recourse to this traditional entry type as sparingly as
possible.)

\mylittlespace If you do provide an \textsf{entrysubtype} field, the
\textsf{misc} type provides a means for citing unpublished letters,
memoranda, private contracts, wills, interviews, and the like, making
it something of an unpublished analogue to the \textsf{letter},
\textsf{article}, and \textsf{review} entry types (which see).
Typically, such an entry will cite part of an archive, and equally
typically the text cited won't have a specific title, but only a
generic one, whereas an \textsf{unpublished} entry will ordinarily
have a specific author and title, and won't come from a named archive.
The \textsf{misc} type with an \textsf{entrysubtype} defined is the
least formatted of all those specified by the \emph{Manual}, so titles
are in plain text, and any location details take no parentheses in
full footnotes.  (It is quite possible, though somewhat unusual, for
archival material to have a specific title, rather than a generic one.
In these cases, you will need to enclose the title inside a
\cmd{mkbibquote} command manually.  Cf.\ shapey:partita.)

\mylittlespace If you are wondering what to put in
\textsf{entrysubtype}, the answer is, currently, anything at all.  You
no longer need to put the exact string \texttt{letter} there in order
to move the date into closer proximity with the \textsf{title}.
Indeed, recent reconsideration of the \emph{Manual} has suggested that
the distinction to be drawn in this class of material hasn't to do
with \emph{where} the date is presented but, rather, with \emph{how}
it is presented.  As I now understand the specification, it draws a
distinction between archival material that is \enquote{letter-like}
(letters, memoranda, reports, telegrams) and that which isn't
(interviews, wills, contracts, or even personal communications you've
received and which you wish to cite).  This may not always be the
easiest distinction to draw, and in previous releases of
\textsf{biblatex-chicago} I have been ignoring it, but once you've
decided to classify it one way or the other you put the date in the
\textsf{origdate} field for letters, etc., and into the \textsf{date}
field for the others.

\mylittlespace In effect, whether it's a \textsf{letter} entry or a
\enquote{letter-like} \textsf{misc} entry (with
\textsf{entrysubtype}), it is by using the \textsf{origdate} field
that you identify when it was written, and the \textsf{origlocation},
if needed, identifies where it was written.  Other sorts of
\textsf{misc} entry (with \textsf{entrysubtype}) use the \textsf{date}
field (but still the \textsf{origlocation}).  This maintains
consistency of usage across entry types and also, I hope, improves
compliance when using the \textsf{misc} type for citing archival
material.  Remember, however, that without an \textsf{entrysubtype}
the entry will be treated as traditional \textsf{misc}, and the title
italicized.  In addition, defining \textsf{entrysubtype} activates the
automatic capitalization mechanism in the \textsf{title} field of
\textsf{misc} entries, on which see \textbf{\textbackslash autocap}
below.  (See 17.205-206, 17.220, 17.222-232; creel:house,
dinkel:agassiz, spock:interview.)

\enlargethispage{\baselineskip}

\mylittlespace As in \textsf{letter} entries, the titles of
unpublished letters are of the form \texttt{Author to Recipient}, and
further information can be given in the \textsf{titleaddon} field,
including the abbreviation \enquote{\texttt{n.d.}}\ (or
\cmd{bibstring\{nodate\}}) for undated examples.  The \textsf{note},
\textsf{organization}, \textsf{institution}, and \textsf{location}
fields (in ascending order of generality) allow the specification of
which manuscript collection now holds the letter, though the
\emph{Manual} specifies (17.228) that well-known depositories don't
usually need a city, state or country specified.  (The traditional
\textsf{misc} fields are all still available, also.)  Both the long
and short note forms can use the same \textsf{title}, but in both
cases you may need to use the \cmd{headlesscite} command to avoid the
awkward repetition of the author's name, though that name will always
appear in the bibliography (creel:house).  If you want to include the
date of a letter in a short note, I have provided the
\cmd{letterdatelong} command for inclusion in the postnote field of
the citation command.  (The standard \textsf{biblatex} command
\cmd{printdate} will work if you need to do the same for interviews.)

\mylittlespace As with \textsf{letter} entries, the \emph{Manual}
(17.223) suggests that bibliography entries contain only the name of
the manuscript collection, unless only one item from that collection
is cited.  The \textsf{crossref} field can be used, as well as the
\textsf{keywords} mechanism (or \texttt{skipbib} option) for
preventing the individual items from turning up in the bibliography.
Obviously, this is a matter for your discretion, and if you're using
only short notes (see the \texttt{short} option,
section~\ref{sec:useropts} below), you may feel the need to include
more information in the note if the bibliography doesn't contain a
full reference to an individual item.

\mylittlespace Finally, if the \textsf{misc} entry isn't a letter,
remember that, as in \textsf{article} and \textsf{review} entries,
words like \texttt{interview} or \texttt{memorandum} needn't be
capitalized unless they follow a period --- the automatic
capitalization routines (with the \textsf{title} field starting with a
lowercase letter [see dinkel:agassiz, spock:interview, and
\textbf{\textbackslash autocap}]) will ensure correctness.  In all
this class of archived material, the \emph{Manual} (17.222) quite
specifically requires more consistency within your own work than
conformity to some external standard, so it is the former which you
should pursue.  I hope that \textsf{biblatex-chicago-notes} proves
helpful in this regard.

\mybigspace This \mymarginpar{\textbf{music}} is one of three new
audiovisual entry types, and is intended primarily to aid in the
presentation of musical recordings that do not have a video component,
though it can also include audio books (auden:reading).  A DVD or VHS
of an opera or other performance, by contrast, should use the
\textbf{video} type instead (handel:messiah).  Because
\textsf{biblatex} --- and \textsc{Bib}\TeX\ before it --- were
designed primarily for citing book-like objects, some choices needed
to be made in assigning the various roles found on the back of a CD to
the fields in a typical .bib entry.  I have also implemented several
new bibstrings to help in identifying these roles within entries.  If
you can think of a simpler way to distribute the roles, please let me
know, so that I can consider making changes before anyone gets used to
the current equivalences.

\mylittlespace These equivalences, in summary form, are:

{\renewcommand{\descriptionlabel}[1]{\qquad\textsf{#1}}
\begin{description}
\item[author =] composer, songwriter, or performer(s),
  depending on whom you wish to emphasize by placing them at the head
  of the entry.
\item[editor, editora, editorb =] conductor, director or
  performer(s).  These will ordinarily follow the \textsf{title} of
  the work, though the usual \texttt{useauthor} and \texttt{useeditor}
  options can alter the presentation within an entry.  Because these
  are non-standard roles, you will need to identify them using the
  following:
\item[editortype, editoratype, editorbtype:] The most common roles,
  all associated with specific bibstrings (or their absence), will be
  \texttt{conductor}, \texttt{director}, \texttt{producer}, and,
  oddly, \texttt{none}.  The last is particularly useful when
  identifying the group performing a piece, as it usually doesn't need
  further specifying and this role prevents \textsf{biblatex} from
  falling back on the default \texttt{editor} bibstring.
\item[title, booktitle, maintitle:] As with the other audiovisual
  types, \textsf{music} serves as an analogue both to books and to
  collections, so the title will either be, e.g., the album title or a
  song title, in which latter case the album title would go into
  \textsf{booktitle}.  The \textsf{maintitle} might be necessary for
  something like a box set of \emph{Complete Symphonies}.
\item[series, number:] These two are closely associated, and are
  intended for presenting the catalog information provided by the
  music publisher, especially in the case when a publisher oversees
  more than one label.  In nytrumpet:art:15, for example, the
  \textsf{series} field holds the label (\texttt{Vox/Turn\-about}) and
  the \textsf{number} field the catalog number (\texttt{PVT 7183}).
  You can certainly put all of this information into one of the above
  fields, but separating it may help make the .bib entry more
  readable.
\item[howpublished/pubstate, date, publisher:] The \emph{Manual}
  (17.268) follows the rather specialized requirements for presenting
  publishing information for musical recordings.  The normal symbol
  for musical copyright is\ \texttt{\textcircledP} (Unicode point
  u+2117, SOUND RECORDING COPYRIGHT), but other copyrights
  \texttt{\textcopyright} are often also asserted.  The
  \textsf{howpublished} field is the place for these symbols, and it
  may also have to hold a year designation if the
  \texttt{\textcircledP} and the \texttt{\textcopyright} apply to
  different years, as sometimes happens.  (The \textsf{pubstate} field
  in this entry type is a synonym for \textsf{howpublished}.  Please
  use only one of them per entry, and note that the usual mechanism
  for automatically printing \cmd{bibstring\{reprint\}} is turned off
  in \textsf{music} entries.)  The \textsf{date} field holds the year
  either of all the symbols or of whichever symbol appears last in
  \textsf{howpublished}, and the \textsf{publisher} field is
  self-explanatory.  (See nytrumpet:art:15.)
\item[type:] As in all the audiovisual entry types, the \textsf{type}
  field holds the medium of the recording, e.g., vinyl, 33 rpm,
  8-track tape, cassette, compact disc, mp3, ogg vorbis.
\end{description}}

I should also note here that I have implemented the standard
\textsf{biblatex} \textsf{eventdate} field, in case you need it to
identify a particular recording session or concert.  It will be
printed just after the \textsf{title}.  The entries in
\textsf{notes-test.bib} should at least give you a good idea of how
this all works, and that file also contains an example of an audio
book presented in a \textsf{music} entry.  If you browse the examples
in the \emph{Manual} you will see some variation from the formatting
choices I have made for \textsf{biblatex-chicago}, but it wasn't
always clear to me that these variations were rules as opposed to
suggestions, so I've ignored some of them in the code.  Arguments as
to why I'm wrong will, of course, be entertained.  (Cf. 17.268;
auden:reading, beethoven:sonata29, bernstein:shostakovich,
nytrumpet:art:15.)

\enlargethispage{\baselineskip}

\mybigspace The \mymarginpar{\textbf{online}} \emph{Manual}'s
instructions (17.142--147, 17.198, 17.234--237) for citing online
materials are slightly different from those suggested by standard
\textsf{biblatex}.  Indeed, this is a case where complete backward
compatibility with other \textsf{biblatex} styles may be impossible,
because as a general rule the \emph{Manual} considers relevant not
only where a source is found, but also the nature of that source,
e.g., if it's an online edition of a book (james:ambassadors), then it
calls for a \textsf{book} entry.  Even if you cite an
\enquote{intrinsically online} source, if that source is structured
more or less like a conventional printed periodical, then you'll
probably want to use \textsf{article} or \textsf{review} instead of
\textsf{online} (stenger:privacy, which cites \emph{CNN.com} ---
\emph{Yahoo!\ News} is another example that would be treated in such a
way).  If the \enquote{standard facts of publication} are missing,
then the \textsf{online} type is usually the best choice
(evanston:library, powell:email).  Some online materials will, no
doubt, make it difficult to choose an entry type, but so long as all
locating information is present, then perhaps that is enough to
fulfill the specification, or at least so I'd like to hope.

\mylittlespace Constructing an \textsf{online} .bib file entry is much
the same as in \textsf{biblatex}.  The \textsf{title} field would
contain the title of the page, the \textsf{organization} field could
hold the title or owner of the whole site.  If there is no specific
title for a page, but only a generic one (powell:email), then such a
title should go in \textsf{titleaddon}, not forgetting to begin that
field with a lowercase letter so that capitalization will work out
correctly.

\mybigspace The \mymarginpar{\textbf{patent}} \emph{Manual} is very
brief on this subject (17.219), but very clear about which information
it wants you to present, so such entries may not work well with other
\textsf{biblatex} styles.  The important date, as far as Chicago is
concerned, is the filing date.  If a patent has been filed but not yet
granted, then you can place the filing date in either the
\textsf{date} field or the \textsf{origdate} field, and
\textsf{biblatex-chicago-notes} will automatically prepend the
bibstring \texttt{patentfiled} to it.  If the patent has been granted,
then you put the filing date in the \textsf{origdate} field, and you
put the date it was issued in the \textsf{date} field, to which the
bibstring \texttt{patentissued} will automatically be prepended.  (In
other words, you no longer need to use a hand-formatted
\textsf{addendum} field, though you can place additional information
in that field if desired, and it will be printed in close association
with the dates.)  The patent number goes in the \textsf{number} field,
and you should use the standard \textsf{biblatex} bibstrings in the
\textsf{type} field.  Though it isn't mentioned by the \emph{Manual},
\textsf{biblatex-chicago-notes} will print the \textsf{holder} after
the \textsf{author}, if you provide one.  See petroff:impurity.


\mybigspace This \mymarginpar{\textbf{periodical}} is the standard
\textsf{biblatex} entry type for presenting an entire issue of a
periodical, rather than one article within it.  It has the same
function in \textsf{biblatex-chicago-notes}, and in the main uses the
same fields, though in keeping with the system established in the
\textsf{article} entry type (which see) you'll need to provide
\textsf{entrysubtype} \texttt{magazine} if the periodical you are
citing is a \enquote{newspaper} or \enquote{magazine} instead of a
\enquote{journal.}  Also, remember that the \textsf{note} field is the
place for identifying strings like \enquote{special issue,} with its
initial lowercase letter to activate the automatic capitalization
routines.  (See \emph{Manual} 17.170; good:wholeissue.)

\mybigspace This \mymarginpar{\textbf{reference}} entry type is
aliased to \textsf{collection} by the standard \textsf{biblatex}
styles, but I intend it to be used in cases where you need to cite a
reference work but not an alphabetized entry or entries in that work.
This could be because it doesn't contain such entries, or perhaps
because you intend the citation to appear in a bibliography rather
than in notes.  Indeed, the only differences between it and
\textsf{inreference} are the lack of a \textsf{lista} field to present
an alphabetized entry, and the fact that any \textsf{postnote} field
will be printed verbatim, rather than formatted as an alphabetized
entry.  (See mla:style for an example of a reference work that uses
numbered sections rather than alphabetized entries, and that appears
in the bibliography as well.)

\mybigspace This \mymarginpar{\textbf{report}} entry type is a
\textsf{biblatex} generalization of the traditional \textsc{Bib}\TeX\
type \textsf{techreport}.  Instructions for such entries are rather
thin on the ground in the \emph{Manual} (17.241), so I have followed
the generic advice about formatting it like a book, and hope that the
results conform to the specification.  Its main peculiarities are the
\textsf{institution} field in place of a \textsf{publisher}, the
\textsf{type} field for identifying the kind of report in question,
and the \textsf{isrn} field containing the International Standard
Technical Report Number of a technical report.  As in standard
\textsf{biblatex}, if you use a \textsf{techreport} entry, then the
\textsf{type} field automatically defaults to
\cmd{bibstring\{techreport\}}.  As with \textsf{booklet} and
\textsf{manual}, you can also use a \textsf{book} entry, putting the
report type in \textsf{note} and the \textsf{institution} in
\textsf{publisher}.  (See herwign:office.)

%\enlargethispage{\baselineskip}

\mybigspace The \colmarginpar{\textbf{review}} \textsf{review} entry
type was added to \textsf{biblatex 0.7}, and it certainly eases the
task of coping with the \emph{Manual}'s complicated requirements for
citing periodicals of all sorts, though it doesn't, I admit, eliminate
all difficulties.  As its name suggests, this entry type was designed
for reviews published in periodicals, and if you've already read the
\textsf{article} instructions above --- if you haven't, I recommend
doing so now --- you'll know that \textsf{review} serves as well for
citing other sorts of material with generic titles, like letters to
the editor, obituaries, interviews, and the like.  The primary rule is
that any piece that has only a generic title, like \enquote{review of
  \ldots,} \enquote{interview with \ldots,} or \enquote{obituary of
  \ldots,} calls for the \textsf{review} type.  Any piece that also
has a specific title, e.g., \enquote{\enquote{Lost in
    \textsc{Bib}\TeX,} an interview with \ldots,} requires an
\textsf{article} entry.  (This assumes the text is found in a
periodical of some sort.  Were it found in a book, then the
\textsf{incollection} type would serve your needs, and you could use
\textsf{title} and \textsf{titleaddon} there.  While we're on the
topic of exceptions, the \emph{Manual} includes an example --- 17.207
--- where the \enquote{Interview} part of the title is considered a
subtitle rather than a titleaddon, said part therefore being included
inside the quotation marks and capitalized accordingly.  Not having
the journal in front of me I'm not sure what prompted that decision,
but \textsf{biblatex-chicago} would obviously have no difficulty
coping with such a situation.)

\mylittlespace Once you've decided to use \textsf{review}, then you
need to determine which sort of periodical you are citing, the rules
for which are the same as for an \textsf{article} entry.  If it is a
\enquote{magazine} or a \enquote{newspaper}, then you need an
\textsf{entrysubtype} \texttt{magazine}.  The generic title goes in
\textsf{title} and the other fields work just as as they do in an
\textsf{article} entry with the same \textsf{entrysubtype}, including
the substitution of the \textsf{journaltitle} for the \textsf{author}
if the latter is missing. (See 17.185, 17.188--194, 17.199--203,
17.207; barcott:review, bundy:macneil, Clemens:letter, gourmet:052006,
kozinn:review, nyt:obittrevor, nyt:trevorobit, unsigned:ranke:15,
wallraff:word.)  If, on the other hand, the piece comes from a
\enquote{journal,} then you don't need an \textsf{entrysubtype}.  The
generic title goes in \textsf{title}, and the remaining fields work
just as they do in a plain \textsf{article} entry.  (See 17.201;
ratliff:review.)

\mylittlespace The onerous details are the same as I described them in
the \textbf{article} section above, but I'll repeat some of them
briefly here.  If anything in the \textsf{title} needs formatting, you
need to provide those instructions yourself, as the default is
completely plain.  In the short note form you no longer need to
provide a formatted \textsf{shortauthor} field for when a
\textsf{journaltitle} replaces an absent \textsf{author}, as the
package automatically prints the former there in the absence of
anything else (gourmet:052006, nyt:trevorobit).  If you wish to keep
the title at the head of an entry, then you'll need to define
\textsf{author} somehow and place \texttt{useauthor=false} in the
\textsf{options} field (as in nyt:obittrevor, by contrast with
nyt:trevorobit.  Please note that the \cmd{isdot} macro alone in the
\textsf{author} field no longer works in \textsf{biblatex} 1.6 and
later, so you may need to check your .bib files when you upgrade.)  As
in \textsf{misc} entries with an \textsf{entrysubtype}, words like
\enquote{interview,} \enquote{review,} and \enquote{letter} only need
capitalization after a full stop, i.e., ordinarily in a bibliography
and not a note, so \textsf{biblatex-chicago-notes} automatically deals
with this problem itself if you start the \textsf{title} field with a
lowercase letter.  The file \textsf{notes-test.bib} and the
documentation of \cmd{autocap} will provide guidance here.

\mybigspace This \mymarginpar{\textbf{suppbook}} is the entry type to
use if the main focus of a reference is supplemental material in a
book or in a collection, e.g., an introduction, afterword, or forward,
either by the same or a different author.  In previous releases of
\textsf{biblatex-chicago} these three just-mentioned types of
material, and only these three types, could be referenced using the
\textsf{introduction}, \textsf{afterword}, or \textsf{foreword}
fields, a system that required you simply to define one of them in any
way and leave the others undefined.  The macros don't use the text
provided by such an entry, they merely check to see if one of them is
defined, in order to decide which sort of pre- or post-matter is at
stake, and to print the appropriate string before the \textsf{title}
in long notes, short notes, list of shorthands, and bibliography.  I
have retained this mechanism both for backward compatibility and
because it works without modification across multiple languages, but
have also added functionality which allows you to cite any sort of
supplemental material whatever, using the \textsf{type} field.  Under
this system, simply put the nature of the material, including the
relevant preposition, in that field, beginning with a lowercase letter
so \textsf{biblatex} can decide whether it needs capitalization
depending on the context.  Examples might be \enquote{\texttt{preface
    to}} or \enquote{\texttt{colophon of}.}  (Please note, however,
that unless you use a \cmd{bibstring} command in the \textsf{type}
field, the resultant entry will not be portable across languages.)

\mylittlespace The other rules for constructing your .bib entry remain
the same.  The \textsf{author} field refers to the author of the
introduction or afterword, while \textsf{bookauthor} refers to the
author of the main text of the work, if the two differ.  If the focus
of the reference is the main text of the book, but you want to mention
the name of the writer of an introduction or afterword for
bibliographical completeness, then the normal \textsf{biblatex} rules
apply, and you can just put their name in the appropriate field of a
\textsf{book} entry, that is, in the \textsf{foreword},
\textsf{afterword}, or \textsf{introduction} field.  (See
\emph{Manual} 17.74--75; polakow:afterw, prose:intro).

\mybigspace This \mymarginpar{\textbf{suppcollection}} fulfills a
function analogous to \textsf{suppbook}.  Indeed, I believe the
\textbf{suppbook} type can serve to present supplemental material in
both types of work, so this entry type is an alias to
\textsf{suppbook}, which see.

\mybigspace This \mymarginpar{\textbf{suppperiodical}} type, new to
\textsf{biblatex} 0.8, is intended to allow reference to
generically-titled works in periodicals, such as regular columns or
letters to the editor.  Previous releases of
\textsf{biblatex-chicago-notes} provided the \textsf{review} type for
this purpose, and now you can use either of these, as I've added
\textsf{suppperiodical} as an alias of \textsf{review}.  Please see
above under \textbf{review} for the full instructions on how to
construct a .bib entry for such a reference.

%\enlargethispage{-\baselineskip}

\mybigspace The \mymarginpar{\textbf{unpublished}}
\textsf{unpublished} entry type works largely as it does in standard
\textsf{biblatex}, though it's worth remembering that you should use a
lowercase letter at the start of your \textsf{note} field (or perhaps
an\ \cmd{autocap} command in the somewhat contradictory
\textsf{howpublished}, if you have one) for material that wouldn't
ordinarily be capitalized except at the beginning of a sentence
(nass:address).

\mybigspace This \mymarginpar{\textbf{video}} is the last of the new
audiovisual entry types, and as its name suggests it is intended for
citing visual media, be it films of any sort or TV shows, broadcast,
on the Net, on VHS, DVD, or Blu-ray.  As with the \textsf{music} type
discussed above, certain choices had to be made when associating the
production roles found, e.g., on a DVD, to those bookish ones provided
by \textsf{biblatex}.  Here are the main correspondences:

{\renewcommand{\descriptionlabel}[1]{\qquad\textsf{#1}}
\begin{description}
\item[author:] This will not infrequently be left undefined, as the
  director of a film should be identified as such and therefore placed
  in the \textsf{editor} field with the appropriate
  \textsf{editortype} (see below).  You will need it, however, to
  identify the composer of, e.g., an oratorio on VHS (handel:messiah),
  or perhaps the provider of commentaries or other extras on a film
  DVD (cleese:holygrail).
\item[editor, editora, editorb =] director or producer, or possibly
  the performer or conductor in recorded musical performances.  These
  will ordinarily follow the \textsf{title} of the work, though the
  usual \texttt{useauthor} and \texttt{useeditor} options can alter
  the presentation within an entry.  Because these are non-standard
  roles, you will need to identify them using the following:
\item[editortype, editoratype, editorbtype:] The most common roles,
  all associated with specific bibstrings (or their absence), will
  likely be \texttt{director}, \texttt{produ\-cer}, and, oddly,
  \texttt{none}.  The last is particularly useful if you want to
  identify performers, as they usually don't need further specifying
  and this role prevents \textsf{biblatex} from falling back on the
  default \texttt{editor} bibstring.
\item[title, titleaddon, booktitle, maintitle:] As with the other
  audiovisual types, \textsf{video} serves as an analogue both to
  books and to collections, so the \textsf{title} may be of a whole
  film DVD or of a TV series, or it may identify one episode in a
  series or one scene in a film.  In the latter cases, the title of
  the whole would go in \textsf{booktitle}.  The \textsf{titleaddon}
  field may be useful for specifying the season and/or episode number
  of a TV series, or for any other information that needs to come
  between the \textsf{title} and the \textsf{booktitle}
  (cleese:holygrail, episode:tv, handel:messiah).  As in the
  \textsf{music} type, \textsf{maintitle} may be necessary for a boxed
  set or something similar.
\item[date, origdate:] The publication details of this sort of
  material are usually straightforward, at least compared with the
  \textsf{music} type, but there will be occasions when you need two
  dates.  When citing an episode of a long-running TV series you may
  need both a date for the episode and a date range for the whole run,
  and when citing a film on DVD you may want to present both the
  original release date and the date of release on DVD.  In both
  cases, the \textsf{origdate} field holds the year of the original
  showing or transmission, while the \textsf{date} field holds either
  the years for an entire run of a TV show or the year of publication
  of the DVD (or other medium).  Cf. episode:tv, hitchcock:nbynw.
\item[type:] As in all the audiovisual entry types, the \textsf{type}
  field holds the medium of the \textsf{title}, e.g., 8 mm, VHS, DVD,
  Blu-ray, MPEG.
\end{description}}

As with the \textsf{music} type, entries in \textsf{notes-test.bib}
should at least give you a good idea of how all this works.  (Cf.\
17.270, 273; cleese:holygrail, episode:tv, handel:messiah,
hitchcock:nbynw, loc:city.)

\subsection{Entry Fields}
\label{sec:entryfields}

The following discussion presents, in alphabetical order, a complete
list of the entry fields you will need to use
\textsf{biblatex-chicago-notes}.  As in section \ref{sec:entrytypes},
I shall include references to the numbered paragraphs of the
\emph{Chicago Manual of Style}, and also to the entries in
\textsf{notes-test.bib}.  Many fields are most easily understood with
reference to other, related fields.  In such cases, cross references
should allow you to find the information you need.

\mybigspace As \mymarginpar{\textbf{addendum}} in standard
\textsf{biblatex}, this field allows you to add miscellaneous
information to the end of an entry, after publication data but before
any \textsf{url} or \textsf{doi} field.  In the \textsf{patent} entry
type (which see), it will be printed in close association with the
filing and issue dates.  In any entry type, if your data begins with a
word that would ordinarily only be capitalized at the beginning of a
sentence, then simply ensure that that word is in lowercase, and the
style will take care of the rest.  Cf.\ \textsf{note}. (See
\emph{Manual} 17.145, 17.123; davenport:attention, natrecoff:camera.)

\mybigspace In most \mymarginpar{\textbf{afterword}} circumstances,
this field will function as it does in standard \textsf{biblatex},
i.e., you should include here the author(s) of an afterword to a given
work.  The \emph{Manual} suggests that, as a general rule, the
afterword would need to be of significant importance in its own right
to require mentioning in the reference apparatus, but this is clearly
a matter for the user's judgment.  As in \textsf{biblatex}, if the
name given here exactly matches that of an editor and/or a translator,
then \textsf{biblatex-chicago-notes} will concatenate these fields in
the formatted references.

%\vspace{\baselineskip}

\mylittlespace As noted above, however, this field has a special
meaning in the \textsf{suppbook} entry type, used to make an
afterword, foreword, or introduction the main focus of a citation.  If
it's an afterword at issue, simply define \textsf{afterword} any way
you please, leave \textsf{foreword} and \textsf{introduction}
undefined, and \textsf{biblatex-chicago-notes} will do the rest. Cf.\
\textsf{foreword} and \textsf{introduction}. (See \emph{Manual} 17.46,
17.74; polakow:afterw.)

\mybigspace At \mymarginpar{\textbf{annotation}} \label{sec:annote}
the request of Emil Salim, \textsf{biblatex-chicago-notes} has, as of
version 0.9, added a package option (see \texttt{annotation} below,
section \ref{sec:useropts}) to allow you to produce annotated
bibliographies.  The formatting of such a bibliography is currently
fairly basic, though it conforms with the \emph{Manual's} minimal
guidelines (16.77).  The default in \textsf{chicago-notes.cbx} is to
define \cmd{DeclareFieldFormat\{an\-notation\}} using
\cmd{par}\cmd{nobreak} \cmd{vskip} \cmd{bibitemsep}, though you can
alter it by re-declaring the format in your preamble.  The
page-breaking algorithms don't always give perfect results here, but
the default formatting looks, to my eyes, fairly decent.  In addition
to tweaking the field formatting you can also insert \cmd{par} (or
even \cmd{vadjust\{\cmd{eject}\}}) commands into the text of your
annotations to improve the appearance.  Please consider the
\texttt{annotation} option a work in progress, but it is usable now.
(N.B.: The \textsc{Bib}\TeX\ field \textsf{annote} serves as an alias
for this.)

\mybigspace I \mymarginpar{\textbf{annotator}} have implemented this
\textsf{biblatex} field pretty much as that package's standard styles
do, even though the \emph{Manual} doesn't actually mention it.  It may
be useful for some purposes.  Cf.\ \textsf{commentator}.

%\enlargethispage{\baselineskip}

\mybigspace For \mymarginpar{\textbf{author}} the most part, I have
implemented this field in a completely standard \textsc{Bib}\TeX\
fashion.  Remember that corporate or organizational authors need to
have an extra set of curly braces around them (e.g.,
\texttt{\{\{Associated Press\}\}}\,) to prevent \textsc{Bib}\TeX\ from
treating one part of the name as a surname (17.47, 17.197;
assocpress:gun, chicago:manual:15).  If there is no \textsf{author},
then \textsf{biblatex-chicago-notes} will look, in sequence, for an
\textsf{editor}, \textsf{translator}, or \textsf{compiler} (actually
\textsf{namec}, currently) and use that name (or those names) instead,
followed by the appropriate identifying string (esp.\ 17.41, also
17.28--29, 17.88, 17.95, 17.172; boxer:china, brown:bremer,
harley:cartography, schellinger:novel, sechzer:women, silver:ga\-wain,
soltes:georgia).  Please note that when a \textsf{namec} appears at
the head of an entry, you'll need to assist \textsf{biblatex}'s
sorting algorithms by providing a \textsf{sortkey} field to ensure
correct alphabetization in the bibliography.  Also, a
\textsf{shortauthor} entry is necessary to provide a name at the head
of the short note form.

\mylittlespace In the rare cases when this substitution mechanism
isn't appropriate, you have two options: either you can
(chaucer:liferecords) put all the information into a \textsf{note}
field rather than individual fields, or you can use the
\textsf{biblatex} options \texttt{useauthor=false},
\texttt{useeditor=false}, \texttt{usetranslator=false}, and
\texttt{usecompiler=\\false} in the \textsf{options} field
(chaucer:alt).  If you look at the chaucer:alt entry in
\textsf{notes-test.bib}, you'll notice a peculiarity of this system of
toggles.  In order to ensure that the \textsf{title} of the book
appears at the head of the entry, you need to use \emph{all four} of
the toggles, even though the entry contains no \textsf{translator}.
Internally, \textsf{biblatex-chicago} is either searching for an
author-substitute, or it is skipping over elements of the ordered,
unidirectional chain \textsf{author -> editor -> translator ->
  compiler -> title}.  If you don't include
\texttt{usetranslator=false} in the \textsf{options} field, then the
package begins its search at \textsf{translator} and continues on to
\textsf{namec}, even though you have \texttt{usecompiler=false} in
\textsf{options}.  The result will be that the compilers' names will
appear at the head of the entry.  If you want to skip over parts of
the chain, you must turn off \emph{all} of the parts up to the one you
wish printed.

\mylittlespace This system of toggles, then, can turn off
\textsf{biblatex-chicago-notes}'s mechanism for finding a name to
place at the head of an entry, but it also very usefully adds the
possibility of citing a work with an \textsf{author} by its editor,
compiler or translator instead (17.45; eliot:pound), something that
wasn't possible before.  For full details of how this works, see the
\textsf{editortype} documentation below.  (Of course, in
\textsf{collection} and \textsf{proceedings} entry types, an
\textsf{author} isn't expected, so there the \textsf{editor} is
required, as in standard \textsf{biblatex}.  Also, in \textsf{article}
or \textsf{review} entries with \textsf{entrysubtype}
\texttt{magazine}, the absence of an \textsf{author} triggers the use
of the \textsf{journaltitle} in its stead.  See those entry types for
further details.)

\mylittlespace \textbf{NB}: The \emph{Manual} provides specific
instructions for formatting the names of both anonymous and
pseudonymous authors (17.32--39).  In the former case, if no author is
known or guessed at, then it may simply be omitted
(virginia:plantation).  The use of \enquote{Anonymous} as the name is
\enquote{generally to be avoided,} but may in some cases be useful
\enquote{in a bibliography in which several anonymous works need to be
  grouped.}  If, on the other hand, \enquote{the authorship is known
  or guessed at but was omitted on the title page,} then you need to
use the \textsf{authortype} field to let
\textsf{biblatex-chicago-notes} know this fact.  If the author is
known (horsley:prosodies), then put \texttt{anon} in the
\textsf{authortype} field, if guessed at (cook:sotweed) put
\texttt{anon?}\ there.  (In both cases,
\textsf{biblatex-chicago-notes} tests for these \emph{exact} strings,
so check your typing if it doesn't work.)  This will have the effect
of enclosing the name in square brackets, with or without the question
mark indicating doubt.  As long as you have the right string in the
\textsf{authortype} field, \textsf{biblatex-chicago-notes} will also
do the right thing automatically in the short note form.

\mylittlespace The \textsf{nameaddon} field furnishes the means to
cope with the case of pseudonymous authorship.  If the author's real
name isn't known, simply put \texttt{pseud.} (or
\cmd{bibstring\{pseudonym\}}) in that field (centinel:letters).  If
you wish to give a pseudonymous author's real name, simply include it
there, formatted as you wish it to appear, as the contents of this
field won't be manipulated as a name by \textsf{biblatex}
(lecarre:quest).  If you have given the author's real name in the
\textsf{author} field, then the pseudonym goes in \textsf{nameaddon},
in the form \texttt{Firstname Lastname, pseud.}\ (creasey:ashe:blast,
creasey:morton:hide, creasey:york:death).  This latter method will
allow you to keep all references to one author's work under different
pseudonyms grouped together in the bibliography, as recommended by the
\emph{Manual}.

%\enlargethispage{-\baselineskip}

\mybigspace In \mymarginpar{\textbf{authortype}}
\textsf{biblatex-chicago}, this field serves a function very much in
keeping with the spirit of standard \textsf{biblatex}, if not with its
letter.  Instead of allowing you to change the string used to identify
an author, the field allows you to indicate when an author is
anonymous, that is, when his or her name doesn't appear on the title
page of the work you are citing.  As I've just detailed under
\textsf{author}, the \emph{Manual} generally discourages the use of
\enquote{Anonymous} as an author, preferring that you simply omit it.
If, however, the name of the author is known or guessed at, then
you're supposed to enclose that name within square brackets, which is
exactly what \textsf{biblatex-chicago} does for you when you put
either \texttt{anon} (author known) or \texttt{anon?} (author guessed
at) in the \textsf{authortype} field.  (Putting the square brackets in
yourself doesn't work right, hence this mechanism.)  The macros test
for these \emph{exact} strings, so check your typing if you don't see
the brackets.  Assuming the strings are correct,
\textsf{biblatex-chicago-notes} will also automatically do the right
thing in the short note form.  Cf.\ \textsf{author}.  (See 17.33--34;
cook:sotweed, horsley:prosodies.)

\mybigspace For \mymarginpar{\textbf{bookauthor}} the most part, as in
\textsf{biblatex}, a \textsf{bookauthor} is the author of a
\textsf{booktitle}, so that, for example, if one chapter in a book has
different authorship from the book as a whole, you can include that
fact in a reference (17.75; will:cohere).  Keep in mind, however, that
the entry type for introductions, forewords and afterwords
(\textsf{suppbook}) uses \textsf{bookauthor} as the author of
\textsf{title} (polakow:afterw, prose:intro).

\mybigspace This, \mymarginpar{\vspace{-12pt}\textbf{bookpagination}}
a standard \textsf{biblatex} field, allows you automatically to prefix
the appropriate string to information you provide in a \textsf{pages}
field.  If you leave it blank, the default is to print no identifying
string (the equivalent of setting it to \texttt{none}), as this is the
practice the \emph{Manual} recommends for nearly all page numbers.
Even if the numbers you cite aren't pages, but it is otherwise clear
from the context what they represent, you can still leave this blank.
If, however, you specifically need to identify what sort of unit the
\textsf{pages} field represents, then you can either hand-format that
field yourself, or use one of the provided bibstrings in the
\textsf{bookpagination} field.  These bibstrings currently are
\texttt{column,} \texttt{line,} \texttt{paragraph,} \texttt{page,}
\texttt{section,} and \texttt{verse}, all of which are used by
\textsf{biblatex's} standard styles.

\mylittlespace There are two points that may need explaining here.
First, all the bibstrings I have just listed follow the Chicago
specification, which may be confusing if they don't produce the
strings you expect.  Second, remember that \textsf{bookpagination}
applies only to the \textsf{pages} field --- if you need to format a
citation's \textsf{postnote} field, then you must use
\textsf{pagination}, which see (15.45--46, 17.128--138).

\mybigspace The \mymarginpar{\textbf{booksubtitle}} subtitle for a
\textsf{booktitle}.  See the next entry for further information.

\mybigspace In \mymarginpar{\textbf{booktitle}} the
\textsf{bookinbook}, \textsf{inbook}, \textsf{incollection},
\textsf{inproceedings}, and \textsf{letter} entry types, the
\textsf{booktitle} field holds the title of the larger volume in which
the \textsf{title} itself is contained as one part.  It is important
not to confuse this with the \textsf{maintitle}, which holds the more
general title of multiple volumes, e.g., \emph{Collected Works}.  It
is perfectly possible for one .bib file entry to contain all three
sorts of title (euripides:orestes, plato:republic:gr).  You may also
find a \textsf{booktitle} in other sorts of entries (e.g.,
\textsf{book} or \textsf{collection}), but there it will almost
invariably be providing information for the \textsc{Bib}\TeX\
cross-referencing apparatus (prairie:state), which I discuss below
(\textbf{crossref}).

\mybigspace An \mymarginpar{\textbf{booktitleaddon}} annex to the
\textsf{booktitle}.  It will be printed in the main text font, without
quotation marks.  If your data begins with a word that would
ordinarily only be capitalized at the beginning of a sentence, then
simply ensure that that word is in lowercase, and
\textsf{biblatex-chicago-notes} will automatically do the right thing.

%\enlargethispage{-\baselineskip}

\mybigspace This \mymarginpar{\textbf{chapter}} field holds the
chapter number, mainly useful only in an \textsf{inbook} or an
\textsf{incollection} entry where you wish to cite a specific chapter
of a book (ashbrook:brain).

\mybigspace I \mymarginpar{\textbf{commentator}} have implemented this
\textsf{biblatex} field pretty much as that package's standard styles
do, even though the \emph{Manual} doesn't actually mention it.  It may
be useful for some purposes.  Cf.\ \textsf{annotator}.

\mybigspace \textsf{Biblatex} \mymarginpar{\textbf{crossref}} uses the
standard \textsc{Bib}\TeX\ cross-referencing mechanism, and has also
introduced a modified one of its own (\textsf{xref}).  The
\textsf{crossref} field works exactly the same as it always has, while
\textsf{xref} attempts to remedy some of the deficiencies of the usual
mechanism by ensuring that child entries will inherit no data at all
from their parents.  Having said all that, a few further instructions
may be in order for users of both \textsf{biblatex} and
\textsf{biblatex-chicago}.  First, remember that fields in a
\textsf{collection} entry, for example, differ from those in an
\textsf{incollection} entry.  In order for the latter to inherit the
\textsf{booktitle} field from the former, the former needs to have
such a field defined, even though a \textsf{collection} entry has no
use itself for such an entry (see ellet:galena, keating:dearborn,
lippincott:chicago, and prairie:state).  Note also that an entry with
a \textsf{crossref} field will mechanically try to inherit all
applicable fields from the entry it cross-references.  In the case of
ellet:galena et al., you can see that this includes the
\textsf{subtitle} field found in prairie:state, which would then,
quite incorrectly, be added to the \textsf{title} of ellet:galena.  In
cases like these, you could just make sure that prairie:state didn't
contain such a field, by placing the entire title + subtitle in the
\textsf{title} field, separated by a colon.  You'd certainly need to
provide a \textsf{shorttitle} field for short footnotes, if you chose
this solution.  Alternatively, as you can see in ellet:galena, you can
just define an empty \textsf{subtitle} field to prevent it inheriting
the unwanted subtitle from prairie:state.

\mylittlespace Turning now more narrowly to
\textsf{biblatex-chicago-notes}, the \emph{Manual} (17.70) specifies
that if you cite several contributions to the same collection, all
(including the collection itself) may be listed separately in the
bibliography, which the package does automatically, using the default
inclusion threshold of 2 in the case both of \textsf{crossref}'ed and
\textsf{xref}'ed entries.  (The familiar \cmd{nocite} command may also
help in some circumstances.)  In footnotes the specification suggests
that, after a citation of any one contribution to the collection, all
subsequent contributions may, even in the first, long footnote, be
cited using a slightly shortened form, thus \enquote{avoiding
  clutter.}  In the bibliography the abbreviated form is appropriate
for all the child entries.  The current version of
\textsf{biblatex-chicago-notes} implements these instructions, but
only if you use a \textsf{crossref} or an \textsf{xref} field, and
only in \textsf{incollection}, \textsf{inproceedings}, or
\textsf{letter} entries (on the last named, see just below).  If you
look at ellet:galena, keating:dearborn, lippincott:chicago, and
prairie:state you'll see this mechanism in action in both notes and
bibliography.  If you wish to disable this, then simply don't use a
\textsf{crossref} or \textsf{xref} field in your entries.

\mylittlespace There is a subtlety involved in this mechanism that I
should address here.  Andrew Goldstone has pointed out to me some
inaccuracies in the package's treatment of these abbreviated
citations, both in notes and bibliography.  Most of the changes I've
made won't affect users in any way, only the actual printed output,
but if you refer separately to chapters in a single-author
\textsf{book}, then the shortened part of the reference, to the whole
book, won't repeat the author's name before the title of the whole.
If, however, you refer separately to parts of a \textsf{collection} or
\textsf{proceedings}, even when the \textsf{editor} of the
\textsf{collection} is the same as the \textsf{author} of an essay in
the collection, you will see the name repeated before the abbreviated
part referencing the whole parent volume.  Because the code tests for
entry type, if you don't use \textsf{collection} or
\textsf{proceedings} for the whole volume, you'll not get the repeated
name, so there may be corner cases where careful choice of the parent
entry type gets you what you want.

%\enlargethispage{-\baselineskip}

\mylittlespace A published collection of letters requires a somewhat
different treatment (17.78).  If you cite more than one letter from
the same collection, then the \emph{Manual} specifies that only the
collection itself should appear in the bibliography.  In footnotes,
you can use the \textsf{letter} entry type, documented above, for
each individual letter, while the collection as a whole may well
require a \textsf{book} entry.  I have, after some consideration,
implemented the system of shortened references in \textsf{letter}
entries, even though the \emph{Manual} doesn't explicitly require it.
As with \textsf{incollection} and \textsf{inproceedings}, mere use of
a \textsf{crossref} or \textsf{xref} field will activate this
mechanism, while avoidance of said fields will disable it.  (See
white:ross:memo, white:russ, and white:total, for examples of the
\textsf{xref} field in action in this way, and please note that the
second of these entries is entirely fictitious, provided merely for
the sake of example.)  How then to keep the individual letters from
appearing in the bibliography?  The simplest mechanism is one provided
by \textsf{biblatex}, which involves the \textsf{keywords} field.
Choose a keyword for any entry you wish excluded from the bibliography
--- I've chosen \texttt{original}, for reasons that will become
clearer later --- then in the optional argument to the
\cmd{printbibliography} command in your document include, e.g.,
\texttt{notkeyword=original}.  (Cf.\ \textbf{keywords} and
\textbf{userf}.)

\mylittlespace If you look closely at the .bib entries for
white:ross:memo and white:russ, you'll see that, despite using
\textsf{xref} instead of \textsf{crossref}, the notes referring to
them inherit data from the parent (white:total).  The citation
mechanism is making a separate call to the parent's .bib entry,
formatting the information there to fill out the bare data provided by
the child, but this only happens in \textsf{letter},
\textsf{incollection}, and \textsf{inproceedings} entries.  It is
perfectly possible that other sorts of entries may make use of
\textsf{crossref} or \textsf{xref} fields --- \textsf{inbook} and
\textsf{bookinbook} come to mind --- but such entries will not result in
the activation of shortened references in notes and bibliography, nor,
when using \textsf{xref}, in the inheritance I have just pointed out.
This is how I interpret the specification, though I'm open to
persuasion on this score.

\mylittlespace I should also take this opportunity to mention that you
need to be careful when using the \textsf{shorthand} field in
conjunction with the \textsf{crossref} or \textsf{xref} fields,
bearing in mind the complicated questions of inheritance posed by all
such cross-references, most especially in \textsf{letter},
\textsf{incollection}, and \textsf{inproceedings} entries.  A
\textsf{shorthand} field in a parent entry is, at least in the current
state of \textsf{biblatex-chicago-notes}, a bad idea.

\mybigspace This \mymarginpar{\textbf{date}} field may be used to
specify an item's complete date of publication, in \textsc{iso}8601
format, i.e., \texttt{yyyy-mm-dd}.  It may also be used to specify a
date range, according to Lehman's instructions in �~2.3.8 of
\textsf{biblatex.pdf}.  Please be aware, however, that \textsf{Biber}
is somewhat more exacting when parsing the \textsf{date} field than
\textsc{Bib}\TeX, so a field looking like \texttt{1968/75} will simply
be ignored --- you need \texttt{1968/1975} instead.  If you want to
present a more compressed year range, or more generally if only part
of a date is required, then the \textsf{month} and \textsf{year}
fields may be more convenient.  The latter may be particularly useful
in some entries because it can hold more than just numerical data, in
contrast to \textsf{date} itself.  Cf.\ the \textsf{misc} entry type
in section~\ref{sec:entrytypes} above for how to use this field to
distinguish between two classes of archival material.  See also
\textsf{origdate} and \textsf{urldate}.

\mylittlespace (Users of the Chicago author-date style who wish to
minimize the labor needed to convert a .bib database for the notes \&\
bibliography style should be aware that, in this release, the latter
style includes compatibility code for the \texttt{cmsdate} (silently
ignored) and \texttt{switchdates} options, along with the mechanism
for reversing \textsf{date} and \textsf{origdate}.  This means that
you can, in theory, leave all of this alone in your .bib file when
making the conversion, though I'm retaining the right to revoke this
if the code in question demonstrably interferes with the functioning
of the notes \&\ bibliography style.)

\enlargethispage{\baselineskip}

\mybigspace This \mymarginpar{\textbf{day}} field, as of
\textsf{biblatex} 0.9, is obsolete, and will be ignored if you use it
in your .bib files.  Use \textsf{date} instead.

\mybigspace Standard \mymarginpar{\textbf{doi}} \textsf{biblatex}
field.  The Digital Object Identifier of the work, which the
\emph{Manual} suggests you can use \enquote{in place of page numbers
  or other locators} (17.181; friedman:learn\-ing).  Cf.\
\textsf{url}.

\mybigspace Standard \mymarginpar{\textbf{edition}} \textsf{biblatex}
field.  If you enter a plain cardinal number, \textsf{biblatex} will
convert it to an ordinal (chicago:manual:15), followed by the
appropriate string.  Any other sort of edition information will be
printed as is, though if your data begins with a word (or
abbreviation) that would ordinarily only be capitalized at the
beginning of a sentence, then simply ensure that that word (or
abbreviation) is in lowercase, and \textsf{biblatex-chicago-notes}
will automatically do the right thing (babb:peru, times:guide).  In
most situations, the \emph{Manual} generally recommends the use of
abbreviations in both bibliography and notes, but there is room for
the user's discretion in specific citations (emerson:nature).

\mylittlespace In a previous release of
\textsf{biblatex-chicago-notes}, I introduced the \textsf{userd} field
to hold this non-numeric information, as \textsf{biblatex} only
accepted an integer in the \textsf{edition} field, but this changed in
version 0.8.  The \textsf{userd} field is now obsolete, and will be
silently ignored.

\mybigspace As \mymarginpar{\textbf{editor}} far as possible, I have
implemented this field as \textsf{biblatex}'s standard styles do, but
the requirements specified by the \emph{Manual} present certain
complications that need explaining.  Lehman points out in his
documentation that the \textsf{editor} field will be associated with a
\textsf{title}, a \textsf{booktitle}, or a \textsf{maintitle},
depending on the sort of entry.  More specifically,
\textsf{biblatex-chicago} associates the \textsf{editor} with the most
comprehensive of those titles, that is, \textsf{maintitle} if there is
one, otherwise \textsf{booktitle}, otherwise \textsf{title}, if the
other two are lacking.  In a large number of cases, this is exactly
the correct behavior (adorno:benj, centinel:letters,
plato:republic:gr, among others).  Predictably, however, there are
numerous cases that require, for example, an additional editor for one
part of a collection or for one volume of a multi-volume work.  For
these cases I have provided the \textsf{namea} field.  You should
format names for this field as you would for \textsf{author} or
\textsf{editor}, and these names will always be associated with the
\textsf{title} (donne:var).

\mylittlespace As you will see below, I have also provided a
\textsf{nameb} field, which holds the translator of a given
\textsf{title} (euripides:orestes).  If \textsf{namea} and
\textsf{nameb} are the same, \textsf{biblatex-chicago} will
concatenate them, just as \textsf{biblatex} already does for
\textsf{editor}, \textsf{translator}, and \textsf{namec} (i.e., the
compiler).  Furthermore, it is conceivable that a given entry will
need separate editors for each of the three sorts of title.  For this,
and for various other tricky situations, there is the \cmd{partedit}
macro (and its siblings), designed to be used in a \textsf{note} field
or in one of the \textsf{titleaddon} fields (chaucer:liferecords).
(Because the strings identifying an editor differ in notes and
bibliography, one can't simply write them out in such a field, hence
the need for a macro, which I discuss further in the commands section
below [\ref{sec:formatcommands}].)  Cf.\ \textsf{namea},
\textsf{nameb}, \textsf{namec}, and \textsf{translator}.

\mybigspace The \mymarginpar{\textbf{editora\\editorb\\editorc}} newer
releases of \textsf{biblatex} provide these fields as a means to
specify additional contributors to texts in a number of editorial
roles.  In the Chicago styles they seem most relevant for the
audiovisual types, especially \textsf{music} and \textsf{video}, where
they help to identify conductors, directors, producers, and
performers.  To specify the role, use the fields \textsf{editoratype},
\textsf{editorbtype}, and \textsf{editorctype}, which see.  (Cf.\
bernstein:shostakovich, handel:messiah.)

%\enlargethispage{\baselineskip}

\mybigspace Normally, \mymarginpar{\textbf{editortype}} with the
exception of the \textsf{article} and \textsf{review} types,
\textsf{biblatex-chicago-notes} will automatically find a name to put
at the head of an entry, starting with an \textsf{author}, and
proceeding in order through \textsf{editor}, \textsf{translator}, and
\textsf{namec} (the compiler).  If all four are missing, then the
\textsf{title} will be placed at the head.  (In \textsf{article} and
\textsf{review} entries with a \texttt{magazine}
\textsf{entrysubtype}, a missing author immediately prompts the use of
\textsf{journaltitle} at the head of an entry.  See above under
\textsf{article} for details.)  The \textsf{editortype} field, added
in \textsf{biblatex 0.7}, provides even greater flexibility, giving
you the ability to put a compiler at the head of an entry without
using \textsf{namec}, freeing you from the need to use a
\textsf{sortkey} and a \textsf{shortauthor}.  You can do this even
though an author is named (eliot:pound shows this mechanism in action
for a standard editor, rather than a compiler).  Two things are
necessary for this to happen.  First, in the \textsf{options} field
you need to set \texttt{useauthor=false}, then you need to put the
name you wish to see at the head of your entry into the
\textsf{editor} or the \textsf{namea} field.  If the \enquote{editor}
is in fact a compiler, then you need to put \texttt{compiler} into the
\textsf{editortype} field, and \textsf{biblatex} will print the
correct string after the name in both the bibliography and in the long
note form.

\mylittlespace There are a few details of which you need to be aware.
Because \textsf{biblatex-chicago} has added the \textsf{namea} field,
which gives you the ability to identify the editor specifically of a
\textsf{title} as opposed to a \textsf{maintitle} or a
\textsf{booktitle}, the \textsf{editortype} mechanism checks first to
see whether a \textsf{namea} is defined.  If it is, that name will be
used at the head of the entry, if it isn't it will go ahead and look
for an \textsf{editor}.  When the \textsf{editor} field is used,
\textsf{biblatex}'s sorting algorithms will work properly, and also
its \textsf{labelname} mechanism, meaning that a shortened form of the
\textsf{editor} will be used in the short note form.  If, however, the
\textsf{namea} field provides the name, then your .bib entry will need
to have a \textsf{sortkey} field to aid in alphabetizing, and it will
also need a \textsf{shorteditor} defined to help with the short note
form, not a \textsf{shortauthor}, ruled out because
\texttt{useauthor=false}.

\mylittlespace In \textsf{biblatex} 0.9 Lehman reworked the string
concatenation mechanism, for reasons he outlined in his RELEASE file,
and I have followed his lead.  In short, if you define the
\textsf{editortype} field, then concatenation is turned off, even if
the name of the \textsf{editor} matches, for example, that of the
\textsf{translator}.  In the absence of an \textsf{editortype}, the
usual mechanisms remain in place, that is, if the \textsf{editor}
exactly matches a \textsf{translator} and/or a \textsf{namec}, or
alternatively if \textsf{namea} exactly matches a \textsf{nameb}
and/or a \textsf{namec}, then \textsf{biblatex} will print the
appropriate strings.  The \emph{Manual} specifically (17.41)
recommends not using these identifying strings in the short note form,
and \textsf{biblatex-chicago-notes} follows their recommendation.  If
you nevertheless need to provide such a string, you'll have to do it
manually in the \textsf{shorteditor} field, or perhaps, in a different
sort of entry, in a \textsf{shortauthor} field.

\mylittlespace It may also be worth noting that because of certain
requirements in the specification -- absence of an \textsf{author},
for example -- the \texttt{useauthor} mechanism won't work properly in
the following entry types: \textsf{collection}, \textsf{letter},
\textsf{patent}, \textsf{periodical}, \textsf{proceedings},
\textsf{review}, \textsf{suppbook}, \textsf{suppcollection}, and
\textsf{suppperiodical}.

\mybigspace These
\mymarginpar{\textbf{editoratype\\editorbtype\\editorctype}} fields
identify the exact role of the person named in the corresponding
\textsf{editor[a-c]} field.  Note that they are not part of the string
concatenation mechanism.  I have implemented them just as the standard
styles do, and they have now found a use particularly in
\textsf{music} and \textsf{video} entries.  Cf.\
bernstein:shostakovich, handel:messiah.

\mybigspace Standard \mymarginpar{\textbf{eid}} \textsf{biblatex}
field, providing a string or number some journals use uniquely to
identify a particular article.  Only applicable to the
\textsf{article} entry type.  Not typically required by the
\emph{Manual}.

\mybigspace Standard \mymarginpar{\textbf{entrysubtype}} and very
powerful \textsf{biblatex} field, left undefined by the standard
styles.  In \textsf{biblatex-chicago-notes} it has four very specific
uses, the first three of which I have designed in order to maintain,
as much as possible, backward compatibility with the standard styles.
First, in \textsf{article}, \textsf{periodical}, and \textsf{review}
entries, the field allows you to differentiate between scholarly
\enquote{journals,} on the one hand, and \enquote{magazines} and
\enquote{newspapers} on the other.  Usage is fairly simple: you need
to put the exact string \texttt{magazine} into the
\textsf{entrysubtype} field if you are citing one of the latter two
types of source, whereas if your source is a \enquote{journal,} then
you need do nothing.

\enlargethispage{\baselineskip}

\mylittlespace The second use involves references to works from
classical antiquity and, according to the \emph{Manual}, from the
Middle Ages, as well.  When you cite such a work using the traditional
divisions into books, sections, lines, etc., divisions which are
presumed to be the same across all editions, then you need to put the
exact string \texttt{classical} into the \textsf{entrysubtype} field.
This has no effect in long notes or in the bibliography, but it does
affect the formatting of short notes, where it suppresses some of the
punctuation.  Ordinarily, you will use this toggle in a \textsf{book}
or a \textsf{bookinbook} entry, but it is possible that a journal
might well also present an edition of such a work.  Given the
tradition of using italics for the titles of such works, this may
require using a \textsf{titleaddon} field (with hand formatting)
instead of a \textsf{title}.  If you wish to reference a classical or
medieval work by the page numbers of a particular, non-standard
edition, then you shouldn't use the \textsf{entrysubtype} toggle.
Also, and the specification is reasonably clear about this, works from
the Renaissance and later, even if cited by the traditional divisions,
have short notes formatted normally, and therefore don't need an
\textsf{entrysubtype} field.  (See \emph{Manual} 17.250--262;
aristotle:metaphy:gr, plato:republic:gr; euripides:orestes is an
example of a translation cited by page number in a modern edition.)

\mylittlespace The third use occurs in \textsf{misc} entries.  If such
an entry contains no \textsf{entrysubtype} field, then the citation
will be treated just as the standard \textsf{biblatex} styles would,
including the use of italics for the \textsf{title}.  Any string at
all in \textsf{entrysubtype} tells \textsf{biblatex-chicago-notes} to
treat the source as part of an unpublished archive.  A \textsf{misc}
entry with \textsf{entrysubtype} defined is the least formatted of all
those specified by the \emph{Manual} --- see
section~\ref{sec:entrytypes} above under \textbf{misc} for all the
details on how these citations work.

\mylittlespace Fourth, and finally, the field can be defined in the
new \textsf{artwork} entry type in order to refer to a work from
antiquity whose title you do not wish to be italicized.  Please see
the documentation of \textsf{artwork} above for the details.

\mybigspace This \mymarginpar{\textbf{eventdate}} is a standard
\textsf{biblatex} field, added recently to the \textbf{music} entry
type in case users need it to identify a particular recording session
or concert.  See the documentation of that type above.  The field will
currently be ignored in any other sort of entry.

\mybigspace As \mymarginpar{\textbf{foreword}} with the
\textsf{afterword} field above, \textsf{foreword} will in general
function as it does in standard \textsf{biblatex}.  Like
\textsf{afterword} (and \textsf{introduction}), however, it has a
special meaning in a \textsf{suppbook} entry, where you simply need to
define it somehow (and leave \textsf{afterword} and
\textsf{introduction} undefined) to make a foreword the focus of a
citation.

\mybigspace A \mymarginpar{\textbf{holder}} standard \textsf{biblatex}
field for identifying a \textsf{patent}'s holder(s), if they differ
from the \textsf{author}.  The \emph{Manual} has nothing to say on the
subject, but \textsf{biblatex-chicago-notes} prints it (them), in
parentheses, just after the author(s).

\enlargethispage{\baselineskip}

\mybigspace Standard \mymarginpar{\textbf{howpublished}}
\textsf{biblatex} field, mainly applicable in the \textsf{booklet}
entry type, where it replaces the \textsf{publisher}.  I have also
retained it in the \textsf{misc} and \textsf{unpublished} entry types,
for historical reasons, and either it or \textsf{pubstate} can be used
in \textsf{music} entries to clarify publication details.

\mybigspace Standard \mymarginpar{\textbf{institution}}
\textsf{biblatex} field.  In the \textsf{thesis} entry type, it will
usually identify the university for which the thesis was written,
while in a \textsf{report} entry it may identify any sort of
institution issuing the report.

\mybigspace As \mymarginpar{\textbf{introduction}} with the
\textsf{afterword} and \textsf{foreword} fields above,
\textsf{introduction} will in general function as it does in standard
\textsf{biblatex}.  Like those fields, however, it has a special
meaning in a \textsf{suppbook} entry, where you simply need to define
it somehow (and leave \textsf{afterword} and \textsf{foreword}
undefined) to make an introduction the focus of a citation.

\mybigspace Standard \mymarginpar{\textbf{isbn}} \textsf{biblatex}
field, for providing the International Standard Book Number of a
publication.  Not typically required by the \emph{Manual}.

\mybigspace Standard \mymarginpar{\textbf{isrn}} \textsf{biblatex}
field, for providing the International Standard Technical Report
Number of a report.  Only relevant to the \textsf{report} entry type,
and not typically required by the \emph{Manual}.

\mybigspace Standard \mymarginpar{\textbf{issn}} \textsf{biblatex}
field, for providing the International Standard Serial Number of a
periodical in an \textsf{article} or a \textsf{periodical} entry.  Not
typically required by the \emph{Manual}.

\mybigspace Standard \mymarginpar{\textbf{issue}} \textsf{biblatex}
field, designed for \textsf{article}, \textsf{periodical}, or
\textsf{review} entries identified by something like \enquote{Spring}
or \enquote{Summer} rather than by the usual \textsf{month} or
\textsf{number} fields (brown:bremer).

\mybigspace The \mymarginpar{\textbf{issuesubtitle}} subtitle for an
\textsf{issuetitle} --- see next entry.

\mybigspace Standard \mymarginpar{\textbf{issuetitle}}
\textsf{biblatex} field, intended to contain the title of a special
issue of any sort of periodical.  If the reference is to one article
within the special issue, then this field should be used in an
\textsf{article} entry (conley:fifthgrade), whereas if you are citing
the entire issue as a whole, then it would go in a \textsf{periodical}
entry, instead (good:wholeissue).  The \textsf{note} field is the
proper place to identify the type of issue, e.g.,\ \texttt{special
  issue}, with the initial letter lower-cased to enable automatic
contextual capitalization.

\mybigspace The \mymarginpar{\textbf{journalsubtitle}} subtitle for a
\textsf{journaltitle} --- see next entry.

\mybigspace Standard \mymarginpar{\textbf{journaltitle}}
\textsf{biblatex} field, replacing the standard \textsc{Bib}\TeX\
field \textsf{journal}, which, however, still works as an alias.  It
contains the name of any sort of periodical publication, and is found
in the \textsf{article} and \textsf{review} entry types.  In the case
where a piece in an \textsf{article} or \textsf{review}
(\textsf{entrysubtype} \texttt{magazine}) doesn't have an author,
\textsf{biblatex-chicago-notes} provides for this field to be used as
the author.  See above (section~\ref{sec:entrytypes}) under
\textbf{article} for details.  The lakeforester:pushcarts and
nyt:trevorobit entries in \textsf{notes-test.bib} will give you some
idea of how this works.

%\enlargethispage{\baselineskip}

\mybigspace This \mymarginpar{\textbf{keywords}} field is
\textsf{biblatex}'s extremely powerful and flexible technique for
filtering bibliography entries, allowing you to subdivide a
bibliography according to just about any criteria you care to invent.
See \textsf{biblatex.pdf} (3.10.4) for thorough documentation.  In
\textsf{biblatex-chicago}, the field can provide a convenient means to
exclude certain entries from making their way into a bibliography.  We
have already seen (\textbf{letter}, above) how the \emph{Manual}
(17.78) requires, in the case of published collections of letters,
that when more than one letter from the same collected is cited, the
bibliography should contain only a reference to the collection as a
whole (white:ross:memo, white:russ, white:total).  Similarly, when
citing both an original text and its translation (see \textbf{userf},
below), the \emph{Manual} (17.66) suggests including the original at
the end of the translation's bibliography entry, a procedure which
requires that the original not also be printed as a separate
bibliography entry (furet:passing:eng, furet:passing:fr,
aristotle:metaphy:trans, aristotle:metaphy:gr).  Finally, citations of
well-known reference works (like the \emph{Encyclopaedia Britannica},
for example), need only be presented in notes, and not in the
bibliography (17.238--239; ency:britannica, wikiped:bibtex; see
\textsf{inreference}, above). In all these cases, I have suggested the
inclusion of \texttt{original} in the \textsf{keywords} field, along
with a \texttt{notkeyword=original} in the optional argument to the
\cmd{printbibliography} command, though of course you can choose any
key you wish.

\mybigspace A \mymarginpar{\textbf{language}} standard
\textsf{biblatex} field, designed to allow you to specify the
language(s) in which a work is written.  As a general rule, the
Chicago style doesn't require you to provide this information, though
it may well be useful for clarifying the nature of certain works, such
as bilingual editions, for example.  There is at least one situation,
however, when the \emph{Manual} does specify this data, and that is
when the title of a work is given in translation, even though no
translation of the work has been published, something that might
happen when a title is in a language deemed to be unparseable by a
majority of your expected readership (17.65--67, 17.166, 17.177;
pirumova, rozner:liberation).  In such a case, you should provide the
language(s) involved using this field, connecting multiple languages
using the keyword \texttt{and}.  (I have retained \textsf{biblatex's}
\cmd{bibstring} mechanism here, which means that you can use the
standard bibstrings or, if one doesn't exist for the language you
need, just give the name of the language, capitalized as it should
appear in your text.  You can also mix these two modes inside one
entry without apparent harm.)

\mylittlespace An alternative arrangement suggested by the
\emph{Manual} is to retain the original title of a piece but then to
provide its translation, as well.  If you choose this option, you'll
need to make use of the \textbf{usere} field, on which see below.  In
effect, you'll probably only ever need to use one of these two fields
in any given entry, and in fact \textsf{biblatex-chicago-notes} will
only print one of them if both are present, preferring \textsf{usere}
over \textsf{language} for this purpose (see kern and weresz).  Note
also that both of these fields are universally associated with the
\textsf{title} of a work, rather than with a \textsf{booktitle} or a
\textsf{maintitle}.  If you need to attach a language or a translation
to either of the latter two, you could probably manage it with special
formatting inside those fields themselves.

\mybigspace I \mymarginpar{\textbf{lista}} intend this field
specifically for presenting citations from reference works that are
arranged alphabetically, where the name of the item rather than a page
or volume number should be given.  The field is a \textsf{biblatex}
list, which means you should separate multiple items with the keyword
\texttt{and}.  Each item receives its own set of quotation marks, and
the whole list will be prefixed by the appropriate string
(\enquote{s.v.,} \emph{sub verbo}, pl.\ \enquote{s.vv.}).
\textsf{Biblatex-chicago-notes} will only print such a field in a
\textsf{book} or an \textsf{inreference} entry, and you should look at
the documentation of these entry types for further details.  (See
\emph{Manual} 17.238--239; ency:britannica, grove:sibelius,
times:guide, wikiped:bibtex.)

\mybigspace This \mymarginpar{\textbf{location}} is
\textsf{biblatex}'s version of the usual \textsc{Bib}\TeX\ field
\textsf{address}, though the latter is accepted as an alias if that
simplifies the modification of older .bib files.  According to the
\emph{Manual} (17.99), a citation usually need only provide the first
city listed on any title page, though a list of cities separated by
the keyword \enquote{\texttt{and}} will be formatted appropriately.
If the place of publication is unknown, you can use
\cmd{autocap\{n\}.p.}\ instead (17.102), though in many or even most
cases this isn't strictly necessary (17.32--34; virginia:plantation).
For all cities, you should use the common English version of the name,
if such exists (17.101).

%\enlargethispage{-\baselineskip}

\mylittlespace Two more details need explanation here.  In
\textsf{article}, \textsf{periodical}, and \textsf{review} entries,
there is usually no need for a \textsf{location} field, but
\enquote{if a journal might be confused with another with a similar
  title, or if it might not be known to the users of a bibliography,}
then this field can present the place or institution where it is
published (17.174, 17.196; lakeforester:pushcarts, kimluu:diethyl, and
garrett).  Less predictably, it is here that \emph{Manual} indicates
that a particular book is a reprint edition (17.123), so in such a
case you can use the \textsf{biblatex-chicago} macro \cmd{reprint},
followed by a comma, space, and the location (aristotle:metaphy:gr,
schweitzer:bach).  (You can also now, somewhat more simply, just put
the string \texttt{reprint} into the \textsf{pubstate} field to
achieve the same result.  See the \textsf{pubstate} documentation
below.)  The \textsf{origdate} field may be used to give the original
date of publication, and of course more complicated situations should
usually be amenable to inclusion in the \textsf{note} field
(emerson:nature).

\mybigspace The \mymarginpar{\textbf{mainsubtitle}} subtitle for a
\textsf{maintitle} --- see next entry.

\mybigspace The \mymarginpar{\textbf{maintitle}} main title for a
multi-volume work, e.g., \enquote{Opera} or \enquote{Collected Works.}
(See donne:var, euripides:orestes, harley:cartography, lach:asia,
pelikan:chris\-tian, and plato:republic:gr.)

%\enlargethispage{\baselineskip}

\mybigspace An \mymarginpar{\textbf{maintitleaddon}} annex to the
\textsf{maintitle}, for which see previous entry.  Such an annex would
be printed in the main text font.  If your data begins with a word
that would ordinarily only be capitalized at the beginning of a
sentence, then simply ensure that that word is in lowercase, and
\textsf{biblatex-chicago-notes} will automatically do the right thing.

\mybigspace Standard \mymarginpar{\textbf{month}} \textsf{biblatex}
field, containing the month of publication.  This should be an
integer, i.e., \texttt{month=\{3\}} not \texttt{month=\{March\}}.  See
\textsf{date} for more information.

\mybigspace This \mymarginpar{\textbf{namea}} is one of the fields
\textsf{biblatex} provides for style writers to use, but which it
leaves undefined itself.  In \textsf{biblatex-chicago} it contains the
name(s) of the editor(s) of a \textsf{title}, if the entry has a
\textsf{booktitle} or \textsf{maintitle}, or both, in which situation
the \textsf{editor} would be associated with one of these latter
fields (donne:var).  You should present names in this field exactly as
you would those in an \textsf{author} or \textsf{editor} field, and
the package will concatenate this field with \textsf{nameb} if they
are identical.  See under \textbf{editor} above for the full details.
Cf.\ also \textsf{nameb}, \textsf{namec}, \textsf{translator}, and the
macros \cmd{partedit}, \cmd{parttrans}, \cmd{parteditandtrans},
\cmd{partcomp}, \cmd{parteditandcomp}, \cmd{parttransandcomp}, and
\cmd{partedittransand\-comp}, for which see
section~\ref{sec:formatcommands}.

\mybigspace This \mymarginpar{\textbf{nameaddon}} field is provided by
\textsf{biblatex}, though not used by the standard styles.  In
\textsf{biblatex-chicago}, it allows you, in most entry types, to
specify that an author's name is a pseudo\-nym, or to provide either
the real name or the pseudonym itself, if the other is being provided
in the \textsf{author} field.  The abbreviation
\enquote{\texttt{pseud}.}\ (always lowercase in English) is specified,
either on its own or after the pseudo\-nym (centinel:letters,
creasey:ashe:blast, creasey:morton:hide, creasey:york:death, and
le\-carre:quest); \cmd{bibstring\{pseudonym\}} does the work for you.
See under \textbf{author} above for the full details.

\mylittlespace In the \textsf{customc} entry type, on the other hand,
which is used to create alphabetized cross-references to other
bibliography entries, the \textsf{nameaddon} field allows you to
change the default string linking the two parts of the
cross-reference.  The code automatically tests for a known bibstring,
which it will italicize.  Otherwise, it prints the string as is.

\mybigspace Like \mymarginpar{\textbf{nameb}} \textsf{namea}, above,
this is a field left undefined by the standard \textsf{biblatex}
styles.  In \textsf{biblatex-chicago}, it contains the name(s) of the
translator(s) of a \textsf{title}, if the entry has a
\textsf{booktitle} or \textsf{maintitle}, or both, in which situation
the \textsf{translator} would be associated with one of these latter
fields (euripides:orestes).  You should present names in this field
exactly as you would those in an \textsf{author} or
\textsf{translator} field, and the package will concatenate this field
with \textsf{namea} if they are identical.  See under the
\textbf{translator} field below for the full details.  Cf.\ also
\textsf{namea}, \textsf{namec}, \textsf{origlanguage},
\textsf{translator}, \textsf{userf} and the macros \cmd{partedit},
\cmd{parttrans}, \cmd{parteditandtrans}, \cmd{partcomp},
\cmd{parteditandcomp}, \cmd{parttransandcomp}, and
\cmd{partedittransandcomp} in section~\ref{sec:formatcommands}.

\mybigspace The \mymarginpar{\textbf{namec}} \emph{Manual} (17.41)
specifies that works without an author may be listed under an editor,
translator, or compiler, assuming that one is available, and it also
specifies the strings to be used with the name(s) of compiler(s).  All
this suggests that the \emph{Manual} considers this to be standard
information that should be made available in a bibliographic
reference, so I have added that possibility to the many that
\textsf{biblatex} already provides, such as the \textsf{editor},
\textsf{translator}, \textsf{commentator}, \textsf{annotator}, and
\textsf{redactor}, along with writers of an \textsf{introduction},
\textsf{foreword}, or \textsf{afterword}.  Since \textsf{biblatex.bst}
doesn't offer a \textsf{compiler} field, I have adopted for this
purpose the otherwise unused field \textsf{namec}.  It is important to
understand that, despite the analogous name, this field does not
function like \textsf{namea} or \textsf{nameb}, but rather like
\textsf{editor} or \textsf{translator}, and therefore if used will be
associated with whichever title field these latter two would be were
they present in the same entry.  Identical fields among these three
will be concatenated by the package, and concatenated too with the
(usually) unnecessary commentator, annotator and the rest.  Also
please note that I've arranged the concatenation algorithms to include
\textsf{namec} in the same test as \textsf{namea} and \textsf{nameb},
so in this particular circumstance you can, if needed, make
\textsf{namec} analogous to these two latter, \textsf{title}-only
fields.  (See above under \textbf{editortype} for details of how you
may, in certain circumstances, use that field to identify a compiler.
This method will be particularly useful if you don't need to
concatenate the \textsf{namec} with any other role, because if you use
the \textsf{editor} field \textsf{biblatex} will automatically attend
to alphabetization and name-replacement in the bibliography, and will
also provide a name for short notes.)

\mylittlespace It might conceivably be necessary at some point to
identify the compiler(s) of a \textsf{title} separate from the
compiler(s) of a \textsf{booktitle} or \textsf{maintitle}, but for the
moment I've run out of available \textsf{name} fields, so you'll have
to fall back on the \cmd{partcomp} macro or the related
\cmd{parteditandcomp}, \cmd{parttransandcomp}, and
\cmd{partedittransandcomp}, on which see Commands
(section~\ref{sec:formatcommands}) below.  (Future releases may be
able to remedy this.)  It may be as well to mention here too that of
the three names that can be substituted for the missing
\textsf{author} at the head of an entry,
\textsf{biblatex-chicago-notes} will choose an \textsf{editor} if
present, then a \textsf{translator} if present, falling back to
\textsf{namec} only in the absence of the other two, and assuming that
the fields aren't identical, and therefore to be concatenated.  In a
change from the previous behavior, these algorithms also now test for
\textsf{namea} or \textsf{nameb}, which will be used instead of
\textsf{editor} and \textsf{translator}, respectively, giving the
package the greatest likelihood of finding a name to place at the head
of an entry.  Please remember, however, that if this name is supplied
by any of the non-standard fields \textsf{name[a-c]}, then you will
need to provide a \textsf{sortkey} to assist with alphabetization in
the bibliography, and also a \textsf{shortauthor} for the short note
form.

\mybigspace As \mymarginpar{\textbf{note}} in standard
\textsf{biblatex}, this field allows you to provide bibliographic data
that doesn't easily fit into any other field.  In this sense, it's
very like \textsf{addendum}, but the information provided here will be
printed just before the publication data.  (See chaucer:alt,
chaucer:liferecords, cook:sotweed, emerson:nature, and rodman:walk for
examples of this usage in action.)  It also has a specialized use in
all the periodical types (\textsf{article}, \textsf{periodical}, and
\textsf{review}), where it holds supplemental information about a
\textsf{journaltitle}, such as \enquote{special issue}
(conley:fifthgrade, good:wholeissue).  In all uses, if your data
begins with a word that would ordinarily only be capitalized at the
beginning of a sentence, then simply ensure that that word is in
lowercase, and \textsf{biblatex-chicago-notes} will automatically do
the right thing.  Cf.\ \textsf{addendum}.

\mybigspace This \mymarginpar{\textbf{number}} is a standard
\textsf{biblatex} field, containing the number of a
\textsf{journaltitle} in an \textsf{article} or \textsf{review} entry,
the number of a \textsf{title} in a \textsf{periodical} entry, or the
volume/number of a book in a \textsf{series}.  Generally, in an
\textsf{article}, \textsf{periodical}, or \textsf{review} entry, this
will be a plain cardinal number, but in such entries
\textsf{biblatex-chicago} now does the right thing if you have a list
or range of numbers (unsigned:ranke:15).  In any \textsf{book}-like entry
the field may well contain considerably more information, including
even a reference to \enquote{2nd ser.,} for example, while the
\textsf{series} field in such an entry will contain the name of the
series, rather than a number.  This field is also the place for the
patent number in a \textsf{patent} entry.  Cf.\ \textsf{issue} and
\textsf{series}.  (See \emph{Manual} 17.90--95 and boxer:china,
palmatary:pottery, wauchope:ceramics; 17.163 and beattie:crime,
conley:fifthgrade, friedman:learning, garrett, gibbard, hlatky:hrt,
mcmillen:antebellum, rozner:liberation, warr:ellison.)

\mylittlespace \textbf{NB}: This may be an opportune place to point
out that the \emph{Manual} (17.129) prefers arabic to roman numerals
in most circumstances (chapters, volumes, series numbers, etc.), even
when such numbers might be roman in the work cited.  The obvious
exception is page numbers, in which roman numerals indicate that the
citation came from the front matter, and should therefore be retained.
Another possible exception is in references to works \enquote{with
  many and complex divisions,} in which \enquote{a mixture of roman
  and arabic} may be \enquote{easier to disentangle.}

\mybigspace A \mymarginpar{\textbf{options}} standard
\textsf{biblatex} field, for setting certain options on a per-entry
basis rather than globally.  Information about some of the more common
options may be found above under \textsf{author} and below in
section~\ref{sec:options}.  See chaucer:alt, eliot:pound,
herwign:office, lecarre:quest, and mla:style for examples of the field
in use.

\mybigspace A \mymarginpar{\textbf{organization}} standard
\textsf{biblatex} field, retained mainly for use in the \textsf{misc},
\textsf{online}, and \textsf{manual} entry types, where it may be of
use to specify a publishing body that might not easily fit in other
categories.  In \textsf{biblatex}, it is also used to identify the
organization sponsoring a conference in a \textsf{proceedings} or
\textsf{inproceedings} entry, and I have retained this as a
possibility, though the \emph{Manual} is silent on the matter.

\mybigspace This \mymarginpar{\textbf{origdate}} is a new
\textsf{biblatex} field, replacing the obsolete \textsf{origyear}, and
allowing more than one full specification for those references which
need to provide more than one date.  As with the analogous
\textsf{date} field, you provide the date (or range of dates) in
\textsc{iso}8601 format, i.e., \texttt{yyyy-mm-dd}.  In most entry
types, you would use \textsf{origdate} to provide the date of first
publication of a work, most usually needed only in the case of reprint
editions, but also recommended by the \emph{Manual} for electronic
editions of older works (17.123, 17.146--7; aristotle:metaphy:gr,
emerson:nature, james:ambassadors, schweitzer:bach).  In the
\textsf{letter} and \textsf{misc} (with \textsf{entrysubtype}) entry
types, the \textsf{origdate} identifies when a letter (or similar) was
written.  In such \textsf{misc} entries, some
\enquote{non-letter-like} materials (like interviews) need the
\textsf{date} field for this purpose, while in \textsf{letter} entries
the \textsf{date} applies to the publication of the whole collection.
If such a published collection were itself a reprint, improvisation in
the \textsf{location} field might be able to rescue the situation.
(See jackson:paulina:letter, white:ross:memo, white:russ, and
white:total for how \textsf{letter} entries usually work; creel:house
shows the field in action in a \textsf{misc} entry, while
spock:interview uses \textsf{date}.)

\mylittlespace Because the \textsf{origdate} field only accepts
numbers, some improvisation may be needed if you wish to include
\enquote{n.d.}\ (\cmd{bibstring\{nodate\}}) in an entry.  In
\textsf{letter} and \textsf{misc}, this information can be placed in
\textsf{titleaddon}, but in other entry types you may need to use the
\textsf{location} field.

\mybigspace In \mymarginpar{\textbf{origlanguage}} keeping with the
\emph{Manual}'s specifications, I have fairly thoroughly redefined
\textsf{biblatex}'s facilities for treating translations.  The
\textsf{origtitle} field isn't used, while the \textsf{language} and
\textsf{origdate} fields have been press-ganged for other duties.  The
\textsf{origlanguage} field, for its part, retains a dual role in
presenting translations in a bibliography.  The details of the
\emph{Manual}'s suggested treatment when both a translation and an
original are cited may be found below under \textbf{userf}.  Here,
however, I simply note that the introductory string used to connect
the translation's citation with the original's is \enquote{Originally
  published as,} which I suggest may well be inaccurate in a great
many cases, as for instance when citing a work from classical
antiquity, which will most certainly not \enquote{originally} have
been published in the Loeb Classical Library.  Although not, strictly
speaking, authorized by the \emph{Manual}, I have provided another way
to introduce the original text, using the \textsf{origlanguage} field,
which must be provided \emph{in the entry for the translation, not the
  original text} (aristotle:metaphy:trans).  If you put one of the
standard \textsf{biblatex} bibstrings there (enumerated below), then
the entry will work properly across multiple languages.  Otherwise,
just put the name of the language there, localized as necessary, and
\textsf{biblatex-chicago} will eschew \enquote{Originally published
  as} in favor of, e.g., \enquote{Greek edition:} or \enquote{French
  edition:}.  This has no effect in notes, where only the work cited
--- original or translation --- will be printed, but it may help to
make the \emph{Manual}'s suggestions for the bibliography more
palatable.

\mylittlespace That was the first usage, in keeping at least with the
spirit of the \emph{Manual}.  I have also, perhaps less in keeping
with that specification, retained some of \textsf{biblatex}'s
functionality for this field.  If an entry doesn't have a
\textsf{userf} field, and therefore won't be combining a text and its
translation in the bibliography, you can also use
\textsf{origlanguage} as Lehman intended it, so that instead of
saying, e.g., \enquote{translated by X,} the entry will read
\enquote{translated from the German by X.}  The \emph{Manual} doesn't
mention this, but it may conceivably help avoid certain ambiguities in
some citations.  As in \textsf{biblatex}, if you wish to use this
functionality, you have to provide \emph{not} the name of the
language, but rather a bibliography string, which may, at the time of
writing, be one of \texttt{american}, \texttt{brazilian},
\texttt{danish}, \texttt{dutch}, \texttt{english}, \texttt{french},
\texttt{german}, \texttt{greek}, \texttt{italian}, \texttt{latin},
\texttt{norwegian}, \texttt{portuguese}, \texttt{spanish}, or
\texttt{swedish}, to which I've added \texttt{russian}.

%\enlargethispage{-\baselineskip}

\mybigspace At \colmarginpar{\textbf{origlocation}} least one example
in the \emph{Manual} provides a more complete specification of a
reprinted book's original publication details than has been possible
using previous releases of \textsf{biblatex-chicago} (17.123).
Starting with this release, you can provide both an
\textsf{origlocation} and an \textsf{origpublisher} to go along with
the \textsf{origdate}, should you so wish, and all of this information
will be printed in long notes and bibliography.  You can now also use
this field in a \textsf{letter} or \textsf{misc} (with
\textsf{entrysubtype}) entry to give the place where a published or
unpublished letter was written (17.76).  (Jonathan Robinson has
suggested that the \textsf{origlocation} may in some circumstances
actually be necessary for disambiguation, his example being early
printed editions of the same material printed in the same year but in
different cities.  The new functionality should make this simple to
achieve.  Cf.\ \textsf{origdate}, \textsf{origpublisher} and
\textsf{pubstate}; schweitzer:bach.)

\mybigspace As \colmarginpar{\textbf{origpublisher}} with the
\textsf{origlocation} field just above, this new field allows you to
provide fuller original publication details for reprinted books
(17.123).  You can now provide an \textsf{origpublisher} and/or an
\textsf{origlocation} in addition to the \textsf{origdate}, and all
will be presented in long notes and bibliography.  (Cf.\
\textsf{origdate}, \textsf{origlocation}, and \textsf{pubstate};
schweitzer:bach.)

\mybigspace This \mymarginpar{\textbf{origyear}} field is, as of
\textsf{biblatex} 0.9, obsolete.  It is ignored if it appears in a
.bib file.

\mybigspace This \mymarginpar{\textbf{pages}} is the standard
\textsf{biblatex} field for providing page references.  In many
\textsf{article} and \textsf{review} entries you'll find this contains
something other than a page number, e.g. a section name or edition
specification (17.188, 17.191, 17.202; kozinn:review, nyt:obittrevor,
nyt:trevorobit).  Of course, the same may be true of almost any sort
of entry, though perhaps with less frequency.  Curious readers may
wish to look at brown:bremer (17.172) for an example of a
\textsf{pages} field used to facilitate reference to a two-part
journal article.  Cf.\ \textsf{number} for more information on the
\emph{Manual}'s preferences regarding the formatting of numerals;
\textsf{bookpagination} and \textsf{pagination} provide details about
\textsf{biblatex's} mechanisms for specifying what sort of division a
given \textsf{pages} field contains; and \textsf{usera} discusses a
different way to present the section information pertaining to a
newspaper article.

\mybigspace This, \mymarginpar{\textbf{pagination}} a standard
\textsf{biblatex} field, allows you automatically to prefix the
appropriate identifying string to information you provide in the
\textsf{postnote} field of a citation command, whereas
\textsf{bookpagination} allows you to prefix a string to the
\textsf{pages} field.  Please see \textbf{bookpagination} above for
all the details on this functionality, as aside from the difference
just mentioned the two fields are equivalent.

\mybigspace Standard \mymarginpar{\textbf{part}} \textsf{biblatex}
field, which identifies physical parts of a single logical volume in
\textsf{book}-like entries, not in periodicals.  It has the same
purpose in \textsf{biblatex-chicago-notes}, but because the
\emph{Manual} (17.88) calls such a thing a \enquote{book} and not a
\enquote{part,} the string printed in notes and bibliography will, at
least in English, be \enquote{\texttt{bk.}\hspace{-2pt}}\ instead of
the plain dot between volume number and part number
(harley:cartography, lach:asia).  This field should only be used in
association with a \textsf{volume} number, so if you need to identify
\enquote{parts} or \enquote{books} that are part of a published
\textsf{series}, for example, then you'll need to use a different
field, (which in this case would be \textsf{number}
[palmatary:pottery]).  Cf.\ \textsf{volume}.

\mybigspace Standard \mymarginpar{\textbf{publisher}}
\textsf{biblatex} field.  Remember that \enquote{\texttt{and}} is a
keyword for connecting multiple publishers, so if a publisher's name
contains \enquote{and,} then you should either use the ampersand (\&)
or enclose the whole name in additional braces.  (See \emph{Manual}
17.103--114; aristotle:metaphy:gr, cohen:schiff, creasey:ashe:blast,
dunn:revolutions.)

\mylittlespace There are, as one might expect, a couple of further
subtleties involved here.  Ordinarily, two publishers will be
separated by a forward slash in both notes and bibliography, but if a
company issues \enquote{certain books through a special publishing
  division or under a special imprint,} then the two names will be
separated by a comma, which you will need to provide in the
\textsf{publisher} field.  The \emph{Manual}'s example (17.112) is
\enquote{\texttt{Ohio University Press, Swallow Press},} which would
cause \textsf{biblatex-chicago-notes} no problems.  If a book has two
co-publishers, \enquote{usually in different countries,} (17.113) then
the simplest thing to do is to choose one, probably the nearest one
geographically.  If you feel it necessary to include both, then
levistrauss:savage demonstrates one way of doing so, using a
combination of the \textsf{publisher} and \textsf{location} fields.
Finally, if the publisher is unknown, then the \emph{Manual}
recommends (17.109) simply using the place (if known) and the date.
If for some reason you need to indicate the absence of a publisher,
the abbreviation given by the \emph{Manual} is \texttt{n.p.}, though
this can also stand for \enquote{no place.}  Some style guides
apparently suggest using \texttt{s.n.}\,(= \emph{sine nomine}) to
specify the lack of a publisher, but the \emph{Manual} doesn't mention
this.

\mybigspace Due \mymarginpar{\textbf{pubstate}} to specific
requirements in the author-date style, I have implemented this field
there as a way of providing accurate citations of reprinted books.  As
the functionality seemed useful, I have also included some of it in
\textsf{biblatex-chicago-notes}.  In previous releases you could
identify a reprint by placing \cmd{bibstring\{reprint\}} in the
\textsf{location} field, followed by a comma, and the style would
print the appropriate string in notes and bibliography.  Now, if it is
more convenient, easier to remember, or if you want to reuse your .bib
database for the author-date style, you can simply put the string
\texttt{reprint} into the \textsf{pubstate} field, and the package
will take care of everything for you.  Both of these methods will now
work just fine, but please choose only one per entry, otherwise the
string will be printed twice.  Please note, also, that this automatic
mechanism has been disabled in \textsf{music} and \textsf{video}
entries, as it isn't appropriate to those sorts of material.  In the
latter, \textsf{pubstate} will be silently ignored, whereas in the
former, for compatibility with the author-date style, the field
functions as a synonym for \textsf{howpublished}, and will be printed
verbatim.  Currently, if you put anything besides \texttt{reprint} in
the \textsf{pubstate} field of anything except \textsf{music} entries,
it too will silently be ignored, but this may change in future
releases.

\mybigspace I \mymarginpar{\textbf{redactor}} have implemented this
field just as \textsf{biblatex}'s standard styles do, even though the
\emph{Manual} doesn't actually mention it.  It may be useful for some
purposes.  Cf.\ \textsf{annotator} and \textsf{commentator}.

\mybigspace \textbf{NB:} \mymarginpar{\textbf{reprinttitle}}
\textbf{Please note that this feature is in an alpha state, and that
  I'm contemplating using a different field in the future for this
  functionality.  I include it here in the hope that it might receive
  some testing in the meantime.}  At the request of Will Small, I have
included a means of providing the original publication details of an
essay or a chapter that you are citing from a subsequent reprint,
e.g., a \emph{Collected Essays} volume.  In such a case, at least
according to the \emph{Manual} (17.73), such details needn't be
provided in notes, only in the bibliography, and then only if these
details are \enquote{of particular interest.}  The data would follow
an introductory phrase like \enquote{originally published as,} making
the problem strictly parallel to that of including details of a work
in the original language alongside the details of its translation.  I
have addressed the latter problem with the \textsf{userf} field, which
provides a sort of cross-referencing method for this purpose, and
\textsf{reprinttitle} works in \emph{exactly} the same way.  In the
.bib entry for the reprint you include a cross-reference to the cite
key of the original location using the \textsf{reprinttitle} field
(which it may help mnemonically to think of as a \enquote{reprinted
  title} field).  The main difference between the two forms is that
\textsf{userf} prints all but the \textsf{author} of the original
work, whereas \textsf{reprinttitle} suppresses both the
\textsf{author} and the \textsf{title} of the original, giving only
the more general details, beginning with, e.g., the
\textsf{journaltitle} or \textsf{booktitle} and continuing from there.
The string prefacing this information will be \enquote{Originally
  published in.}  Please see the documentation on \textsf{userf} below
for all the details on how to create .bib entries for presenting your
data.

\mybigspace A \mymarginpar{\textbf{series}} standard \textsf{biblatex}
field, usually just a number in an \textsf{article},
\textsf{periodical}, or \textsf{review} entry, almost always the name
of a publication series in \textsf{book}-like entries.  If you need to
attach further information to the \textsf{series} name in a
\textsf{book}-like entry, then the \textsf{number} field is the place
for it, whether it be a volume, a number, or even something like
\enquote{2nd ser.} or \enquote{\cmd{bibstring\{oldseries\}}.}  Of
course, you can also use \cmd{bibstring\{oldseries\}} or
\cmd{bibstring\{newseries\}} in an \textsf{article} entry, but there
you would place it in the \textsf{series} field itself.  (In fact, the
\textsf{series} field in \textsf{article}, \textsf{periodical}, and
\textsf{review} entries is one of the places where \textsf{biblatex}
allows you just to use the plain bibstring \texttt{oldseries}, for
example, rather than making you type \cmd{bibstring\{oldseries\}}.
The \textsf{type} field in \textsf{manual}, \textsf{patent},
\textsf{report}, and \textsf{thesis} entries also has this
auto-detection mechanism in place; see the discussion of
\cmd{bibstring} below for details.)  In whatever entry type, these
bibstrings produce the required abbreviation, which thankfully is the
same in both notes and bibliography.  (For books and similar entries,
see \emph{Manual} 17.90--95; boxer:china, browning:aurora,
palmatary:pottery, plato:republic:gr, wauchope:ceramics; for
periodicals, see 17.178; garaud:gatine, sewall:letter.)  Cf.\
\textsf{number} for more information on the \emph{Manual}'s
preferences regarding the formatting of numerals.

\mybigspace This \mymarginpar{\textbf{shortauthor}} is a standard
\textsf{biblatex} field, but \textsf{biblatex-chicago-notes} makes
considerably grea\-ter use of it than the standard styles.  For the
purposes of the Chicago style, the field provides the name to be used
in the short form of a footnote.  In the vast majority of cases, you
don't need to specify it, because the \textsf{biblatex} system selects
the author's last name from the \textsf{author} field and uses it in
such a reference, but in a few cases this default behavior won't work.
In books without an author and listed under an editor,
\textsf{biblatex} does the right thing and uses the surname of the
editor in a short note (zukowsky:chicago), but if the work is listed
under a compiler (or any of the non-standard names
\textsf{name[a-c]}), you need to provide that person's name in
\textsf{shortauthor}, and also remember to provide a \textsf{sortkey}
to make sure the work will be alphabetized correctly in the
bibliography.  (The current version of \textsf{biblatex} will now
automatically alphabetize by \textsf{translator} if that is the name
given at the head of an entry.)  You no longer, however, need to
provide one in an author-less \textsf{article} or \textsf{review}
entry (\textsf{entrysubtype} \texttt{magazine}), where you allow
\textsf{biblatex-chicago-notes} to use the \textsf{journaltitle} as
the author, nor in author-less \textsf{manual} entries, where the
\textsf{organization} will be so used.  The style now automatically
provides the same substitution in the short note form, though you'll
still need to help the alphabetization routines by providing a
\textsf{sortkey} field in such cases (dyna:browser, gourmet:052006,
lakeforester:pushcarts, nyt:trevorobit).

\mylittlespace As mentioned under \textsf{editortype}, the
\emph{Manual} (17.41) recommends against providing the identifying
string (e.g., ed.\ or trans.)\ in the short note form, and
\textsf{biblatex-chicago-notes} follows their recommendation.  If you
need to provide these strings in such a citation, then you'll have to
do so by hand in the \textsf{shortauthor} field, or in the
\textsf{shorteditor} field, whichever you are using.

\mybigspace Like \mymarginpar{\textbf{shorteditor}}
\textsf{shortauthor}, a field to provide a name for a short footnote,
in this case for, e.g., a \textsf{collection} entry that typically
lacks an author.  The \textsf{shortauthor} field works just as well in
most situations, but if you have set \texttt{useauthor=false} (and not
\texttt{useeditor=false}) in an entry's \textsf{options} field, then
only \textsf{shorteditor} will be recognized.  Cf.\
\textsf{editortype}, above.

%\enlargethispage{\baselineskip}

\mybigspace This \mymarginpar{\textbf{shorthand}} is
\textsf{biblatex}'s mechanism for using abbreviations in place of the
usual short note form, and I've left it effectively unmodified in
\textsf{biblatex-chicago-notes}, apart from a few formatting tweaks.
Any entry which contains such a field will produce a normal first
note, either long or short according to your package options,
informing the reader that the work will hereafter be cited by this
abbreviation.  As in \textsf{biblatex}, the \cmd{printshorthands}
command will produce a formatted list of abbreviations for reference
purposes, a list which the \emph{Manual} suggests should be placed
either in the front matter (when using footnotes) or before the
endnotes, in case these are used.  (See 16.39--40, and also
\textsf{biblatex.pdf} for more information.)

\mylittlespace As I mentioned above under \textbf{crossref}, extra
care is needed when using shorthands with cross-references, and I
would avoid them in all parent entries, at least in the current state
of \textsf{biblatex-chicago-notes}.

\mybigspace When \mymarginpar{\textbf{shorthandintro}} you include a
\textsf{shorthand} in an entry, it will ordinarily appear the first
time you cite the work, at the end of a long note, prefaced by the
phrase \enquote{Henceforth cited as.}  With this standard
\textsf{biblatex} field, you can change that phrase to suit your
needs.  Please note, first, that you need to include the shorthand in
this field as you intend it to appear and, second, that you still need
the \textsf{shorthand} field present in order to ensure the
appropriate presentation of the shorthand in later citations and in
the list of shorthands.

\mybigspace A \colmarginpar{\textbf{shorttitle}} standard
\textsf{biblatex} field, primarily used to provide an abbreviated
title for short notes.  In \textsf{biblatex-chicago-notes}, you need
to take particular care with \textsf{letter} entries, where, as
explained above, the \emph{Manual} requires a special format
(\enquote{\texttt{to Recipient}}).  (See 17.76--78;
jackson:paulina:letter, white:ross:memo, white:russ.)  Some
\textsf{misc} entries (with an \textsf{entrysubtype}) also need
special attention.  (See creel:house, where the full \textsf{title} is
used as the \textsf{shortauthor} + \textsf{shorttitle} by using
\cmd{headlesscite} commands.  Placing \cmd{isdot} into the
\textsf{shortauthor} field no longer works in \textsf{biblatex} 1.6
and later, so be sure to check your .bib files when you upgrade.)
Remember, also, that the generic titles in \textsf{review} and
\textsf{misc} entries may not want capitalization in all contexts, so,
as with the \textsf{title} field, if you begin a \textsf{shorttitle}
with a lowercase letter the style will do the right thing
(barcott:review, bundy:macneil, Clemens:letter, kozinn:review,
ratliff:review, unsigned:ranke:15).

\mybigspace A \mymarginpar{\textbf{sortkey}} standard
\textsf{biblatex} field, designed to allow you to specify how you want
an entry alphabetized in a bibliography.  In general, if an entry
doesn't turn up where you expect or want it, this field should provide
the solution.  More particularly, entries without an \textsf{author}
or an \textsf{editor}, or with a corporate author beginning with the
definite or indefinite article, will usually require your assistance
in this way (chaucer:alt, cotton:manufacture:15, gourmet:052006,
lakeforester:pushcarts, nyt:obittrevor, nyt:trevorobit, silver:gawain,
unsigned:ranke:15, virginia:plantation).  Lehman also provides
\textbf{sortname}, \textbf{sorttitle}, and \textbf{sortyear} for more
fine-grained control.  Please consult \textsf{biblatex.pdf} for the
details.

\mybigspace The \mymarginpar{\textbf{subtitle}} subtitle for a
\textsf{title} --- see next entry.

\mybigspace In \mymarginpar{\textbf{title}} the vast majority of
cases, this field works just as it always has in \textsc{Bib}\TeX, and
just as it does in \textsf{biblatex}.  Nearly every entry will have
one, the most likely exceptions being \textsf{incollection} or
\textsf{online} entries with a merely generic title, instead of a
specific one (centinel:letters, powell:email).  The main source of
difficulties flows from the \emph{Manual}'s rules for formatting
\textsf{titles}, rules which also hold for \textsf{booktitles} and
\textsf{maintitles}.  The whole point of using a
\textsc{Bib}\TeX-based system is for it to do the formatting for you,
and in most cases \textsf{biblatex-chicago-notes} does just that,
surrounding titles with quotation marks, italicizing them, or
occasionally just leaving them alone.  When, however, a title is
quoted within a title, then you need to know some of the rules.  A
summary here should serve to clarify them, and help you to understand
when \textsf{biblatex-chicago-notes} might need your help in order to
comply with them.

\mylittlespace The internal rules of \textsf{biblatex-chicago-notes}
are as follows:

\begin{description}
\item[\qquad Italics:] \textsf{booktitle}, \textsf{maintitle}, and
  \textsf{journaltitle} in all entry types; \textsf{title} of
  \textsf{artwork}, \textsf{book}, \textsf{bookinbook},
  \textsf{booklet}, \textsf{collection}, \textsf{inbook},
  \textsf{manual}, \textsf{misc} (with no \textsf{entrysubtype}),
  \textsf{periodical}, \textsf{proceedings}, \textsf{report},
  \textsf{suppbook}, and \textsf{suppcollection} entry types.
\item[\qquad Quotation Marks:] \textsf{title} of \textsf{article},
  \textsf{image}, \textsf{incollection}, \textsf{inproceedings},
  \textsf{online}, \textsf{patent}, \textsf{periodical},
  \textsf{thesis}, and \textsf{unpublished} entry types,
  \textsf{issuetitle} in \textsf{article}, \textsf{periodical}, and
  \textsf{review} entry types.
\item[\qquad Unformatted:] \textsf{booktitleaddon},
  \textsf{maintitleaddon}, and \textsf{titleaddon} in all entry types,
  \textsf{title} of \textsf{customc}, \textsf{letter}, \textsf{misc}
  (with an \textsf{entrysubtype}), \textsf{review}, and
  \textsf{suppperiodical} entry types.
\item[\qquad Italics or Quotation Marks:] All of the audiovisual entry
  types --- \textsf{audio}, \textsf{music}, and \textsf{video} ---
  have to serve as analogues both to \textsf{book} and to
  \textsf{inbook}.  Therefore, if there is both a \textsf{title} and a
  \textsf{booktitle}, then the \textsf{title} will be in quotation
  marks.  If there is no \textsf{booktitle}, then the \textsf{title}
  will be italicized.
\end{description}

Now, the rules for which entry type to use for which sort of work tend
to be fairly straightforward, but in cases of doubt you can consult
section \ref{sec:entrytypes} above, the examples in
\textsf{notes-test.bib}, or go to the \emph{Manual} itself,
8.164--210.  Assuming, then, that you want to present a title within a
title, and you know what sort of formatting each of the two would, on
its own, require, then the following rules apply:

\begin{enumerate}
\item Inside an italicized title, all other titles are enclosed in
  quotation marks and italicized, so in such cases all you need to do
  is provide the quotation marks using \cmd{mkbibquote}, which will
  take care of any following punctuation that needs to be brought
  within the closing quotation mark(s) (17.58; donne:var,
  mchugh:wake).
\item Inside a quoted title, you should present another title as it
  would appear if it were on its own, so in such cases you'll need to
  do the formatting yourself.  Within the double quotes of the title
  another quoted title would take single quotes --- the
  \cmd{mkbibquote} command does this for you automatically, and also,
  I repeat, takes care of any following punctuation that needs to be
  brought within the closing quotation mark(s).  (See 17.157; garrett,
  loften:hamlet, murphy:silent, white:callimachus.)
\item Inside a plain title (most likely in a \textsf{review} entry or
  a \textsf{titleaddon} field), you should present another title as it
  would appear on its own, once again formatting it yourself using
  \cmd{mkbibemph} or \cmd{mkbibquote}.  (barcott:review, gibbard,
  osborne:poison, ratliff:review, unsigned:ranke:15).
\end{enumerate}

\enlargethispage{-2\baselineskip}

The \emph{Manual} provides a few more rules, as well.  A word normally
italicized in text should also be italicized in a quoted or plain-text
title, but should be in roman (\enquote{reverse italics}) in an
italicized title.  A quotation used as a (whole) title (with or
without a subtitle) retains its quotation marks when it is quoted or
plain, but loses them when it is italicized (17.60, 17.157; lewis).  A
word or phrase in quotation marks, but that isn't a quotation, retains
those marks in all title types (kimluu:diethyl).

\mylittlespace Finally, please note that in all \textsf{review} (and
\textsf{suppperiodical}) entries, and in \textsf{misc} entries with an
\textsf{entrysubtype}, and only in those entries,
\textsf{biblatex-chicago-notes} will automatically capitalize the
first word of the \textsf{title} after sentence-ending punctuation,
assuming that such a \textsf{title} begins with a lowercase letter in
your .bib database.  See \textbf{\textbackslash autocap} below for
more details.

\mybigspace Standard \mymarginpar{\textbf{titleaddon}}
\textsf{biblatex} intends this field for use with additions to titles
that may need to be formatted differently from the titles themselves,
and \textsf{biblatex-chicago-notes} uses it in just this way, with the
additional wrinkle that it can, if needed, replace the \textsf{title}
entirely, and this in, effectively, any entry type, providing a fairly
powerful, if somewhat complicated, tool for getting \textsc{Bib}\TeX\
to do what you want (cf.\ centinel:letters, powell:email).  This field
will always be unformatted, that is, neither italicized nor placed
within quotation marks, so any formatting you may need within it
you'll need to provide manually yourself.  The single exception to
this rule is when your data begins with a word that would ordinarily
only be capitalized at the beginning of a sentence, in which case you
need then simply ensure that that word is in lowercase, and
\textsf{biblatex-chicago-notes} will automatically do the right thing.
See\ \textbf{\textbackslash autocap}, below.  (Cf.\ brown:bremer,
osborne:poison, reaves:rosen, and white:ross:memo for examples where
the field starts with a lowercase letter; morgenson:market provides an
example where the \textsf{titleaddon} field, holding the name of a
regular column in a newspaper, is capitalized, a situation that is
handled as you would expect.)

\mybigspace As \mymarginpar{\textbf{translator}} far as possible, I
have implemented this field as \textsf{biblatex}'s standard styles do,
but the requirements specified by the \emph{Manual} present certain
complications that need explaining.  Lehman points out in his
documentation that the \textsf{translator} field will be associated
with a \textsf{title}, a \textsf{booktitle}, or a \textsf{maintitle},
depending on the sort of entry.  More specifically,
\textsf{biblatex-chicago} associates the \textsf{translator} with the
most comprehensive of those titles, that is, \textsf{maintitle} if
there is one, otherwise \textsf{booktitle}, otherwise \textsf{title},
if the other two are lacking.  In a large number of cases, this is
exactly the correct behavior (adorno:benj, centinel:letters,
plato:republic:gr, among others).  Predictably, however, there are
numerous cases that require, for example, an additional translator for
one part of a collection or for one volume of a multi-volume work.
For these cases I have provided the \textsf{nameb} field.  You should
format names for this field as you would for \textsf{author} or
\textsf{editor}, and these names will always be associated with the
\textsf{title} (euripides:orestes).

\mylittlespace I have also provided a \textsf{namea} field, which
holds the editor of a given \textsf{title} (euripides:orestes).  If
\textsf{namea} and \textsf{nameb} are the same,
\textsf{biblatex-chicago} will concatenate them, just as
\textsf{biblatex} already does for \textsf{editor},
\textsf{translator}, and \textsf{namec} (i.e., the compiler).
Furthermore, it is conceivable that a given entry will need separate
translators for each of the three sorts of title.  For this, and for
various other tricky situations, there is the \cmd{parttrans} macro
(and its siblings), designed to be used in a \textsf{note} field or in
one of the \textsf{titleaddon} fields (ratliff:review).  (Because the
strings identifying a translator differ in notes and bibliography, one
can't simply write them out in such a field, hence the need for a
macro, which I discuss further in the commands section below
[\ref{sec:formatcommands}].)

\mylittlespace Finally, as I detailed above under \textbf{author}, in
the absence of an \textsf{author} or an \textsf{editor}, the
\textsf{translator} will be used at the head of an entry
(silver:gawain), and the bibliography entry alphabetized by the
translator's name, behavior that can be controlled with the
\texttt{{usetranslator}} switch in the \textsf{options} field.  Cf.\
\textsf{author}, \textsf{editor}, \textsf{namea}, \textsf{nameb}, and
\textsf{namec}.

\mybigspace This \mymarginpar{\textbf{type}} is a standard
\textsf{biblatex} field, and in its normal usage serves to identify
the type of a \textsf{manual}, \textsf{patent}, \textsf{report}, or
\textsf{thesis} entry.  \textsf{Biblatex} 0.7 introduced the ability,
in some circumstances, to use a bibstring without inserting it in a
\cmd{bibstring} command, and in these entry types the \textsf{type}
field works this way, allowing you simply to input, e.g.,
\texttt{patentus} rather than \cmd{bibstring\{patentus\}}, though both
will work.  (See petroff:impurity; herwign:office, murphy:silent, and
ross:thesis all demonstrate how the \textsf{type} field may sometimes
be automatically set in such entries by using one of the standard
entry-type aliases).

\mylittlespace With the arrival of Lehman's remarkable
punctuation-tracking code in \textsf{biblatex} 0.8, there can be
almost no use for the \textsf{type} field as a switch for the
\cmd{custpunct} macro, so I have been able to reuse it in order to
generalize the functioning of the \textsf{suppbook} entry type, and of
its alias \textsf{suppcollection}.  In such entries, you can now use
the \textsf{type} field to specify what sort of supplemental material
you are citing, e.g., \enquote{\texttt{preface to}} or
\enquote{\texttt{postscript to}.}  Cf.\ \textsf{suppbook} above for the
details.  (See \emph{Manual} 17.74--75; polakow:afterw, prose:intro).

\mylittlespace You can also use the \textsf{type} field in
\textsf{artwork}, \textsf{audio}, \textsf{image}, \textsf{music}, and
\textsf{video} entries to identify the medium of the work, e.g.,
\texttt{oil on canvas}, \texttt{albumen print}, \texttt{compact disc}
or \texttt{MPEG}.  If the first word in this field would normally only
be capitalized at the beginning of a sentence, then leave it in
lowercase in your .bib file and \textsf{biblatex} will automatically
do the right thing in citations.  Cf.\ \textsf{artwork},
\textsf{audio}, \textsf{image}, \textsf{music}, and \textsf{video},
above, for all the details.  (See auden:reading, bedford:photo,
cleese:holygrail, leo:madonna, nytrumpet:art:15.)

%\enlargethispage{-\baselineskip}

\mybigspace Standard \mymarginpar{\textbf{url}} \textsf{biblatex}
field, it holds the url of an online publication, though you can
provide one for all entry types.  The required \LaTeX\ package
\textsf{url} will ensure that your documents format such references
properly, in the text and in the reference apparatus.

\mybigspace Standard \mymarginpar{\textbf{urldate}} \textsf{biblatex}
field, it identifies exactly when you accessed a given url.  This
field would contain the whole date, in \textsc{iso}8601 format
(evanston:library, grove:sibelius, hlatky:hrt, osborne:poison,
sirosh:visualcortex, wikiped:bibtex).  Please note that the
\textbf{urlday}, \textbf{urlmonth}, and \textbf{urlyear} fields are
all now obsolete.

\mybigspace A \mymarginpar{\textbf{usera}} supplemental
\textsf{biblatex} field which functions in \textsf{biblatex-chicago}
almost as a \enquote{\textsf{journaltitleaddon}} field.  In
\textsf{article}, \textsf{periodical}, and \textsf{review} entries
with \textsf{entrysubtype} \texttt{magazine}, the contents of this
field will be placed, unformatted and between commas, after the
\textsf{journaltitle} and before the date.  The main use is for
identifying the broadcast network when you cite a radio or television
program (bundy:macneil), though you may also want to use it to
identify the section of a newspaper in which you've found a particular
article (morgenson:market).  (See \emph{Manual} 17.190, 17.207.  As
far as I can work out, newspaper section information may be placed
either before the date [\textsf{usera}] or after it [\textsf{pages}].
Cp. kozinn:review [17.202] and morgenson:market [17.190].  The choice
would appear to be yours.)

\mybigspace \textbf{NB:} \mymarginpar{\textbf{userb}} \textbf{this
  field is now deprecated, mainly because it is very unlikely you will
  have any further need for the \textbackslash custpunct macros.  I
  leave the code, and the instructions for how to use it, in place,
  because it's barely possible that a need for it might still arise.}
A supplemental \textsf{biblatex} field, with a very specific use in
\textsf{biblatex-chicago-notes}.  If the occasion does arise when you
need to supply some context-sensitive punctuation yourself, then
usually the \cmd{custpunct} command will then be needed, controlled in
certain circumstances by a toggle in the \textsf{type} field.  If,
however, you already need the \textsf{type} field for its regular
usage in a \textsf{suppbook}, \textsf{manual}, \textsf{patent},
\textsf{report}, or \textsf{thesis} entry, and if you need to control
the \cmd{custpunct} with a toggle, then you'll have to use
\cmd{custpunctb}, toggled by putting the exact string \texttt{plain}
in \textsf{userb}.

\mybigspace I \mymarginpar{\textbf{userc}} have now implemented this
supplemental \textsf{biblatex} field as part of Chicago's name
cross-referencing system.  (The \enquote{c} part is meant as a sort of
mnemonic for this function, though it's perfectly possible to use the
field in other contexts.)  If you use the \textbf{customc} entry type
to include alphabetized cross-references to other, separate entries in
a bibliography, it is unlikely that you will cite the \textsf{customc}
entry in the body of your text.  Therefore, in order for it to appear
in the bibliography, you have two choices.  You can either include the
entry key of the \textsf{customc} entry in a \cmd{nocite} command
inside your document, or you can place that entry key in the
\textsf{userc} field of another .bib entry that you will be citing.
In the latter case, \textsf{biblatex-chicago} will call \cmd{nocite}
for you, and this method should ensure that there will be at least one
entry in the bibliography to which the cross-reference will point.
(See 17.39--40; lecarre:cornwell, lecarre:quest.)

\mybigspace NB: \mymarginpar{\textbf{userd}} this field is now
obsolete.  If it appears in a .bib file it will be ignored.

\mybigspace Another \mymarginpar{\textbf{usere}} supplemental
\textsf{biblatex} field, which \textsf{biblatex-chicago} uses
specifically to provide a translated \textsf{title} of a work,
something that may be needed if you deem the original language
unparseable by a significant portion of your likely readership.  The
\emph{Manual} offers two alternatives in such a situation: either you
can translate the title and use that translation in your
\textsf{title} field, providing the original language in
\textsf{language}, or you can give the original title in
\textsf{title} and the translation in \textsf{usere}.  If you choose
the latter, you may need to provide a \textsf{shorttitle} so that the
short note form is also parseable.  Cf.\ \textbf{language}, above.
(See 17.65--67, 17.166, 17.177; kern, weresz.)

%\enlargethispage{\baselineskip}

\mybigspace This \mymarginpar{\textbf{userf}} is the last of the
supplemental fields which \textsf{biblatex} provides, used by
\textsf{biblatex-chicago} for a very specific purpose.  When you cite
both a translation and its original, the \emph{Manual} (17.66)
recommends that, in the bibliography at least, you combine references
to both texts in one entry, though the presentation in notes is pretty
much up to you.  In order to follow this specification, I have
provided a third cross-referencing system (the others being
\textsf{crossref} and \textsf{xref}), and have chosen the name
\textsf{userf} because it might act as a mnemonic for its function.

\mylittlespace In order to use this system, you should start by
entering both the original and its translation into your .bib file,
just as you normally would.  The mechanism works for any entry type,
and the two entries need not be of the same type.  In the entry for
the \emph{translation}, you put the cite key of the original into the
\textsf{userf} field.  In the \emph{original's} entry, you need to
include a toggle in the \textsf{keywords} field that will prevent that
entry from being printed separately in the bibliography --- I have
chosen the string \texttt{original}, and use
\texttt{notkeyword=original} in the \cmd{printbibliography} command,
though you can use anything you want.  In this standard case, the data
for the translation will be printed first, followed by the string
\texttt{originally published as}, followed by the original, author
omitted, in what amounts to the same format that the \emph{Manual}
uses for long footnotes (furet:passing:eng, furet:passing:fr).  As
explained above (\textbf{origlanguage}), I have also included a way to
modify the string printed before the original.  In the entry for the
\emph{translation}, you put the original's language in
\textsf{origlanguage}, and instead of \texttt{originally published
  as}, you'll get \texttt{French edition:} or \texttt{Latin edition:},
etc.\ (aristotle:metaphy:gr, aristotle:metaphy:trans).

\mybigspace Standard \mymarginpar{\textbf{venue}} \textsf{biblatex}
offers this field for use in \textsf{proceedings} and
\textsf{inproceedings} entries, but I haven't yet implemented it,
mainly because the \emph{Manual} has nothing to say about it.  Perhaps
the \textsf{organization} field could be used, for the moment,
instead.  Anything in a \textsf{venue} field will be ignored.

\mybigspace Standard \mymarginpar{\textbf{version}} \textsf{biblatex}
field, currently only available in \textsf{misc} and \textsf{patent}
entries in \textsf{biblatex-chicago-notes}.

\mybigspace Standard \mymarginpar{\textbf{volume}} \textsf{biblatex}
field.  It holds the volume of a \textsf{journaltitle} in
\textsf{article} (and some \textsf{review}) entries, and also the
volume of a multi-volume work in many other sorts of entry.  Cf.\
\textsf{part}.

\mybigspace Standard \mymarginpar{\textbf{volumes}} \textsf{biblatex}
field.  It holds the total number of volumes of a multi-volume work,
and its use in an entry triggers particular behavior in short notes
referring to such an entry, which notes will not print any punctuation
between the title of the work and the volume+page reference given in
the optional postnote field of the relevant \cmd{cite} command
(17.134; meredith:letters).  If this behavior is inconvenient in a
particular entry, you may need to provide a \textsf{shorttitle} field
ending in an \cmd{addcomma}, though in such a case you'd need to
ensure that the \cmd{cite} command's postnote field contained
something, as otherwise the note would end, wrongly, with a comma.
(The \emph{Manual} appears to be somewhat inconsistent on this
question [cf.\ 16.47], so if this feature proves onerous in use I
could remove it.)

\mybigspace A \mymarginpar{\textbf{xref}} modified \textsf{crossref}
field provided by \textsf{biblatex}.  See \textbf{crossref}, above.

\mybigspace Standard \mymarginpar{\textbf{year}} \textsf{biblatex}
field.  It usually identifies the year of publication, though unlike
the \textsf{date} field it allows non-numeric input, so you can put
\enquote{n.d.}\ (or, to be language agnostic,
\cmd{bibstring\{nodate\}}) here if required, or indeed any other sort
of non-numerical date information.  If you can guess the date then you
can include that guess in square brackets instead of, or after, the
\enquote{n.d.}\ abbreviation.  Cf.\ bedford:photo, clark:mesopot,
ross:leo, thesis:madonna.

\subsection{Commands}
\label{sec:commands}

In this section I shall attempt to document all those commands you may
need when using \textsf{biblatex-chicago-notes} that I have either
altered with respect to the standard provided by \textsf{biblatex} or
that I have provided myself.  Some of these, unfortunately, will make
your .bib file incompatible with other \textsf{biblatex} styles, but
I've been unable to avoid this.  Any ideas for more elegant, and more
compatible, solutions will be warmly welcomed.

\subsubsection{Formatting Commands}
\label{sec:formatcommands}

These commands allow you to fine-tune the presentation of your
references in both notes and bibliography.  You can find many examples
of their usage in \textsf{notes-test.bib}, and I shall try to point
you toward a few such entries in what follows.  \textbf{NB:}
\textsf{biblatex's} \cmd{mkbibquote} command is now mandatory in some
situations.  See its entry below.

\enlargethispage{\baselineskip}

\mybigspace Version \mymarginpar{\textbf{\textbackslash autocap}} 0.8
of \textsf{biblatex} introduced the \cmd{autocap} command, which
capitalizes a word inside a note or bibliography entry if that word
follows sentence-ending punctuation, and leaves it lowercase
otherwise.  As this command is both more powerful and more elegant
than the kludge I designed for a previous version of
\textsf{biblatex-chicago-notes} (see\ \textbf{\textbackslash
  bibstring} below), you should be aware that the use of the
single-letter \cmd{bibstring} commands in your .bib file is obsolete.

\mylittlespace In order somewhat to reduce the burden on users even
further, I have, following Lehman's example, implemented a new system
which automatically tracks the capitalization of certain fields in
your .bib file.  I chose these fields after a non-scientific survey of
entries in my own databases, so of course if you have ideas for the
extension of this facility I would be most interested to hear them.
In order to take advantage of this functionality, all you need do is
begin the data in the appropriate field with a lowercase letter,
e.g.,\ \texttt{note = \{with the assistance of X\}}.  If the data
begins with a capital letter --- and this is not infrequent --- that
capital will always be retained.  (cf., e.g., creel:house,
morgenson:market.)  If, on the other hand, you for some reason need
such a field always to start with a lowercase letter, then you can try
using the \cmd{isdot} macro at the start, which turns off the
mechanism without printing anything itself.  Here, then, is the
complete list of fields where this functionality is active:

\begin{enumerate}
\item The \textbf{addendum} field in all entry types.
\item The \textbf{booktitleaddon} field in all entry types.
\item The \textbf{edition} field in all entry types.  (Numerals work
  as you expect them to here.)
\item The \textbf{maintitleaddon} field in all entry types.
\item The \textbf{note} field in all entry types.
\item The \textbf{shorttitle} field in the \textsf{review}
  (\textsf{suppperiodical}) entry type and in the \textsf{misc} type,
  in the latter case, however, only when there is an
  \textsf{entrysubtype} defined, indicating that the work cited is
  from an archive.
\item The \textbf{title} field in the \textsf{review}
  (\textsf{suppperiodical}) entry type and in the \textsf{misc} type,
  in the latter case, however, only when there is an
  \textsf{entrysubtype} defined, indicating that the work cited is
  from an archive.
\item The \textbf{titleaddon} field in all entry types.
\item The \textbf{type} field in \textsf{artwork}, \textsf{audio},
  \textsf{image}, \textsf{music}, \textsf{suppbook},
  \textsf{suppcollection}, and \textsf{video} entry types.
\end{enumerate}

In any other cases --- and there are only two examples of this in
\textsf{notes-test.bib} (centinel:letters, powell:email) --- you'll
need to provide the \cmd{autocap} command yourself.  Indeed, if you
accidentally do so in one of the above fields, it shouldn't matter at
all, and you'll still get what you want, but taking advantage of the
automatic provisions should at least save some typing.

\mybigspace This \mymarginpar{\textbf{\textbackslash bibstring}} is
Lehman's very powerful mechanism to allow \textsf{biblatex}
automatically to provide a localized version of a string, and to
determine whether that string needs capitalization, depending on where
it falls in an entry.  In the first release of
\textsf{biblatex-chicago-notes}, the style relied very heavily on this
macro, particularly on an extension I provided by defining all 26
letters of the (ASCII) alphabet as \texttt{bibstrings}
(\cmd{bibstring\{a\}}, \cmd{bibstring\{b\}}, etc.)  While you should
continue to use the standard, whole-word bibstrings, \textbf{all use
  of the single-letter variants I formerly provided is obsolete, and
  will generate an error}.  This functionality has been replaced by
the \cmd{autocap} command, which does the same thing, only more
elegantly.  This command was designed by Philipp Lehman, and has now
been included in version 0.8 of \textsf{biblatex}.  For yet greater
convenience I have implemented, following Lehman's example, a system
automating this functionality in all of the entry fields where its use
was, by my reckoning, most frequent.  This means that, when you
require this functionality, all you need do is input the data in such
a field starting with a lowercase letter, and
\textsf{biblatex-chicago-notes} will do the rest with no further
assistance.  In my \textsf{notes-test.bib} file, this new mechanism in
effect eliminated all need for the single-letter \texttt{bibstrings}
and very nearly all need for the \cmd{autocap} command ---
centinel:letters and powell:email being the only exceptions.  Please
see \textbf{\textbackslash autocap} above for full details.

%\enlargethispage{\baselineskip}

\mylittlespace I should also mention here that \textsf{biblatex 0.7}
introduced a new functionality which sometimes allows you simply to
input, for example, \texttt{newseries} instead of
\cmd{bib\-string\{newseries\}}, the package auto-detecting when a
bibstring is involved and doing the right thing, though in all such
cases either form will work.  This functionality is available in the
\textsf{series} field of \textsf{article}, \textsf{periodical}, and
\textsf{review} entries; in the \textsf{type} field of
\textsf{manual}, \textsf{patent}, \textsf{report}, and \textsf{thesis}
entries; in the \textsf{location} field of \textsf{patent} entries; in
the \textsf{language} field in all entry types; and in the
\textsf{nameaddon} field in \textsf{customc} entries.  These are the
places, as far as I can make out, where \textsf{biblatex's} standard
styles support this feature, and I have added the last,
style-specific, one.  If Lehman generalizes it still further in a
future release, I shall do the same, if possible.

\mybigspace In \mymarginpar{\textbf{\textbackslash custpunct} \\
  \textbf{\textbackslash custpunctb}} common with other American
citation styles, the \emph{Manual} requires that the commas and
periods separating units of a reference go inside any quotation marks
that happen to be present.  As of version 0.8c, \textsf{biblatex}
contains truly remarkable code that handles this situation in very
nearly complete generality, detecting punctuation after the closing
quotation mark and moving it inside when necessary, and also
controlling which punctuation marks can be printed after which other
punctuation marks, whether quotation marks intervene or not.  This
functionality is now mature, and \textsf{biblatex-chicago-notes}
relies on this code to place punctuation in the \enquote{American
  style,} rather than on complicated \cmd{DeclareFieldFormat}
instructions that attempt to anticipate all possible permutations.
One result of this, thankfully, is that both \cmd{custpunct} and
\cmd{custpunctb} are now basically unnecessary, as their only purpose
was to supply context-appropriate punctuation inside any quotation
marks that users themselves provided as part of various entry fields.
A second consequence, and I've already recommended this in previous
releases anyway, is that users now \emph{must} use \cmd{mkbibquote}
instead of \cmd{enquote} or the usual \LaTeX\ mechanisms inside their
.bib files.  For further details, please see the \cmd{mkbibquote}
entry below.

%\enlargethispage{-\baselineskip}

\mylittlespace I have retained the code for the \cmd{custpunct}
commands in \textsf{chicago-notes.cbx}, in case a particularly gnarly
entry might still require them, but I have already started to re-use
the \textsf{type} field, which formerly served as a switch for
\cmd{custpunct}, in other contexts (see \textbf{artwork},
\textbf{image}, and \textbf{suppbook} above).

\mybigspace This \colmarginpar{\textbf{\textbackslash isdot}} is a
standard \textsf{biblatex} macro, which in previous releases of
\textsf{biblatex-chicago} could function as a convenient placeholder
in entry fields that, for one reason or another, you may have wanted
to have defined and yet to print nothing.  With the release of
\textsf{biblatex} 1.6, this no longer works as before, a situation
which has revealed a number of inconsistencies and bugs in my code,
the rectification of which may therefore require some changes to your
.bib files, assuming you've taken advantage of this mechanism.  I
believe that all the situations formerly calling for this specific use
of the macro can now be addressed by more standard means, i.e., the
\cmd{headlesscite} commands and the \texttt{useauthor=false}
declaration in the \textsf{options} field.  (See creel:house,
nyt:obittrevor, sewall:letter, unsigned:ranke:15, and white:total.)

\mybigspace I \mymarginpar{\textbf{\textbackslash letterdatelong}}
have provided this macro mainly for use in the optional postnote field
of the various citation commands.  When citing a letter (published or
unpublished, \textsf{letter} or \textsf{misc}), it may be useful to
append the date to the usual short note form in order to disambiguate
references.  This macro simply prints the date of a letter, or indeed
of any other sort of correspondence.  (If your main document language
isn't English, it's better just to use the standard \textsf{biblatex}
command \cmd{printorigdate}.)

\mybigspace This \mymarginpar{\textbf{\textbackslash mkbibquote}} is
the standard \textsf{biblatex} command, which requires attention here
because it is a crucial part of the mechanism of Lehman's
\enquote{American} punctuation system.  If you look in
\textsf{chicago-notes.cbx} you'll see that the quoted fields, e.g., an
\textsf{article} or \textsf{incollection title}, have this command in
their formatting, which does most of the work for you.  If, however,
you need to provide additional quotation marks in a field --- a quoted
title within a title, for example --- then you may need to use this
command so that any following period or comma will be brought within
the closing quotation marks.  Its use is \emph{required} when the
quoted material comes at the end of a field, and I recommend always
using it in your .bib database, as it does no harm even when that
condition is not fulfilled.  A few examples from
\textsf{notes-test.bib} should help to clarify this.

\mylittlespace In an \textsf{article} entry, the \textsf{title}
contains a quoted phrase:

\begin{quotation}
  \noindent\texttt{title = \{Diethylstilbestrol and Media Coverage of the \\
    \indent\cmd{mkbibquote}\{Morning After\} Pill\}}
\end{quotation}

Here, because the quoted text doesn't come at the end of title, and no
punctuation will ever need to be drawn within the closing quotation
mark, you could instead use \texttt{\cmd{enquote}\{Morning After\}} or
even \texttt{`Morning After'}. (Note the single quotation marks here
--- the other two methods have the virtue of taking care of nesting
for you.)  All of these will produce the formatted
\enquote{Diethylstilbestrol and Media Coverage of the \enquote{Morning
    After} Pill.}  Here, by contrast, is a \textsf{book title}:

\begin{quotation}
  \noindent \texttt{title = \{Annotations to
    \cmd{mkbibquote}\{Finnegans Wake\}\}}
\end{quotation}

Because the quoted title within the title comes at the end of the
field, and because this bibliographical unit will be separated from
what follows by a period in the bibliography, then the
\cmd{mkbibquote} command is necessary to bring that period within the
final quotation marks, like so: \emph{Annotations to
  \enquote{Finnegans Wake.}}

\mylittlespace Let me also add that this command interacts well with
Lehman's \textsf{csquotes} package, which I highly recommend, though
the latter isn't strictly necessary in texts using an American style,
to which \textsf{biblatex} defaults when \textsf{csquotes} isn't
loaded.

\mybigspace This \mymarginpar{\textbf{\textbackslash reprint}} and the
following 7 macros all help \textsf{biblatex-chicago-notes} cope with
the fact that many bibstrings in the Chicago system differ between
notes and bibliography, the former sometimes using abbreviated forms
when the latter prints them in full.  In the current case, if a book
is a reprint, then the macro \cmd{reprint}, followed by a comma,
should go in the \textsf{location} field before the city of
publication (aristotle:metaphy:gr, schweitzer:bach).  See
\textbf{location}, above.

\mylittlespace \textbf{NB:} The rules for employing abbreviated or
full bibstrings in the \emph{Manual} are remarkably complex, but I
have attempted to make them as transparent for users as possible.  In
\textsf{biblatex-chicago-notes}, if you don't see it mentioned in this
section, then in theory you should always provide an abbreviated
version, using the \cmd{bibstring} mechanism, if necessary
(babb:peru).  The standard \textsf{biblatex} bibstrings should also
work (palmatary:pottery), and any that won't should be covered by the
series of macros beginning here with \cmd{reprint} and ending below
with \cmd{parttransandcomp}.

\mybigspace Since \mymarginpar{\textbf{\textbackslash partcomp}} the
\emph{Manual} specifies that the strings \texttt{editor},
\texttt{translator}, and \texttt{compiler} all require different forms
in notes and bibliography, and since it mentions these three apart
from all the others \textsf{biblatex} provides (\textsf{annotator},
\textsf{commentator}, et al.), and further since it may indeed happen
that the available fields (\textsf{editor}, \textsf{namea},
\textsf{translator}, \textsf{nameb}, and \textsf{namec}) aren't
adequate for presenting some entries, I have provided 7 macros to
allow you to print the correct strings for these functions in both
notes and bibliography.  Their names all begin with \cmd{part}, as
originally I intended them for use when a particular name applied only
to a specific \textsf{title}, rather than to a \textsf{maintitle} or
\textsf{booktitle} (cf.\ \textbf{namea} and \textbf{nameb}, above).

\enlargethispage{\baselineskip}

\mylittlespace In the present instance, you can use \cmd{partcomp} to
identify a compiler when \textsf{namec} won't do, e.g., in a
\textsf{note} field or the like.  In such a case,
\textsf{biblatex-chicago-notes} will print the appropriate string in
your references.

\mybigspace Use \mymarginpar{\textbf{\textbackslash partedit}} this
macro when identifying an editor whose name doesn't conveniently fit
into the usual fields (\textsf{editor} or \textsf{namea}).  (N.B.: If
you are writing in French and using \textsf{cms-french.lbx}, then
currently you'll need to add either \texttt{de} or \texttt{d'} after
this command in your .bib files to make the references come out right.
I'm working on this.)  See chaucer:liferecords.

\mybigspace As \mymarginpar{\textbf{\textbackslash
    partedit-\\andcomp}} before, but for use when an editor is also a
compiler.

\vspace{1.3\baselineskip} As \mymarginpar{\textbf{\textbackslash
    partedit-\\andtrans}} before, but for when when an editor is also a
translator (ratliff:review).

\mybigspace As \mymarginpar{\textbf{\textbackslash
    partedit-\\transandcomp}} before, but for when an editor is also a
translator and a compiler.

\vspace{1.3\baselineskip} As \mymarginpar{\textbf{\textbackslash
    parttrans}} before, but for use when identifying a translator
whose name doesn't conveniently fit into the usual fields
(\textsf{translator} and \textsf{nameb}).

\mybigspace As \mymarginpar{\textbf{\textbackslash
    parttrans-\\andcomp}} before, but for when a translator is also a
compiler.

\subsubsection{Citation Commands}
\label{sec:citecommands}

The \textsf{biblatex} package is particularly rich in citation
commands, some of which (e.g., \cmd{supercite(s)}, \cmd{citeyear})
provide functionality that isn't really needed by the Chicago notes
and bibliography style offered here.  If you are getting unexpected
behavior when using them please have a look in your .log file.  A
command like \cmd{textcite}, listed in �~3.6.2 of the
\textsf{biblatex} manual but not defined by \textsf{biblatex-chicago},
defaults to \cmd{cite}, and leaves a warning in the .log.  Others
(e.g., \cmd{cite\-url}), though I haven't tested them extensively,
should pretty much work out of the box.  What remains are the commands
I have found most useful and necessary for following the
\emph{Manual}'s specifications, and I document in this section any
alterations I have made to these.  As always, if there are standard
commands that don't work for you, or new commands that would be
useful, please let me know, and it should be possible to fix or add
them.

\mylittlespace A number of users have run into a problem that appears
when they've used a command like \cmd{cite} inside a \cmd{footnote}
macro.  In this situation, the automatic capitalization routines will
not be in operation at the start of the footnote, so instead of
\enquote{Ibid.,} for example, you'll see \enquote{ibid.}  If you need
to use the \cmd{cite} command within a \cmd{footnote} command, the
solution is to use \cmd{Cite} instead.  Alternatively, don't use a
\cmd{footnote} macro at all, rather try \cmd{footcite} or
\cmd{autocite} with the optional prenote and postnote arguments.  Cf.\
\cmd{Citetitle} below, and also section~3.6 of \textsf{biblatex.pdf}.

\mybigspace I \mymarginpar{\textbf{\textbackslash autocite}} haven't
adapted this in the slightest, but I thought it worth pointing out
that \textsf{biblatex-chicago-notes} sets this command to use
\cmd{footcite} as the default option. It is, in my experience, much
the most common citation command you will use, and also works fine in
its multicite form, \textbf{\textbackslash autocites}.

\mybigspace While \mymarginpar{\textbf{\textbackslash cite*}} the
\cmd{cite} command works just as you would expect it to, I have also
provided a starred version for the rare situations when you might need
to turn off the ibidem tracking mechanism.  \textsf{Biblatex} provides
very sophisticated algorithms for using \enquote{Ibid} in notes, so in
general you won't find a need for this command, but in case you'd
prefer a longer citation where you might automatically find
\enquote{Ibid,} I've provided this.  Of course, you'll need to put it
inside a \cmd{footnote} command manually.  (See also section
\ref{sec:useropts}, below.)

\mybigspace I \colmarginpar{\textbf{\textbackslash citeauthor}} have
adapted this standard \textsf{biblatex} command only very slightly to
bring it into line with \textsf{biblatex-chicago's} needs.  Its main
usage will probably be for references to works from classical
antiquity, when an \textsf{author's} name (abbreviated or not)
sometimes suffices in the absence of a \textsf{title}, e.g.,
Thuc.\ 2.40.2--3 (17.252).  You'll need to put it inside a
\cmd{footnote} command manually.  (Cf.\ also \textsf{entrysubtype} in
section~\ref{sec:entryfields}, above.)

\mybigspace This \colmarginpar{\textbf{\textbackslash citejournal}}
command provides an alternative short form when citing journal
\textsf{articles}, giving the \textsf{journaltitle} and
\textsf{volume} number instead of the article \textsf{title} after the
\textsf{author's} name.  The \emph{Manual} suggests that this format
might be helpful \enquote{in the absence of a full bibliography}
(17.179).

\mybigspace This \mymarginpar{\textbf{\textbackslash Citetitle}}
simply prepends \cmd{bibsentence} to the usual \cmd{citetitle}
command.  Some titles may need this for the automatic contextual
capitalization facility to work correctly.  (Included as standard from
\textsf{biblatex} 0.8d.)

\mybigspace Joseph \mymarginpar{\textbf{\textbackslash citetitles}}
Reagle noticed that, because of the way
\textsf{biblatex-chicago-notes} formats titles in quotation marks,
using the \cmd{citetitle} command will often get you punctuation you
don't want, especially when presenting a list of titles.  I've
included this multicite command to enable you to present such a list,
if the need arises.  Remember that you'll have to put it inside a
\cmd{footnote} command manually.

\mybigspace Another \mymarginpar{\textbf{\textbackslash footfullcite}}
standard \textsf{biblatex} command, modified to work properly with
\textsf{biblatex-chicago-notes}, and provided in case you find
yourself in a situation where you really need the full citation in a
footnote, but where \cmd{autocite} would print a short note or even
\enquote{Ibid.}  This may be particularly useful if you've chosen to use all
short notes by setting the \texttt{short} option in the arguments to
\cmd{usepackage\{biblatex\}}, yet still feel the need for the
occasional full citation.

\mybigspace This, \mymarginpar{\textbf{\textbackslash fullcite}} too,
is a standard command, and it too provides a full citation, but unlike
the previous command it doesn't automatically place it in a footnote.
It may be useful within long textual notes.

\mybigspace Matthew \mymarginpar{\textbf{\textbackslash headlesscite}}
Lundin requested a more generalized \cmd{headlesscite} macro,
suppressing the author's name in specific contexts while allowing
users not to worry about whether a particular citation needs the long
or short form, a responsibility thereby handed over to
\textsf{biblatex's} tracking mechanisms.  This citation command
attempts to fulfill this request.  Please note that, in the short
form, the result will be rather like a \cmd{citetitle} command, which
may or may not be what you want.  Note, also, that as I have provided
only the most flexible form of the command, you'll have to wrap it in
a \cmd{footnote} yourself.  Please see the next entry for further
discussion of some of the needs this command might help address.

\mybigspace I \mymarginpar{\textbf{\textbackslash
    headless-\\fullcite}} have provided this command in case you want
to print a full citation without the author's name.  The \emph{Manual}
(17.31, 17.42) suggests this for brevity's sake in cases where that
name is already obvious enough from the title, and where repetition
might seem awkward (creel:house, feydeau:farces, meredith:letters, and
sewall:letter).  \textsf{Letter} entries --- and only such entries ---
do this for you automatically, and of course the repetition is
tolerated in bibliographies for the sake of alphabetization, but in
notes this command may help achieve greater elegance, even if it isn't
strictly necessary.  As I've provided only the most flexible form of
the command, you'll have to wrap it in a \cmd{footnote} yourself.

\mybigspace I \mymarginpar{\textbf{\textbackslash shortcite}} have
provided this command in case, for any reason, you specifically
require the short form of a note, and \textsf{biblatex} thinks you
want something else.  Again, I've provided only the most flexible form
of the command, so you'll have to wrap it in a \cmd{footnote}
manually.

\enlargethispage{\baselineskip}

\mylittlespace If you look at \textsf{chicago-notes.cbx}, you'll see a
number of other citation commands, but those are intended for internal
use only, mainly in cross-references of various sorts.  Use at your
own risk.

\subsection{Package Options}
\label{sec:options}

\subsubsection{Pre-Set \textsf{biblatex} Options}
\label{sec:presetopts}

Although a quick glance through \textsf{biblatex-chicago.sty} will
tell you which \textsf{biblatex} options the package sets for you, I
thought I might gather them here also for your perusal.  These
settings are, I believe, consistent with the specification, but you
can alter them in the options to \textsf{biblatex-chicago} in your
preamble or by loading the package via
\cmd{usepackage[style=chicago-notes]\{biblatex\}}, which gives you the
\textsf{biblatex} defaults unless you redefine them yourself inside
the square brackets.

\mylittlespace By \mymarginpar{\texttt{abbreviate=\\false}} default,
\textsf{biblatex-chicago-notes} prints the longer bibstrings, mainly
for use in the bibliography, but since notes require the shorter forms
of many of them, I've had to define many new strings for use there.

\mylittlespace \textsf{Biblatex-chicago-notes}
\mymarginpar{\texttt{autocite=\\footnote}} places references in
footnotes by default.

\mybigspace The \mymarginpar{\texttt{citetracker=\\true}} citetracker
for the \cmd{ifciteseen} test is enabled globally.

\mybigspace The \mymarginpar{\texttt{alldates=comp}} specification calls
for the long format when presenting dates, slightly shortened when
presenting date ranges.

\mylittlespace The \mymarginpar{\texttt{dateabbrev=\\false}}
\emph{Manual} prefers full month names in the notes \&\ bibliography
style.

\mybigspace This \mymarginpar{\texttt{ibidtracker=\\constrict}}
enables the use of \enquote{Ibid} in notes, but only in the most
strictly-defined circumstances.  Whenever there might be any
ambiguity, \textsf{biblatex} should default to printing a more
informative reference.  Remember also that you can use the \cmd{cite*}
command to disable this functionality in any given reference, or
indeed one of the \texttt{fullcite} commands if you need the long note
form for any reason.

%\enlargethispage{\baselineskip}

\mylittlespace This \marginpar{\texttt{loccittracker\\=constrict}}
allows the package to determine whether two consecutive citations of
the same source also cite the same page of that source.  In such a
case, \texttt{Ibid} alone will be printed, without the page reference,
following the specification (16.47).

\mylittlespace These \colmarginpar{\textsf{\texttt{maxbibnames\\=10\\
      minbibnames\\=7}}} two options are new, and control the number
of names printed in the bibliography when that number exceeds 10.
These numbers follow the recommendations of the \emph{Manual}
(17.29--30), and they are different from those for use in notes.  With
\textsf{biblatex} 1.6 and later you can no longer redefine
\texttt{maxnames} and \texttt{minnames} in the \cmd{printbibliography}
command at the bottom of your document, so \textsf{biblatex-chicago}
now does this automatically for you, though of course you can change
them in your document preamble.

\mylittlespace This \mymarginpar{\texttt{pagetracker=\\true}} enables
page tracking for the \cmd{iffirstonpage} and \cmd{ifsamepage}
commands for controlling, among other things, the printing of
\enquote{Ibid.}  It tracks individual pages if \LaTeX\ is in oneside
mode, or whole spreads in twoside mode.

\mylittlespace This \colmarginpar{\texttt{sortcase=\\false}} turns off
the sorting of uppercase and lowercase letters separately, a practice
which the \emph{Manual} doesn't appear to recommend.

\mylittlespace This \mymarginpar{\texttt{usetranslator\\=true}}
enables automatic use of the \textsf{translator} at the head of
entries in the absence of an \textsf{author} or an \textsf{editor}.
In the bibliography, the entry will be alphabetized by the
translator's surname.  You can disable this functionality on a
per-entry basis by setting \texttt{usetranslator=false} in the
\textsf{options} field.  Cf.\ silver:gawain.

\subsubsection*{Other \textsf{biblatex} Formatting Options}
\label{sec:formatopts}

I've chosen defaults for many of the general formatting commands
provided by \textsf{biblatex}, including the vertical space between
bibliography items and between items in the list of shorthands
(\cmd{bibitemsep} and \cmd{lositemsep}).  I define many of these in
\textsf{biblatex-chicago.sty}, and of course you may want to redefine
them to your own needs and tastes.  It may be as well you know that
the \emph{Manual} does state a preference for two of the formatting
options I've implemented by default: the 3-em dash as a replacement
for repeated names in the bibliography (16.103--106); and the
formatting of note numbers, both in the main text and at the bottom of
the page / end of the essay (superscript in the text, in-line in the
notes; 16.25).  The code for this last formatting is also in
\textsf{biblatex-chicago.sty}, and I've wrapped it in a test that
disables it if you are using the \textsf{memoir} class, which I
believe has its own commands for defining these parameters.  You can
also disable it by using the \texttt{footmarkoff} package option, on
which see below.

\subsubsection{{Pre-Set \textsf{chicago} Options}}
\label{sec:chicpreset}

At \mymarginpar{\texttt{bookpages=\\true}} the request of Scot
Becker, I have included this rather specialized option, which controls
the printing of the \textsf{pages} field in \textsf{book} entries.
Some bibliographic managers, apparently, place the total page count in
that field by default, and this option allows you to stop the printing
of this information in notes and bibliography.  It defaults to true,
which means the field is printed, but it can be set to false either in
the preamble, for the whole document, or on a per-entry basis in the
\textsf{options} field (though rather than use this latter method it
would make sense to eliminate the \textsf{pages} field from the
affected entries).

\mylittlespace This \mymarginpar{\texttt{doi=true}} option controls
whether any \textsf{doi} fields present in the .bib file will be
printed in notes and bibliography.  It defaults to true, and can be
set to false either in the preamble, for the whole document, or on a
per-entry basis, in the \textsf{options} field.

\mylittlespace This \mymarginpar{\texttt{isbn=true}} option controls
whether any \textsf{isan}, \textsf{isbn}, \textsf{ismn},
\textsf{isrn}, \textsf{issn}, and \textsf{iswc} fields present in the
.bib file will be printed in notes and bibliography.  It defaults to
true, and can be set to false either in the preamble, for the whole
document, or on a per-entry basis, in the \textsf{options} field.

\mylittlespace Once \mymarginpar{\texttt{numbermonth=\\true}} again
at the request of Scot Becker, I have included this option, which
controls the printing of the \textsf{month} field in all the
periodical-type entries when a \textsf{number} field is also present.
Some bibliographic software, apparently, always includes the month of
publication even when a \textsf{number} is present.  When all this
information is available the \emph{Manual} (17.181) prints everything,
so this option defaults to true, which means the field is printed, but
it can be set to false either in the preamble, for the whole document,
or on a per-entry basis in the \textsf{options} field.

\mylittlespace This \mymarginpar{\texttt{url=true}} option controls
whether any \textsf{url} fields present in the .bib file will be
printed in notes and bibliography.  It defaults to true, and can be
set to false either in the preamble, for the whole document, or on a
per-entry basis, in the \textsf{options} field.  Please note that, as
in standard \textsf{biblatex}, the \textsf{url} field is always
printed in \textsf{online} entries, regardless of the state of this
option.

\mylittlespace This \mymarginpar{\texttt{includeall=\\true}} is the
one option that rules the five preceding, either printing all the
fields under consideration --- the default --- or excluding all of
them.  It is set to \texttt{true} in \textsf{chicago-notes.cbx}, but
you can change it either in the preamble for the whole document or in
the \textsf{options} field of individual entries.  The rationale for
all of these options is the availability of bibliographic managers
that helpfully present as much data as possible, in every entry, some
of which may not be felt to be entirely necessary.  Setting
\texttt{includeall} to \texttt{true} probably works just fine for
those compiling their .bib databases by hand, but others may find that
some automatic pruning helps clear things up, at least to a first
approximation.  Some per-entry work afterward may then polish up the
details.

\mylittlespace This \mymarginpar{\texttt{usecompiler=\\true}} option
enables automatic use of the name of the compiler (in the
\textsf{namec} field) at the head of an entry, usually in the absence
of an \textsf{author}, \textsf{editor}, or \textsf{translator}, in
accordance with the specification (\emph{Manual} 17.41).  It may also,
like \texttt{useauthor}, \texttt{useeditor}, and
\texttt{usetranslator}, be disabled on a per-entry basis by setting
\texttt{usecompiler=false} in the \textsf{options} field.  Please
remember that, because \textsf{namec} isn't a standard
\textsf{biblatex} field, this name won't be part of its standard name
algorithms, and that any entry headed by a \textsf{namec} will
therefore need a \textsf{shortauthor} for short notes and a
\textsf{sortkey} or the like in order to have it appear in the correct
place in the bibliography.  (The exception to this is when you modify
the \textsf{editor's} identifying string using the \textsf{editortype}
field, which is the procedure I recommend if the entry-heading
compiler is only a compiler, and not also, e.g., an editor or a
translator.)

\subsubsection{Style Options -- Preamble}
\label{sec:useropts}

These are parts of the specification that not everyone will wish to
enable.  All except the second can be used even if you load the
package in the old way via a call to \textsf{biblatex}, but most users
can just place the appropriate string(s) in the options to the
\cmd{usepackage\{biblatex-chicago\}} call in your preamble.

\mylittlespace At \mymarginpar{\texttt{annotation}} the request of
Emil Salim, I included in \textsf{biblatex-chicago} the ability to
produce annotated bibliographies.  If you turn this option on then the
contents of your \textsf{annotation} (or \textsf{annote}) field will
be printed after the bibliographical reference.  (You can also use
external files to store annotations -- please see
\textsf{biblatex.pdf} �~3.10.7 for details on how to do this.)  This
functionality is currently in a beta state, so before you use it
please have a look at the documentation for the \textsf{annotation}
field, on page~\pageref{sec:annote} above.

\mylittlespace Although \mymarginpar{\texttt{footmarkoff}} the
\emph{Manual} (16.25) recommends specific formatting for footnote (and
endnote) marks, i.e., superscript in the text and in-line in foot- or
endnotes, Charles Schaum has brought it to my attention that not all
publishers follow this practice, even when requiring Chicago style.  I
have retained this formatting as the default setup, but if you include
the \texttt{footmarkoff} option, \textsf{biblatex-chicago-notes} will
not alter \LaTeX 's (or the \textsf{endnote} package's) defaults in
any way, leaving you free to follow the specifications of your
publisher.  I have placed all of this code in
\textsf{biblatex-chicago.sty}, so if you load the package with a call
to \textsf{biblatex} instead, then once again footnote marks will
revert to the \LaTeX\ default, but of course you also lose a fair
amount of other formatting, as well.  See section~\ref{sec:loading},
below.

\mylittlespace The \mymarginpar{\texttt{juniorcomma}} \emph{Manual}
(6.49) states that \enquote{commas are no longer required around
  \emph{Jr.}\ and \emph{Sr.},} so by default \textsf{biblatex-chicago}
has followed standard \textsf{biblatex} in using a simple space in
names like \enquote{John Doe Jr.}  Charles Schaum has pointed out that
traditional \textsc{Bib}\TeX\ practice was to include the comma, and
since the \emph{Manual} has no objections to this, I have provided an
option which allows you to turn this behavior back on, either for the
whole document or on a per-entry basis.  Please note, first, that
numerical suffixes (John Doe III) never take the comma.  The code
tests for this situation, and detects cardinal numbers well, but if
you are using ordinals you may need to set this to \texttt{false} in
the \textsf{options} field of some entries.  Second, I have fixed a
bug in older releases which always printed the \enquote{Jr.}\ part of
the name immediately after the surname, even when the surname came
before the given names (as in a bibliography).  The package now
correctly puts the \enquote{Jr.}\ part at the end, after the given
names, and in this position it always takes a comma, the presence of
which is unaffected by this option.

\mylittlespace This \mymarginpar{\texttt{natbib}} may look like the
standard \textsf{biblatex} option, but to keep the coding of
\textsf{biblatex-chicago.sty} simpler for the moment I have
reimplemented it there, from whence it is merely passed on to
\textsf{biblatex}.  If you load the Chicago style with
\cmd{usepackage\{biblatex-chicago\}}, then the option should simply
read \texttt{natbib}, rather than \texttt{natbib=true}.  The shorter
form also works if you load the style using
\cmd{usepackage[style=chicago-notes]\{biblatex\}}, so I hope this
requirement isn't too onerous.

\mylittlespace At \mymarginpar{\texttt{noibid}} the request of an
early tester, I have included this option to allow you globally to
turn off the \texttt{ibidem} mechanism that
\textsf{biblatex-chicago-notes} uses by default.  Some publishers, it
would appear, require this.  Setting this option will mean that all
possible instances of \emph{ibid.}\ will be replaced by the short note
form.  For more fine-grained control of individual citations you'll
probably want to use specialized citation commands, instead.  See
section \ref{sec:citecommands}.

%\enlargethispage{\baselineskip}

\mylittlespace This \mymarginpar{\texttt{short}} option means that
your text will only use the short note form, even in the first
citation of a particular work.  The \emph{Manual} (16.3) recommends
this space-saving format only when you provide a \emph{full}
bibliography, though even with such a bibliography you may feel it
easier for your readers to present long first citations.  If you do
use the \texttt{short} option, remember that there are several
citation commands which allow you to present the full reference in
specific cases (see section \ref{sec:citecommands}).  If your
bibliography is not complete, then you should not use this option.

\mylittlespace Chris Sparks \mymarginpar{\texttt{shorthandibid}}
pointed out that \textsf{biblatex-chicago-notes} would never use
\emph{ibid.}\ in the case of entries containing a \textsf{shorthand}
field, but rather that consecutive references to such an entry
continued to provide the shorthand, instead.  The \emph{Manual} isn't,
as far as I can tell, completely clear on this question.  In 17.252,
discussing references to works from classical antiquity, it states
that \enquote{when abbreviations are used, these rather than
  \emph{ibid.}\ should be used in succeeding references to the same
  work,} but I can't make out whether this rule is specific to
classical references or has more general scope.  Given this ambiguity,
I don't think it unreasonable to provide an option to allow printing
of \emph{ibid.}\ instead of the shorthand in such circumstances,
though the default behavior remains the same as it always has.

\mylittlespace This \mymarginpar{\texttt{strict}} still-experimental
option attempts to follow the \emph{Manual}'s recommendations (16.57)
for formatting footnotes on the page, using no rule between them and
the main text unless there is a run-on note, in which case a short
rule intervenes to emphasize this continuation.  I haven't tested this
code very thoroughly, and it's possible that frequent use of floats
might interfere with it.  Let me know if it causes problems.


\subsection{General Usage Hints}
\label{sec:hints}

\subsubsection{Loading the Style}
\label{sec:loading}

With the addition of the author-date style to the package, I have
provided two keys for choosing which style to load, \texttt{notes} and
\texttt{authordate}, one of which you put in the options to the
\cmd{usepackage} command.  The default way of loading the notes +
bibliography style has therefore slightly changed.  With early
versions of \textsf{biblatex-chicago-notes}, the standard way of
loading the package was via a call to \textsf{biblatex}, e.g.:
\begin{quote}
  \cmd{usepackage[style=chicago-notes,strict,backend=bibtex8,\%\\
    babel=other,bibencoding=inputenc]\{biblatex\}}
\end{quote}
Now, the default way to load the style, and one that will in the
vast majority of standard cases produce the same results as the old
invocation, will look like this:
\begin{quote}
  \cmd{usepackage[notes,strict,backend=bibtex8,babel=other,\%\\
    bibencoding=inputenc]\{biblatex-chicago\}}
\end{quote}

(In point of fact, the previous \textsf{biblatex-chicago} loading
method without the \texttt{notes} option will still work, but only
because I've made the notes \&\ bibliography style the default if no
style is explicitly requested.)  If you read through
\textsf{biblatex-chicago.sty}, you'll see that it sets a number of
\textsf{biblatex} options aimed at following the Chicago
specification, as well as setting a few formatting variables intended
as reasonable defaults (see section~\ref{sec:presetopts}, above).
Some parts of this specification, however, are plainly more
\enquote{suggested} than \enquote{required,} and indeed many
publishers, while adopting the main skeleton of the Chicago style in
citations, nonetheless maintain their own house styles to which the
defaults I have provided do not conform.

\mylittlespace If you only need to change one or two parameters, this
can easily be done by putting different options in the call to
\textsf{biblatex-chicago} or redefining other formatting variables in
the preamble, thereby overriding the package defaults.  If, however,
you wish more substantially to alter the output of the package,
perhaps to use it as a base for constructing another style altogether,
then you may want to revert to the old style of invocation above.
You'll lose all the definitions in \textsf{biblatex-chicago.sty},
including those to which I've already alluded and also the code that
sets the note number in-line rather than superscript in endnotes or
footnotes.  Also in this file is the code that calls
\textsf{cms-american.lbx}, which means that you'll lose all the
Chicago-specific bibstrings I've defined unless you provide, in your
preamble, a \cmd{DeclareLanguageMapping} command adapted for your
setup, on which see section~\ref{sec:international} below and also
��~4.9.1 and 4.11.7 in Lehman's \textsf{biblatex.pdf}.

%\enlargethispage{-\baselineskip}

\mylittlespace What you \emph{will not} lose is the ability to call
the package options \texttt{annotation, strict, short,} and
\texttt{noibid} (section~\ref{sec:useropts}, above), in case these
continue to be useful to you when constructing your own modifications.
There's very little code, therefore, actually in
\textsf{biblatex-chicago.sty}, but I hope that even this minimal
separation will make the package somewhat more adaptable.  Any
suggestions on this score are, of course, welcome.

\subsubsection{Other Hints}
\label{sec:otherhints}

One useful rule, when you are having difficulty creating a .bib entry,
is to ask yourself whether all the information you are providing is
strictly necessary.  The Chicago specification is a very full one, but
the \emph{Manual} is actually, in many circumstances, fairly relaxed
about how much of the data from a work's title page you need to fit
into a reference.  Authors of introductions and afterwords, multiple
publishers in different countries, the real names of authors more
commonly known under pseudonyms, all of these are candidates for
exclusion if you aren't making specific reference to them, and if you
judge that their inclusion won't be of particular interest to your
readers.  Of course, any data that may be of such interest, and
especially any needed to identify and track down a reference, has to
be present, but sometimes it pays to step back and reevaluate how much
information you're providing.  I've tried to make
\textsf{biblatex-chicago-notes} robust enough to handle the most
complex, data-rich citations, but there may be instances where you can
save yourself some typing by keeping it simple.

\mylittlespace Scot Becker has pointed out to me that the inverse
problem not only exists but may well become increasingly common, to
wit, .bib database entries generated by bibliographic managers which
helpfully provide as much information as is available, including
fields that users may well wish not to have printed (ISBN, URL, DOI,
\textsf{pagetotal}, inter alia).  The standard \textsf{biblatex}
styles contain a series of options, detailed in \textsf{biblatex.pdf}
�3.1.2.2, for controlling the printing of some of these fields, and
with this release I have implemented the ones that are relevant to
\textsf{biblatex-chicago}, along with a couple that Scot requested and
that may be of more general usefulness.  There is also a general
option to excise with one command all the fields under consideration
-- please see section~\ref{sec:chicpreset} above.

\mylittlespace If you are having problems with the interaction of
punctuation and quotation marks in notes or bibliography, first please
check that you've used \cmd{mkbibquote} in the relevant part of your
.bib file.  If you are still getting errors, please let me know, as it
may well be a bug.

\mylittlespace For the \textsf{biblatex-chicago-notes} style, I have
fully adopted \textsf{biblatex's} system for providing punctuation at
the end of entries.  Several users noted insufficiencies in previous
releases of \textsf{biblatex-chicago}, sometimes related to the
semicolon between multiple citations, sometimes to ineradicable
periods after long notes, bugs that were byproducts of my attempt to
fix other end-of-entry errors.  One of the side effects of this older
code was (wrongly) to put a period after a long note produced, e.g.,
by a command like \cmd{footnote\{\textbackslash headlessfullcite\}},
whereas only the \enquote{foot} cite commands (including
\cmd{autocite} in the default \textsf{biblatex-chicago-notes} set up)
should do so.  If you came to rely on this side effect, please note
now that you'll have to put the period in yourself when explicitly
calling \cmd{footnote}, like so: \cmd{footnote\{\textbackslash
  headlessfullcite\{key\}.\}}

\mylittlespace When you use abbreviations at the ends of fields in
your .bib file (e.g., \enquote{\texttt{n.d.}} or
\enquote{\texttt{Inc.},}) \textsf{biblatex-chicago-notes} should deal
automatically with adding (or suppressing) appropriate punctuation
after the final dot.  This includes retaining periods after such dots
when a closing parenthesis intervenes, as in (n.d.).  Merely entering
the abbreviation without informing \textsf{biblatex} that the final
dot is a dot and not a period should always work, though you do have
to provide manual formatting in those rare cases when you need a comma
after the author's initials in a bibliography, usually in a
\textsf{misc} entry (see house:papers).  If you find you need to
provide such formatting elsewhere, please let me know.

\mylittlespace Finally, allow me to reiterate what Philipp Lehman says
in \textsf{biblatex.pdf}, to wit, use \textsf{bibtex8}, rather than
standard \textsc{Bib}\TeX, and avoid the cryptic errors that ensue
when your .bib file gets to a certain size.

\section{The Specification: Author-Date}
\label{sec:authdate}

In what follows, I attempt to explain all the parts of
\textsf{biblatex-chicago-authordate} that might be considered somehow
\enquote{non standard,} at least with respect to the styles included
with \textsf{biblatex} itself, though in the section on entry fields I
have also duplicated a lot of the information in
\textsf{biblatex.pdf}, which I hope won't badly annoy expert users of
the system.  Headings in \mycolor{green} \colmarginpar{\textsf{New in
    this release}} indicate material new to this release, or
occasionally old material that has undergone significant revision.
Numbers in parentheses refer to sections of the \emph{Chicago Manual
  of Style}, 15th edition.  The file \textsf{dates-test.bib} contains
many examples from the \emph{Manual} which, when processed using
\textsf{biblatex-chicago-authordate}, should produce the same output
as you see in the \emph{Manual} itself, or at least compliant output,
where the specifications are vague or open to interpretation, a state
of affairs which does sometimes occur.  I have provided
\textsf{cms-dates-sample.pdf}, which shows how my system processes
\textsf{dates-test.bib}, and I have also included the reference keys
from the latter file below in parentheses.

\subsection{Entry Types}
\label{sec:types:authdate}

The complete list of entry types currently available in
\textsf{biblatex-chicago-authordate}, minus the odd \textsf{biblatex}
alias, is as follows: \textbf{article}, \textbf{artwork},
\textbf{audio}, \textbf{book}, \textbf{bookinbook}, \textbf{booklet},
\textbf{collection}, \textbf{customc}, \textbf{image},
\textbf{inbook}, \textbf{incollection}, \textbf{inproceedings},
\textbf{inreference}, \mycolor{\textbf{letter}}, \textbf{manual},
\mycolor{\textbf{misc}}, \textbf{music}, \textbf{online} (with its
alias \textbf{www}), \textbf{patent}, \textbf{periodical},
\textbf{proceedings}, \textbf{reference}, \textbf{report} (with its
alias \textbf{techreport}), \textbf{review}, \textbf{suppbook},
\textbf{suppcollection}, \textbf{suppperiodical}, \textbf{thesis}
(with its aliases \textbf{mastersthesis} and \textbf{phdthesis}),
\textbf{unpublished}, and \textbf{video}.

\mylittlespace What follows is an attempt to specify all the
differences between these types and the standard provided by
\textsf{biblatex}.  If an entry type isn't discussed here, then it is
safe to assume that it works as it does in the standard styles.  In
general, I have attempted not to discuss specific entry fields here,
unless such a field is crucial to the overall operation of a given
entry type.  As a general and important rule, most entry types require
very few fields when you use \textsf{biblatex-chicago-authordate}, so
it seemed to me better to gather information pertaining to fields in
the next section.

\mybigspace The \mymarginpar{\textbf{article}} \emph{Chicago Manual of
  Style} (17.148) recognizes three different sorts of periodical
publication, \enquote{journals,} \enquote{magazines,} and
\enquote{newspapers.}  The first (17.150) includes \enquote{scholarly
  or professional periodicals available mainly by subscription,} while
the second refers to \enquote{weekly or monthly} publications that are
\enquote{available either by subscription or in individual issues at
  bookstores or newsstands.}  \enquote{Magazines} will tend to be
\enquote{more accessible to general readers,} and typically won't have
a volume number.

\mylittlespace Now, for articles in \enquote{journals} you can simply
use the traditional \textsc{Bib}\TeX\ --- and indeed \textsf{biblatex}
--- \textsf{article} entry type, which will work as expected and set
off the page numbers with a colon in the list of references, as
required by the \emph{Manual}.  If, however, you need to refer to a
\enquote{magazine} or a \enquote{newspaper,} then you need to add an
\textsf{entrysubtype} field containing the exact string
\texttt{magazine}.  The main formatting differences between a
\texttt{magazine} (which includes both \enquote{magazines} and
\enquote{newspapers}) and a plain \textsf{article} are that time
specifications (month, day, season) aren't placed within parentheses,
and that page numbers are set off by a comma rather than a colon.
Otherwise, the two sorts of reference have much in common.  (For
\textsf{article}, see \emph{Manual} 17.154--181; batson,
beattie:crime, chu:panda, connell:chronic, conway:evolution,
friedman:learning, garaud:gatine, garrett:15, hlatky:hrt, kern,
lewis:15, loften:hamlet, loomis:structure:15, rozner:liberation,
schneider:mittelpleistozaene, terborgh:preservation, wall:ra\-di\-o,
white:callima\-chus. With \textsf{entrysubtype} \texttt{magazine},
cf.\ 17.166, 17.182--198; assocpress:gun, lakeforester:pushcarts,
morgenson:market, reaves:ro\-sen, stenger:privacy.)

\mylittlespace The \emph{Manual} suggests that \enquote{a list of
  works cited need not list newspaper items if these have been
  documented in the text} (17.191).  This involves giving the title of
the journal and the full date of publication in a parenthetical
reference, including any other information in the main text, thereby
obviating the need to present such an entry in the list of references.
To utilize this method in the author-date style, in addition to a
\texttt{magazine} \textsf{entrysubtype}, you'll need to place
\mycolor{\texttt{cmsdate=full}} into the \textsf{options} field,
including \texttt{skipbib} there as well to stop the entry printing in
the list of references.  If the entry only contains a \textsf{date}
and \textsf{journaltitle} that's enough, but if it's a fuller entry
also containing an \textsf{author} then you'll also need
\texttt{useauthor=false} in the \textsf{options} field.  Other surplus
fields will be ignored.

\mylittlespace If you are familiar with the notes \&\ bibliography
style, you'll know that the \emph{Manual} treats reviews (of books,
plays, performances, etc.) as a sort of recognizable subset of
\enquote{journals,} \enquote{magazines,} and \enquote{newspapers,}
distinguished mainly by the way one formats the title of the review
itself.  In the author-date style, however, since both a generic title
like \enquote{review of \ldots} and a specific one (cf. gibbard:15;
osborne:poison:15) are formatted in the same way (no quotation marks or
italics, sentence-style capitalization), all you really need,
conveniently, is the \textsf{article} type, with the
\textsf{entrysubtype} toggle to distinguish the sort of periodical
which contains the review.

%\enlargethispage{-\baselineskip}

\mylittlespace In the case of a review with a specific as well as a
generic title, the former goes in the \textsf{title} field, and the
latter in the \textsf{titleaddon} field.  Standard \textsf{biblatex}
intends this field for use with additions to titles that may need to
be formatted differently from the titles themselves, and
\textsf{biblatex-chicago-authordate} uses it in just this way, with
the additional wrinkle that it can, if needed, replace the
\textsf{title} entirely, and this in, effectively, any entry type,
providing a fairly powerful, if somewhat complicated, tool for getting
\textsc{Bib}\TeX\ to do what you want.  Here, however, if all you need
is a generic title like \enquote{review of \ldots,} then you can
simply use the \textsf{title} field for it, ignoring
\textsf{titleaddon}.

\mylittlespace No less than seven more things need explication under
this heading.  First, since the \emph{Manual} specifies, for the
author-date style, that what goes into the \textsf{title} or the
\textsf{titleaddon} fields of \textsf{article} entries stays
unformatted --- no italics, no quotation marks --- this plain style
(with sentence-style capitalization, as usual) is the default for such
text, which means that you'll have to format any titles within these
fields yourself, e.g., with \cmd{mkbibemph\{\}}.  Second, the
\emph{Manual} specifies a similar plain style for the titles of other
sorts of material found in \enquote{magazines} and
\enquote{newspapers,} e.g., obituaries, letters to the editor,
interviews, the names of regular columns, and the like, though the
names of regular columns, please note, need to be capitalized headline
style.  References may contain both the title of an individual article
and the name of the regular column, in which case the former should
go, as usual, in a \textsf{title} field, and the latter in
\textsf{titleaddon}.  (See 17.188, 17.190, 17.193; morgenson:market,
reaves:rosen.)

\mylittlespace Third, the \emph{Manual} suggests that
\enquote{unsigned newspaper articles or features are best dealt with
  in text or notes.  But if a \ldots\ reference-list entry should be
  needed, the name of the newspaper stands in place of the author}
(17.192).  It doesn't always carry through on this in its own
presentation of newspaper citations (see esp.\ 17.188), but I've
implemented their recommendation nonetheless, which means that in an
\textsf{article} (or a \textsf{review}) entry, \textsf{entrysubtype}
\texttt{magazine}, and only in such an entry, a missing
\textsf{author} field results in the name of the periodical (in the
\textsf{journaltitle} field) being used as the missing author.  You
can also place \mycolor{\texttt{cmsdate=full}} and \texttt{skipbib}
into the \textsf{options} field to produce an augmented in-text
citation whilst keeping this material out of the reference list.  If
the citation does appear in the reference list, the new default
sorting scheme in \textsf{biblatex-chicago-authordate} means that you
no longer need the \textsf{sortkey} field to alphabetize by
\textsf{journaltitle} instead of \textsf{title}, though you will still
need one if you retain the definite or indefinite article at the
beginning of the \textsf{journaltitle}.  Also, if you want to
abbreviate the \textsf{journaltitle} for use in citations, then the
\textsf{shortauthor} field, somewhat surprisingly, is the place for
it.  (See section~\ref{sec:authformopts}, below;
lakeforester:pushcarts, nyt:trevorobit, unsigned:ranke.)

%\enlargethispage{-\baselineskip}

\mylittlespace Fourth, in certain fields, just beginning your data
with a lowercase letter activates the mechanism for capitalizing that
letter depending on its context within a list of references entry.
This is less important in the author-date style, where this
information only turns up in the reference list and not in citations,
but you can consult \textbf{\textbackslash autocap} below for all the
details.  Both the \textsf{titleaddon} and \textsf{note} fields are
among those treating their data this way, and since both appear
regularly in \textsf{article} entries, I thought the problem merited a
preliminary mention here.

\mylittlespace Fifth, if you need to cite an entire issue of any sort
of periodical, rather than one article in an issue, then the
\textsf{periodical} entry type, once again with or without the
\texttt{magazine} toggle in \textsf{entrysubtype,} is what you'll
need.  (You can also use the \textsf{article} type, placing what would
normally be the \textsf{issuetitle} in the \textsf{title} field and
retaining the usual \textsf{journaltitle} field, but this arrangement
isn't compatible with standard \textsf{biblatex}.)  The \textsf{note}
field is where you place something like \enquote{special issue} (with
the small \enquote{s} enabling the automatic capitalization routines),
whether you are citing one article or the whole issue
(conley:fifthgrade, good:wholeissue).  Indeed, this is a somewhat
specialized use of \textsf{note}, and if you have other sorts of
information you need to include in an \textsf{article} or
\textsf{periodical} entry, then you shouldn't put it in the
\textsf{note} field, but rather in \textsf{titleaddon} or perhaps
\textsf{addendum} (brown:bremer).

\mylittlespace Sixth, I would suggest that if you wish to cite a
television or radio broadcast, the \textsf{article} type,
\textsf{entrysubtype} \texttt{magazine} is the place for it.  The name
of the program would go in \textsf{journaltitle}, with the name of the
episode in \textsf{title}.  The network's name goes into the
\textsf{usera} field.  (8.196, 17.207; see bundy:macneil for an
example of how this all might look in a .bib file.)

\mylittlespace Finally, if you're planning to use the same .bib file
for the author-date and for the notes \&\ bibliography style, you may
need to look under the \textsf{article} and \textsf{review} entries in
section~\ref{sec:entrytypes} above for the full instructions on their
differences.  Any well-constructed \textsf{review} entry will work
just fine under author-date, but if you only need the author-date
style then you can avoid the complexity of learning another entry
type.

\mylittlespace If you're still with me, allow me to recommend that you
browse through \textsf{dates-test.bib} to get a feel for just how many of
the \emph{Manual}'s complexities the \textsf{article} and
\textsf{periodical} types attempt to address.  It may be that in
future releases of \textsf{biblatex-chicago} I'll be able to simplify
these procedures somewhat, but if you are only using author-date at
least you've avoided some of the worst of it.

\mybigspace Arne \mymarginpar{\textbf{artwork}} Kjell Vikhagen has
pointed out to me that none of the standard entry types were
straightforwardly adaptable when referring to visual artworks.  The
\emph{Manual} doesn't give any thorough specifications for such
references, and indeed it's unclear that it believes it necessary to
include them in the reference apparatus at all.  Still, it's easy to
conceive of contexts in which a list of artworks studied might be
desirable, and \textsf{biblatex} includes entry types for just this
purpose, though the standard styles leave them undefined.  The two I
have adopted are \textsf{artwork} and \textsf{image}, the former
intended for paintings, sculptures, etchings, and the like, the latter
for photographs.  The two entry types work in exactly the same way as
far as constructing your .bib entry, and when printed the only
difference will be that the titles of \textsf{artworks} are
italicized, those of \textsf{images} in plain text.

\mylittlespace As one might expect, the artist goes in \textsf{author}
and the name of the work in \textsf{title}.  The \textsf{type} field
is intended for the medium --- e.g., oil on canvas, charcoal on paper
--- and the \textsf{version} field might contain the state of an
etching.  You can place the dimensions of the work in \textsf{note},
and the current location in \textsf{organization},
\textsf{institution}, and/or \textsf{location}, in ascending order of
generality.  The \textsf{type} field, as in several other entry types,
uses \textsf{biblatex's} automatic capitalization routines, so if the
first word only needs a capital letter at the beginning of a sentence,
use lowercase in the .bib file and let \textsf{biblatex} handle it for
you.  (See \emph{Manual} 12.33; leo:madonna, bedford:photo.)

\mylittlespace As a final complication, the \emph{Manual} (8.206) says
that \enquote{the names of works of antiquity \ldots\,are usually set
  in roman.}  If you should need to include such a work in the
reference apparatus, you can either define an \textsf{entrysubtype}
for an \textsf{artwork} entry --- anything will do --- or you could
use the \textsf{image} type, or you could try the \textsf{misc} entry
type with an \textsf{entrysubtype}.  Fortunately, in this instance the
other fields in a \textsf{misc} entry function pretty much as in
\textsf{artwork} or \textsf{image}.

\mybigspace For \mymarginpar{\textbf{audio}} this release of
\textsf{biblatex-chicago}, following the request of Johan Nordstrom, I
have included three new entry types, all undefined by the standard
styles, designed to allow users to present audiovisual sources in
accordance with the Chicago specifications.  The \emph{Manual's}
presentation of such sources (17.263--273), though admirably brief,
seems to me somewhat inconsistent.  I attempted to condense all the
requirements into two new entry types, but ended up relying on three,
the differences between which I shall attempt to delineate here.
There are likely to be occasions when your choice of entry type is not
obvious, but at the very least \textsf{biblatex-chicago} should help
you maintain consistency.  For users of the author-date style, it is
as well to note here that the \emph{Manual} (17.265) suggests that
\enquote{such materials are best mentioned in running text and grouped
  in the reference list under a subhead,} a suggestion you may wish to
follow, particularly if your audiovisual entries don't typically
contain the information --- \textsf{author} and \textsf{date} ---
needed to produce a parseable parenthetical citation.

\mylittlespace The \textbf{music} type is intended for all musical
recordings that do not have a video component.  This means, for
example, digital media (whether on CD or hard drive), vinyl records,
and tapes.  The \textbf{video} type includes (nearly) all visual
media, whether it be films, TV shows, tapes and DVDs of the preceding
or of any sort of performance (including music), or online multimedia.
Finally, the \textbf{audio} type, our current concern, fills gaps in
the two others, and presents its sources in a more \enquote{book-like}
manner.  Published musical scores need this type --- unpublished ones
would use \textsf{misc} with an \textsf{entrysubtype}
(shapey:partita:15) --- as do such favorite educational formats as the
slideshow and the filmstrip (greek:filmstrip:15, schubert:muellerin,
verdi:corsaro).  The \emph{Manual} (17.269--270) sometimes uses a
similar format for audio books and even for films (twain:audio,
weed:flatiron), though elsewhere these sorts of material are presented
as \textsf{music} and \textsf{video}, respectively.  It would appear
to depend on which sorts of publication facts you wish to present ---
cf.\ \emph{Manual} 17.269.

\mylittlespace Once you've accepted the analogy of composer to
\textsf{author}, constructing an \textsf{audio} entry should be fairly
straightforward, since many of the fields function just as they do in
\textsf{book} or \textsf{inbook} entries.  Indeed, please note that I
compare it to both these other types as, in common with the other
audiovisual types, \textsf{audio} has to do double duty as an analogue
for both books and collections, so while there will normally be an
\textsf{author}, a \textsf{title}, a \textsf{publisher}, a
\textsf{date}, and a \textsf{location}, there may also be a
\textsf{booktitle} and/or a \textsf{maintitle} --- see
schubert:muellerin for an entry that uses all three in citing one song
from a cycle.  If the medium in question needs specifying, the
\textsf{type} field is the place for it.  (It is characteristic of
this entry type that such information is placed after the publisher
information, whereas in the other audiovisual types their order is
reversed.)  Finally, the \textsf{titleaddon} field can specify
functions for which \textsf{biblatex-chicago} provides no automated
handling, e.g., a librettist (verdi:corsaro).

\mybigspace This \mymarginpar{\textbf{bookinbook}} type provides the
means of referring to parts of books that are considered, in other
contexts, themselves to be books, rather than chapters, essays, or
articles.  (Older versions of \textsf{biblatex-chicago} used
\textbf{customb} for this purpose, but this is now deprecated.)  Such
an entry can have a \textsf{title} and a \textsf{maintitle}, but it
can also contain a \textsf{booktitle}, all three of which will be
italicized in the reference matter.  In general usage it is,
therefore, rather like the traditional \textsf{inbook} type, only with
its \textsf{title} in italics rather than in plain text.  (See
\emph{Manual} 17.72, 17.89, 17.93; bernard:boris, euripides:orestes,
plato:republic:gr.)

\mylittlespace \textbf{NB}: The Euripides play receives slightly
different presentations in 17.89 and 17.93.  Although the
specification is very detailed, it doesn't eliminate all choice or
variation.  Using a system like \textsc{Bib}\TeX\ should help to
maintain consistency.

\mybigspace This \mymarginpar{\textbf{booklet}} is the first of two
entry types --- the other being \textsf{manual}, on which see below
--- which are traditional in \textsc{Bib}\TeX\ styles, but which the
\emph{Manual} (17.241) suggests may well be treated basically as
books.  In the interests of backward compatibility,
\textsf{biblatex-chicago-authordate} will so format such an entry,
which uses the \textsf{howpublished} field instead of a standard
\textsf{publisher}, though of course if you do decide just to use a
\textsf{book} entry then any information you might have given in a
\textsf{howpublished} field should instead go in \textsf{publisher}.
(See clark:mesopot.)

\mybigspace This \mymarginpar{\textbf{customa}} entry type is now
obsolete, and any such entries in your .bib file will trigger an
error.  Please use the standard \textsf{biblatex} \textbf{letter} type
instead.

%\enlargethispage{-\baselineskip}

\mybigspace This \mymarginpar{\textbf{customb}} entry type is now
obsolete, and any such entries in your .bib file will trigger an
error.  Please use the standard \textsf{biblatex} \textbf{bookinbook}
type instead.

\mybigspace This \mymarginpar{\textbf{customc}} entry type has
undergone a metamorphosis with this release, as I previously warned
both here and in the RELEASE file.  Rather than being a (deprecated)
alias of the standard \textsf{biblatex} \textbf{suppbook}, it now
allows you to include the expansions of abbreviations and shorthands
\emph{inside} the list of references itself, as recommended by the
\emph{Manual} (17.47), rather than in the list of shorthands.

\mylittlespace In its simplest form, you need merely place the
abbreviation of the (often institutional) author's name into the
\textsf{author} field, and its expansion into the \textsf{title}
field.  To make sure it appears in the list of references, you can
either manually include the entry key in a \cmd{nocite} command, or
you can put that entry key in the \textsf{userc} field in the main
.bib entry, in which case \textsf{biblatex-chicago-authordate} will
print the expanded abbreviation if and only if you cite the main
entry.  (See abbrev:BSI, abbrev:ISO, bsi:abbreviation:15,
iso:electrodoc:15.)  Under ordinary circumstances,
\textsf{biblatex-chicago} will connect the abbreviation and its
expansion with the word \enquote{\emph{See}} --- or its equivalent in
the document's language --- in italics.  If you wish to present the
cross-reference differently, you can put the connecting word(s) into
the \textsf{nameaddon} field.

\mylittlespace I haven't disabled in any way the mechanisms
\textsf{biblatex} uses to produce the list of shorthands, so the
\cmd{printshorthands} command will still print that list in your
document.  Judicious use of \texttt{skiplos} in the \textsf{options}
field will therefore enable simultaneous use of several methods of
shorthand presentation, if that is useful.  It may be as well to
mention here that the \emph{Manual} (17.39--40) offers other uses for
this alphabetized cross-referencing system in reference lists, which
may also prove helpful for the author-date style.  See the discussion
of \textsf{customc} in section~\ref{sec:entrytypes} of the notes \&\
bibliography docs, above.

\mylittlespace Finally, you may need to use this entry type if you
wish to include a comment inside the parentheses of a citation, as
specified by the \emph{Manual} (16.111).  If you have a
\textsf{postnote}, then you can manually provide the punctuation and
comment there, e.g., \cmd{autocite[4; the unrevised
  trans.]\{stendhal:parma\}}.  Without a \textsf{postnote}, you'll
need a separate \textsf{misc} or \textsf{customc} entry containing
just the text of the comment in the \textsf{title} field,
\textsf{entrysubtype} \texttt{classical}, and \textsf{options}
\texttt{skipbib}.  An \cmd{autocites} command calling both the main
text and the comment will do the trick, e.g.,
\cmd{autocites\{chicago:manual:15\}\{chicago:comment\}}.

\mybigspace This \mymarginpar{\textbf{image}} entry type, left
undefined in the standard styles, is in
\textsf{biblatex-chicago-authordate} intended for referring to
photographs.  Excluding the possible use of the \textsf{entrysubtype}
field, which in an \textsf{image} entry would be ignored, this type is
a clone of \textsf{artwork}, so you should consult the latter's
documentation above to see how to construct your .bib entry.  (See
\emph{Manual} 12.33; bedford:photo.)

\enlargethispage{\baselineskip}

\mybigspace These \mymarginpar{\textbf{inbook}\\\textbf{incollection}}
two standard \textsf{biblatex} types have very nearly identical
formatting requirements as far as the Chicago specification is
concerned, but I have retained both of them for compatibility.
\textsf{Biblatex.pdf} (�~2.1.1) intends the first for \enquote{a part
  of a book which forms a self-contained unit with its own title,}
while the second would hold \enquote{a contribution to a collection
  which forms a self-contained unit with a distinct author and its own
  title.}  The \textsf{title} of both sorts will be in plain text, and
in general you can use either type for most material falling into
these categories.  There is, however, an important difference between
them, as it is only in \textsf{incollection} entries that I implement
the \emph{Manual's} recommendations for space-saving abbreviations in
the list of references when you cite multiple pieces from the same
\textsf{collection}.  These abbreviations are activated when you use
the \textsf{crossref} or \textsf{xref} field in \textsf{incollection}
entries, and not in \textsf{inbook} entries, mainly because the
\emph{Manual} (17.70) here specifies a \enquote{multiauthor book.}
(For more on this mechanism see \textbf{crossref}, below, and note
that it is also active in \textsf{letter} and \textsf{inproceedings}
entries.  There is, of course, nothing to prevent you from using the
mechanism when referring to, e.g., chapters from a single-author book,
but you'll have to use \textsf{incollection} instead of
\textsf{inbook}.)  If the part of a book to which you are referring
has had a separate publishing history as a book in its own right, then
you may wish to use the \textsf{bookinbook} type, instead, on which
see above.  (See \emph{Manual} 17.68--72; \textsf{inbook}:
ashbrook:brain, phibbs:diary, will:cohere; \textsf{incollection}:
centinel:letters, contrib:contrib, sirosh:visualcortex; ellet:galena,
keating:dearborn, and lippincott:chicago [and the \textsf{collection}
entry prairie:state] demonstrate the use of the \textsf{crossref}
field with its attendant abbreviations in the list of references.)

\mylittlespace \textbf{NB}: The \emph{Manual} suggests that, when
referring to a chapter, one use either a chapter number or the
inclusive page numbers, not both.  In-text citations, of course,
require any \textsf{postnote} field to specify if it is a whole
chapter to which you are referring.

\mybigspace This \mymarginpar{\textbf{inproceedings}} entry type works
pretty much as in standard \textsf{biblatex}.  Indeed, the main
differences between it and \textsf{incollection} are the lack of an
\textsf{edition} field and the possibility that an
\textsf{organization} may be cited alongside the \textsf{publisher},
even though the \emph{Manual} doesn't specify its use (17.71).  Please
note, also, that the \textsf{crossref} and \textsf{xref} mechanism for
shortening citations of multiple pieces from the same
\textsf{proceedings} is operative here, just as it is in
\textsf{incollection} entries.  See \textbf{crossref}, below, for more
details.

\mybigspace This \mymarginpar{\textbf{inreference}} entry type is
aliased to \textsf{incollection} in the standard styles, but the
\emph{Manual's} requirements for the notes \&\ bibliography style
prompted a thoroughgoing revision.  Unfortunately, instructions for
the author-date style are considerably less copious, so parts of what
follows are my best guess at following the specification
(17.238--239).

\mylittlespace One thing, at least, seems clear.  If your reference
work can easily or conveniently be presented like a regular book, that
is, with an author or editor, a year of publication, and a title, and
if you you will be citing it by page or section number, then you
should almost certainly simply choose the \textsf{book} entry type for
your .bib entry. (Cf.\ mla:style, schellinger:novel, times:guide.  The
latter was presented as an \textsf{inreference} entry for the notes
\&\ bibliography style, but because the \textsf{book} entry type can
also present references to alphabetized headings [see below], at least
in the list of references, then it seemed better just to choose a
\textsf{book} entry for the author-date style.)

\mylittlespace If you simply cannot make your source fit the template
for a \textsf{book}, then you may need to use the \textsf{inreference}
type, the main feature of which is the \textsf{lista} field, which you
use to present citations from \enquote{alphabetically arranged} works
by named article rather than by page number.  You should present these
article names just as they appear in the work, separated by the
keyword \enquote{\texttt{and}} if there is more than one, and
\textsf{biblatex-chicago-authordate} will provide the appropriate
prefatory string (\texttt{s.v.}, plural \texttt{s.vv.}), and enclose
each in its own set of quotation marks (times:guide).  More relevant
to the author-date style is the fact that the \textsf{postnote} field
works the same way in \textsf{inreference} entries, the only
limitation on this system being that this field, unlike
\textsf{lista}, is not a list, and therefore for the formatting to
work correctly you can only put one article name in it.  In the case
of \enquote{[w]ell-known reference books, such as major dictionaries
  and encyclopedias,} you are encouraged not to include them in the
list of references, so the \textsf{lista} field actually may be of
less use than this special formatting of \textsf{postnote}.  You may
want to look at ency:britannica, where only a (carefully-formatted)
\textsf{shorttitle} and an \textsf{options} field are necessary to
allow you to produce in-text citations that look like (\emph{Ency.\
  Brit.}\ 15th ed., s.v. \enquote{Article}).

\mylittlespace If it seems appropriate to include such a work in the
list of references, perhaps because the work is not so well known that
a short citation will be parseable by your readers, or perhaps because
it is an online work, which requires you to provide a \textsf{urldate}
(see below), be aware that the contents of the \textsf{lista} field
will also be presented there, which may not be what you want.  A
separate \textsf{inreference} or \textsf{reference} entry might solve
this problem, but you may also need a \textsf{sortkey} field to ensure
proper alphabetization, as \textsf{biblatex} will attempt to use an
\textsf{editor} or \textsf{author} name, if either is present.  In a
typical \textsf{inreference} entry, very few fields are needed, as
\enquote{the facts of publication are often omitted, but the edition
  (if not the first) must be specified.}  In practice, this means a
\textsf{title} and possibly an \textsf{edition} field.  The
\textsf{author} field holds the author of the specific article (in
\textsf{lista}), not the author of the \textsf{title} as a whole.
This name will be printed in parentheses after the entry's name
(grove:sibelius).

\mylittlespace All of these rules apply to online reference works, as
well, for which you need to provide not only a \textsf{url} but also,
always, a \textsf{urldate}, as these sources are in constant flux
(wikiped:bibtex, grove:sibelius).  For author-date, it may be
convenient to duplicate the \textsf{urlyear} in the \textsf{year}, as
this will help to present and categorize the material both in
citations and in the list of references.  Please note, however, that
the automatic provision of the \enquote{n.d.} abbreviation when a
\textsf{year} is missing has been turned off for \textsf{inreference}
entries, as for \textsf{misc} and \textsf{reference} entries.

\mylittlespace Some of these presentational difficulties might make
switching between the two Chicago styles rather more difficult,
depending on the nature of your sources.  The advice I offer in
section~\ref{sec:twostyles} below may be of assistance.

\mybigspace This \colmarginpar{\textbf{letter}} entry type was designed
to be used for citing letters, memoranda, or similar texts, but
\emph{only} when they appear in a published collection.  (Unpublished
material of this nature needs a \textsf{misc} entry, for which see
below.)  The author-date specification (17.77), however, recommends
against individual letters appearing in a list of references,
suggesting instead that you put the whole published collection in a
\textsf{book} entry and use a notice in the text to specify the letter
(white:total).

\mylittlespace If you absolutely must include individual letters in
the list of references, for whatever reason, then the instructions
above for the notes \&\ bibliography style in
section~\ref{sec:entrytypes}, s.v.\ \enquote{\textsf{letter,}} should
get you started.  There are a few wrinkles, related to date
specifications, that I shall attempt to clarify here.  If you look at
white:ross:memo and white:russ, you'll see two letters from the same
published collection, both written in the same year.  You can now
simply use the \textsf{origdate} field in both of them, because in the
absence of a \textsf{date} (or an \textsf{eventdate}) \textsf{Biber}
and \textsf{biblatex} will use the \textsf{origyear} as the
\textsf{labelyear}, putting it at the head of the entry and in the
citation, and also ensuring that the letters \texttt{a,b,c} are
appended to disambiguate the two sources.  You no longer need anything
in the \textsf{options} field at all, thanks to the way
\cmd{DeclareLabelyear} works through the possibilities and finds a
date to head the entry.  In this case, it works because we are using
the \textsf{xref} mechanism to refer to the whole published collection
(white:total), so a separate citation of that entry provides the
\textsf{date} for the shortened cross-reference included in the list
of references, and the \textsf{letter} entry never sees that
\textsf{date} at all.

\mylittlespace If this all seems clear as mud, I'm not surprised, but
let me suggest that you experiment with the different date settings to
see what kinds of effects they have on the final result, and also read
the documentation of the \textsf{date} field in
section~\ref{sec:fields:authdate} below.

\mybigspace This \mymarginpar{\textbf{manual}} is the second of two
traditional \textsc{Bib}\TeX\ entry types that the \emph{Manual}
suggests formatting as books, the other being \textsf{booklet}. As
with this latter, I have retained it in
\textsf{biblatex-chicago-authordate} for backward compatibility, its
main peculiarity being that, in the absence of a named author, the
\textsf{organization} producing the manual will be provided both as
author and as publisher.  (You can give a shortened form of the
\textsf{organization} in the \textsf{shortauthor} field for text
citations, if needed.)  Of course, if you were to use a \textsf{book}
entry for such a reference, then you would need to define both
\textsf{author} and \textsf{publisher} using the name you here might
have put in \textsf{organization}.  (See 17:47; chicago:manual:15,
dyna:browser, natrecoff:camera.)

\mybigspace As \colmarginpar{\textbf{misc}} its name suggests, the
\textsf{misc} entry type was designed as a hold-all for citations that
didn't quite fit into other categories.  In \textsf{biblatex-chicago},
I have somewhat extended its applicability, while retaining its
traditional use.  Put simply, with no \textsf{entrysubtype} field, a
\textsf{misc} entry will retain backward compatibility with the
standard styles, so the usual \textsf{howpublished}, \textsf{version},
and \textsf{type} fields are all available for specifying an otherwise
unclassifiable text, and the \textsf{title} will be italicized.  (The
\emph{Manual}, you may wish to note, doesn't give specific
instructions on how such citations should be formatted, so when using
the Chicago style I would recommend you have recourse to this
traditional entry type as sparingly as possible.)

\mylittlespace If you do provide an \textsf{entrysubtype} field, the
\textsf{misc} type provides a means for citing unpublished letters,
memoranda, private contracts, wills, interviews, and the like, making
it something of an unpublished analogue to the \textsf{letter} and
\textsf{article} entry types (which see).  Typically, such an entry
will cite part of an archive, and equally typically the text cited
won't have a specific title, but only a generic one, whereas an
\textsf{unpublished} entry will ordinarily have a specific author and
title, and won't come from a named archive.  As a rule, and as with
the \textsf{letter} type, the \emph{Manual} (17.233) suggests that the
list of references will usually contain only the name of the whole
archived collection, with more specific information about individual
items provided in the text, \enquote{outside the parentheses.}  If, on
the other hand, \enquote{only one item from a collection has been
  mentioned in text, the entry may begin with the writer's name (if
  known).}  (See 17.205-206, 17.220, 17.222-232; house:papers cites a
whole archive, while creel:house, dinkel:agassiz, and spock:interview
cite individual pieces.)

\mylittlespace As far as constructing your .bib entry goes, you should
first know that, like the \textsf{inreference} and \textsf{reference}
types, the absence of any date will not result in the \enquote{n.d.}
abbreviation automatically being provided.  As for presenting the
date, the \emph{Manual} draws a distinction between archival material
that is \enquote{letter-like} (letters, memoranda, reports, telegrams)
and that which isn't (interviews, wills, contracts, or even personal
communications you've received and which you wish to cite).  This may
not always be the easiest distinction to draw, and in previous
releases of \textsf{biblatex-chicago} I have been ignoring it, but
once you've decided to classify it one way or the other you put the
date in the \textsf{origdate} field for letters, etc.\ (creel:house),
and into the \textsf{date} field for the others (spock:interview).
Like with the \textsf{letter} type, if the only date present is an
\textsf{origdate}, you no longer need to set the \texttt{cmsdate}
option in your .bib entry to make sure that that year appears at the
head of the entry (and in citations) --- this now happens
automatically.  (Cf.\ particularly the documentation in
section~\ref{sec:fields:authdate} below, s.v.\ \enquote{date}, and
also the \textsf{letter} type above for some of the date-related
complications that can arise, and how you can address them with
judicious use of the \textsf{options}, \textsf{date}, and
\textsf{origdate} fields.)

\mylittlespace As in \textsf{letter} entries, the titles of
unpublished letters are of the form \texttt{Author to Recipient},
further information can be given in the \textsf{titleaddon} field,
while the \textsf{origlocation} field can hold the place where the
letter was written.  Interviews or similar pieces will have a
different sort of title, but all types will use the \textsf{note},
\textsf{organization}, \textsf{institution}, and \textsf{location}
fields (in ascending order of generality) to identify the archive,
though the \emph{Manual} specifies (17.228) that well-known
depositories don't usually need a city, state or country specified.
(The traditional \textsf{misc} fields are all still available, also.)

\mylittlespace When your .bib entry refers to an entire archived
collection, then you may wish to use the word
\enquote{\texttt{classical}} as your \textsf{entrysubtype}, which will
have no effect on the list of references but will change the look of
the in-text citations (house:papers).  Instead of any date, the
citation will include the \textsf{title}, separated from the
\textsf{author's} name by a space, e.g., (House Papers).  This same
arrangement, happily, allows you easily to cite individual books of
the Bible, and also certain other sacred texts (17.246--49; genesis).
Please see under \textsf{entrysubtype} in
section~\ref{sec:fields:authdate} below for all the details of the
\texttt{classical} toggle.

\mylittlespace In all this class of archived material, the
\emph{Manual} (17.222) quite specifically requires more consistency
within your own work than conformity to some external standard, so it
is the former which you should pursue.  I hope that
\textsf{biblatex-chicago} proves helpful in this regard.

\mybigspace This \mymarginpar{\textbf{music}} is one of three new
audiovisual entry types, and is intended primarily to aid in the
presentation of musical recordings that do not have a video component,
though it can also include audio books (auden:reading:15).  A DVD or
VHS of an opera or other performance, by contrast, should use the
\textbf{video} type instead (handel:messiah:15).  Because
\textsf{biblatex} --- and \textsc{Bib}\TeX\ before it --- were
designed primarily for citing book-like objects, some choices needed
to be made in assigning the various roles found on the back of a CD to
the fields in a typical .bib entry.  I have also implemented several
new bibstrings to help in identifying these roles within entries.  If
you can think of a simpler way to distribute the roles, please let me
know, so that I can consider making changes before anyone gets used to
the current equivalences.

\mylittlespace These equivalences, in summary form, are:

{\renewcommand{\descriptionlabel}[1]{\qquad\textsf{#1}}
\begin{description}
\item[author =] composer, songwriter, or performer(s),
  depending on whom you wish to emphasize by placing them at the head
  of the entry.
\item[editor, editora, editorb =] conductor, director or
  performer(s).  These will ordinarily follow the \textsf{title} of
  the work, though the usual \texttt{useauthor} and \texttt{useeditor}
  options can alter the presentation within an entry.  Because these
  are non-standard roles, you will need to identify them using the
  following:
\item[editortype, editoratype, editorbtype:] The most common roles,
  all associated with specific bibstrings (or their absence), will be
  \texttt{conductor}, \texttt{director}, \texttt{producer}, and,
  oddly, \texttt{none}.  The last is particularly useful when
  identifying the group performing a piece, as it usually doesn't need
  further specifying and this role prevents \textsf{biblatex} from
  falling back on the default \texttt{editor} bibstring.
\item[title, booktitle, maintitle:] As with the other audiovisual
  types, \textsf{music} serves as an analogue both to books and to
  collections, so the title will either be, e.g., the album title or a
  song title, in which latter case the album title would go into
  \textsf{booktitle}.  The \textsf{maintitle} might be necessary for
  something like a box set of \emph{Complete Symphonies}.
\item[series, number:] These two are closely associated, and are
  intended for presenting the catalog information provided by the
  music publisher, especially in the case when a publisher oversees
  more than one label.  In nytrumpet:art:15, for example, the
  \textsf{series} field holds the label (\texttt{Vox/Turn\-about}) and
  the \textsf{number} field the catalog number (\texttt{PVT 7183}).
  You can certainly put all of this information into one of the above
  fields, but separating it may help make the .bib entry more
  readable.
\item[howpublished/pubstate:] The \emph{Manual} (17.268) follows the
  rather specialized requirements for presenting publishing
  information for musical recordings.  The normal symbol for musical
  copyright is\ \texttt{\textcircledP} (Unicode point u+2117, SOUND
  RECORDING COPYRIGHT), but other copyrights \texttt{\textcopyright}
  are often also asserted.  The \textsf{howpublished} field is the
  place for these symbols, and it may also have to hold a year
  designation if the \texttt{\textcircledP} and the
  \texttt{\textcopyright} apply to different years, as sometimes
  happens.  In keeping with its general usage in the author-date
  style, but also recognizing the peculiarities of this entry type, I
  have made the \textsf{pubstate} field a synonym for
  \textsf{howpublished}.  Please choose only one of them per entry,
  and remember that the automatic presentation of reprints via the
  \textsf{pubstate} field is disabled in \textsf{music} entries.  (See
  nytrumpet:art:15.)
\item[date, publisher:] Ordinarily, you can use a combination of the
  \textsf{date} and \textsf{origdate} fields, along with the
  \texttt{cmsdate} entry option, to present the various dates of
  publication and republication of a work.  This will mostly still
  work in \textsf{music} entries, but, as I just pointed out, the
  automatic presentation of reprints via the \textsf{pubstate} field
  has been turned off, as it doesn't provide a good fit for the
  materials at hand.  Instead, the \textsf{howpublished} or
  \textsf{pubstate} field can be used manually to present the
  publishing complexities, including year information that won't be
  placed at the head of an entry.  Any year data you do wish placed
  there needs to go in \textsf{date} or \textsf{origdate}, as usual.
  Thankfully, the \textsf{publisher} field itself is self-explanatory.
\item[type:] As in all the audiovisual entry types, the \textsf{type}
  field holds the medium of the recording, e.g., vinyl, 33 rpm,
  8-track tape, cassette, compact disc, mp3, ogg vorbis.
\end{description}}

I should also note here that I have implemented the standard
\textsf{biblatex} \textbf{eventdate} field, in case you need it to
identify a particular recording session or concert.  It will be
printed just after the \textsf{title}.  The entries in
\textsf{dates-test.bib} should at least give you a good idea of how
this all works, and that file also contains an example of an audio
book presented in a \textsf{music} entry.  If you browse the examples
in the \emph{Manual} you will see that no author-date examples are
given, so I have generally adopted the formatting decisions I made for
the notes \&\ bibliography style.  Arguments as to why I'm wrong will,
of course, be entertained.  (Cf. 17.268; auden:reading:15,
beethoven:sonata29, bernstein:shostakovich, nytrumpet:art:15.)

\mybigspace The \mymarginpar{\textbf{online}} \emph{Manual}'s
instructions (17.142--147, 17.198, 17.234--237) for citing online
materials are slightly different from those suggested by standard
\textsf{biblatex}.  Indeed, this is a case where complete backward
compatibility with other \textsf{biblatex} styles may be impossible,
because as a general rule the \emph{Manual} considers relevant not
only where a source is found, but also the nature of that source,
e.g., if it's an online edition of a book (james:ambassadors), then it
calls for a \textsf{book} entry.  Even if you cite an
\enquote{intrinsically online} source, if that source is structured
more or less like a conventional printed periodical, then you'll
probably want to use \textsf{article} or \textsf{review} instead of
\textsf{online} (stenger:privacy, which cites \emph{CNN.com} ---
\emph{Yahoo!\ News} is another example that would be treated in such a
way).  If the \enquote{standard facts of publication} are missing,
then the \textsf{online} type is usually the best choice
(evanston:library, powell:email).  Some online materials will, no
doubt, make it difficult to choose an entry type, but so long as all
locating information is present, then perhaps that is enough to
fulfill the specification, or at least so I'd like to hope.

\mylittlespace Constructing an \textsf{online} .bib file entry is much
the same as in \textsf{biblatex}.  The \textsf{title} field would
contain the title of the page, the \textsf{organization} field could
hold the title or owner of the whole site.  If there is no specific
title for a page, but only a generic one (powell:email), then in the
author-date style the \textsf{title} will serve just as well as
\textsf{titleaddon}, which latter is required for the notes \&\
bibliography style.

%\enlargethispage{\baselineskip}

\mybigspace The \mymarginpar{\textbf{patent}} \emph{Manual} is very
brief on the subject of patents (17.219), but very clear about which
information it wants you to present, so such entries may not work well
with other \textsf{biblatex} styles.  The important date, as far as
Chicago is concerned, is the filing date.  If a patent has been filed
but not yet granted, then you can place the filing date in either the
\textsf{date} field or the \textsf{origdate} field, and
\textsf{biblatex-chicago-authordate} will automatically prepend the
bibstring \texttt{patentfiled} to it.  If the patent has been granted,
then you put the filing date in the \textsf{origdate} field, and you
put the date it was issued in the \textsf{date} field, to which the
bibstring \texttt{patentissued} will automatically be prepended.  In
this entry type, and in no other, the \texttt{cmsdate=on} option is
turned on by default, so that the filing date will be at the head of
the entry in the list of references and in the citation, as well.  If
you have more than one patent by the same author(s) filed in the same
year, and if one or both of them have also been granted, then you'll
need to reverse the dates (or put \texttt{switchdates} in the
\textsf{options} field) so that \enquote{a,b,c} etc.\ can be appended
to the year.  (If there is just the one, filing, date, please don't
use the \texttt{switchdates} option.)  The patent number goes in the
\textsf{number} field, and you should use the standard
\textsf{biblatex} bibstrings in the \textsf{type} field.  Though it
isn't mentioned by the \emph{Manual},
\textsf{biblatex-chicago-authordate} will print the \textsf{holder}
after the \textsf{author}, if you provide one.  See petroff:impurity.

\mybigspace This \mymarginpar{\textbf{periodical}} is the standard
\textsf{biblatex} entry type for presenting an entire issue of a
periodical, rather than one article within it.  It has the same
function in \textsf{biblatex-chicago}, and in the main uses the same
fields, though in keeping with the system established in the
\textsf{article} entry type (which see) you'll need to provide
\textsf{entrysubtype} \texttt{magazine} if the periodical you are
citing is a \enquote{newspaper} or \enquote{magazine} instead of a
\enquote{journal.}  Also, remember that the \textsf{note} field is the
place for identifying strings like \enquote{special issue,} with its
initial lowercase letter to activate the automatic capitalization
routines, though this isn't strictly necessary in the author-date
style.  (See \emph{Manual} 17.170; good:wholeissue.)

\mybigspace This \mymarginpar{\textbf{reference}} entry type is
aliased to \textsf{collection} by the standard \textsf{biblatex}
styles, but I intend it to be used in cases where you need to cite a
reference work but not an alphabetized article or articles in that
work.  This could be because it doesn't contain such articles, and yet
you still want the entry in the list of references to start with the
\textsf{title}.  Indeed, the only differences between it and
\textsf{inreference} are the lack of a \textsf{lista} field to present
an alphabetized entry, and the fact that any \textsf{postnote} field
will be printed verbatim, rather than formatted as an alphabetized
entry.  (Cf.\ \textsf{inreference}, above.)

\mybigspace This \mymarginpar{\textbf{report}} entry type is a
\textsf{biblatex} generalization of the traditional \textsc{Bib}\TeX\
type \textsf{techreport}.  Instructions for such entries are rather
thin on the ground in the \emph{Manual} (17.241), so I have followed
the generic advice about formatting it like a book, and hope that the
results conform to the specification.  Its main peculiarities are the
\textsf{institution} field in place of a \textsf{publisher}, the
\textsf{type} field for identifying the kind of report in question,
and the \textsf{isrn} field containing the International Standard
Technical Report Number of a technical report.  As in standard
\textsf{biblatex}, if you use a \textsf{techreport} entry, then the
\textsf{type} field automatically defaults to
\cmd{bibstring\{techreport\}}.  As with \textsf{booklet} and
\textsf{manual}, you can also use a \textsf{book} entry, putting the
report type in \textsf{note} and the \textsf{institution} in
\textsf{publisher}.  (See herwign:office.)

\mybigspace As \mymarginpar{\textbf{review}} its name suggests, this
entry type was designed for reviews published in periodicals, and if
you've already read the \textsf{article} documentation above you'll
know that I haven't yet found an example where you absolutely need to
use it for the author-date style.  The code to process such an entry
remains in \textsf{biblatex-chicago-authordate}, so if you are
building a .bib file for use with both Chicago styles then any
\textsf{review} entries in it will work fine in both, but otherwise
the \textsf{article} type will suffice.  If you find I'm wrong about
this, please let me know.  (Cf.\ barcott:review:15, bundy:macneil,
Clemens:letter, gourmet:052006, kozinn:review, nyt:trevorobit,
ratliff:review:15, unsigned:ranke, wallraff:word.)

%\enlargethispage{\baselineskip}

\mybigspace This \mymarginpar{\textbf{suppbook}} is the entry type to
use if the main focus of a reference is supplemental material in a
book or in a collection, e.g., an introduction, afterword, or forward,
either by the same or by a different author.  There are two mechanisms
in \textsf{biblatex-chicago} for producing such a citation.  First,
these three just-mentioned types of material, and only these three
types, can be referenced using the \textsf{introduction},
\textsf{afterword}, or \textsf{foreword} fields, a system that
requires you simply to define one of them in any way and leave the
others undefined.  The macros don't use the text provided by such an
entry, they merely check to see if one of them is defined, in order to
decide which sort of pre- or post-matter is at stake, and to print the
appropriate string before the \textsf{title} in the list of
references, and possibly also in the list of shorthands.  This
mechanism works without modification across multiple languages, but I
have also provided functionality which allows you to cite any sort of
supplemental material whatever, using the \textsf{type} field.  Under
this second system, simply put the nature of the material, including
the relevant preposition, in that field, beginning with a lowercase
letter so \textsf{biblatex} can decide whether it needs capitalization
depending on the context.  Examples might be \enquote{\texttt{preface
    to}} or \enquote{\texttt{colophon of}.}  (Please note, however,
that unless you use a \cmd{bibstring} command in the \textsf{type}
field, the resultant entry will not be portable across languages.)

\mylittlespace The other rules for constructing your .bib entry remain
the same.  The \textsf{author} field refers to the author of the
introduction or afterword, while \textsf{bookauthor} refers to the
author of the main text of the work, if the two differ.  If the focus
of the reference is the main text of the book, but you want to mention
the name of the writer of an introduction or afterword for
completeness, then the normal \textsf{biblatex} rules apply, and you
can just put their name in the appropriate field of a \textsf{book}
entry, that is, in the \textsf{foreword}, \textsf{afterword}, or
\textsf{introduction} field.  (See \emph{Manual} 17.74--75;
friedman:intro, polakow:afterw, prose:intro).

\mybigspace This \mymarginpar{\textbf{suppcollection}} fulfills a
function analogous to \textsf{suppbook}.  Indeed, I believe the
\textbf{suppbook} type can serve to present supplemental material in
both types of work, so this entry type is an alias to
\textsf{suppbook}, which see.

\mybigspace This \mymarginpar{\textbf{suppperiodical}} type is
intended to allow reference to generically-titled works in
periodicals, such as regular columns or letters to the editor.
\textsf{Biblatex} also provides the \textsf{review} type for this
purpose, and in the notes \&\ bibliography style
\textsf{suppperiodical} is an alias of \textsf{review}.  In the
author-date style, however, as discussed above, you really only need
the \textsf{article} entry type for this purpose, though I have
retained \textsf{suppperiodical} in order to facilitate switching
between the two Chicago styles.

\mybigspace This \mymarginpar{\textbf{video}} is the last of the new
audiovisual entry types, and as its name suggests it is intended for
citing visual media, be it films of any sort or TV shows, broadcast,
on the Net, on VHS, DVD, or Blu-ray.  As with the \textsf{music} type
discussed above, certain choices had to be made when associating the
production roles found, e.g., on a DVD, to those bookish ones provided
by \textsf{biblatex}.  Here are the main correspondences:

{\renewcommand{\descriptionlabel}[1]{\qquad\textsf{#1}}
\begin{description}
\item[author:] This will not infrequently be left undefined, as the
  director of a film should be identified as such and therefore placed
  in the \textsf{editor} field with the appropriate
  \textsf{editortype} (see below).  You will need it, however, to
  identify the composer of, e.g., an oratorio on VHS
  (handel:messiah:15), or perhaps the provider of commentaries or
  other extras on a film DVD (cleese:holygrail).
\item[editor, editora, editorb:] The director or producer, or possibly
  the performer or conductor in recorded musical performances.  These
  will ordinarily follow the \textsf{title} of the work, though the
  usual \texttt{useauthor} and \texttt{useeditor} options can alter
  the presentation within an entry.  Because these are non-standard
  roles, you will need to identify them using the following:
\item[editortype, editoratype, editorbtype:] The most common roles,
  all associated with specific bibstrings (or their absence), will
  likely be \texttt{director}, \texttt{produ\-cer}, and, oddly,
  \texttt{none}.  The last is particularly useful if you want to
  identify performers, as they usually don't need further specifying
  and this role prevents \textsf{biblatex} from falling back on the
  default \texttt{editor} bibstring.
\item[title, titleaddon, booktitle, maintitle:] As with the other
  audiovisual types, \textsf{video} serves as an analogue both to
  books and to collections, so the \textsf{title} may be of a whole
  film DVD or of a TV series, or it may identify one episode in a
  series or one scene in a film.  In the latter cases, the title of
  the whole would go in \textsf{booktitle}.  The \textsf{titleaddon}
  field may be useful for specifying the season and/or episode number
  of a TV series, or for any other information that needs to come
  between the \textsf{title} and the \textsf{booktitle}
  (cleese:holygrail, episode:tv, handel:messiah:15).  As in the
  \textsf{music} type, \textsf{maintitle} may be necessary for a boxed
  set or something similar.
\item[date, origdate, pubstate:] The publication details of this sort
  of material are usually straightforward, at least compared with the
  \textsf{music} type, but there will be occasions when you need two
  dates.  When citing an episode of a long-running TV series you may
  need both a date for the episode and either a range for the whole
  run or a year for the release of the box set, and when citing a film
  on DVD you may want to present both the original release date and
  the date of release on DVD.  In these cases, the usual
  \textsf{chicago-authordate} mechanisms for choosing the date to
  appear at the head of an entry apply.  You can also use the standard
  \textsf{pubstate} field with \texttt{reprint} in it to control the
  printing of the original date in parentheses at the end of the
  entry, though I have altered the string that is printed there (see
  next item).  I have also disabled the printing of the
  \cmd{bibstring\{reprint\}} before the publication information, as it
  doesn't really apply to this class of material.  Cf.\ episode:tv,
  hitchcock:nbynw; \textsf{pubstate}, below.
\item[entrysubtype:] In most entry types, the string printed in
  parentheses to date the original appearance of a work is
  \enquote{\texttt{Orig.\ pub.}}  This won't work in \textsf{video},
  so by default in this entry type it will print
  \enquote{\texttt{Orig.\ released.}}  As this isn't quite right for
  TV shows, you can place the exact string \texttt{tv} into the
  \textsf{entrysubtype} field to obtain \enquote{\texttt{Orig.\
      shown.}}  Alternatively, you can put whatever you like in the
  \textsf{pubstate} field, including parentheses and the year if you
  want them, and all of it will appear where it should.  (The
  \emph{Manual} gives no guidance on presenting televisual materials,
  so I've improvised.  Any improvements will be gratefully
  considered.)
\item[type:] As in all the audiovisual entry types, the \textsf{type}
  field holds the medium of the \textsf{title}, e.g., 8 mm, VHS, DVD,
  Blu-ray, MPEG.
\end{description}}

As with the \textsf{music} type, entries in \textsf{dates-test.bib}
should at least give you a good idea of how all this works.  (Cf.\
17.270, 273; cleese:holygrail, episode:tv, handel:messiah:15,
hitchcock:nbynw, loc:city.)

\subsection{Entry Fields}
\label{sec:fields:authdate}

The following discussion presents, in alphabetical order, a complete
list of the entry fields you will need to use
\textsf{biblatex-chicago-authordate}.  As in
section~\ref{sec:types:authdate}, I shall include references to the
numbered paragraphs of the \emph{Chicago Manual of Style}, and also to
the entries in \textsf{dates-test.bib}.  Many fields are most easily
understood with reference to other, related fields.  In such cases,
cross references should allow you to find the information you need.

\mybigspace As \mymarginpar{\textbf{addendum}} in standard
\textsf{biblatex}, this field allows you to add miscellaneous
information to the end of an entry, after publication data but before
any \textsf{url} or \textsf{doi} field.  In the \textsf{patent} entry
type (which see), it will be printed in close association with the
filing and issue dates.  In any entry type, if your data begins with a
word that would ordinarily only be capitalized at the beginning of a
sentence, then simply ensure that that word is in lowercase, and the
style will take care of the rest.  Cf.\ \textsf{note}. (See
\emph{Manual} 17.145, 17.123; davenport:attention, natrecoff:camera.)

\mybigspace In most \mymarginpar{\textbf{afterword}} circumstances,
this field will function as it does in standard \textsf{biblatex},
i.e., you should include here the author(s) of an afterword to a given
work.  The \emph{Manual} suggests that, as a general rule, the
afterword would need to be of significant importance in its own right
to require mentioning in the reference apparatus, but this is clearly
a matter for the user's judgment.  As in \textsf{biblatex}, if the
name given here exactly matches that of an editor and/or a translator,
then \textsf{biblatex-chicago} will concatenate these fields in the
formatted references.

\mylittlespace As noted above, however, this field has a special
meaning in the \textsf{suppbook} entry type, used to make an
afterword, foreword, or introduction the main focus of a citation.  If
it's an afterword at issue, simply define \textsf{afterword} any way
you please, leave \textsf{foreword} and \textsf{introduction}
undefined, and \textsf{biblatex-chicago} will do the rest. Cf.\
\textsf{foreword} and \textsf{introduction}. (See \emph{Manual} 17.46,
17.74; polakow:afterw.)

\enlargethispage{\baselineskip}

\mybigspace At \mymarginpar{\textbf{annotation}} the request of Emil
Salim, \textsf{biblatex-chicago} has, as of version 0.9, added a
package option (see \texttt{annotation} below, section
\ref{sec:useropts}) to allow you to produce annotated lists of
references.  The formatting of such a list is currently fairly basic,
though it conforms with the \emph{Manual's} minimal guidelines
(16.77).  The default in \textsf{chicago-authordate.cbx} is to define
\cmd{DeclareFieldFormat\{an\-notation\}} using \cmd{par}\cmd{nobreak}
\cmd{vskip} \cmd{bibitemsep}, though you can alter it by re-declaring
the format in your preamble.  The page-breaking algorithms don't
always give perfect results here, but the default formatting looks, to
my eyes, fairly decent.  In addition to tweaking the field formatting
you can also insert \cmd{par} (or even \cmd{vadjust\{\cmd{eject}\}})
commands into the text of your annotations to improve the appearance.
Please consider the \texttt{annotation} option a work in progress, but
it is usable now.  (N.B.: The \textsc{Bib}\TeX\ field \textsf{annote}
serves as an alias for this.)

\mybigspace I \mymarginpar{\textbf{annotator}} have implemented this
\textsf{biblatex} field pretty much as that package's standard styles
do, even though the \emph{Manual} doesn't actually mention it.  It may
be useful for some purposes.  Cf.\ \textsf{commentator}.

\mybigspace For \mymarginpar{\textbf{author}} the most part, I have
implemented this field in a completely standard \textsc{Bib}\TeX\
fashion.  Remember that corporate or organizational authors need to
have an extra set of curly braces around them (e.g.,
\texttt{\{\{Associated Press\}\}}\,) to prevent \textsc{Bib}\TeX\ from
treating one part of the name as a surname (17.47, 17.197;
assocpress:gun, chicago:manual:15).  If there is no \textsf{author}, then
\textsf{biblatex-chicago} will look, in sequence, for an
\textsf{editor}, \textsf{translator}, or \textsf{compiler} (actually
\textsf{namec}, currently) and use that name (or those names) instead,
followed by the appropriate identifying string (esp.\ 17.41, also
17.28--29, 17.88, 17.95, 17.172; boxer:china, brown:bremer,
harley:cartography, schellinger:novel, sechzer:women, silver:ga\-wain,
soltes:georgia).  \textsf{Biber} now takes care of alphabetizing
entries no matter which name appears at their head, and the package
also automatically provides a name for citations.

\mylittlespace As its name suggests, the author-date style very much
wants to have a name of some sort present both for the entries in the
list of references and for the in-text citations.  Indeed,
\enquote{this system works best where all or most of the sources are
  easily convertible to author-date references} (16.4).  The
\emph{Manual} is nothing if not flexible, however, so with unsigned
articles or encyclopedia entries the \textsf{journaltitle} or
\textsf{title} may take the place of the \textsf{author}
(gourmet:052006, lakeforester:pushcarts, nyt:trevorobit,
unsigned:ranke, wikipedia:bibtex).  Even in such \textsf{article}
entries, however, it may be advantageous to provide a (formatted and
abbreviated) \textsf{shortauthor} field to keep the in-text citations
to a reasonable length, though not at the expense of making it hard to
find the relevant entries in the reference list.

\mylittlespace If you wish to emphasize the activity of an editor or a
translator, you can use the \textsf{biblatex} and
\textsf{biblatex-chicago} options \texttt{useauthor=false},
\texttt{useeditor=false}, \texttt{usetranslator=false}, and
\texttt{usecompiler=false} in the \textsf{options} field to choose
which one appears at the head of an entry.  A peculiarity of this
system of toggles is that in order to ensure that the \textsf{title}
of a book appears at the head of an entry, you would need to use
\emph{all four} of the toggles, even though the hypothetical entry
contains no \textsf{translator}.  Internally,
\textsf{biblatex-chicago} is either searching for an
author-substitute, or it is skipping over elements of the ordered,
unidirectional chain \textsf{author -> editor -> translator ->
  compiler -> title}.  If you don't include
\texttt{usetranslator=false} in the \textsf{options} field, then the
package begins its search at \textsf{translator} and continues on to
\textsf{namec}, even though you have \texttt{usecompiler=false} in
\textsf{options}.  The result will be that the compilers' names will
appear at the head of the entry.  If you want to skip over parts of
the chain, you must turn off \emph{all} of the parts up to the one you
wish printed.  Another peculiarity of the system is that setting the
Chicago-specific \texttt{usecompiler} option to \texttt{false} doesn't
remove \textsf{namec} from the sorting list, whereas the other
standard \textsf{biblatex} toggles \emph{do} remove their names from
the sorting list, so in some corner cases you may need the
\textsf{sortkey} field.  See \cmd{DeclareSortingScheme} in
section~\ref{sec:authformopts}, below.

\mylittlespace This system of toggles, then, can turn off
\textsf{biblatex-chicago}'s mechanism for finding a name to place at
the head of an entry, but it also very usefully adds the possibility
of citing a work with an \textsf{author} by its editor, compiler or
translator instead (17.45; eliot:pound), something that wasn't
possible before.  For full details of how this works, see the
\textsf{editortype} documentation below.  (Of course, in
\textsf{collection} and \textsf{proceedings} entry types, an
\textsf{author} isn't expected, so there the \textsf{editor} is
required, as in standard \textsf{biblatex}.  Also, in \textsf{article}
entries with \textsf{entrysubtype} \texttt{magazine}, the absence of
an \textsf{author} triggers the use of the \textsf{journaltitle} in
its stead.  See those entry types for further details.)

\mylittlespace \textbf{NB}: The \emph{Manual} provides specific
instructions for formatting the names of both anonymous and
pseudonymous authors (17.32--39).  The use of \enquote{Anonymous} as
the name is \enquote{generally to be avoided,} but may in some cases
be useful \enquote{in a bibliography in which several anonymous works
  need to be grouped.}  I would add that sometimes it's the simplest
option for a difficult citation --- cf.\ virginia:plantation:15, where
placing \enquote{\texttt{Anon.}\hspace{-2pt}}\ in the \textsf{author}
field seems about the only way to fit this text into the author-date
style.  If \enquote{the authorship is known or guessed at but was
  omitted on the title page,} then you need to use the
\textsf{authortype} field to let \textsf{biblatex-chicago} know this
fact.  If the author is known (horsley:prosodies), then put
\texttt{anon} in the \textsf{authortype} field, if guessed at
(cook:sotweed) put \texttt{anon?}\ there.  (In both cases,
\textsf{biblatex-chicago} tests for these \emph{exact} strings, so
check your typing if it doesn't work.)  This will have the effect of
enclosing the name in square brackets, with or without the question
mark indicating doubt.  As long as you have the right string in the
\textsf{authortype} field, \textsf{biblatex-chicago-authordate} will
also do the right thing automatically in text citations.

\mylittlespace The \textsf{nameaddon} field furnishes the means to
cope with the case of pseudonymous authorship.  If the author's real
name isn't known, simply put \texttt{pseud.}\ (or
\cmd{bibstring\{pseudonym\}}) in that field (centinel:letters).  If
you wish to give a pseudonymous author's real name, simply include it
there, formatted as you wish it to appear, as the contents of this
field won't be manipulated as a name by \textsf{biblatex}
(lecarre:quest, stendhal:parma).  If you have given the author's real
name in the \textsf{author} field, then the pseudonym goes in
\textsf{nameaddon}, in the form \texttt{Firstname Lastname,\,pseud.}\
(creasey:ashe:blast, creasey:morton:hide, creasey:\\york:death).  This
latter method will allow you to keep all references to one author's
work under different pseudonyms grouped together in the list of
references, a method recommended by the \emph{Manual}.

\mylittlespace One final piece of advice.  An institutional author's
name, or a journal's name being used in place of an author, can be
rather too long for in-text citations.  In unsigned:ranke I placed an
abbreviated form of the \textsf{journaltitle} into
\textsf{shortauthor}, a practice condoned by the \emph{Manual}
(17.159) even in author-date reference lists.  In iso:electrodoc:15, I
provided a \textsf{shorthand} field, which by default in
\textsf{biblatex-chicago-authordate} will appear in text citations.
With this release, you can expand the abbreviation inside the list of
references itself, as suggested by the \emph{Manual}, (17.47).  Please
see \textsf{customc} above and \textsf{shorthands} below for the
details.  (A list of shorthands can still clarify the abbreviation, if
you wish.)

\mybigspace In \mymarginpar{\textbf{authortype}}
\textsf{biblatex-chicago}, this field serves a function very much in
keeping with the spirit of standard \textsf{biblatex}, if not with its
letter.  Instead of allowing you to change the string used to identify
an author, the field allows you to indicate when an author is
anonymous, that is, when his or her name doesn't appear on the title
page of the work you are citing.  As I've just detailed under
\textsf{author}, the \emph{Manual} generally discourages the use of
\enquote{Anonymous} (or \enquote{Anon.} as an author, though in some
cases it may well be your best option.  If, however, the name of the
author is known or guessed at, then you're supposed to enclose that
name within square brackets, which is exactly what
\textsf{biblatex-chicago} does for you when you put either
\texttt{anon} (author known) or \texttt{anon?} (author guessed at) in
the \textsf{authortype} field.  (Putting the square brackets in
yourself doesn't work right, hence this mechanism.)  The macros test
for these \emph{exact} strings, so check your typing if you don't see
the brackets.  Assuming the strings are correct,
\textsf{biblatex-chicago} will also automatically do the right thing
in citations.  (See the \textsf{author} docs just above.  Also
\emph{Manual} 17.33--34; cook:sotweed, horsley:prosodies.)

%\enlargethispage{-\baselineskip}

\mybigspace For \mymarginpar{\textbf{bookauthor}} the most part, as in
\textsf{biblatex}, a \textsf{bookauthor} is the author of a
\textsf{booktitle}, so that, for example, if one chapter in a book has
different authorship from the book as a whole, you can include that
fact in a reference (17.75; will:cohere).  Keep in mind, however, that
the entry type for introductions, forewords and afterwords
(\textsf{suppbook}) uses \textsf{bookauthor} as the author of
\textsf{title} (polakow:afterw, prose:intro).

\mybigspace This, \mymarginpar{\vspace{-12pt}\textbf{bookpagination}}
a standard \textsf{biblatex} field, allows you automatically to prefix
the appropriate string to information you provide in a \textsf{pages}
field.  If you leave it blank, the default is to print no identifying
string (the equivalent of setting it to \texttt{none}), as this is the
practice the \emph{Manual} recommends for nearly all page numbers.
Even if the numbers you cite aren't pages, but it is otherwise clear
from the context what they represent, you can still leave this blank.
If, however, you specifically need to identify what sort of unit the
\textsf{pages} field represents, then you can either hand-format that
field yourself, or use one of the provided bibstrings in the
\textsf{bookpagination} field.  These bibstrings currently are
\texttt{column,} \texttt{line,} \texttt{paragraph,} \texttt{page,}
\texttt{section,} and \texttt{verse}, all of which are used by
\textsf{biblatex's} standard styles.

\mylittlespace There are two points that may need explaining here.
First, all the bibstrings I have just listed follow the Chicago
specification, which may be confusing if they don't produce the
strings you expect.  Second, remember that \textsf{bookpagination}
applies only to the \textsf{pages} field --- if you need to format a
citation's \textsf{postnote} field, then you must use
\textsf{pagination}, which see (15.45--46, 17.128--138).

\mybigspace The \mymarginpar{\textbf{booksubtitle}} subtitle for a
\textsf{booktitle}.  See the next entry for further information.

\mybigspace In \mymarginpar{\textbf{booktitle}} the
\textsf{bookinbook}, \textsf{inbook}, \textsf{incollection},
\textsf{inproceedings}, and \textsf{letter} entry types, the
\textsf{booktitle} field holds the title of the larger volume in which
the \textsf{title} itself is contained as one part.  It is important
not to confuse this with the \textsf{maintitle}, which holds the more
general title of multiple volumes, e.g., \emph{Collected Works}.  It
is perfectly possible for one .bib file entry to contain all three
sorts of title (euripides:orestes, plato:republic:gr).  You may also
find a \textsf{booktitle} in other sorts of entries (e.g.,
\textsf{book} or \textsf{collection}), but there it will almost
invariably be providing information for the \textsc{Bib}\TeX\
cross-referencing apparatus (prairie:state), which I discuss below
(\textbf{crossref}).  The \textsf{booktitle} takes sentence-style
capitalization in author-date.

%\enlargethispage{\baselineskip}

\mybigspace An \mymarginpar{\textbf{booktitleaddon}} annex to the
\textsf{booktitle}.  It will be printed in the main text font, without
quotation marks.  If your data begins with a word that would
ordinarily only be capitalized at the beginning of a sentence, then
simply ensure that that word is in lowercase, and
\textsf{biblatex-chicago} will automatically do the right thing.

\mybigspace This \mymarginpar{\textbf{chapter}} field holds the
chapter number, mainly useful only in an \textsf{inbook} or an
\textsf{incollection} entry where you wish to cite a specific chapter
of a book (ashbrook:brain).

\mybigspace I \mymarginpar{\textbf{commentator}} have implemented this
\textsf{biblatex} field pretty much as that package's standard styles
do, even though the \emph{Manual} doesn't actually mention it.  It may
be useful for some purposes.  Cf.\ \textsf{annotator}.

\mybigspace \textsf{Biblatex} \mymarginpar{\textbf{crossref}} uses the
standard \textsc{Bib}\TeX\ cross-referencing mechanism, and has also
introduced a modified one of its own (\textsf{xref}).  The
\textsf{crossref} field works exactly the same as it always has, while
\textsf{xref} attempts to remedy some of the deficiencies of the usual
mechanism by ensuring that child entries will inherit no data at all
from their parents.  Having said all that, a few further instructions
may be in order for users of both \textsf{biblatex} and
\textsf{biblatex-chicago}.  First, remember that fields in a
\textsf{collection} entry, for example, differ from those in an
\textsf{incollection} entry.  In order for the latter to inherit the
\textsf{booktitle} field from the former, the former needs to have
such a field defined, even though a \textsf{collection} entry has no
use itself for such an entry (see ellet:galena, keating:dearborn,
lippincott:chicago, and prairie:state).  Note also that an entry with
a \textsf{crossref} field will mechanically try to inherit all
applicable fields from the entry it cross-references.  In the case of
ellet:galena et al., you can see that this includes the
\textsf{subtitle} field found in prairie:state, which would then,
quite incorrectly, be added to the \textsf{title} of ellet:galena.  In
cases like these, you could just make sure that prairie:state didn't
contain such a field, by placing the entire title + subtitle in the
\textsf{title} field, separated by a colon.  Alternatively, as you can
see in ellet:galena, you can just define an empty \textsf{subtitle}
field to prevent it inheriting the unwanted subtitle from
prairie:state.

\mylittlespace Turning now more narrowly to \textsf{biblatex-chicago},
the \emph{Manual} (17.70) specifies that if you cite several
contributions to the same collection, all (including the collection
itself) may be listed separately in the list of references, which the
package does automatically, using the default inclusion threshold of 2
in the case both of \textsf{crossref}'ed and \textsf{xref}'ed entries.
(The familiar \cmd{nocite} command may also help in some
circumstances.)  In the list of references an abbreviated form will be
appropriate for all the child entries.  The current version of
\textsf{biblatex-chicago-authordate} implements these instructions,
but only if you use a \textsf{crossref} or an \textsf{xref} field, and
only in \textsf{incollection}, \textsf{inproceedings}, or
\textsf{letter} entries (on the last named, see just below).  If you
look at ellet:galena, keating:dearborn, lippincott:chicago, and
prairie:state you'll see this mechanism in action in the list of
references.  If you wish to disable this, then simply don't use a
\textsf{crossref} or \textsf{xref} field in your entries.

%\enlargethispage{\baselineskip}

\mylittlespace A published collection of letters requires a somewhat
different treatment (17.77--78).  If you cite more than one letter
from the same collection, then the \emph{Manual} specifies that only
the collection itself --- probably in a \textsf{book} entry --- should
appear in the list of references.  In the author-date style, it
discourages individual letters from appearing in that list at all,
even if only one is cited from a collection.  If you have special
reason to do so, however, you can still present individual published
letters there, and they too can use the system of shortened references
just outlined, even though the \emph{Manual} doesn't explicitly
require it.  As with \textsf{incollection} and \textsf{inproceedings},
mere use of a \textsf{crossref} or \textsf{xref} field will activate
this mechanism, while avoidance of said fields will disable it.  (See
white:ross:memo, white:russ, and white:total, for examples of the
\textsf{xref} field in action in this way, and please note that the
second of these entries is entirely fictitious, provided merely for
the sake of example.)

\mylittlespace I should also take this opportunity to mention that you
need to be careful when using the \textsf{shorthand} field in
conjunction with the \textsf{crossref} or \textsf{xref} fields,
bearing in mind the complicated questions of inheritance posed by all
such cross-references, most especially in \textsf{letter},
\textsf{incollection}, and \textsf{inproceedings} entries.  A
\textsf{shorthand} field in a parent entry is, at least in the current
state of \textsf{biblatex-chicago}, a bad idea.

\mybigspace Predictably, \colmarginpar{\textbf{date}} this is one of
the key fields for the author-date style, and one which, as a general
rule, every .bib entry designed for this system ought to contain.  So
important is it, that \textsf{biblatex-chicago-authordate} will, in
most entry types, supply a missing \cmd{bibstring\{nodate\}} if there
is no date otherwise provided; citations will look like (Author n.d.),
and entries in the list of references will begin: Author, Firstname.\
n.d.  This seems simple enough, but there are a surprising number of
complications which require attention.

\mylittlespace First, with \textsf{Biber}, an absent \textsf{date}
will automatically provoke it into searching for other sorts of dates
in the entry, in the order \textsf{date, eventdate, origdate,
  urldate}.  Only when it finds no year at all will it fall back on
\cmd{bibstring\{nodate\}}.  You can eliminate some of these dates from
the running, or change the search order, using the
\cmd{DeclareLabelyear} command in your preamble, but please be aware
that I have hard-coded this order into the author-date style in order
to cope with some tricky corners of the specification.  If you reorder
these dates, and your references enter these tricky corners, the
results might be surprising.  (Cf.\ section~4.5.2 in
\textsf{biblatex.pdf} for the \cmd{DeclareLabelyear} command.)
Second, the entry types in which this automatic provision is turned
off are \textsf{inreference}, \textsf{misc}, and \textsf{reference},
none of which may be expected in the standard case to have a date
provided.  In all other entry types \enquote{\texttt{n.d.}}\ will
appear if no date is provided, though you can turn this off throughout
the document in all entry types with the option \texttt{nodates=false}
when loading \textsf{biblatex-chicago} in your preamble.  (See
section~\ref{sec:authpreset}, below.)  Third, if you wish to provide
the \enquote{\texttt{n.d.}}\ yourself in the \textsf{year} field,
please instead put \cmd{bibstring\{nodate\}} there, as otherwise the
punctuation in citations will come out (subtly) wrong.  Fourth, while
we're on the subject, the \textsf{year} field is also the place for
things like \enquote{\texttt{forthcoming},} though you should use the
\cmd{autocap} macro there to make sure the word comes out correctly in
both citations and the list of references.  The reason for this is
that the \textsf{date} field accepts only numerical data, in
\textsc{iso}8601 format (\texttt{yyyy-mm-dd}), whereas \textsf{year}
can, conveniently, hold just about anything.  It may be worth noting
here that \textsf{Biber} is somewhat more exacting when parsing the
\textsf{date} field than \textsc{Bib}\TeX, so a field looking like
\texttt{1968/75} will simply be ignored, producing
\enquote{\texttt{n.d.}}\ in the output --- you need \texttt{1968/1975}
instead.  If you want a more compressed year range, then you'll want
to use the \textsf{year} field.

\mylittlespace Fifth, for most entry types, only a year is really
necessary, and in all types only the year --- or year range --- will
be printed in text citations and at the head of entries in the list of
references.  More specific \textsf{date} fields are often present,
however, in \textsf{article}, \textsf{misc}, \textsf{online},
\textsf{patent}, and \textsf{unpublished} entries, for all of which
any day or month provided will be printed later in the reference list
entry.  If you follow the recommendations of the \emph{Manual} and
document newspaper articles within running text (17.191), then the
citations need to contain the complete \textsf{date} along with the
\textsf{journaltitle}.  Placing \mycolor{\texttt{cmsdate=full}} (and
\texttt{skipbib}) in the \textsf{options} field of an \textsf{article}
or a \textsf{review} entry, alongside a possible
\texttt{useauthor=false}, should allow you to achieve this.  The
\emph{Manual} is a little inconsistent when presenting the names of
months in the author-date style, but currently
\textsf{biblatex-chicago-authordate} uses abbreviated forms, which you
can change by setting the option \texttt{dateabbrev=false} in your
document preamble.  (Cf.\ assocpress:gun, barcott:review:15, batson,
creel:house, nass:address, petroff:impurity, powell:email.)

\mylittlespace Sixth, the \emph{Manual} (17.125--7) provides a number
of options for when a particular entry --- a reprinted edition, say
--- has more than one date, and \textsf{biblatex-chicago-authordate}
allows you to choose among all of them.  The user interface is a
little more complicated than I had hoped, but I shall attempt to
explain it here as clearly as I can.  If a reprinted book, say, has
both a \textsf{date} of publication for the reprint edition and an
\textsf{origdate} for the original edition, then by default
\textsf{biblatex-chicago-authordate} will use the \textsf{date} in
citations and at the head of the entry in the reference list.  If you
inform \textsf{biblatex-chicago} that the book is a reprint by putting
the string \texttt{reprint} in the \textsf{pubstate} field, then a
parenthetical notice will be printed at the end of the entry saying
\enquote{(Orig.\ pub.\ 1898.)}  With no \textsf{pubstate} field (and
no \texttt{cmsdate} option), the algorithms will ignore the
\textsf{origdate}.

\mylittlespace If, for any reason, you wish the \textsf{origdate} to
appear at the head of the entry, then you need to use the
\texttt{cmsdate} toggle in the \textsf{options} field.  This has 4
possible states relevant to this context, though there is a fifth
state (\mycolor{\texttt{full}}) which I've discussed two paragraphs
up:

\begin{enumerate}
\item \texttt{cmsdate=on} prints the \textsf{origdate} at the head of
  the entry in the list of references and in citations: (Author 1898).
\item \texttt{cmsdate=new} prints both the \textsf{origdate} and the
  \textsf{date}, using the \emph{Manual's}\ \enquote{new} format:
  (Author 1898/1952).
\item \texttt{cmsdate=old} prints both the \textsf{origdate} and the
  \textsf{date}, using the \emph{Manual's}\ \enquote{old} format:
  (Author [1898] 1952).
\item \texttt{cmsdate=off} is the default, discussed above:
  (Author 1952).
\end{enumerate}

In the first three cases, if you put the string \texttt{reprint} in
the \textsf{pubstate} field, then the publication data in the list of
references will include a notice, formatted according to the
specifications, that the modern, cited edition is a reprint.  In the
first case, since the \textsf{date} hasn't yet been printed, this
publication data will also include the date of the modern reprint.

\mylittlespace Let us imagine, however, that your list of references
contains another book by the same author, also a reprint edition:
(Author 1896/1974).  How will these two works be ordered in the list
of references?  By whatever appears in the \textsf{date} field, which
appears first in the default definition of \cmd{DeclareLabelyear}, and
which in this case will be wrong, because the entries should always be
ordered by the \emph{first} date to appear there, in this case the
contents of \textsf{origdate}.  In this example, the solution can be
as simple as a \textsf{sortyear} field set to something earlier than
the date of the other work, e.g., 1951.

\mylittlespace And if the original publication dates of the two works
are the same?  Just as when it is ordering entries, \textsf{biblatex}
will always first process the contents of the \textsf{date} field when
it is deciding whether to add the alphabetical suffix (\texttt{a,b,c}
etc.)  to the year to distinguish different works by the same author
published in the same year.  You can't even put the suffix on yourself
because the \textsf{origdate} field only accepts numerical data.
Citations of the two works should read, e.g., (Author 1898a) and
(Author 1898b), but will in fact read, ambiguously, (Author 1898) and
(Author 1898).  Here we are forced to resort to an unusual expedient,
which amounts to switching the two date fields, placing the earlier
date in \textsf{date} and the later one in \textsf{origdate}.
\textsf{Biblatex-chicago-authordate} tests for this condition using a
simple arithmetical comparison between the two years, then printing
the two dates according to the state of the \texttt{cmsdate} toggle.
The four states of this toggle are the same as before, but there are
only three possible outcomes, as follows:

\begin{enumerate}
\item \texttt{cmsdate=off} (the default) and \texttt{cmsdate=on}
  \emph{both} print the \textsf{date} at the head of the entry in the
  list of references and in citations: (Author 1898a), (Author 1898b).
\item \texttt{cmsdate=new} prints both the \textsf{date} and the
  \textsf{origdate}, using the \emph{Manual's}\ \enquote{new} format:
  (Author 1898a/1952), (Author 1898b/1974).
\item \texttt{cmsdate=old} prints both the \textsf{date} and the
  \textsf{origdate}, using the \emph{Manual's}\ \enquote{old} format:
  (Author [1898a] 1952), (Author [1898b] 1974).
\end{enumerate}

If, for some reason, the automatic switching of the dates cannot be
achieved, perhaps in crossref'd \textsf{letter} entries that you
really want to have in your list of references (white:ross:memo,
white:russ), or perhaps in a reprint edition that hasn't yet appeared
in print (preventing the comparison between a year and the word
\enquote{forthcoming}), then you can use the per-entry option
\texttt{switchdates} in the \textsf{options} field to achieve the
required effects.  It may also be worth remarking that the
instructions in the \emph{Manual} aren't entirely clear on the subject
of the alphabetical affix when both dates are used in a citation or at
the head of an entry.  It's possible that the differentiation between
(Author 1898/1952) and (Author 1898/1974) is good enough without
affixing anything to the first year, but then in this situation you
would have to be using either \texttt{cmsdate=new} or
\texttt{cmsdate=old}, so the switching functionality at least allows
maximum flexibility.

\mylittlespace Finally, in the \textsf{misc} entry type this field can
help to distinguish between two classes of archival material, letters
and \enquote{letter-like} sources using \textsf{origdate} while others
(interviews, wills, contracts) use \textsf{date}.  (See \textsf{misc}
in section~\ref{sec:types:authdate} for the details.)  If such an
entry, as may well occur, contains only an \textsf{origdate}, as can
also be the case in the \textsf{letter} entries I mentioned in the
previous paragraph, \textsf{Biber} and the default
\cmd{DeclareLabelyear} definition now make it possible to do without a
\texttt{cmsdate} option, as \textsf{biblatex} will in such a case use
the \textsf{origdate} to order the entries in a reference list, and
will also append the alphabetical suffix if more than one entry by the
same author has the same \textsf{origyear}.  I recommend that you have
a look through \textsf{dates-test.bib} to see how all these
complications will affect the construction of your .bib database,
especially at aristotle:metaphy:gr, creel:house, emerson:nature,
james:ambassadors, maitland:canon, maitland:equity, schweit\-zer:bach,
spock:in\-terview, white:ross:me\-mo, and white:russ.  Cf.\ also
\textsf{origdate} and \textsf{year}, below, and the \texttt{cmsdate},
\texttt{nodates}, and \texttt{switchdates} options in
sections~\ref{sec:preset:authdate} and \ref{sec:authentryopts}.

\mybigspace This \mymarginpar{\textbf{day}} field, as of
\textsf{biblatex} 0.9, is obsolete, and will be ignored if you use it
in your .bib files.  Use \textsf{date} instead.

\mybigspace Standard \mymarginpar{\textbf{doi}} \textsf{biblatex}
field.  The Digital Object Identifier of the work, which the
\emph{Manual} suggests you can use \enquote{in place of page numbers
  or other locators} (17.181; friedman:learn\-ing).  Cf.\
\textsf{url}.

%\enlargethispage{-\baselineskip}

\mybigspace Standard \mymarginpar{\textbf{edition}} \textsf{biblatex}
field.  If you enter a plain cardinal number, \textsf{biblatex} will
convert it to an ordinal (chicago:manual:15), followed by the appropriate
string.  Any other sort of edition information will be printed as is,
though if your data begins with a word (or abbreviation) that would
ordinarily only be capitalized at the beginning of a sentence, then
simply ensure that that word (or abbreviation) is in lowercase, and
\textsf{biblatex-chicago} will automatically do the right thing
(babb:peru, times:guide).  In most situations, the \emph{Manual}
generally recommends the use of abbreviations in the list of
references, but there is room for the user's discretion in specific
citations (emerson:nature).

\mybigspace As \mymarginpar{\textbf{editor}} far as possible, I have
implemented this field as \textsf{biblatex}'s standard styles do, but
the requirements specified by the \emph{Manual} present certain
complications that need explaining.  Lehman points out in his
documentation that the \textsf{editor} field will be associated with a
\textsf{title}, a \textsf{booktitle}, or a \textsf{maintitle},
depending on the sort of entry.  More specifically,
\textsf{biblatex-chicago} associates the \textsf{editor} with the most
comprehensive of those titles, that is, \textsf{maintitle} if there is
one, otherwise \textsf{booktitle}, otherwise \textsf{title}, if the
other two are lacking.  In a large number of cases, this is exactly
the correct behavior (adorno:benj, centinel:letters,
plato:republic:gr, among others).  Predictably, however, there are
numerous cases that require, for example, an additional editor for one
part of a collection or for one volume of a multi-volume work.  For
these cases I have provided the \textsf{namea} field.  You should
format names for this field as you would for \textsf{author} or
\textsf{editor}, and these names will always be associated with the
\textsf{title} (donne:var:15).

\mylittlespace As you will see below, I have also provided a
\textsf{nameb} field, which holds the translator of a given
\textsf{title} (euripides:orestes).  If \textsf{namea} and
\textsf{nameb} are the same, \textsf{biblatex-chicago} will
concatenate them, just as \textsf{biblatex} already does for
\textsf{editor}, \textsf{translator}, and \textsf{namec} (i.e., the
compiler).  Furthermore, it is conceivable that a given entry will
need separate editors for each of the three sorts of title.  For this,
and for various other tricky situations, there is the \cmd{partedit}
macro (and its siblings), designed to be used in a \textsf{note}
field, in one of the \textsf{titleaddon} fields, or even in a
\textsf{number} field (howell:marriage).  (Because the strings
identifying an editor differ in notes and bibliography, one can't
simply write them out in such a field when using the notes \&\
bibliography style, but you can certainly do so in the author-date
style, if you wish.  Using the macros will make your .bib file more
portable across both Chicago specifications, and also across multiple
languages, but they are otherwise unnecessary.
Cf. section~\ref{sec:international}, and also \textsf{namea},
\textsf{nameb}, \textsf{namec}, and \textsf{translator}.)

\mybigspace The \mymarginpar{\textbf{editora\\editorb\\editorc}} newer
releases of \textsf{biblatex} provide these fields as a means to
specify additional contributors to texts in a number of editorial
roles.  In the Chicago styles they seem most relevant for the
audiovisual types, especially \textsf{music} and \textsf{video}, where
they help to identify conductors, directors, producers, and
performers.  To specify the role, use the fields \textsf{editoratype},
\textsf{editorbtype}, and \textsf{editorctype}, which see.  (Cf.\
bernstein:shostakovich, handel:messiah:15.)

\mybigspace Normally, \mymarginpar{\textbf{editortype}} with the
exception of the \textsf{article} type,
\textsf{biblatex-chicago-authordate} will automatically find a name to
put at the head of an entry, starting with an \textsf{author}, and
proceeding in order through \textsf{editor}, \textsf{translator}, and
\textsf{namec} (the compiler).  If all four are missing, then the
\textsf{title} will be placed at the head.  (In \textsf{article}
entries with a \texttt{magazine} \textsf{entrysubtype}, a missing
author immediately prompts the use of \textsf{journaltitle} at the
head of an entry.  See above under \textsf{article} for details.)  The
\textsf{editortype} field provides even greater flexibility, allowing
you to choose from a variety of editorial roles while only using the
\textsf{editor} field.  You can do this even though an author is named
(eliot:pound shows this mechanism in action for a standard editor,
rather than a compiler).  Two things are necessary for this to happen.
First, in the \textsf{options} field you need to set
\texttt{useauthor=false} (if there is an \textsf{author)}, then you
need to put the name you wish to see at the head of your entry into
the \textsf{editor} or the \textsf{namea} field.  If the
\enquote{editor} is in fact a compiler, then you need to put
\texttt{compiler} into the \textsf{editortype} field, and
\textsf{biblatex} will print the correct string after the name in the
list of references.

\mylittlespace There are a few details of which you need to be aware.
Because \textsf{biblatex-chicago} has added the \textsf{namea} field,
which gives you the ability to identify the editor specifically of a
\textsf{title} as opposed to a \textsf{maintitle} or a
\textsf{booktitle}, the \textsf{editortype} mechanism checks first to
see whether a \textsf{namea} is defined.  If it is, that name will be
used at the head of the entry, if it isn't it will go ahead and look
for an \textsf{editor}.  \textsf{Biblatex}'s sorting algorithms, and
also its \textsf{labelname} mechanism, should both work properly no
matter sort of name you provide, thanks to \textsf{Biber} and the
(default) Chicago-specific definitions of \cmd{DeclareLabelname} and
\cmd{DeclareSortingScheme}.  (Cf.\ section~\ref{sec:authformopts},
below).  If, however, the \textsf{namea} field provides the name, and
that name isn't automatically shortened properly by \textsf{biblatex},
then your .bib entry will need to have a \textsf{shorteditor} defined
to help with in-text citations, not a \textsf{shortauthor}, possibly
ruled out because \texttt{useauthor=false}.

\mylittlespace In \textsf{biblatex} 0.9 Lehman has reworked the string
concatenation mechanism, for reasons he outlines in his RELEASE file,
and I have followed his lead.  In short, if you define the
\textsf{editortype} field, then concatenation is turned off, even if
the name of the \textsf{editor} matches, for example, that of the
\textsf{translator}.  In the absence of an \textsf{editortype}, the
usual mechanisms remain in place, that is, if the \textsf{editor}
exactly matches a \textsf{translator} and/or a \textsf{namec}, or
alternatively if \textsf{namea} exactly matches a \textsf{nameb}
and/or a \textsf{namec}, then \textsf{biblatex} will print the
appropriate strings.  The \emph{Manual} specifically (17.41)
recommends not using these identifying strings in citations, and
\textsf{biblatex-chicago-authordate} follows that recommendation.  If
you nevertheless need to provide such a string, you'll have to do it
manually in the \textsf{shorteditor} field, or perhaps, in a different
sort of entry, in a \textsf{shortauthor} field.

\mylittlespace It may also be worth noting that because of certain
requirements in the specification -- absence of an \textsf{author},
for example -- the \texttt{useauthor} mechanism won't work properly in
the following entry types: \textsf{collection}, \textsf{letter},
\textsf{patent}, \textsf{periodical}, \textsf{proceedings},
\textsf{suppbook}, \textsf{suppcollection}, and
\textsf{suppperiodical}.

\mybigspace These
\mymarginpar{\textbf{editoratype\\editorbtype\\editorctype}} fields
identify the exact role of the person named in the corresponding
\textsf{editor[a-c]} field.  Note that they are not part of the string
concatenation mechanism.  I have implemented them just as the standard
styles do, and they have now found a use particularly in
\textsf{music} and \textsf{video} entries.  Cf.\
bernstein:shostakovich, handel:messiah:15.

%\enlargethispage{\baselineskip}

\mybigspace Standard \mymarginpar{\textbf{eid}} \textsf{biblatex}
field, providing a string or number some journals use uniquely to
identify a particular article.  Only applicable to the
\textsf{article} entry type.  Not typically required by the
\emph{Manual}.

\mybigspace Standard \mymarginpar{\textbf{entrysubtype}} and very
powerful \textsf{biblatex} field, left undefined by the standard
styles.  In \textsf{biblatex-chicago-authordate} it has five very
specific uses, the first three of which I have designed in order to
maintain, as much as possible, backward compatibility with the
standard styles.  First, in \textsf{article} and \textsf{periodical}
entries, the field allows you to differentiate between scholarly
\enquote{journals,} on the one hand, and \enquote{magazines} and
\enquote{newspapers} on the other.  Usage is fairly simple: you need
to put the exact string \texttt{magazine} into the
\textsf{entrysubtype} field if you are citing one of the latter two
types of source, whereas if your source is a \enquote{journal,} then
you need do nothing.

\mylittlespace The second use involves references to works from
classical antiquity and, according to the \emph{Manual}, from the
Middle Ages, as well.  When you cite such a work using the traditional
divisions into books, sections, lines, etc., divisions which are
presumed to be the same across all editions, then you need to put the
exact string \texttt{classical} into the \textsf{entrysubtype} field.
This has no effect in the list of references, which will still present
the particular edition you are using, but it does affect the
formatting of in-text citations, in two ways.  First, it suppresses
some of the punctuation.  Second, and more importantly, it suppresses
the \textsf{date} field in favor of the \textsf{title}, so that
citations look like (Aristotle \emph{Metaphysics} 3.2.996b5--8)
instead of (Aristotle 1997, 3.2.996b5--8).  This mechanism may also
prove useful in \textsf{misc} entries for citations from the Bible or
other sacred texts (cf.\ genesis), and for citing archival collections
(house:papers), where it produces citations of the form (House
Papers).  (Cf.\ the next but one paragraph.)

\mylittlespace If you wish to reference a classical or medieval work
by the page numbers of a particular, non-standard edition, then you
shouldn't use the \texttt{classical} \textsf{entrysubtype} toggle.
Also, and the specification isn't entirely clear about this, works
from the Renaissance and later, even if cited by the traditional
divisions, seem to have citations formatted normally, and therefore
don't need an \textsf{entrysubtype} field.  (See \emph{Manual}
17.246--262; aristotle:metaphy:gr, plato:republic:gr;
euripides:orestes is an example of a translation cited by page number
in a modern edition.)

\mylittlespace The third use of the \textsf{entrysubtype} field occurs
in \textsf{misc} entries.  If such an entry contains no such field,
then the citation will be treated just as the standard
\textsf{biblatex} styles would, including the use of italics for the
\textsf{title}.  Any string at all in \textsf{entrysubtype} tells
\textsf{biblatex-chicago} to treat the source as part of an
unpublished archive.  Please see section~\ref{sec:types:authdate}
above under \textbf{misc} for all the details on how these citations
work.

\mylittlespace Fourth, the field can be defined in the
\textsf{artwork} entry type in order to refer to a work from antiquity
whose title you do not wish to be italicized.  Please see the
documentation of \textsf{artwork} above for the details.  Fifth, and
finally, you can use the exact string \texttt{tv} to identify
televisual material as a subset of the \textsf{video} entry type.
This will only affect the entry in the list of references if you use
\texttt{reprint} in the \textsf{pubstate} field in order to print
the date a program was originally shown in parentheses at the end of
such an entry.  (It's a niche usage but it at least maintains
consistency for the \texttt{reprint} mechanism.  Cf.\
\textsf{pubstate}, below.)

\mybigspace Kazuo
\colmarginpar{\textbf{eprint}\\\textbf{eprintclass}\\\textbf{eprinttype}}
Teramoto suggested adding \textsf{biblatex's} excellent
\textsf{eprint} handling to \textsf{biblatex-chicago}, and he sent me
a patch implementing it.  With minor alterations, I have applied it to
this release, so these three fields now work more or less as they do
in standard \textsf{biblatex}.  They may prove helpful in providing
more abbreviated references to online content than conventional URLs,
though I can find no specific reference to them in the \emph{Manual}.

\mybigspace This \colmarginpar{\textbf{eventdate}} is a standard
\textsf{biblatex} field, added recently to the \textbf{music} entry
type in case users need it to identify a particular recording session
or concert.  See the documentation of that type above.  In the default
configuration of \cmd{DeclareLabel\-year}, an entry missing a
\textsf{date} will use the \textsf{eventdate} to find a year for the
citation and list of references, though the rest of the field would in
such a case be ignored in any entry other than \textsf{music}.

\mybigspace As \mymarginpar{\textbf{foreword}} with the
\textsf{afterword} field above, \textsf{foreword} will in general
function as it does in standard \textsf{biblatex}.  Like
\textsf{afterword} (and \textsf{introduction}), however, it has a
special meaning in a \textsf{suppbook} entry, where you simply need to
define it somehow (and leave \textsf{afterword} and
\textsf{introduction} undefined) to make a foreword the focus of a
citation.

\mybigspace A \mymarginpar{\textbf{holder}} standard \textsf{biblatex}
field for identifying a \textsf{patent}'s holder(s), if they differ
from the \textsf{author}.  The \emph{Manual} has nothing to say on the
subject, but \textsf{biblatex-chicago} prints it (them), in
parentheses, just after the author(s).

%\enlargethispage{\baselineskip}

\mybigspace Standard \mymarginpar{\textbf{howpublished}}
\textsf{biblatex} field, mainly applicable in the \textsf{booklet}
entry type, where it replaces the \textsf{publisher}.  I have also
retained it in the \textsf{misc} and \textsf{unpublished} entry types,
for historical reasons, and either it or \textsf{pubstate} can be used
in \textsf{music} entries to clarify publication details.

\mybigspace Standard \mymarginpar{\textbf{institution}}
\textsf{biblatex} field.  In the \textsf{thesis} entry type, it will
usually identify the university for which the thesis was written,
while in a \textsf{report} entry it may identify any sort of
institution issuing the report.

\mybigspace As \mymarginpar{\textbf{introduction}} with the
\textsf{afterword} and \textsf{foreword} fields above,
\textsf{introduction} will in general function as it does in standard
\textsf{biblatex}.  Like those fields, however, it has a special
meaning in a \textsf{suppbook} entry, where you simply need to define
it somehow (and leave \textsf{afterword} and \textsf{foreword}
undefined) to make an introduction the focus of a citation.

\mybigspace Standard \mymarginpar{\textbf{isbn}} \textsf{biblatex}
field, for providing the International Standard Book Number of a
publication.  Not typically required by the \emph{Manual}.

\mybigspace Standard \mymarginpar{\textbf{isrn}} \textsf{biblatex}
field, for providing the International Standard Technical Report
Number of a report.  Only relevant to the \textsf{report} entry type,
and not typically required by the \emph{Manual}.

\mybigspace Standard \mymarginpar{\textbf{issn}} \textsf{biblatex}
field, for providing the International Standard Serial Number of a
periodical in an \textsf{article} or a \textsf{periodical} entry.  Not
typically required by the \emph{Manual}.

\mybigspace Standard \mymarginpar{\textbf{issue}} \textsf{biblatex}
field, designed for \textsf{article} or \textsf{periodical} entries
identified by something like \enquote{Spring} or \enquote{Summer}
rather than by the usual \textsf{month} or \textsf{number} fields
(brown:bremer).

\mybigspace The \mymarginpar{\textbf{issuesubtitle}} subtitle for an
\textsf{issuetitle} --- see next entry.

\mybigspace Standard \mymarginpar{\textbf{issuetitle}}
\textsf{biblatex} field, intended to contain the title of a special
issue of any sort of periodical.  If the reference is to one article
within the special issue, then this field should be used in an
\textsf{article} entry (conley:fifthgrade), whereas if you are citing
the entire issue as a whole, then it would go in a \textsf{periodical}
entry, instead (good:wholeissue).  The \textsf{note} field is the
proper place to identify the type of issue, e.g.,\ \texttt{special
  issue}, with the initial letter lower-cased to enable automatic
contextual capitalization.

\mybigspace The \mymarginpar{\textbf{journalsubtitle}} subtitle for a
\textsf{journaltitle} --- see next entry.

\mybigspace Standard \mymarginpar{\textbf{journaltitle}}
\textsf{biblatex} field, replacing the standard \textsc{Bib}\TeX\
field \textsf{journal}, which, however, still works as an alias.  It
contains the name of any sort of periodical publication, and is found
in the \textsf{article} entry type.  In the case where a piece in an
\textsf{article} (\textsf{entrysubtype} \texttt{magazine}) doesn't
have an author, \textsf{biblatex-chicago} provides for this field to
be used as the author.  See above (section~\ref{sec:fields:authdate})
under \textbf{article} for details.  The lakeforester:pushcarts and
nyt:trevorobit entries in \textsf{dates-test.bib} will give you some
idea of how this works.

\mybigspace This \mymarginpar{\textbf{keywords}} field is
\textsf{biblatex}'s extremely powerful and flexible technique for
filtering entries in a list of references, allowing you to subdivide
it according to just about any criteria you care to invent.  See
\textsf{biblatex.pdf} (3.10.4) for thorough documentation.  In
\textsf{biblatex-chicago}, the field provides one convenient means to
exclude certain entries from making their way into a list of
references, though the toggle \texttt{skipbib} in the \textsf{options}
field works just as well, and perhaps more simply.  There are a few
reasons for so excluding entries.  When citing both an original text
and its translation (see \textbf{userf}, below), the \emph{Manual}
(17.66) suggests including the original at the end of the
translation's reference list entry, a procedure which requires that
the original not also be printed as a separate entry
(furet:passing:eng, furet:passing:fr, aristotle:metaphy:trans,
aristotle:metaphy:gr).  Well-known reference works (like the
\emph{Encyclopaedia Britannica}, for example) and many sacred texts
need only be presented in citations, and not in the list of references
(17.238--239; ency:britannica, genesis, wikiped:bibtex; see
\textsf{inreference} and \textsf{misc}, above).

\mybigspace A \mymarginpar{\textbf{language}} standard
\textsf{biblatex} field, designed to allow you to specify the
language(s) in which a work is written.  As a general rule, the
Chicago style doesn't require you to provide this information, though
it may well be useful for clarifying the nature of certain works, such
as bilingual editions, for example.  There is at least one situation,
however, when the \emph{Manual} does specify this data, and that is
when the title of a work is given in translation, even though no
translation of the work has been published, something that might
happen when a title is in a language deemed to be unparseable by a
majority of your expected readership (17.65--67, 17.166, 17.177;
chu:panda, pirumova, rozner:liberation).  In such a case, you should
provide the language(s) involved using this field, connecting multiple
languages using the keyword \texttt{and}.  (I have retained
\textsf{biblatex's} \cmd{bibstring} mechanism here, which means that
you can use the standard bibstrings or, if one doesn't exist for the
language you need, just give the name of the language, capitalized as
it should appear in your text.  You can also mix these two modes
inside one entry without apparent harm.)

\mylittlespace An alternative arrangement suggested by the
\emph{Manual} is to retain the original title of a piece but then to
provide its translation, as well.  If you choose this option, you'll
need to make use of the \textbf{usere} field, on which see below.  In
effect, you'll probably only ever need to use one of these two fields
in any given entry, and in fact \textsf{biblatex-chicago} will only
print one of them if both are present, preferring \textsf{usere} over
\textsf{language} for this purpose (see kern, pirumova:russian, and
weresz).  Note also that both of these fields are universally
associated with the \textsf{title} of a work, rather than with a
\textsf{booktitle} or a \textsf{maintitle}.  If you need to attach a
language or a translation to either of the latter two, you could
probably manage it with special formatting inside those fields
themselves.

%\enlargethispage{\baselineskip}

\mybigspace I \mymarginpar{\textbf{lista}} intend this field
specifically for presenting citations from reference works that are
arranged alphabetically, where the name of the article rather than a
page or volume number should be given.  The field is a
\textsf{biblatex} list, which means you should separate multiple items
with the keyword \texttt{and}.  Each item receives its own set of
quotation marks, and the whole list will be prefixed by the
appropriate string (\enquote{s.v.,} \emph{sub verbo}, pl.\
\enquote{s.vv.}).  \textsf{Biblatex-chicago} will only print such a
field in a \textsf{book} or an \textsf{inreference} entry, and you
should look at the documentation of these entry types for further
details.  (See \emph{Manual} 17.238--239; grove:sibelius, times:guide,
wikiped:bibtex.)

\mybigspace This \mymarginpar{\textbf{location}} is
\textsf{biblatex}'s version of the usual \textsc{Bib}\TeX\ field
\textsf{address}, though the latter is accepted as an alias if that
simplifies the modification of older .bib files.  According to the
\emph{Manual} (17.99), a citation usually need only provide the first
city listed on any title page, though a list of cities separated by
the keyword \enquote{\texttt{and}} will be formatted appropriately.
If the place of publication is unknown, you can use
\cmd{autocap\{n\}.p.}\ instead (17.102), though in many or even most
cases this isn't strictly necessary (17.32--34;
virginia:plantation:15).  For all cities, you should use the common
English version of the name, if such exists (17.101).

%\enlargethispage{\baselineskip}

\mylittlespace One other use needs explanation here.  In
\textsf{article} and \textsf{periodical} entries, there is usually no
need for a \textsf{location} field, but \enquote{if a journal might be
  confused with another with a similar title, or if it might not be
  known to the users of a bibliography,} then this field can present
the place or institution where it is published (17.174, 17.196;
garrett:15, kimluu:diethyl:15, and lakeforester:pushcarts).

\mybigspace The \mymarginpar{\textbf{mainsubtitle}} subtitle for a
\textsf{maintitle} --- see next entry.

\mybigspace The \mymarginpar{\textbf{maintitle}} main title for a
multi-volume work, e.g., \enquote{Opera} or \enquote{Collected Works.}
It takes sentence-style capitalization in author-date.  (See
donne:var:15, euripides:\-orestes, harley:cartography, lach:asia,
pelikan:chris\-tian, and plato:republic:gr.)

\mybigspace An \mymarginpar{\textbf{maintitleaddon}} annex to the
\textsf{maintitle}, for which see previous entry.  Such an annex would
be printed in the main text font.  If your data begins with a word
that would ordinarily only be capitalized at the beginning of a
sentence, then simply ensure that that word is in lowercase, and
\textsf{biblatex-chicago} will automatically do the right thing.

\mybigspace Standard \mymarginpar{\textbf{month}} \textsf{biblatex}
field, containing the month of publication.  This should be an
integer, i.e., \texttt{month=\{3\}} not \texttt{month=\{March\}}.  See
\textsf{date} for more information.

\mybigspace This \mymarginpar{\textbf{namea}} is one of the fields
\textsf{biblatex} provides for style writers to use, but which it
leaves undefined itself.  In \textsf{biblatex-chicago} it contains the
name(s) of the editor(s) of a \textsf{title}, if the entry has a
\textsf{booktitle} or \textsf{maintitle}, or both, in which situation
the \textsf{editor} would be associated with one of these latter
fields (donne:var:15).  You should present names in this field exactly
as you would those in an \textsf{author} or \textsf{editor} field, and
the package will concatenate this field with \textsf{nameb} if they
are identical.  See under \textbf{editor} above for the full details.
Cf.\ also \textsf{nameb}, \textsf{namec}, \textsf{translator}, and the
macros \cmd{partedit}, \cmd{parttrans}, \cmd{parteditandtrans},
\cmd{partcomp}, \cmd{parteditandcomp}, \cmd{parttransandcomp}, and
\cmd{partedittransand\-comp}, for which see
section~\ref{sec:formatting:authdate}.

\mybigspace This \mymarginpar{\textbf{nameaddon}} field is provided by
\textsf{biblatex}, though not used by the standard styles.  In
\textsf{biblatex-chicago}, it allows you to specify that an author's
name is a pseudo\-nym, or to provide either the real name or the
pseudonym itself, if the other is being provided in the
\textsf{author} field.  The abbreviation
\enquote{\texttt{pseud.}\hspace{-2pt}}\ (always lowercase in English)
is specified, either on its own or after the pseudonym
(centinel:letters, creasey:ashe:blast, creasey:morton:hide,
creasey:york:death, and le\-carre:quest); \cmd{bibstring\{pseudonym\}}
does the work for you.  See under \textbf{author} above for the full
details.

\mylittlespace In the \textsf{customc} entry type, on the other hand,
which is used to create alphabetized expansions of
\textsf{shorthands}, the \textsf{nameaddon} field allows you to change
the default string linking the two parts of the expansion.  The code
automatically tests for a known bibstring, which it will italicize.
Otherwise, it prints the string as is.

\mybigspace Like \mymarginpar{\textbf{nameb}} \textsf{namea}, above,
this is a field left undefined by the standard \textsf{biblatex}
styles.  In \textsf{biblatex-chicago}, it contains the name(s) of the
translator(s) of a \textsf{title}, if the entry has a
\textsf{booktitle} or \textsf{maintitle}, or both, in which situation
the \textsf{translator} would be associated with one of these latter
fields (euripides:orestes).  You should present names in this field
exactly as you would those in an \textsf{author} or
\textsf{translator} field, and the package will concatenate this field
with \textsf{namea} if they are identical.  See under the
\textbf{translator} field below for the full details.  Cf.\ also
\textsf{namea}, \textsf{namec}, \textsf{origlanguage},
\textsf{translator}, \textsf{userf} and the macros \cmd{partedit},
\cmd{parttrans}, \cmd{parteditandtrans}, \cmd{partcomp},
\cmd{parteditandcomp}, \cmd{parttransandcomp}, and
\cmd{partedittransandcomp} in section~\ref{sec:formatting:authdate}.

\mybigspace The \mymarginpar{\textbf{namec}} \emph{Manual} (17.41)
specifies that works without an author may be listed under an editor,
translator, or compiler, assuming that one is available, and it also
specifies the strings to be used with the name(s) of compiler(s).  All
this suggests that the \emph{Manual} considers this to be standard
information that should be made available in a bibliographic
reference, so I have added that possibility to the many that
\textsf{biblatex} already provides, such as the \textsf{editor},
\textsf{translator}, \textsf{commentator}, \textsf{annotator}, and
\textsf{redactor}, along with writers of an \textsf{introduction},
\textsf{foreword}, or \textsf{afterword}.  Since \textsf{biblatex.bst}
doesn't offer a \textsf{compiler} field, I have adopted for this
purpose the otherwise unused field \textsf{namec}.  It is important to
understand that, despite the analogous name, this field does not
function like \textsf{namea} or \textsf{nameb}, but rather like
\textsf{editor} or \textsf{translator}, and therefore if used will be
associated with whichever title field these latter two would be were
they present in the same entry.  Identical fields among these three
will be concatenated by the package, and concatenated too with the
(usually) unnecessary commentator, annotator and the rest.  Also
please note that I've arranged the concatenation algorithms to include
\textsf{namec} in the same test as \textsf{namea} and \textsf{nameb},
so in this particular circumstance you can, if needed, make
\textsf{namec} analogous to these two latter, \textsf{title}-only
fields.  (See above under \textbf{editortype} for details of how you
can use that field to identify a compiler.)

\mylittlespace It might conceivably be necessary at some point to
identify the compiler(s) of a \textsf{title} separate from the
compiler(s) of a \textsf{booktitle} or \textsf{maintitle}, but for the
moment I've run out of available \textsf{name} fields, so you'll have
to fall back on the \cmd{partcomp} macro or the related
\cmd{parteditandcomp}, \cmd{parttransandcomp}, and
\cmd{partedittransandcomp}, on which see Commands
(section~\ref{sec:formatting:authdate}) below.  (Future releases may
be able to remedy this.)  It may be as well to mention here too that
of the three names that can be substituted for the missing
\textsf{author} at the head of an entry, \textsf{biblatex-chicago}
will choose an \textsf{editor} if present, then a \textsf{translator}
if present, falling back to \textsf{namec} only in the absence of the
other two, and assuming that the fields aren't identical, and
therefore to be concatenated.  In a change from the previous behavior,
these algorithms also now test for \textsf{namea} or \textsf{nameb},
which will be used instead of \textsf{editor} and \textsf{translator},
respectively, giving the package the greatest likelihood of finding a
name to place at the head of an entry.  \textsf{Biblatex}'s sorting
algorithms, and also its \textsf{labelname} mechanism, should both
work properly no matter sort of name you provide, thanks to
\textsf{Biber} and the (default) Chicago-specific definitions of
\cmd{DeclareLabelname} and \cmd{DeclareSortingScheme}.  (Cf.\
section~\ref{sec:authformopts}, below).

\mybigspace As \mymarginpar{\textbf{note}} in standard
\textsf{biblatex}, this field allows you to provide bibliographic data
that doesn't easily fit into any other field.  In this sense, it's
very like \textsf{addendum}, but the information provided here will be
printed just before the publication data.  (See chaucer:alt,
cook:sotweed, emerson:nature, and rodman:walk for examples of this
usage in action.)  It also has a specialized use in the periodical
types (\textsf{article} and \textsf{periodical}), where it holds
supplemental information about a \textsf{journaltitle}, such as
\enquote{special issue} (conley:fifthgrade, good:wholeissue).  In all
uses, if your data begins with a word that would ordinarily only be
capitalized at the beginning of a sentence, then simply ensure that
that word is in lowercase, and \textsf{biblatex-chicago} will
automatically do the right thing.  Cf.\ \textsf{addendum}.

\mybigspace This \mymarginpar{\textbf{number}} is a standard
\textsf{biblatex} field, containing the number of a
\textsf{journaltitle} in an \textsf{article} entry, the number of a
\textsf{title} in a \textsf{periodical} entry, or the volume/number of
a book in a \textsf{series}.  Generally, in an \textsf{article} or
\textsf{periodical} entry, this will be a plain cardinal number, but
in such entries \textsf{biblatex-chicago} now does the right thing if
you have a list or range of numbers (unsigned:ranke).  In any
\textsf{book}-like entry it may well contain considerably more
information, including even a reference to \enquote{2nd ser.,} for
example, while the \textsf{series} field in such an entry will contain
the name of the series, rather than a number.  This field is also the
place for the patent number in a \textsf{patent} entry.  Cf.\
\textsf{issue} and \textsf{series}.  (See \emph{Manual} 17.90--95 and
boxer:china, palmatary:pottery, wauchope:ceramics; 17.163 and
beattie:crime, conley:fifthgrade, friedman:learning, garrett:15,
gibbard:15, hlatky:hrt, mcmillen:antebellum, rozner:liberation,
warr:ellison.)

\mylittlespace \textbf{NB}: This may be an opportune place to point
out that the \emph{Manual} (17.129) prefers arabic to roman numerals
in most circumstances (chapters, volumes, series numbers, etc.), even
when such numbers might be roman in the work cited.  The obvious
exception is page numbers, in which roman numerals indicate that the
citation came from the front matter, and should therefore be retained.
Another possible exception is in references to works \enquote{with
  many and complex divisions,} in which \enquote{a mixture of roman
  and arabic} may be \enquote{easier to disentangle.}

\mybigspace A \mymarginpar{\textbf{options}} standard
\textsf{biblatex} field, for setting certain options on a per-entry
basis rather than globally.  Information about some of the more common
options may be found above under \textsf{author} and \textsf{date},
and below in section~\ref{sec:authuseropts}.  See creel:house,
eliot:pound, emerson:nature, ency:britannica, herwign:office,
lecarre:quest, and maitland:canon for examples of the field in use.

% \enlargethispage{\baselineskip}

\mybigspace A \mymarginpar{\textbf{organization}} standard
\textsf{biblatex} field, retained mainly for use in the \textsf{misc},
\textsf{online}, and \textsf{manual} entry types, where it may be of
use to specify a publishing body that might not easily fit in other
categories.  In \textsf{biblatex}, it is also used to identify the
organization sponsoring a conference in a \textsf{proceedings} or
\textsf{inproceedings} entry, and I have retained this as a
possibility, though the \emph{Manual} is silent on the matter.

\mybigspace This \colmarginpar{\textbf{origdate}} is a standard
\textsf{biblatex} field which replaced the obsolete \textsf{origyear},
and which therefore allows more than one full date specification for
those references which need to provide more than just one.  As with
the analogous \textsf{date} field, you provide the date (or range of
dates) in \textsc{iso}8601 format, i.e., \texttt{yyyy-mm-dd}.  In most
entry types, you would use \textsf{origdate} to provide the date of
first publication of a work, most usually needed only in the case of
reprint editions, but also recommended by the \emph{Manual} for
electronic editions of older works (17.123, 17.146--7;
aristotle:metaphy:gr, emerson:nature, james:ambassadors,
schweitzer:bach).  In both the \textsf{letter} and \textsf{misc} (with
\textsf{entrysubtype)} entry types, the \textsf{origdate} identifies
when a letter (or similar) was written.  In such \textsf{misc}
entries, some \enquote{non-letter-like} materials (like interviews)
need the \textsf{date} field for this purpose, while in
\textsf{letter} entries the \textsf{date} applies to the publication
of the whole collection.  If such a published collection were itself a
reprint, judicious use of the \textsf{pubstate} field or perhaps
improvisation in the \textsf{location} field might be able to rescue
the situation.  (See white:ross:memo, white:russ, and white:total for
how \textsf{letter} entries can work; creel:house shows the field in
action in a \textsf{misc} entry, while spock:interview uses
\textsf{date} instead.)

\mylittlespace Because of the importance of date specifications in the
author-date style, \textsf{biblatex-chicago-authordate} provides a
series of options and automated behaviors to allow you to emphasize
the \textsf{origdate} in citations and at the head of entries in the
list of references.  In entries which have \emph{only} an
\textsf{origdate} --- usually \textsf{misc} with an
\textsf{entrysubtype} --- \textsf{Biber} and the default
\cmd{DeclareLabelyear} configuration now make it possible to do
without a \texttt{cmsdate} option, as the \textsf{origdate} will
automatically appear where and as it should.  In entries with both
dates you have a choice of which appears where.  In some cases it may
even be necessary to reverse the two date fields, putting the earlier
year in \textsf{date} and the later in \textsf{origdate}.  Please see
above under \textbf{date} for all the details on how these options
interact.

\mylittlespace Because the \textsf{origdate} field only accepts
numbers, some improvisation may be needed if you wish to include
\enquote{n.d.}\ (\cmd{bibstring\{nodate\}}) in an entry.  In
\textsf{letter} and \textsf{misc}, this information can be placed in
\textsf{titleaddon}, but in other entry types you may need to use the
\textsf{location} field.

%\enlargethispage{\baselineskip}

\mybigspace In \mymarginpar{\textbf{origlanguage}} keeping with the
\emph{Manual}'s specifications, I have fairly thoroughly redefined
\textsf{biblatex}'s facilities for treating translations.  The
\textsf{origtitle} field isn't used, while the \textsf{language} and
\textsf{origdate} fields have been press-ganged for other duties.  The
\textsf{origlanguage} field, for its part, retains a dual role in
presenting translations in a list of references.  The details of the
\emph{Manual}'s suggested treatment when both a translation and an
original are cited may be found below under \textbf{userf}.  Here,
however, I simply note that the introductory string used to connect
the translation's citation with the original's is \enquote{Originally
  published as,} which I suggest may well be inaccurate in a great
many cases, as for instance when citing a work from classical
antiquity, which will most certainly not \enquote{originally} have
been published in the Loeb Classical Library.  Although not, strictly
speaking, authorized by the \emph{Manual}, I have provided another way
to introduce the original text, using the \textsf{origlanguage} field,
which must be provided \emph{in the entry for the translation, not the
  original text} (aristotle:metaphy:trans).  If you put one of the
standard \textsf{biblatex} bibstrings there (enumerated below), then
the entry will work properly across multiple languages.  Otherwise,
just put the name of the language there, localized as necessary, and
\textsf{biblatex-chicago} will eschew \enquote{Originally published
  as} in favor of, e.g., \enquote{Greek edition:} or \enquote{French
  edition:}.  This has no effect in citations, where only the work
cited --- original or translation --- will be printed, but it may help
to make the \emph{Manual}'s suggestions for the list of references
more palatable.

\mylittlespace That was the first usage, in keeping at least with the
spirit of the \emph{Manual}.  I have also, perhaps less in keeping
with that specification, retained some of \textsf{biblatex}'s
functionality for this field.  If an entry doesn't have a
\textsf{userf} field, and therefore won't be combining a text and its
translation in the list of references, you can also use
\textsf{origlanguage} as Lehman intended it, so that instead of
saying, e.g., \enquote{translated by X,} the entry will read
\enquote{translated from the German by X.}  The \emph{Manual} doesn't
mention this, but it may conceivably help avoid certain ambiguities in
some citations.  As in \textsf{biblatex}, if you wish to use this
functionality, you have to provide \emph{not} the name of the
language, but rather a bibstring, which may, at the time of writing,
be one of \texttt{american}, \texttt{brazilian}, \texttt{danish},
\texttt{dutch}, \texttt{english}, \texttt{french}, \texttt{german},
\texttt{greek}, \texttt{italian}, \texttt{latin}, \texttt{norwegian},
\texttt{portuguese}, \texttt{spanish}, or \texttt{swedish}, to which
I've added \texttt{russian}.

%\enlargethispage{\baselineskip}

\mybigspace At \colmarginpar{\textbf{origlocation}} least one notes
\&\ bibliography example in the \emph{Manual} provides a more complete
specification of a reprinted book's original publication details than
has been possible using previous releases of \textsf{biblatex-chicago}
(17.123).  It seems reasonable to include this in the author-date
style, as well, so starting with this release, you can provide both an
\textsf{origlocation} and an \textsf{origpublisher} to go along with
the \textsf{origdate}, should you so wish, and all of this information
will be printed in the reference list.  You can now also use this
field in a \textsf{letter} or \textsf{misc} (with
\textsf{entrysubtype}) entry to give the place where a published or
unpublished letter was written (17.76).  (Jonathan Robinson has
suggested that the \textsf{origlocation} may in some circumstances
actually be necessary for disambiguation, his example being early
printed editions of the same material printed in the same year but in
different cities.  The new functionality should make this simple to
achieve.  Cf.\ \textsf{origdate}, \textsf{origpublisher} and
\textsf{pubstate}; schweitzer:bach.)

\mybigspace As \colmarginpar{\textbf{origpublisher}} with the
\textsf{origlocation} field just above, this new field allows you to
provide fuller original publication details for reprinted books.  You
can now provide an \textsf{origpublisher} and/or an
\textsf{origlocation} in addition to the \textsf{origdate}, and all
will be presented in long notes and bibliography.  (Cf.\
\textsf{origdate}, \textsf{origlocation}, and \textsf{pubstate};
schweitzer:bach.)

\mybigspace This \mymarginpar{\textbf{origyear}} field is, as of
\textsf{biblatex} 0.9, obsolete.  It is ignored if it appears in a
.bib file.  Please use \textsf{origdate} instead.

\mybigspace This \mymarginpar{\textbf{pages}} is the standard
\textsf{biblatex} field for providing page references.  In many
\textsf{article} entries you'll find this contains something other
than a page number, e.g. a section name or edition specification
(17.188, 17.191, 17.202; kozinn:review, nyt:trevorobit).  Of course,
the same may be true of almost any sort of entry, though perhaps with
less frequency.  Curious readers may wish to look at brown:bremer
(17.172) for an example of a \textsf{pages} field used to facilitate
reference to a two-part journal article.  Cf.\ \textsf{number} for
more information on the \emph{Manual}'s preferences regarding the
formatting of numerals; \textsf{bookpagination} and
\textsf{pagination} provide details about \textsf{biblatex's}
mechanisms for specifying what sort of division a given \textsf{pages}
field contains; and \textsf{usera} discusses a different way to
present the section information pertaining to a newspaper article.

\mybigspace This, \mymarginpar{\textbf{pagination}} a standard
\textsf{biblatex} field, allows you automatically to prefix the
appropriate identifying string to information you provide in the
\textsf{postnote} field of a citation command, whereas
\textsf{bookpagination} allows you to prefix a string to the
\textsf{pages} field.  Please see \textbf{bookpagination} above for
all the details on this functionality, as aside from the difference
just mentioned the two fields are equivalent.

\mybigspace Standard \mymarginpar{\textbf{part}} \textsf{biblatex}
field, which identifies physical parts of a single logical volume in
\textsf{book}-like entries, not in periodicals.  It has the same
purpose in \textsf{biblatex-chicago}, but because the \emph{Manual}
(17.88) calls such a thing a \enquote{book} and not a \enquote{part,}
the string printed in the list of references will, at least in
English, be \enquote{\texttt{bk.}\hspace{-2pt}}\ instead of the plain
dot between volume number and part number (harley:cartography,
lach:asia).  This field should only be used in association with a
\textsf{volume} number, so if you need to identify \enquote{parts} or
\enquote{books} that are part of a published \textsf{series}, for
example, then you'll need to use a different field, (which in the case
of a series would be \textsf{number} [palmatary:pottery]).  Cf.\
\textsf{volume}; iso:electrodoc:15.

\mybigspace Standard \mymarginpar{\textbf{publisher}}
\textsf{biblatex} field.  Remember that \enquote{\texttt{and}} is a
keyword for connecting multiple publishers, so if a publisher's name
contains \enquote{and,} then you should either use the ampersand (\&)
or enclose the whole name in additional braces.  (See \emph{Manual}
17.103--114; aristotle:metaphy:gr, cohen:schiff, creasey:ashe:blast,
dunn:revolutions.)

\mylittlespace There are, as one might expect, a couple of further
subtleties involved here.  Ordinarily, two publishers will be
separated by a forward slash in the list of references, but if a
company issues \enquote{certain books through a special publishing
  division or under a special imprint,} then the two names will be
separated by a comma, which you will need to provide in the
\textsf{publisher} field.  The \emph{Manual}'s example (17.112) is
\enquote{\texttt{Ohio University Press, Swallow Press},} which would
cause \textsf{biblatex-chicago} no problems.  If a book has two
co-publishers, \enquote{usually in different countries,} (17.113) then
the simplest thing to do is to choose one, probably the nearest one
geographically.  If you feel it necessary to include both, then
levistrauss:savage demonstrates one way of doing so, using a
combination of the \textsf{publisher} and \textsf{location} fields.
Finally, if the publisher is unknown, then the \emph{Manual}
recommends (17.109) simply using the place (if known) and the date.
If for some reason you need to indicate the absence of a publisher,
the abbreviation given by the \emph{Manual} is \texttt{n.p.}, though
this can also stand for \enquote{no place.}  Some style guides
apparently suggest using \texttt{s.n.}\,(= \emph{sine nomine}) to
specify the lack of a publisher, but the \emph{Manual} doesn't mention
this.

\mybigspace A \mymarginpar{\textbf{pubstate}} standard
\textsf{biblatex} field, new to version 0.9.  Because the author-date
specification has fairly complicated rules about presenting reprinted
editions, I have adopted this field as a means of simplifying the
problem for users.  Instead of hand-formatting in the
\textsf{location} field, you can now simply put the string
\texttt{reprint} into the \textsf{pubstate} field, and depending on
which date you have chosen to appear at the head of the entry,
\textsf{biblatex-chicago-authordate} will either print the (localized)
string \texttt{Repr.}\ in the proper place or otherwise provide a
parenthesized notice at the end of the entry detailing the original
publication date.  See under \textbf{date} above for the available
permutations. (Cf.\ aristotle:metaphy:gr, maitland:canon,
maitland:equity, schweitzer:bach.)  If the field contains something
other than the word \texttt{reprint}, then it will be treated as in
the standard styles, and printed after the publication information.
In \textsf{music} entries, its literal contents will always be printed
as part of the publication data.

\mybigspace I \mymarginpar{\textbf{redactor}} have implemented this
field just as \textsf{biblatex}'s standard styles do, even though the
\emph{Manual} doesn't actually mention it.  It may be useful for some
purposes.  Cf.\ \textsf{annotator} and \textsf{commentator}.

\mybigspace \textbf{NB:} \mymarginpar{\textbf{reprinttitle}}
\textbf{Please note that this feature is in an alpha state, and that
  I'm contemplating using a different field in the future for this
  functionality.  I include it here in the hope that it might receive
  some testing in the meantime.}  At the request of Will Small, I have
included a means of providing the original publication details of an
essay or a chapter that you are citing from a subsequent reprint,
e.g., a \emph{Collected Essays} volume.  In such a case, at least
according to the \emph{Manual} (17.73), such details need be provided
only if they are \enquote{of particular interest.}  The data would
follow an introductory phrase like \enquote{originally published as,}
making the problem strictly parallel to that of including details of a
work in the original language alongside the details of its
translation.  I have addressed the latter problem with the
\textsf{userf} field, which provides a sort of cross-referencing
method for this purpose, and \textsf{reprinttitle} works in
\emph{exactly} the same way.  In the .bib entry for the reprint you
include a cross-reference to the cite key of the original location
using the \textsf{reprinttitle} field (which it may help mnemonically
to think of as a \enquote{reprinted title} field).  The main
difference between the two forms is that \textsf{userf} prints all but
the \textsf{author} of the original work, whereas
\textsf{reprinttitle} suppresses both the \textsf{author} and the
\textsf{title} of the original, giving only the more general details,
beginning with, e.g., the \textsf{journaltitle} or \textsf{booktitle}
and continuing from there.  The string prefacing this information will
be \enquote{Orig.\ pub.\ in.}  Please see the documentation on
\textsf{userf} below for all the details on how to create .bib entries
for presenting your data.

%\enlargethispage{-\baselineskip}

\mybigspace A \mymarginpar{\textbf{series}} standard \textsf{biblatex}
field, usually just a number in an \textsf{article},
\textsf{periodical}, or \textsf{review} entry, almost always the name
of a publication series in \textsf{book}-like entries.  If you need to
attach further information to the \textsf{series} name in a
\textsf{book}-like entry, then the \textsf{number} field is the place
for it, whether it be a volume, a number, or even something like
\enquote{2nd ser.} or \enquote{\cmd{bibstring\{oldseries\}}.}  Of
course, you can also use \cmd{bibstring\{oldseries\}} or
\cmd{bibstring\{newseries\}} in an \textsf{article} entry, but there
you would place it in the \textsf{series} field itself.  (In fact, the
\textsf{series} field in \textsf{article} and \textsf{periodical}
entries is one of the places where \textsf{biblatex} allows you just
to use the plain bibstring \texttt{oldseries}, for example, rather
than making you type \cmd{bibstring\{oldseries\}}.  The \textsf{type}
field in \textsf{manual}, \textsf{patent}, \textsf{report}, and
\textsf{thesis} entries also has this auto-detection mechanism in
place; see the discussion of \cmd{bibstring} below for details.)  In
whatever entry type, these bibstrings produce the required
abbreviation.  (For books and similar entries, see \emph{Manual}
17.90--95; boxer:china, browning:aurora, palmatary:pottery,
plato:republic:gr, wauchope:ceramics; for periodicals, see 17.178;
garaud:gatine, sewall:letter.)  Cf.\ \textsf{number} for more
information on the \emph{Manual}'s preferences regarding the
formatting of numerals.

\mybigspace This \mymarginpar{\textbf{shortauthor}} is a standard
\textsf{biblatex} field, but \textsf{biblatex-chicago} makes
considerably grea\-ter use of it than the standard styles.  For the
purposes of the author-date specification, the field provides the name
to be used in text citations.  In the vast majority of cases, you
don't need to specify it, because the \textsf{biblatex} system selects
the author's last name from the \textsf{author} field and uses it in
such a reference, and if there is no \textsf{author} it will search
\textsf{namea}, \textsf{editor}, \textsf{nameb}, \textsf{translator},
and \textsf{namec}, in that order.  The current versions of
\textsf{biblatex} and \textsf{biber} will now automatically
alphabetize by any of these names if they appear at the head of an
entry.  If, in an author-less \textsf{article} entry
(\textsf{entrysubtype} \texttt{magazine}), you allow
\textsf{biblatex-chicago} to use the title of the periodical as the
author --- the default behavior --- then your \textsf{shortauthor}
field can optionally contain an abbreviated form of the periodical
name, formatted appropriately, which usually means something like
\enquote{\cmd{mkbibemph\{Abbrev.\ Period.\ Title\}}} (gourmet:052006,
lakeforester:pushcarts, nyt:trevorobit, unsigned:ranke).  Indeed, with
long, institutional authors, a shortened version in
\textsf{shortauthor} may save space in the running text
(cotton:manufacture, evanston:library).  See just below under
\textbf{shorthand} for another method of saving space.

\mylittlespace As mentioned under \textsf{editortype}, the
\emph{Manual} (17.41) recommends against providing the identifying
string (e.g., ed.\ or trans.)\ in text citations, and
\textsf{biblatex-chicago} follows their recommendation.  If you need
to provide these strings in such a citation, then you'll have to do so
by hand in the \textsf{shortauthor} field, or in the
\textsf{shorteditor} field, whichever you are using.

\mybigspace Like \mymarginpar{\textbf{shorteditor}}
\textsf{shortauthor}, a field to provide a name for a text citation,
in this case for, e.g., a \textsf{collection} entry that typically
lacks an author.  The \textsf{shortauthor} field works just as well in
most situations, but if you have set \texttt{useauthor=false} (and not
\texttt{useeditor=false}) in an entry's \textsf{options} field, then
only \textsf{shorteditor} will be recognized.  Cf.\
\textsf{editortype}, above.

%\enlargethispage{\baselineskip}

\mybigspace This \mymarginpar{\textbf{shorthand}} is
\textsf{biblatex}'s mechanism for using abbreviations in citations.
For \textsf{biblatex-chicago-authordate} I have modified it somewhat
to conform to the needs of the specification, though there is a
package option to revert the behavior to something closer to the
\textsf{biblatex} standard --- see under \texttt{cmslos} in
section~\ref{sec:authpreset}.  The main problem when presenting
readers with an abbreviation is to ensure that they know how to expand
it.  In the notes \&\ bibliography style this is accomplished with a
notice in the first footnote citing a given work, which explains that
henceforth the abbreviation will be used instead, and also, if needed,
with a list of shorthands that summarizes all the abbreviations used
in a particular text.  The first part of this system isn't available
in the author-date style of citation, and indeed these citations are
in themselves already highly-abbreviated keys to the fuller
information to be found in the list of references.  There are cases,
however, particularly when institutions or \textsf{journaltitles}
appear as authors, when you may feel the need to provide a shortened
version for citations.  I have already discussed one option available
to you just above (cf.\ \textbf{shortauthor}), but for this to work
the abbreviation must either be instantly recognizable to your
readership or at least easily parseable by them.

\mylittlespace The \emph{Manual's} suggestion, particularly when
\enquote{long names are cited several times} (17.47), is to provide an
abbreviation which is then explained by an alphabetized
cross-refer\-ence inside the list of references itself.  In this
release of \textsf{biblatex-chicago}, I have reclaimed one of the
custom entry types (\textbf{customc}) to implement exactly this
system.  See the explanation of this entry type above, but basically
you can put the abbreviation itself in the \textsf{author} field and
its expansion in the \textsf{title} field, then use either
\cmd{nocite} or a \textsf{userc} field to ensure the entry appears in
the list of references.  (Cf.\ abbrev:BSI, abbrev:ISO,
bsi:abbreviation:15, iso:electrodoc:15.)

\mylittlespace The author-date style will still automatically print
the cross-reference in the list of shorthands too, as in standard
\textsf{biblatex}, if the \cmd{printshorthands} command appears in
your document.  You can place \texttt{skiplos} in the \textsf{options}
field to exclude a particular entry from the list of shorthands if you
do decide to print that list, giving maximum flexibility.  Indeed,
\colmarginpar{New!} for this release, I have provided two new options
to add to this flexibility.  First, I have included two new
\texttt{bibenvironments} for use with the \texttt{env} option to the
\cmd{printshorthands} command: \mycolor{\texttt{losnotes}} is designed
to allow a list of shorthands to appear inside footnotes, while
\mycolor{\texttt{losendnotes}} does the same for endnotes.  Their main
effect is to change the font size, and in the latter case to clear up
some spurious punctuation and white space that I see on my system when
using endnotes.  (You'll probably also want to use the option
\texttt{heading=none} in order to get rid of the [oversized] default,
providing your own within the \cmd{footnote} command.)  Second, I have
provided a new package option, \mycolor{\texttt{shorthandfull}}, which
prints entries in the list of shorthands which contain full
bibliographical information, effectively allowing you to eschew the
list of references in favor of a fortified shorthand list.  This
option will only work if used in tandem with \texttt{cmslos=false}, as
the whole functionality seems to me closely related to the more
standard ways of presenting shorthands.  (See 16.39--40, and also
\textsf{biblatex.pdf} for more information.)

%\enlargethispage{-\baselineskip}

\mylittlespace As I mentioned above under \textbf{crossref}, extra
care is needed when using shorthands with cross-references, and I
would avoid them in all parent entries, at least in the current state
of \textsf{biblatex-chicago}.

\mybigspace A \mymarginpar{\textbf{shorttitle}} standard
\textsf{biblatex} field, primarily used to provide an abbreviated
title for citation styles that need one.  In
\textsf{biblatex-chicago-authordate} such a field will be necessary
only very rarely (unlike in the notes \&\ bibliography style), and is
most likely to turn up in \textsf{inreference} or \textsf{reference}
entries (where the \textsf{title} takes the place of the
\textsf{author}), or in any sort of entry with a \texttt{classical}
\textsf{entrysubtype}.  This latter toggle makes citations use
\textsf{author} and \textsf{title} instead of \textsf{author} and
\textsf{year}, and if an abbreviated version of that title would save
space in your running text this is the field where you can provide it.
(Cf.\ ency:britannica, grove:sibelius, aristotle:metaphy:gr.)

\mybigspace A \mymarginpar{\textbf{sortkey}} standard
\textsf{biblatex} field, designed to allow you to specify how you want
an entry alphabetized in a list of references.  In general, if an
entry doesn't turn up where you expect or want it, this field should
provide the solution.  More particularly, entries with a corporate
author beginning with the definite or indefinite article will usually
require your assistance in this way (cotton:manufacture,
grove:sibelius).  The default settings of \cmd{DeclareSortingScheme}
now include the three supplemental name fields (\textsf{name[a-c]})
and also the \textsf{journaltitle} in the sorting algorithm, so you
should find these algorithms needing less help than before.  There may
be circumstances --- several reprinted books by the same author, for
example --- when the \textbf{sortyear} field is more appropriate, on
which see below.  Lehman also provides \textbf{sortname} and
\textbf{sorttitle} for equally fine-grained control.  Please consult
\textsf{biblatex.pdf} for the details.

\mybigspace A \mymarginpar{\textbf{sortyear}} standard
\textsf{biblatex} field, provided by Lehman for more fine-grained
control over the sorting of entries in a list of references, and
possibly useful in \textsf{biblatex-chicago-authordate} to help
present several reprinted books by the same author.  See
\textsf{sortkey} above and maitland:equity.

\mybigspace The \mymarginpar{\textbf{subtitle}} subtitle for a
\textsf{title} --- see next entry.

%\enlargethispage{-\baselineskip}

\mybigspace In \mymarginpar{\textbf{title}} the vast majority of
cases, this field works just as it always has in \textsc{Bib}\TeX, and
just as it does in \textsf{biblatex}.  Nearly every entry for the
author-date specification will have such a field, any exceptions
likely stemming from the need to re-use a database for the notes \&\
bibliography style.  Aside from this, the main source of difficulties
flows from the \emph{Manual}'s rules for formatting \textsf{titles},
rules which also hold for \textsf{booktitles} and \textsf{maintitles}.
The whole point of using a \textsc{Bib}\TeX-based system is for it to
do the formatting for you, and in most cases
\textsf{biblatex-chicago-authordate} does just that, capitalizing them
sentence-style, italicizing them, and sometimes both.  There are two
situations that require user intervention.  First, in titles that take
sentence-style capitalization, you need, as always in traditional
\textsc{Bib}\TeX, to assist the algorithms by placing anything that
needs to remain capitalized within an extra pair of curly braces.
Second, when a title is quoted within a title, you need to know some
of the rules of the Chicago style.  A summary here should serve to
clarify them, and help you to understand when
\textsf{biblatex-chicago-authordate} might need your help in order to
comply with them.

\mylittlespace With regard to sentence-style capitalization, the rules
of the Chicago author-date style are fairly simple:

\begin{description}
\item[\qquad Headline Style:] \textsf{journaltitle} in all types,
  \textsf{series} in all \textsf{book}-like entries (i.e., not in
  \textsf{articles}), and \textsf{title} in \textsf{periodical}
  entries.
\item[\qquad Sentence Style:] Every other \textsf{title},
  \emph{except} in \textsf{letter} entries and in \textsf{misc}
  entries with an \textsf{entrysubtype}.  Also, the
  \textsf{booktitle}, \textsf{issuetitle}, and \textsf{maintitle} in
  all entry types.
\item[\qquad Contextual Capitalization of First Word:]
  \textsf{titleaddon}, \textsf{booktitleaddon},
  \textsf{maintitleaddon} in all entry types, and the \textsf{title}
  of \textsf{misc} entries with an \textsf{entrysubtype}.
\item[\qquad Plain:] \textsf{title} in \textsf{letter} entries.
\end{description}

What this means in practice is that to get a title like \emph{The
  Chicago manual of style}, your .bib entry needs to have a field that
looks something like this:
\begin{quote}
  \texttt{title = \{The \{Chicago\} Manual of Style\}}
\end{quote}

This is completely straightforward, but remember that if an
\textsf{article} has a title like: Review of \emph{The Chicago manual
  of style}, then the curly braces enclosing material to be formatted
in italics will cause the capitalization algorithm to stop and leave
all of that material as it is, so your .bib entry would need to have a
field something like this:

\begin{quote}
  \texttt{title = \{}\cmd{bibstring\{reviewof\}} \cmd{mkbibemph\{The
    Chicago manual of style\}\}}
\end{quote}

(As an aside, the use of the \texttt{reviewof} bibstring isn't
strictly necessary here, but it helps with portability across
languages and across the two Chicago styles.  If you've noticed a lot
of lowercase letters starting fields in \textsf{dates-test.bib},
they're present because in the notes \&\ bibliography style
capitalization is complicated by notes using commas where the
bibliography uses periods, and words like \enquote{review} start in
uppercase only if the context demands it.  There's considerably less
of this in the author-date style [note the \textsf{*titleaddon}
fields], but it still pays to be aware of the issue.)

\mylittlespace With regard to italics, the rules of
\textsf{biblatex-chicago-authordate} are as follows:

\begin{description}
\item[\qquad Italics:] \textsf{booktitle}, \textsf{maintitle}, and
  \textsf{journaltitle} in all entry types; \textsf{title} of
  \textsf{artwork}, \textsf{book}, \textsf{bookinbook},
  \textsf{booklet}, \textsf{collection}, \textsf{inbook},
  \textsf{manual}, \textsf{misc} (with no \textsf{entrysubtype}),
  \textsf{periodical}, \textsf{proceedings}, \textsf{report},
  \textsf{suppbook}, and \textsf{suppcollection} entry types.
\item[\qquad Main Text Font (Roman):] \textsf{title} of
  \textsf{article}, \textsf{image}, \textsf{incollection},
  \textsf{inproceedings}, \textsf{letter}, \textsf{misc} (with an
  \textsf{entrysubtype}), \textsf{online}, \textsf{patent},
  \textsf{periodical}, \textsf{suppperiodical}, \textsf{thesis}, and
  \textsf{unpublished} entry types, \textsf{issuetitle} in
  \textsf{article} and \textsf{periodical} entry types.
  \textsf{booktitleaddon}, \textsf{maintitleaddon}, and
  \textsf{titleaddon} in all entry types.
\item[\qquad Italics or Roman:] All of the audiovisual entry types ---
  \textsf{audio}, \textsf{music}, and \textsf{video} --- have to serve
  as analogues both to \textsf{book} and to \textsf{inbook}.
  Therefore, if there is both a \textsf{title} and a
  \textsf{booktitle}, then the \textsf{title} will be in the main text
  font.  If there is no \textsf{booktitle}, then the \textsf{title}
  will be italicized.
\end{description}

Now, the rules for which entry type to use for which sort of work tend
to be fairly straightforward, but in cases of doubt you can consult
section~\ref{sec:types:authdate} above, the examples in
\textsf{dates-test.bib}, or go to the \emph{Manual} itself,
8.164--210.  Assuming, then, that you want to present a title within a
title, and you know what sort of formatting each of the two would, on
its own, require, then the following rules apply:

\begin{enumerate}
\item Inside an italicized title, all other titles are enclosed in
  quotation marks and italicized, so in such cases all you need to do
  is provide the quotation marks using \cmd{mkbibquote}, which will
  take care of any following punctuation that needs to be brought
  within the closing quotation mark(s) (17.58; donne:var:15,
  mchugh:wake:15).
\item Inside a plain-text title, you should present another title as
  it would appear if it were on its own, so in such cases you'll need
  to do the formatting yourself, using \cmd{mkbibemph} or
  \cmd{mkbibquote}.  (See 17.157; barcott:review:15, garrett:15,
  gibbard:15, loften:hamlet, loomis:structure:15, murphy:silent:15,
  osborne:poi\-son, ratliff:review:15, unsigned:ranke,
  white:\\callimachus.)
\end{enumerate}

\enlargethispage{\baselineskip}

The \emph{Manual} provides a few more rules, as well.  A word normally
italicized in text should also be italicized in a plain-text title,
but should be in roman (\enquote{reverse italics}) in an italicized
title.  A quotation used as a (whole) title (with or without a
subtitle) retains its quotation marks when it is plain, but loses them
when it is italicized (17.60, 17.157; lewis:15).  A word or phrase in
quotation marks, but that isn't a quotation, retains those marks in
all title types (kimluu:diethyl:15).

\mylittlespace Finally, please note that in all \textsf{review} (and
\textsf{suppperiodical}) entries (if you happen to be using those),
and in \textsf{misc} entries with an \textsf{entrysubtype}, and only
in those entries, \textsf{biblatex-chicago-authordate} will
automatically capitalize the first word of the \textsf{title} after
sentence-ending punctuation, assuming that such a \textsf{title}
begins with a lowercase letter in your .bib database.  See
\textbf{\textbackslash autocap} below for more details.

\mybigspace Standard \mymarginpar{\textbf{titleaddon}}
\textsf{biblatex} intends this field for use with additions to titles
that may need to be formatted differently from the titles themselves,
and \textsf{biblatex-chicago} uses it in just this way, with the
additional wrinkle that it can, if needed, replace the \textsf{title}
entirely, and this in, effectively, any entry type, providing a fairly
powerful, if somewhat complicated, tool for getting \textsc{Bib}\TeX\
to do what you want (cf.\ centinel:letters).  This field will always
be unformatted, that is, neither italicized nor placed within
quotation marks, so any formatting you may need within it you'll need
to provide manually yourself.  The single exception to this rule is
when your data begins with a word that would ordinarily only be
capitalized at the beginning of a sentence, in which case you need
then simply ensure that that word is in lowercase, and
\textsf{biblatex-chicago} will automatically do the right thing.  See\
\textbf{\textbackslash autocap}, below.  (Cf.\ brown:bremer,
osborne:poison:15, reaves:rosen, and white:ross:memo for examples
where the field starts with a lowercase letter; morgenson:market
provides an example where the \textsf{titleaddon} field, holding the
name of a regular column in a newspaper, is capitalized, a situation
that is handled as you would expect.)

%\enlargethispage{-\baselineskip}

\mybigspace As \mymarginpar{\textbf{translator}} far as possible, I
have implemented this field as \textsf{biblatex}'s standard styles do,
but the requirements specified by the \emph{Manual} present certain
complications that need explaining.  Lehman points out in his
documentation that the \textsf{translator} field will be associated
with a \textsf{title}, a \textsf{booktitle}, or a \textsf{maintitle},
depending on the sort of entry.  More specifically,
\textsf{biblatex-chicago} associates the \textsf{translator} with the
most comprehensive of those titles, that is, \textsf{maintitle} if
there is one, otherwise \textsf{booktitle}, otherwise \textsf{title},
if the other two are lacking.  In a large number of cases, this is
exactly the correct behavior (adorno:benj, centinel:letters,
plato:republic:gr, among others).  Predictably, however, there are
numerous cases that require, for example, an additional translator for
one part of a collection or for one volume of a multi-volume work.
For these cases I have provided the \textsf{nameb} field.  You should
format names for this field as you would for \textsf{author} or
\textsf{editor}, and these names will always be associated with the
\textsf{title} (euripides:orestes).

\mylittlespace I have also provided a \textsf{namea} field, which
holds the editor of a given \textsf{title} (euripides:orestes).  If
\textsf{namea} and \textsf{nameb} are the same,
\textsf{biblatex-chicago} will concatenate them, just as
\textsf{biblatex} already does for \textsf{editor},
\textsf{translator}, and \textsf{namec} (i.e., the compiler).
Furthermore, it is conceivable that a given entry will need separate
translators for each of the three sorts of title.  For this, and for
various other tricky situations, there is the \cmd{parttrans} macro
(and its siblings), designed to be used in a \textsf{note} field or in
one of the \textsf{titleaddon} fields (ratliff:review:15).  (Because the
strings identifying a translator differ in notes and bibliography, one
can't simply write them out in such a field when using the notes \&\
bibliography style, but you can certainly do so in the author-date
style, if you wish.  Using the macros will make your .bib file more
portable across both Chicago specifications, and also across multiple
languages, but they are otherwise unnecessary.  [See
section~\ref{sec:international}].)

\mylittlespace Finally, as I detailed above under \textbf{author}, in
the absence of an \textsf{author} or an \textsf{editor}, the
\textsf{translator} will be used at the head of an entry
(silver:gawain), and the reference list entry alphabetized by the
translator's name, behavior that can be controlled with the
\texttt{{usetranslator}} switch in the \textsf{options} field.  Cf.\
\textsf{author}, \textsf{editor}, \textsf{namea}, \textsf{nameb}, and
\textsf{namec}.

% \enlargethispage{\baselineskip}

\mybigspace This \mymarginpar{\textbf{type}} is a standard
\textsf{biblatex} field, and in its normal usage serves to identify
the type of a \textsf{manual}, \textsf{patent}, \textsf{report}, or
\textsf{thesis} entry.  \textsf{Biblatex} implements the possibility,
in some circumstances, to use a bibstring without inserting it in a
\cmd{bibstring} command, and in these entry types the \textsf{type}
field works this way, allowing you simply to input, e.g.,
\texttt{patentus} rather than \cmd{bibstring\{patentus\}}, though both
will work.  (See petroff:impurity; herwign:office, murphy:silent:15,
and ross:thesis all demonstrate how the \textsf{type} field may
sometimes be automatically set in such entries by using one of the
standard entry-type aliases).

\mylittlespace Another use for the field is to generalize the
functioning of the \textsf{suppbook} entry type, and of its alias
\textsf{suppcollection}.  In such entries, the \textsf{type} field can
specify what sort of supplemental material you are citing, e.g.,
\enquote{\texttt{preface to}} or \enquote{\texttt{postscript to}.}
Cf.\ \textsf{suppbook} above for the details.  (See \emph{Manual}
17.74--75; polakow:afterw, prose:intro).

\mylittlespace You can also use the \textsf{type} field in
\textsf{artwork}, \textsf{audio}, \textsf{image}, \textsf{music}, and
\textsf{video} entries to identify the medium of the work, e.g.,
\texttt{oil on canvas}, \texttt{albumen print}, \texttt{compact disc},
or \texttt{MPEG}.  If the first word in this field would normally only
be capitalized at the beginning of a sentence, then leave it in
lowercase in your .bib file and \textsf{biblatex} will automatically
do the right thing in citations.  Cf.\ \textsf{artwork},
\textsf{audio}, \textsf{image}, \textsf{music}, and \textsf{video},
above, for all the details.  (See auden:reading:15, bedford:photo,
cleese:holygrail, leo:madonna, nytrumpet:art:15.)

\mybigspace Standard \mymarginpar{\textbf{url}} \textsf{biblatex}
field, it holds the url of an online publication, though you can
provide one for all entry types.  The required \LaTeX\ package
\textsf{url} will ensure that your documents format such references
properly, in the text and in the reference apparatus.

%\enlargethispage{\baselineskip}

\mybigspace Standard \colmarginpar{\textbf{urldate}} \textsf{biblatex}
field, it identifies exactly when you accessed a given url.  This
field would contain the whole date, in \textsc{iso}8601 format
(evanston:library, grove:sibelius, hlatky:hrt, osborne:poison:15,
sirosh:visualcortex, wikiped:bib\-tex).  In the default setting of
\cmd{DeclareLabelyear}, any entry without a \textsf{date},
\textsf{eventdate}, or \textsf{origdate} will now use the
\textsf{urldate} to find a year for citations and the list of
references (grove:sibelius, wikiped:bibtex).  Please note that the
\textbf{urlday}, \textbf{urlmonth}, and \textbf{urlyear} fields are
all now obsolete.

%\enlargethispage{\baselineskip}

\mybigspace A \mymarginpar{\textbf{usera}} supplemental
\textsf{biblatex} field which functions in \textsf{biblatex-chicago}
almost as a \enquote{\textsf{journaltitleaddon}} field.  In
\textsf{article} and \textsf{periodical} entries with
\textsf{entrysubtype} \texttt{maga\-zine}, the contents of this field
will be placed, unformatted and between commas, after the
\textsf{journaltitle} and before the date.  The main use is for
identifying the broadcast network when you cite a radio or television
program (bundy:macneil), though you may also want to use it to
identify the section of a newspaper in which you've found a particular
article (morgenson:market).  (See \emph{Manual} 17.190, 17.207.  As
far as I can work out, newspaper section information may be placed
either before the date [\textsf{usera}] or after it [\textsf{pages}].
Cp. kozinn:review [17.202] and morgenson:market [17.190].  The choice
would appear to be yours.)

\mybigspace I \mymarginpar{\textbf{userc}} have now implemented this
supplemental \textsf{biblatex} field as part of the Chicago
author-date style's handling of the \textsf{shorthand} field and other
cross-references within the list of references.  (The \enquote{c} part
is meant as a sort of mnemonic for this latter, general function,
though its main use will probably be in association with the former.)
If you use the \textbf{customc} entry type to include alphabetized
expansions of \textsf{shorthands} in the reference list, or indeed to
provide cross-references of any sort to separate entries in that list,
it is unlikely that you will cite the \textsf{customc} entry itself in
the body of your text.  Therefore, in order for it to appear in the
reference list, you have two choices.  You can either include the
entry key of the \textsf{customc} entry in a \cmd{nocite} command
inside your document, or you can place that entry key in the
\textsf{userc} field of the .bib entry that actually uses the
\textsf{shorthand}.  In the latter case, \textsf{biblatex-chicago}
will call \cmd{nocite} for you when you cite the main entry.  (See
17.39--40,47; abbrev:BSI, abbrev:ISO, bsi:abbreviation:15,
iso:electrodoc:15.)

\mybigspace Another \mymarginpar{\textbf{usere}} supplemental
\textsf{biblatex} field, which \textsf{biblatex-chicago} uses
specifically to provide a translated \textsf{title} of a work,
something that may be needed if you deem the original language
unparseable by a significant portion of your likely readership.  The
\emph{Manual} offers two alternatives in such a situation: either you
can translate the title and use that translation in your
\textsf{title} field, providing the original language in
\textsf{language}, or you can give the original title in
\textsf{title} and the translation in \textsf{usere}.  Cf.\
\textbf{language}, above.  (See 17.65--67, 17.166, 17.177; kern,
pirumova:russian, weresz.)

\mybigspace This \mymarginpar{\textbf{userf}} is the last of the
supplemental fields which \textsf{biblatex} provides, used by
\textsf{biblatex-chicago} for a very specific purpose.  When you cite
both a translation and its original, the \emph{Manual} (17.66)
recommends that, in a reference list at least, you combine references
to both texts in one entry.  Lacking specific instructions about the
author-date style, I have nonetheless chosen to implement this
possibility also for a list of references, though in-text citations
will still only refer to individual works.  In order to follow this
specification, I have provided a third cross-referencing system (the
others being \textsf{crossref} and \textsf{xref}), and have chosen the
name \textsf{userf} because it might act as a mnemonic for its
function.

%\enlargethispage{-\baselineskip}

\mylittlespace In order to use this system, you should start by
entering both the original and its translation into your .bib file,
just as you normally would.  The mechanism works for any entry type,
and the two entries need not be of the same type.  In the entry for
the \emph{translation}, you put the cite key of the original into the
\textsf{userf} field.  In the \emph{original's} entry, you need to
include some means of preventing it appearing separately in the list
of references, either a toggle in the \textsf{keywords} field or
perhaps \texttt{skipbib} in the \textsf{options} field.  In this
standard case, the data for the translation will be printed first,
followed by the string \texttt{orig. pub. as}, followed by the
original, author omitted (furet:passing:eng, furet:passing:fr).  As
explained above (\textbf{origlanguage}), I have also included a way to
modify the string printed before the original.  In the entry for the
\emph{translation}, you put the original's language in
\textsf{origlanguage}, and instead of \texttt{originally published
  as}, you'll get \texttt{French edition:} or \texttt{Latin edition:},
etc.\ (aristotle:metaphy:gr, aristotle:metaphy:trans).

\mybigspace Standard \mymarginpar{\textbf{venue}} \textsf{biblatex}
offers this field for use in \textsf{proceedings} and
\textsf{inproceedings} entries, but I haven't yet implemented it,
mainly because the \emph{Manual} has nothing to say about it.  Perhaps
the \textsf{organization} field could be used, for the moment,
instead.  Anything in a \textsf{venue} field will be ignored.

\mybigspace Standard \mymarginpar{\textbf{version}} \textsf{biblatex}
field, currently only available in \textsf{misc} and \textsf{patent}
entries in \textsf{biblatex-chicago}.

\mybigspace Standard \mymarginpar{\textbf{volume}} \textsf{biblatex}
field.  It holds the volume of a \textsf{journaltitle} in
\textsf{article} entries, and also the volume of a multi-volume work
in many other sorts of entry.  Cf.\ \textsf{part}; conway:evolution
shows how sometimes this field may hold series information, as well.

\mybigspace Standard \mymarginpar{\textbf{volumes}} \textsf{biblatex}
field.  It holds the total number of volumes of a multi-volume work,
and in such references you should provide the volume and page numbers
in the \textsf{postnote} field of the relevant \cmd{cite} command,
e.g.:

\begin{quote}
\cmd{autocite}\texttt{[3:25]\{bibfile:key\}}.  
\end{quote}

Cf.\ 16.110; meredith:letters, tillich:system, weber:saugetiere,
wright:evolution.  The entry wright:theory presents one volume of such
a multi-volume work, so you would no longer need to give the volume in
any \textsf{postnote} field when citing it.

\mybigspace A \mymarginpar{\textbf{xref}} modified \textsf{crossref}
field provided by \textsf{biblatex}.  See \textbf{crossref}, above.

\mybigspace Standard \mymarginpar{\textbf{year}} \textsf{biblatex}
field, especially important for the author-date specification.  Please
see all the details under \textbf{date} above.  Unlike the
\textsf{date} field \textsf{year} allows non-numeric input, so you can
put \cmd{bibstring\{nodate\}} here if required, or indeed any other
sort of non-numerical date information.  If you can guess the date
then you can include that guess in square brackets instead of
\cmd{bibstring\{nodate\}}.  Cf.\ bedford:photo, clark:mesopot,
ross:leo, thesis:madonna.

\subsection{Commands}
\label{sec:commands:authdate}

In this section I shall attempt to document all those commands you may
need when using \textsf{biblatex-chicago-authordate} that I have either
altered with respect to the standard provided by \textsf{biblatex} or
that I have provided myself.  Some of these, unfortunately, will make
your .bib file incompatible with other \textsf{biblatex} styles, but
I've been unable to avoid this.  Any ideas for more elegant, and more
compatible, solutions will be warmly welcomed.

\subsubsection{Formatting Commands}
\label{sec:formatting:authdate}

These commands allow you to fine-tune the presentation of your
references in both citations and list of references.  You can find
many examples of their usage in \textsf{dates-test.bib}, and I shall
try to point you toward a few such entries in what follows.
\textbf{NB:} \textsf{biblatex's} \cmd{mkbibquote} command is now
mandatory in some situations.  See its entry below.

\mybigspace Version \mymarginpar{\textbf{\textbackslash autocap}} 0.8
of \textsf{biblatex} introduced the \cmd{autocap} command, which
capitalizes a word inside a citation or list of references entry if
that word follows sentence-ending punctuation, and leaves it lowercase
otherwise.  The whole question of capitalization is considerably more
complicated in the notes \&\ bibliography style, where the former uses
commas and the latter (often) periods to separate blocks of
information, whereas the more streamlined author-date specification
has few such issues.  In \textsf{dates-test.bib} there are only two
places where the \cmd{autocap} macro is necessary, and they both
involve the string \texttt{forthcoming} in the \textsf{year} field
(author:forthcoming, contrib:contrib).

\mylittlespace I have nonetheless retained the system developed,
following Lehman's example, for the notes \&\ bibliography style,
which automatically tracks the capitalization of certain fields in
your .bib file.  I chose these fields after a non-scientific survey of
entries in my own databases, so of course if you have ideas for the
extension of this facility I would be most interested to hear them.
In order to take advantage of this functionality, all you need do is
begin the data in the appropriate field with a lowercase letter,
e.g.,\ \texttt{note = \{with the assistance of X\}}.  If the data
begins with a capital letter --- and this is not infrequent --- that
capital will always be retained.  (cf., e.g., creel:house,
morgenson:market.)  If, on the other hand, you for some reason need
such a field always to start with a lowercase letter, then you can try
using the \cmd{isdot} macro at the start, which turns off the
mechanism without printing anything itself.  Here, then, for reference
purposes, is the complete list of fields where this functionality is
active:

\begin{enumerate}
\item The \textbf{addendum} field in all entry types.
\item The \textbf{booktitleaddon} field in all entry types.
\item The \textbf{edition} field in all entry types.  (Numerals work
  as you expect them to here.)
\item The \textbf{maintitleaddon} field in all entry types.
\item The \textbf{note} field in all entry types.
\item The \textbf{shorttitle} field in the \textsf{review}
  (\textsf{suppperiodical}) entry type and in the \textsf{misc} type,
  in the latter case, however, only when there is an
  \textsf{entrysubtype} defined, indicating that the work cited is
  from an archive.
\item The \textbf{title} field in the \textsf{review}
  (\textsf{suppperiodical}) entry type and in the \textsf{misc} type,
  in the latter case, however, only when there is an
  \textsf{entrysubtype} defined, indicating that the work cited is
  from an archive.
\item The \textbf{titleaddon} field in all entry types.
\item The \textbf{type} field in \textsf{artwork}, \textsf{audio},
  \textsf{image}, \textsf{music}, \textsf{suppbook},
  \textsf{suppcollection}, and \textsf{video} entry types.
\end{enumerate}

If you accidentally use the \cmd{autocap} macro in one of the above
fields, it frankly shouldn't matter at all, and you'll still get what
you want, but taking advantage of the automatic provisions should at
least save some typing.

\mybigspace This \mymarginpar{\textbf{\textbackslash bibstring}} is
Lehman's very powerful mechanism to allow \textsf{biblatex}
automatically to provide a localized version of a string, and to
determine whether that string needs capitalization, depending on where
it falls in an entry.  \textsf{Biblatex} also provides functionality
which allows you sometimes simply to input, for example,
\texttt{newseries} instead of \cmd{bib\-string\{newseries\}}, the
package auto-detecting when a bibstring is involved and doing the
right thing, though in all such cases either form will work.  This
functionality is available in the \textsf{series} field of
\textsf{article}, \textsf{periodical}, and \textsf{review} entries; in
the \textsf{type} field of \textsf{manual}, \textsf{patent},
\textsf{report}, and \textsf{thesis} entries; in the \textsf{location}
field of \textsf{patent} entries; in the \textsf{language} field in
all entry types; and in the \textsf{nameaddon} field in
\textsf{customc} entries.  These are the places, as far as I can make
out, where \textsf{biblatex's} standard styles support this feature,
though I have added the last, style-specific, one.  If Lehman
generalizes it still further in a future release, I shall do the same,
if possible.

%\enlargethispage{\baselineskip}

\mybigspace This \mymarginpar{\textbf{\textbackslash mkbibquote}} is
the standard \textsf{biblatex} command, which requires attention here
because it is a crucial part of the mechanism of Lehman's
\enquote{American} punctuation system.  No titles in the author-date
system require quotation marks, but titles-within-titles frequently
do, so it is best to get accustomed to using this command to make sure
any periods or commas appearing in the neighborhood of the closing
quotes will appear inside them automatically.  A few examples from
\textsf{dates-test.bib} should help to clarify this.

\mylittlespace In an \textsf{article} entry, the \textsf{title}
contains a quoted phrase:

\begin{quotation}
  \noindent\texttt{title = \{Diethylstilbestrol and Media Coverage of the \\
    \indent\cmd{mkbibquote}\{morning after\} Pill\}}
\end{quotation}

Here, because the quoted text doesn't come at the end of title, and no
punctuation will ever need to be drawn within the closing quotation
mark, you could instead use \texttt{\cmd{enquote}\{morning after\}} or
even \texttt{``morning after''}. (Note the double quotation marks here
--- the other two methods have the virtue of taking care of nesting
for you.)  All of these will produce the formatted: Diethylstilbestrol
and media coverage of the \enquote{morning after} pill.

\mylittlespace Here, by contrast, is a \textsf{book title}:

\begin{quotation}
  \noindent \texttt{title = \{Annotations to
    \cmd{mkbibquote}\{Finnegans wake\}\}}
\end{quotation}

Because the quoted title within the title comes at the end of the
field, and because this reference unit will be separated from
what follows by a period in the list of references, then the
\cmd{mkbibquote} command is necessary to bring that period within the
final quotation marks, like so: \emph{Annotations to
  \enquote{Finnegans wake.}}

\mylittlespace Note in both cases how you need to be careful with the
sentence-style capitalization inside the curly brackets.  The
automatic algorithms assume anything inside the brackets doesn't need
alteration, so you need to provide lower- or uppercase as they should
appear in the list of references.

\mylittlespace Let me also add that this command interacts well with
Lehman's \textsf{csquotes} package, which I highly recommend, though
the latter isn't strictly necessary in texts using an American style,
to which \textsf{biblatex} defaults when \textsf{csquotes} isn't
loaded.

\mybigspace This \mymarginpar{\textbf{\textbackslash partcomp}} and
the following 6 macros were all designed to help
\textsf{biblatex-chicago} cope with the fact that many bibstrings in
the notes \&\ bibliography style differ between notes and
bibliography, the former sometimes using abbreviated forms when the
latter prints them in full.  These problems do not arise in the
author-date style, but using these macros will make your .bib database
more portable across languages and across both Chicago styles, and may
be slightly easier to remember than the strings themselves.  On the
other hand, of course, they will make your .bib file less portable
across multiple \textsf{biblatex} styles.

\mylittlespace These macros allow you to provide an \texttt{editor}, a
\texttt{translator}, and/or a \texttt{compiler} in situations where
the available fields (\textsf{editor}, \textsf{namea},
\textsf{translator}, \textsf{nameb}, and \textsf{namec}) aren't
adequate.  Their names all begin with \cmd{part}, as originally I
intended them for use when a particular name applied only to a
specific \textsf{title}, rather than to a \textsf{maintitle} or
\textsf{booktitle} (cf.\ \textbf{namea} and \textbf{nameb}, above).

%\enlargethispage{\baselineskip}

\mylittlespace In the present instance, you can use \cmd{partcomp} to
identify a compiler when \textsf{namec} (or \textsf{editortype}) won't
do, e.g., in a \textsf{note} field or the like.  In such a case,
\textsf{biblatex-chicago} will print the appropriate string in your
references.

\mybigspace Use \mymarginpar{\textbf{\textbackslash partedit}} this
macro when identifying an editor whose name doesn't conveniently fit
into the usual fields (\textsf{editor} or \textsf{namea}).  (N.B.: If
you are writing in French and using \textsf{cms-french.lbx}, then
currently you'll need to add either \texttt{de} or \texttt{d'} after
this command in your .bib files to make the references come out right.
I'm working on this.)  See howell:marriage.

\mybigspace As \mymarginpar{\textbf{\textbackslash
    partedit-\\andcomp}} before, but for use when an editor is also a
compiler.

\vspace{1.3\baselineskip} As \mymarginpar{\textbf{\textbackslash
    partedit-\\andtrans}} before, but for when when an editor is also a
translator (ratliff:review:15).

\mybigspace As \mymarginpar{\textbf{\textbackslash
    partedit-\\transandcomp}} before, but for when an editor is also a
translator and a compiler.

\vspace{1.3\baselineskip} As \mymarginpar{\textbf{\textbackslash
    parttrans}} before, but for use when identifying a translator
whose name doesn't conveniently fit into the usual fields
(\textsf{translator} and \textsf{nameb}).

\mybigspace As \mymarginpar{\textbf{\textbackslash
    parttrans-\\andcomp}} before, but for when a translator is also a
compiler.

\subsubsection{Citation Commands}
\label{sec:cite:authordate}

The \textsf{biblatex} package is particularly rich in citation
commands, most of which, in \textsf{biblatex-chicago-authordate},
function as they do in the standard author-date styles.  If you are
getting unexpected behavior when using them please have a look in your
.log file.  A command like \cmd{supercite}, listed in �~3.6.2 of the
\textsf{biblatex} manual but not defined by
\textsf{biblatex-chicago-authordate} or by core \textsf{biblatex},
defaults to \cmd{cite}, and leaves a warning in the .log.  The
following commands may require some minimal explanation, but if there
are standard commands that don't work for you, or new commands that
would be useful, please let me know, and it should be possible to fix
or add them.

\mybigspace I \mymarginpar{\textbf{\textbackslash autocite}} haven't
adapted this in the slightest, but I thought it worth pointing out
that \textsf{biblatex-chicago-authordate} sets this command to use
\cmd{parencite} as the default option.  It is, in my experience, much
the most common citation command you will use, and also works fine in
its multicite form, \textbf{\textbackslash autocites}.

\mybigspace In \mymarginpar{\textbf{\textbackslash textcite}} standard
\textsf{biblatex} this command searches first for a
\textsf{labelname}, usually taken from the \textsf{author} or
\textsf{shortauthor} field, then uses the \textsf{shorthand} field if
the former doesn't exist.  Because of the way the Chicago author-date
specification recommends handling abbreviations, I have switched this
around, and the command now searches for a \textsf{shorthand} first.
This holds also for the multicite form \textbf{\textbackslash
  textcites}, though both commands revert to their standard
\textsf{biblatex} behavior when you give the \texttt{cmslos=false}
option in the preamble.

\subsection{Package Options}
\label{sec:opts:authdate}

\subsubsection{Pre-set \textsf{biblatex} Options}
\label{sec:preset:authdate}

Although a quick glance through \textsf{biblatex-chicago.sty} will
tell you which \textsf{biblatex} options the package sets for you, I
thought I might gather them here also for your perusal.  These
settings are, I believe, consistent with the specification, but you
can alter them in the options to \textsf{biblatex-chicago} in your
preamble or by loading the package using
\cmd{usepackage[style=chicago-authordate]\{biblatex\}}, which gives
you the \textsf{biblatex} defaults unless you redefine them yourself
inside the square brackets.

\mylittlespace \textsf{Biblatex-chicago-authordate}
\mymarginpar{\texttt{autocite=\\inline}} places references in
footnotes by default.

\mybigspace The \mymarginpar{\texttt{citetracker=\\true}} citetracker
for the \cmd{ifciteseen} test is enabled globally.

\mybigspace The \mymarginpar{\texttt{alldates=comp}} specification
calls for the long format when presenting dates, slightly shortened
when presenting date ranges.  Please note that because of the
author-date style's complicated requirements with respect to dates,
there will be cases when printed ranges don't look exactly right ---
cf., e.g., nass:address.  I'm working on this.

\mylittlespace This \mymarginpar{\texttt{ibidtracker=\\constrict}}
enables an \emph{ibidem} mechanism in citations, but only in the most
strictly-defined circumstances.  The Chicago author-date style doesn't
print \enquote{Ibid} in citations, but in general a repeated citation
on the same page will print only the page reference.  Technically,
this should only occur when a source is cited \enquote{more than once
  in one paragraph} (16.114), so you can use the \cmd{citereset}
command from \textsf{biblatex} to achieve the greatest compliance, as
the package only offers automatic resetting on part, chapter, section,
and subsection boundaries, while \textsf{biblatex-chicago}
automatically resets the tracker at page breaks.  (Cf.\
\textsf{biblatex.pdf} �3.1.2.1.)  Whenever there might be any
ambiguity, \textsf{biblatex} should default to printing a more
informative reference.

\mylittlespace If you are going to repeat a source, make sure that the
cite command provides a \textsf{postnote} --- if you don't need to cite a
specific page, then it's better only to use one citation rather than
two, as otherwise, in the current state of the code, you'll get empty
parentheses, like so: ().

%\enlargethispage{\baselineskip}

\mylittlespace This \mymarginpar{\texttt{labelyear=\\true}} option
tells \textsf{biblatex} to provide the special \textsf{labelyear} and
\textsf{extrayear} fields for author-date styles.

\mylittlespace These \colmarginpar{\textsf{\texttt{maxbibnames\\=10\\
      minbibnames\\=7}}} two options are new, and control the number
of names printed in the list of references when that number exceeds
10.  These numbers follow the recommendations of the \emph{Manual}
(17.29--30), and they are different from those for use in citations.
With \textsf{biblatex} 1.6 and later you can no longer redefine
\texttt{maxnames} and \texttt{minnames} in the \cmd{printbibliography}
command at the bottom of your document, so \textsf{biblatex-chicago}
now does this automatically for you, though of course you can change
them in your document preamble.  Please see
section~\ref{sec:otherhints:auth} below (and the file
\textsf{cms-dates-sample.pdf}) for hints on dealing with entries with
more than three authors.

\mylittlespace This \mymarginpar{\texttt{pagetracker=\\true}} enables
page tracking for the \cmd{iffirstonpage} and \cmd{ifsamepage}
commands for controlling, among other things, the \emph{ibidem}
mechanism.  It tracks individual pages if \LaTeX\ is in oneside mode,
or whole spreads in twoside mode.

\mylittlespace This \mymarginpar{\texttt{punctfont=\\true}} fixes a
minor problem with punctuation in titles, ensuring that the colon
between a title and a subtitle appears in the correct, matching font.

\mylittlespace This \mymarginpar{\texttt{sortcase=\\false}} turns off
the sorting of uppercase and lowercase letters separately, a practice
which the \emph{Manual} doesn't appear to recommend.

\mylittlespace This \colmarginpar{\texttt{sorting=cms}} setting, new
in this release, takes advantage of the \cmd{DeclareSortingScheme}
command provided by \textsf{biblatex} and \textsf{biber}, in effect
implementing a default sorting order in the list of references
tailored to comply with the Chicago author-date specification.  Please
see the documentation of \cmd{DeclareSortingScheme} in
section~\ref{sec:authformopts}, below.

\mylittlespace This \mymarginpar{\texttt{uniquelist=\\minyear}}
enables \textsf{biblatex-chicago-authordate} to disambiguate entries
which have more than three \textsf{authors}, but which differ
\emph{after} the first name in the list.  This will only occur when
two such entries have the same \textsf{year} (16.118).  The option is
\textsf{Biber}-only, like the following, which means that this
next-generation \textsc{Bib}\TeX\ replacement is now required for the
author-date style.  Please see \textsf{cms-dates-sample.pdf} and
section~\ref{sec:otherhints:auth}, below, for further details.

\mylittlespace This \mymarginpar{\texttt{uniquename=\\minfull}}
enables the package to distinguish different authors who share a
surname, using initials in the first instance, and whole names if
initials aren't enough (16.108).  The option is now
\textsf{Biber}-only, like the previous one.

\mylittlespace This \mymarginpar{\texttt{usetranslator\\=true}}
enables automatic use of the \textsf{translator} at the head of
entries in the absence of an \textsf{author} or an \textsf{editor}.
In the list of references, the entry will be alphabetized by the
translator's surname.  You can disable this functionality on a
per-entry basis by setting \texttt{usetranslator=false} in the
\textsf{options} field.  Cf.\ silver:gawain.

\subsubsection*{Other \textsf{biblatex} Formatting Options}
\label{sec:authformopts}

I've chosen defaults for many of the general formatting commands
provided by \textsf{biblatex}, including the vertical space between
items in the list of references and between items in the list of
shorthands (\cmd{bibitemsep} and \cmd{lositemsep}).  I define many of
these in \textsf{biblatex-chicago.sty}, and of course you may want to
redefine them to your own needs and tastes.  It may be as well you
know that the \emph{Manual} does state a preference for two of the
formatting options I've implemented by default: the 3-em dash as a
replacement for repeated names in the list of references
(16.103--106); and the formatting of note numbers, both in the main
text and at the bottom of the page / end of the essay (superscript in
the text, in-line in the notes; 16.25).  The code for this last
formatting is also in \textsf{biblatex-chicago.sty}, and I've wrapped
it in a test that disables it if you are using the \textsf{memoir}
class, which I believe has its own commands for defining these
parameters.  You can also disable it by using the \texttt{footmarkoff}
package option, on which see below.

\mylittlespace Gildas Hamel pointed out that my default definition, in
\textsf{biblatex-chicago.sty}, of \textsf{biblatex's}
\cmd{bibnamedash} didn't work well with many fonts, leaving a line of
three dashes separated by gaps.  He suggested an alternative, which
I've adopted, with a minor tweak to make the dash thicker, though you
can toy with all the parameters to find what looks right with your
chosen font.  The default definition is:
\cmd{renewcommand*\{\textbackslash bibnamedash\}\{\textbackslash
  rule[.4ex]\{3em\}\{.6pt\}\}}.

\mylittlespace With \colmarginpar{\texttt{losnotes}
  \&\\\texttt{losendnotes}} this release, and at the request of
Kenneth Pearce, I have added two new \texttt{bibenvironments} to
\textsf{chicago-authordate.bbx}, for use with the \texttt{env} option
to the \cmd{printshorthands} command.  The first, \texttt{losnotes},
is designed to allow a list of shorthands to appear inside footnotes,
while \texttt{losendnotes} does the same for endnotes.  Their main
effect is to change the font size, and in the latter case to clear up
some spurious punctuation and white space that I see on my system when
using endnotes.  (You'll probably also want to use the option
\texttt{heading=none} in order to get rid of the [oversized] default,
providing your own within the \cmd{footnote} command.)  Please see the
documentation of \textsf{shorthand} in
section~\ref{sec:fields:authdate} above for further options available
to you for presenting and formatting the list of shorthands.

\mylittlespace The next-generation backend \textsf{Biber} offers
enhanced functionality in many areas, two of which I've implemented in
this release.  If the default definitions don't work well for you, you
can redefine all of them in your document preamble --- see
\textsf{biblatex.pdf} ��4.5.1 and 4.5.2.

\mylittlespace This \colmarginpar{\cmd{Declare-}\\\texttt{Labelname}}
option allows you to add name fields for consideration when
\textsf{biblatex} is attempting to find a shortened name for in-text
citations.  This, for example, allows a compiler (=\textsf{namec}) to
appear in citations without any other intervention from the user,
rather than requiring a \textsf{shortauthor} field as previous
releases of \textsf{biblatex-chicago} did.  The default definition
currently is
\texttt{\{shortauthor,author,\\shorteditor,namea,editor,nameb,translator,namec\}}.

\mylittlespace The
\colmarginpar{\cmd{Declare-}\\\texttt{SortingScheme}} third
\textsf{Biber} enhancement I have implemented allows you to include
almost any field whatsoever in \textsf{biblatex's} sorting algorithms
for the list of references, so that a great many more entries will be
sorted correctly automatically rather than requiring manual
intervention in the form of a \textsf{sortkey} field or the like.
Code in \textsf{biblatex-chicago.sty} sets the \textsf{biblatex}
option \texttt{sorting=cms}, which is a custom scheme, basically a
Chicago-specific variant of the default \texttt{nyt}.  You can find
its definition in \textsf{chicago-authordate.cbx}.

\mylittlespace The advantages of this scheme are, specifically, that
any entry headed by one of the supplemental name fields
(\textsf{name[a-c]}), a \textsf{manual} entry headed by an
\textsf{organization}, or an \textsf{article} or \textsf{review} entry
headed by a \textsf{journaltitle} will no longer need a
\textsf{sortkey} set.  The main disadvantage should only occur very
rarely, and appears because the supplemental name fields are treated
differently from the standard name fields by \textsf{biblatex}.
Ordinarily, you can set, for example, \texttt{useauthor=false} in the
\textsf{options} field to remove the \textsf{author's} name from
consideration for sorting purposes.  The Chicago-specific option
\textsf{usecompiler=false}, however, doesn't remove \textsf{namec}
from such consideration, so in some rare corner cases you may need a
\textsf{sortkey}.

\subsubsection{{Pre-set \textsf{chicago} Options}}
\label{sec:authpreset}

At \mymarginpar{\texttt{bookpages=\\true}} the request of Scot
Becker, I have included this rather specialized option, which controls
the printing of the \textsf{pages} field in \textsf{book} entries.
Some bibliographic managers, apparently, place the total page count in
that field by default, and this option allows you to stop the printing
of this information in the reference list.  It defaults to true, which
means the field is printed, but it can be set to false either in the
preamble, for the whole document, or on a per-entry basis in the
\textsf{options} field (though rather than use this latter method it
would make sense to eliminate the \textsf{pages} field from the
affected entries).

\mylittlespace This \mymarginpar{\texttt{doi=true}} option controls
whether any \textsf{doi} fields present in the .bib file will be
printed in the reference list.  It defaults to true, and can be set to
false either in the preamble, for the whole document, or on a
per-entry basis, in the \textsf{options} field.

\mylittlespace This \colmarginpar{\texttt{eprint=true}} option controls
whether any \textsf{eprint} fields present in the .bib file will be
printed in the list of references.  It defaults to true, and can be
set to false either in the preamble, for the whole document, or on a
per-entry basis, in the \textsf{options} field.  In \textsf{online}
entries, the \textsf{eprint} field will always be printed.

\mylittlespace This \mymarginpar{\texttt{isbn=true}} option controls
whether any \textsf{isan}, \textsf{isbn}, \textsf{ismn},
\textsf{isrn}, \textsf{issn}, and \textsf{iswc} fields present in the
.bib file will be printed in the list of references.  It defaults to
true, and can be set to false either in the preamble, for the whole
document, or on a per-entry basis, in the \textsf{options} field.

%\enlargethispage{-\baselineskip}

\mylittlespace Once \mymarginpar{\texttt{numbermonth=\\true}} again
at the request of Scot Becker, I have included this option, which
controls the printing of the \textsf{month} field in all the
periodical-type entries when a \textsf{number} field is also present.
Some bibliographic software, apparently, always includes the month of
publication even when a \textsf{number} is present.  When all this
information is available the \emph{Manual} (17.181) prints everything,
so this option defaults to true, which means the field is printed, but
it can be set to false either in the preamble, for the whole document,
or on a per-entry basis in the \textsf{options} field.

\mylittlespace This \mymarginpar{\texttt{url=true}} option controls
whether any \textsf{url} fields present in the .bib file will be
printed in the reference list.  It defaults to true, and can be set to
false either in the preamble, for the whole document, or on a
per-entry basis, in the \textsf{options} field.  Please note that, as
in standard \textsf{biblatex}, the \textsf{url} field is always
printed in \textsf{online} entries, regardless of the state of this
option.

\mylittlespace This \mymarginpar{\texttt{includeall=\\true}} is the
one option that rules the six preceding, either printing all the
fields under consideration --- the default --- or excluding all of
them.  It is set to \texttt{true} in \textsf{chicago-authordate.cbx},
but you can change it either in the preamble for the whole document or
in the \textsf{options} field of individual entries.  The rationale
for all of these options is the availability of bibliographic managers
that helpfully present as much data as possible, in every entry, some
of which may not be felt to be entirely necessary.  Setting
\texttt{includeall} to \texttt{true} probably works just fine for
those compiling their .bib databases by hand, but others may find that
some automatic pruning helps clear things up, at least to a first
approximation.  Some per-entry work afterward may then polish up the
details.

\mylittlespace This \mymarginpar{\texttt{cmslos=true}} option alters
\textsf{biblatex's} standard behavior when processing the
\textsf{shorthand} field.  Chicago's author-date style only seems to
recommend the use of shorthands as abbreviations for long authors'
names, particularly institutional names, which means the shorthand
will replace only the name part in citations rather than the whole
citation (17.47; bsi:abbreviation:15, iso:electrodoc:15).  It suggests
placing the expansion of the abbreviation into the reference list
itself, a procedure that I have implemented in this release using the
\textbf{customc} entry type, which see.  Please note that you can
still print a list of shorthands if you wish, and you can also get
back something approaching the \enquote{standard} behavior of
shorthands if you give the \texttt{cmslos=false} option to
\textsf{biblatex-chicago} in your document preamble.  Cf.\
section~\ref{sec:fields:authdate}, s.v. \enquote{\textbf{shorthand}}
above, and also \textsf{cms-dates-sample.pdf}.

\mylittlespace This \mymarginpar{\texttt{nodates=true}} option means
that for all entry types except \textsf{inreference}, \textsf{misc},
and \textsf{reference}, \textsf{biblatex-chicago} will automatically
provide \cmd{bibstring\{nodates\}} for any entry that doesn't
otherwise provide a date for citations and for the heads of entries in
the list of references.  If you set \texttt{nodates=false} in your
preamble, then the package won't perform this substitution in any
entry type whatsoever.  (The bibstring expands to
\enquote{\texttt{n.d.}} in English.)

\mylittlespace This \mymarginpar{\texttt{usecompiler=\\true}} option
enables automatic use of the name of the compiler (in the
\textsf{namec} field) at the head of an entry, usually in the absence
of an \textsf{author}, \textsf{editor}, or \textsf{translator}, in
accordance with the specification (\emph{Manual} 17.41).  It may also,
like \texttt{useauthor}, \texttt{useeditor}, and
\texttt{usetranslator}, be disabled on a per-entry basis by setting
\texttt{usecompiler=false} in the \textsf{options} field.  The only,
subtle, difference between this switch and those standard
\textsf{biblatex} switches is that this one won't remove
\textsf{namec} from the sorting list, whereas \texttt{useauthor=false}
and \texttt{useeditor=false} do remove the \textsf{author} and
\textsf{editor}.  You may, therefore, in corner cases, require a
\textsf{sortkey} in the entry.

\subsubsection{Style Options -- Preamble}
\label{sec:authuseropts}

These are parts of the specification that not everyone will wish to
enable.  All except the second can be used even if you load the
package in the old way via a call to \textsf{biblatex}, but most users
can just place the appropriate string(s) in the options to the
\cmd{usepackage\{biblatex-chicago\}} call in your preamble.

\mylittlespace At \mymarginpar{\texttt{annotation}} the request of
Emil Salim, I have added to this version of \textsf{biblatex-chicago}
the ability to produce annotated reference lists.  If you turn this
option on then the contents of your \textsf{annotation} (or
\textsf{annote}) field will be printed after the reference.  (You can
also use external files to store annotations -- please see
\textsf{biblatex.pdf} �~3.10.7 for details on how to do this.)  This
functionality is currently in a beta state, so before you use it
please have a look at the documentation for the \textsf{annotation}
field, in section~\ref{sec:fields:authdate} above.

\mylittlespace Although \mymarginpar{\texttt{footmarkoff}} the
\emph{Manual} (16.25) recommends specific formatting for footnote (and
endnote) marks, i.e., superscript in the text and in-line in foot- or
endnotes, Charles Schaum has brought it to my attention that not all
publishers follow this practice, even when requiring Chicago style.  I
have retained this formatting as the default setup, but if you include
the \texttt{footmarkoff} option, \textsf{biblatex-chicago} will not
alter \LaTeX 's (or the \textsf{endnote} package's) defaults in any
way, leaving you free to follow the specifications of your publisher.
I have placed all of this code in \textsf{biblatex-chicago.sty}, so if
you load the package with a call to \textsf{biblatex} instead, then
once again footnote marks will revert to the \LaTeX\ default, but of
course you also lose a fair amount of other formatting, as well.  See
section~\ref{sec:loading:auth}, below.

\mylittlespace Several \colmarginpar{\texttt{headline}} users have
requested an option that turns off the automatic transformations that
produce sentence-style capitalization in the title fields of the
author-date style.  If you set this option, the word case in your
title fields will not be changed in any way, that is, this doesn't
automatically transform your titles into headline-style, but rather
allows the .bib file to determine capitalization.  It works by
redefining the command \cmd{MakeSentenceCase}, so in the unlikely
event you are using the latter anywhere in your document please be
aware that it will also be turned off there.

\mylittlespace The \mymarginpar{\texttt{juniorcomma}} \emph{Manual}
(6.49) states that \enquote{commas are no longer required around
  \emph{Jr.}\ and \emph{Sr.},} so by default \textsf{biblatex-chicago}
has followed standard \textsf{biblatex} in using a simple space in
names like \enquote{John Doe Jr.}  Charles Schaum has pointed out that
traditional \textsc{Bib}\TeX\ practice was to include the comma, and
since the \emph{Manual} has no objections to this, I have provided an
option which allows you to turn this behavior back on, either for the
whole document or on a per-entry basis.  Please note, first, that
numerical suffixes (John Doe III) never take the comma.  The code
tests for this situation, and detects cardinal numbers well, but if
you are using ordinals you may need to set this to \texttt{false} in
the \textsf{options} field of some entries.  Second, I have fixed a
bug in older releases which always printed the \enquote{Jr.}\ part of
the name immediately after the surname, even when the surname came
before the given names (as in a reference list).  The package now
correctly puts the \enquote{Jr.}\ part at the end, after the given
names, and in this position it always takes a comma, the presence of
which is unaffected by this option.

%\enlargethispage{\baselineskip}

\mylittlespace This \mymarginpar{\texttt{natbib}} may look like the
standard \textsf{biblatex} option, but to keep the coding of
\textsf{biblatex-chicago.sty} simpler for the moment I have
reimplemented it there, from whence it is merely passed on to
\textsf{biblatex}.  If you load the Chicago style with
\cmd{usepackage\{biblatex-chicago\}}, then the option should simply
read \texttt{natbib}, rather than \texttt{natbib=true}.  The shorter
form also works if you use \cmd{usepackage}\\
\texttt{[style=chicago-authordate]\{biblatex\}}, so I hope this
requirement isn't too onerous.

\mylittlespace At \mymarginpar{\texttt{noibid}} the request of an
early tester, I have included this option to allow you globally to
turn off the \texttt{ibidem} mechanism that
\textsf{biblatex-chicago-authordate} uses by default.  This mechanism
doesn't actually print \enquote{Ibid,} but rather includes only the
\textsf{postnote} information in a citation, i.e., it will print (224)
instead of (Author 2000, 224).  Setting this option will mean that
none of these shortened citations will appear automatically.  For more
fine-grained control of individual citations you'll probably want to
use the \cmd{citereset} command, allied possibly with the
\textsf{biblatex}\ \texttt{citereset} option, on which see
\textsf{biblatex.pdf} �3.1.2.1.

%\enlargethispage{\baselineskip}

\mylittlespace Kenneth Pearce \colmarginpar{\texttt{shorthandfull}}
has suggested that, in some fields of study, a list of shorthands
providing full bibliographical information may replace the list of
references itself.  This option, which must be used in tandem with
\texttt{cmslos=false}, prints this full information in the list of
shorthands, though of course you should remember that any .bib entry
not containing a \textsf{shorthand} field won't appear in such a list.
Please see the documentation of the \textsf{shorthand} field in
section~\ref{sec:fields:authdate} above for information on further
options available to you for presenting and formatting the list of
shorthands.

\mylittlespace This \mymarginpar{\texttt{strict}} still-experimental
option attempts to follow the \emph{Manual}'s recommendations (16.57)
for formatting footnotes on the page, using no rule between them and
the main text unless there is a run-on note, in which case a short
rule intervenes to emphasize this continuation.  I haven't tested this
code very thoroughly, and it's possible that frequent use of floats
might interfere with it.  Let me know if it causes problems.

\subsubsection{Style Options -- Entry}
\label{sec:authentryopts}

These options are settable on a per-entry basis in the
\textsf{options} field; both relate to the presentation of dates in
citations and the list of references.

\mylittlespace The \colmarginpar{\texttt{cmsdate}} \emph{Manual}
outlines a series of options for entries with more than one date
(17.124--27).  All of these possibilities are available in
\textsf{biblatex-chicago} using the \texttt{cmsdate} entry option.  It
has 4 possible states relevant to this problem, alongside a fifth
which I discuss below.  An example should make this clearer.  Let us
assume that an entry presents a reprinted edition of a work by Smith,
first published in 1926 (the \textsf{origdate}) and reprinted in 1985
(the \textsf{date}):

\begin{description}
\item[\qquad \textbf{off}:] This is the default.  The citation will
  look like (Smith 1985).
\item[\qquad \textbf{on}:] The citation will look like (Smith 1926).
\item[\qquad \textbf{new}:] The citation will look like (Smith
  1926/1985).
\item[\qquad \textbf{old}:] The citation will look like (Smith [1926]
  1985).
\end{description}

As I explained in detail above in section~\ref{sec:fields:authdate},
s.v.\ \enquote{\textbf{date},}\ because \textsf{biblatex's} sorting
algorithms and automatic creation of the \textsf{extrayear} field
refer by default to the \textsf{date} before the \textsf{origdate}
when both are present, there may be situations when you need to have
the \emph{earlier} year in the \textsf{date} field, and the later one
in \textsf{origdate}, e.g., if you have another reprinted work by the
same author originally printed in the same year.
\textsf{Biblatex-chicago-authordate} will automatically detect this
switch, and given the same reprinted work as above, the results will
be as follows:

\begin{description}
\item[\qquad \textbf{off}:] This is the default.  The citation will
  look like (Smith 1926a).
\item[\qquad \textbf{on}:] The citation will look like (Smith 1926a).
\item[\qquad \textbf{new}:] The citation will look like (Smith
  1926a/1985).
\item[\qquad \textbf{old}:] The citation will look like (Smith [1926a]
  1985).
\end{description}

If, \mymarginpar{\texttt{switchdates}} for any reason, simply
switching the \textsf{date} and the \textsf{origdate} isn't possible
in a given entry, then you can put \texttt{switchdates} in the
\textsf{options} field to achieve the same result.  Please take a look
at the full documentation of the \textbf{date} field to which I
referred just above, and also at \textsf{cms-dates-sample.pdf} and
\textsf{dates-test.bib} for examples of how all this works.

\mylittlespace The \colmarginpar{New!} \emph{Manual} specifies that
\enquote{a list of works cited need not list newspaper items if these
  have been documented in the text} (17.191).  This will apply mainly
to \textsf{article} and \textsf{review} entries with
\textsf{entrysubtype} \texttt{magazine}, and involves a parenthetical
citation giving the \textsf{journaltitle} and then the full
\textsf{date}, not just the year, with any other relevant identifying
information incorporated into running text.  In order to facilitate
this, I have added a further switch to the \texttt{cmsdate} option
\colmarginpar{\mycolor{\texttt{cmsdate=full}}} ---
\mycolor{\textbf{full}} --- which \emph{only} affects the presentation
of citations, and causes the printing of the full date specification
there.  You can use the standard \textsf{biblatex} \texttt{skipbib}
option to keep such entries from appearing in the list of references,
and you may, if your .bib entry is a complete one, also need
\texttt{useauthor=false} in order to ensure that the
\textsf{journaltitle} appears in the citations rather than the
\textsf{author}.

\mylittlespace As a final note, I should point out that the code in
\textsf{chicago-authordate.cbx} allows \texttt{cmsdate} to be used in
the document preamble as a general setting.  This leads to a world of
pain, so I very strongly advise against it, though I'm leaving it in
for testing purposes.

\subsection{General Usage Hints}
\label{sec:hints:auth}

\subsubsection{Loading the Style}
\label{sec:loading:auth}

With the addition of the author-date style to the package, I have
provided two keys for choosing which style to load, \texttt{notes} and
\texttt{authordate}, one of which you put in the options to the
\cmd{usepackage} command.  With early versions of
\textsf{biblatex-chicago}, the standard way of loading the package was
via a call to \textsf{biblatex}, e.g.:
\begin{quote}
  \cmd{usepackage[style=chicago-authordate,strict,backend=biber,\%\\
    babel=other,bibencoding=inputenc]\{biblatex\}}
\end{quote}
Now, the default way to load the style, and one that will in the
vast majority of standard cases produce the same results as the old
invocation, will look like this:
\begin{quote}
  \cmd{usepackage[authordate,strict,backend=biber,babel=other,\%\\
    bibencoding=inputenc]\{biblatex-chicago\}}
\end{quote}

If you read through \textsf{biblatex-chicago.sty}, you'll see that it
sets a number of \textsf{biblatex} options aimed at following the
Chicago specification, as well as setting a few formatting variables
intended as reasonable defaults (see section~\ref{sec:preset:authdate},
above).  Some parts of this specification, however, are plainly more
\enquote{suggested} than \enquote{required,} and indeed many
publishers, while adopting the main skeleton of the Chicago style in
citations, nonetheless maintain their own house styles to which the
defaults I have provided do not conform.

\mylittlespace If you only need to change one or two parameters, this
can easily be done by putting different options in the call to
\textsf{biblatex-chicago} or redefining other formatting variables in
the preamble, thereby overriding the package defaults.  If, however,
you wish more substantially to alter the output of the package,
perhaps to use it as a base for constructing another style altogether,
then you may want to revert to the old style of invocation above.
You'll lose all the definitions in \textsf{biblatex-chicago.sty},
including those to which I've already alluded and also the code that
sets the note number in-line rather than superscript in endnotes or
footnotes.  Also in this file is the code that calls
\textsf{cms-american.lbx}, which means that you'll lose all the
Chicago-specific bibstrings I've defined unless you provide, in your
preamble, a \cmd{DeclareLanguageMapping} command adapted for your
setup, on which see section~\ref{sec:international} below and also
��~4.9.1 and 4.11.7 in Lehman's \textsf{biblatex.pdf}.

\mylittlespace What you \emph{will not} lose is the ability to call
the package options \texttt{annotation, strict, cmslos=false} and
\texttt{noibid} (section~\ref{sec:authuseropts}, above), in case these
continue to be useful to you when constructing your own modifications.
There's very little code, therefore, actually in
\textsf{biblatex-chicago.sty}, but I hope that even this minimal
separation will make the package somewhat more adaptable.  Any
suggestions on this score are, of course, welcome.

\subsubsection{Other Hints}
\label{sec:otherhints:auth}

Starting with \textsf{biblatex} version 1.5, in order to adhere to the
author-date specification you will need to use \textsf{Biber} to
process your .bib files, as \textsc{Bib}\TeX\ (and its more recent
variants) will no longer provide all the required features.  The
previous release of \textsf{Biber} (0.9.5), however, contained bugs
that made it tricky to use with \textsf{biblatex-chicago}.  These bugs
have been addressed in 0.9.6, so I recommend you upgrade to it and to
the latest \textsf{biblatex} (1.7), which are designed to work
together.  This document assumes that you are using \textsf{Biber}; if
you wish to continue using \textsc{Bib}\TeX\ then you need
\textsf{biblatex} version 1.4c and, if you have any problems with the
current release, possibly \textsf{biblatex-chicago} 0.9.7a.

\mylittlespace If your .bib file contains a large number of entries
with more than three authors, then you may run into some limitations
of the \textsf{biblatex-chicago} code.  The default settings in the
package are \texttt{maxnames=3,minnames=1} in citations and
\texttt{max\-bibnames=10,minbibnames=7} in the list of references.  In
practice, this means that an entry like hlatky:hrt, with 5 authors,
will present all of them in the list of references but will truncate
to one in citations, like so: (Hlatky et al. 2002).  For the vast
majority of circumstances, these settings are exactly right for the
Chicago author-date specification.  However, if \enquote{a reference
  list includes another work \emph{of the same date} that would also
  be abbreviated as [\enquote{Hlatky et al.}] but whose coauthors are
  different persons or listed in a different order, the text citations
  must distinguish between them} (16.118).  The new
(\textsf{Biber}-only) \textsf{biblatex} option \texttt{uniquelist},
set for you in \textsf{biblatex-chicago.sty}, will automatically
handle many of these situations for you, but it is as well to
understand that it does so by temporarily suspending the limits,
listed above, on how many names to print in a citation.  Without
\texttt{uniquelist}, \textsf{biblatex} would present such a work as,
e.g., (Hlatky et al. 2002b), while hlatky:hrt would be (Hlatky et
al. 2002a).  This does distinguish between them, but inaccurately, as
it suggests that the two different author lists are exactly the same.
With \texttt{uniquelist}, the two citations might look like (Hlatky,
Boothroyd et al.\ 2002) and (Hlatky, Smith et al.\ 2002), which is
what the specification requires.

\mylittlespace If, however, the distinguishing name occurs further
down the author list --- in fourth or fifth position in our examples
--- then the default settings would produce citations with all 4 or 5
names printed, which can become awkwardly long.  In such a situation,
you can provide \textsf{shortauthor} fields that look like this:
\{\{Hlatky et al., \textbackslash mkbibquote\{Quality of Life,\}\}\}
and \{\{Hlatky et al., \textbackslash mkbibquote\{Depressive
Symptoms,\}\}\}, using a shortened title to distinguish the
references.  This would produce (Hlatky et al., \enquote{Quality of
  Life,} 2002) and (Hlatky et al., \enquote{Depressive Symptoms,}
2002), again as the spec requires.  There is, unfortunately, no
simpler way that I know of to deal with this situation.

\mylittlespace One useful rule, when you are having difficulty
creating a .bib entry, is to ask yourself whether all the information
you are providing is strictly necessary.  The Chicago specification is
a very full one, but the \emph{Manual} is actually, in many
circumstances, fairly relaxed about how much of the data from a work's
title page you need to fit into a reference.  Authors of introductions
and afterwords, multiple publishers in different countries, the real
names of authors more commonly known under pseudonyms, all of these
are candidates for exclusion if you aren't making specific reference
to them, and if you judge that their inclusion won't be of particular
interest to your readers.  Of course, any data that may be of such
interest, and especially any needed to identify and track down a
reference, has to be present, but sometimes it pays to step back and
reevaluate how much information you're providing.  I've tried to make
\textsf{biblatex-chicago} robust enough to handle the most complex,
data-rich citations, but there may be instances where you can save
yourself some typing by keeping it simple.

\enlargethispage{\baselineskip}

\mylittlespace Scot Becker has pointed out to me that the inverse
problem not only exists but may well become increasingly common, to
wit, .bib database entries generated by bibliographic managers which
helpfully provide as much information as is available, including
fields that users may well wish not to have printed (ISBN, URL, DOI,
\textsf{pagetotal}, inter alia).  The standard \textsf{biblatex}
styles contain a series of options, detailed in \textsf{biblatex.pdf}
�3.1.2.2, for controlling the printing of some of these fields, and
with this release I have implemented the ones that are relevant to
\textsf{biblatex-chicago}, along with a couple that Scot requested and
that may be of more general usefulness.  There is also a general
option to excise with one command all the fields under consideration
-- please see section~\ref{sec:authpreset} above.

\mylittlespace Finally, allow me to reiterate what Philipp Lehman says
in \textsf{biblatex.pdf}, to wit, use \textsf{bibtex8}, rather than
standard \textsc{Bib}\TeX, and avoid the cryptic errors that ensue
when your .bib file gets to a certain size.

\section{Internationalization}
\label{sec:international}

Several users have requested that, in line with analogous provisions
in other \enquote{American} \textsf{biblatex} styles (e.g.,
\textsf{biblatex-apa} and \textsf{biblatex-mla}), I include facilities
for producing a Chicago-like style in other languages.  I have
supplied three lbx files, \textsf{cms-german.lbx}, its clone
\textsf{cms-ngerman.lbx}, and \textsf{cms-french.lbx}, in at least
partial fulfillment of this request.  For this release, Baldur
Kristinsson has very kindly provided
\mycolor{\textsf{cms-icelandic.lbx}} for speakers of that language,
while H�kon Malmedal has equally kindly provided
\mycolor{\textsf{cms-norsk.lbx}},
\mycolor{\textsf{cms-norwegian.lbx}}, and
\mycolor{\textsf{cms-nynorsk.lbx}}.  I have added
\mycolor{\textsf{cms-british.lbx}} in order to simplify and to improve
the package's handling of non-American typographical conventions in
English.  This means that all --- or at least most --- of the
Chicago-specific bibstrings are now available for documents and
reference apparatuses written in these languages, with, as I intend,
more languages to follow, limited mainly by my finite time and
even-more-finite competence.  (If you would like to provide bibstrings
for a language in which you want to work, or indeed correct
deficiencies in the lbx files contained in the package, please contact
me.)

\mylittlespace Using \mymarginpar{\textbf{babel}} these facilities is
fairly simple.  By default, and this functionality remains the same as
it was in the previous release of \textsf{biblatex-chicago}, calls to
\cmd{DeclareLanguage\-Mapping} in \textsf{biblatex-chicago.sty} will
automatically load the American strings, and also \textsf{biblatex's}
American-style punctuation tracking, when you:
\begin{enumerate}
\item Load \textsf{babel} with \texttt{american} as the main text
  language.
\item Load \textsf{babel} with \texttt{english} as the main text
  language.
\item[] \qquad \emph{or}
\item Do not load \textsf{babel} at all.
\end{enumerate}
(This last is a change from the \textsf{biblatex} defaults --- cp.\
�~3.9.1 in \textsf{biblatex.pdf} --- but it seems to me reasonable, in
an American citation style, to expect this arrangement to work well
for the majority of users.)

\mylittlespace If, \colmarginpar{New!} for whatever reason, you wanted
to use \textsf{biblatex-chicago} but retain British typographical
conventions --- punctuation outside of quotation marks, outer quotes
single rather than double, etc.\ --- then you no longer need to follow
the complicated rules outlined in previous releases of
\textsf{biblatex-chicago}.  Instead, simply load \textsf{babel} with
the \texttt{british} option.

\mylittlespace If you want to use French, German, Icelandic, or
Norwegian strings in the reference apparatus, then you can load
\textsf{babel} with \texttt{french}, \texttt{german},
\texttt{icelandic}, \texttt{ngerman}, \texttt{norsk}, or
\texttt{nynorsk} as the main document language.  You no longer need
any calls to \cmd{DeclareLanguageMapping} in your document preamble,
since \textsf{bib\-latex-chicago.sty} now automatically provides these
if you load the package in the standard way.

\mylittlespace You can also define which bibstrings to use on an
entry-by-entry basis by using the \textsf{hyphenation} field in your
bib file, but you will have to make sure that the Chicago-specific
strings for the given language are loaded using a
\cmd{DeclareLanguageMapping} call in the preamble.  Indeed, if
\texttt{american} isn't the main text language when loading
\textsf{babel}, then in order to have access to those strings you'll
need \cmd{DeclareLanguageMapping\{american\}\{cms-american\}} in your
preamble, as \textsf{biblatex-chicago.sty} won't load it for you.

\mylittlespace Three other hints may be in order here.  Please note,
first, that I haven't altered the standard punctuation procedures used
in any of the other available languages, so commas and full stops will
appear outside of quotation marks, and those quotation marks
themselves will be language-specific.  If, for whatever reason, you
wish to follow the Chicago specification and move punctuation inside
quotation marks, then you'll need a declaration of this sort in your
preamble:

\begin{quote}
  \cmd{DefineBibliographyExtras\{german\}\{\%}\\
  \hspace*{2em}\cmd{DeclareQuotePunctuation\{.,\}\}}
\end{quote}

Second, depending on the nature of your bibliography database, it will
only rarely be possible to process the same bib file in different
languages and obtain completely satisfactory results.  Fields like
\textsf{note} and \textsf{addendum} will often contain
language-specific information that won't be translated when you switch
languages, so manual intervention will be necessary.  If you suspect
you may have a need to use the same bib file in different languages,
you can minimize the amount of manual intervention required by using
the bibstrings defined either by \textsf{biblatex} or by
\textsf{biblatex-chicago}.  Here, a quick read through
\textsf{notes-test.bib} and/or \textsf{dates-test.bib} should give you
an idea of what is available for this purpose --- see esp.\ the
strings \texttt{by}, \texttt{nodate}, \texttt{newseries},
\texttt{number}, \texttt{numbers}, \texttt{oldseries},
\texttt{pseudonym}, \texttt{reviewof}, \texttt{revisededition}, and
\texttt{volume}, and also section \ref{sec:formatcommands} above,
esp.\ s.v.\ \enquote{\cmd{partedit}.}

\mylittlespace Finally, the French and German bibstrings I have
provided may well break with established bibliographical traditions in
those languages, but my main concern when choosing them was to remain
as close as possible to the quirks of the Chicago specification.  I
have entirely relied on the judgment of the creators of the Icelandic
and Norwegian localizations in those instances.  If you have strong
objections to any of the strings, or indeed to any of my formatting
decisions, please let me know.

\section{One .bib Database, Two Chicago Styles}
\label{sec:twostyles}

I have, when designing this package, attempted to keep at least half
an eye on the possibility that users might want to re-use a .bib
database in documents using the two different Chicago styles.  I have
no idea whether this will even be a common concern, but I thought I
might gather in this section the issues that a hypothetical user might
face.  The two possible conversion vectors are by no means
symmetrical, so I provide two lists, items within the lists appearing
in no particular order.  These may well be incomplete, so any
additions are welcome.

\subsection{Notes -> Author-Date }
\label{sec:conv:notesauth}

This is, I believe, the simpler conversion, as most well-constructed
.bib entries for the notes \& bibliography style will nearly
\enquote{just work} in author-date, but here are a few caveats
nonetheless:

\begin{enumerate}
\item You'll need curly brackets in titles, subtitles, etc., to
  protect capitalization in the sentence-style defaults of
  author-date.
\item You may need to reevaluate your use of shorthands, given that by
  default the author-date style uses them in place of authors rather
  than in place of the whole citation.  The preamble option
  \textsf{cmslos=false} may help, but this may leave your document
  out-of-spec.
\item The potential problem with multiple author lists containing more
  than three names doesn't arise in the notes \& bibliography style,
  so the \textsf{shortauthor} fields in such entries may need
  alteration according to the instructions in
  section~\ref{sec:otherhints:auth} above.
\item Date presentation is relatively simple in notes \& bibliography,
  but you'll need to contemplate the \texttt{cmsdate} options from
  section~\ref{sec:authentryopts} when doing the conversion to
  author-date.
\end{enumerate}

\subsection{Author-date -> Notes}
\label{sec:conv:authnotes}

It is my impression that an author-date .bib database is somewhat
easier to construct in the first instance, but subsequently converting
it to notes \& bibliography is a little more onerous.  Here are some
of the things you may need to address:

\begin{enumerate}
\item If you've decided against using the \cmd{partedit} macro and
  friends from section~\ref{sec:formatting:authdate} above, commands
  not strictly necessary for author-date, you'll need to insert them
  now.
\item In general, you need to be more careful in notes \& bibliography
  about capitalization issues.  Fields which only appear once in
  author-date --- in the list of references --- may appear in both
  long notes and in the bibliography, in different syntactic contexts,
  so a quick perusal of the documentation of the \cmd{autocap} macro
  in section~\ref{sec:formatting:authdate} above may help.
\item You also need to be more careful about the use of abbreviations,
  e.g., in journal names, where the author-date style is more liberal
  in their use than the notes \&\ bibliography style.  (Cf.\ 17.159.)
  The bibstrings mechanism and package options sort much of this out
  automatically, but not all.
\item Although you can get away with the \textsf{article} type for all
  sorts of periodical entries in author-date, you'll need the
  \textsf{review} type for notes \& bibliography.  Any
  well-constructed \textsf{review} entry should work just fine in
  author-date, so this is a one-time conversion.  Please see the
  documentation in section~\ref{sec:entrytypes},
  s.vv. \enquote{article} and \enquote{review,} above.
\item The \textsf{shorttitle} field is used extensively in notes \&
  bibliography to keep short notes short, so you may find that you
  need to add a fair number of these to an author-date database.  In
  general this field is ignored by the latter style, so this, too,
  will be a one-time conversion.
\item You may need to add \textsf{letter} entries if you are citing
  just one letter from a published collection.  See
  section~\ref{sec:entrytypes}, s.v. \enquote{letter,} above.
\item The default shorthand presentation differs from one style to the
  other.  You may need to reconsider how you use this field when
  making the conversion.
\item As I explained above in section~\ref{sec:entryfields}, s.v.\
  \enquote{date,} I have included compatibility code in
  \textsf{biblatex-chicago-notes} for the \texttt{cmsdate} (silently
  ignored) and \texttt{switchdates} options, along with the automatic
  mechanism for reversing \textsf{date} and \textsf{origdate}.  This
  means that you can, in theory, leave all of this alone in your .bib
  file when making the conversion, though I'm retaining the right to
  revise this if the code in question demonstrably interferes with the
  functioning of the notes \&\ bibliography style.
\end{enumerate}

\section{Interaction with Other Packages}
\label{sec:otherpacks}

For \mymarginpar{\textbf{endnotes}} users of the \textsf{endnotes}
package --- or of \textsf{pagenote} --- \textsf{biblatex} 0.9 offers
considerably enhanced functionality.  Please read Lehman's RELEASE
file and the documentation of the \texttt{notetype} option in
\textsf{biblatex.pdf} �~3.1.2.1.

\mylittlespace Another \mymarginpar{\textbf{memoir}} problem I have
found occurs because the \textsf{memoir} class provides its own
commands for the formatting of foot- and end-note marks.  By default,
\textsf{biblatex-chicago} uses superscript numbers in the text, and
in-line numbers in foot- or end-notes, but I have turned this off when
the \textsf{memoir} class is loaded, reasoning that users of that
package may well have their own ideas about such formatting.

\mylittlespace The \mymarginpar{\textbf{ragged2e}} footnote mark code
I've just mentioned also causes problems for the \textsf{rag\-ged2e}
package, but in this case a simple workaround is to load
\textsf{biblatex} \emph{after} you've loaded \textsf{ragged2e} in your
document preamble.

%\enlargethispage{\baselineskip}

\mylittlespace Nick \mymarginpar{\textbf{Xe\LaTeX}} Andrewes alerted
me to problems that appeared when he used the Xe\LaTeX\ engine to
process his files.  These included spurious punctuation after
quotation marks in some situations, and also failures in the automatic
capitalization routines.  Some of these problems disappeared when I
switched to using Lehman's punctuation-tracking code for
\enquote{American} styles, but some remained.  A bug report from
J. P. E.~Harper-Scott suggested a new way of addressing the issue, and
the newest version of Lehman's \textsf{csquotes} package (4.4)
incorporates a full fix.  This, thankfully, doesn't require turning
off any of Xe\LaTeX 's features, and indeed merely involves upgrading
to the latest version of \textsf{csquotes}, which I recommend doing in
any case.  Compatibility with the EU1 encoding is now standard in that
package.

\section{TODO \&\ Known Bugs}
\label{sec:bugs}

This release, belatedly, implements the specifications of the 16th
edition of the \emph{Chicago Manual of Style}.  I am maintaining the
15th-edition styles for those who may need them, but the majority of
my time will now be devoted to improving, and extending the scope of,
the styles devoted to the most recent edition.  I encourage all users
of the notes \&\ bibliography style to upgrade as soon as possible,
and any users of the author-date style who don't require the (until
now) traditional title formatting should do the same.  The next
release will contain an author-date style that allows you to maintain
that title formatting while switching in all other respects to the
latest specification.  If you still have feature requests for the
older specification, I'll do what I can to include them.

\mylittlespace Regardless of which edition you are considering, there
are a number of things I haven't implemented.  The solution in
brown:bre\-mer to multi-part journal articles obviously isn't optimal,
and I should investigate a way of making it simpler.  If the kludge
presented there doesn't appeal, you can always, for the time being,
refer separately to the various parts.  Legal citations are another
thorny issue, and implementing them would involve choosing a
particular documentation scheme (for which there exist at least three
widely-used standards in the US), then providing what would be, it has
seemed to me, an entirely separate \textsf{biblatex} style, bearing
little or no relation to the usual look of Chicago citations.  Indeed,
the \emph{Manual} (17.275) even makes it clear that you should be
using a different reference book if you are presenting work in the
field, so I've thought it prudent to stay clear of those waters so
far.  I have received a request for this feature, however, so when I
have finished the updates for the 16th edition I shall look at it more
closely.  If you have other issues with particular sorts of citation,
I'm of course happy to take them on board.  The \emph{Manual} covers
an enormous range of materials, but if we exclude the legal citations
it seems to me that the available entry types could be pressed into
service to address the vast majority of them.  If this optimism proves
misguided, please let me know.

\mylittlespace Kenneth L. Pearce has reported a bug that appears when
using multiple citation commands inside the \textsf{annotation} field
of annotated bibliographies.  As late as I am with the update to the
16th edition of the \emph{Manual}, I shall attempt to address this in
a future release.  If you run into this problem, he suggests placing
all the citations together in parentheses at the end of the
annotation, though on my machine this doesn't always work too well,
either.

\enlargethispage{\baselineskip}

\mylittlespace Version 1.5 of \textsf{biblatex} revised the way the
package deals with breaking long URLs and DOIs across lines.  The new
code is designed to deal as elegantly as possible with as wide a
variety of cases as possible, but in a few of my test entries it has
caused some line-breaking issues of its own.  Depending on the nature
of your cited sources, it may be useful for you to revert to the
older, pre-1.5 \textsf{biblatex} behavior, something which is easily
done by copying and pasting the old definition of the
\cmd{biburlsetup} command into your document preamble.  If you look in
the preambles of \textsf{cms-notes-sample.tex} or
\textsf{cms-dates-sample.tex}, you can see the redefinition and copy
it from there, just to see whether it helps your situation.
\textsf{Biblatex} 1.7 also now provides several new counters for
tuning the formatting of URLs, and these may serve you better than the
old code.  I have tested them in \textsf{cms-dates-sample.tex} and
they work well.  Cf.\ \textsf{biblatex.pdf}, �~4.10.3.

\mylittlespace The switch to \textsf{Biber} for the author-date
specification means that \textsf{biblatex} now provides considerably
enhanced handling of the various date fields.  I have attempted to
document the relevant changes in \textsf{cms-dates-sample.pdf} and in
the \textbf{date} discussion in section~\ref{sec:fields:authdate},
above, but it's possible the package may need some changes to cope
with all the permutations.  Please let me know if you find something
that looks like a bug.

\mylittlespace Recent versions of \textsf{biblatex} have introduced
some new entry types for citing multi-volume works.  These are largely
aimed at those already using \textsf{Biber}, which provides
much-improved functionality for the \textsf{crossref} field compared
to standard \textsc{Bib}\TeX\ or \textsf{bibtex8}.  Depending upon the
time required to implement the changes for the 16th edition, these
types may make it into the package for the next major release, or they
may have to wait until the one after.

\mylittlespace Roger Hart, Pierric Sans, and a number of other users
have reported a bug in the formatting of title fields.  This, as far
as I can tell, has to do with the interaction between
\cmd{MakeSentenceCase} and certain characters at the start of the
title, particularly Unicode ones.  It may help for the moment to put
an empty set of curly braces \{\}\ at the start of the field, but I
shall look into this further.

\mylittlespace This release fixes the other formatting errors of which
I am aware, though users writing in French should be aware of problems
with the \cmd{partedit} command in section~\ref{sec:formatcommands}
above.  There also remain the larger issues I've discussed throughout
this documentation, which mainly represent my inability to make all of
\textsf{biblatex-chicago's} formatting functions transparent for the
user, but thankfully Lehman's superb punctuation-tracking code has
preemptively fixed a great many small errors, some of which I hadn't
even noticed before I began testing that functionality.  That there
are other micro-bugs seems certain --- if you report them I'll do my
best to fix them.

\mylittlespace I haven't looked closely at the standard
\textsc{Bib}\TeX\ style by Glenn Paulley, contained in
the \textsf{chicago} package on CTAN, which implements the
author-date specification from the 13th edition of the \emph{Manual}.
If anyone is still using the style, and requires some compatibility
code for it, let me know, and I'll look into it.

\section{Revision History}
\label{sec:history}

\textbf{0.9.9g: Released \today}
\begin{itemize}
\item No changes to the 15th-edition styles.
\end{itemize}

\textbf{0.9.9f: Released August 15. 2014}
\begin{itemize}
\item Small changes to both styles so that they compile with the
  latest \textsf{biblatex} (2.9a).  No other fixes.
\end{itemize}

\textbf{0.9.9e: Released January 29, 2014}
\begin{itemize}
\item No changes to the 15th-edition styles.
\end{itemize}

\textbf{0.9.9d: Released October 30, 2013}
\begin{itemize}
\item I have made just enough changes to the styles to ensure they
  compile the test files correctly, but I am marking them as
  \enquote{strongly deprecated,} and encourage all users to upgrade as
  soon as is practicable to the 16th-edition styles.  In the next
  release, I shall mark them as \enquote{obsolete,} and they will
  receive no further updates.

\end{itemize}

\textbf{0.9.9c: Released March 15, 2013}
\begin{itemize}
\item No changes to either style, but I am marking them as
  \enquote{deprecated,} and encourage all users to upgrade as soon as
  is practicable to the 16th-edition styles.
\end{itemize}

\textbf{0.9.9b: Released December 6, 2012}
\begin{itemize}
\item I have updated calls to \cmd{DeclareLabelname} and
  \cmd{DeclareLabelyear} in .cbx files so that the package works
  correctly with the most recent version (2.4) of \textsf{biblatex}.
\item I am grateful to Baldur Kristinsson for providing an Icelandic
  localization file for \textsf{biblatex-chicago}, called
  \mycolor{\textsf{cms-icelandic.lbx}}.  You'll see if you look
  through it that it is still something of a work in progress, but it
  should cover most needs in that language very well.  If you would
  like to fill in some of the gaps please let me know.
\item I am also grateful to H�kon Malmedal for providing Norwegian
  localizations for \textsf{biblatex-chicago}, contained in the files
  \mycolor{\textsf{cms-norsk.lbx}},
  \mycolor{\textsf{cms-norwe\-gian.lbx}}, and
  \mycolor{\textsf{cms-nynorsk.lbx}}.
\item I have added a new British localization
  (\mycolor{\textsf{cms-british.lbx}}) that should make it much
  simpler for users to produce documents adhering to that tradition.
  For further details on the usage of all these localizations please
  see section~\ref{sec:international}, above.
\end{itemize}

\textbf{0.9.9a: Released July 30, 2012}
\begin{itemize}
\item I have made a few changes to \textsf{biblatex-chicago.sty} to
  allow the package to work with the latest version (2.0) of
  \textsf{biblatex}.  In all other respects this release is identical
  to 0.9.9.  If you do use the package with \textsf{biblatex} 2.0,
  please let me know if there are issues I need to address.  Thanks to
  Charles Schaum for alerting me to some of them.
\end{itemize}

\textbf{0.9.9: Released July 5, 2012}

\mylittlespace \label{deprec:obsol} This release, for the first time,
provides style files implementing the specifications of the 16th
edition of \emph{The Chicago Manual of Style}.  As I am continuing to
maintain the older files, here follows a list of changes to the
15th-edition styles since the last release:
\begin{itemize}
\item To continue using the 15th-edition styles, for whatever reason,
  please remember to specify either \texttt{notes15} or
  \texttt{authordate15} when loading \textsf{biblatex-chicago} in your
  preamble.
\item For reprinted books, you can now present more detailed
  information about the original edition using the new
  \mycolor{\textbf{origlocation}} and \mycolor{\textbf{origpublisher}}
  fields.  You can also use this field in \textsf{letter} or
  \textsf{misc} (with \textsf{entrysubtype}) entries to give the place
  where a published or unpublished letter was written.  These uses
  apply to both styles.
\item Thanks to a patch sent by Kazuo Teramoto, you can now take
  advantage of \textsf{biblatex's} facilities for citing
  \mycolor{\textbf{eprint}} resources.  There is also a new
  \mycolor{\texttt{eprint}} option, set to \texttt{true} by default,
  which controls the printing of this field in the author-date style.
  You can set the option both in the preamble and in the
  \textsf{options} field of individual entries.  The field will always
  print in \textbf{online} entries.
\item I have added a new citation command,
  \mycolor{\cmd{citejournal}}, to the notes \&\ bibliography style to
  allow you to present journal articles using an alternative short
  note form, which may be a clearer form of reference in certain
  circumstances.  Such short notes will present the name of the
  \textsf{author}, the \textsf{journaltitle}, and the \textsf{volume}
  number.
\item I have included a very slightly modified version of the standard
  \textsf{biblatex} \cmd{citeauthor} command, which may be useful for
  references to works from classical antiquity.
\item I have added a new \texttt{cmsdate=\mycolor{full}} switch to the
  author-date style, which only affects citations in the text, and
  means that a full date specification will appear there, rather than
  just the year.  If you follow the \emph{Manual's} recommendations
  concerning newspaper and magazine articles only appearing in running
  text and not in the reference list, this option will help.
\item I have provided a new option, \mycolor{\texttt{headline}}, which
  turns off the automatic transformations that produce sentence-style
  capitalization in the title fields of the author-date style.  If you
  set this option, the word case in your title fields will not be
  changed in any way, that is, this doesn't automatically transform
  your titles into headline-style, but rather allows the .bib file to
  determine capitalization.
\item Following a request by Kenneth Pearce, I have added new
  facilities for presenting \textbf{shorthands} in the author-date
  style.  There are two new \texttt{bibenvi\-ronments} which you can set
  using the \texttt{env} option to the \cmd{printshorthands} command:
  \mycolor{\texttt{losnotes}} formats the list of shorthands so that
  it can be presented in a footnote, while
  \mycolor{\texttt{losendnotes}} does the same for endnotes.  There is
  also a new preamble option, \mycolor{\texttt{shorthandfull}}, which
  prints the full bibliographical information of each entry inside the
  list of shorthands, allowing such a list effectively to replace a
  list of references.  You need to set the \texttt{cmslos=false}
  option as well in order for this to work.
\item Thanks to a coding suggestion from Gildas Hamel, I have
  redefined the \cmd{bibnamedash} in \textsf{biblatex-chicago.sty},
  which should now by default look a little better in a wider variety
  of fonts.
\item At the request of Baldur Kristinsson, I have added
  \cmd{DeclareLanguageMap\-ping} commands to
  \textsf{biblatex-chicago.sty} for all the languages
  \textsf{biblatex-chicago} currently provides.  If you load the style
  in the standard way, you no longer need to provide these mappings
  manually yourself.
\item I have improved the date handling in both styles, particularly
  with regard to date ranges.
\end{itemize}

\textbf{0.9.8d: Released November 15, 2011}
\begin{itemize}
\item Some minor fixes to both styles for compatibility with
  \textsf{biblatex} 1.7.
\item Kenneth Pearce found an error in the formatting of
  \textsf{bookinbook} titles in the author-date style's list of
  shorthands.  This should work properly now.
\item Jonathan Robinson spotted some inconsistencies in the way the
  notes \&\ bibliography style interacts with the \textsf{hyperref}
  package.  Following his suggestion, short notes now point to long
  notes when the latter are available, but to bibliography entries
  instead when you have set the \texttt{short} option.
\end{itemize}

%\enlargethispage{\baselineskip}

\textbf{0.9.8c: Released October 12, 2011}
\begin{itemize}
\item Emil Salim pointed out some rather basic errors in the
  presentation of \textsf{inproceedings} and \textsf{proceedings}
  entries, errors that have been present from the first release of the
  style(s).  These should now, belatedly, have been put right.
\item Minor improvements to coding and documentation.
\end{itemize}

\textbf{0.9.8b: Released September 29, 2011}
\begin{itemize}
\item Bad Dates: Christian Boesch alerted me to some date-formatting
  errors produced when using the styles with the \texttt{german}
  option to \textsf{babel}.  A little further investigation revealed
  similar problems with \texttt{french}, and before long it became
  clear that date handling in \textsf{biblatex-chicago} was generally,
  and significantly, sub-optimal.  The whole system should now be more
  robust and more accurate.
\item The new date-handling code shouldn't require any changes to your
  .bib files, but users of the author-date style may want to have a
  look at the documentation of the \textsf{letter} and \textsf{misc}
  entry types, and of the four date fields, for some information about
  how the changes could simplify the creation of their databases.
\item Various other minor improvements.
\end{itemize}

\textbf{0.9.8a: Released September 21, 2011}
\begin{itemize}
\item Fixed a series of unsightly errors in the author-date style,
  discovered while working on the pending update to the 16th edition.
\item Fixed bugs uncovered in both the author-date and the notes \&\
  bibliography styles thanks to Charles Schaum's adventurous use of
  the \textsf{origyear} field.
\item Added two new bibstrings to the cms-*.lbx files to fix potential
  bugs in some of the audiovisual entry types.
\end{itemize}

\textbf{0.9.8: Released August 31, 2011}

\mylittlespace Obsolete and Deprecated Features:
\begin{itemize}
\item Starting with \textsf{biblatex} version 1.5, in order to adhere
  to the author-date specification you will need to use \textsf{Biber}
  to process your .bib files, as \textsc{Bib}\TeX\ (and its more
  recent variants) will no longer provide all the required features.
  Unfortunately, however, the current release of \textsf{Biber}
  (0.9.5) contains bugs that make it tricky to use with
  \textsf{biblatex-chicago}.  These bugs have been addressed in 0.9.6
  beta, which is available for various operating systems in the
  \texttt{development} subdirectory of your SourceForge mirror, e.g.,
  \href{http://www.mirrorservice.org/sites/download.sourceforge.net/pub/sourceforge/b/project/bi/biblatex-biber/biblatex-biber/development/binaries/}{UK
    mirror}.  (If, by the time you read this, \textsf{Biber} 0.9.6 has
  already been released, then so much the better.)  Please see the
  start of \textsf{cms-dates-sample.pdf} for more details.
\item The switch to \textsf{Biber} for the author-date specification
  means that \textsf{biblatex} now provides considerably enhanced
  handling of the various date fields.  I have attempted to document
  the relevant changes in \textsf{cms-dates-sample.pdf} and in the
  \textbf{date} discussion in section~\ref{sec:fields:authdate},
  above, but in my testing the only alterations I've so far had to
  make to my .bib files involve adhering more closely to the
  instructions for specifying date ranges.  \textsf{Biber} doesn't
  like \{\texttt{1968/75}\}, and will ignore it.  Either use
  \{\texttt{1968/1975}\} or use \{\texttt{1968-{}-75}\} in the
  \textsf{year} field.
\item In the notes \&\ bibliography style, and mainly in
  \textsf{article}, \textsf{letter}, \textsf{misc}, and
  \textsf{review} entries, previous releases of
  \textsf{biblatex-chicago} recommended using the \cmd{isdot} macro
  when you needed both to define a field and not have it appear in the
  printed output.  This mechanism no longer works in \textsf{biblatex}
  1.6, and while addressing the problem I realized that relying on it
  covered over some inconsistencies and bugs in my code, so from this
  release forward you will need to modify your .bib and .tex files to
  use other, more standard mechanisms to achieve the same ends, in
  particular the \cmd{headlesscite} commands and declaring
  \texttt{useauthor=false} in the \textsf{options} field.  Please
  consult the documentation in section~\ref{sec:formatcommands}, s.v.\
  \enquote{\cmd{isdot},} for a list of example entries where you can
  see these changes at work.
\end{itemize}

Other New Features:

\begin{itemize}
\item Fixed the \cmd{smartcite} citation command in, and added a
  \cmd{smartcites} command to, \textsf{chicago-notes.cbx}, so that the
  notes \& bibliography style no longer prints parentheses around
  citations produced using \cmd{autocite(s)} commands inside
  \cmd{footnote} commands.  Many thanks to Louis-Dominique Dubeau for
  pointing out this error.
\item Rembrandt Wolpert and Aaron Lambert pointed out an issue with a
  command (\cmd{lbx@fromlang}) that \textsf{biblatex} no longer
  defines, and Charles Schaum very kindly suggested a temporary
  workaround in a newsgroup post, a workaround that should no longer
  be necessary.
\item Version 1.6 of \textsf{biblatex} no longer allows you to
  redefine the \texttt{minnames} and \texttt{maxnames} options in the
  \cmd{printbibliography} command, so I've defined
  \texttt{minbibnames} and \texttt{maxbibnames} in
  \textsf{biblatex-chicago.sty}, instead.  These parameters have been
  available since version 1.1, so this is now the earliest version of
  \textsf{biblatex} that will work with the Chicago styles.  Of
  course, if the (Chicago-recommended) values of these options don't
  suit your needs, you can redefine them in your document preamble.
\end{itemize}

\textbf{0.9.7a: Released March 17, 2011}
\begin{itemize}
\item Added \cmd{smartcite} command to \textsf{chicago-notes.cbx} so
  that the notes \&\ bibliography style will work with
  \textsf{biblatex} 1.3.
\item Added bibstrings \texttt{byconductor} and \texttt{cbyconductor}
  to the .lbx files, mistakenly omitted in version 0.9.7.
\item Minor fixes to the docs.
\end{itemize}

\textbf{0.9.7: Released February 15, 2011}

\mylittlespace Obsolete and Deprecated Features:
\begin{itemize}
\item The \textbf{customa} and \textbf{customb} entry types are now
  obsolete.  Any such entries will be ignored.  Please change any that
  remain to \textbf{letter} and \textbf{bookinbook}, respectively.
\item If you still have any \textbf{customc} entries containing
  introductions, prefaces, or the like, please change them to
  \textbf{suppbook}.  I have recycled \textsf{customc} for another
  purpose, on which see below.
\end{itemize}

Other New Features:

\begin{itemize}
\item At the request of Johan Nordstrom, I have added three new
  audiovisual entry types to both styles, \textbf{audio},
  \textbf{music}, and \textbf{video}.  The documentation of
  \textsf{audio} in sections~ \ref{sec:entrytypes} and
  \ref{sec:types:authdate} above contains an overview of the three,
  and the details for each type are to be found under their individual
  headings.
\item I have transformed the \textbf{customc} entry type to enable
  alphabetized cross-references --- the \enquote{c} is meant to be
  mnemonic --- to other, separate entries in a reference list or
  bibliography.  In particular, this facilitates cross-references to
  other names in a list, rather than to other works.  In author-date,
  in a procedure recommended by the \emph{Manual}, this now allows you
  to expand shorthands inside the reference list rather than in a list
  of shorthands.  In both styles, you can now provide a pointer to the
  main entry if a reader is looking an author up under, e.g., a
  pseudonym or other alternative name.
\item I have introduced the \textbf{userc} field, intended to simplify
  the printing of the cross-references provided by \textsf{customc}
  entries.  The standard \cmd{nocite} command works as well, but the
  additional mechanism may be more convenient in some circumstances.
\item You can now provide an \textbf{eventdate} in \textsf{music}
  entries to identify, e.g., a particular recording session.  It will
  be printed just after the \textsf{title}.
\item In the notes \&\ bibliography style, I have now implemented the
  \textbf{shorthandintro} field, which allows you to change the string
  introducing a shorthand in the first, long note.  It works just as
  it does in the standard \textsf{biblatex} styles.
\item At the request of Scot Becker, I have added six new
  field-exclusion options to both styles, all of which can be set both
  in the document preamble and/or in the \textsf{options} field of
  individual .bib entries.  Three of these --- \texttt{doi},
  \texttt{isbn}, and \texttt{url} --- are standard \textsf{biblatex}
  options, the others --- \texttt{bookpages}, \texttt{includeall}, and
  \texttt{numbermonth} --- are \textsf{chicago}-specific.  See the
  docs in sections~\ref{sec:chicpreset} and \ref{sec:authpreset},
  above.
\item At the request of Charles Schaum, I've added the
  \texttt{juniorcomma} option to both styles, which can be set in the
  document preamble and/or in the \textsf{options} field of individual
  entries.  It allows you to get the traditional comma between a
  surname and \enquote{Jr.} or \enquote{Sr.}
\item Fixed an old inaccuracy in the presentation of \enquote{Jr.} and
  \enquote{Sr.,} so that they now appear at the end of names printed
  surname first in bibliographies and reference lists.
\item Thanks to Andrew Goldstone, I fixed some old inaccuracies in the
  syntax of shortened notes and bibliography entries presenting
  multiple contributions to one multi-author (or single-author)
  volume.
\item I've altered the directory structure of the archive containing
  this release.  Files were multiplying, and look set to multiply
  still further, so I've copied the structure used by Lehman for
  \textsf{biblatex} itself.
\item Fixed an old bug, which I'd guess was triggered quite rarely, in
  the formatting of publication information in long notes.
\item Fixed another bug in author-date where the colon separating
  titles and subtitles was in the wrong font.  The \textsf{biblatex}
  \texttt{punctfont} option solved this.
\item Fixed a punctuation bug in \textsf{InReference} entries in the
  notes \&\ bibliography style.  Also fixed \textsf{title}
  presentation in \textsf{Reference} entries in author-date.
\item Fixed some inaccuracies in the tests establishing priority
  between \textsf{date} and \textsf{origdate} fields.  These arose
  when date ranges were involved, and it's possible I haven't yet
  addressed all possible permutations of the problem.
\item Added several new bibstrings to the \textsf{cms-*.lbx} files for
  the new audiovisual entry types.  This means that the
  \textsf{editortype} fields can now be set to \texttt{director},
  \texttt{producer}, or \texttt{conductor}, depending on your needs.
  You can also set the fields to \texttt{none}, which eliminates all
  identifying strings, and which is useful for identifying performers
  of various sorts.
\item Minor improvements to documentation.
\end{itemize}

\textbf{0.9.5a: Released September 7, 2010}
\begin{itemize}
\item Quick fix for an elementary and show-stopping mistake in
  \textsf{biblatex-chica\-go.sty}, a mistake disguised if you load
  \textsf{csquotes}, which I do in all my test files.  Mea culpa.
  Many thanks indeed to Israel Jacques and Emil Salim for pointing
  this out to me.
\end{itemize}

\textbf{0.9.5: Released September 3, 2010}

\mylittlespace Obsolete and Deprecated Features:
\begin{itemize}
\item All the custom entry types --- \textbf{customa},
  \textbf{customb}, and \textbf{customc} --- are now deprecated.  They
  will still work for the time being, but please be aware that in the
  next major release they will no longer function, at least not as you
  might be expecting.  Please change your .bib files to use
  \textbf{letter} (=\textbf{customa}), \textbf{bookinbook}
  (=\textbf{customb}), and \textbf{suppbook} (=\textbf{customc})
  instead.
\item If by some chance anyone is still using the old \cmd{custpunctc}
  macro, it is now obsolete.  It really shouldn't be needed, but let
  me know if I'm wrong.
\end{itemize}

%\vspace{2\baselineskip}

Other New Features:
\begin{itemize}
\item The Chicago author-date style is now implemented in the
  package, and is fully documented in section~\ref{sec:authdate},
  above.
\item The default way of loading the style(s) has slightly changed.
  You should put either \texttt{notes} or \texttt{authordate} in the
  options to \textsf{biblatex-chicago}, e.g.:
  \begin{quote}
    \cmd{usepackage[authordate,more options%
     \,\ldots]\{biblatex-chicago\}}
  \end{quote}
\item With the addition of the second Chicago style, I have thought it
  appropriate to alter both the name of the package and the names of
  the files it contains.  The package is now \textsf{biblatex-chicago}
  instead of \textsf{biblatex-chicago-notes-df}, and the following
  files have been renamed:
  \begin{itemize}
  \item \textsf{chicago-notes-df.cbx} is now \textsf{chicago-notes.cbx}
  \item \textsf{chicago-notes-df.bbx} is now \textsf{chicago-notes.bbx}
  \item \textsf{sample.tex} is now \textsf{cms-notes-sample.tex}
  \item \textsf{sample.pdf} is now \textsf{cms-notes-sample.pdf}
  \item \textsf{chicago-test.bib} is now \textsf{notes-test.bib}
  \item \textsf{biblatex-chicago-notes-df.pdf} (this file) is now
    \textsf{biblatex-chicago.pdf}
  \end{itemize}
  The following files have been added:
  \begin{itemize}
  \item \textsf{chicago-authordate.cbx}
  \item \textsf{chicago-authordate.bbx}
  \item \textsf{cms-dates-sample.tex}
  \item \textsf{cms-dates-sample.pdf}
  \item \textsf{dates-test.bib}
  \end{itemize}
  The following files have retained their old names:
  \begin{itemize}
  \item \textsf{cms-american.lbx}
  \item \textsf{cms-french.lbx}
  \item \textsf{cms-german.lbx}
  \item \textsf{cms-ngerman.lbx}
  \item \textsf{biblatex-chicago.sty}
  \end{itemize}
\item I have implemented the \textsf{pubstate} field, slightly
  differently yet compatibly in the two styles, to provide a simpler
  mechanism for identifying a reprinted book.  In the author-date
  style, it is highly recommended you use it, as it sorts out some
  complicated formatting questions automatically.  In the notes \&\
  bibliography style it isn't strictly necessary, but may be useful
  anyway and easier to remember than the old system.  See the
  documentation under \textsf{pubstate} in
  sections~\ref{sec:entryfields} and \ref{sec:fields:authdate}, above.
\item Users of \textsf{biblatex-chicago-notes} no longer need a
  \textsf{shortauthor} field in author-less \textsf{manual} entries,
  or in author-less \textsf{article} or \textsf{review} entries with a
  \texttt{maga\-zine} \textsf{entrysubtype}.  The package will now
  automatically take an author for short notes from the
  \textsf{organization} field for \textsf{manual} entries and from the
  \textsf{journaltitle} field for the others.  You can still use a
  \textsf{shortauthor} field if you want, but it's no longer
  necessary.  (This also holds for \textsf{chicago-authordate}.)
\item Date presentation in the \textsf{misc} entry type (with
  \textsf{entrysubtype}) has changed to fix an inaccuracy.  You can
  now use the \textsf{date} and \textsf{origdate} fields to
  distinguish between two sorts of archival source: letters and
  \enquote{letter-like} sources use \textsf{origdate}, interviews and
  other non-letters use \textsf{date}.  The only difference is in how
  the date is printed, so current .bib entries will continue to work
  fine, albeit with minor inaccuracies in the case of non-letter-like
  sources.  See the docs on \textbf{misc} in
  sections~\ref{sec:entrytypes} and \ref{sec:types:authdate}, above.
\item When only one date is presented in a \textsf{patent} entry ---
  either in the \textsf{date} or \textsf{origdate} field --- this will
  now always be used as the filing date.  In
  \textsf{biblatex-chicago-notes}, this makes a change from the
  previous (incorrect) behavior.
\item I have included the option \texttt{dateabbrev=false} in the
  default settings for \textsf{biblatex-chicago-notes}.  This ensures
  that the long month names are printed, as otherwise recent releases
  of \textsf{biblatex} print the abbreviated ones by default.
\item The provision of punctuation in \textsf{entrysubtype}
  \texttt{classical} entries has been improved, allowing the comma to
  appear before certain kinds of location specifiers even when citing
  works by their traditional divisions.  See \emph{Manual} 17.253.
  (This applies to both Chicago styles.)
\item The \textsf{number} field in \textsf{article},
  \textsf{periodical}, and \textsf{review} entries now allows you to
  include a series or range of numbers in the field, with the style
  automatically providing the correct bibstring (singular or plural).
\item I have removed and altered bibstrings in the .lbx files to take
  advantage of the new \cmd{bibsstring} and \cmd{biblstring} commands
  in \textsf{biblatex}, and added one new string
  (\texttt{origpubyear}) needed by
  \textsf{biblatex-chicago-authordate}.
\end{itemize}

\textbf{0.9a: Released March 20, 2010}
\begin{itemize}
\item Quick fixes for compatibility with \textsf{biblatex} 0.9a.
\end{itemize}

\textbf{0.9: Released March 18, 2010}

\mylittlespace Obsolete and Deprecated Features:
\begin{itemize}
\item The \textbf{userd} field is now obsolete.  All information it
  used to hold should be placed in the \textsf{edition} field.
\item The \textbf{origyear} field is now obsolete in
  \textsf{biblatex}.  It has been replaced by \textbf{origdate}, and
  because the latter allows a full date specification, I have been
  able to make the operation of \textsf{customa} (=\,\textsf{letter}),
  \textsf{misc} (with an \textsf{entrysubtype}), and \textsf{patent}
  entries more intuitive.  The RELEASE file contained in this package
  gives the short instructions on how to update your .bib files, and
  you can also consult the documentation of those entry types above.
\item The modified \textsf{csquotes.cfg} file I provided in earlier
  releases is now obsolete, and has been removed from the package.
  Please upgrade to the latest version of \textsf{csquotes} and, if
  you are still using my modified .cfg file, remove it from your \TeX\
  search path, or at the very least excise the code I provided.
\end{itemize}

Other New Features:
\begin{itemize}
\item Added the files \textsf{cms-german.lbx} (with its clone
  \textsf{cms-ngerman.lbx}) and \textsf{cms-french.lbx}, which allow
  the creation of Chicago-like references in those languages.  See
  section \ref{sec:international} above for details on usage.
\item Added the \texttt{annotation} package option to allow the
  creation of annotated bibliographies.  This code is still not
  entirely polished yet, but it is usable.  Please see page
  \pageref{sec:annote} above for instructions and hints.
\item Added \textsf{biblatex's} new \textbf{bookinbook} entry type,
  which currently functions as an alias of the \textsf{customb} type.
  As \textsf{biblatex} now provides standard equivalents for all of
  the custom types I initially found it necessary to provide ---
  \textsf{letter}~= \textsf{customa}, \textsf{bookinbook}~=
  \textsf{customb}, and \textsf{suppbook} \& \textsf{suppcollection}~=
  \textsf{customc} --- it may soon be time to prune out the custom
  types to enhance compatibility with other \textsf{biblatex} styles.
  I shall give plenty of warning before I do so.
\item In line with the new system adopted in \textsf{biblatex} 0.9,
  using the \textsf{editortype} field turns off the usual string
  concatenation mechanisms of the Chicago style.  See Lehman's RELEASE
  file for a discussion of this.
\item I have added support for the new \textsf{editor[a--c]} and
  \textsf{editor[a--c]type} fields, and they work just as in standard
  \textsf{biblatex}, though I'm uncertain how much use they'll get
  from users of the Chicago style.
\item I have added many bibstrings to the .lbx files to help with
  internationalization.  The new ones that you might want to use in
  your .bib files include: \texttt{pseudonym}, \texttt{nodate},
  \texttt{revisededition}, \texttt{numbers}, and \texttt{reviewof}.
  Please see section~\ref{sec:international} for a fuller list.
\end{itemize}

\textbf{0.8.9d: Released February 17, 2010}
\begin{itemize}
\item Chris Sparks and Aaron Lambert both found formatting bugs in the
  0.8.9c code.  I've fixed these bugs, and am releasing this version
  now, the last in the 0.8.9 series.  The next release of
  \textsf{biblatex-chicago-notes-df}, due as soon as possible, will
  contain many more significant changes, including those necessary for
  it to function properly with the recently-released \textsf{biblatex}
  version 0.9.  In the meantime, at least version 0.8.9d should produce
  more accurate output.
\end{itemize}

\textbf{0.8.9c: Released November 4, 2009}
\begin{itemize}
\item Emil Salim noticed that the \emph{ibidem} mechanism wasn't
  working properly, printing the page number after \enquote{Ibid} even
  when the page reference of the preceding citation was identical.
  The fix for this involved setting \texttt{loccittracker=constrict}
  in \textsf{biblatex-chicago.sty}, something you'll have to do
  manually yourself if you're loading the package via a call to
  \textsf{biblatex} rather than to \textsf{biblatex-chicago}.
\item Several users have reported unwanted behavior when repeated
  names in bibliographies are replaced with the \texttt{bibnamedash}.
  This release should fix both when the \texttt{bibnamedash} appears
  and what punctuation follows it.
\end{itemize}

\textbf{0.8.9b: Released September 9, 2009}
\begin{itemize}
\item Fixed a long-standing bug in formatting names in the
  bibliography.  The package now correctly places a comma after the
  reversed name that begins the entry, using \textsf{biblatex's}
  \cmd{revsdnamedelim} command.  Many thanks to Johanna Pink for
  catching my rather egregious error.
\item While fixing some formatting errors that cropped up when using
  the newest version of \textsf{biblatex} (0.8h at time of writing), I
  also spotted some more venerable bugs in the code for using
  shortened cross-references for citing multiple entries in a
  collection of essays or letters.  I believe this now works
  correctly, but please let me know if you discover differently.
\item Joseph Reagle noticed that endnote marks (produced using the
  \textsf{endnotes} package) did not receive the
  same treatment as footnote marks.  I have rectified this, placing
  the code in \textsf{biblatex-chicago.sty} so that you can turn it
  off either by using the old package-loading system or by setting the
  \texttt{footmarkoff} package option when loading
  \textsf{biblatex-chicago}.
\item Updates to Lehman's \textsf{csquotes} package have rendered my
  modifications in \textsf{csquotes.cfg} obsolete.  Please use the
  latest version of \textsf{csquotes} (4.4a at time of writing) and
  ignore my file, which will disappear in a later release.
\item At the request of Will Small, I have included some code, still
  in an alpha state, to allow you to specify, in the bibliography, the
  original publication details of essays which you are citing from
  later reprints (a \emph{Collected Essays} volume, for example).  See
  the documentation above under the \textsf{\mycolor{reprinttitle}}
  field if you would like to test this functionality.
\end{itemize}

%\enlargethispage{-3\baselineskip}

\textbf{0.8.9a: Released July 5, 2009}
\begin{itemize}
\item Slight changes for compatibility with \textsf{biblatex} 0.8e.
  The package still works with 0.8c and 0.8d, as well.
\end{itemize}

\textbf{0.8.9: Released July 2, 2009}

\mylittlespace Obsolete and Deprecated Features:
\begin{itemize}
\item The \textbf{single-letter bibstrings} (\cmd{bibstring\{a\}},
  \cmd{bibstring\{b\}}, etc.) are now obsolete.  You should replace
  any still present in your .bib file with \cmd{autocap} commands ---
  see �~3.8.4 of \textsf{biblatex.pdf}.
\end{itemize}

Other New Features:
\begin{itemize}
\item The default way of loading the package is now with

  \cmd{usepackage[further-options]\{biblatex-chicago\}}

  rather than

  \cmd{usepackage[style=chicago-notes-df,further-options]\{biblatex\}}.

  Please see section~\ref{sec:loading} above for details and hints.
\item Package-specific bibstrings have been removed from the .cbx and
  .bbx files and are now gathered in a new file,
  \textbf{cms-american.lbx}, which changes the way the package
  interacts with \textbf{babel}.  It is now somewhat simpler if you
  want the defaults, but somewhat more complex if you require
  non-standard features.  Please see section~\ref{sec:otherpacks}
  above for more details.
\item Two new entry types have been added: \textbf{artwork} for works
  of visual art excluding photographs, and \textbf{image} for
  photographs.  See the documentation of \textsf{artwork} for how to
  create .bib entries for both types.
\item Added the new bibliography and entry option
  \textbf{usecompiler}, set to \texttt{true} by default.  This
  streamlines the code that finds a name to head an entry
  (\textbf{author -> editor [or namea] -> translator [or nameb] ->
    compiler [namec] -> title}).  The whole system should work more
  consistently now, but do see the \textsf{author} and \textsf{namec}
  documentation for improved notes on how to use it.
\item Added the new bibliography option \textbf{footmarkoff}, to turn
  off the optional in-line (as opposed to superscript) formatting of
  the marks in foot- or endnotes.  You only need this if you load the
  package with the new default \cmd{usepackage\{biblatex-chicago\}};
  users loading it the old way get default \LaTeX\ formatting.
\item At Matthew Lundin's request, I have added the citation command
  \textbf{\textbackslash head\-lesscite}, which works like
  \cmd{headlessfullcite} but allows \textsf{biblatex} to decide
  whether to print the full or the short note.
\item Fully adopted \textsf{biblatex's} system for providing
  end-of-entry punctuation, which should solve some of the bugs users
  have been finding.  See section~\ref{sec:otherhints}, above, and do
  please let me know if inconsistencies remain.
\item Added a modified \textbf{csquotes.cfg} file to address issues
  users were having when using the \textbf{Xe\LaTeX} engine in
  combination with \textsf{biblatex-chicago}.  See
  section~\ref{sec:otherpacks}, above.
\item Added \texttt{natbib} option to allow users of the default setup
  to continue to benefit from \textsf{biblatex's} \textsf{natbib}
  compatibility code.  Thanks to Bennett Helm for pointing out this
  issue.
\item Added a \textbf{shorthandibid} option to allow the printing of
  \emph{ibid.}\ in consecutive references to an entry that contains a
  \textsf{shorthand} field.  Thanks to Chris Sparks for calling my
  attention to this problem.
\item While investigating the preceding, I noticed failures when
  combining the \texttt{short} option with a \textsf{shorthand} field.
  The package now actually does what it has always claimed to do under
  \textbf{shorthand}.
\item Many small bug fixes and improvements to the documentation.
\end{itemize}

To Do:

\begin{itemize}
\item The shorthand vs \emph{ibid.}\ question may need more careful
  addressing in some cross references, and also in relation to the
  \texttt{noibid} package option.
\item Charles Schaum has quite rightly pointed out the inconsistency
  in my naming conventions --- \textsf{biblatex-chicago.sty} as
  opposed to \textsf{chicago-notes-df.cbx}, for example.  I'm going to
  delay a decision on which way to go with this until a later release.
\end{itemize}

\textbf{0.8.5a: Released June 14, 2009}

\begin{itemize}
\item Quick and dirty fixes to bibliography strings to allow
  compatibility with \textsf{biblatex} version 0.8d.  If you are still
  using 0.8c, then I would wait for the next version of
  \textsf{biblatex-chicago-notes-df}, which is due soon.  See README.
\end{itemize}

\textbf{0.8.5: Released January 10, 2009}

\mylittlespace Obsolete and Deprecated Features:

  \begin{itemize}
  \item The \textbf{\textbackslash custpunct} commands are now
    deprecated --- Lehman's \enquote{American} punctuation tracking
    facilities should handle quoted text automatically, assuming you
    remember always to use \textbf{\textbackslash mkbibquote} in your
    database.  If you still need \cmd{custpunct}, please let me know,
    as it may be an error in the style.
  \item With \cmd{custpunct} no longer needed, the toggles activated
    by placing \enquote{\texttt{plain}} in the \textbf{type} or
    \textbf{userb} fields are also deprecated.
  \end{itemize}

Other New Features:

\begin{itemize}
\item At least \textbf{biblatex 0.8b} is now required --- 0.8c works
  fine, as well.
\item I now \emph{strongly recommend} that you use \textbf{babel} with
  \enquote{\texttt{american}} as the main text language.  See
  section~\ref{sec:otherpacks} above for further details.
\item The \textbf{customc} entry type has been revised, allowing you
  to cite any sort of supplementary material using the \textbf{type}
  field instead of relying on toggles in the \textsf{introduction},
  \textsf{afterword}, and \textsf{foreword} fields, though these
  latter still work.  The two new entry types \textbf{suppbook} and
  \textbf{suppcollection} are both aliased to \textsf{customc}, and
  therefore work in exactly the same way.
\item The new entry type \textbf{suppperiodical} is aliased to
  \textbf{review}.
\item The new entry type \textbf{letter} is aliased to
  \textbf{customa}.
\item In \textbf{inreference} entries the \textsf{postnote} field of
  all \cmd{cite} commands is now treated like data in \textsf{lista},
  that is, it will be placed within quotation marks and prefaced with
  the appropriate string.  The only difference is that you can only
  put one such article name in \textsf{postnote}, as it isn't a list
  field.
\item I've set the new \textsf{biblatex} option \texttt{usetranslator}
  to \texttt{true} by default, which means entries will automatically
  be alphabetized by their \textsf{translator} in the absence of an
  \textsf{author} or an \textsf{editor}.
\item A host of small formatting errors were eliminated, nearly all of
  them through adopting Lehman's punctuation tracker.
\item In the main body of this documentation, I've added some
  \mycolor{\textbf{color coding}} to help you more quickly to identify
  entry types and fields that are either new or that have undergone
  significant revision.
\end{itemize}

To Do:

\begin{itemize}
\item Separate out \enquote{options} from the basic citation
  \enquote{style,} using a \LaTeX\ style file.  This is an
  architectural change recommended by Lehman.
\end{itemize}

\textbf{0.8.2.2: Released November 24, 2008}

\begin{itemize}
\item Fixed spurious commas appearing in some bibliography entries,
  spotted by Nick Andrewes.  While investigating this I noticed a more
  general problem with punctuation after italicized titles ending with
  question marks or exclamation points.  This will be addressed in
  forthcoming revisions both of \textsf{biblatex} and of this package.
\item Nick also reported some problems with spurious punctuation in
  the bibliography when using XeLaTeX.  I haven't yet been able to pin
  down the exact cause of these, but if you are using XeLaTeX and are
  having (or have solved) similar problems I'd be interested to hear
  from you.
\end{itemize}

\textbf{0.8.2: Released November 3, 2008}

\begin{itemize}
\item Fixed several formatting glitches between citations in multicite
  commands (spotted by Joseph Reagle) and also after some prenotes. 
\end{itemize}

\textbf{0.8.1: Released October 22, 2008}

\mylittlespace Obsolete and Deprecated Features:

\begin{itemize}
\item The \textbf{origlocation} field is now obsolete, and has been
  replaced by \textbf{lista}.  Please update your .bib files
  accordingly.
\item The single-letter \textbf{\textbackslash bibstring} commands I
  provided in version 0.7 are now deprecated.  In most cases, you'll
  be able to take advantage of the automatic contextual capitalization
  facilities introduced in this release, but if you still need the
  single-letter \cmd{bibstring} functionality then you should switch
  to \cmd{autocap}, as I shall be removing the single-letter
  \texttt{bibstrings} in a future release.  See above under
  \textbf{\textbackslash autocap} for all the details.
\item The \textbf{userd} field is now deprecated, as \textsf{biblatex}
  0.8 allows all forms of data to be included in the \textsf{edition}
  field.  I shall be removing \textsf{userd} in a future release, so
  please update your .bib files as soon as is convenient.
\end{itemize}

Other New Features:

\begin{itemize}
\item Updated the .bbx and .cbx files to work with \textsf{biblatex}
  0.8.  This most recent version of \textsf{biblatex} is now required
  for \textsf{biblatex-chicago-notes-df} to work.
\item Added the \textbf{usera} field, which holds supplemental
  information about a \textsf{journaltitle} in \textsf{article} and
  \textsf{review} entries.  See the documentation of the field for
  details.
\item Added the \textbf{\textbackslash citetitles} multicite command
  to fix a problem with spurious punctuation when multiple titles were
  listed.
\item Added the \textbf{\textbackslash Citetitle} command to help with
  automatic capitalization of titles when they occur at the beginning
  of a note.
\item Minor punctuation fixes in \textsf{biblatex-chicago-notes-df.bbx}.
\end{itemize}

To Do:

\begin{itemize}
\item Integrate \textsf{biblatex's} American punctuation facilities.
\item Separate out \enquote{options} from the basic citation
  \enquote{style,} using a \LaTeX\ style file.  This is an
  architectural change recommended by Lehman.
\item Investigate and possibly integrate the new entry types provided
  in \textsf{biblatex} 0.8.
\end{itemize}

\textbf{0.7: First public release, September 18, 2008}

\end{document}
