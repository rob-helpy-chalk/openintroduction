\documentclass[twoside, openright]{book}

%glossary and indexing stuff
\usepackage[toc, nopostdot, numberedsection]{glossaries}
\usepackage{datatool}
%\newglossary*{catstatements}{Chapter \ref{chap:catstatements} Key Terms}
%\newglossary*{catsyllogisms}{Key Terms}
\renewcommand{\glsnamefont}[1]{\makefirstuc{#1}}
\makeglossaries

%General packages
\usepackage{answers}
\usepackage{textcomp}
\usepackage{anyfontsize}	
\usepackage{geometry}
\usepackage{url}
\usepackage{changepage}
\usepackage{syntonly}
\usepackage{enumitem}
\usepackage{turnstile}
\usepackage[normalem]{ulem}
\usepackage{fixltx2e}	
\usepackage{wasysym}
\usepackage{tocloft}
\usepackage{fancyhdr}
\usepackage{fancyref}
\usepackage{etoolbox}
\usepackage[utf8]{inputenc}
%\usepackage{amsthm} for some reason, this conflicts with the fitch.sty part of openlogic.sty. 

%%% Bibstuff
\usepackage[authordate,autocite=inline,backend=biber, natbib]{biblatex-chicago}
\bibliography{tex/z-openlogic}
% To typeset the bibliography, you need to run "biber --output-safechars openintroduction" from the command line. You can't use the function within TeXworks, because it doen't have the --output-safechars flag. Without that flag, biber is unable to handle many characters used for Sanskrit words, like the n with a dot under it.




%%%  graphics packages %%%
\usepackage{tikz}
\usetikzlibrary{shapes,backgrounds,matrix,arrows,decorations,positioning}
\usepgflibrary{arrows.new}
\usepackage{graphicx}
\usepackage{xcolor}


%%%    Table and figure packages %%%
\usepackage{float}
\usepackage[singlelinecheck=false, skip=0pt]{caption} %left aligns captions for tables, moves them closer to table. 
\usepackage{tabularx}
\usepackage{longtable}
\usepackage{tabu}
\usepackage[framemethod=1]{mdframed} 
\usepackage{wrapfig} %I used this to put a frame around tables.
\tabulinesep=.75ex
\usepackage{colortbl}
\floatstyle{plain} 
\restylefloat{figure}
\usepackage[export]{adjustbox}
\usepackage{multirow}
\usepackage{rotating}
\usepackage{booktabs}
	

%linking and bookmarks in the pdf.
\usepackage{hyperref}
\hypersetup{pdftex,colorlinks=true,allcolors=blue}
\usepackage{hypcap}	
	

\usepackage{openintroduction}
	
\pdfinfo{
  /Title (An Open Introduction to Logic)
  /Author (J. Robert Loftis, Cathal Woods, and P.D. Magnus)
  /Subject (An open access introductory textbook in logic and critical thinking)
  )
}

\begin{document}

%\label{showanswers} %uncomment this tag and typeset twice to show answers
%\label{blank_prob_set} %uncomment this tag and typeset twice to create a blank problem set sheet. Don't use with \label{showanswers} uncommented

\raggedright
\setlength{\parindent}{1em}
\setlength{\parskip}{1em}	

\frontmatter
\pagestyle{plain} %Says there are no running heads, only page numbers centered at the bottom. 
\setlength{\parindent}{0em}
\thispagestyle{empty}

%\vfill

%{\Huge The Basics of Argument,} \\ 
%\vspace{3pt}
%{\Huge for Bioethics PHLY 161}
%

%
{\large Excerpts from}\\
\vfill
{\LARGE An Open Introduction to Logic}\\
\vfill 
%{\huge For All \script{X}}\\ \vspace{24pt}
%{\Large \emph{The Lorain County Remix}}\\


%{\large Selections for Critical Thinking (PHLY 174), Spring 2016} 
%{\large Selections for Introduction to Logic (PHLY 171), Fall 2015} 


\vfill

{\sf P.D. Magnus}\\
\emph{University at Albany, State University of New York}

and

{\sf Cathal Woods}\\
\emph{Virginia Wesleyan University}

and

{\sf J. Robert Loftis}\\
\emph{Lorain County Community College}

\vfill

{\sf
	This book is offered under a Creative Commons license.\\
	(Attribution-NonCommercial-ShareAlike 4.0 International )
}

\includegraphics[width=66pt, height=23pt, keepaspectratio=true]{img/cc-by-nc-sa.png}


%\begin{frame}
%	\titlepage
%	\vfill
%	\begin{center}
%		\CcGroupByNcSa{0.83}{0.95ex}\\[2.5ex]
%		{\tiny\CcNote{\CcLongnameByNcSa}}
%		\vspace*{-2.5ex}
%	\end{center}
%\end{frame}

\thispagestyle{empty}%
%\clearpage\setcounter{page}{1}
%\cftpagenumbersoff{chapter}
This is version 0.1 of An Open Introduction to Logic. It is current as of \today. %Complete version information is available at \textbookhomepage.

\vfill

{\copyright\ \ifthenelse{\year=2005}{\number\year}{2005--\number\year} by Cathal Woods, P.D. Magnus, and J. Robert Loftis. Some rights reserved. }

\begin{adjustwidth}{2em}{0em}
{\footnotesize Licensed under a Creative Commons license.\\
	(Attribution-NonComercial-ShareAlike 4.0 International )
	\url{https://creativecommons.org/licenses/by-nc-sa/4.0/}


\includegraphics[width=66pt, height=23pt, keepaspectratio=true]{img/cc-by-nc-sa.png}

}

\end{adjustwidth}

\vfill

This book incorporates material from \emph{An Introduction to Reasoning} by Cathal Woods, available at \url{sites.google.com/site/anintroductiontoreasoning/}
and \emph{For All X} by P.D. Magnus (version 1.27 [090604]), available at \url{www.fecundity.com/logic}.


\textit{Introduction to Reasoning} \copyright\ 2007--2014 by Cathal Woods. Some rights reserved.

\begin{adjustwidth}{2em}{0em}
{\footnotesize Licensed under a Creative Commons license: Attribution-NonCommercial-ShareAlike 3.0 Unported. \url{http://creativecommons.org/licenses/by-nc-sa/3.0/}}
\end{adjustwidth}

\textit{For All} X \copyright\  2005--2010 by P.D. Magnus. Some rights reserved.

\begin{adjustwidth}{2em}{0em}
{\footnotesize Licensed under a Creative Commons license: Attribution ShareAlike \url{http://creativecommons.org/licenses/by-sa/3.0/}}
\end{adjustwidth}

\vfill

J. Robert Loftis compiled this edition and wrote original material for it. He takes full responsibility for any mistakes remaining in this version of the text.


\vfill

Typesetting was carried out entirely in \LaTeX$2\varepsilon$. The style for typesetting proofs is based on fitch.sty (v0.4) by Peter Selinger, University of Ottawa.




\iflabelexists{part:formal_logic}{
\pagebreak

``When you come to any passage you don't understand, \emph{read it again}: if you \emph{still} don't understand it, \emph{read it again}: if you fail, even after \emph{three} readings, very likely your brain is getting a little tired. In that case, put the book away, and take to other occupations, and next day, when you come to it fresh, you will very likely find that it is \emph{quite} easy.'' \\
-- Charles Dodgson (Lewis Carroll) \emph{Symbolic Logic} \parencite*{Dodgson1896}
}{}


	


{
\setlength{\parskip}{0em}
\cftpagenumbersoff{part}
%\cftpagenumbersoff{chapter}

\renewcommand{\cftpartpresnum}{\sf\Large\partname\ }
\tableofcontents
}


\setlength{\parindent}{1em}
\chapter{About this Book}

This book was created by combining two previous books on logic and critical thinking, both made available under a Creative Commons license, and then adding some material so that the coverage matched that of commonly used logic textbooks.

P.D. Magnus' \textit{For All} X \parencite*{Magnus2008} formed the basis of Part \ref{part:formal_logic}: Formal Logic. I began using \textit{For All} X in my own logic classes in 2009, but I quickly realized I needed to make changes to make it appropriate for the community college students I was teaching. In 2010 I began developing \textit{For All} X \textit{: The Lorain County Remix} and using it in my classes. The main change I made was to separate the discussions of sentential and quantificational logic and to add exercises. It is this remixed version that became the basis for Part \ref{part:formal_logic}: Formal Logic complete version of this text. 

Similarly, Part \ref{part:CT}: Critical Thinking\iflabelexists{part:inductive_scientific}{ and Part \ref{part:inductive_scientific}: Inductive and Scientific Reasoning.}{ } grew out of Cathal Woods' \textit{Introduction to Reasoning}. In the Spring of 2011, I began to use an early version of this text (\cite{Woods2011}) in my critical thinking courses. I kept up with the updates and changes to the text until the release of \cite{Woods2014}, all the while gradually merging the material with the work in \textit{For All X}. After that point, my version forks from Woods's.

On May 20, 2016, I posted the combined textbook to Github and all subsequent changes have been tracked there: \url{https://github.com/rob-helpy-chalk/openintroduction}



 \begin{adjustwidth}{2em}{0em} 
 J. Robert Loftis \\
\noindent \emph{Elyria, Ohio, USA} 
\end{adjustwidth}

	

\setlength{\parindent}{1em}
\chapter{Acknowledgments}

Thanks first of all go to the authors of the textbooks here stitched together: P.D. Magnus for \emph{For All} X and Cathal Woods for \emph{Introduction to Reasoning}. My thanks go to them for writing the excellent textbooks that have been incorporated into this one, for making those publicly available under Creative Commons licenses, and for giving their blessing to this derivative work. 

In general, this book would not be possible without a culture of sharing knowledge.   The book was typeset using \LaTeX$2\varepsilon$ developed by Leslie Lamport. Lamport was building on \TeX by Donald Knuth. Peter Selinger built on what Lamport made by developing the Fitch typesetting format that the proofs were laid out in. Diagrams were made in PikZ by Till Tantu. All of these are coding systems are not only freely available online, they have extensive user support communities. Add-on packages are designed, manuals are written, questions are answered in discussion forums, all by people who are donating their time and expertise.

The culture of sharing isn't just responsible for the typesetting of this book; it was essential to the content. Essential background information comes from the free online \textit{Stanford Encyclopedia of Philosophy}. Primary sources from the history of logic came from \textit{Project Gutenberg}. Logicians, too, can and should create free knowledge.

Many early adopters of this text provided invaluable feedback, including Jeremy Dolan, Terry Winant, Benjamin Lennertz, Ben Sheredos, and Michael Hartsock. Lennertz, in particular, provided useful edits. Helpful comments were also made by Ben Cordry, John Emerson, Andrew Mills, Nathan Smith, Vera Tobin, Cathal Woods, and many more that I have forgot to mention, but whose emails are probably sitting on my computer somewhere. 

I would also like to thank Lorain County Community College for providing the sabbatical leave that allowed me to write the sections of this book on Aristotelian logic. Special thanks goes to all the students at LCCC who had to suffer through earlier versions of this work and provided much helpful feedback. Most importantly, I would like to thank Molly, Caroline and Joey for their incredible love and support. 

 \begin{adjustwidth}{2em}{0em} 
 J. Robert Loftis \\
\noindent \emph{Elyria, Ohio, USA} 
\end{adjustwidth}

\pagebreak
	
\thispagestyle{empty}

                                    
\noindent Intellectual debts too great to articulate are owed to scholars too many to enumerate. At different points in the work, readers might detect the influence of various works of Aristotle, Toulmin (especially \cite*{Toulmin1958}), Fisher and Scriven \parencite*{Fisher1997}, Walton (especially \cite*{Walton1996}), Epstein \parencite*{Epstein2002}, Johnson-Laird (especially \cite*{johnson2006we}), Scriven \parencite*{Scriven1962}, Giere \parencite*{giere1997understanding} and the works of the Amsterdam school of pragma-dialectics \citep{van2002argumentation}.

Thanks are due to Virginia Wesleyan College for providing me with Summer Faculty Development funding in 2008 and 2010 and a Batten professorship in 2011. These funds, along with some undergraduate research funds (also provided by VWC), allowed me to hire students Gaby Alexander (2008), Ksera Dyette (2009), Mark Jones (2008), Andrea Medrano (2011), Lauren Perry (2009), and Alan Robertson (2010). My thanks to all of them for their hard work and enthusiasm.

For feedback on the text, thanks are due to James Robert (Rob) Loftis (Lorain County Community College) and Bill Roche (Texas Christian University). Answers (to exercises) marked with “(JRL)” are by James Robert Loftis.

Particular thanks are due to my (once) Ohio State colleague Bill Roche. The book began as a collection of lecture notes, combining work by myself and Bill. 

\begin{adjustwidth}{2em}{0em}
Cathal Woods\\
\noindent\emph{Norfolk, Virginia, USA}\\
\noindent(Taken from \emph{Introduction to Reasoning} (\citeyear{Woods2014}))
\end{adjustwidth}


\vspace{3cm}

\noindent The author would like to thank the people who made this project possible. Notable among these are Cristyn Magnus, who read many early drafts; Aaron Schiller, who was an early adopter and provided considerable, helpful feedback; {and} Bin Kang, Craig Erb, Nathan Carter, Wes McMichael, and the students of Introduction to Logic, who detected various errors in previous versions of the book.


\begin{adjustwidth}{2em}{0em}
P.D. Magnus \\
\noindent\emph{Albany, New York, USA}\\
\noindent(Taken from \emph{For All X} (\citeyear{Magnus2008})) 
\end{adjustwidth}





\mainmatter
\setlength{\parindent}{1em}
\pagestyle{headings} % puts the running heads back.
\label{full_version} %Include this label to make cross references work right when typesetting full text


%\part{Basic Concepts} \label{part:basic_concepts}
%\chapter{What Is \iflabelexists{CTVersion}{Critical Thinking?}{Logic?}}
\label{Chap:what_is_logic}
\markright{Ch. \ref{Chap:what_is_logic}: What Is \iflabelexists{CTVersion}{Critical Thinking?}{Logic?}}

\label{noexplanation}

% *****************************
% *		Introduction                      *
% ****************************

\section{Introduction}

\iflabelexists{CTVersion}{Critical thinking is a part of a more general field called logic.}{} Logic is a part of the study of human reason, the ability we have to think abstractly, solve problems, explain the things that we know, and infer new knowledge on the basis of evidence. Traditionally, logic has focused on the last of these items, the ability to make inferences on the basis of evidence. \iflabelexists{CTVersion}{Sometimes we study our ability to make inferences for the purpose of improving that ability in practical sitations. When we are doing logic for practical purposes like this, we say we are studying critical thinking.
}{}

\iflabelexists{CTVersion}{Making inferences}{This} is an activity you engage in every day. Consider, for instance, the game of Clue. (For those of you who have never played, Clue is a murder mystery game where players have to decide who committed the murder, what weapon they used, and where they were.) A player in the game might decide that the murder weapon was the candlestick by ruling out the other weapons in the game: the knife, the revolver, the rope, the lead pipe, and the wrench. This evidence lets the player know something they did not know previously, namely, the identity of the murderer.

\iflabelexists{CTVersion}{In logic and critical thinking,}{In logic,} we use the word ``argument'' to refer to the attempt to show that certain evidence supports a conclusion. This is very different from the sort of argument you might have when you are mad at someone, which could involve screaming and throwing things. We are going to use the word ``argument'' a lot in this book, so you need to get used to thinking of it as a name for a rational process, and not a word that describes what happens when people disagree.

A logical argument is structured to give someone a reason to believe some conclusion. Here is the argument about a game of Clue written out in a way that shows its structure. 


\label{argClue}
\begin{earg}
\item[P$_1$:] In a game of Clue, the possible murder weapons are the knife, the candlestick, the revolver, the rope, the lead pipe, and the wrench.
\item[P$_2$:] The murder weapon was not the knife.
\item[P$_3$:] The murder weapon was also not the revolver, the rope, the lead pipe, or the wrench.
\vspace{-.5em}
\item [] \rule{0.9\linewidth}{.5pt} 
\item[C:] Therefore, the murder weapon was the candlestick.
\end{earg} 

In the argument above, statements P$_1$--P$_3$ are the evidence. We call these the \emph{premises}. The word ``therefore'' indicates that the final statement, marked with a C, is the \emph{conclusion} of the argument. If you believe the premises, then the argument provides you with a reason to believe the conclusion. You might use reasoning like this purely in your own head, without talking with anyone else. You might wonder what the murder weapon is, and then mentally rule out each item, leaving only the candlestick. On the other hand, you might use reasoning like this while talking to someone else, to convince them that the murder weapon is the candlestick. (Perhaps you are playing as a team.) Either way the structure of the reasoning is the same. 

\newglossaryentry{logic}
{
name=logic,
description={the part of the study of reasoning that focuses on argument.}
}

We can define \textsc{\Gls{logic}}\label{def:logic} then more precisely as the part of the study of reasoning that focuses on argument. In more casual situations, we will follow ordinary practice and use the word ``logic'' to either refer to the business of studying human reason or the thing being studied, that is, human reasoning itself. While logic focuses on argument, other disciplines, like decision theory and cognitive science, deal with other aspects of human reasoning, like abstract thinking and problem solving more generally. Logic, as the study of argument, has been pursued for thousands of years by people from civilizations all over the globe. The initial motivation for studying logic is generally practical. Given that we use arguments and make inferences all the time, it only makes sense that we would want to learn to do these things better. \iflabelexists{CTVersion}{So the earliest attempts at logic could be classified as critical thinking.}{} Once people begin to study logic, however, they quickly realize that it is a fascinating topic in its own right. Thus the study of logic quickly moves from being a practical business to a theoretical endeavor people pursue for its own sake. 

\newglossaryentry{metareasoning}
{
name=metareasoning,
description={Using reasoning to study reasoning. See also \emph{metacognition}.}
}

\newglossaryentry{metacognition}
{
name=metacognition,
description={Thought processes that are applied to other thought processes See also \emph{metareasoning}.}
}

In order to study reasoning, we have to apply our ability to reason to our reason itself. This reasoning about reasoning is called \textsc{\gls{metareasoning}}\label{def:Metareasoning}. It is part of a more general set of processes called \textsc{\gls{metacognition}}\label{def:Metacognition}, which is just any kind of thinking about thinking. When we are pursing logic as a practical discipline, one important part of metacognition will be awareness of your own thinking, especially its weakness and biases, as it is occurring. More theoretical metacognition will be about attempting to understand the structure of thought itself. 


\newglossaryentry{content neutrality}
{
name=content neutrality,
description={the feature of the study of logic that makes it indifferent to the topic being argued about. If a method of argument is considered rational in one domain, it should be considered rational in any other domain, all other things being equal.}
}

Whether we are pursuing logical for practical or theoretical reasons, our focus is on argument. The key to studying argument is to set aside the subject being argued about and to focus on the \emph{way} it is argued \emph{for}. The section opened with an example that was about a game of Clue. However, the kind of reasoning used in that example was just the process of elimination. Process of elimination can be applied to any subject. Suppose a group of friends is deciding which restaurant to eat at, and there are six restaurants in town. If you could rule out five of the possibilities, you would use an argument just like the one above to decide where to eat. Because logic sets aside what an argument is about, and just looks at how it works rationally, logic is said to have \textsc{\gls{content neutrality}}. \label{def:content_neutrality} If we say an argument is good, then the same kind of argument applied to a different topic will also be good.  If we say an argument is good for solving murders, we will also say that the same kind of argument is good for deciding where to eat, what kind of disease is destroying your crops, or who to vote for. 

\newglossaryentry{formal logic}
{
name=formal logic,
description={A way of studying logic that achieves content neutrality by replacing parts of the arguments being studied with abstract symbols. Often this will involve the construction of full formal languages.}
}


When logic is studied for theoretical reasons, it typically is pursued as \textsc{\gls{formal logic}}. \label{def:Formal_logic} In formal logic we get content neutrality by replacing parts of the argument we are studying with abstract symbols. For instance, we could turn the argument above into a formal argument like this:

\label{argClueformal}
\begin{earg}
\item[P$_1$:] There are six possibilities: A, B, C, D, E, and F.
\item[P$_2$:] A is false.
\item[P$_3$:] B, D, E, and F are also false.
\vspace{-.5em}
\item [] \rule{0.6\linewidth}{.5pt} 
\item[C:]  $\therefore$ The correct answer is C.
\end{earg} 

Here we have replaced the concrete possibilities in the first argument with abstract letters that could stand for anything. We have also replaced the English word ``therefore'' with the symbol ``$\therefore$,'' which means therefore. This lets us see the formal structure of the argument, which is why it works in any domain you can think of. In fact, we can think of formal logic as the method for studying argument that uses abstract notation to identify the formal structure of argument.  Formal logic is closely allied with mathematics, and studying formal logic often has the sort of puzzle-solving character one associates with mathematics. \iflabelexists{full_version}{You will see this when we get to Part \ref{part:formal_logic}, which covers formal logic.}
	{\iflabelexists{part:CT}{}
	{\iflabelexists{part:cat_logic}{You will see this when we get to Parts \ref{part:cat_logic} and \ref{part:sent_logic}, which cover formal logic.}
{\iflabelexists{part:quant_logic}{You will see this when we get to Parts \ref{part:sent_logic} and \ref{part:quant_logic}, which cover formal logic.}}	
	{}	}}
 
\newglossaryentry{critical thinking}
{
name=critical thinking,
description={The use of metareasoning to improve our reasoning in practical situations. Sometimes the term is also used to refer to the results of this effort at self improvement, that is, reasoning in practical situations that has been sharpened by reflection and metareasoning.}
}

\newglossaryentry{critical thinker}
{
name=critical thinker,
description={A person who has both sharpened their reasoning abilities using metareasoning and deploys those sharpened abilities in real world situations..}
}

\newglossaryentry{informal logic}
{
name=informal logic,
description={The study of arguments given in ordinary language.}
}


\iflabelexists{CTVersion}{As we said before, when}{When} logic is studied for practical reasons, it is typically called critical thinking. We will define \textsc{\gls{critical thinking}}\label{def:Critical_Thinking}  narrowly as the use of metareasoning to improve our reasoning in practical situations.  Sometimes we will use the term ``critical thinking'' more broadly to refer to the results of this effort at self-improvement.  You are ``thinking critically'' when you reason in a way that has been sharpened by reflection and metareasoning. A \textsc{\gls{critical thinker}}\label{def:critical_thinker} someone who has both sharpened their reasoning abilities using metareasoning and deploys those sharpened abilities in real world situations.

Critical thinking is generally pursued as \textsc{\gls{informal logic}}, rather than formal logic. This means that we will keep arguments in ordinary language and draw extensively on your knowledge of the world to evaluate them. In contrast to the clarity and rigor of formal logic, informal logic is suffused with ambiguity and vagueness. There are problems  with multiple correct answers, or where reasonable people can disagree with what the correct answer is. This is because you will be dealing with reasoning in the real world, which is messy. \label{messiness_warning} \label{ver_var} \iflabelexists{part:CT}{You will learn more about this in the chapters on critical thinking Part \ref{part:CT}}{}
  

You can think of the difference between formal logic and informal logic as the difference between a laboratory science and a field science. \label{lab_vs._field_science} If you are studying, say, mice, you could discover things about them by running experiments in a lab, or you can go out into the field where mice live and observe them in their natural habitat.  Informal logic is the field science for arguments: you go out and study arguments in their natural habitats, like newspapers, courtrooms, and scientific journal articles. Like studying mice scurrying around a meadow, the process takes patience, and often doesn't yield clear answers but it lets you see how things work in the real world. Formal logic takes arguments out of their natural habitat and performs experiments on them to see what they are capable of. The arguments here are like lab mice. They are pumped full of chemicals and asked to perform strange tasks, as it were. They live lives very different than their wild cousins. Some of the arguments will wind up looking like the ``ob/ob mouse'', a genetically engineered obese mouse scientists use to study type II diabetes (See Figure \ref{fig:ob_ob_mouse}). These arguments will be huge, awkward, and completely unable to survive in the wild. But they will tell us a lot about the limits of logic as a process.

\begin{figure}
\begin{mdframed}[style=mytableclearbox]
\begin{center}
\includegraphics*[scale=.8]{img/Fatmouse}
\end{center}
\end{mdframed}
\caption{The ob/ob mouse (left), a laboratory mouse which has been genetically engineered to be obese, and an ordinary mouse (right). Formal logic, which takes arguments out of their natural environment, often winds up studying arguments that look like the ob/ob mouse. They are huge, awkward, and unable to survive in the wild, but they tell us a lot about the limits of logic as a process. Photo from \cite{WikimediaCommons2006}.}
\label{fig:ob_ob_mouse}
\end{figure}


\newglossaryentry{rhetoric}
{
name=rhetoric,
description={The study of effective persuasion.}
}


Our main goal in studying arguments is to separate the good ones from the bad ones. The argument about Clue we saw earlier is a good one, based on the process of elimination.  It is good because it leads to truth. If I've got all the premises right, the conclusion will also be right. The textbook \textit{Logic: Techniques of Formal Reasoning} \citep{Kalish1980} had a nice way of capturing the meaning of logic: ``logic is the study of virtue in argument.'' \label{virtue_in_argument} This textbook will accept this definition, with the caveat that an argument is virtuous if it helps us get to the truth.

Logic is different from \textsc{\gls{rhetoric}}, which is the study of effective persuasion. Rhetoric does not look at virtue in argument. It only looks at the power of arguments, regardless of whether they lead to truth. An advertisement might convince you to buy a new truck by having a gravelly voiced announcer tell you it is ``ram tough'' and showing you a picture of the truck on top of a mountain, where it no doubt actually had to be airlifted. This sort of persuasion is often more effective at getting people to believe things than logical argument, but it has nothing to do with whether the truck is really the right thing to buy. In this textbook we will only be interested in rhetoric to the extent that we need to learn to defend ourselves against the misleading rhetoric of others. \iflabelexists{part:CT}{The sections of this text on critical thinking will emphasize becoming aware of our biases and how others might use misleading rhetoric to exploit them. }{}This will not, however, be anything close to a full treatment of the study of rhetoric.


% ******************************************
% *		Statement, Argument, Premise, Conclusion  *
% ******************************************

\section{Statement, Argument, Premise, Conclusion}
\label{sec:SAPC}

\newglossaryentry{statement}
{
name=statement,
description={A unit of language that can be true or false.}
}

\iflabelexists{CTVersion}{So far we have defined logic as the study of argument and critical thinking as logic done for practical purposes.}{So far we have defined logic as the study of argument and outlined its relationship to related fields.} To go any further, we are going to need a more precise definition of what exactly an argument is. We have said that an argument is not simply two people disagreeing; it is an attempt to prove something using evidence. More specifically, an argument is composed of statements. In \iflabelexists{CTVersion}{logic and critical thinking}{logic}, we define a \textsc{\gls{statement}} \label{def:statement} as a unit of language that can be true or false. To put it another way, it is some combination of words or symbols that have been put together in a way that lets someone agree or disagree with it. All of the items below are statements.

\begin{enumerate}[label=(\alph*)]
\item \label{itm:t.rex_true}\emph{Tyrannosaurus rex} went extinct 65 million years ago. 
\item \label{itm:t.rex_false}\emph{Tyrannosaurus rex} went extinct last week.
\item \label{itm:t.rex_unknown}On this exact spot, 100  million years ago, a \emph{T. rex} laid a clutch of eggs. 
\item \label{itm:silly}George W. Bush is the king of Jupiter. 
\item \label{itm:moral}Murder is wrong. 
\item \label{itm:opinion1}Abortion is murder. 
\item \label{itm:opinion2}Abortion is a woman's right. 
\item \label{itm:opinion3}Lady Gaga is pretty.
\item \label{itm:definition}Murder is the unjustified killing of a person.
\item \label{itm:nonsense}The slithy toves did gyre and gimble in the wabe.
\item \label{itm:history}The murder of logician Richard Montague was never solved. 
\end{enumerate}

Because a statement is something that can be true \emph{or} false, statements include truths like \ref{itm:t.rex_true} and falsehoods like \ref{itm:t.rex_false}. A statement can also be something that that must either be true or false, but we don't know which, like \ref{itm:t.rex_unknown}. A statement can be something that is completely silly, like \ref{itm:silly}. Statements in logic include statements about morality, like \ref{itm:moral}, and things that in other contexts might be called ``opinions,'' like \ref{itm:opinion1} and \ref{itm:opinion2}. People disagree strongly about whether \ref{itm:opinion1} or \ref{itm:opinion2} are true, but it is definitely possible for one of them to be true. The same is true about \ref{itm:opinion3}, although it is a less important issue than \ref{itm:opinion1} and \ref{itm:opinion2}. A statement in logic can also simply give a definition, like \ref{itm:definition}.  Statements can include nonsense words like \ref{itm:nonsense}, because we don't really need to know what the statement is about to see that it is the sort of thing that can be true or false. All of this relates back to the content neutrality of the study of argument. The statements we study can be about dinosaurs, abortion, Lady Gaga, and even the history of logic itself, as in statement \ref{itm:history}, which is true.

% You have now put the normative/descriptive distinction only in the CT version of the text}

\iflabelexists{CTVersion}{
\newglossaryentry{descriptive statement}
{
name=descriptive statement,
description={A statement which talks about the way the world is. These are contrasted with normative statements, which talk about the way the world should be.}
}


\newglossaryentry{normative statement}
{
name=normative statement,
description={A statement which talks about the way the world should be. These are contrasted with descriptive statements, which talk about the way the world is.}
}



So you see that statement is a broad category that can include all kinds of things that you might not normally lump together. One important division within the class of statements is the distinction between statements which talk about the way the world \textit{is} and statements which talk about the way the world should be. Statements that are about the way the world is are called \textsc{\glspl{descriptive statement}}\label{def:descriptive_statement}. These include true descriptions of the world, like statement \ref{itm:t.rex_true} and false descriptions like \label{itm:silly}. Statements that are about the way the world should be are called \textsc{\glspl{normative statement}}\label{def:normative_statement}. These include statements with an explicit ``should'' in them, like ``you shouldn't chew with your mouth open.'' They also include statements that contain an implicit ``should'' like statements  \ref{itm:opinion1} and \ref{itm:opinion2}. If you assert that abortion is murder, you are saying that people \textit{shouldn't} have abortions and that abortions \textit{should} be illegal. Conversely, you say that abortion is a woman's right, you are saying it \textit{should} be legal. All of this is different than describing what people actually do or what is actually legal. 

Some kinds of statements can be interpreted normatively or discriptively. Definitions like statement \ref{itm:definition} are like that. It could describe the way people actually use a word or announce that people should use a certain way. Sometimes the same definition can be used different ways. Dictionaries, for instance, are written to be descriptive: they simply summarize how a word has been used up to this point. However, when dictionaries are actually used, they are generally used normatively. We use them to correct people's use of words, to say that they \textit{should} use words differently.}{}

We are treating statements primarily as units of language or strings of symbols, and most of the time the statements you will be working with will just be words printed on a page. However, it is important to remember that statements are also what philosophers call ``speech acts.'' They are actions people take when they speak (or write). If someone makes a statement they are typically telling other people that they believe the statement to be true, and will back it up with evidence if asked to. When people make statements, they always do it in a context---they make statements at a place and a time with an audience. Often the context statements are made in will be important for us, so when we give examples, statements, or arguments we will sometimes include a description of the context. When we do that, we will give the context in \textit{italics.} See Figure \ref{fig:statements_and_context} for examples. \label{context_marker} For the most part, the context for a statement or argument will be important in the chapters on critical thinking, when we are pursing the study of logic for practical reasons. In the chapters on formal logic, context is less important, and we will be more likely to skip it. 

\begin{figure}
\begin{mdframed}[style=mytableclearbox]
\includegraphics*[width=\linewidth]{img/statement_and_contexts}
\end{mdframed}
\caption{A statement in different contexts, or no context.} \label{fig:statements_and_context}
\end{figure}


``Statements' in this text does \emph{not} include questions, commands, exclamations, or sentence fragments. Someone who asks a \emph{question} like ``Does the grass need to be mowed?'' is typically not claiming that anything is true or false. Generally, \emph{questions} will not count as statements, but \emph{answers} will. ``What is this course about?'' is not a statement. ``No one knows what this course is about,'' is a statement.

For the same reason \emph{commands} do not count as statements for us. If someone bellows ``Mow the grass, now!'' they are not saying whether the grass has been mowed or not. You might infer that they believe the lawn has not been mowed, but then again maybe they think the lawn is fine and just want to see you exercise. 

An exclamation like ``Ouch!'' is also neither true nor false. On its own, it is not a statement. We will treat ``Ouch, I hurt my toe!'' as meaning the same thing as ``I hurt my toe.'' The ``ouch'' does not add anything that could be true or false.

Finally, a lot of possible strings of words will fail to qualify as statements simply because they don't form a complete sentence. In your composition classes, these were probably referred to as sentence fragments. This includes strings of words that are parts of sentences, such as noun phrases like ``The tall man with the hat'' and verb phrases, like ``ran down the hall.'' Phrases like these are missing something they need to make a claim about the world. The class of sentence fragments also includes completely random combinations of words, like ``The up if blender route,'' which don't even have the form of a statement about the world.  

Other logic textbooks describe the components of argument as ``propositions,'' or ``assertions,'' and we will use these terms sometimes as well.  There is actually a great deal of disagreement about what the differences between all of these things are and which term is best used to describe parts of arguments. However, none of that makes a difference for this textbook. We could have used any of the other terms in this text, and it wouldn't change anything. Some textbooks will also use the term ``sentence'' here. We will not use the word ``sentence'' to mean the same thing as ``statement.'' Instead, we will use ``sentence'' the way it is used in ordinary grammar, to refer generally to statements, questions, and commands. 

Sometimes the outward form of a speech act does not match how it is actually being used. A rhetorical question, for instance, has the outward form of a question, but is really a statement or a command. If someone says ``don't you think the lawn needs to be mowed?'' they may actually mean a statement like ``the lawn needs to be mowed'' or a command like ``mow the lawn, now.'' Similarly one might disguise a command as a statement. ``You will respect my authority'' \emph{is} either true or false---either you will or you will not. But the speaker may intend this as an order---''Respect me!''---rather than a prediction of how you will behave.

When we study argument, we need to express things as statements, because arguments are composed of statements. Thus if we encounter a rhetorical question while examining an argument, we need to convert it into a statement. ``Don't you think the lawn needs to be mowed'' will become ``the lawn needs to be mowed.'' Similarly, commands will become should statements. ``Mow the lawn, now!'' will need to be transformed into ``You should mow the lawn.'' 

\newglossaryentry{practical argument}
{
name=practical argument,
description={An argument whose conclusion is a statement that someone should do something.}
}

The latter kind of change will be important in critical thinking, because critical thinking often studies arguments whose goal is to an get audience to do something. These are called \textsc{\glspl{practical argument}}\label{def:practical_argument}. Most advertising and political speech consists of practical arguments, and these are crucial topics for critical thinking.

\newglossaryentry{argument}
{
name=argument,
description={a connected series of statements designed to convince an audience of another statement.}
}

\newglossaryentry{premise}
{
name=premise,
description={a statement in an argument that provides evidence for the conclusion}
}

\newglossaryentry{conclusion}
{
name=conclusion,
description={the statement that an argument is trying to convince an audience of.}
}

 
Once we have a collection of statements, we can use them to build arguments. An \textsc{\gls{argument}} \label{def:Argument} is a connected series of statements designed to convince an audience of another statement. Here an audience might be a literal audience sitting in front of you at some public speaking engagement. Or it might be the readers of a book or article. The audience might even be yourself as you reason your way through a problem. Let's start with an example of an argument given to an external audience. This passage is from an essay by Peter Singer called ``Famine, Affluence, and Morality'' in which he tries to convince people in rich nations that they need to do more to help people in poor nations who are experiencing famine.

\begin{quotation}\noindent \textit{A contemporary philosopher writing in an academic journal} If it is in our power to prevent something bad from happening, without thereby sacrificing anything of comparable moral importance, we ought, morally, to do so. Famine is something bad, and it can be prevented without sacrificing anything of comparable moral importance. So, we ought to prevent famine. \citep{Singer1972} \label{singer_quote} \end{quotation} 

Singer wants his readers to work to prevent famine. This is represented by the last statement of the passage, ``we ought to prevent famine,'' which is called the conclusion of the passage. The \textsc{\gls{conclusion}} \label{def:conclusion} of an argument is the statement that the argument is trying to convince the audience of. The statements that do the convincing are called the \textsc{\glspl{premise}}. \label{def:premise}In this case, the argument has three premises: (1) ``If it is in our power to prevent something bad from happening, without thereby sacrificing anything of comparable moral importance, we ought, morally, to do so''; (2) ``Famine is something bad''; and (3) ``it can be prevented without sacrificing anything of comparable moral importance.''

Now let's look at an example of internal reasoning. 

\begin{quotation}\noindent\textit{Jack arrives at the track, in bad weather.} There is no one here. I guess the race is not happening. \label{racetrack}
\end{quotation}

In the passage above, the words in \textit{italics} explain the context for the reasoning, and the words in regular type represent what Jack is actually thinking to himself. \nix{(We will talk more about his way of representing reasoning in section \ref{sec:arguments_and_context}, below.)} This passage again has a premise and a conclusion. The premise is that no one is at the track, and the conclusion is that the race was canceled. The context gives another reason why Jack might believe the race has been canceled, the weather is bad. You could view this as another premise--it is very likely a reason Jack has come to believe that the race is canceled. In general, when you are looking at people's internal reasoning, it is often hard to determine what is actually working as a premise and what is just working in the background of their unconscious. %[We will talk more about this in section...]


\newglossaryentry{premise indicator}
{
name=premise indicator,
description={a word or phrase such as ``because'' used to indicate that what follows is the premise of an argument.}
}

\newglossaryentry{conclusion indicator}
{
name=conclusion indicator,
description={a word or phrase such as ``therefore'' used to indicate that what follows is the conclusion of an argument.}
}

When people give arguments to each other, they typically use words like ``therefore'' and ``because.'' These are meant to signal to the audience that what is coming is either a premise or a conclusion in an argument. Words and phrases like ``because'' signal that a premise is coming, so we call these \textsc{\glspl{premise indicator}}. Similarly, words and phrases like ``therefore'' signal a conclusion and are called \textsc{\glspl{conclusion indicator}}. The argument from Peter Singer (on page \pageref{singer_quote}) uses the conclusion indicator word, ``so.'' Table \ref{table:Indicators} is an incomplete list of indicator words and phrases in English.


\begin{table}
\begin{mdframed}[style=mytablebox]

\begin{longtabu}{X[1,p]X[2,p]}
\textbf{Premise Indicators:} & because, as, for, since, given that, for the reason that \\
\textbf{Conclusion Indicators:} & therefore, thus, hence, so, consequently, it follows that, in conclusion, as a result, then, must, accordingly, this implies that, this entails that, we may infer that \\
\end{longtabu}
\end{mdframed}
\caption{Premise and Conclusion Indicators.}
\label{table:Indicators}
\end{table}

\newglossaryentry{standard form}
{
name=standard form,
description={a method for representing arguments where each premise is written on a separate, numbered, line, followed by a horizontal bar and then the conclusion. Statements in the argument might be paraphrased for brevity and indicator words are removed.}
}


The two passages we have looked at in this section so far have been simply presented as quotations. But often it is extremely useful to rewrite arguments in a way that makes their logical structure clear. One way to do this is to use something called ``standard form.''   An argument written in \textsc{\gls{standard form}} \label{def:canonical_form}has each premise numbered and written on a separate line. Indicator words and other unnecessary material should be removed from the premises. Although you can shorten the premises and conclusion, you need to be sure to keep them all complete sentences with the same meaning, so that they can be true or false. The argument from Peter Singer, above, looks like this in standard form:

\begin{earg}
\item[P$_1$:] If we can stop something bad from happening, without sacrificing anything of comparable moral importance, we ought to do so. 
\item[P$_2$:] Famine is something bad.
\item[P$_3$:] Famine can be prevented without sacrificing anything of comparable moral importance.
\vspace{-.5em}
\item [] \rule{0.9\linewidth}{.5pt} 
\item[C:] We ought to prevent famine.
\end{earg} 

Each statement has been written on its own line and given a number. The statements have been paraphrased slightly, for brevity, and the indicator word ``so'' has been removed. Also notice that the ``it'' in the third premise has been replaced by the word ``famine,'' so that statements reads naturally on its own.  

Similarly, we can rewrite the argument Jack gives at the racetrack, on page \pageref{racetrack}, like this:

\begin{earg}
\item[P:] There is no one at the race track.
\vspace{-.5em}
\item [] \rule{0.4\linewidth}{.5pt} 
\item[C:] The race is not happening. 
\end{earg} 

Notice that we did not include anything from the part of the passage in italics. The italics represent the context, not the argument itself. Also, notice that the ``I guess'' has been removed. When we write things out in standard form, we write the content of the statements, ignore information about the speaker's mental state, like ``I believe'' or ``I guess.'' 

One of the first things you have to learn to do in logic is to identify arguments and rewrite them in standard form. This is a foundational skill for everything else we will be doing in this text, so we are going to run through a few examples now, and there will be more in the exercises. The passage below is paraphrased from the ancient Greek philosopher Aristotle. 

\begin{quotation}\noindent \textit{An ancient philosopher, writing for his students} Again, our observations of the stars make it evident that the earth is round. For quite a small change of position to south or north causes a manifest alteration in the stars which are overhead. (\cite{Aristotle:heavens}, 298a2-10)
\label{on_the_heavens} \end{quotation}

The first thing we need to do to put this argument in standard form is to identify the conclusion. The indicator words are the best way to do this. The phrase ``make it evident that'' is a conclusion indicator phrase. He is saying that everything else is \textit{evidence} for what follows. So we know that the conclusion is that the earth is round. ``For'' is a premise indicator word---it is sort of a weaker version of ``because.''  Thus the premise is that the stars in the sky change if you move north or south. In standard form, Aristotle's argument that the earth is round looks like this.\\


\begin{earg}
\item[P:] There are different stars overhead in the northern and southern parts of the earth.
\vspace{-.5em}
\item [] \rule{0.9\linewidth}{.5pt} 
\item[C:] The earth is spherical in shape. 
\end{earg} 

That one is fairly simple, because it just has one premise. Here's another example of an argument, this time from the book of Ecclesiastes in the Bible. The speaker in this part of the bible is generally referred to as The Preacher, or in Hebrew, Koheleth. In this verse, Koheleth uses both a premise indicator and a conclusion indicator to let you know he is giving reasons for enjoying life.

\begin{quotation}
\noindent \textit{The words of the Preacher, son of David, King of Jerusalem} There is something else meaningless that occurs on earth: the righteous who get what the wicked deserve, and the wicked who get what the righteous deserve. \ldots So I commend the enjoyment of life, because there is nothing better for a person under the sun than to eat and drink and be glad. (Ecclesiastes 8:14-15, New International Version)
\end{quotation}

Koheleth begins by pointing out that good things happen to bad people and bad things happen to good people. This is his first premise. (Most Bible teachers provide some context here by pointing that that the ways of God are mysterious and this is an important theme in Ecclesiastes.) Then Koheleth gives his conclusion, that we should enjoy life, which he marks with the word ``so.'' Finally he gives an extra premise, marked with a ``because,'' that there is nothing better for a person than to eat and drink and be glad. In standard form, the argument would look like this.


\begin{earg}
\item[P$_1$:] Good things happen to bad people and bad things happen to good people.
\item[P$_2$:] There is nothing better for people than to eat, to drink and to enjoy life.
\vspace{-.5em}
\item [] \rule{0.8\linewidth}{.5pt} 
\item[C:] You should enjoy life.
\end{earg} 

Notice that in the original passages, Aristotle put the conclusion in the first sentence, while Koheleth put it in the middle of the passage, between two premises. In ordinary English, people can put the conclusion of their argument where ever they want. However, when we write the argument in standard form, the conclusion goes last.

Unfortunately, indicator words aren't a perfect guide to when people are giving an argument. Look at this passage from a newspaper:

\begin{quotation}
\noindent \textit{From the general news section of a national newspaper} The new budget underscores the consistent and paramount importance of tax cuts in the Bush philosophy. His first term cuts affected more money than any other initiative undertaken in his presidency, including the costs thus far of the war in Iraq. All told, including tax incentives for health care programs and the extension of other tax breaks that are likely to be taken up by Congress, the White House budget calls for nearly \$300 billion in tax cuts over the next five years, and \$1.5 trillion over the next 10 years.  \citep{Toner2006}
\end{quotation}

Although there are no indicator words, this is in fact an argument. The writer wants you to believe something about George Bush: tax cuts are his number one priority. The next two sentences in the paragraph give you reasons to believe this. You can write the argument in standard form like this.

\begin{earg}
\item[P$_1$:] Bush's first term cuts affected more money than any other initiative undertaken in his presidency, including the costs thus far of the war in Iraq. 
\item[P$_2$:] The White House budget calls for nearly \$300 billion in tax cuts over the next five years, and \$1.5 trillion over the next 10 years. 
\vspace{-.5em}
\item [] \rule{0.9\linewidth}{.5pt} 
\item[C:] Tax cuts are of consistent and paramount importance of in the Bush philosophy.
\end{earg} 

The ultimate test of whether something is an argument is simply whether some of the statements provide reason to believe another one of the statements. If some statements support others, you are looking at an argument. The speakers in these two cases use indicator phrases to let you know they are trying to give an argument.

\newglossaryentry{inference}
{
name=inference,
description={the act of coming to believe a conclusion on the basis of some set of premises.}
}

A final bit of terminology for this section. An \textsc{\gls{inference}} \label{def:Inference} is the act of coming to believe a conclusion on the basis of some set of premises. When Jack in the example above saw that no one was at the track, and came to believe that the race was not on, he was making an inference. We also use the term inference to refer to the connection between the premises and the conclusion of an argument. If your mind moves from premises to conclusion, you make an inference, and the premises and the conclusion are said to be linked by an inference. In that way inferences are like argument glue: they hold the premises and conclusion together. 

%%%% Practice Problems


\practiceproblems
Throughout the book, you will find a series of practice problems that review and explore the material covered in the chapter. There is no substitute for actually working through some problems, because \iflabelexists{CTVersion}{critical thinking}{logic} is more about a way of thinking than it is about memorizing facts. %The answers to some of the problems are provided at the end of the book in Appendix \ref{app.solutions}; the problems that are solved in the appendix are marked with a star (\solutions.)

\noindent\problempart Decide whether the following passages are statements, as the term is used in this textbook, and give reasons for your answers.

\begin{longtabu}{p{.1\linewidth}p{.9\linewidth}}
\textbf{Example}: & Did you follow the instructions? \\
\textbf{Answer}: & Not a statement, a question. \\
\end{longtabu}


\begin{exercises}
\item England is smaller than China. \answerblank{\underline{Statement}}{\vspace{.25in}}
\item Greenland is south of Jerusalem. \answerblank{\underline{Statement}}{\vspace{.25in}}
\item Is New Jersey east of Wisconsin? \answerblank{\underline{A question, not a Statement.}}{\vspace{.25in}}
\item The atomic number of helium is 2. \answerblank{\underline{Statement}}{\vspace{.25in}}
\item The atomic number of helium is $\pi$. \answerblank{\underline{Statement}}{\vspace{.25in}}
\item I hate overcooked noodles. \answerblank{\underline{Statement}}{\vspace{.25in}}
\item Blech! Overcooked noodles! \answerblank{\underline{An exclamation, not a statement.}}{\vspace{.25in}}
\item Overcooked noodles are disgusting.\answerblank{\underline{Statement}}{\vspace{.25in}}
\item Take your time. \answerblank{\underline{A command, not a Statement}}{\vspace{.25in}}
\item This is the last question. \answerblank{\underline{Statement}}{\vspace{.25in}}
\end{exercises}


\noindent\problempart Decide whether the following passages are statements, as the term is used in this textbook, and give reasons for your answers.
\answer{Answers from Ben Sheredos.}
\begin{exercises}
\item Is this a question? \answer{\underline{Question, not a statement.}}
\item Nineteen out of the 20 known species of Eurasian elephants are extinct. \answer{\underline{Statement; has to be true or false (might be false bc 20 is the wrong number, or because they are not extinct, etc.)}}
\item The government of the United Kingdom has formally apologized for the way it treated the logician Alan Turing. \answer{\underline{ Statement: has to be true or false; they either have or have not apologized}} 

\item Texting while driving \answer{\underline{Not a statement, but a sentence fragment}}
\item Texting while driving is dangerous. \answer{\underline{Statement; has to be true or false.}}
\item Insanity ran in the family of logician Bertrand Russell, and he had a life-long fear of going mad. \answer{\underline{Complex, but a statement: both halves are true or false, so is the whole.}}
\item For the love of Pete, put that thing down before someone gets hurt!  \answer{\underline{Not a statement: First bit is an exclamation, second is a command.}}
\item Don't try to make too much sense of this. \answer{\underline{Not a statement, a command.}}
\item Never look a gift horse in the mouth.  \answer{\underline{Not a statement, a command.}}
\item The physical impossibility of death in the mind of someone living  \answer{\underline{ Not a statement, sentence fragment.}}
\end{exercises}

\noindent\problempart Rewrite each of the following arguments in standard form. Be sure to remove all indicator words and keep the premises and conclusion as complete sentences. Write the indicator words and phrases separately and state whether they are premise or conclusion indicators. 

%NTS: when writing these problems, be sure to include a mix of conclusion-first, conclusion-last and conclusion middle, as well as a mix of arguments with true and false premises and a variety of indicator words (or lack thereof).

\begin{longtabu}{p{.1\linewidth}p{.9\linewidth}}	
\textbf{Example}: & \textit{An ancient philosopher writes} We should not be distressed or concerned by the thought of our our own death in any way. Why? Look back on the time before you were born: It is a time you did not exist, but it does not trouble you in any way. The time after you die is also a time when you will not exist, so it shouldn't trouble you either. (Based on Lucretius \citetitle{Lucretius2001} 3.972--75)\\
\textbf{Answer}: & 
\vspace{-16pt}
\begin{earg}
\item[P$_1$:] The time before you were born is a time you did not exist.
\item[P$_2$:] You are not troubled by the time before you were born. 
\item[P$_3$:] The time after you die is also a time you will not exist.
\vspace{-.5em}
\item [] \rule{0.6\linewidth}{.5pt} 
\item[C:] We should not be distressed or concerned by the thought of our our own death. 
\end{earg} 
Premise indicator: So
\\
\end{longtabu}
	
\begin{exercises}

\item \textit{A detective is speaking: }Henry's finger-prints were found on the stolen computer. So, I infer that Henry stole the computer.  

\answerblank{
\begin{earg*}
\item Henry's finger-prints were found on the stolen computer
\itemc Henry stole the computer.  
\end{earg*}
Conclusion indicator word: So}{\vspace{1.5in}}


\item \textit{Monica is wondering about her co-workers political opinions} You cannot both oppose abortion and support the death penalty, unless you think there is a difference between fetuses and felons. Steve opposes abortion and supports the death penalty. Therefore Steve thinks there is a difference between fetuses and felons. 
		%Conclusion-last

\answerblank{
\begin{earg*}
\item You cannot both oppose abortion and support the death penalty, unless you think there is a difference between fetuses and felons. 
\item Steve opposes abortion and supports the death penalty. 
\itemc Steve thinks there is a difference between fetuses and felons. 
\end{earg*}
Conclusion Indicator: Therefore}{\vspace{1.5in}}


\item \textit{The Grand Moff of Earth defense considers strategy} We know that whenever people from one planet invade another, they always wind up being killed by the local diseases, because in 1938, when Martians invaded the Earth, they were defeated because they lacked immunity to Earth's diseases. Also, in 1942, when Hitler's forces landed on the Moon, they were killed by Moon diseases.
		%Conclusion-first

\answerblank{
\begin{earg} 
\item[1.] In 1938, when Martians invaded the Earth, they were defeated because they lacked immunity to Earth's diseases. 
\item[2.] In 1942, when Hitler's forces landed on the Moon, they were killed by Moon diseases.
\item [] \noindent\hrulefill 
\item[$\therefore$] Whenever people from one planet invade another, they always wind up being killed by the local diseases, 
\end{earg}
Premise indicator: Because }{\vspace{1.5in}}


\item If you have slain the Jabberwock, my son, it will be a frabjous day. The Jabberwock lies there dead, its head cleft with your vorpal sword. This is truly a fabjous day. 
%Conclusion-last
\answerblank{ 
\begin{earg*} 
\item  If you have slain the Jabberwock, my son, it will be a frabjous day. 
\item The Jabberwock lies there dead
 
\itemc This is truly a fabjous day 
\end{earg*}
Indicators: none		
}{\vspace{1.5in}}	

\item \textit{A detective trying to crack a case thinks to herself} Miss Scarlett was jealous that Professor Plum would not leave his wife to be with her. Therefore she must be the killer, because she is the only one with a motive. 
%Conclusion-middle
\answerblank{
\begin{earg*} 
\item Miss Scarlett was jealous that Professor Plum would not leave his wife to be with her. 
\item Miss Scarlett is the only one with a motive. 
 
\itemc Miss Scarlett must be the killer
\end{earg*}

Premise Indicator: Because \\
Conclusion Indicator: Therefore}{\vspace{1.5in}}
\end{exercises}



\noindent\problempart Rewrite each of the following arguments in standard form. Be sure to remove all indicator words and keep the premises and conclusion as complete sentences. Write the indicator words and phrases separately and state whether they are premise or conclusion indicators. 

\answer{Answers from Ben Sheredos.}

\begin{enumerate}[label=\arabic*), topsep=0pt, parsep=0pt, itemsep=6pt]
\item \textit{A pundit is speaking on a Sunday political talk show} Hillary Clinton should drop out of the race for Democratic Presidential nominee. For every day she stays in the race, McCain gets a day free from public scrutiny and the members of the Democratic party get to fight one another.  
			%Conclusion-first

\answer{ 
	\begin{earg*} 
		\item For every day Hillary stays in the race, McCain gets a day free from public scrutiny and the members of the Democratic party get to fight one another.
		\itemc Hillary Clinton should drop out of the race for Democratic Presidential Nominee.
	\end{earg*}
	"For" could be a premise-indicator, functioning like "since."
}


\item You have to be smart to understand the rules of Dungeons and Dragons. Most smart people are nerds. So, I bet most people who play D\&D are nerds.  
			%Conclusion-last

\answer{ 
			\begin{earg*} 
				\item You have to be smart to understand the rules of D\&D.
				\item Most smart people are nerds.
				\itemc $[I bet]$ most people who play D\&D are nerds.
			\end{earg*}
			"So" is definitely a conclusion-indicator; "I bet" is probably part of a conclusion-indicator as well, with the speaker indicating that they think this argument is a bit weak.
		}

\item Any time the public receives a tax rebate, consumer spending increases. Since the public just received a tax rebate, consumer spending will increase. 
		%Conclusion-last

\answer{ 
	\begin{earg*} 
		\item Any time the public receives a tax rebate, consumer spending increases. 
		\item The public just received a tax rebate.
		\itemc Consumer spending will increase.
	\end{earg*}
	"Since" is a premise-indicator, but the last sentence needs to be split up into premise and conclusion. This would be more clear if the speaker said "\underline{Since} the public just received a tax rebate, \underline{it follows that} consumer spending will increase." Our speaker is lazy.
}

\item Isabelle is taller than Jacob. Kate must also be taller than Jacob, because she is taller than Isabelle. 
%conclusion-middle

\answer{ 
	\begin{earg*} 
		\item Isabelle is taller than Jacob.
		\item Kate is taller than Isabelle.
		\itemc Kate is taller than Jacob.
	\end{earg*}
	"Must" is a conclusion indicator, "because" is a premise-indicator, and so the last sentence has to be split up to put this argument into standard form.
}
\end{enumerate}

% * **********************************
% *     Arguments and Nonarguments          *
% ************************************

\section{Arguments and Nonarguments}
\label{sec:arguments_and_nonarguments}

We just saw that arguments are made of statements. However, there are lots of other things you can do with statements. Part of learning what an argument is involves learning what an argument is not, so in this section and the next we are going to look at some other things you can do with statements besides make arguments. 

The list below of kinds of nonarguments is not meant to be exhaustive: there are all sorts of things you can do with statements that are not discussed. Nor are the items on this list meant to be exclusive. One passage may function as both, for instance, a narrative and a statement of belief. Right now we are looking at real world reasoning, so you should expect a lot of ambiguity and imperfection. If your class is continuing on into the critical thinking portions of this textbook, you will quickly get used to this. 

\subsection{Simple Statements of Belief}

\newglossaryentry{simple statement of belief}
{
name=simple statement of belief,
description={A kind of nonargumentative passage where the speaker simply asserts what they believe without giving reasons. }
}

An argument is an attempt to persuade an audience to believe something, using reasons. Often, though, when people try to persuade others to believe something, they skip the reasons, and give a \textsc{\gls{simple statement of belief}}. \label{def:simple_statement_of_belief} This is a kind of nonargumentative passage where the speaker simply asserts what they believe without giving reasons. Sometimes simple statements of belief are prefaced with the words ``I believe,'' and sometimes they are not. A simple statements of belief can be a profoundly inspiring way to change people's hearts and minds. Consider this passage from Dr. Martin Luther King's Nobel acceptance speech.

\begin{quotation} \noindent I believe that even amid today's mortar bursts and whining bullets, there is still hope for a brighter tomorrow. I believe that wounded justice, lying prostrate on the blood-flowing streets of our nations, can be lifted from this dust of shame to reign supreme among the children of men. I have the audacity to believe that peoples everywhere can have three meals a day for their bodies, education and culture for their minds, and dignity, equality and freedom for their spirits. \citep{King2001} \end{quotation}

This actually is a part of a longer passage that consists almost entirely of statements that begin with some variation of ``I believe.''It is incredibly powerful oration, because the audience, feeling the power of King's beliefs, comes to share in those beliefs. The language King uses to describe how he believes is important, too. He says his belief in freedom and equality requires audacity, making the audience feel his courage and want to share in this courage by believing the same things. 

These statements are moving, but they do not form an argument. None of these statements provide evidence for any of the other statements. In fact, they all say roughly the same thing, that good will triumph over evil. So the study of this kind of speech belongs to the discipline of rhetoric, not of logic.  
  
\subsection{Expository Passages}

Perhaps the most basic use of a statement is to convey information. Often if we have a lot of information to convey, we will sometimes organize our statements around a theme or a topic. Information organized in this fashion can often appear like an argument, because all of the statements in the passage relate back to some central statement. However, unless the other statements are given as reasons to believe the central statement, the passage you are looking at is not an argument. Consider this passage:

\begin{quotation}\noindent\textit{From a college psychology textbook.} Eysenck advocated three major behavior techniques that have been used successfully to treat a variety of phobias. These techniques are modeling, flooding, and systematic desensitization. In \textbf{modeling} phobic people watch nonphobics cope successfully with dreaded objects or situations.In \textbf{flooding} clients are exposed to dreaded objects or situations for prolonged periods of time in order to extinguish their fear. In contrast to flooding, \textbf{systematic desensitization} involves gradual, client-controlled exposure to the anxiety eliciting object or situation. (Adapted from Ryckman \cite*{Ryckman2007}) \end{quotation}

\newglossaryentry{expository passage}
{
name=expository passage,
description={A nonargumentative passage that organizes statements around a central theme or topic statement.}
}

We call this kind of passage an expository passage. In an \textsc{\gls{expository passage}}, \label{def:expository_passage} statements are organized around a central theme or topic statement. The topic statement might look like a conclusion, but the other statements are not meant to be evidence for the topic statement. Instead, they elaborate on the topic statement by providing more details or giving examples. In the passage above, the topic statement is ``Eysenck advocated three major behavioral techniques \ldots.'' The statements describing these techniques elaborate on the topic statement, but they are not evidence for it. Although the audience may not have known this fact about Eysenk before reading the passage, they will typically accept the truth of this statement instantly, based on the textbook's authority. Subsequent statements in the passage merely provide detail. 

Deciding whether a passage is an argument or an expository passage is complicated by the fact that sometimes people argue by example: 

\begin{adjustwidth}{2em}{0em}
\begin{longtabu}{p{.1\linewidth}p{.8\linewidth}}
\textbf{Steve:} & Kenyans are better distance runners than everyone else. \\
\textbf{Monica:} & Oh come on, that sounds like an exaggeration of a stereotype that isn't even true.\\
\textbf{Steve:} & What about Dennis Kimetto, the Kenyan who set the world record for running the marathon? And you know who the previous record holder was: Emmanuel Mutai, also Kenyan. \\
\end{longtabu}
\end{adjustwidth}
\vspace{-1.5cm}

Here Steve has made a general statement about all Kenyans. Monica clearly doubts this claim, so Steve backs it up with some examples that seem to match his generalization. This isn't a very strong way to argue: moving from two examples to statement about all Kenyans is probably going to be a kind of bad argument known as a hasty generalization. (This mistake is covered in the complete version of this text in the chapter on induction\nix{Chapter \ref{chap:induction} on induction.}\label{ver_var}) The point here however, is that Steve is just offering it as an argument. 

The key to telling the difference between expository passages and arguments by example is whether there is a conclusion that they audience needs to be convinced of. In the passage from the psychology textbook, ``Eysenck advocated three major behavioral techniques'' doesn't really work as a conclusion for an argument. The audience, students in an introductory psychology course, aren't likely to challenge this assertion, the way Monica  challenges Steve's overgeneralizing claim. 

Context is very important here, too. The Internet is a place where people argue in the ordinary sense of exchanging angry words and insults. In that context, people are likely to actually give some arguments in the logical sense of giving reasons to believe a conclusion. 

\subsection{Narratives} 

Statements can also be organized into descriptions of events and actions, as in this snippet from book V of \textit{Harry Potter}.

\begin{quotation} \noindent But she [Hermione] broke off; the morning post was arriving and, as usual, the \textit{Daily Prophet} was soaring toward her in the beak of a screech owl, which landed perilously close to the sugar bowl and held out a leg. Hermione pushed a Knut into its leather pouch, took the newspaper, and scanned the front page critically as the owl took off again. \citep{Rowling2003} \end{quotation} 

\newglossaryentry{narrative}
{
name=narrative,
description={A nonargumentative passage that describes a sequence of events or actions.}
}

We will use the term \textsc{\gls{narrative}} \label{def:narrative} loosely to refer to any passage that gives a sequence of events or actions. A narrative can be fictional or nonfictional. It can be told in regular temporal sequence or it can jump around, forcing the audience to try to reconstruct a temporal sequence. A narrative can describe a short sequence of actions, like Hermione taking a newspaper from an owl, or a grand sweep of events, like this passage about the  rise and fall of an empire in the ancient near east:

\begin{quotation}\noindent The Guti were finally expelled from Mesopotamia by the Sumerians of Erech (\textit{c}. 2100), but it was left to the kings of Ur's famous third dynasty to re-establish the Sargonoid frontiers and write the final chapter of the Sumerian History. The dynasty lasted through the twenty first century at the close of which the armies of Ur were overthrown by the Elamites and Amorites \citep{McEvedy1967}. \end{quotation} 

This passage does not feature individual people performing specific actions, but it is still united by character and action. Instead of Hermione at breakfast, we have the Sumarians in Mesopotamia. Instead of retrieving a message from an owl, the conquer the Guti, but then are conquered by the Elamites and Amorites. The important thing is that the statements in a narrative are not related as premises and conclusion. Instead, they are all events which are united common characters acting in specific times and places. 

%%%%%%% Practice Problems

\practiceproblems
\problempart Identify each passage below as an argument or a nonargument, and give reasons for your answers. If it is a nonargument, say what kind of nonargument you think it is. If it is an argument, write it out in standard form.

\begin{longtabu}{p{.1\linewidth}p{.9\linewidth}}
\textbf{Example}: & \textit{One student speaks to another student who has missed class:} The instructor passed out the syllabus at 9:30. Then he went over some basic points about reasoning, arguments and explanations. Then he said we should call it a day. \\
\textbf{Answer}: & Not an argument, because none of the statements provide any support for any of the others. This is probably better classified as a narration because the events are in temporal sequence. \\
\end{longtabu}

\begin{exercises}
%\item \textit{An anthropology teacher is speaking to her class }Different gangs use different colors to distinguish themselves. Here are some illustrations: biologists tend to wear some blue, while the philosophy gang wears black. 
%\answerblank{\\ Not an argument. Expository passage. The students probably will believe the teacher as soon as she makes an assertion. The word ``illustration'' is also a clue.}{\vspace{1.5in}}

\item \textit{From Bob Willis “Payrolls Jump Casts Doubt on Fed Rate Pledge” - Feb 3, 2012, in Bloomberg news.} The unemployment rate dropped to 8.3 percent, the lowest since February 2009, Labor Department figures showed today in Washington. The 243,000 increase in jobs was the biggest in nine months and exceeded the most optimistic forecast in a Bloomberg News survey. Service industries grew by the most in a year, according to a separate report.
\answerblank{\\ Not an argument. A report or narration.}{\vspace{1.5in}}

\item \textit{A caller on a radio call-in show has a theory.} The economy has been in trouble recently. And it's certainly true that cell phone use has been rising during that same period. So, I suspect increasing cell phone use is bad for the economy. 
\answerblank{\\  Argument. The indicator ``so'' is a clue. 
\begin{earg*}
\item The economy has been in trouble recently. 
\item Cell phone use has been rising during that same period. 
\itemc Cell phone use is bad for the economy. 
\end{earg*}
}{\vspace{1.5in}}


\item \textit{At Widget-World Corporate Headquarters:} We believe that our company must deliver a quality product to our customers. Our customers also expect first-class customer service. At the same time, we must make a profit. 

%\vspace{6pt}
\answerblank{ Not an argument. The speaker is not using any of the propositions as reasons to believe or explain any of the others; rather she is simply asserting various things.}{\vspace{1.5in}}
      
\item \textit{Jack is at the breakfast table and shows no sign of hurrying. Gill says:} You should leave now. It's almost nine a.m. and it takes three hours to get there.

\answerblank{Arguing. Jack's inaction suggests that he does believe that he needs to leave now and so Gill provides reasons that might convince him. Notice that there are no argument flag words or phrases.

This example also includes the word ``should'' in its conclusion. Words such as ``ought'' and ``should'' indicate that the speaker is trying to get the audience to do or believe something that they are not currently doing or believing.

\begin{earg*}
\item It's almost nine a.m. 
\item It takes three hours to get there.
\itemc  You should leave now.
\end{earg*}
}{\vspace{1.5in}}
      
\item \textit{In a text book on the brain:} Axons are distinguished from dendrites by several features, including shape (dendrites often taper while axons usually maintain a constant radius), length (dendrites are restricted to a small region around the cell body while axons can be much longer), and function (dendrites usually receive signals while axons usually transmit them).

\answerblank{Not an argument. Expository passage. The features named just fill in the first statement.}{\vspace{1.5in}}

\end{exercises}
%
\problempart Identify each passage below as an argument or a nonargument, and give reasons for your answers. If it is a nonargument, say what kind of nonargument you think it is. If it is an argument, write it out in standard form.

%
\begin{exercises}
\item \textit{Suzi doesn't believe she can quit smoking. Her friend Brenda says} Some people have been able to give up cigarettes by using their will-power. Everyone can draw on their will-power. So, anyone who wants to give up cigarettes can do so.

\item \textit{The words of the Preacher, son of David, King of Jerusalem} I have seen something else under the sun: The race is not to the swift or the battle to the strong, nor does food come to the wise or wealth to the brilliant or favor to the learned; but time and chance happen to them all. (Ecclesiastes 9:11, New International Version)

\item \textit{An economic development expert is speaking.} The introduction of cooperative marketing into Europe greatly increased the prosperity of the farmers, so we may be confident that a similar system in Africa will greatly increase the prosperity of our farmers.

\item \textit{From the CBS News website, US section.} Headline: ``FBI nabs 5 in alleged plot to blow up Ohio bridge.'' Five alleged anarchists have been arrested after a months-long sting operation, charged with plotting to blow up a bridge in the Cleveland area, the FBI announced Tuesday. CBS News senior correspondent John Miller reports the group had been involved in a series of escalating plots that ended with their arrest last night by FBI agents. The sting operation supplied the anarchists with what they thought were explosives and bomb-making materials. At no time during the case was the public in danger, the FBI said. \citep{CBSNews2012}


\item \textit{At a school board meeting.} Since creationism can be discussed effectively as a scientific model, and since evolutionism is fundamentally a religious philosophy rather than a science, it is unsound educational practice for evolution to be taught and promoted in the public schools to the exclusion or detriment of special creation. (Kitcher \cite*{Kitcher1982}, p. 177, citing Morris \cite*{Morris1975}.)

\end{exercises}

\iflabelexists{CTVersion}{text for CT version}{

% * **********************************
% *     Arguments and Explanations          *
% ************************************

\section{Arguments and Explanations}
\label{arguments_and_explanations}

Explanations are are not arguments, but they they share important characteristics with arguments, so we should devote a separate section to them. Both explanations and arguments are parts of reasoning, because both feature statements that act as reasons for other statements. The difference is that explanations are not used to convince an audience of a conclusion.  

Let's start with workplace example. Suppose you see your co-worker, Henry, removing a computer from his office. You think to yourself ``Gosh, is he stealing from work?'' But when you ask him about it later, Henry says, ``I took the computer because I believed that it was scheduled for repair.'' Henry's statement looks like an argument. It has the indicator word ``because'' in it, which would mean that the statement ``I believed it was scheduled for repairs'' would be a premise. If it was, we could put the argument in standard form, like this: 

\begin{earg}
\item[P:] I believed the computer was scheduled for repair
\vspace{-.5em}
\item [] \rule{0.6\linewidth}{.5pt} 
\item[C:] I took the computer from the office. 
\end{earg} 

But this would be awfully weird as an argument. If it were an argument, it would be trying to convince us of the conclusion, that Henry took the computer from the office. But you don't need to be convinced of this. You already know it---that's why you were talking to him in the first place. 
  
Henry is giving reasons here, but they aren't reasons that try to \textit{prove} something. They are reasons that \textit{explain} something. When you explain something with reasons, you increase your understanding of the world by placing something you already know in a new context. You already knew that Henry took the computer, but now you know \textit{why} Henry took the computer, and can see that his action was completely innocent (if his story checks out). 


\newglossaryentry{explanation}
{
name=explanation,
description={A kind of reasoning where reasons are used to provide a greater understanding of something that is already known.}
}

\newglossaryentry{explainer}
{
name=explainer,
description={The part of an explanation that provides greater understanding of the explainee.}
}

\newglossaryentry{explainee}
{
name=explainee,
description={The part of an explanation that one gains a greater understanding of as a result of the explainer.}
}

\newglossaryentry{reason}
{
name=reason,
description={The premise of an argument or the explainer in an explanation; the part of reasoning that provides logical support for the target proposition.}
}

\newglossaryentry{target proposition}
{
name=target proposition,
description={The conclusion of an argument or the explainee in an explanation; the part of reasoning that is logically supported by the reasons.}
}



Both arguments and explanations both involve giving reasons, but the reasons function differently in each case. An \textsc{\gls{explanation}} \label{def:explanation}is defined as a kind of reasoning where reasons are used to provide a greater understanding of something that is already known.  

Because both arguments and explanations are parts of reasoning, we will use parallel language to describe them. In the case of an argument, we called the reasons ``premises.'' In the case of an explanation, we will call them \textsc{\glspl{explainer}}. \label{def:explainer} Instead of a ``conclusion,'' we say that the explanation has an \textsc{\gls{explainee}}.  \label{def:explainee} We can use the generic term \textsc{\glspl{reason}} \label{def:reason} to refer to either premises or explainers and the generic term \textsc{\gls{target proposition}} \label{def:target_proposition} to refer to either conclusions or explainees. Figure \ref{fig:arguments_explanations} shows this relationship. 


\begin{figure}
\begin{mdframed}[style=mytableclearbox, userdefinedwidth=.65\textwidth]
\begin{tikzpicture}

\tikzstyle{mynode} = [rectangle, draw, fill=light-gray, rounded corners=3pt,] 

\path
	(0,0) node [mynode] (premises) {Premises}
	(0,-2) node [mynode] (conclusion) {Conclusion}
	(3,0) node  [mynode] (explainers) {Explainers}
	(3,-2) node [mynode] (explainee) {Explainee}
	(5,0) node  [anchor=west](reasons) {Reasons}
	(5,-2) node [anchor=west](target) {Target Proposition};

\tikzstyle{myblockarrow} = [thick, fill=light-gray,decoration={markings,mark=at position
   1 with {\arrow[semithick]{open triangle 60}}},
   double distance=2pt, shorten >= 5.5pt,
   preaction = {decorate},
   postaction = {draw,line width=2pt, white,shorten >= 4.5pt}]


\draw [myarrow2] (premises) to node [left] {\color{black}Prove} (conclusion);
\draw [myarrow2] (explainers) to node [right] {\color{black}Clarify} (explainee);
\draw [myarrow1, shorten >=.5cm] (reasons) to (explainers);
\draw [myarrow1, shorten >=.5cm] (target) to (explainee);


\end{tikzpicture}
\end{mdframed}
\caption{Arguments vs. Explanations.} \label{fig:arguments_explanations}
\end{figure}

We can put explanations in standard form, just like arguments, but to distinguish the two, we will simply number the statements, rather than writing Ps and Cs, and we will put an E next to the line that separates explainers and exaplainee, like this:

\begin{tikzpicture}
\path
	(0,0) node [anchor=west] {1. Henry believed the computer was scheduled for repair}
	(9,-8pt) node [anchor=west]{E}
	(0,-20pt) node[anchor=west] {2. Henry took the computer from the office.};
\draw (.5,-8pt) -- (9,-8pt);
\end{tikzpicture}

Cases where the target proposition is something that is completely common sense are clearcut cases of explanation. Consider the following passage.

\begin{quotation}
\noindent\textit{From Livescience, a science education website, under the headline “Why is grass green?”} Like many plants, most species of grass produce a bright pigment called chlorophyll. Chlorophyll absorbs blue light (high energy, short wavelengths) and red light (low energy, longer wavelengths) well, but mostly reflects green light, which accounts for your lawn's color. \citep{Mauk2013}
\end{quotation}

The passage contains reasoning. The nature of chlorophyll ``accounts for'' the color of grass. But in this case the audience does not need to be convinced that grass is green. Everyone knows that. The audience went to the Livescience website because they wanted an \emph{explanation} for why grass was green. 

Often the same piece of reasoning can work as either an argument or an explanation, depending on the situation where it is used. Consider this short dialogue

\begin{adjustwidth}{2em}{0em}
\begin{longtabu}{p{.1\linewidth}p{.8\linewidth}}
\multicolumn{2}{p{.9\linewidth}}{\textit{Monica visits Steve's cubical}.}\\
\textbf{Monica:} &All your plants are dead.\\
\textbf{Steve:} & It's because I never water them.
\end{longtabu}
\end{adjustwidth}
\vspace{-1cm}


In the passage above, Steve uses the word ``because,'' which we've seen in the past is a premise indicator word. But if it were a premise, the conclusion would be ``All Steve's plants are dead.'' But Steve can't possibly be trying to convince Monica that all his plants are dead. It is something that Monica herself says, and that they both can see. The ``because'' here indicates a reason, but here Steve is giving an explanation, not an argument. He takes something that Steve and Monica already know---that the plants are dead---and puts it in a new light by explaining how it came to be. In this case, the plants died because they didn't get water, rather than dying because they didn't get enough light or were poisoned by a malicious co-worker. The reasoning is best represented like this:

\begin{tikzpicture}
\path
	(0,0) node [anchor=west] {1. Steve never waters his plants.}
	(5.5,-8pt) node [anchor=west]{E}
	(0,-20pt) node[anchor=west] {2. All the plants are dead.};
\draw (.5,-8pt) -- (5.5,-8pt);
\end{tikzpicture}

But the same piece of reasoning can change form an explanation into an argument simply by putting it into a new situation:


\begin{adjustwidth}{2em}{0em}
\begin{longtabu}{p{.1\linewidth}p{.8\linewidth}}
\multicolumn{2}{p{.9\linewidth}}{\textit{Monica and Steve are away from the office}.}\\
\textbf{Monica:} &Did you have someone water your plants while you were away?\\
\textbf{Steve:}& No.\\
\textbf{Monica:}& I bet they are all dead.
\end{longtabu}
\end{adjustwidth}
\vspace{-1cm}

Here Steve and Monica do not know that Steve's plants are dead. Monica is inferring this idea based on the premise which she learns from Steve, that his plants are not being watered. This time ``Steve's plants are not being watered'' is a premise and ``The plants are dead'' is a conclusion. We represent the argument like this:

\begin{earg}
\item[P.] Steve never waters his plants. 
\vspace{-.5em}
\item [] \rule{0.3\linewidth}{.5pt} 
\item[C.] All the plants are dead. 
\end{earg}

In the example of Steve's plants, the same piece of reasoning can function either as an argument or an explanation, depending on the context where it is given. This is because the reasoning in the example of the plants is causal: the \textit{causes} of the plants dying are given as reasons for the death, and we can appeal to causes either to explain something that we know happened or to predict something that we think might have happened. 

Not all kinds of reasoning are flexible like that, however. Reasoning from authority can be used in some kinds of argument, but often makes a lousy explanation. Consider another conversation between Steve and Monica:

\begin{adjustwidth}{2em}{0em}
\begin{longtabu}{p{.1\linewidth}p{.8\linewidth}}
\textbf{Monica:} & I saw on a documentary last night that the universe is expanding and probably will keep expanding for ever. \\
\textbf{Steve:} & Really?\\
\textbf{Monica:} &Yeah, Steven Hawking said so. \\
\end{longtabu}
\end{adjustwidth}
\vspace{-1cm}

There aren't any indicator words here, but it looks like Monica is giving an argument. She states that the universe is expanding, and Steve gives a skeptical ``really?'' Monica then replies by saying that she got this information from the famous physicist Steven Hawking. It looks like Steve is supposed to believe that the universe will expand indefinitely because Hawking, an authority in the relevant field, said so. This makes for an ok argument: 

 \begin{earg}
\item[P:] Steven Hawking said that the universe is expanding and will continue to do so indefinitely.
\vspace{-.5em}
\item [] \rule{\linewidth}{.5pt} 
\item[C:] The universe is expanding and will continue to do so indefinitely.
\end{earg} 

Arguments from authority aren't very reliable, but for very many things they are all we have to go on. We can't all be experts on everything. But now try to imagine this argument as an explanation. What would it mean to say that the expansion of the universe can be \textit{explained} by the fact that Steven Hawking said that it should expand. It would be as if Hawking were a god, and the universe obeyed his commands! Arguments from authority are acceptable, but not ideal. Explanations from authority, on the other hand, are completely illegitimate. \label{no_exp_from_authority}

In general, arguments that appeal to how the world works are more satisfying than ones which appeals to the authority or expertise of others. Compare the following pair of arguments:

\begin{enumerate}[label=(\alph*)]
\item Jack says traffic will be bad this afternoon. So, traffic will be bad this afternoon. 
\item Oh no! Highway repairs begin downtown today. And a bridge lift is scheduled for the middle of rush hour. Traffic is going to be terrible \end{enumerate}

Even though the second passage is an argument, the reasons used to justify the conclusion could be used in an explanation. Someone who accepts this argument will also have an explanation ready to offer if someone should later ask ``Traffic was terrible today! I wonder why?''. This is not true of the first passage: bad traffic is not explained by saying ``Jack said it would be bad.'' The argument that refers to the drawbridge going up is appealing to a more powerful sort of reason, one that works in both explanations and arguments. This simply makes for a more satisfying argument, one that makes for a deeper understanding of the world, than one that merely appeals to authority. 

Although arguments based on explanatory premises are preferred, we must often rely on other people for our beliefs, because of constraints on our time and access to evidence. But the other people we rely on should hopefully hold the belief on the basis of an empirical understanding. And if \textit{those} people are just relying on authority, then we should hope that at some point the chain of testimony ends with someone who is relying on something more than mere authority. In [cross ref] we'll look more closely at sources and how much you should trust them.

We just have seen that they same set of statements can be used as an argument or an explanation depending on the context. This can cause confusion between speakers as to what is going on. Consider the following case:

\begin{adjustwidth}{2em}{0em}
\begin{longtabu}{p{.1\linewidth}p{.8\linewidth}}
\multicolumn{2}{p{.9\linewidth}}{\textit{Bill and Henry have just finished playing basketball.}}\\
\textbf{Bill:} & Man, I was terrible today. \\
\textbf{Henry:} & I thought you played fine. \\
\textbf{Bill:} & Nah. It's because I have a lot on my mind from work. \\
\end{longtabu}
\end{adjustwidth}
\vspace{-1cm}

Bill and Henry disagree about what is happening---arguing or explaining. Henry doubts Bill's initial statement, which should provoke Bill to argue. But instead, he appears to plough ahead with his explanation. What Henry can do in this case, however, is take the reason that Bill offers as an explanation (that Bill is preoccupied by issues at work) and use it as a premise in an argument for the conclusion ``Bill played terribly.'' Perhaps Henry will argue (to himself) something like this: ``It's true that Bill has a lot on his mind from work. And whenever a person is preoccupied, his basketball performance is likely to be degraded. So, perhaps he did play poorly today (even though I didn't notice).''

In other situations, people can switch back and forth between arguing and explaining. Imagine that Jones says ``The reservoir is at a low level because of several releases to protect the down-stream ecology.'' Jones might intend this as an explanation, but since Smith does not share the belief that the reservoir's water level is low, he will first have to be given reasons for believing that it is low. The conversation might go as follows:

\begin{adjustwidth}{2em}{0em}
\begin{longtabu}{p{.1\linewidth}p{.8\linewidth}}
\textbf{Jones:} & The reservoir is at a low level because of several releases to protect the down-stream ecology. \\
\textbf{Smith:} & Wait. The reservoir is low?\\
\textbf{Jones:} & Yeah. I just walked by there this morning. You haven't been up there in a while? \\
\textbf{Smith:} & I guess not. \\
\textbf{Jones:} & Yeah, it's because they've been releasing a lot of water to protect the ecology lately. \\
\end{longtabu}
\end{adjustwidth}
\vspace{-1cm}

When challenged, Smith offers evidence from his memory: he saw the reservoir that morning. Once Smith accepts that the water level is low, Jones can restate his explanation.

Some forms of explanation overlap with other kinds of nonargumentative passages. We are dealing right now with thinking in the real world, and as we mentioned on page \pageref{messiness_warning} the real world is full of messiness and ambiguity. One effect of this is that all the categories we are discussing will wind up overlapping. Narratives and expository passages, for instance, can also function as explanations. Consider this passage

\begin{quotation} \noindent\textit{From the sports section} Duke beat Butler 61-59 for the national championship Monday night. Gordon Hayward's half-court, 3-point heave for the win barely missed to leave tiny Butler one cruel basket short of the Hollywood ending. (Based on \cite{AP2010}) \end{quotation}

On the one hand, this is clearly a narrative---retelling a sequence of events united by time, place, and character. But it also can work as an explanation about how Duke won, if the audience immediately accepts the result. 'The last shot was a miss\textit{ }and then Duke won' can be understood as 'the last shot was a miss and so Duke won'.


%%%%%% Practice problems


\practiceproblems 
\problempart Identify each of the passages below as an argument, an explanation, or neither, and justify your answer. If the passage is an argument write it in standard form, with premises marked P$_1$ etc., then a line, and then the conclusion marked with a C. If the argument is an explanation, write it in the standard form for an explanation, with the explainers numbered and an ``E'' after the line that separates the explainers and the explainee. If the argument is neither an argument nor an explanation, state what kind of nonargument you think it is, such as a narrative or an expository passage.
 
\begin{longtabu}{p{.1\linewidth}p{.9\linewidth}}
\textbf{Example}: & \textit{Henry arrives at work late and wonders to himself: }Bill is not here. He very rarely arrives late. So, he is not coming in today. \\
\textbf{Answer}: & \textit{Argument} You can tell Henry is giving an argument to himself here because the conclusion is something that he did not already believe. \\
&\begin{earg}
\item[P$_1$:] Bill is not here. 
\item[P$_2$:] Bill very rarely arrives late. 
\vspace{-.5em}
\item [] \rule{0.6\linewidth}{.5pt} 
\item[C:] Bill is not coming in today
\end{earg} 
\end{longtabu}


\begin{exercises}

\item \textit{From a science education website run by NASA, also promoted by Google as the answer to the question “Why is the sky blue?”}  Blue light is scattered in all directions by the tiny molecules of air in Earth's atmosphere. Blue is scattered more than other colors because it travels as shorter, smaller waves. This is why we see a blue sky most of the time. \citep{NASA2015}

\answerblank{\vspace{6pt}This is an explanation, because the target proposition is common knowledge, as in the ``Grass is green'' example in your text. \vspace{6pt}

\begin{tikzpicture}
\path
	(0,0) node [anchor=west] {1. Blue travels in shorter wavelengths. }
	(0,-11pt) node [anchor=west] {2. Blue is scattered more than other colors.}
	(0,-22pt)  node [anchor=west] {3. Light is scattered by molecules in the atmosphere.}
	(9, -33pt) node [anchor=west] {E}
	(0, -44 pt) node[anchor=west] {4. We see a blue sky most of the time.};
\draw (.5,-33pt) -- (9,-33pt);
\end{tikzpicture}

\vspace{6pt}There are actually two separate levels of explanation in this passage, each marked with separate indicator words. The first ``because'' relates the sentence ``Blue travels in shorter wavelengths'' to the sentence ``Blue is scattered more than other colors.'' The short wavelengths explain why blue is scattered more. The fact that blue is scattered more and that light is scattered when it enters the atmosphere in turn explains why we see the sky as blue. If you wanted to be very precise, you would represent the explanations with two separate diagrams. First

\begin{tikzpicture}
\path
	(0,0) node [anchor=west] {1. Blue travels in shorter wavelengths. }
	(9, -11pt) node [anchor=west] {E}
	(0,-22pt) node [anchor=west] {2. Blue is scattered more than other colors.};
\draw (.5,-11pt) -- (9,-11pt);
\end{tikzpicture}

and then \vspace{6pt}

\begin{tikzpicture}
\path
	(0,0) node [anchor=west] {1. Blue is scattered more than other colors.}
	(0,-11pt) node [anchor=west] {2. Light is scattered by molecules in the atmosphere.}
	(9,-22pt)  node [anchor=west] {E}
	(0, -33 pt) node[anchor=west] {3. We see a blue sky most of the time.};
\draw (.5,-22pt) -- (9,-22pt);
\end{tikzpicture}

}{\vspace{1.5in}}



\item \textit{Jack is reading a popular science magazine. It reads: }Recent research has shown that people who rate themselves as ``very happy'' are less successful financially than those who rate themselves as ``moderately happy.'' \textit{He says,} ``Huh! It seems that a little unhappiness is good in life.''  


\answerblank{
\begin{earg*}
\item People who rate themselves as ``very happy'' are less successful financially than those who rate themselves as ``moderately happy.''
\itemc A little unhappiness is good in life.
\end{earg*}

Jack is arguing to himself. He makes an inference about happiness based on the premise he read in the magazine. Other people may have drawn a different conclusion from that premise. }{\vspace{1.5in}}

\item \textit{An anthropologist is speaking. }People get nicknames based on some distinctive feature they possess. And so, Mark, for example, who is 6'6'' is  (ironically) called ``Smalls'', while Matt, who looks young, is called ``Baby Face.'' John looks just like his dad, and is called ``Chip.'' 

\answerblank{\vspace{6pt}
Neither. Mark, Matt and John are examples of the general proposition stated in the first statement. However, the examples aren't being used to prove the first statement, nor do they explain why the first statement is true.}{\vspace{1.5in}} 




\item \textit{Two teenaged friends are talking. Analyze Saida's reasoning.}
\vspace{-6pt}
\begin{adjustwidth}{2em}{0em}
\begin{longtabu}{p{.1\linewidth}p{.8\linewidth}}
\textbf{Saida}: &I can't go to the show tonight. \\
\textbf{Jordan}:& Bummer. \\
\textbf{Saida}: &I know! My mother wouldn't let me go out when I asked. 
\end{longtabu}
\end{adjustwidth}
\vspace{-.9cm}

\answerblank{\vspace{6pt} Explanation or expository passage. It is not an argument because Jordan is going to believe Saida right away because they are friends. So Saida doesn't need to prove that she can't go to the show. Most likely this is an explanation. "My mother won't let me," \textit{explains} why she can't go. \\

\begin{tikzpicture}
\path
	(0,0) node [anchor=west] {1. My mother won't let me go to the show.}
	(9,-8pt) node [anchor=west]{E}
	(0,-19pt) node[anchor=west] {2. I can't go to the show.};
\draw (.5,-8pt) -- (9,-8pt);
\end{tikzpicture}}{\vspace{1.5in}}


\item \textit{A mother is speaking to her teenage son. }You should always listen 
to your mother. I say ``no\texttt.'' So, you have to stay in tonight. 

\answerblank{

\begin{earg*}
\item You should always listen to your mother. 
\item Your mother says ``no.'' 
\itemc You have to stay in tonight. 
\end{earg*}
Conclusion indicator word: So\\

Arguing. The mother is giving her son a reason to stay home--her authority. We will talk more about arguments from authority later.
}{\vspace{1.5in}}

\item \textit{An economist is speaking. }Any time the public receives a tax rebate, consumer spending increases. Since the public just received a tax rebate, consumer spending will increase. 


\answerblank{ 
\begin{earg*}
\item Any time the public receives a tax rebate, consumer spending increases. 
\item The public just received a tax rebate
\itemc Consumer spending will increase.
\end{earg*}

Premise indicator word: Since\\
\vspace{6pt}
Arguing. ``Since'' is a flag work here that shows some kind of inference is being made. But the increase in spending hasn't happened yet, so the audience needs to be convinced by the economist that it will happen. Therefore the passage is an argument. If the increase in spending had already happened, and the audience already believe it, then the economist would be explaining. In general, if the target proposition is a prediction, then the passage is likely to be an argument, because the audience doesn't already know the future. % (JRL)
}{\vspace{1.5in}}


\item \textit{In a letter to the editor. }Today's kids are all slackers. American 
society is doomed. 

\answerblank{
\begin{earg*}
\item Today's kids are all slackers. 
\itemc American society is doomed. 
\end{earg*}
Argument. This is an argument for the same reason as the last one. It is a prediction about the future.}{\vspace{1.5in}}

\item  \textit{On Monday, Jack is told that his unit ships to Iraq in two days: }I 
was hoping to go to Henry's birthday party next weekend. But I'm shipping out on Wednesday. So, I will miss it. 

\answerblank{
\begin{earg*}
\item I was hoping to go to Henry's birthday party next weekend. 
\item I'm shipping out on Wednesday. 
\itemc I will miss it.
\end{earg*}

Arguing. Jack learns a fact, in this case that he is shipping out, and then infers another fact, that he will miss the party. I have no idea why missing a party is the first thing on his mind when he is given this news. }{\vspace{1.5in}}%(JRL) 



\item \textit{A student is speaking to her instructor: }I was late for class because the battery in my mobile phone, which I was using as an alarm clock, ran out.
\answerblank{Explaining. The instructor already knows the student is late for class. \\

\begin{tikzpicture}
\path
	(0,0) node [anchor=west] {1. I use my mobile phone for an alarm clock.}
	(0,-11pt) node [anchor=west] {2. The battery on my phone ran out.}
	(9,-19pt) node [anchor=west]{E}
	(0,-30pt) node[anchor=west] {3. I was late to class.};
\draw (.5,-19pt) -- (9,-19pt);
\end{tikzpicture}
}{\vspace{1.5in}}

\item There is a lot of positive talk concerning parenthood because people tend to think about the positive effects that have a child brings and they tend to exclude the numerous negatives that it brings.
\answerblank{Explaining. The flag word ``because'' indicates reasoning and that the target comes first. The kind of reasoning here is likely explaining, because the target is a commonly held belief. \\

\begin{tikzpicture}
\path
	(0,0) node [anchor=west] {1. People tend to think about the positive effects that have a child brings.}
	(0,-11pt) node [anchor=west] {2. People tend to tend to exclude the numerous negatives that it brings.}
	(9,-19pt) node [anchor=west]{E}
	(0,-30pt) node[anchor=west] {3. There is a lot of positive talk concerning parenthood.};
\draw (.5,-19pt) -- (9,-19pt);
\end{tikzpicture}
}{\vspace{1.5in}}

\end{exercises}


\noindent\problempart Identify each of the passages below as an argument, an explanation, or neither, and justify your answer. If the passage is an argument write it in standard form, with premises marked P$_1$ etc., then a line, and then the conclusion marked with a C. If the argument is an explanation, write it in the standard form for an explanation, with the explainers numbered and an ``E'' after the line that separates the explainers and the explainee. If the argument is neither an argument nor an explanation, state what kind of nonargument you think it is, such as a narrative or an expository passage.

 
\begin{exercises}

\item You have to be smart to understand the rules of Dungeons and Dragons. Most smart people are nerds. So, I bet most people who play D\&D are nerds.

% use this passage as a basis for some problems
%Notice that knowledge of an explanation can be used (on a different occasion) to make an argument for the truth of a conclusion. For example, if extremely cold weather in Europe is explained by the movement of air from Siberia, on a future occasion the movement of air from Siberia can be used to argue that it is or will be extremely cold. 

%also this one
%
%\begin{quote} The IPCC, a panel of experts from various countries, has stated that human activity has an impact on climate. So, that's how it is.\end{quote}

%In this passage, a speaker provides a reason for believing \textit{that} human activity has an impact on climate, namely, that an international panel believes so. That is, the speaker provides a premise which might justify adopting the conclusion as a belief. This premise, however, it does not explain \textit{why }or \textit{how}human activity impacts climate. It might thus be a justification, but it could not be used as an explanation. If a speaker says something is so because some source says it, you are looking at an argument. 



\item \textit{A coach is emailing parents in a neighborhood youth soccer league.} The game is canceled since it is raining heavily.


\item  \textit{At the market. }You know, granola bars generally aren't healthy. The ingredients include lots of processed sugars.

\item  \textit{At the pet store.}
\vspace{-8pt}
\begin{adjustwidth}{2em}{0em}
\begin{longtabu}{p{.1\linewidth}p{.8\linewidth}}
\textbf{Salesman}:     &A small dog makes just as effective a guard dog for your 
home as a big dog.\\
\textbf{Henry}:        &   No way!\\
\textbf{Salesman}: &    It might seem strange. But smaller ``yappy'' dogs bark readily and they also generate distinctive higher-pitched sounds. Most of a dog's effectiveness as a guard is due to making a sound, not physical size. \end{longtabu}
\end{adjustwidth}
\vspace{-.9cm}

\item \textit{A child is thinking out loud. }I think my cat must be dead. It isn't in any of its usual places. And when I asked my mother if she had seen it, she couldn't look me in the eyes. 

\item {\color{white}flurm}
\vspace{-24pt}
\begin{adjustwidth}{2em}{0em}
\begin{longtabu}{p{.1\linewidth}p{.8\linewidth}}
\textbf{Smith:} & I can solve any puzzle more quickly than you.\\
\textbf{Jones:}& Get out of here. \\
\textbf{Smith:} & It's true! I'm a member of MENSA, and you're not. 
\end{longtabu}
\end{adjustwidth}
\vspace{-.9cm}

\item \textit{In the comments on a biology blog: }According to Darwin's theory, my ancestors were monkeys. But since that's ridiculous, Darwin's theory is false. 

\item If you believe in [the Christian] God and turn out to be incorrect, you have lost nothing. But if you don't believe in God and turn out to be incorrect, you will go to hell. Believing in God is better in both cases. One should therefore believe in God. (A formulation of ``Pascal's Wager'' by Blaise Pascal.) 

\item \textit{Bill and Henry are in Columbus.}
\vspace{-8pt}
\begin{adjustwidth}{2em}{0em}
\begin{longtabu}{p{.1\linewidth}p{.8\linewidth}}
\textbf{Bill:} & Good news---I just accepted a job offer in Omaha. \\
\textbf{Henry:} & That's great. Congratulations! I suppose this means you'll be leaving us, then?\\
\textbf{Bill:} & Yes, I'll need to move sometime before September.  \\
\end{longtabu}
\end{adjustwidth}
\vspace{-.9cm}

\item You already know that God kicked humanity out of Eden before they could eat of the tree of life but only after they had eaten of the tree of knowledge of good and evil. That was because Satan wanted to take over God's throne and was responsible for their eating from the tree. If humans had eaten of both trees they could have been a threat to God. 

\end{exercises}
}


% *****************************************
% *  		Recognizing Arguments in Wild		*
% *****************************************

% a section for working with newspapers and field projects.

%
%\section{Recognizing Arguments in Wild}
%%When faced with a passage or dialogue, you must first determine whether or not it contains reasoning, and in particular whether the reasons involved are reasons-to-believe or reasons-which-explain. 
%
%%reiterate flag words. 
% 
%       
%%      There are an infinitely large number of flag words and phrases.
%%
%%      These flag words and phrases indicate reasoning because they indicate premises or target, but they do not distinguish between arguing and explaining. Moreover, passages sometimes do not have any flag words. So, we need other ways of telling whether and what kind of reasoning is taking place.
%
%
%Discuss replacing pronouns, making each statement stand on its own, paraphrasing for length. Use lots of real world examples.
%
% 
%   Discuss reports of arguments here. 
%
%Arguments vs. explanations (again)
%
%``Should'' statements in the conclusion are generally a sign of arguing. Predictions are a sign of arguing.
%
%
%\begin{quote}Highway repairs begin downtown today. And a bridge lift is scheduled for the middle of rush hour. I predict that traffic is going to be terrible.\end{quote}
%
%\begin{quote}Yeah, I know traffic is going to be terrible. It's because repairs begin downtown today. And a bridge lift is scheduled for the middle of rush hour.\end{quote}
%
%The words ``I predict'' in the first passage suggest the conclusion is a novel belief, (in fact, it's novel even to the speaker). The second passage starts out with the speaker saying ``I know'' about what is clearly the target, because of the reasons offered subsequently. In the first, therefore, the speaker is making an inference and trying to convince someone (perhaps herself) that the proposition ``Traffic is going to be terrible.'' is true. The second, on the other hand, is an explanation. The speaker is not trying to increase her (or anyone else's) store of knowledge; she is trying to describe connections between states of affairs.
%
%      Sometimes you need to use your knowledge of what various specific people know, as well as your general knowledge of the knowledge that people have, including your knowledge of what you can reasonably expect people or different ages (children, teens, adults) or different backgrounds (people from your own country or region as opposed to foreigners) and so on. This is called epistemic score-keeping.
%      




\section*{Key Terms}
\begin{multicols}{2}
\begin{sortedlist}
\sortitem{Logic}{}
\sortitem{Metareasoning}{}
\sortitem{Metacognition}{} 	
\sortitem{Content neutrality}{}
\sortitem{Formal logic}{}
\sortitem{Critical thinking}{}
\sortitem{Informal logic}{}
\sortitem{Rhetoric}{}
\sortitem{Standard form}{}
\sortitem{Conclusion indicator}{} 
\sortitem{Premise indicator}{}
\sortitem{Statement}{}
\sortitem{Argument}{}
\sortitem{Conclusion}{}
\sortitem{Premise}{}
\sortitem{Inference}{}
\sortitem{Simple statement of belief}{}
\sortitem{Expository passage}{}
\sortitem{Narrative}{}
\sortitem{Explanation}{}
\sortitem{Explainer}{}
\sortitem{Explainee}{}
\sortitem{Reason}{}
\sortitem{Target proposition}{}
\sortitem{Critical thinker}{}
\sortitem{Practical argument}{}
\iflabelexists{CTVersion}{\sortitem{Descriptive statement}{}}{}
\iflabelexists{CTVersion}{\sortitem{Normative statement}{}}{}
\end{sortedlist}
\end{multicols}	
%\chapter{The Basics of Evaluating Argument}
\markright{Ch. \ref{chap:basicevaluation}: The Basics of Evaluating Argument}
\label{chap:basicevaluation}
\setlength{\parindent}{1em}

% **************************************************** 	
% *			Two ways that arguments can go wrong			*
% ****************************************************


\section{Two Ways an Argument Can Go Wrong}
\label{sec:two_ways}

Arguments are supposed to lead us to the truth, but they don't always succeed. There are two ways they can fail in their mission. First, they can simply start out wrong, using false premises. Consider the following argument. 

\begin{earg*}
\item It is raining heavily.
\item If you do not take an umbrella, you will get soaked.
\itemc You should take an umbrella.
\end{earg*}

If premise (1) is false---if it is sunny outside---then the argument gives you no reason to carry an umbrella.The argument has failed its job. Premise (2) could also be false: Even if it is raining outside, you might not need an umbrella. You might wear a rain poncho or keep to covered walkways and still avoid getting soaked. Again, the argument fails because a premise is false.

Even if an argument has all true premises, there is still a second way it can fail. Suppose for a moment that both the premises in the argument above are true. You do not own a rain poncho. You need to go places where there are no covered walkways. Now does the argument show you that you should take an umbrella? Not necessarily. Perhaps you enjoy walking in the rain, and you would like to get soaked. In that case, even though the premises were true, the conclusion would be false. The premises, although true, do not \emph{support} the conclusion. Back on page \pageref{def:Inference} we defined an inference, and said  it was like argument glue: it holds the premises and conclusion together. When an argument goes wrong because the premises do not support the conclusion, we say there is something wrong with the inference. %When there is something wrong with the inference, that means there is something wrong with the \emph{logical form} of the argument: Premises of the kind given do not necessarily lead to a conclusion of the kind given. We will be interested primarily in the logical form of arguments. We will learn to identify bad inferences by identifying bad logical forms. 

Consider another example: 

\begin{earg*}
\item You are reading this book.
\item This is a logic book.
\itemc[.3] You are a logic student.
\end{earg*}

This is not a terrible argument. Most people who read this book are logic students. Yet, it is possible for someone besides a logic student to read this book. If your roommate picked up the book and thumbed through it, they would not immediately become a logic student. So the premises of this argument, even though they are true, do not guarantee the truth of the conclusion. Its inference is less than perfect.

Again, for any argument, there are two ways that it could fail. First, one or more of the premises might be false.  Second, the premises might fail to support the conclusion. Even if the premises were true, the form of the argument might be weak, meaning the inference is bad.


%  ********************************************************
% *				Valid, Sound									* 
% ********************************************************

\section{Valid, Sound}

\newglossaryentry{valid}
{
name=valid,
description={A property of arguments where it is impossible for the premises to be true and the conclusion false.}
}

In logic, we are mostly concerned with evaluating the quality of inferences, not the truth of the premises. The truth of various premises will be a matter of whatever specific topic we are arguing about, and, as we have said, logic is content neutral.

The strongest inference possible would be one where the premises, if true, would somehow force the conclusion to be true. This kind of inference is called valid. There are a number of different ways to make this idea of the premises forcing the truth of the conclusion more precise. Here are a few:
 
An argument is valid if and only if\ldots 
\begin{enumerate}[label=(\alph*)]
\item it is impossible to consistently both (i) accept the premises and (ii) reject the conclusion

\item \label{itm:our_def} it is impossible for the premises to be true and the conclusion false

\item \label{itm:necessary} the premises, if true, would necessarily make the conclusion true.

\item \label{itm:imagination} the conclusion is true in every imaginable scenario in which the premises are true

\item \label{itm:story} it is impossible to write a consistent story (even fictional) in which the premises are true and the conclusion is false

\end{enumerate} 
 
In the glossary, we formally adopt item \ref{itm:our_def} as the definition for this textbook: an argument is \textsc{\gls{valid}} \label{def:valid} if and only if it is impossible for the premises to be true and the conclusion false.  However, nothing will really ride on the differences between the definitions in the list above, and we can look at all of them in order to give us a sense of what logicians mean when they use the term ``valid''.  
 
The important thing to see is that all the definitions in the list above try to get at what \textit{would} happen if the premises were true. None of them assert that the premises actually \textit{are} true. This is why definitions \ref{itm:imagination} and \ref{itm:story} talk about what would happen if you somehow \textit{pretend} the premises are true, for instance by telling a story. The argument is valid if, when you pretend the premises are true, you also have to pretend the conclusion is true. Consider the argument in Figure \ref{fig:Gaga_valid}


\begin{figure}
\begin{mdframed}[style=mytablebox]
\begin{earg*}
\item Lady Gaga is from Mars. 
\itemc[.4] Lady Gaga is from the fourth planet from our sun.
\end{earg*}
\end{mdframed}
\caption{A \textbf{valid} argument.} \label{fig:Gaga_valid}
\end{figure}

The American pop star Lady Gaga is not from Mars. (She's from New York City.) Nevertheless, if you imagine she's from Mars, you simply have to imagine that she is from the fourth planet from our sun, because mars simply is the fourth planet form our sun. Therefore this argument is valid. 

This way of understanding validity is based on what you can imagine, but not everyone is convinced that the imagination is a reliable tool in logic. That is why definitions like \ref{itm:necessary} and \ref{itm:our_def} talk about what is necessary or impossible. If the premises are true, the conclusion necessarily must be true. Alternately, it is impossible for the premises to be true and the conclusion false. The idea here is that instead of talking about the imagination, we will just talk about what can or cannot happen at the same time. The fundamental notion of validity remains the same, however: the truth of the premises would simply guarantee the truth of conclusion. 

So, assessing validity means wondering about whether the conclusion would be true \textit{if} the premises were true. This means that valid arguments can have false conclusions. This is important to keep in mind because people naturally tend to think that any argument must be good if they agree with the conclusion. And the more passionately people believe in the conclusion, the more likely we are to think that any argument for it must be brilliant. Conversely, if the conclusion is something we don't believe in, we naturally tend to think the argument is poor. And the more we don't like the conclusion, the less likely we are to like the argument. 

\newglossaryentry{fallacy}
{
name=fallacy,
plural=fallacies,
description={A common mistake in reasoning. Fallacies are generally conceived of as mistake forms of inference and are generally explained by arguments represented in canonical form. See also \emph{cognitive bias}.}
}

\newglossaryentry{myside fallacy}
{
name=myside fallacy,
description={The common mistake of evaluating an argument based merely on whether one agrees or disagrees with the conclusion.}
}


But this is not the way to evaluate inferences at all. The quality of the inference is entirely independent of the truth of the conclusion. You can have great arguments for false conclusions and horrible arguments for true conclusions. Confusing the truth of the conclusion with the quality of of the inference is a mistake in logic we can call the myside fallacy. A \textsc{\gls{fallacy}} \label{def:fallacy} is any common mistake in reasoning \iflabelexists{Chap:what_is_ct}{More detail on the nature of fallacies is given in Chapter \ref{Chap:what_is_ct}, on page \label{fallacy_detail}}{} The \textsc{\gls{myside fallacy}} \label{def:myside_fallacy} is specifically the common mistake of evaluating an argument based merely on whether one agrees or disagrees with the conclusion. You can also think of this as the fallacy of mistaking the conclusion for the argument.

An argument is valid if it is impossible for the premises to be true and the conclusion false. This means that you can have valid arguments with false conclusions, they just have to also have false premises. Consider the example in Figure \ref{fig:valid_oranges}


\begin{figure}[t]
\begin{mdframed}[style=mytablebox]
\begin{earg*}
\item Oranges are either fruits or musical instruments.
\item Oranges are not fruits.
\itemc Oranges are musical instruments.
\end{earg*}
\end{mdframed}
\caption{A \textbf{valid} argument} \label{fig:valid_oranges}
\end{figure}

\label{valid_arg_false_premises}

The conclusion of this argument is ridiculous. Nevertheless, it follows validly from the premises. This is a valid argument. \emph{If} both premises were true, \emph{then} the conclusion would necessarily be true.

This shows that a valid argument does not need to have true premises or a true conclusion. Conversely, having true premises and a true conclusion is not enough to make an argument valid. Consider the example in Figure \ref{fig:invalid_paris}


\begin{figure}[b]
\begin{mdframed}[style=mytablebox]
\begin{earg*}
\item London is in England.
\item Beijing is in China.
\itemc[.3] Paris is in France.
\end{earg*}
\end{mdframed}
\caption{An \textbf{invalid} argument.} \label{fig:invalid_paris}
\end{figure}
\label{invalid_true_premises_and_conclusion}


\newglossaryentry{invalid}
{
name=invalid,
description={A property of arguments that holds when the premises do not force the truth of the conclusion. The opposite of valid.}
}
 

The premises and conclusion of this argument are, as a matter of fact, all true. This is a terrible argument, however, because the premises have nothing to do with the conclusion. Imagine what would happen if Paris declared independence from the rest of France. Then the conclusion would be false, even though the premises would both still be true. Thus, it is \emph{logically possible} for the premises of this argument to be true and the conclusion false. The argument is not valid.  If an argument is not valid, it is called \textsc{\gls{invalid}}. \label{def:invalid} As we shall see, this term is a little misleading, because less than perfect arguments can be very useful. But before we do that, we need to look more at the concept of validity.

In general, then, the \textit{actual }truth or falsity of the premises, if known, do not tell you whether or not an inference is valid. There is one exception: when the premises are true and the conclusion is false, the inference cannot be valid, because valid reasoning can only yield a true conclusion when beginning from true premises. 
 
Figure \ref{fig:invalid_animals} has another invalid argument:

\begin{figure}
\begin{mdframed}[style=mytablehalfbox]
\begin{earg*}
\item All dogs are mammals
\item All dogs are animals
\itemc All animals are mammals.
\end{earg*}
\end{mdframed}
\caption{An \textbf{invalid} argument.} \label{fig:invalid_animals}
\end{figure}

In this case, we can see that the argument is invalid by looking at the truth of the premises and conclusion. We know the premises are true. We know that the conclusion is false. This is the one circumstance that a valid argument is supposed to make impossible. 

Some invalid arguments are hard to detect because they resemble valid arguments. Consider the one in Figure \ref{fig:invalid_stimulus}

\begin{figure}[b]
\begin{mdframed}[style=mytablebox]
\begin{earg*}
\item An economic stimulus package will allow the U.S. to avoid a depression. 
\item There is no economic stimulus package
\itemc[.3] The U.S. will go into a depression. 
\end{earg*}
\end{mdframed}
\caption{An \textbf{invalid} argument} \label{fig:invalid_stimulus}
\end{figure}


This reasoning is not valid since the premises do not \textit{definitively} support the conclusion. To see this, assume that the premises are true and then ask, "Is it possible that the conclusion could be false in such a situation?". There is no inconsistency in taking the premises to be true without taking the conclusion to be true. The first premise says that the stimulus package will allow the U.S. to avoid a depression, but it does not say that a stimulus package is the \textit{only }way to avoid a depression. Thus, the mere fact that there is no stimulus package does not necessarily mean that a depression will occur. 

Here is another, trickier, example. I will give it first in ordinary language. 

\begin{quotation} \noindent\textit{A pundit is speaking on a cable news show} If the U.S. economy were in recession and inflation were running at more than 4\%, then the value of the U.S. dollar would be falling against other major currencies. But this is not happening --- the dollar continues to be strong. So, the U.S. is not in recession. \end{quotation}

The conclusion is "The U.S. economy is not in recession." If we put the argument in canonical form, it looks like figure \ref{fig:invalid_recession}

\begin{figure}
\begin{mdframed}[style=mytablebox]
\begin{earg*}
\item If the U.S. were in a recession with more than 4\% inflation, then the dollar would be falling
\item The dollar is not falling
\itemc[.3] The U.S. is not in a recession. 
\end{earg*}
\end{mdframed}
\caption{An \textbf{invalid} argument} \label{fig:invalid_recession}
\end{figure}

The conclusion does not follow necessarily from the premises. It does follow necessarily from the premises that (i) the U.S. economy is not in recession or (ii) inflation is running at more than 4\%, but they do not guarantee (i) in particular, which is the conclusion. For all the premises say, it is possible that the U.S. economy is in recession but inflation is less than 4\%. So, the inference does not \textit{necessarily} establish that the U.S. is not in recession. A parallel inference would be "Jack needs eggs and milk to make an omelet. He can't make an omelet. So, he doesn't have eggs.". 

\newglossaryentry{sound}
{
name=sound,
description={A property of arguments that holds if the argument is valid and has all true premises.}
}

If an argument is not only valid, but also has true premises, we call it \textsc{\gls{sound}}. \label{def:sound} ``Sound'' is the highest compliment you can pay an argument. If logic is the study of virtue in argument, sound arguments are the most virtuous. We said in Section \ref{sec:two_ways} that there were two ways an argument could go wrong, either by having false premises or weak inferences. Sound arguments have true premises and undeniable inferences. If someone gives a sound argument in a conversation, you have to believe the conclusion, or else you are irrational.  

The argument on the left in Figure \ref{fig:valid_sound} is valid, but not sound. The argument on the right is both valid and sound.

\begin{figure}[b]
\begin{mdframed}[style=mytablebox]
\begin{longtabu}{X[l,c]X[l,c]}
\vspace{-16pt}
\begin{earg*}
\item Socrates is a person.
\item All people are carrots.
\itemc[.5] Therefore, Socrates is a carrot.
\end{earg*}
&
\vspace{-16pt}
\begin{earg*}
\item Socrates is a person.
\item All people are mortal.
\itemc[.5] Therefore, Socrates is mortal.
\end{earg*}
\\
\textbf{Valid, but not sound}&
\textbf{Valid and sound}
\end{longtabu}
\end{mdframed}
\caption{These two arguments are valid, but only the one on the right is sound} \label{fig:valid_sound}
\end{figure}

Both arguments have the exact same form. They say that a thing belongs to a general category and everything in that category has a certain property, so the thing has that property. Because the form is the same, it is the same valid inference each time. The difference in the arguments is not the validity of the inference, but the truth of the second premise. People are not carrots, therefore the argument on the left is not sound. People are mortal, so the argument on the right is sound. 

Often it is easy to tell the difference between validity and soundness if you are using completely silly examples. Things become more complicated with false premises that you might be tempted to believe, as in the argument in Figure \ref{fig:valid_unsound}.

\begin{figure}
\begin{mdframed}[style=mytablehalfbox]
\begin{earg*}
\item Every Irishman drinks Guinness
\item Smith is an Irishman
\itemc Smith drinks Guinness.
\end{earg*}
\end{mdframed}
\caption{An argument that is \textbf{valid} but not \textit{sound}} \label{fig:valid_unsound}
\end{figure}


You might have a general sense that the argument in Figure \ref{fig:valid_unsound} is bad---you shouldn't assume that someone drinks Guinness just because they are Irish. But the argument is completely valid (at least when it is expressed this way.) The inference here is the same as it was in the previous two arguments. The problem is the first premise. Not all Irishmen drink Guinness, but if they did, and Smith was an Irishman, he would drink Guinness. 

The important thing to remember is that validity is not about the actual truth or falsity of the statements in the argument. Instead, it is about the way the premises and conclusion are put together. It is really about the \emph{form} of the argument. A valid argument has perfect logical form. The premises and conclusion have been put together so that the truth of the premises is incompatible with the falsity of the conclusion. 

A general trick for determining whether an argument is valid is to try to come up with just one way in which the premises could be true but the conclusion false. If you can think of one (just one! anything at all! but no violating the laws of physics!), the reasoning is \textit {invalid.}    
 

% Practice Problems %%%%%%%%%%%%%%%

\practiceproblems

\noindent\problempart  Put the following arguments in canonical form and then decide whether they are valid. If the argument is invalid, explain why.

\begin{longtabu}{p{.1\linewidth}p{.9\linewidth}}
\textbf{Example}: & \textit{Monica is looking for her coworker} Jack is in his office. Jack's office is on the second floor. So, Jack is on the second floor. \\
\textbf{Answer}: & \vspace{-16pt}
\begin{earg*}
\item Jack is in his office. 
\item Jack's office is on the second floor.
\itemc Jack is on the second floor.
\end{earg*}
Valid
\\
\end{longtabu}

\begin{exercises}
\item All dinosaurs are people, and all people are fruit. Therefore all dinosaurs are fruit. 

\answer{
\begin{earg*}
\item All dinosaurs are people
\item All people are fruit. 
\itemc[.4] All dinosaurs are fruit. 
\end{earg*}

\underline{Valid}}
%% F, F, conclusion last

\item All dogs are mammals. Therefore, Fido is a mammal, because Fido is a dog.  

\answer{
\begin{earg*}
\item All dogs are mammals
\item Fido is a dog.   
\itemc[.4] Fido is a mammal, 
\end{earg*}

\underline{Valid}}
%%Made up, conclusion middle

\item Abe Lincoln must have been from France, because he was either from France or from Luxemborg, and we know was not from Luxemborg. 
\answer{

\begin{earg*}
\item Abe Lincoln was either from France or from Luxemborg
\item Abe Lincoln was not from Luxemborg. 
\itemc[.4] Abe Lincoln was from France. 
\end{earg*}

\underline{Valid}}
%%F, F, conclusion first

\item Love is blind. God is love. Ray Charles is blind. Therefore, Ray Charles is God
\answer{
\begin{earg*}
\item  Love is blind.
\item  God is love. 
\item  Ray Charles is blind. 
\itemc[.4]  Ray Charles is God
\end{earg*}
\underline{Invalid}
Inductive fallacy--equivocation, }%conclusion last

\item If the world were to end today, then I would not need to get up tomorrow morning. I will need to get up tomorrow morning. Therefore, the world will not end today.

\answer{
\begin{earg*}
\item If the world were to end today, then I would not need to get up tomorrow morning.
\item I will need to get up tomorrow morning.
\itemc[.4]  The world will not end today.
\end{earg*}
\underline{Valid}}
%%Made up

\item All people are mortal. Socrates is mortal. Therefore all people are Socrates. 
\answer{
\begin{earg*}
\item  All people are mortal.
\item Socrates is mortal. 
\itemc[.4]  All people are Socrates. 
\end{earg*}
\underline{Invalid}}
%%Formal fallacy


\item \textit{A forest ranger is surveying the park} I can tell that bears have been down by the river, because there are tracks in the mud. Tracks like these are made by bears in almost every case. 

\answer{
\begin{earg*}
\item  There are tracks in the mud.
\item Tracks like these are made by bears in almost every case. 
\itemc[.4]  Bears have been down by the river
\end{earg*}
\underline{Invalid}}
%%Inductive, conclusion first 

\item If the triceratops were a dinosaur, it would be extinct. Therefore, the triceratops is extinct, because the triceratops was a dinosaur. 
\answer{
\begin{earg*}
\item  If the triceratops were a dinosaur, it would be extinct.
\item The triceratops is extinct 
\itemc[.4] The triceratops was a dinosaur. 
\end{earg*}

\underline{Valid}}
%%T, T, conclusion middle

\item If George Washington was assassinated, he is dead. George Washington is dead. Therefore George Washington was assassinated.
\answer{
\begin{earg*}
\item  If George Washington was assassinated, he is dead.
\item George Washington is dead.
\itemc[.4] George Washington was assassinated.
\end{earg*}
\underline{Invalid}}
%% Formal fallacy
%
\item Jack prefers Pepsi to Coke. After all, about 52\% of people prefer Pepsi to Coke, and Jack is a person. 

\answer{
\begin{earg*}
\item  About 52\% of people prefer Pepsi to Coke
\item Jack is a person. 
\itemc[.4] Jack prefers Pepsi to Coke.
\end{earg*}

\underline{invalid}}
%%Inductive, conclusion first 
\end{exercises}

\noindent\problempart Put the following arguments in canonical form and then decide whether they are valid. If the argument is invalid, explain why.
\answer{Answers by Ben Sheredos}
\begin{exercises}
\item Cindy Lou Who lives in Whoville. You can tell because Cindy Lou Who is a Who, and all Whos live in Whoville.  

\answer{ 
	\begin{earg*} 
		\item Cindy Lou Who is a Who.
		\item All Whos live in Whoville.
		\itemc Cindy Lou Who lives in Whoville.
	\end{earg*}
Valid}
%Made up, conclusion first

\item If Frog and Toad like each other, they are friends. Frog and Toad like each other. Therefore, Frog and Toad are friends. 
\answer{
	\begin{earg*} 
		\item If Frog and Toad like each other, they are friends.
		\item Frog and Toad like each other.
		\itemc Frog and Toad are friends.
	\end{earg*}
Valid}
%Made up

\item If Cindy Lou Who is no more than two, then she is not five years old. Cindy Lou Who is not five. Therefore Cindy Lou Who is two or more.
\answer{
	\begin{earg*} 
		\item If Cindy Lou Who is no more than two, then she is not five years old.
		\item Cindy Lou Who is not five.
		\itemc Cindy Lou Who is two or more.
	\end{earg*}
Invalid. This starts out as a formal fallacy, affirming the consequent. Then it goes even further wrong by swapping "no more than two" for "two or more."}
% Formal fallacy

\item \textit{Jack's suspicious house mate is in the kitchen} Jack has moved my leftover slice of pizza. Jack must have moved it, because Jack is the only person who has been in the house, and the pizza is no longer in the fridge.
\answer{ 
	\begin{earg*} 
		\item Jack is the only person who has been in the house, and the pizza is no longer in the fridge,
		\itemc Jack has moved my leftover slice of pizza.
	\end{earg*}
Alternatively:
	\begin{earg*} 
	\item Jack is the only person who has been in the house.
	\item The pizza is no longer in the fridge.
	\itemc Jack has moved my leftover slice of pizza.
	\end{earg*}
Invalid. Maybe pets or robots can open the fridge? Or possibly, Jack opened the fridge, but did so with his cell phone is his hand and right at that moment received a funny video from a friend and left the fridge door open while he watched it, allowing the dog to steal the pizza. Not likely, admittedly, but if it could happen, the argument is \underline{not valid}.}
% Inductive, conclusion first
 
\item Jack is Smith's work colleague. So, Jack and Smith are friends.
\answer{ 
	\begin{earg*} 
		\item Jack is Smith's work colleague.
		\itemc Jack and Smith are friends.
	\end{earg*}
Invalid. Not all coworkers are friends.}

%Inductive 

\item Abe Lincoln was either born in Illinois or he was once president. Therefore Abe Lincoln was born in Illinois, because he was never president. 
\answer{ 
	\begin{earg*} 
		\item Lincoln was either born in Illinois or he was once president.
		\item Lincoln was never president.
		\itemc Lincoln was born in Illinois.
	\end{earg*}
Valid. Probably every statement here is false, but what matters is that IF the premises were true, the conclusion would have to be true.}
%F, T, conclusion middle

\item Politicians get a generous allowance for transportation costs. Enda Kenny is a politician. Therefore Kenny gets a generous transportation allowance.
\answer{ 
	\begin{earg*} 
		\item Politicians get a generous allowance for transportation costs.
		\item Enda Kenny is a politician.
		\itemc Kenny gets a generous transportation allowance.
	\end{earg*}
Valid. If the plural "politicians" is understood to mean "all politicians" rather than "most", the inference is valid. If you wrote "invalid" and explained that you thought "politicians" only meant "most politicians," that would be OK, as long as you made it clear. English is ambiguous like that.}

% Valid. If the plural "politicians" is understood to mean "all politicians" rather than "most", the inference is valid.
%might as well be made up

\item Jones is taller than Bill, because Smith is taller than Jones and Bill is shorter than Smith. 
\answer{
	\begin{earg*} 
		\item Smith is taller than Jones.
		\item Bill is shorter than Smith.
		\itemc Jones is taller than Bill.
	\end{earg*}
Invalid. Jones and Bill could be the same height.}

%Formal fallacy, conclusion first

\item If grass is green, then I am the pope. Grass is green. So, I am the pope.
\answer{
	\begin{earg*} 
		\item If grass is green, then I am the pope.
		\item Grass is green.
		\itemc I am the pope.
	\end{earg*}
Valid. IF premises are true, conclusion has to be.}

%F, F

\item Smith is paid more than Jack. They are both paid weekly. So, Smith has more money than Jack.
\answer{
	\begin{earg*} 
		\item Smith is paid more than Jack.
		\item Both Smith and Jones are paid weekly.
		\itemc Smith has more money than Jack.
	\end{earg*}
Invalid. There are sources of wealth other than what one is paid.}% Weak
\end{exercises}

\noindent\problempart Put the following arguments in canonical form and then decide whether they are valid. If the argument is invalid, explain why.

\begin{exercises}
\item Jack is close to the pond. The pond is close to the playground. So, Jack is close to the playground.
\answer{
\begin{earg*}
\item  Jack is close to the pond. 
\item The pond is close to the playground. 
\itemc[.4] Jack is close to the playground.
\end{earg*}
\underline{Invalid}}
%% General fallacy

\item \textit{Jack is at work: }I have up to half an hour to get to the bank, because work ends at 5:00 and the bank closes at 5:30. 
\answer{
\begin{earg*}
\item  Work ends at 5:00
\item The bank closes at 5:30. 
\itemc[.4] I have up to half an hour to get to the bank
\end{earg*}

\underline{Valid}}
%% Made up, conclusion first.

\item Jack and Gill ate at Guadalajara restaurant earlier and both of them feel nauseated now. So, something they ate there is making them sick.
\answer{
\begin{earg*}
\item  Jack and Gill ate at Guadalajara restaurant earlier
\item  Jack and Gill feel nauseated now. 
\itemc[.4] Something they ate there is making them sick.
\end{earg*}

\underline{ Invalid}}
%% Inductive

\item Zhaoquing must be west of Huizhou, because Zhaoquing is west of Guangzhou, which is west of Huizhou. 
\answer{
\begin{earg*}
\item   Zhaoquing is west of Guangzhou
\item  Guangzhou is west of Huizhou.  
\itemc[.4] Zhaoquing is west of Huizhou
\end{earg*}
\underline{Valid}}
%%T, T (?), conclusion first
%
\item \textit{Henry can't find his glasses. }I remember I had them when I came in from the car. So, they are in the house somewhere.
\answer{
\begin{earg*}
\item   Henry had his glasses when he came in from the car
\itemc[.4]  
\end{earg*} His glasses are in the house somewhere.
\underline{Invalid}}
% False dilemma
%% General fallacy

\item I was talking about tall John---the one who is over 6'4''---but Jack was talking about short John, who is at most 5'2''. So, we were talking about two different Johns.
\answer{
\begin{earg*}
\item I was talking about tall John  
\item   Jack was talking about short John
\itemc[.4]  We were talking about two different Johns.
\end{earg*}\underline{Valid}}
%% Made up

\item Tomorrow's trip to Ensenada will take about 10 hours, because the last time I drove there from here it took 10 hours. 
\answer{
\begin{earg*}
\item   The last time I drove to Ensenada from here it took 10 hours. 
\itemc[.4]  Tomorrow's trip to Ensenada will take about 10 hours
\end{earg*}\underline{Invalid}}
%%Induction, conclusion first.

\end{exercises}

\noindent\problempart Put the following arguments in canonical form and then decide whether they are valid. If the argument is invalid, explain why.
\answer{Answers by Ben Sheredos}
\begin{exercises}
\item \textit{Monica is surveying the crowd that showed up for her talk} There must be at least 150 people here. That's how many people the auditorium holds, and every seat is full and people are beginning to sit on the stairs at the side. 
\answer{
	\begin{earg*} 
		\item The auditorium holds 150 people.
		\item Every seat in the auditorium is full and people are beginning to sit on the stairs at the side.
		\itemc There are at least 150 people here.
	\end{earg*}
Valid}
% Made up, conclusion first

\item The fire bell in the building is ringing. There is sometimes a fire in the building when the alarm goes off. So, there is a fire.
\answer{
	\begin{earg*} 
		\item The fire bell in the building is ringing.
		\item There is sometimes a fire in the building when the alarm goes off.
		\itemc There is a fire (in the building).
	\end{earg*}
Invalid. "Sometimes" isn't "always," so the conclusion is not necessarily true.}
% Inductive

\item I cannot drive on the motorways yet, because I just passed my driving test and anyone who passes can drive on the roads but not on the motorway for six months.  
\answer{
	\begin{earg*} 
		\item I just passed my driving test.
		\item Anyone who passes can drive on the roads but not on the motorway for 6 months.
		\itemc I cannot drive on the motorways yet.
	\end{earg*}
Valid; it's implied pretty strongly that "just passing" means "passed within the past 6 months." There is room for equivocation here, but it looks pretty solid.}

% Made up

\item Yesterday's the temperature reached 91 degrees Fahrenheit. Today it is 94. So, today is warmer than yesterday.
\answer{
	\begin{earg*} 
		\item Yesterday the temp. reached 91F.
		\item Today the tepm. is 94F.
		\itemc Today is warmer than yesterday.
	\end{earg*}
Valid, unless the speaker inexplicably changes to a Celsius scale or something, but more likely the idea is that they just told you what scale they were using, and so they don't repeat it.}
% Made up

\item  My car is functioning well at the moment. So, all of the parts in my car are functioning well.
\answer{
	\begin{earg*} 
		\item My car is functioning well at the moment.
		\itemc All the parts of my car are functioning well.
	\end{earg*}
Probably invalid -- probably a fallacy of "Composition and Division." Suppose the speaker added, as P2: ''I mean, the antenna fell off, so I can't listen to Jazz 98.3, but I'll fix that later'' We wouldn't jump on them and say ''\textit{A-HA!} so your car \textit{isn't} functioning well!''}

\item It has been sunny every day for the last five days. So, it will be sunny today.

\answer{
	\begin{earg*} 
		\item It has been sunny every day for the past 5 days.
		\itemc It will be sunny today.
	\end{earg*}
Not valid in the logical sense defined here. Five days in a row is no guarantee that the sixth day will be the same.}

% Inductive

\item Jack is in front of Gill. So, Gill is behind Jack.
\answer{
	\begin{earg*} 
		\item Jack is front of Gill.
		\itemc Gill is behind Jack.
	\end{earg*}
Valid}

% made up

\item \textit{Gill is returning home}: The door to my house is still locked. So, my possessions are still inside.
\answer{
	\begin{earg*} 
		\item The door to my house is still locked.
		\itemc My possessions are still inside.
	\end{earg*}
Invalid. Oh simple, naive Gil. A pro would definitely pick the lock, rob you blind, and lock the door on the way out so as not to arouse suspicions. By now your possessions have been pawned, and the thief is halfway to Vegas.}

%Inductive.

\end{exercises}




% *******************************************
% *			Strong, Cogent, Deductive, Inductive	    *	
% *******************************************


\section{Strong, Cogent, Deductive, Inductive}

We have just seen that sound arguments are the very best arguments. Unfortunately, sound arguments are really hard to come by, and when you do find them, they often only prove things that were already quite obvious, like that Socrates (a dead man) is mortal. Fortunately, arguments can still be worthwhile, even if they are not sound. Consider this one:

\begin{earg*}
\item In January 1997, it rained in San Diego.
\item In January 1998, it rained in San Diego.
\item In January 1999, it rained in San Diego.
\itemc[.6] It rains every January in San Diego.
\end{earg*}


This argument is not valid, because the conclusion could be false even though the premises are true. It is possible, although unlikely, that it will fail to rain next January in San Diego. Moreover, we know that the weather can be fickle. No amount of evidence should convince us that it rains there \emph{every} January. Who is to say that some year will not be a freakish year in which there is no rain in January in San Diego? Even a single counterexample is enough to make the conclusion of the argument false.

\newglossaryentry{strong}
{
name=strong,
description={A property of arguments which holds when the premises, if true, mean the conclusion must be likely to be true.}
}


\newglossaryentry{cogent}
{
name=cogent,
description={A property of arguments that holds when the argument is strong and the premises are true.}
}



\newglossaryentry{weak}
{
name=weak,
description={A property of arguments that are neither valid nor strong. In a weak argument, the premises would not even make the conclusion likely, even if they were true.}
}



Still, this argument is pretty good. Certainly, the argument could be made stronger by adding additional premises: In January 2000, it rained in San Diego. In January 2001$\ldots$ and so on. Regardless of how many premises we add, however, the argument will still not be deductively valid. Instead of being valid, this argument is strong. An argument is \textsc{\gls{strong}} \label{def:strong} if the premises would make the conclusion more likely, were they true. In a strong argument, the premises don't guarantee the truth of the conclusion, but they do make it a good bet. If an argument is strong, and it has true premises, we say that it is \textsc{\gls{cogent}} \label{def:cogent} Cogency is the equivalent of soundness in strong arguments. If an inference is neither valid, nor strong, we say it is \textsc{\gls{weak}}. \label{def:weak}In a weak argument, the premises would not even make the conclusion likely, even if they were true.

You may have noticed that the word ``likely'' is a little vague. How likely do the premises have to make the conclusion before we can count the argument as strong? The answer is a very unsatisfying ``it depends.'' It depends on what is at stake in the decision to believe the conclusion. What happens if you are wrong? What happens if you are right? The phrase ``make the conclusion a good bet'' is really quite apt. Whether something is a good bet depends a lot on how much money is at stake and how much you are willing to lose. Sometimes people feel comfortable taking a bet that has a 50\% chance of doubling their money, sometimes they don't. 

The vagueness of the word ``likely'' brings out an interesting feature of strong arguments: some strong arguments are stronger than others. The argument about rain in San Diego, above, has three premises referring to three previous Januaries. The argument is pretty strong, but it can become stronger if we go back farther into the past, and find more years where it rains in January. The more evidence we have, the better a bet the conclusion is. Validity is not like this. Validity is a black-or-white matter. You either have it, and you're perfect, or you don't, and you're nothing. There is no point in adding premises to an argument that is already valid. 

\newglossaryentry{deductive}
{
name=deductive,
description={A style of arguing where one attempts to use valid arguments.}
}

\newglossaryentry{inductive}
{
name=inductive,
description={A style of arguing where one attempts to use strong arguments.}
}

Arguments that are valid, or at least try to be, are called \textsc{\gls{deductive}} \label{def:deductive}, and people who attempt to argue using valid arguments are said to be arguing \textit{deductively.} The notion of validity we are using here is, in fact, sometimes called \textit{deductive validity}. Deductive argument is difficult, because, as we said, in the real world sound arguments are hard to come by, and people don't always recognize them as sound when they find them. Arguments that purport to merely be strong rather than valid are called \textsc{\gls{inductive}}. \label{def:inductive} The most common kind of inductive argument includes arguments like the one above about rain in San Diego, which generalize from many cases to a conclusion about all cases.

Deduction is possible in only a few contexts. You need to have clear, fixed meanings for all of your terms and rules that are universal and have no exceptions.   One can find situations like this if you are dealing with things like legal codes, mathematical systems or logical puzzles. One can also create, as it were, a context where deduction is possible by imagining a universal, exceptionless rule, even if you know that no such rule exists in reality. In the example above about rain in San Diego, we can change the argument from inductive to deductive by adding a universal, exceptionless premise like ``It always rains in January in San Diego.'' This premise is unlikely to be true, but it can make the inference valid. (For more about trade offs between the validity of the inference and the truth of the premise, see the chapter on incomplete arguments in the complete version of this text. \label{ver_var} \nix{Chapter \ref{chap:incomplete_arguments}}

Here is an example in which the context is an artificial code --- the tax code: 

\begin{quotation} \noindent\textit{From a the legal code posted on a government website} A tax credit for energy-efficient home improvement is available at 30\% of the cost, up to \$1,500 total, in 2009 \& 2010, ONLY for existing homes, NOT new construction, that are your "principal residence" for Windows and Doors (including sliding glass doors, garage doors,~storm doors and storm windows), Insulation, Roofs (Metal and Asphalt), HVAC: Central Air Conditioners, Air Source Heat Pumps, Furnaces and Boilers, Water Heaters: Gas, Oil, \& Propane Water Heaters, Electric Heat Pump Water Heaters, Biomass Stoves. \end{quotation}

This rule describes the conditions under which a person can or cannot take a certain tax credit. Such a rule can be used to reach a valid conclusion that the tax credit can or cannot be taken.

As another example of an inference in an artificial situation with limited and clearly defined options, consider a Sudoku puzzle. The rules of Sudoku are that each cell contains a single number from 1 to 9, and each row, each column and each 9-cell square contain one occurrence of each number from 1 to 9. Consider the following partially completed board:

\begin{center}
\noindent \includegraphics*[width=2.45in, height=2.45in, keepaspectratio=false]{img/sudoku}
\end{center}

The following inference shows that, in the first column, a 9 must be entered below the 7:

\begin{quotation} The 9 in the first column must go in one of the open cells in the column. It cannot go in the third cell in the column, because there is already a 9 in that 9-cell square. It cannot go in the eighth or ninth cell because each of these rows already contains a 9, and a row cannot contain two occurrences of the same number. Therefore, since there must be a 9 somewhere in this column, it must be entered in the seventh cell, below the 7.\end{quotation}

The reasoning in this inference is valid: if the premises are true, then the conclusion must be true. Logic puzzles of all sorts operate by artificially restricting the available options in various ways. This then means that each conclusion arrived at (assuming the reasoning is correct) is necessarily true. 

One can also create a context where deduction is possible by imagining a rule that holds without exception. This can be done with respect to any subject matter at all. Speakers often exaggerate the connecting premise in order to ensure that the justificatory or explanatory power of the inference is as strong as possible. Consider Smith's words in the following passage: 


\begin{adjustwidth}{2em}{0em}
\begin{longtabu}{p{.1\linewidth}p{.8\linewidth}}
\textbf{Smith:} & I'm going to have some excellent pizza this evening. \\
\textbf{Jones:} & I'm glad to hear it. How do you know?\\
\textbf{Smith:} & I'm going to Adriatico's. They always make a great pizza. \\
\end{longtabu}
\end{adjustwidth}
\vspace{-1cm}

Here, Smith justifies his belief that the pizza will be excellent --- it comes from Adriatico's, where the pizza, he claims, is \textit{always }great: in the past, present and future. 

As stated by Smith, the inference that the pizza will be great this evening is valid. However, making the inference valid in this way often means making the general premise false: it's not likely that the pizza is great \textit{every single }time; Smith is overstating the case for emphasis. Note that Smith does not need to use a universal proposition in order to convince Jones that the pizza will \textit{very likely} be good. The inference to the conclusion would be strong (though not valid) if he had said that the pizza is "almost always" great, or that the pizza has been great on all of the many occasions he has been at that restaurant in the past. The strength of the inference would fall to some extent---it would not be guaranteed to be great this evening---but a slightly weaker inference seems appropriate, given that sometimes things go contrary to expectation. 

Sometimes the laws of nature make constructing contexts for valid arguments more reasonable. Now consider the following passage, which involves a scientific law:

\begin{quotation}\noindent Jack is about to let go of Jim's leash. The operation of gravity makes all unsupported objects near the Earth's surface fall toward the center of the Earth. Nothing stands in the way. Therefore, Jim's leash will fall. \end{quotation}

(Or, as Spock said in a Star Trek episode, "If I let go of a hammer on a planet that has a positive gravity, I need not see it fall to know that it has in fact fallen.") The inference above is represented in canonical form as follows:

\begin{adjustwidth}{2em}{2em}
\begin{earg*}
\item  Jack is about to let go of Jim's leash. 
\item  The operation of gravity makes all unsupported objects near the Earth's surface fall toward the center of the Earth. 
\item  Nothing stands in the way of the leash falling. 
\itemc  Jim's leash will fall toward the center of the Earth.
\end{earg*}
\end{adjustwidth}

As stated, this argument is valid. That is, if you pretend that they are true or accept them "for the sake of argument", you would \textit{necessarily }also accept the conclusion. Or, to put it another way, there is no way in which you could hold the premises to be true and the conclusion false.

Although this argument is valid, it involves idealizing assumptions similar to the ones we saw in the pizza example. P$_2$ states a physical law which is about as well confirmed as any statement about the world around us you care to name. However, physical laws make assumptions about the situations they apply to---they typically neglect things like wind resistance. In this case, the idealizing assumption is just that nothing stands in the way of the leash falling. This can be checked just by looking, but this check can go wrong. Perhaps there is an invisible pillar underneath Jack's hand? Perhaps a huge gust of wind will come? These events are much less likely than Adriatico's making a lousy pizza, but they are still possible. 

Thus we see that using scientific laws to create a context where deductive validity is possible is a much safer bet than simply asserting whatever exceptionless rule pops into your head. However, it still involves improving the quality of the inference by introducing premises that are less likely to be true. 

So deduction is possible in artificial contexts like logical puzzles and legal codes. It is also possible in cases where we make idealizing assumptions or imagine exceptionless rules. The rest of the time we are dealing with induction. When we do induction, we try for strong inferences, where the premises, assuming they are true, would make the truth of the conclusion very likely, though not necessary. Consider the two arguments in Figure \ref{fig:strong_weak}


\begin{figure}
\begin{mdframed}[style=mytablebox]
\begin{tabu}{X[1,c]X[1,c]}
\begin{earg*}
\item  92\% of Republicans from Texas voted for Bush in 2000. 
\item  Jack is a Republican from Texas. 
\itemc  Jack voted for Bush. 
\end{earg*}
&
\begin{earg*}
\item  Just over half of drivers are female. 
\item  There's a person driving the car that just cut me off. 
\itemc  The person driving the car that just cut me off is female.
\end{earg*}
\\
A \textbf{strong} argument &
A \textbf{weak} argument
\end{tabu}
\end{mdframed}
\caption{Neither argument is valid, but one is strong and one is weak} \label{fig:strong_weak}
\end{figure}


Note that the premises in neither inference \textit{guarantee }the truth of the conclusion. For all the premises in the first one say, Jack could be one of the 8\% of Republicans from Texas who did not vote for Bush; perhaps, for example, Jack soured on Bush, but not on Republicans in general, when Bush served as governor. Likewise for the second; the driver could be one of the 49\%. 

So, neither inference is valid. But there is a big difference between how much support the premises, if true, would give to the conclusion in the first and how much they would in the second. The premises in the first, assuming they are true, would provide very strong reasons to accept the conclusion. This, however, is not the case with the second: if the premises in it were true then they would give only weak reasons for believing the conclusion. thus, the first is strong while the second is weak.

As we said earlier, there there are only two options with respect to validity---valid or not valid. On the other hand, strength comes in degrees, and sometimes arguments will have percentages that will enable you to exactly quantify their strength, as in the two examples in Figure \ref{fig:strong_weak}. 

However, even where the degree of support is made explicit by a percentage there is no firm way to say at what degree of support an inference can be classified as strong and below which it is weak. In other words, it is difficult to say whether or not a conclusion is \textit{very likely} to be true. For example, In the inference about whether Jack, a Texas Republican, voted for Bush. If 92\% of Texas Republicans voted for Bush, the conclusion, if the premises are granted, would very probably be true. But what if the number were 85\%? Or 75\%? Or 65\%? Would the conclusion very likely be true? Similarly, the second inference involves a percentage greater than 50\%, but this does not seem sufficient. At what point, however, would it be sufficient? 

In order to answer this question, go back to basics and ask yourself: "If I accept the truth of the premises, would I then have sufficient reason to believe the conclusion?". If you would not feel safe in adopting the conclusion as a belief as a result of the inference, then you think it is weak, that is, you do not think the premises give sufficient support to the conclusion. 

Note that the same inference might be weak in one context but strong in another, because the degree of support needed changes. For example, if you merely have a deposit to make, you might accept that the bank is open on Saturday based on your memory of having gone to the bank on Saturday at some time in the past. If, on the other hand, you have a vital mortgage payment to make, you might not consider your memory sufficient justification. Instead, you will want to call up the bank and increase your level of confidence in the belief that it will be open on Saturday.

Most inferences (if successful) are strong rather than valid. This is because they deal with situations which are in some way open-ended or where our knowledge is not precise. In the example of Jack voting for Bush, we know only that 92\% of Republicans voted for Bush, and so there is no definitive connection between being a Texas Republican and voting for Bush. Further, we have only statistical information to go on. This statistical information was based on polling or surveying a sample of Texas voters and so is itself subject to error (as is discussed in the chapter on induction in the complete version of this text.\label{ver_var} \nix{Chapter \ref{chap:induction} on induction).} A more precise version of the premise might be "92\% $\pm$ 3\% of Texas Republicans voted for Bush.".

At the risk of redundancy, let's consider a variety of examples of valid, strong and weak inferences, presented in standard form. 

\begin{earg*}
\item  David Duchovny weighs more than 200 pounds. 
\itemc  David Duchovny weighs more than 150 pounds.
\end{earg*}

The inference here is valid. It is valid because of the number system (here applied to weight): 200 is more than 150. It might be false, as a matter of fact, that David Duchovny weighs more than 200 pounds, and false, as a matter of fact, that David Duchovny weighs more than 150 pounds. But if you \textit{suppose }or \textit{grant }or \textit{imagine }that David Duchovny weighs more than 200 pounds, it would then \textit{have }to be true that David Duchovny weighs more than 150 pounds. Next:

\begin{earg*}
\item  Armistice Day is November 11th, each year. 
\item  Halloween is October 31st, each year.
\itemc  Armistice Day is later than Halloween, each year. 
\end{earg*}

This inference is valid. It is valid because of order of the months in the Gregorian calendar and the placement of the New Year in this system. Next:

\begin{earg*}
\item  All people are mortal. 
\item  Professor Pappas is a person. 
\itemc  Professor Pappas is mortal. 
\end{earg*}

As written, this inference is valid. If you accept for the sake of argument that all men are mortal (as the first premise says) and likewise that Professor Pappas is a man (as the second premise says), then you would have to accept also that Professor Pappas is mortal (as the conclusion says). You could not consistently both (i) affirm that all men are mortal and that Professor Pappas is a man and (ii) deny that Professor Pappas is mortal. If a person accepted these premises but denied the conclusion, that person would be making a mistake in logic.

This inference's validity is due to the fact that the first premise uses the word "all". You might, however, wonder whether or not this premise is true, given that we believe it to be true only on our experience of men \textit{in the past}. This might be a case of over-stating a premise, which we mentioned earlier. Next:

\begin{earg*}
\item  In 1933, it rained in Columbus, Ohio on 175 days.
\item  In 1934, it rained in Columbus, Ohio on 177 days.
\item  In 1935, it rained in Columbus, Ohio on 171 days.
\itemc  In 1936, it rained in Columbus, Ohio on at least 150 days.
\end{earg*}

This inference is strong. The premises establish a record of days of rainfall that is well above 150. It is possible, however, that 1936 was exceptionally dry, and this possibility means that the inference does not achieve validity. Next:

\begin{earg*}
\item  The Bible says that homosexuality is an abomination.
\itemc  Homosexuality is an abomination.
\end{earg*}

This inference is an appeal to a source. Appeals to sources are discussed in the sections on arguments from authority in the complete version of this text. \label{ver_var} \nix{Section \ref{sec:sources} of Part \ref{part:CT_and_informal_logic}} of this book. In brief, you should think about whether the source is reliable, is biased, and whether the claim is consistent with what other authorities on the subject say. You should apply all these criteria to this argument for yourself. You should ask what issues, if any, the Bible is reliable on. If you believe humans had any role in writing the Bible, you can ask about what biases and agendas they might have had. And you can think about what other sources---religious texts or moral experts---say on this issue. You can certainly find many who disagree. Given the controversial nature of this issue, we will not give our evaluation. We will only encourage you to think it through systematically.


\begin{earg*}
\item  Some professional philosophers published books in 2007.
\item  Some books published in 2007 sold more than 100,000 copies. 
\itemc  Some professional philosophers published books in 2007 that sold more than 100,000 copies. 
\end{earg*}

This reasoning is weak. Both premises use the word "some" which doesn't tell you a lot about many professional philosophers published books and how many books sold more than 100,000 copies in 2007. This means that you cannot be confident that even one professional philosopher sold more than 100,000 copies. Next:

\begin{earg*}
\item  Lots of Russians prefer vodka to bourbon. 
\itemc  George Bush was the President of the United States in 2006.
\end{earg*}

No one (in her right mind) would make an inference like this. It is presented here as an example only: it is clearly weak. It's hard to see how the premise justifies the conclusion to any extent at all.  
      
To sum up this section, we have seen that there are two styles of reasoning, deductive and inductive. The former tries to use valid arguments, while the latter contents itself to give arguments that are merely strong. The section of this book on formal logic will deal entirely with deductive reasoning. Historically, most of formal logic has been devoted to the study of deductive arguments, although many great systems have been developed for the formal treatment of inductive logic. On the other hand, the sections of this book on informal logic and critical thinking will focus mostly on inductive logic, because these arguments are more readily available in the real world. 



\practiceproblems

\noindent\problempart For each inference, (i) say whether it is valid, strong, or weak and (ii) explain your answer.

\begin{longtabu}{p{.1\linewidth}p{.8\linewidth}}
\textbf{Example}: & The patient has a red rash covering the extremities and head, but not the torso. The only cause of such a rash is a deficiency in vitamin K. So, the patient must have a vitamin K deficiency. \\
\textbf{Answer}: & \noindent (i) Valid. \newline
\noindent (ii) The word "only" means it must be vitamin K deficiency.
\\
\end{longtabu}

\begin{exercises}

\item On 2003-06-19 in Norfolk, VA, a violent storm blew through and the power went out over much of the city. So, the storm caused the power to go out.
\answer{\underline{Strong} \\Storms often cause power outages, but other things can cause them, too. In general, causal inferences are part of inductive reasoning, and are therefore at best strong. }

\item  All men are things with purple hair, and all things with purple hair have nine legs. Therefore, all men have nine legs.

\answer{\underline{Valid.}\\ You can see this by substituting in different things for ``purple hair'' and ``has nine legs.'' For instance, you could use ``mammals'' and ``has hair.'' There is no way to do this that will make the premises true and the conclusion false. So the argument is valid. }


\item  Elvis Presley was known as The King. Elvis had 18 songs reach \#1 in the Billboard charts. So, The King had 18 \#1 hits.

\answer{ \underline{Valid} \\This is an example of substituting different names for the same thing to create a valid argument. }

\item Most philosophers are right-handed. Terence Irwin is a philosopher. So, he is right-handed.

\answer{\underline{Strong} (?)\\

The conclusion isn't necessarily true, so the inference is not valid. Is it strong or weak? If you read "most" as "very many" or something like that, it would be strong; if you read "most" as "a majority" (in the sense of 'somewhere between 51\% and 99\%' ), it would probably be weak. Advertisers sometimes use the vagueness of "most" to get you to feel that lots of people are buying a certain product or service, when in fact only a small majority is. }

\item  Jack has purple hair, and purple toe nails. Hence, he has toe nails.

\answer{\underline{Valid}}

\item  The Ohio State football team beat the Miami football team on 2003-01-03 for the college national championship. So, the Ohio State football team was the best team in college football in the 2002-2003 season.

\answer{\underline{Strong} (?) \\

What are the chances that the non-best team would win the championship? If you think that a non-best time wins somewhat frequently, this inference would be weak. If, on the other hand, you think winning the championship is  a good way to judge the best team, you think the inference is strong.}

\item Willie Mosconi made almost all of the pool shots he took from 1940-1945. He took a bunch of shots in 1941. So, he made almost every shot he took in 1941.

\answer{\underline{Strong}   \\
"Almost all" is fairly vague. So we are not sure how many missed shots we are talking about in the five year period. If there were all clustered in 1941, it is possible that the success rate for 1941 would no longer qualify as "almost all," but this is unlikely. }


\item Some philosophers are people who are right-handed. Therefore, some people who are right-handed are philosophers.

\answer{\underline{Valid.} \\ The conclusion can't be false if the premise is true.}

\item U.S. President Obama firmly believed that Iran is planning a nuclear attack against Israel. We can conclude that Iran is planning a nuclear attack on Israel.

\answer{\underline{Weak}  \\

Even if you think Obama's judgment is generally reliable here, a nuclear attack on Israel is an incredibly unlikely event. After all, this land is holy to Muslims, too. Extraordinary claims require extraordinary evidence, and I don't think any one's person's judgment is enough to go on here. }

\item Since the Spanish American War occurred before the American Civil War, and since the American Civil War occurred after the Korean War, it follows that the Spanish American War occurred before the Korean War.

\answer{\underline{Weak.}\\ If A is before B and C is before B, we know nothing about the relationship between B and C. A and C could be at the same time or either one before the other.}

\item There are exactly 10 humans in Carnegie Hall right now. Every human in Carnegie Hall right now has exactly ten legs. And, of course, no human in Carnegie Hall shares any legs with another human. Thus, there are at least 100 legs in Carnegie Hall right now.

\answer{\underline{Valid}}

\item Amy Bishop is an evolutionary biologist (who shot a number of her colleagues to death in 2010). Evolutionary biology is incompatible with [Christian] scriptural teaching. Scriptural teaching is the only grounding for morality. Thus, evolutionary biologists are immoral.

\answer{\underline{Weak}\\

Many beliefs are incompatible with scriptural teaching on some point or other, but it's not clear that the people who hold those beliefs are immoral, unless morality is defined as following every single scriptural edict.}

\item Corrupt people do harm to those around them, and no one intentionally wants to be done harm. Therefore, I [Socrates] did not corrupt my associates intentionally.

\answer{\underline{Valid} \\ The argument works if you think the premises are true always and everywhere. They seem like natural enough statements to make, but are they really perfectly and unequivocally true?}

\item Taxation means paying some of your earned income to the government. Some of this income is distributed to others. Paying so that someone else can benefit is slavery. Therefore, taxation is slavery.

\answer{\underline{Valid.}\\ Chain argument. However, the definition of slavery here is contentious, to say the least.}

%\item Attempts have been made recently to carry bombs or bomb-making materials onto planes in the underwear and in other personal areas. These types of procedure provide a large measure of security against such attempts. Thus, flyers are required to submit to either a full-body scan or a thorough pat-down.
%
%%{\color{red}This problem shouldn't have been here, because it is really an explanation and not an argument.}
\end{exercises}


\noindent\problempart For each inference, (i) say whether it is valid, strong, or weak and (ii) explain your answer.
\answer{Answers by Ben Sheredos}
\begin{exercises} 
\item The sun has come up in the east every day in the past. So, the sun will come up in the east tomorrow.

\answer{Invalid, but strong. That's a huge number of cases to generalize from, so the conclusion is very likely to be true, even if it is not \textit{certain}}

\item Jack's dog Jim will die before the age of 73 (in human years). After all, you are familiar with lots of dogs, and lots of different kinds of dogs, and any dog that is now dead died before the age of 73 (in human years).

\answer{Invalid, but strong. That's a huge number of cases to generalize from, so the conclusion is very likely to be true. Don't get confused because you yourself are \textit{absolutely certain} that no dog will live to 73 in human years. The question is how well \textit{this argument} supports that claim.}

\item Any time the public receives a tax rebate, consumer spending increases, and the economy is stimulated. Since the public just received a tax rebate, consumer spending will increase.

\answer{Valid. One could continue on to infer that the economy will be stimulated. The key is that the premise is that \textit{every time} there is a tax rebate, spending increases. This might be false, but \textit{if} it is true, the conclusion follows.}

\item  90\% of the marbles in the box are blue. So, about 90\% of the 20 I pick at random will be blue.

\answer{Invalid, and pretty weak. This is a common error in statistical reasoning. The 20 marbles you pick out are not connected in any way, so if you pick out a non-blue marble, that doesn't increase the odds of you picking out a blue one next time. You might happen to pick nothing but marbles from the 10\% of marbles that are not blue.}

\item  According to the world-renowned physicist Stephen Hawking, quarks are one of the fundamental particles of matter. So, quarks are one of the fundamental particles of matter.

\answer{This is a simple appeal to authority, which is a fallacy. The argument is weak. It might be supplemented to be made stronger (''Hawking is \textit{the world's foremost authority} on this topic, and he has put forth a convincing argument that quarks are a fundamental particle''). But as it is stated here, it's garbage. }

\item Sean Penn, Susan Sarandon and Tim Robbins are actors, and Democrats. So, most actors are Democrats.

\answer{Invalid and weak. This is a very hasty generalization.}


\item  The President's approval rating has now fallen to 53\%, employment is at a 10 year high, and he is in charge of two foreign wars. He would not win another term in two years' time, if he were to run.

\answer{Weak. There are some suppressed premises here, concerning how voters are likely to respond to the claims presumed in the premises. Only by filling them in could the argument be made strong.}


\item If Bill Gates owns a lot of gold then Bill Gates is rich, and Bill Gates doesn't own a lot of gold. So, Bill Gates isn't rich.

\answer{Weak, since (\textit{a}) owning gold is not necessary for being rich, and (\textit{b}) Bill Gates is demonstrably rich even though (suppose) he owns little gold.}

\item All birds have wings, and all vertebrates have wings. So, all birds are vertebrates.

\answer{ Weaksauce, even if the premises are true. Compare: ''All students in PHIL 10 are enrolled at UCSD, and all students in PHIL 163 are enrolled at UCSD. So all students in Phil 10 are in PHIL 163.'' Clearly wrongheaded.}

\item U.S. President Obama gave a speech in Berlin shortly after his inauguration. Berlin, of course, is where Hitler gave many speeches. Thus, Obama intends to establish a socialist system in the U.S.

\answer{ Obviously Weak. Doing something as general as ``giving a speech'' in a place where Hitler gave a speech does not make one relevantly like Hitler to draw this conclusion.}

\item Einstein said that he believed in a god only in the sense of a pantheistic god equivalent with nature. Thus, there is no god in the Judeo-Christian sense.

\answer{Weak. Appeal to authority. The argument concludes that something is true just because one person believed it; why trust Einstein on this? No support is provided. What, are we just supposed to be impressed because it was Einstein?}

\item The United States Congress has more members than there are days in the year. Thus, at least two members of the United States Congress celebrate their birthdays on the same day of the year.

\answer{Valid. There are 365 days in a year. In any group of 366 people, at least 2 people have to share birthdays. (And don't try weaseling in that leap-year nonsense.)}

\item The base at Guantanamo ought to be closed. The continued incarceration of prisoners without any move to try or release them provides terrorist organizations with an effective recruiting tool, perhaps leading to attacks against Americans overseas.

\answer{Pretty strong? The first sentence is the conclusion, and reasons are provided for thinking it is true. Maybe there are countervailing reasons that tell against...? Informal reasoning is tricky.}

\item Smith and Jones surveyed teenagers (13-19 years old) at a local mall and found that 94\% of this group owned a mobile phone. Therefore, they concluded, about 94\% of all teenagers own mobile phone.

\answer{Weak. Why suppose an un-specified number of teenagers in one place are representative of that entire group of people, worldwide?}

\item Janice Brooks is an unfit mother. Her Facebook and Twitter records show that in the hour prior to the youngest son's accident she had sent 50 messages --- any parent who spends this much time on social media when they have kids is not giving them proper attention. 

\answer{Weak. Does Janice Brooks even live with her youngest son? Was there any reason she should've known his accident was impending? What, are people with children never allowed to have a day chatting with friends?}

\end{exercises}

\section*{Key Terms}
\begin{sortedlist}
\sortitem{Valid}{} 	
\sortitem{Invalid}{} 	
\sortitem{Sound}{} 
\sortitem{Strong}{} 
\sortitem{Cogent}{} 
\sortitem{Deductive}{} 
\sortitem{Inductive}{} 
\sortitem{Fallacy of mistaking the conclusion for the argument}{}
\sortitem{Fallacy}{}
\sortitem{Weak}{}
\end{sortedlist}




%\part{Formal Logic} 
%\chapter{What is Formal Logic?}
\label{chap:whatisformallogic}
\markright{Ch. \ref{chap:whatisformallogic}: What is Formal Logic?}
\setlength{\parindent}{1em}

% **************************************************
% *	3.1 Formal as in  Concerned with the Form of Things             *
% **************************************************

\section{Formal as in Concerned with the Form of Things}


The chapters in 
\iflabelexists{part:formal_logic}{Part \ref{part:formal_logic}} %This prints ``Part $N$ if there is a single section for all of formal logic
{Parts \iflabelexists{part:cat_logic}{\ref{part:cat_logic} and \ref{part:sent_logic}} %this prints ``Parts $N$ and $M'$' if there is a section for cat logic, where $N$ and $M$ are the part numbers for cat and sent logic.
{\ref{part:sent_logic} and \ref{part:quant_logic}}} %this prints ``Parts $N$ and $M'$' if there is not a section for cat logic, where $N$ and $M$ are the part numbers for sent and quant logic.
deal with formal logic. Formal logic is distinguished from other branches of logic by the way it achieves content neutrality. Back on page \pageref{def:content_neutrality}, we said that a distinctive feature of logic is that it is neutral about the content of the argument it evaluates. If a kind of argument is strong---say, a kind of statistical argument---it will be strong whether it is applied to sports, politics, science or whatever. Formal logic takes radical measures to ensure content neutrality: it removes the parts of a statement that tie it to particular objects in the world and replaces them with abstract symbols. (See page \pageref{def:Formal_logic})

Consider the two arguments from Figure \ref{fig:valid_sound} again:
\begin{multicols}{2}
\begin{earg*}
\item Socrates is a person.
\item All persons are mortal.
\itemc Socrates is mortal.
\end{earg*}

\begin{earg*}
\item Socrates is a person.
\item All people are carrots.
\itemc Socrates is a carrot.
\end{earg*}

\end{multicols}

These arguments are both valid. In each case, if the premises were true, the conclusion would have to be true. (In the case of the first argument, the premises are actually true, so the argument is sound, but that is not what we are concerned with right now.) What makes these arguments valid is that they are put together the right way. Another way of thinking about this is to say that they have the same logical form. Both arguments can be written like this:

\begin{earg*}
\item $S$ is $M$.
\item All $M$ are $P$.
\itemc[.2] $S$ is $P$.
\end{earg*}

In both arguments $S$ stands for Socrates and $M$ stands for person. In the first argument, $P$ stands for mortal; in the second, $P$ stands for carrot. \iflabelexists{chap:catstatements}{(The reason we chose these letters will become clear in Chapters \ref{chap:catstatements} and \ref{chap:cat_syllogisms}.)}{} The letters `S', `M', and `P' are variables. They are just like the variables you may have learned about in algebra class. In algebra, you had equations like $y = 2x + 3$, where $x$ and $y$ were variables that could stand for any number. Just as $x$ could stand for any number in algebra, `S' can stand for any name in logic. In fact, this is one of the original uses of variables. Long before variables were used to stand for numbers in algebra, they were used to stand for classes of things, like people or carrots, by Aristotle in his book the \cite*{Aristotle:prior}. At about the same time, over in India, the ancient grammarian and linguist P\={a}\d{n}ini was also using variables to represent possible sounds that could be used in different forms of a word \citep{Panini2015}. Both thinkers introduce their variables fairly causally, as if their readers would be familiar with the idea, so it may be that people prior to them actually invented the variable.

Whoever invented it, the variable was one of the most important conceptual innovations in human history, right up there with the invention of the zero, or alphabetic writing. The importance of the variable for the history of mathematics is obvious. But it was also incredibly important in one of its original fields of application, logic. For one thing, it allows logicians to be more content neutral. We can set aside any associations we have with people, or carrots, or whatever, when we are analyzing an argument. More importantly, once we set aside content in this way, we discover that something incredibly powerful is left over, the logical structure of the sentence itself. This is what we investigate when we study formal logic. In the case of the two arguments above, identifying the logical structure of statements reveals not only that the two arguments have the same logical form, but they have an impeccable logical form. Both arguments are valid, and any other arguments that have this form will be valid. 

When Aristotle introduced the variable to the study of logic he used it the way we did in the argument above. His variables stood for names and categories in simple two-premise arguments called syllogisms. The system of logic Aristotle outlined became the dominant logic in the Western world for more than two millennia. It was studied and elaborated on by philosophers and logicians from Baghdad to Paris. The thinkers that carried on Aristotelian tradition were divided by language and religion. They were pagans, Muslims, Jews, and Christians writing typically in Greek, Latin or Arabic. But they were all united by the sense that the tools Aristotle had given them allowed them to see something profound about the nature of reality. They were looking at abstract structures which somehow seemed to be at the foundation of things. As the philosopher and historian of logic Catarina Dutilh Novaes points out, the logic that the thinkers of all these religious traditions were pursuing was formal in that it concerned the \textit{forms} of things \citep{DutilhNovaes2011}. As formal logic evolved, however, the idea of being ``formal'' would take on an additional meaning. 


% *********************************************************
% *				3.2 Formal meaning strictly following rules              *
% ********************************************************


\section{Formal as in Strictly Following Rules}


\newglossaryentry{artificial language}
{
name=artificial language,
description={A language that was consciously developed by identifiable individuals for some purpose.}
}

\newglossaryentry{natural language}
{
name=natural language,
description={A language that develops spontaneously and learned by infants as their first language.}
}

\newglossaryentry{formal language}
{
name=formal language,
description={An artificial language designed to bring out the logical structure of  ideas and remove all the ambiguity and vagueness that plague natural languages like English. Sometimes, formal languages are also said to be languages that can be implemented by a machine.}
}

Despite its historical importance, Aristotelean logic has largely been superseded. Starting in the 19th century people learned to do more than simply replace categories with variables. They learned to replicate the whole structure of sentences with a formal system that brought out all sorts of features of the logical form of arguments. The result was the creation of entire artificial languages. An \textsc{\gls{artificial language}} \label{def:artificial_language} is a language that was consciously developed by identifiable individuals for some purpose. Esperanto, for instance, is an artificial language developed by Ludwig Lazarus Zamenhof in the 19th century with the hope of promoting world peace by creating a common language for all. J.R.R. Tolkien invented several languages to flesh out the fictional world of his fantasy novels, and even created timelines for their evolution. For Tolkien, the creation of languages was an art form in itself, ``An art for which life is not long enough, indeed: the construction of imaginary languages in full or outline for amusement, for the pleasure of the constructor or even conceivably of any critic that might occur'' (\cite*{Tolkien1931}). And it is an art that is really beginning to catch on, especially with Hollywood commissioning languages to be constructed for blockbuster films. 

Artificial languages contrast with \textsc{\glspl{natural language}}, \label{def:natural_language} which develop spontaneously and are learned by infants as their first language. Natural languages include all the well-known languages spoken around the world, like English or Japanese or Arabic. It also includes more recently developed languages and evolved spontaneously amongst groups of people. For instance, whenever you put deaf children together, for instance in a boarding school, they will spontaneously develop their own sign language. This phenomenon was important for the development of American Sign Language (ASL) and is part of why ASL counts as a \textit{natural} language. For similar reasons Nicaraguan Sign Language counts as a natural language, even though it emerged very recently---in the late 1970s and 80s, when the new Sandinista government set up schools for the deaf for the first time. Natural languages can also develop by creolization, when languages merge and children grow up speaking the merged language as their first language. Haitian Creole is the most famous example of this.   

The languages developed by logicians are artificial, not natural. Their goal is not to promote global harmony, like Zamenhof's Esperanto. Nor are they creating art for art's sake, as Tolkein was, although logical languages can have a great deal of beauty. When the languages first started being developed in the late 19th and early 20th centuries, the goal was, in fact, to have a logically pure language, free of the irrationalities the plague natural languages. More specifically, they had two distinct goals: first, remove all ambiguity and vagueness, and second, to make the logical structure of the language immediately apparent, so that the language wore its logical structure on its face, as it were. If such a language could be developed, it would help us solve all kinds of problems. The logician and philosopher Rudolf Carnap, for instance, felt that the right artificial language could simply make philosophical problems disappear \citep{Carnap1928}.

The languages developed by logicians in the late 19th and early 20th centuries got labeled formal languages, in part because the logicians in question were working in the tradition of formal logic that was already established. A shift began to happen here with the meaning of formal, however, a change which is well documented by Dutilh Novaes  \citeyear{DutilhNovaes2011}. Logicians began to hope that the languages that were being developed were so logical that everything about them could be characterized by a machine. A machine could be used to create sentences in this language, and then again to identify all the valid arguments in this language. This brings out another sense of the word ``formal.'' As Dutilh Novaes puts it (\citeyear{DutilhNovaes2011}) instead of being ``formal'' in the sense of concerning the forms of things, logic was formal in the sense that it followed rules perfectly precisely. You might compare this to the way a ``formal hearing'' in a court of law follows the rule of law to the letter. 

For the purposes of this textbook, we will say that the core idea of a  \textsc{\gls{formal language}} \label{def:formal_language} is that it is an artificial language designed to bring out the logical structure of ideas and remove all the ambiguity and vagueness that plague natural languages like English. We will further add that sometimes, formal languages are languages that can be implemented by a machine. Creating formal languages always involves all kinds of trade offs. On the one hand, we are trying to create a language that makes a logical structure clear and obvious. This will require simplifying things, removing excess baggage from the language. On the other hand, we want to make the language perfectly precise, free of vagueness and ambiguity. This will mean adding complexity to the language. The other thing was that it was very important for the people developing these languages that you be able to prove the all the truths of mathematics in them. This meant that the languages had to have a certain scope.

This was a trade off no logician was ever able to get perfectly correct, because, as it turns out, a logically pure language is impossible. No formal language can do everything that a natural language can do. Logicians became convinced of this, naturally enough, because of a pair of logical proofs. In 1931, the logician Kurt G\"{o}del showed that you couldn't do all of mathematics in a consistent logical system, which was enough to persuade most of the logicians engaged in the project to drop it. There is a more general problem with the idea of a purely logical language, though, which is that that many of the features logicians were trying to remove from language were actually necessary to make it function. Arika Okrent puts the point quite well. For Okrent, the failure of artificial languages is precisely what illuminates the virtues of natural language. 

\begin{quotation}\noindent [By studying artificial languages we] gain a deeper appreciation of natural language and the messy qualities that give it so much flexibility and power and that a simple communication device. The ambiguity and lack of precision allow to serve as a instrument of thought \textit{formation}, of experimentation and discovery. We don't know exactly what we mean before we speak; we can figure it out as we go along,. We can talk just to talk, to be social, to feel connected, to participate. At the same time natural language still works as an instrument of thought transmission, one that can be \textit{made} extremely precise and reliable when we need it to be, or left loose and sloppy when we can't spare the time or effort \citep{Okrent2009} \end{quotation} 

The languages developed in the late 19th and early 20th centuries had goals that were theoretical, rather than practical. They languages were meant to improve our understanding of the world for the sake of improving our understanding of the world. They failed at this theoretical goal, but they wound up having a practical spin-off of world-historical proportions, which is why formal logic is a thriving discipline to this day. Remember that in this period people started thinking of formal languages as languages that could be implemented mechanically. At first, the idea of a a mechanistic language was a metaphor. The rules that were being followed to the letter were to be followed by a human being actually writing down symbols. This human being was generally referred to as a ``computer,'' because they were computing things. The world changed when a logician named Alan Turing started using literal machines to be computers.

In the 1930s, Turing developed the idea of a reasoning machine that could compute any function. At first, this was just an abstract idea: it involved an infinite stretch of tape. But during World War II, Turing went to work the British code breaking effort at Bletchley Park. The Nazis encoded messages using a device called the Enigma Machine. The Allies had captured one, but since they settings on the machine were reshuffled for each message, it didn't do them much good. Turing, together with people like the mathematicians Gordon Welchman and Joan Clarke, managed to build another machine that could test Enigma settings rapidly to identify the configuration being used. People had made computing machines before, but now the science of logic was so much more advanced that they real power of mechanical computing could be exploited. The human computers became the fully programmable machines we know today, and the formal languages logicians created for theoretical reasons came the computer languages the world of the 21st century depends on. (All of this information, plus lots of fascinating pictures and diagrams, is available at www.turing.org.uk.)



%This version is for the complete text, where all formal sections are covered in a unified Part II.
\iflabelexists{part:formal_logic}{Part \ref{part:formal_logic} of this book begins by exploring the world of Aristotelian logic, where logic is ``formal'' in the sense of being about the forms of things. Chapter \ref{chap:catstatements} looks at the logical structure of the individual statements studied by the Aristotelian tradition. Chapter \ref{chap:cat_syllogisms} then builds these into valid arguments. After we study Aristotelian logic, we will develop two formal languages, called SL and QL.  Chapters \ref{chap:SL} through \ref{chap:proofsinSL} develop SL. In SL, the smallest units are individual statements. Simple statements are represented as letters and connected with {logical connectives} like \emph{and} and \emph{not} to make more complex statements. Chapters \ref{chap:QL} through \ref{chap:proofsinQL} develop QL. In QL, the basic units are objects, properties of objects, and relations between objects.}{
%% This version of the paragraph is for texts that just do cat and sent.
\iflabelexists{part:cat_logic}{Part \ref{part:cat_logic} of this book explores the world of Aristotelian logic. Chapter \ref{chap:catstatements} looks at the logical structure of the individual statements studied by the Aristotelian tradition. Chapter \ref{chap:cat_syllogisms} then builds these into valid arguments. Part \ref{part:sent_logic} develops a full-blown formal language, called Sentential logic, or SL. In SL Simple statements are represented as letters and connected with logical connectives like \emph{and} and \emph{not} to make more complex statements.}{
%This version of the paragraph is for texts that just do sent and quant
In this book we will be developing two formal languages, called SL and QL. Part \ref{part:sent_logic} develops SL.In SL, the smallest units are individual statements. Simple statements are represented as letters and connected with logical connectives like \emph{and} and \emph{not} to make more complex statements. Part \ref{part:quant_logic} develops QL. In QL, the basic units are objects, properties of objects, and relations between objects.} }




%\iflabelexists{part:formal_logic}{ and QL.  Chapters \ref{chap:SL} through \ref{chap:proofsinSL} develop SL. In SL, the smallest units are individual statements. Several chapters in the complete version of this text \label{ver_var}\nix{Chapters \ref{chap:QL} through \ref{chap:proofsinQL}} develop QL. In QL, the basic units are objects, properties of objects, and relations between objects.


%% **********************************************
%% *			On Learning a Formal Logical System        *
%% **********************************************
%\section{On Learning a Formal Logical System}
%\label{sec:On_learning_a_formal_logical_system}
%
%You may be reading this book because you have a keen interest in logic and are excited to learn more about it. You may also be reading this book because it was assigned in a class that you need to fulfill a distribution requirement. As the chapters on formal logic roll on, and the pages begin to fill up with unfamiliar squiggles, you may even begin to question whether the study of logic is for you. Rest assured, if you have a human brain capable of reading this sentence, you are also capable of doing formal logic---and you can benefit from doing so, too. In this section, we are going to talk about why you can be confident in your ability to do logic, even if you are new to it. We are also going to offer some strategies for studying formal logic, so even if you are already quite confident in your abilities, it will be worth reading the rest of this section. 
%
%All of the basic mental skills used in a formal logical system are just that: basic mental skills. They are things you do whenever you use language. A basic part of formal logic is using abstract symbols to refer to a group of things that aren't specified. So earlier we used ``$P$'' in place of the words ``mortal'' and ``carrot'' and a whole bunch of other words that might occupy that spot in an argument. This is the same thing you do when you use a word like ``dog'' to refer to Spot and Fido and a whole bunch of other dogs that you don't know about. We are also going to spend time transforming things in one logical form into another. Again, this is something you already do when you speak. You know that ``Jane gave the ball to Sally'' can be changes to ``Sally was given the ball by Jane'' without changing its meaning. The kinds of things we are doing in this text are no different. 
%


% **********************************************
% *			More Logical Notions for Formal Logic      *
% **********************************************
\section{More Logical Notions for Formal Logic}
\label{sec:other_logical_notions}
\setlength{\parindent}{1em}

Part \ref{part:basic_concepts} covered the basic concepts you need to study any kind of logic. When we study formal logic, we will be interested in some additional logical concepts, which we will explain here. 

%1.5.1 Truth values

\subsection{Truth values}

\newglossaryentry{truth value}
{
  name=truth value,
  description={The status of a statement with relationship to truth. For  this textbook, this means the status of a statement as true or false}
}

\newglossaryentry{bivalent}
{
  name=bivalent,
  description={A property of logical systems which is present when the system only has two truth values, generally ``true'' and ``false.''}
}



A truth value is the status of a statement as true or false. Thus the truth value of the sentence ``All dogs are mammals'' is ``True,'' while the truth value of ``All dogs are reptiles'' is false. More precisely, a \textsc{\gls{truth value}} \label{def:Truth_value} is the status of a statement with relationship to truth. We have to say this, because there are systems of logic that allow for truth values besides ``true'' and ``false,'' like ``maybe true,'' or ``approximately true,'' or ``kinda sorta true.'' For instance, some philosophers have claimed that the future is not yet determined. If they are right, then statements about \emph{what will be the case} are not yet true or false. Some systems of logic accommodate this by having an additional truth value. Other formal languages, so-called paraconsistent logics, allow for statements that are both true \emph{and} false. We won't be dealing with those in this textbook, however. For our purposes, there are two truth values, ``true'' and ``false,'' and every statement has exactly one of these. Logical systems like ours are called \textsc{\gls{bivalent}}. \label{def:Bivalent}







%1.5.2 Tautology, Contingent Statement, Contradiction

\subsection{Tautology, contingent statement, contradiction}

In considering arguments formally, we care about what would be true \emph{if} the premises were true. Generally, we are not concerned with the actual truth value of any particular statements--- whether they are \emph{actually} true or false. Yet there are some statements that must be true, just as a matter of logic.

Consider these statements:
\begin{enumerate}[label=(\alph*)]
\item \label{itm:ex_contingent} It is raining.
\item \label{itm:ex_tautology} Either it is raining, or it is not.
\item \label{itm:ex_contradiction} It is both raining and not raining.
\end{enumerate}
In order to know if statement \ref{itm:ex_contingent} is true, you would need to look outside or check the weather channel. Logically speaking, it might be either true or false. Statements like this are called \emph{contingent} statements.


\newglossaryentry{tautology}
{
name=tautology,
description={A statement that must be true, as a matter of logic.}
}

Statement \ref{itm:ex_tautology} is different. You do not need to look outside to know that it is true. Regardless of what the weather is like, it is either raining or not. If it is drizzling, you might describe it as partly raining or in a way raining and a way not raining. However, our assumption of bivalence means that we have to draw a line, and say at some point that it is raining. And if we have not crossed this line, it is not raining. Thus the statement ``either it is raining or it is not'' is always going to be true, no matter what is going on outside. A statement that has to be true, as a matter of logic is called a \textsc{\gls{tautology}} \label{def:tautology} or logical truth. 

\newglossaryentry{contradiction}
{
name=contradiction,
description={A statement that must be false, as a matter of logic.}
}

You do not need to check the weather to know about statement \ref{itm:ex_contradiction}, either. It must be false, simply as a matter of logic. It might be raining here and not raining across town, it might be raining now but stop raining even as you read this, but it is impossible for it to be both raining and not raining here at this moment. The third statement is \emph{logically false}; it is false regardless of what the world is like. A logically false statement is called a \textsc{\gls{contradiction}}. \label{def:contradiction}

\newglossaryentry{contingent statement}
{
name=contingent statement,
description={A statement that is neither a tautology nor a contradiction.}
}

We have already said that a contingent statement is one that could be true, or could be false, as far as logic is concerned. To be more precise, we should define a \textsc{\gls{contingent statement}}  \label{def:contingent_statement} as a statement that is neither a tautology nor a contradiction. This allows us to avoid worrying about what it means for something to be logically possible. We can just piggyback on the idea of being logically necessary or logically impossible. 

A statement might \emph{always} be true and still be contingent. For instance, it may be the case that in no time in the history of the universe was there ever an elephant with tiger stripes. Elephants only ever evolved on Earth, and there was never any reason for them to evolve tiger stripes. The statement ``Some elephants have tiger stripes,'' is therefore always false. It is, however, still a contingent statement. The fact that it is always false is not a matter of logic. There is no contradiction in considering a possible world in which elephants evolved tiger stripes, perhaps to hide in really tall grass. The important question is whether the statement \emph{must} be true, just on account of logic.

When you combine the idea of tautologies and contradictions with the notion of deductive validity, as we have defined it, you get some curious results. For one thing, any argument with a tautology in the conclusion will be valid, even if the premises are not relevant to the conclusion. This argument, for instance, is valid.

\begin{earg*}
\item There is coffee in the coffee pot.
\item There is a dragon playing bassoon on the armoire.
\itemc All bachelors are unmarried men.
\end{earg*}

The statement ``All bachelors are unmarried men'' is a tautology. No matter what happens in the world, all bachelors have to be unmarried men, because that is how the word ``bachelor'' is defined. But if the conclusion of the argument is a tautology, then there is no way that the premises could be true and the conclusion false. So the argument must be valid.

Even though it is valid, something seems really wrong with the argument above. The premises are not relevant to the conclusion. Each sentence is about something completely different. This notion of relevance, however, is something that we don't have the ability to capture in the kind of simple logical systems we will be studying. The logical notion of validity we are using here will not capture everything we like about arguments.

Another curious result of our definition of validity is that any argument with a contradiction in the premises will also be valid. In our kind of logic, once you assert a contradiction, you can say anything you want. This is weird, because you wouldn't ordinarily say someone who starts out with contradictory premises is arguing well. Nevertheless, an argument with contradictory premises is valid.

%1.5.3 Logical equivalence. 

\subsection{Logically Equivalent and Contradictory Pairs of Sentences}

We can also ask about the logical relations \emph{between} two statements. For example:

\begin{enumerate}[label=(\alph*)]
\item John went to the store after he washed the dishes.
\item John washed the dishes before he went to the store.
\end{enumerate}

\newglossaryentry{logical equivalence}
{
name={logical equivalence},
text={logically equivalent},
description={A property held by a pair of sentences that must always have the same truth value.}
}

These two statements are both contingent, since John might not have gone to the store or washed dishes at all. Yet they must have the same truth value. If either of the statements is true, then they both are; if either of the statements is false, then they both are. When two statements necessarily have the same truth value, we say that they are \textsc{\gls{logical equivalence}}. \label{def:logical_equivalence}

\newglossaryentry{contradictories}
{
name=contradictories,
description={Two statements that must have opposite truth values, so that one must true and the other false.}
}

On the other hand, if two sentences must have opposite truth values, we say that they are \textsc{\gls{contradictories}}. \label{def:contradictory}Consider these two sentences 

\begin{enumerate}[label=(\alph*)]
\item Susan is taller than Monica.
\item Susan is shorter or the same height as Monica.
\end{enumerate}

One of these sentences must be true, and if one of the sentences is true, the other one is false. It is important to remember the difference between a single sentence that is a \emph{contradiction} and a pair of sentences that are \emph{contradictory}. A single sentence that is a contradiction is in conflict with itself, so it is never true. When a pair of sentences is contradictory, one must always be true and the other false.

%%%%%%%%%%%%%%  consistency

\subsection{Consistency}
Consider these two statements:

\begin{enumerate}[label=(\alph*)]
\item \label{itm:taller} My only brother is taller than I am.
\item \label{itm:shorter} My only brother is shorter than I am.
\end{enumerate}

Logic alone cannot tell us which, if either, of these statements is true. Yet we can say that \emph{if} the first statement \ref{itm:taller} is true, \emph{then} the second statement \ref{itm:shorter} must be false. And if \ref{itm:shorter}  is true, then \ref{itm:taller} must be false. It cannot be the case that both of these statements are true. It is possible, however that both statements can be false. My only brother could be the same height as I am. 

\newglossaryentry{inconsistency}
{
name=inconsistency,
text={inconsistent},
description={A property possessed by a set of sentences when they cannot all be true at the same time, but they may all be false at the same time.}
}

\newglossaryentry{consistency}
{
name=consistency,
text={consistent},
description={A property possessed by a set of sentences when they can all be true at the same time, but are not necessarily so.}
}

If a set of statements could not all be true at the same time, they are said to be \textsc{\gls{inconsistency}}. \label{def:inconsistency} Otherwise, they are \textsc{\gls{consistency}}. \label{def:consistency} 

We can ask about the consistency of any number of statements. For example, consider the following list of statements:

\label{MartianGiraffes}
\begin{enumerate}[label=(\alph*)]
\item \label{itm:at_least_four}There are at least four giraffes at the wild animal park.
\item \label{itm:exactly_seven} There are exactly seven gorillas at the wild animal park.
\item \label{itm:not_more_than_two} There are not more than two Martians at the wild animal park.
\item \label{itm:martians} Every giraffe at the wild animal park is a Martian.
\end{enumerate}

Statements \ref{itm:at_least_four} and \ref{itm:martians} together imply that there are at least four Martian giraffes at the park. This conflicts with \ref{itm:not_more_than_two}, which implies that there are no more than two Martian giraffes there. So the set of statements \ref{itm:at_least_four}--\ref{itm:martians} is inconsistent. Notice that the inconsistency has nothing at all to do with \ref{itm:exactly_seven}. Statement \ref{itm:exactly_seven} just happens to be part of an inconsistent set.

Sometimes, people will say that an inconsistent set of statements ``contains a contradiction.'' By this, they mean that it would be logically impossible for all of the statements to be true at once. A set can be inconsistent even when all of the statements in it are either contingent or tautologous. When a single statement is a contradiction, then that statement alone cannot be true.

%%%%%%%%%%%  Practice Problems %%%%%%%%%%%


\practiceproblems
\noindent \problempart \label{pr.EnglishTautology} Label the following tautology, contradiction, or contingent statement.

\begin{longtabu}{p{.1\linewidth}p{.9\linewidth}}
\textbf{Example}: & Caesar crossed the Rubicon. \\
\textbf{Answer}: & Contingent statement. \\
&The Rubicon is a river in Italy. When General Julius Caesar took his army across it, he was committing to a revolution against the Roman Republic. Since that time, ``crossing the Rubicon'' has been a expression referring to making an irreversible decision. This kind of decision certainly seems to be contingent. Caesar could have decided otherwise.\\
\end{longtabu}

\begin{exercises}
\item Someone once crossed the Rubicon. \answer{\underline{Contingent statement}}
\item No one has ever crossed the Rubicon. \answer{\underline{Contingent  statement}}
\item If Caesar crossed the Rubicon, then someone has. \answer{\underline{Tautology}}
\item Even though Caesar crossed the Rubicon, no one has ever crossed the Rubicon. \answer{\underline{Contradiction}}
\item If anyone has ever crossed the Rubicon, it was Caesar. \answer{\underline{Contingent statement}}
\end{exercises}

\noindent \problempart Label the following tautology, contradiction, or contingent statement.
\begin{exercises}
\item Elephants dissolve in water. \answer{\underline{Contingent}}
\item Wood is a light, durable substance useful for building things. \answer{\underline{Contingent}}
\item If wood were a good building material, it would be useful for building things. \answer{\underline{Tautology}}
\item I live in a three story building that is two stories tall. \answer{\underline{Contradiction}}
\item If gerbils were mammals they would nurse their young. \answer{\underline{Tautology}}
\end{exercises}

\noindent \problempart Label the following logically equivalent, contradictory, or neither. 

\begin{longtabu}{p{.1\linewidth}p{.9\linewidth}}
\textbf{Example}: &  All students who study will pass the test. \\
& If Jeremy studies, he will pass the test. \\
\textbf{Answer}: & Neither. \\
&If the first statement is true, then the second statement has to be true, but the reverse is not the case. It might be that Jeremy will pass the test if he studies, but some other students are going to fail no matter what.\\
\end{longtabu}

 
\begin{exercises}
\item Elephants dissolve in water.	\\
	If you put an elephant in water, it will dissolve.
\answer{\\\underline{Logically equivalent}}	

\item All mammals dissolve in water.\\		
	If you put an elephant in water, it will dissolve. 
\answer{\\ \underline{Neither}}

\item Elephants are bigger than lions. \\                                                                                        
Elephants are smaller or the same size as lions.
\answer{\\ \underline{Contradictory}}

\item The Eurasian elephant is an herbivore \\
All the Eurasian elephant sometimes eats meat
\answer{\\ \underline{Contradictory}}

\item Elephants dissolve in water. 	\\	
	All mammals dissolve in water. 
\answer{\\ \underline{Neither}}

\end{exercises}


\noindent \problempart Label the following logically equivalent, contradictory, or neither. 

\begin{exercises}
\item  Thelonious Monk played piano.	\\
John Coltrane played tenor sax. 
\answer{\\ \underline{Neither}}

\item  Thelonious Monk played gigs with John Coltrane.	\\
	John Coltrane played gigs with Thelonious Monk.
\answer{\\ \underline{Logically equivalent}}

\item  All professional piano players have big hands.	\\
	Piano player Bud Powell had big hands.
	\answer{\\ \underline{Neither}}

\item  Bud Powell suffered from severe mental illness.	 \\
	All piano players suffer from severe mental illness.
	\answer{\\ \underline{Neither}}

\item John Coltrane was deeply religious.	 \\
John Coltrane was moderately or not at all religious 
\answer{\\ \underline{Contradictory}}
\end{exercises}


\noindent \problempart Consider again the statements on p.\pageref{MartianGiraffes}: 
\begin{enumerate}[label=(\alph*)]
\item \label{itm:at_least_four}There are at least four giraffes at the wild animal park.
\item \label{itm:exactly_seven} There are exactly seven gorillas at the wild animal park.
\item \label{itm:not_more_than_two} There are not more than two Martians at the wild animal park.
\item \label{itm:martians} Every giraffe at the wild animal park is a Martian.
\end{enumerate}
Now mark each of the following sets of statements consistent or inconsistent.
\begin{longtabu}{p{.1\linewidth}p{.9\linewidth}}
\textbf{Example}: & Statements \ref{itm:at_least_four}, \ref{itm:not_more_than_two}, and \ref{itm:martians}\\
\textbf{Answer}: & Inconsistent. If there are at least four giraffes, and every one of them is Martian, there can't be no more than two Martians in the park.\\
\end{longtabu}



\begin{exercises}
\item Statements \ref{itm:exactly_seven}, \ref{itm:not_more_than_two}, and \ref{itm:martians} \answer{\underline{consistent}}
\item Statements \ref{itm:at_least_four}, \ref{itm:exactly_seven}, \ref{itm:not_more_than_two}, and \ref{itm:martians} \answer{\underline{inconsistent}}
\item Statements \ref{itm:at_least_four}, \ref{itm:exactly_seven}, and \ref{itm:martians}\answer{\underline{consistent}}
\item Statements \ref{itm:at_least_four}, \ref{itm:exactly_seven}, and \ref{itm:not_more_than_two} \answer{\underline{consistent}}
\end{exercises}

\noindent \problempart Consider the following set of statements.
\begin{enumerate}[label=(\alph*)]
\item \label{itm:allmortal} All people are mortal.
\item \label{itm:socperson} Socrates is a person.
\item \label{itm:socnotdie} Socrates will never die.
\item \label{itm:socmortal} Socrates is mortal.
\end{enumerate}
Which combinations of statements form consistent sets? Mark each “consistent” or “inconsistent.”
\begin{exercises}
\item Statements \ref{itm:allmortal}, \ref{itm:socperson}, and \ref{itm:socnotdie}  \answer{\underline{Inconsistent}}
\item Statements \ref{itm:socperson}, \ref{itm:socnotdie}, and \ref{itm:socmortal} \answer{\underline{Inconsistent}}
\item Statements \ref{itm:socperson} and \ref{itm:socnotdie} \answer{\underline{Consistent}}
\item Statements \ref{itm:allmortal} and \ref{itm:socmortal} \answer{\underline{Consistent}}
\item Statements \ref{itm:allmortal}, \ref{itm:socperson}, \ref{itm:socnotdie}, and \ref{itm:socmortal} \answer{\underline{Inconsistent}} 
\end{exercises}

\noindent \problempart \label{pr.EnglishCombinations} Which of the following is possible? If it is possible, give an example. If it is not possible, explain why.


\begin{longtabu}{p{.1\linewidth}p{.9\linewidth}}
\textbf{Example}: & A valid argument that has one false premise and one true premise.\\
\textbf{Answer}: & Possible: Example: If Taylor Swift were a kangaroo, she would be a marsupial (true). Taylor Swift is a kangaroo. (False.) Therefore Taylor Swift is a marsupial (false.)\\ &Remember, if an argument is valid, the only thing that can't happen is for it to have all true premises and a false conclusion. So if you don't specify a false conclusion anything is possible.\\
\end{longtabu}



\begin{exercises}
\item A false tautology. 

\answer{Impossible. Tautologies, by definition, are always true.}

\item A valid argument that has a false conclusion

\answer{\underline{Possible}. Example: If grass is green, then I am the pope. (False) Grass is green. (True) \therefore  I am the pope. (False)}

\item A valid argument, the conclusion of which is a contradiction

\answer{\underline{Possible}. The conclusion is always false, but if the premises are also always false, you are fine. Example: If A, then not A. \therefore If B, then not B. \\}

\item An invalid argument, the conclusion of which is a tautology

\answer{\underline{Impossible}. If the conclusion is always true, then the there is no way for all the premises to be true and conclusion false.\\}

\item A tautology that is contingent

\answer{\underline{Impossible}. Contradictions, contingencies, and tautologies are exclusive categories. If you are one, you can't be either of the others. \\}


\item Two logically equivalent sentences, both of which are tautologies

\answer{\underline{Possible} In fact, all tautologies are logically equivalent. Logically equivalent sentences always have the same truth value, and all tautologies are always true. \\}


\item Two logically equivalent sentences, one of which is a tautology and one of which is contingent

\answer{\underline{Impossible}. A tautology is always true, but contingent sentences can be false. Therefore they can have different truth values. \\}


\item Two logically equivalent sentences that together are an inconsistent set

\answer{\underline{Possible} Two contradictions are logically equivalent, however it is impossible for them to both be true, because it is impossible for either one to be true. \\}


\item A consistent set of sentences that contains a contradiction

\answer{\underline{Impossible}. The contradiction can never be true, so the whole set cannot never all be true. \\}


\item An inconsistent set of sentences that contains a tautology
\answer{\underline{Possible}. Example: A, Not A, If A then A.} 
\end{exercises}

\noindent \problempart Which of the following is possible? If it is possible, give an example. If it is not possible, explain why.
\answer{All answers, except for the last question, are by Ben Sheredos}
\begin{exercises}
\item A valid argument, whose premises are all tautologies, and whose conclusion is contingent
\answer{Not Possible. If the argument is valid, then the conclusion must be true if the premises are true. If the premises are \textit{tautologies}, then the premises are \textit{always} true, and so the conclusion also must always be true.}

\item A valid argument with true premises and a false conclusion
\answer{ \textit{Absolutely not!} This contradicts the very definition of a valid argument.
}
\item A consistent set of sentences that contains two sentences that are not logically equivalent
\answer{ Most definitely. Here are two sentences that are consistent but not logically equivalent: ``Today is a Wednesday'' and ``I like pie.''
}
\item A consistent set of sentences, all of which are contingent
\answer{For sure. See the examples given in the previous answer. Both are contingent (sometimes it's not Wednesday today, and I might've hated pie.)
}
\item A false tautology
\answer{Not possible. By definition, a tautology is always true.
}
\item A valid argument with false premises
\answer{ Yup. Because validity only requires that \textit{if} the premises are true, \textit{then} the conclusion must be true. But all of them could be false, and the argument would remain valid. 
}
\item A logically equivalent pair of sentences that are not consistent
\answer{ Careful here. Our definition of consistency is that a set of statements are consistent if they could all be true at the same time. Well, consider the case of 2 statements which are logically equivalent, and which are both \textit{contradictions}. Neither can be true. So they cannot \textit{both} be true. So they are not consistent. 
}
\item A tautological contradiction
\answer{ Impossible. This is gibberish-nonsense.
}
\item A consistent set of sentences that are all contradictions
\answer{ Nope: see again \#7 above. If a set of statements contains nothing but contradictions, then none of them can be true. But if none of them can be true, then they cannot be true together, and so they cannot be consistent.
}

\item A valid argument, whose premises are all tautologies, and whose conclusion is contingent.
\answer{Impossible. If the conclusion is contingent, then it could be false, in which case you would have true premises and a false conclusion, which would make the argument invalid.}

\end{exercises}

\section*{Key Terms}
\begin{sortedlist}
\sortitem{Truth value}{} 	
\sortitem{Natural language}{}
\sortitem{Artificial language}{}
\sortitem{Formal language}{}
\sortitem{Tautology}{}
\sortitem{Contradiction}{}
\sortitem{Contingent statement}{}
\sortitem{Logically equivalent}{}
\sortitem{Contradictories}{}
\sortitem{Consistent}{}
\sortitem{Inconsistent}{}
\sortitem{Formal logic as concern for logical form}{}
\sortitem{Formal logic as strictly following rules}{}
\sortitem{Bivalent}{}
\end{sortedlist}
 \label{part:formal_logic}
%\part{Categorical Logic}
%\label{part:cat_logic}
%\chapter{Categorical Statements}
\label{chap:catstatements}
\markright{Chap. \ref{chap:catstatements}: Categorical Statements}
\setlength{\parindent}{1em}

% *********************************************
% * Quantified Categorical Statements
% **********************************************

\section{Quantified Categorical Statements}
\label{sec:qcatstatements}

Back in Chapter \ref{Chap:what_is_logic}, we saw that a statement was a unit of language that could be true or false. In this chapter and the next we are going to look at a particular kind of statement, called a quantified categorical statement, and begin to develop a formal theory of how to create arguments using these statements. This kind of logic is generally called ``categorical'' or ``Aristotelian'' logic, because it was originally invented by the great logician and philosopher Aristotle in the fourth century \textsc{bce}. This kind of logic dominated the European and Islamic worlds for 20 centuries afterward, and was expanded in all kinds of fascinating ways, some of which we will look at here.
 
%add a cross reference to the interesting historical sidebar for the Aristotelian tradition. 

Consider the following propositions:

\begin{enumerate}[label=(\alph*)]
\item \label{itm:dogs} All dogs are mammals.

\item \label{itm:physicists} Most physicists are male.

\item \label{itm:teachers} Few teachers are rock climbers.

\item \label{itm:no_dogs} No dogs are cats. 

\item \label{itm:americans} Some Americans are doctors. 

\item \label{itm:adults}Some adults are not logicians. 

\item \label{itm:canadians} Thirty percent of Canadians speak French.

\item \label{itm:chair}One chair is missing.

\end{enumerate}

\newglossaryentry{quantified categorical statement}
{
  %type=catstatements,
  name=quantified categorical statement,
  description={A statement that makes a claim about a certain quantity of the members of a class or group.}
}

These are all examples of quantified categorical statements. A  \textsc{\gls{quantified categorical statement}} \label{def:quantified_categorical_statement} is a statement that makes a claim about a certain quantity of the members of a class or group. (Sometimes we will just call these ``categorical statements'') Statement \ref{itm:dogs}, for example, is about the class of dogs and the class of mammals. These statements make no mention of any particular members of the categories or classes or types they are about. The propositions are also \textit{quantified} in that they state \textit{how many} of the things in one class are also members of the other. For instance, statement \ref{itm:physicists} talks about \textit{most} physicists, while statement \ref{itm:teachers} talks about \textit{few} teachers.  

\newglossaryentry{quantifier}
{
  %type=catstatements,
  name=quantifier,
  description={The part of a categorical sentence that specifies a portion of a class.}
}

\newglossaryentry{subject class}
{
  %type=catstatements,
  name=subject class,
  description={The first class named in a quantified categorical statement.}
}

\newglossaryentry{predicate class}
{
  %type=catsyllogisms,
  name=predicate class,
  description={The second class named in a quantified categorical statement.}
  }

\newglossaryentry{copula}
{
  name=copula,
  description={The form of the verb ``to be'' that links subject and predicate.}
}

Categorical statements can be broken down into four parts: the quantifier, the subject term, the predicate term, and the copula. The \textsc{\gls{quantifier}} \label{def:quantifier} is the part of a categorical sentence that specifies a portion of a class. It is the ``how many'' term. The quantifiers in the sentences above are all, most, few, no, some, thirty percent, and one. Notice that the ``no'' in sentence \ref{itm:no_dogs} counts as a quantifier, the same way zero counts as a number. The subject and predicate terms are the two classes the statement talks about. The \textsc{\gls{subject class}} \label{def:subject_class} is the first class mentioned in a quantified categorical statement, and the \textsc{\gls{predicate class}} \label{def:predicate_class} is the second. In sentence \ref{itm:americans}, for instance, the subject class is the class of Americans and the predicate class is the class of doctors.  The \textsc{\gls{copula}} \label{def:copula} is simply the form of the verb ``to be'' that links subject and predicate. Notice that the quantifier is always referring to the subject. The statement ``Thirty percent of Canadians speak French'' is saying something about a portion of Canadians, not about a portion of French speakers.

Sentence \ref{itm:canadians} is a little different than the others. In sentence \ref{itm:canadians} the subject is the class of Canadians and the predicate is the class of people who speak French. That's not quite the way it is written, however. There is no explicit copula, and instead of giving a noun phrase for the predicate term, like ``people who speak French,'' it has a verb phrase, ``speak French.'' If you are asked to identify the copula and predicate for a sentence like this, you should say that the copula is implicit and transform the verb phrase into a noun phrase. You would do something similar for sentence \ref{itm:chair}: the subject term is ``chair,'' and the predicate term is ``things that are missing.'' We will go into more detail about these issues in Section \ref{sec:transformation}.  


\begin{figure}
\begin{mdframed}[style=mytableclearbox]
\includegraphics*[width=\linewidth]{img/partsofacategoricalstatement}
\end{mdframed}
\caption{Parts of a quantified categorical statement.}
\label{fig:Partsofacategoricalstatement}
\end{figure}

In the previous chapter we noted that formal logic achieves content neutrality by replacing some or all of the ordinary words in a statement with symbols. For categorical logic, we are only going to be making one such substitution. Sometimes we will replace the classes referred to in a quantified categorical statement with capital letters that act as variables. Typically we will use the letter $S$ when referring to the class in the subject term and $P$ when referring to the predicate term, although sometimes more letters will be needed. Thus the sentence ``Some Americans are doctors,'' above, will sometimes become ``Some $S$ are $P$.'' The sentence ``No dogs are cats'' will sometimes become``No $S$ is $P$.''

 %%%%%%%%%%%%%% practice problems %%%%%%%%%%% 

\practiceproblems
\problempart For each of the following sentences identify the quantifier, the subject term, the predicate term, and the copula. Some of these are like the example ``Thirty percent of Canadians speak French'' where the copula is implicit and the predicate needs to be transformed into a noun phrase. 

\begin{longtabu}{p{.1\linewidth}p{.9\linewidth}}
\textbf{Example}: & Some dinosaurs had feathers\\
\textbf{Answer}: & Quantifier: Some \\
&Subject term: Dinosaurs \\
&Copula: Implicit \\
& Predicate term: Things with feathers \\
\end{longtabu}

\begin{exercises}
\item Some politicians are not members of tennis clubs.

\answer{
 Quantifier: Some \\
 Subject term: Politicians \\
 Copula: Are \\
 Predicate term: Things that are not members of tennis clubs \\ 
 }
\item All dogs go to heaven.

\answer{
 Quantifier:  All \\
 Subject term:  Dogs \\ 
 Copula:  Implicit \\
 Predicate term: Things that go to heaven \\
 }

\item Most things in the fridge are moldy.

\answer{
 Quantifier:  Most \\
 Subject term: Things in the fridge  \\
 Copula:  Are \\
 Predicate term: Things that are moldy  \\
}
\item Some birds do not fly.

\answer{
 Quantifier:  Some \\
 Subject term: Birds \\
 Copula: Implicit \\
 Predicate term: Things that do not fly. \\  
 }

\item Few people have seen Janet relaxed and happy.

\answer{
 Quantifier: Few  \\
 Subject term: People \\  
 Copula:  Implicit \\ 
 Predicate term:  People who have seen Janet relaxed and happy. \\
 }
\item No elephants are pocket-sized.

\answer{
 Quantifier: No \\
 Subject term:  Elephants \\
 Copula: Are  \\
 Predicate term: Things that are pocket sized. \\
}
\item Two thirds of Americans are obese or overweight.

\answer{
 Quantifier:  Two thirds \\
 Subject term:  Americans \\
 Copula:  Are\\
 Predicate term: People who are obese or overweight. \\
}
\item All applicants must submit to a background check. 

\answer{
 Quantifier:  All \\
 Subject term:  Applicants \\
 Copula:  Implicit \\
 Predicate term:  People who must submit to a background check. \\
 }
\item All handguns are weapons.

\answer{
 Quantifier:  All \\
 Subject term: Handguns  \\
 Copula:  Are\\
 Predicate term: Weapons  \\
 } 
\item One man stands alone against injustice.

\answer{
 Quantifier:  One \\
 Subject term:  Man \\
 Copula:  Implicit \\
 Predicate term:  People who stand alone against injustice. \\
 }
\end{exercises}



\noindent \problempart For each of the following sentences identify the quantifier, the subject term, the predicate term, and the copula. Some of these are like the example ``Thirty percent of Canadians speak French'' where the copula is implicit and the predicate needs to be transformed into a noun phrase. 


\begin{exercises}
\item No dog has been to Mars.

\answer{
 Quantifier:  No\\
 Subject term:  Dog\\
 Copula:  implicit\\
 Predicate term:  Things that have been to Mars \\
 }

\item All human beings are mortal.

\answer{
 Quantifier:  All\\
 Subject term:  Human beings\\
 Copula:  Are\\
 Predicate term:  Things that are mortal \\
 }

\item Some spears are six feet long.

\answer{
 Quantifier:  Some\\
 Subject term:  Spears \\
 Copula:  Are \\
 Predicate term:  Things that are six feet long \\
 }

\item Most dogs are friendly 

\answer{
 Quantifier:  Most\\
 Subject term:  Dogs\\
 Copula:  Are\\
 Predicate term:  Things that are friendly \\
 }

\item Eighty percent of Americans graduate from high school.

\answer{
 Quantifier:  Eighty percent\\
 Subject term:  Americans \\
 Copula:  Implicit\\
 Predicate term:   People who graduate from high school\\
 }

\item Few doctors are poor. 

\answer{
 Quantifier:  Few\\
 Subject term:  Doctors\\
 Copula:  Are\\
 Predicate term:   People who are poor\\
 }

\item All squids are cephalopods. 

\answer{
 Quantifier:  All\\
 Subject term:  Squids\\
 Copula:  Are \\
 Predicate term:   cephalopods\\
 }

\item No fish can sing.

\answer{
 Quantifier:  No\\
 Subject term:  Fish\\
 Copula:  Implicit\\
 Predicate term:  Things that can sing \\
 }

\item Some songs are sad.

\answer{
 Quantifier:  Some\\
 Subject term:  Songs\\
 Copula:  Are\\
 Predicate term:  Things that are sad \\
 }

\item Two dogs are playing in the backyard.

\answer{
 Quantifier:  Two \\
 Subject term:  Dogs \\
 Copula:  Are \\
 Predicate term:   Things that are playing in the backyard\\
 }

\end{exercises}


% **********************************
% * Quantity, Quality, and Distribution     *
% **********************************

\section{Quantity, Quality, Distribution, and Venn Diagrams}
\label{sec:QQDVD}
Ordinary English contains all kinds of quantifiers, including the counting numbers themselves. In this chapter and the next, however, we are only going to deal with two quantifiers: ``all,'' and ``some.'' We are restricting ourselves to the quantifiers ``all'' and ``some'' because they are the ones that can easily be combined to create valid arguments using the system of logic that was invented by Aristotle. We will deal with other quantifiers in chapter in the larger version of the text on induction. \label{ver_var} \nix{Chapter \ref{chap:induction}, on induction.} There we will talk about much more specific quantified statements, like ``Thirty percent of Canadians speak French,'' and do a little bit of work with modern statistical methods. 

\newglossaryentry{quantity}
{
name=quantity,
description={The portion of the subject class described by a categorical statement. Generally ``some'' or ``none.''}
}

\newglossaryentry{universal}
{
name=universal,
description={The quantity of a statement that uses the quantifier ``all.''}
}

\newglossaryentry{particular}
{
name=particular,
description={The quantity of a statement that uses the quantifier ``some.''}
}

The quantifier used in a statement is said to give the \textsc{\gls{quantity}} \label{def:Quantity} of the statement. Statements with the quantifier ``All'' are said to be ``\textsc{\gls{universal}}'' and those with the quantifier ``some'' are said to be ``\textsc{\gls{particular}}.''

Here ``some'' will just mean ``at least one.'' So, ``some people in the room are standing'' will be true even if there is only one person standing. Also, because ``some'' means ``at least one,'' it is compatible with ``all'' statements. If I say ``some people in the room are standing'' it might actually be that \textit{all} people in the room are standing, because if all people are standing, then at least one person is standing. This can sound a little weird, because in ordinary circumstances, you wouldn't bother to point out that something applies to some members of a class when, in fact, it applies to all of them. It sounds odd to say ``\textit{some} dogs are mammals,'' when in fact they \textit{all} are. Nevertheless, when ``some'' means ``at least one'' it is perfectly true that some dogs are mammals. 


\newglossaryentry{quality}
{
name=quality,
description={The status of a categorical statement as affirmative or negative.}
}

\newglossaryentry{negative}
{
name=negative,
description={The quality of a statement containing a ``not'' or ``no.''}
}

\newglossaryentry{affirmative}
{
name=affirmative,
description={The quality of a statement without a ``not'' or a ``no.''}
}



\newglossaryentry{statement mood}
{
name=statement mood,
description={The classification of a categorical statement based on its quantity and quality.}
}

\newglossaryentry{mood-A statement}
{
name=mood-A statement,
description={A quantified categorical statement of the form ``All $S$ are $P$.''}
}

\newglossaryentry{mood-E statement}
{
name=mood-E statement,
description={A quantified categorical statement of the form ``No $S$ are $P$.''}
}

\newglossaryentry{mood-I statement}
{
name=mood-I statement,
description={A quantified categorical statement of the form ``Some $S$ are $P$.''}
}

\newglossaryentry{mood-O statement}
{
name=mood-O statement,
description={A quantified categorical statement of the form ``Some $S$ are not $P$.''}
}



In addition to talking about the quantity of statements, we will talk about their \textsc{\gls{quality}}. \label{defQuality} The quality of a statement refers to whether the statement is negated. Statements that include the words ``no'' or ``not'' are \textsc{\gls{negative}}, and other statements are \textsc{\gls{affirmative}}. Combining quantity and quality gives us four basic types of quantified categorical statements, which we call the \textsc{\glspl{statement mood}} or just ``moods.'' The four moods are labeled with the letters A, E, I, and O. Statements that are universal and affirmative are \textsc{\glspl{mood-A statement}}. Statements that are universal and negative are \textsc{\glspl{mood-E statement}}. Particular and affirmative statements are \textsc{\glspl{mood-I statement}}, and particular and negative statements are \textsc{\glspl{mood-O statement}}. (See Table \ref{tab:moods}.)


Aristotle didn't actually use those letters to name the kinds of categorical propositions. His later followers writing in Latin came up with the idea. They remembered the labels because the ``A'' and the ``I'' were in the Latin word ``\textbf{a}ff\textbf{i}rmo,'' (``I affirm'') and the ``E'' and the ``O'' were in the Latin word ``n\textbf{e}g\textbf{o}'' (``I deny''). 

\begin{table}[t]
\begin{mdframed}[style=mytablebox]
\begin{tabu}{p{.1\linewidth}p{.3\linewidth}p{.3\linewidth}}
  \underline{Mood}& \underline{Form} & \underline{Example} \\ 
A & All $S$ are $P$ & All dogs are mammals. \\
E & No S are $P$ & No dogs are reptiles. \\
I & Some $S$ are $P$ & Some birds can fly. \\
O & Some $S$ are not $P$ & Some birds cannot fly.\\
\end{tabu}
\end{mdframed}
\caption{The four moods of a categorical statement} \label{tab:moods}
\end{table}

\newglossaryentry{distribution}
{
name=distribution,
description={A property of the terms of a categorical statement that is present when the statement makes a claim about the whole term.}
}

The \textsc{\gls{distribution}} of a categorical statement refers to how the statement describes its subject and predicate class. A term in a sentence is said to be distributed \label{def:Distribution} if a claim is being made about the whole class. In the sentence ``All dogs are mammals,'' the subject class, dogs, is distributed, because the quantifier ``All'' refers to the subject. The sentence is asserting that every dog out there is a mammal. On the other hand, the predicate class, mammals, is not distributed, because the sentence isn't making a claim about all the mammals. We can infer that at least some of them are dogs, but we can't infer that all of them are dogs. So in mood-A statements, only the subject is distributed. 

On the other hand, in an I sentence like ``Some birds can fly'' the subject is not distributed. The quantifier ``some'' refers to the subject, and indicates that we are not saying something about all of that subject. We also aren't saying anything about all flying things, either. So in mood-I statements, neither subject nor predicate is distributed. 

Even though the quantifier always refers to the subject, the predicate class can be distributed as well. This happens when the statement is negative. The sentence ``No dogs are reptiles'' is making a claim about all dogs: they are all not reptiles. It is also making a claim about all reptiles: they are all not dogs. So mood-E statements distribute both subject and predicate. Finally, negative particular statements (mood-O) have only the predicate class distributed. The statement ``some birds cannot fly'' does not say anything about all birds. It does, however say something about all flying things: the class of all flying things excludes some birds. 

The quantity, quality, and distribution of the four forms of a categorical statement are given in Table \ref{tab:quantity}. The general rule to remember here is that universal statements distribute the subject, and negative statements distribute the predicate. 


\begin{table}[b]
\begin{mdframed}[style=mytablebox]
\begin{tabu}{p{.1\linewidth}p{.2\linewidth}p{.15\linewidth}p{.15\linewidth}p{.4\linewidth}}
 \underline{Mood} & \underline{Form} &  \underline{Quantity} & \underline{Quality} & \underline{Terms Distributed} \\ 
A & All $S$ are $P$ & Universal &  Affirmative & S\\
E & No $S$ are $P$ & Universal & Negative & S and P\\
I & Some $S$ are $P$ & Particular & Affirmative & None\\
O &Some $S$ are not $P$ & Particular &Negative & P \\
\end{tabu}
\end{mdframed}
\caption{Quantity, quality, and distribution.}\label{tab:quantity}
\end{table}

\newglossaryentry{Venn diagram}
{
name=Venn diagram,
description={A diagram that represents categorical statements using circles that stand for classes.}
}


In 1880 English logician John Venn published two essays on the use of diagrams with circles to represent categorical propositions (Venn \citeyear{Venn1880a}, \citeyear{Venn1880b}). Venn noted that the best use of such diagrams so far had come from the brilliant Swiss mathematician Leonhard Euler, but they still had many problems, which Venn felt could be solved by bringing in some ideas about logic from his fellow English logician George Boole. Although Venn only claimed to be building on the long logical tradition he traced, since his time these kinds of circle diagrams have been known as \textsc{\glspl{Venn diagram}}.

In this section we are going to learn to use Venn diagrams to represent our four basic types of categorical statement. Later in this chapter, we will find them useful in evaluating arguments. Let us start with a statement in mood A: ``All $S$ are $P$.'' We are going to use one circle to represent $S$ and another to represent $P$. There are a couple of different ways we could draw the circles if we wanted to represent ``All $S$ are $P$.'' One option would be to draw the circle for $S$ entirely inside the circle for $P$, as in Figure \ref{fig:euler_circles}

\begin{figure}
\begin{mdframed}[style=mytablehalfbox]
\begin{center}
\begin{tikzpicture}
\def\firstcircle{(0,0) circle (.5 cm)}
\def\secondcircle{(0:0) circle (1.25 cm)}
\draw \firstcircle node[outer sep=.25cm, above left] (s) {$S$};
\draw \secondcircle node[outer sep=.9cm, above right] (p) {$P$};
\end{tikzpicture}
\end{center}
\end{mdframed}
\caption{Euler Circles} \label{fig:euler_circles}
\end{figure}

\begin{figure}
\begin{mdframed}[style=mytableclearbox, userdefinedwidth=.3\textwidth]
\begin{center}
\includegraphics*{img/OriginalVenn}
\end{center}
\end{mdframed}
\caption{Venn's original \\ diagram for an mood-A statement \\(Venn \citeyear{Venn1880a}).}
\end{figure}


It is clear from Figure \ref{fig:euler_circles} that all $S$ are in fact $P$. And outside of college logic classes, you may have seen people use a diagram like this to represent a situation where one group is a subclass of another. You may have even seen people call concentric circles like this a Venn diagram. But Venn did not think we should put one circle entirely inside the other if we just want to represent ``All $S$ is $P$.'' Technically speaking Figure \ref{fig:euler_circles} shows Euler circles.

Venn pointed out that the circles in Figure \ref{fig:euler_circles} don't just say that ``All $S$ are $P$.'' They also says that ``All $P$ are $S$'' is false. But we don't necessarily know that if we have only asserted ``All $S$ are $P$.'' The statement ``All $S$ are $P$'' leaves it open whether the $S$ circle should be smaller than or the same size as the $P$ circle.

Venn suggested that to represent just the content of a single proposition, we should always begin by drawing partially overlapping circles. This means that we always have spaces available to represent the four possible ways the terms can combine: 

\begin{center}
\begin{tikzpicture}
\def\firstcircle{(0,0) circle (1cm)}
\def\secondcircle{(0:1.33cm) circle (1cm)}
\draw \firstcircle node[outer sep=.75cm, above left] (s) {$S$} 
	node [xshift=-.25cm] (1) {1}
	node [xshift=.66cm] (2){2};
\draw \secondcircle node[outer sep=.75cm, above right] (p) {$P$}
	node [xshift=.25cm] (3) {3}
	node [xshift=1.4cm] (4){4};
\end{tikzpicture}
\label{fig:two_circle_venn}
\end{center}

Area 1 represents things that are $S$ but not $P$; area 2, things that are $S$ and $P$; area 3, things that are just $P$; and area 4 represents things that are neither $S$ nor $P$. We can then mark up these areas to indicate whether something is there or could be there. We shade a region of the diagram to represent the claim that nothing can exist in that region. For instance, if we say ``All $S$ are $P$,'' we are asserting that nothing can exist that is in the $S$ circle unless it is also in the $P$ circle. So we shade out the part of the $S$ circle that doesn't overlap with $P$. 

\begin{center}
\begin{tikzpicture}
\def\firstcircle{(0,0) circle (1cm)}
\def\secondcircle{(0:1.33cm) circle (1cm)}
     \begin{scope}[shift={(4cm,0cm)}]
        \begin{scope}[even odd rule]% first circle without the second
            \clip \secondcircle (-1,-1) rectangle (1,1);
        \fill[gray] \firstcircle;
        \end{scope}
        \draw \firstcircle node[outer sep=.75cm, above left] (s) {$S$};
        \draw \secondcircle node[outer sep=.75cm, above right] (p) {$P$};
    \end{scope} 
\end{tikzpicture}
\end{center}


If we want to say that something does exist in a region, we put an ``x'' in it. This is the diagram for ``Some $S$ are $P$'': 

\begin{center}
\begin{tikzpicture}
\def\firstcircle{(0,0) circle (1cm)}
\def\secondcircle{(0:1.33cm) circle (1cm)}
\draw \firstcircle node[outer sep=.75cm, above left] (s) {$S$};
\node[outer sep=.45cm, right] (x) {x};
\draw \secondcircle node[outer sep=.75cm, above right] (p) {$P$};
\end{tikzpicture}
\end{center}

If a region of a Venn diagram is blank, if it is neither shaded nor has an x in it, it could go either way. Maybe such things exist, maybe they do not.

The Venn diagrams for all four basic forms of categorical statements are in Figure \ref{fig:fourvenns}. Notice that when we draw diagrams for the two universal forms, A and E, we do not draw any x's. For these forms we are only ruling out possibilities, not asserting that things actually exist. This is part of what Venn learned from Boole, and we will see its importance in Section \ref{sec:ExistentialImport}. 

Finally, notice that so far, we have only been talking about categorical statements involving the variables $S$ and $P$. Sometimes, though, we will want to represent statements in regular English. To do this, we will include a dictionary saying what the variables $S$ and $P$ represent in this case. For instance, this is the diagram for ``No dogs are reptiles.''

\begin{figure}[H]
\begin{center}
\begin{tikzpicture}
\def\firstcircle{(0,0) circle (1cm)}
\def\secondcircle{(0:1.33cm) circle (1cm)}
\draw \firstcircle node[outer sep=.75cm, above left] (s) {$S$};
\draw \secondcircle node[outer sep=.75cm, above right] (p) {$P$};
    \begin{scope}
      \clip \firstcircle;
      \fill[gray] \secondcircle;
    \end{scope}
\end{tikzpicture}
\captionsetup{singlelinecheck=on}
\caption*{$S$: Dogs \\ $P$: Reptiles}
\end{center}
\end{figure}



\begin{figure}
\begin{mdframed}[style=mytablebox]
\begin{tabu}{X[1,c,m]X[1,c,m]}

\begin{tikzpicture}
\def\firstcircle{(0,0) circle (1cm)} 
\def\secondcircle{(0:1.33cm) circle (1cm)}
     \begin{scope}[shift={(4cm,0cm)}]
        \begin{scope}[even odd rule]% first circle without the second
            \clip \secondcircle (-1,-1) rectangle (1,1);
        \fill[gray] \firstcircle;
        \end{scope}
        \draw \firstcircle node[outer sep=.75cm, above left] (s) {$S$};
        \draw \secondcircle node[outer sep=.75cm, above right] (p) {$P$};
    \end{scope} 
\end{tikzpicture}

&


\begin{tikzpicture}
\def\firstcircle{(0,0) circle (1cm)}
\def\secondcircle{(0:1.33cm) circle (1cm)}
\draw \firstcircle node[outer sep=.75cm, above left] (s) {$S$};
\draw \secondcircle node[outer sep=.75cm, above right] (p) {$P$};
    \begin{scope}
      \clip \firstcircle;
      \fill[gray] \secondcircle;
    \end{scope}
\end{tikzpicture}


\\

\textbf{A}: All $S$ are $P$

&

\textbf{E}: No $S$ are $P$


\\
\begin{tikzpicture}
\def\firstcircle{(0,0) circle (1cm)}
\def\secondcircle{(0:1.33cm) circle (1cm)}
\draw \firstcircle node[outer sep=.75cm, above left] (s) {$S$};
\node[outer sep=.44cm, right] (x) {\Large{x}};
\draw \secondcircle node[outer sep=.75cm, above right] (p) {$P$};
\end{tikzpicture}

&

\begin{tikzpicture}
\def\firstcircle{(0,0) circle (1cm)}
\def\secondcircle{(0:1.33cm) circle (1cm)}
\draw \firstcircle node[outer sep=.75cm, above left] (s) {$S$};
\node[outer sep=.3cm] (x) {\Large{x}};
\draw \secondcircle node[outer sep=.75cm, above right] (p) {$P$};
\end{tikzpicture}

\\

\textbf{I}: Some $S$ are $P$
&
\textbf{O}: Some $S$ are not $P$

\end{tabu}
\end{mdframed}
\caption{Venn Diagrams for the Four Basic Forms of a Categorical Statement}
\label{fig:fourvenns}
\end{figure}

%%%%%%%%%%%%%%% Practice problems %%%%%%%%%%

\practiceproblems

\problempart Identify each of the following sentences as A, E, I, or O; state its quantity and quality; and state which terms are distributed. Then draw the Venn Diagram for each.

\begin{longtabu}{p{.1\linewidth}p{.9\linewidth}}
\textbf{Example}: & Some dinosaurs are not herbivores \\
\textbf{Answer}: & Form: O\\
&Quantity: particular \\
&Quality: negative \\
&
%\vspace{.25em}
\noindent \begin{tikzpicture}
\def\firstcircle{(0,0) circle (.75cm)}
\def\secondcircle{(0:1cm) circle (.75cm)}
\draw \firstcircle node[outer sep=.66cm, above left] (s) {$S$};
\node[outer sep=.3cm] (x) {\Large{x}};
\draw \secondcircle node[outer sep=.66cm, above right] (p) {$P$};
\end{tikzpicture}\\
& $S$: Dinosaurs \\
& $P$: Herbivores
\end{longtabu}

\begin{exercises}

\item All gerbils are rodents.

\answer{
Form: A\\
Quantity: Universal \\
Quality: Affirmative\\

\vspace{.25em}

\noindent \begin{tikzpicture}
\def\firstcircle{(0,0) circle (.75cm)}
\def\secondcircle{(0:1cm) circle (.75cm)}
       \begin{scope}[even odd rule]% first circle without the second
            \clip \secondcircle (-1,-1) rectangle (1,1);
        \fill[red] \firstcircle;
        \end{scope}
 \draw \firstcircle node[outer sep=.66cm, above left] (s) {$S$};
%\node[outer sep=.3cm] (x) {\Large{x}};
\draw \secondcircle node[outer sep=.66cm, above right] (p) {$P$};
\end{tikzpicture}\\
$S$: Gerbils\\
$P$: Rodents
}

\item Some planets do not have life.

\answer{
Form: O\\
Quantity: Particular \\
Quality: Negative \\

\vspace{.25em}
\noindent \begin{tikzpicture}
\def\firstcircle{(0,0) circle (.75cm)}
\def\secondcircle{(0:1cm) circle (.75cm)}
\draw \firstcircle node[outer sep=.66cm, above left] (s) {$S$};
\node[outer sep=.3cm] (x) {\Large{x}};
\draw \secondcircle node[outer sep=.66cm, above right] (p) {$P$};
\end{tikzpicture}\\
 $S$: Planets\\
 $P$: Things that have life
}

\item Some manatees are not rappers.

\answer{
Form: O\\
Quantity: Particular \\
Quality: Negative \\

\vspace{.25em}
\noindent \begin{tikzpicture}
\def\firstcircle{(0,0) circle (.75cm)}
\def\secondcircle{(0:1cm) circle (.75cm)}
\draw \firstcircle node[outer sep=.66cm, above left] (s) {$S$};
\node[outer sep=.3cm] (x) {\Large{x}};
\draw \secondcircle node[outer sep=.66cm, above right] (p) {$P$};
\end{tikzpicture}\\
 $S$: Manatees\\
 $P$: Rappers
}


\item All rooms have televisions.

\answer{
Form: A\\
Quantity: Universal \\
Quality: Affirmative\\

\vspace{.25em}

\noindent \begin{tikzpicture}
\def\firstcircle{(0,0) circle (.75cm)}
\def\secondcircle{(0:1cm) circle (.75cm)}
       \begin{scope}[even odd rule]% first circle without the second
            \clip \secondcircle (-1,-1) rectangle (1,1);
        \fill[red] \firstcircle;
        \end{scope}
 \draw \firstcircle node[outer sep=.66cm, above left] (s) {$S$};
%\node[outer sep=.3cm] (x) {\Large{x}};
\draw \secondcircle node[outer sep=.66cm, above right] (p) {$P$};
\end{tikzpicture}\\
$S$: Rooms\\
$P$: Things that have televisions
}


\item All stores are closed.

\answer{
Form: A\\
Quantity: Universal \\
Quality: Affirmative\\

\vspace{.25em}

\noindent \begin{tikzpicture}
\def\firstcircle{(0,0) circle (.75cm)}
\def\secondcircle{(0:1cm) circle (.75cm)}
       \begin{scope}[even odd rule]% first circle without the second
            \clip \secondcircle (-1,-1) rectangle (1,1);
        \fill[red] \firstcircle;
        \end{scope}
 \draw \firstcircle node[outer sep=.66cm, above left] (s) {$S$};
%\node[outer sep=.3cm] (x) {\Large{x}};
\draw \secondcircle node[outer sep=.66cm, above right] (p) {$P$};
\end{tikzpicture}\\
$S$: Stores\\
$P$: Things that are closed
}


\item Some dancers are graceful. 

\answer{
Form: I\\
Quantity: Particular \\
Quality: Affirmative\\

\vspace{.25em}

\noindent \begin{tikzpicture}
\def\firstcircle{(0,0) circle (.75cm)}
\def\secondcircle{(0:1cm) circle (.75cm)}
%       \begin{scope}[even odd rule]% first circle without the second
%            \clip \secondcircle (-1,-1) rectangle (1,1);
%        \fill[red] \firstcircle;
%        \end{scope}
 \draw \firstcircle node[outer sep=.66cm, above left] (s) {$S$};
%\node[outer sep=.3cm] (x) {\Large{x}};
\node[outer sep=.3cm, xshift=.5cm] (x) {\Large{x}};
\draw \secondcircle node[outer sep=.66cm, above right] (p) {$P$};
\end{tikzpicture}\\
$S$: Dancers\\
$P$: Things that are graceful
}


\item No extraterrestrials are in Cleveland. 

\answer{
Form: E\\
Quantity: Universal \\
Quality: Negative\\

\vspace{.25em}

\noindent \begin{tikzpicture}
\def\firstcircle{(0,0) circle (.75cm)}
\def\secondcircle{(0:1cm) circle (.75cm)}
%       \begin{scope}[even odd rule]% first circle without the second
%            \clip \secondcircle (-1,-1) rectangle (1,1);
%        \fill[red] \firstcircle;
%        \end{scope}
    \begin{scope}
      \clip \firstcircle;
      \fill[red] \secondcircle;
    \end{scope}
\draw \firstcircle node[outer sep=.66cm, above left] (s) {$S$};
%\node[outer sep=.3cm] (x) {\Large{x}};
%\node[outer sep=.3cm, xshift=.5cm] (x) {\Large{x}};
\draw \secondcircle node[outer sep=.66cm, above right] (p) {$P$};
\end{tikzpicture}\\
$S$: Extraterrestrial\\
$P$: Things in Cleveland
}

\item Some crates are empty.

\answer{
Form: I\\
Quantity: Particular \\
Quality: Affirmative\\

\vspace{.25em}

\noindent \begin{tikzpicture}
\def\firstcircle{(0,0) circle (.75cm)}
\def\secondcircle{(0:1cm) circle (.75cm)}
%       \begin{scope}[even odd rule]% first circle without the second
%            \clip \secondcircle (-1,-1) rectangle (1,1);
%        \fill[red] \firstcircle;
%        \end{scope}
%    \begin{scope}
%      \clip \firstcircle;
%      \fill[red] \secondcircle;
%    \end{scope}
\draw \firstcircle node[outer sep=.66cm, above left] (s) {$S$};
%\node[outer sep=.3cm] (x) {\Large{x}};
\node[outer sep=.3cm, xshift=.5cm] (x) {\Large{x}};
\draw \secondcircle node[outer sep=.66cm, above right] (p) {$P$};
\end{tikzpicture}\\
$S$: Crates\\
$P$: Things that are empty
}



\item No customers are mistaken.


\answer{
Form: E\\
Quantity: Universal \\
Quality: Negative\\

\vspace{.25em}

\noindent \begin{tikzpicture}
\def\firstcircle{(0,0) circle (.75cm)}
\def\secondcircle{(0:1cm) circle (.75cm)}
%       \begin{scope}[even odd rule]% first circle without the second
%            \clip \secondcircle (-1,-1) rectangle (1,1);
%        \fill[red] \firstcircle;
%        \end{scope}
    \begin{scope}
      \clip \firstcircle;
      \fill[red] \secondcircle;
    \end{scope}
\draw \firstcircle node[outer sep=.66cm, above left] (s) {$S$};
%\node[outer sep=.3cm] (x) {\Large{x}};
%\node[outer sep=.3cm, xshift=.5cm] (x) {\Large{x}};
\draw \secondcircle node[outer sep=.66cm, above right] (p) {$P$};
\end{tikzpicture}\\
$S$: Customers\\
$P$: Things that are mistaken}


\item All cats love catnip.

\answer{
Form: A\\
Quantity: Universal \\
Quality: Affirmative\\

\vspace{.25em}

\noindent \begin{tikzpicture}
\def\firstcircle{(0,0) circle (.75cm)}
\def\secondcircle{(0:1cm) circle (.75cm)}
       \begin{scope}[even odd rule]% first circle without the second
            \clip \secondcircle (-1,-1) rectangle (1,1);
        \fill[red] \firstcircle;
        \end{scope}
%    \begin{scope}
%      \clip \firstcircle;
%      \fill[red] \secondcircle;
%    \end{scope}
\draw \firstcircle node[outer sep=.66cm, above left] (s) {$S$};
%\node[outer sep=.3cm] (x) {\Large{x}};
%\node[outer sep=.3cm, xshift=.5cm] (x) {\Large{x}};
\draw \secondcircle node[outer sep=.66cm, above right] (p) {$P$};
\end{tikzpicture}\\
$S$: Cats\\
$P$: Things that love catnip
}

\end{exercises}

\noindent \problempart Identify each of the following sentences as A, E, I, or O; state its quantity and quality; and state which terms are distributed. Then draw the Venn Diagram for each.

\begin{exercises}


\item No appeals are rejected.

\item All bagels are boiled.

\item Some employees are late.

\item All forgeries are discovered eventually.

\item Some shirts are purple.

\item Some societies are matriarchal.

\item No sunflowers are blue.

\item Some appetizers are filling. 

\item Some jokes are funny.

\item Some arguments are invalid. 

\end{exercises}

\noindent\problempart Transform the following sentences by switching their quantity, but not their quality.

\textbf{Example}: Some dogs have fleas. \\
\textbf{Answer}: All dogs have fleas.

\begin{exercises}
\item Some trees are not evergreen. \answer{ No trees are evergreen}
\item All smurfs are blue. \answer{ Some smurfs are blue}
\item Some swords are sharp. \answer{ All swords are sharp}
\item Some sweaters are not soft. \answer{ No sweaters are soft}
\item All snails are invertebrates. \answer{Some snails are invertebrates}
\end{exercises}

\noindent\problempart Transform the following sentences by switching their quantity, but not their quality.

\begin{exercises}
\item Some puffins are not large. 
\item Some Smurfs are female.
\item All guitars are stringed instruments. 
\item No lobsters are extraterrestrial.
\item Some metals are alloys 
\end{exercises}

\noindent\problempart Transform the following sentences by switching their quality, but not their quantity.

\textbf{Example}: Some elephants are in zoos. \\
\textbf{Answer}: Some elephants are not in zoos. 

\begin{exercises}
\item Some lobsters are white. \answer{Some lobsters are not white.}

\item Some responsibilities are onerous. \answer{ Some responsibilities are not onerous}

\item No walls are bridges. \answer{All walls are bridges}

\item Some riddles are not clever.\answer{ Some riddles are clever}
 
\item All red things are colored. \answer{No red things are colored}
\end{exercises}

\noindent\problempart Transform the following sentences by switching their quality, but not their quantity.

\begin{exercises}
\item All drums are musical instruments. 
\item No grandsons are female. 
\item Some crimes are felonies.
\item Some airplanes are not commercial. 
\item All scorpions are arachnids. 
\end{exercises}

\noindent\problempart Transform the following sentences by switching both their quality and quantity.

\textbf{Example}: No sharks are virtuous. \\
\textbf{Answer}: Some sharks are virtuous. 

\begin{exercises}
\item No lobsters are vertebrates. \answer{Some lobsters are vertebrates}

\item Some colors are not pastel. \answer{All colors are pastel}

\item All tents are temporary structures.\answer{Some tents are not temporary structures}

\item No goats are bipeds. \answer{Some goats are bipeds}
 
\item Some shirts are plaid.\answer{ No shirts are plaid}
\end{exercises}

\noindent\problempart Transform the following sentences by switching both their quality and quantity.

\begin{exercises}
\item No shirts are pants. 
\item All ducks are birds.  
\item Some possibilities are not likely events.
\item Some raincoats are blue. 
\item Some days are holidays. 
\end{exercises}

% *****************************************************
% * Transforming English into logically structured English.        *
% *****************************************************

\section{Transforming English into Logically Structured English} \label{sec:transformation} 

\newglossaryentry{logically structured English}
{
name=logically structured English,
description={English that has been regimented into a standard form to make its logical structure clear and to remove ambiguity. A stepping stone to full-fledged formal languages.}
}

Because the four basic forms are stated using variables, they have a great deal of generality. We can expand on that generality by showing how many different kinds of English sentences can be represented as sentences in our four basic forms. We already touched on this a little in section \ref{sec:qcatstatements}, when we look at sentences like ``Thirty percent of Canadians speak French.'' There we saw that the predicate was not explicitly a class. We needed to change ``speak French'' to ``people who speak French.'' In this section, we are going to expand on that to show how ordinary English sentences can be transformed into something we will call ``logically structured English.'' \textsc{\gls{logically structured English}} \label{def:LSE} is English that has been put into a standardized form that allows us to see its logical structure more clearly and removes ambiguity.  Doing this is a step towards the creation of formal languages, which we will start doing in Chapter \ref{chap:SL}. 

Transforming English sentences into logically structured English is fundamentally a matter of understanding the meaning of the English sentence and then finding the logically structured English statements with the same or similar meaning. Sometimes this will require judgment calls. English, like any natural  language, is fraught with ambiguity. One of our goals with logically structured English is to reduce the amount of ambiguity. Clarifying ambiguous sentences will always require making judgments that can be questioned. Things will only get harder when we start using full blown formal languages in Chapter \ref{chap:SL}, which are supposed to be completely free of ambiguity.

\newglossaryentry{standard form for a categorical statement}
{
name=standard form for a categorical statement,
description={A categorical statement that has been put into logically structured English, with the following elements in the following order: (1) The quantifiers ``all,'' ``some,'' or ``no''; (2) the subject term; (3) the copula ``are'' or ``are not''; and (4) the predicate term.}
}


To transform a quantified categorical statement into logically structured English, we have to put all of its elements in a fixed order and be sure they are all of the right type. All statements must begin with the quantifies ``All'' or ``Some'' or the negated quantifier ``No.'' Next comes the subject term, which must be a plural noun, a noun phrase, or a variable that stands for any plural noun or noun phrase. Then comes the copula ``are'' or the negated copula ``are not.'' Last is the predicate term, which must also be a plural noun or noun phrase. We also specify that you can only say ``are not'' with the quantifier ``some,'' that way the universal negative statement is always phrased ``No $S$ are $P$,'' not ``All S are not $P$.'' Taken together, these criteria define the \textsc{\gls{standard form for a categorical statement}} in logically structured English. \label{def:standard_form_cat_statement}

The subsections below identify different kinds of changes you might need to make to put a statement into logically structured English. Sometimes translating a sentence will require using multiple changes.

\subsection{Change the Predicate into a Noun Phrase}
\label{subsec:predicate_noun_phrase}
In section \ref{sec:qcatstatements} we saw that ``Some Canadians speak French'' has a verb phrase ``speaks French'' instead of a copula and a plural noun phrase. To transform these sentences into logically structured English, you need to add the copula and turn all the terms into plural nouns or plural noun phrases. Adding a plural noun phrase means you have to come up with some category, like ``people'' or ``animals.'' When in doubt, you can always use the most general category, ``things.'' Below are some examples

\begin{longtabu}{p{.5\linewidth}p{.5\linewidth}}
\underline{English} &
\underline{Logically Structured English} \\
\endhead 
No cats bark. &
No cats are animals that bark.  \\ 

All birds can fly. &
All birds are animals that can fly.  \\

Some thoughts should be left unsaid. &
Some thoughts are things that should be left unsaid. 
\end{longtabu}

\noindent Sometimes English sentences will have a copula and an adjective or adjective phrase as the predicate. These need to be changed to noun phrases, just as the verb phrases did. 

\begin{longtabu}{p{.5\linewidth}p{.5\linewidth}}
\underline{English} &
\underline{Logically Structured English}  \\
\endhead 
Some roses are red. &
Some roses are red flowers. \\

Football players are strong. &
All football players are strong persons. \\

Some names are hurtful. &
Some names are hurtful things.
\end{longtabu}

\noindent Again, you will have to come up with a category for the predicate, and when it doubt, you can just use ``things.''

\subsection{Standardize the Quantifier}
\label{subsec:standardize_quantifier}

English has a wide variety of ways to express quantity. We need to reduce all of these to either ``all'' or ``some,'' plus negations.  Here are some examples

\begin{longtabu}{p{.5\linewidth}p{.5\linewidth}}
\underline{English} &
\underline{Logically Structured English} \\
\endhead 

Most people with a PhD in psychology are female. &
Some people with a PhD in psychology are female. \\

Among the things that Sylvia inherited was a large mirror &
Some things that Sylvia inherited were large mirrors\\

There are Americans that are doctors. &
Some Americans are doctors. \\

At least a few Americans are doctors.&
Some Americans are doctors. \\

A man is walking down the street. &
Some men are things that are walking down the street.\\


Every day is a blessing. &
All days are blessings. \\

Whatever is a dog is not a cat. &
No dogs are cats. \\

Not a single dog is a cat. &
No dogs are cats. \\

Take nothing for granted &
No things are things that should be taken for granted \\

Something is rotten in Denmark &
Some things are things that are rotten in Denmark\\

Everything is coming up roses & 
Sll things are things that are coming up roses\\


``What does not destroy me, makes me stronger.'' --Friedrich Nietzsche & 
All things that do not destroy me are things that make me stronger. \\


\end{longtabu}

Notice in the last case we are losing quite a bit of information when we transform the sentence into logically structured English. ``Most'' means more that fifty percent, while ``some'' could be any percentage less than a hundred. This is simply a price we have to pay in creating a standard logical form. As we will see when we move to constructing artificial languages in Chapter \ref{chap:SL}, no logical language has the expressive richness of a natural language. 

Sometimes universal statements in English don't have an explicit quantifier. Instead they use a plural noun or indefinite article to express generality. 

\begin{longtabu}{p{.5\linewidth}p{.5\linewidth}}
\underline{English} &
\underline{Logically Structured English} \\
\endhead 
Boots are footwear. &
All boots are footwear.\\

Giraffes are tall. &
All giraffes are tall things.\\

A dog is not a cat. & 
No dogs are cats.\\

A lion is a fierce creature. &
All lions are fierce creatures.\\

\end{longtabu}

\noindent Notice that in the second sentence we had to make two changes, adding both the words ``All'' and ``things.''

In the last two sentences, the indefinite article ``a'' is being used to create a kind of generic sentence. Not all sentences using the indefinite article work this way. The list before this one included the example ``A man is walking down the street.'' This sentence is not talking about all men generically. It is talking about a specific man whose identity is unknown. Here the indefinite article is being used like a nonstandard version of the quantifier ``some,'' which is why it appeared in the earlier list. You will have to use your good judgment and understanding of context to know when the indefinite article is being used like the word ``all'' and when it is being used like the word ``some.''

English also uses specialized adverbial phrases as quantifiers for people, places and times. If we want to talk about all people, we use a specialized quantifier like ``everyone,'' ``someone'' or ``no one.'' We use ``everywhere,'' ``somehwere,'' and ``nowhere'' for places, and ``always,'' ``sometimes,'' and ``never'' for times.  All of these need to be transformed into using the simple quantifiers ``all'' or ``some,'' plus negations.  

\tabulinesep=1.25ex
\begin{longtabu}{p{.5\linewidth}p{.5\linewidth}}
\underline{English} &
\underline{Logically Structured English} \\
\endhead 
Someone in America is a doctor. &
Some Americans are doctors. \\

Not everyone who is an adult is a logician. &
Some adults are not logicians. \\

``Whenever you need me, I'll be there.'' -- Michael Jackson &
All times that you need me are times that I will be there. \\

``We are never, ever, ever getting back together.'' -- Taylor Swift &
No times are times when we will get back together.\\

``Whoever fights with monsters should be careful lest he thereby become a monster.'' --Friedrich Nietzsche & 
All persons who fight with monsters are persons who should be careful lest they become a monster.\\
 
\end{longtabu}
%\tabulinesep=.75ex

\subsection{Standardize Alternative Universal Forms}
\label{subsec:alternative_universals}

Many constructions in English can be represented as universal statements in Logically Structured English, either affirmative (A) or negative (E)

For instance, it turns out that statements about individual people or specific objects can be represented by A or E statements. This is not something Aristotle originally noticed. For him a statement like ``Socrates is mortal,'' for Aristotle, were neither universal nor particular. They were a third class he called ``singular.'' The power of categorical logic was expanded considerably when it was realized singular statements can converted into universal statements. The trick is to add a phrase like ``All things identical to\ldots'' to our singular sentence. Essentially we are adding a universal quantifier that only picks out one specific object.

\begin{longtabu}{p{.5\linewidth}p{.5\linewidth}}
\underline{English} &
\underline{Logically Structured English} \\
\endhead 
Socrates is mortal.&
All persons identical with Socrates are mortal. \\

The Empire State Building is tall. &
All things identical to The Empire State Building are tall things. \\

Ludwig was not happy. &
No persons identical with Ludwig are happy. \\

\end{longtabu}


%
%\subsection{``It Is False That''}
%
%In English, we can say ``It is false that \ldots'' or ``It is not the case that \ldots'' to indicate that a statement is false. A lawyer, for instance, might say ``It is not the case that you signed the consent form before the doctor did the procedure.'' In logically structured English, we can make these simpler by converting the negated proposition to its contradictory. A negated A statement will become an O statement, and vice versa. Likewise a negating E statement will become and I statement, and vice versa. See the examples below. This time we have marked their form A, E, I or O, with a ``not-'' for the cases where they are negated. 
%
%\begin{tabu}{p{.5\linewidth}p{.5\linewidth}}
%\underline{English} &
%\underline{Logically Structured English} \\
%
%\textbf{not-A}: It is not the case that all dogs are pets &
%\textbf{O}: Some dogs are not pets.\\
%
%\textbf{not-I}: It is not the case that some dogs are reptiles &
%\textbf{E}: No dogs are reptiles\\
%
%\textbf{not-E}: It is not the case that no dogs are pets &
%\textbf{I}: Some dogs are pets \\
%
%\textbf{not-O}: It is not the case that some dogs are not mammals &
%\textbf{A}: All dogs are mammals. 
%\end{tabu}

Another kind of statement that can be transformed into a universal statement is a conditional. A conditional is a statement of the form ``If \ldots then \ldots.'' They will become a big focus of our attention starting in Chapter \ref{chap:SL} when we begin introducing modern formal languages. They are not given special treatment in the Aristotelian tradition, however. Instead, where we can, we just treat them as categorical generalizations:

\begin{longtabu}{p{.5\linewidth}p{.5\linewidth}}
\underline{English} &
\underline{Logically Structured English} \\
\endhead 
If something is a cat, then it is a feline. &
All cats are feline.\\

If something is a dog, then it's not a cat. &
No dogs are cats. \\
\end{longtabu}

The word ``only'' is used in a couple of different constructions in English that can be represented as universal statements. ``Exclusive propositions'' are statements that say the subject excludes everything except what is in the predicate. For instance the sentence ``Only people over 21 may drink'' says that the class of people who may drink excludes everyone except those who are over 21. In English exclusive propositions are created using the words ``only,'' ``none but,'' or ``none except.'' These statements become A statements when translated into logically structured English. So ``Only people over 21 may drink'' becomes ``If you may drink, you  are over 21.'' It is important to see that in each case these words are used to introduce the predicate, not the subject. In the sentence ``Only people over 21 may drink,'' the term ``people over 21'' is actually the  predicate, and ``people who may drink'' is the subject. 

\begin{longtabu}{p{.5\linewidth}p{.5\linewidth}}
\underline{English} &
\underline{Logically Structured English} \\
\endhead 
Only people over 21 may drink. &
All people who drink are over 21.\\

No one, except those with a ticket, may enter the theater. &
All people who enter the theater have a ticket. \\

None but the strong survive. &
All people who survive are strong people. \\
\end{longtabu}


Sentences with ``The only'' are a little different than sentences that just have ``only'' in them. The sentence ``Humans are the only animals that talk on cell phones'' should be translated as ``All animals who talk on cell phones are humans.'' In this sentence, ``the only'' introduces the subject, rather than the predicate. The statement still asserts that the subject excludes everything except what is in the predicate, and we still represent them using mood A statements.

\begin{longtabu}{p{.5\linewidth}p{.5\linewidth}}
\underline{English} &
\underline{Logically Structured English} \\
\endhead 
Humans are the only animals who talk on cell phones. &
All animals who talk on cell phones are human.\\

Shrews are the only venomous mammal in North America. &
All venomous mammals in North America are shrews.\\

\end{longtabu}

Transforming sentences into Logically Structured English requires judgment and attention to the nuances of meaning in English. You must be able to recognize which of the transformations describe above needs to be applied and apply it correctly. One frequent mistake by people starting out is to overgeneralize. We saw at the start of the subsection on alternative universal forms that singular propositions can be turned into universal propositions by adding the phrase ``Things identical to \ldots'' Once you get in the habit of doing this, it becomes tempting to add the phrase ``things identical to \ldots'' to everything, even when it isn't necessary or doesn't make sense. The sentence ``Fido is a dog'' should become ``all things identical to Fido are dogs'' in logically structured English, because ``Fido'' is a singluar term referring to an individual dog. But with the sentence ``dogs are mammals,'' you do not need to add the phrase ``All things identical to\ldots'', because ``dogs'' is already a collective noun, not an individual.   

The same is true for the phrases we use to transform adjective and verb phrases into noun phrases. The sentence ``No cats bark'' has to be changed, because ``bark'' is a verb, so it becomes ``No cats are animals that bark'' in Logically Structured English. But the sentence ``No cats are reptiles'' already has a noun, ``reptiles,'' for a predicate, so you do not need to transform it into ``No cats are animals that are reptiles.'' The key is not only knowing when o use the transformations we describe, but knowing when not to use them. 


%%%%%%%%%% practice problems %%%%%%%%%%%%

\practiceproblems
\noindent\problempart Transform the following into logically structured English; identify it as A, E, I, or O; and provide the appropriate Venn diagram.


\begin{longtabu}{p{.1\linewidth}p{.9\linewidth}}
\textbf{Example}: & If you can't stand the heat, get out of the kitchen \\
\textbf{Answer}:  & All people who cannot stand the heat are people who should get out of the kitchen. \\
& Form: A\\
&
\noindent \begin{tikzpicture}
\def\firstcircle{(0,0) circle (.75cm)}
\def\secondcircle{(0:1cm) circle (.75cm)}
     \begin{scope}[shift={(4cm,0cm)}]
        \begin{scope}[even odd rule]% first circle without the second
            \clip \secondcircle (-1,-1) rectangle (1,1);
        \fill[gray] \firstcircle;
        \end{scope}
\draw \firstcircle node[outer sep=.66cm, above left] (s) {$S$};
\draw \secondcircle node[outer sep=.66cm, above right] (p) {$P$};
        \end{scope}

\end{tikzpicture}\\
& $S$: People who can't stand the heat\\
& $P$: People who should get out of the kitchen

\end{longtabu}

\begin{exercises}

\item If something is worth doing, it is worth doing well. \answer{\\
All things that are worth doing are things that are worth doing well. \\
Form: A \\
\begin{venns}
\MoodAStatementRed
\end{venns}

$S$: Things that are worth doing \\
$P$: Things that are worth doing well.
}

\item Cats are not herbivores.\answer{\\
No cats are herbivores. \\
Form: E \\
\begin{venns}
\MoodEStatementRed
\end{venns}

$S$: Cats\\
$P$: Herbivores
}

\item Some chimpanzees know sign language. \answer{\\
Some chimpanzees are things that know sign language. \\
Form: I \\
\begin{venns}
\MoodIStatementRed
\end{venns}
	
$S$: Chimpanzees \\
$P$: Things that know sign language.
}

\item Some dogs are not loyal. \answer{\\
Some dogs are not things that are loyal. \\
Form: O \\
\begin{venns}
\MoodOStatementRed
\end{venns}

$S$: Dogs \\
$P$: Things that are loyal.
}

\item Monotremes are the only egg-laying mammals.\answer{\\
All egg-laying mammals are monotremes. \\
Form: A \\
\begin{venns}
\MoodAStatementRed
\end{venns}

$S$: Egg-laying mammals \\
$P$: Monotremes\\
\vspace{11pt}
Remember that ``the only'' introduces the subject, not the predicate. 
}

\item Whenever a bell rings, an angel gets its wings. \answer{\\
All times that a bell rings are times that an angel gets its wings. \\
Form: A \\
\begin{venns}
\MoodAStatementRed
\end{venns}

$S$: Times that a bell rings \\
$P$: Times that an angel gets its wings\\
\vspace{11pt}
Remember that words like ``whenever'' and ``where ever'' are really specialized quantifiers for places and times, and need to be replaced by ``all places'' and ``all times.'' See page 71.
}


\item At least one person in this room is a liar. \answer{\\
Some people in this room are liars. \\
Form: I \\
\begin{venns}
\MoodIStatementRed
\end{venns}
	
$S$: People in this room \\
$P$: Liars\\
\vspace{11pt}
}


\item Only natural born citizens can be president of the United States.\answer{\\
All presidents of the United States are natural born citizens. \\
Form: A \\
\begin{venns}
\MoodAStatementRed
\end{venns}

$S$: Presidents of the United States\\
$P$: Natural born citizens\\
\vspace{11pt}
}

\item Gottlob Frege suffered from severe depression. \answer{\\
All people identical with Gottlob Frege are people who suffer from severe depression. \\
Form: A \\
\begin{venns}
\MoodAStatementRed
\end{venns}

$S$: People identical with Gottlob Frege\\
$P$: People who suffer from severe depression\\
}

\item ``Anyone who ever had a heart, wouldn't turn around and break it.'' --Lou Reed. \answer{\\
No things who ever had a heart are things that would turn around and break it. \\
Form: E \\

\begin{venns}
\MoodEStatementRed 
\end{venns}

$S$: Things who ever had a heart\\
$P$: Things that would turn around a break it\\
}

\end{exercises}

\noindent\problempart Transform the following into logically structured English; identify it as A, E, I, or O; and provide the appropriate Venn diagram.

\begin{exercises}
\item If a muffin has frosting, then it is a cupcake. \answer{\\
All muffins with frosting are cupcakes\\
Mood: A\\
\begin{venns}
\MoodAStatementRed 
\end{venns}

$S$: Muffins with frosting\\
$P$: Cupcakes\\
}

\item Some birds eat fish. \answer{\\
Some birds are things that eat fish \\
Mood: I\\
\begin{venns}
\MoodIStatementRed
\end{venns}\\
$S$: Birds\\
$P$: Things that eat fish\\
}

\item Dragons don't take kindly to strangers \answer{\\
No dragons are things that take kindly to strangers.\\
Mood: E\\
\begin{venns}
\MoodEStatementRed
\end{venns}\\
$S$:  Dragons\\
$P$:  Things that take kindly to strangers\\
}

\item Some logicians are not mentally ill \answer{\\
Some logicians are not people who are mentally ill\\
Mood: O\\
\begin{venns}
\MoodOStatementRed
\end{venns}\\
$S$:  Logicians\\
$P$:  People who are not mentally ill\\
}

\item There's no milk in the fridge \answer{\\
No things in the fridge are milk \\
Mood: E\\
\begin{venns}
\MoodEStatementRed
\end{venns}\\
$S$:  Things in the fridge\\
$P$:  Milk\\
}
 
\item Seahorses are the only fish species in which the male carries the babies. \answer{\\
All fish species where the male carries the babies are seahorses   \\
Mood: A\\
\begin{venns}
\MoodAStatementRed
\end{venns}\\
$S$:  Fish species where the male carries the babies\\
$P$:  Seahorses\\
}
 
\item Seahorses are animals that mate for life. \answer{\\
 All seahorses are animals that made for life  \\
Mood:  A\\
\begin{venns}
\MoodAStatementRed
\end{venns} \\
$S$:  Seahorses\\
$P$:  Animals that mate for life\\
}
 
\item Few dogs are fans of classical music. \answer{\\
Some dogs are fans of classical music   \\
Mood:  I\\
\begin{venns}
\MoodIStatementRed
\end{venns}\\
$S$:  Dogs\\
$P$:  Fans of classical music\\
}
 
\item Ruth Barcan Marcus was a member of the Yale faculty. \answer{\\
 All people identical to Ruth Barcan Marcus were members of the Yale faculty  \\
Mood: A \\
\begin{venns}
\MoodAStatementRed
\end{venns}\\
$S$:  People identical to Ruth Barcan Marcus \\
$P$:  Members of the Yale faculty  \\
}
 
\item Only zombies are brain eaters. \answer{\\
 All brain eaters are zombies  \\
Mood:  A \\
\begin{venns}
\MoodAStatementRed
\end{venns}\\
$S$:  Brain eaters\\
$P$:  Zombies\\
}
 
\end{exercises}

\noindent\problempart Transform the following into logically structured English; identify it as A, E, I, or O; and provide the appropriate Venn diagram. Some problems will require multiple transformations.

\begin{longtabu}{p{.1\linewidth}p{.9\linewidth}}
\textbf{Example}: & Bertrand Russell was married four times. \\
\textbf{Answer}:  & All people who are identical to Bertrand Russell are people who were married four times. \\
& Form: A\\
&
\noindent \begin{tikzpicture}
\def\firstcircle{(0,0) circle (.75cm)}
\def\secondcircle{(0:1cm) circle (.75cm)}
     \begin{scope}[shift={(4cm,0cm)}]
        \begin{scope}[even odd rule]% first circle without the second
            \clip \secondcircle (-1,-1) rectangle (1,1);
        \fill[gray] \firstcircle;
        \end{scope}
\draw \firstcircle node[outer sep=.66cm, above left] (s) {$S$};
\draw \secondcircle node[outer sep=.66cm, above right] (p) {$P$};
        \end{scope}
\end{tikzpicture}\\
& $S$: People who are identical to Bertrand Russell\\
& $P$: People who were married four times
\end{longtabu}

\begin{exercises}

\item Many logicians work in computer science. \answer{\\
Some logicians are people who work in computer science. \\
Form: I \\
\begin{venns}
\MoodIStatementRed
\end{venns}\\
$S$: Logicians\\
$P$: People that work in computer science\\
}


\item Ludwig Wittgenstein severed in the Austro-Hungarian Army in World War I.\answer{\\
All people identical to Ludwig Wittgenstein are people that served in the Austro-Hungarian Army in World War I \\
Form: A \\
\begin{venns}
\MoodAStatementRed
\end{venns}\\
$S$: People identical to Ludwig Wittgenstein\\
$P$: People that served int he Austro-Hungarian Army in WWI\\
}

\item Martians can be found nowhere on Earth.\answer{\\
No place on earth is a place where you can find Martians. \\
Form: E \\
\begin{venns}
\MoodEStatementRed
\end{venns}\\
$S$: Places on Earth\\
$P$: Places you can find Martians\\
}

\item One of our cats is not awake. \answer{\\
Some of our cats are not animals that are awake\\
Form: O \\
\begin{venns}
\MoodOStatementRed
\end{venns}\\
$S$: Our cats\\
$P$: Animals that are awake\\
}

\item Grover Cleveland was the only president to serve two nonconsecutive terms. \answer{\\
All presidents who served two nonconsecutive terms are people identical to Grover Cleveland. \\
Form: A \\
\begin{venns}
\MoodAStatementRed
\end{venns}\\
$S$: Presidents who served two nonconsecutive terms \\
$P$: People identical to Grover Cleveland\\
}

\item Grover Cleveland was not a Muppet \answer{\\
No people identical to Grover Cleveland are Muppets \\
Form: E \\
\begin{venns}
\MoodEStatementRed
\end{venns}\\
$S$: People identical to Grover Cleveland\\
$P$: Muppets\\
}



\item The band's only singer also plays guitar. \answer{\\
All people identical to the singer of the band are people who play guitar. \\
Form: A \\
\begin{venns}
\MoodAStatementRed
\end{venns}\\
$S$: People identical to the singer of the band\\
$P$: People who play guitar\\
}

\item If a dog has a collar, it is someone's pet.\answer{\\
All dogs that have a collar are animals that are someone's pet.\\
Form: A \\
\begin{venns}
\MoodAStatementRed
\end{venns}\\
$S$: Dogs that have a collar\\
$P$: Animals that are someone's pet. \\
}

\item Only the basketball players in the class were tall. \answer{\\
All tall people in the class were basketball players\\
Form: A \\
\begin{venns}
\MoodAStatementRed
\end{venns}\\
$S$: Tall people in the class\\
$P$: Basketball players\\
}

\item If you study, then you will not fail the test.\answer{\\
All things that study are things that will pass the test. \\
Form: E \\
\begin{venns}
\MoodEStatementRed
\end{venns}\\
$S$: People who study \\
$P$: People who fail the test\\
}

\end{exercises}


\noindent\problempart Transform the following into logically structured English; identify it as A, E, I, or O; and provide the appropriate Venn diagram. Some problems will require multiple transformations.

\begin{exercises}
\item People have walked on the moon at least once. 
\item Basketball players are tall.
\item Most senior citizens vote.
\item If a bird is a crow, then it is very intelligent. 
\item Whoever ate the last cookie is in trouble.
\item ``Euclid alone has looked on Beauty bare.'' --Edna St. Vincent Millay.
\item If something is a dog, then it is man's best friend.
\item More than a few students will fail the test.
\item Mercury is the only metal that is liquid at room temperature. 
\item Bertrand Russell was married four times.
\end{exercises}

%Something is a dog only if it's not a cat.
%Dogs are not cats. &
%No dogs are cats.\\

% *****************************************
% * Conversion, Obversion, and Contraposition   *
% *****************************************

\section{Conversion, Obversion, and Contraposition}
\label{sec:conv_obv_cont}
Now that we have shown the wide range of statements that can be represented in our four standard logical forms A, E, I, and O, it is time to begin constructing arguments with them. The arguments we are going to look at are sometimes called ``immediate inferences'' because they only have one premise. We are going to learn to identify some valid forms of these one-premise arguments by looking at ways you can transform a true sentence that maintain its truth value. For instance, ``No dogs are reptiles'' and ``No reptiles are dogs'' have the same truth value and basically mean the same thing. On the other hand if you change ``All dogs are mammals'' into ``All mammals are dogs'' you turn a true sentence into a false one. In this section we are going to look at three ways of transforming categorical statements---conversion, obversion, and contraposition---and use Venn diagrams to determine whether these transformations also lead to a change in truth value. From there we can identify valid argument forms. 

\subsection{Conversion}

\newglossaryentry{conversion}
{
name=conversion,
description={The process of changing a sentence by reversing the subject and predicate.}
}


The two examples in the last paragraph are examples of conversion. \textsc{\gls{conversion}} \label{defConversion} is the process of transforming a categorical statement by switching the subject and the predicate. When you convert a statement, it keeps its form---an A statement remains an A statement, an E statement remains an E statement---however it might change its truth value.  The Venn diagrams in Figure \ref{fig:conversion} illustrate this.

\begin{figure}
\begin{mdframed}[style=mytablebox]
\begin{tabu}{p{.5\linewidth}p{.5\linewidth}}

\underline{Original Statement}

&

\underline{Converse} \\

\begin{tikzpicture}
\def\firstcircle{(0,0) circle (.66cm)}
\def\secondcircle{(0:.88cm) circle (.66cm)}
     \begin{scope}[shift={(4cm,0cm)}]
        \begin{scope}[even odd rule]% first circle without the second
            \clip \secondcircle (-.66,-.66) rectangle (.66,.66);
        \fill[gray] \firstcircle;
        \end{scope}
        \draw \firstcircle node[outer sep=.4cm, above left] (s) {$S$};
        \draw \secondcircle node[outer sep=.4cm, above right] (p) {$P$};
    \end{scope} 
\end{tikzpicture}

&

\begin{tikzpicture}
\def\firstcircle{(0,0) circle (.66cm)}
\def\secondcircle{(0:.88cm) circle (.66cm)}
     \begin{scope}[shift={(4cm,0cm)}]
        \begin{scope}[even odd rule]
            \clip \firstcircle (-.66,-.66) rectangle (1.66,.88);
        \fill[gray] \secondcircle;
        \end{scope}
        \draw \firstcircle node[outer sep=.4cm, above left] (s) {$S$};
        \draw \secondcircle node[outer sep=.4cm, above right] (p) {$P$};
    \end{scope} 
\end{tikzpicture}

\\

\textbf{A:} All $S$ are $P$

&

\textbf{A:} All $P$ are $S$

\\

\begin{tikzpicture}
\def\firstcircle{(0,0) circle (.66cm)}
\def\secondcircle{(0:.88cm) circle (.66cm)}
\draw \firstcircle node[outer sep=.4cm, above left] (s) {$S$};
\draw \secondcircle node[outer sep=.4cm, above right] (p) {$P$};
    \begin{scope}
      \clip \firstcircle;
      \fill[gray] \secondcircle;
    \end{scope}
\end{tikzpicture}

&

\begin{tikzpicture}
\def\firstcircle{(0,0) circle (.66cm)}
\def\secondcircle{(0:.88cm) circle (.66cm)}
\draw \firstcircle node[outer sep=.4cm, above left] (s) {$S$};
\draw \secondcircle node[outer sep=.4cm, above right] (p) {$P$};
    \begin{scope}
      \clip \firstcircle;
      \fill[gray] \secondcircle;
    \end{scope}
\end{tikzpicture}

\\

\textbf{E}: No $S$ are $P$

&
\textbf{E}: No $P$ are $S$


\\

\begin{tikzpicture}
\def\firstcircle{(0,0) circle (.66cm)}
\def\secondcircle{(0:.88cm) circle (.66cm)}
\draw \firstcircle node[outer sep=.4cm, above left] (s) {$S$};
\node[outer sep=.22cm, right] (x) {x};
\draw \secondcircle node[outer sep=.4cm, above right] (p) {$P$};
\end{tikzpicture}

&

\begin{tikzpicture}
\def\firstcircle{(0,0) circle (.66cm)}
\def\secondcircle{(0:.88cm) circle (.66cm)}
\draw \firstcircle node[outer sep=.4cm, above left] (s) {$S$};
\node[outer sep=.22cm, right] (x) {x};
\draw \secondcircle node[outer sep=.4cm, above right] (p) {$P$};
\end{tikzpicture}

\\

\textbf{I}: Some $S$ are $P$

&

\textbf{I}: Some $P$ are $S$

\\

\begin{tikzpicture}
\def\firstcircle{(0,0) circle (.66cm)}
\def\secondcircle{(0:.88cm) circle (.66cm)}
\draw \firstcircle node[outer sep=.4cm, above left] (s) {$S$};
\node[outer sep=.3cm] (x) {x};
\draw \secondcircle node[outer sep=.4cm, above right] (p) {$P$};
\end{tikzpicture}

&


\begin{tikzpicture}
\def\firstcircle{(0,0) circle (.66cm)}
\def\secondcircle{(0:.88cm) circle (.66cm)}
\draw \firstcircle node[outer sep=.4cm, above left] (s) {$S$};
\node[outer sep=.66cm, right] (x) {x};
\draw \secondcircle node[outer sep=.4cm, above right] (p) {$P$};
\end{tikzpicture}


\\

\textbf{O}: Some $S$ are not $P$

&

\textbf{O}: Some $P$ are not $S$

\end{tabu}
\end{mdframed}
\caption{Conversions of the Four Basic Forms}
\label{fig:conversion} 	
\end{figure}

As you can see, the Venn diagram for the converse of an E statement is exactly the same as the original E statement, and likewise for I statements. This means that the two statements are logically equivalent. Recall that two statements are logically equivalent if they always have the same truth value. (See page \ref{def:logical_equivalence}). In this case, that means that if an E statement is true, then its converse is also true, and if an E statement is false, then its converse is also false. For instance, ``No dogs are reptiles'' is true, and so is ``No reptiles are dogs.'' On the other hand ``No dogs are mammals'' is false, and so is ``No mammals are dogs.''

Likewise, if an I statement is true, its converse is true, and if an I statement is false, than its converse is false. ``Some dogs are pets'' is true, and so is ``Some pets are dogs.'' On the other hand ``Some dogs can fly'' is false and so is ``Some flying things are dogs.''

The converses of A and O statements are not so illuminating. As you can see from the Venn diagrams, these statements are not identical to their converses. They also don't contradict their converses. If we know that an A or O statement is true, we still don't know anything about their converses. We say their truth value is undetermined. 

Because E and I statements are logically equivalent to their converses, we can use them to construct valid arguments. Recall from Chapter 2 (page \pageref{def:valid}) that an argument is valid if it is impossible for its conclusion to be false whenever its premises are true. Because E and I are logically equivalent to their converses, the two argument forms in Figure \ref{fig:conversion_arguments} are valid. 


\begin{figure}
\begin{mdframed}[style=mytablebox]
\begin{tabu}{p{.5\linewidth}p{.5\linewidth}}

\begin{earg}
\item[P.] No $S$ are $P$.
\vspace{-.5em}
\item [] \rule{0.6\linewidth}{1pt} 
\item[C.] No $P$ are $S$.
\end{earg} 

&

\begin{earg}
\item[P.] Some $S$ are $P$.
\vspace{-.5em}
\item [] \rule{0.6\linewidth}{1pt} 
\item[C.] Some  $P$ are $S$.
\end{earg} 
\end{tabu}
\end{mdframed}
\caption{Valid Arguments by Conversion} \label{fig:conversion_arguments}
\end{figure}

Notice that these are argument forms, with variables in the place of the key terms. This means that these arguments will be valid no matter what; $S$ and $P$ could be people, or squirrels, or the Gross Domestic Product of industrialized nations, or anything, and the arguments are still valid. While these particular argument forms may seem trivial and obvious, we are beginning to see some of the power of formal logic here. We have uncovered a very general truth about the nature of validity with these two argument forms. 

The truth value of the converses of A and O statements, on the other hand, are undetermined by the truth value of the original statements. This means we cannot construct valid arguments from them. Imagine you have an argument with an A or O statement as its premise and the converse of that statement as the conclusion. Even if the premise is true, we know nothing about the truth of the conclusion. So there are no valid argument forms to be found here. 

\subsection{Obversion}

\newglossaryentry{complement}
{
name=complement,
description={The class of everything that is not in a given class.}
}


Obversion is a more complex process. To understand what an obverse is, we first need to define the complement of a class. The \textsc{\gls{complement}} \label{def:Complement} of a class is everything that is not in the class. So the complement of the class of dogs is everything that is not a dog, including not just cats, but battleships, pop songs, and black holes. In English we can easily create a name for the complement of any class using the prefix ``non-''. So the complement of the class of dogs is the class of non-dogs. We will use complements in defining both obversion and contraposition. 

\newglossaryentry{obversion}
{
name=obversion,
description={The process of transforming a categorical statement by changing its quality and replacing the predicate with its complement.}
}


The \textsc{\gls{obversion}} \label{def:Obversion} of a categorical proposition is a new proposition created by changing the quality of the original proposition and switching its predicate to its complement. Obversion is thus a two step process. Take, again, the proposition ``All dogs are mammals.'' For step 1, we change its quality, in this case going from affirmative to negative. That gives us ``No dogs are mammals.'' For step 2, we take the complement of the predicate. The predicate in this case is ``mammals'' so the complement is ``non-mammals.'' That gives us the obverse ``No dogs are non-mammals.''

We can map this process out using Venn diagrams. Let's start with an A statement.

\begin{center}
\begin{tikzpicture}
\def\firstcircle{(0,0) circle (1cm)}
\def\secondcircle{(0:1.33cm) circle (1cm)}
     \begin{scope}[shift={(4cm,0cm)}]
        \begin{scope}[even odd rule]% first circle without the second
            \clip \secondcircle (-1,-1) rectangle (1,1);
        \fill[gray] \firstcircle;
        \end{scope}
        \draw \firstcircle node[outer sep=.75cm, above left] (s) {$S$};
        \draw \secondcircle node[outer sep=.75cm, above right] (p) {$P$};
    \end{scope}
\end{tikzpicture} \\
\textbf{A}: All $S$ are $P$.
\end{center}

Changing the quality turns it into an E statement.

\begin{center}
\begin{tikzpicture}
\def\firstcircle{(0,0) circle (1cm)}
\def\secondcircle{(0:1.33cm) circle (1cm)}
\draw \firstcircle node[outer sep=.75cm, above left] (s) {$S$};
\draw \secondcircle node[outer sep=.75cm, above right] (p) {$P$};
    \begin{scope}
      \clip \firstcircle;
      \fill[gray] \secondcircle;
    \end{scope}
\end{tikzpicture} \\
\textbf{E}: No $S$ are $P$.
\end{center}

Now what happens when we take the complement of $P$? That means we will shade in all the parts of S that are non-$P$, which puts us back where we started. We still have an E statement, but it is now equivalent to the A statement. 


\begin{figure}[H]
\begin{center}
\begin{tikzpicture}
\def\firstcircle{(0,0) circle (1cm)}
\def\secondcircle{(0:1.33cm) circle (1cm)}
     \begin{scope}%[shift={(4cm,0cm)}]
        \begin{scope}[even odd rule]% first circle without the second
            \clip \secondcircle (-1,-1) rectangle (1,1);
        \fill[gray] \firstcircle;
        \end{scope}
        \draw \firstcircle node[outer sep=.75cm, above left] (s) {$S$};
        \draw \secondcircle node[outer sep=.75cm, above right] (p) {$P$};
    \end{scope}
\end{tikzpicture} \\
\captionsetup{singlelinecheck=on}
\caption*{\textbf{E}: No $S$ are non-$P$.}
\end{center}
\end{figure}

The final statement is logically equivalent to the original A statement. It has the same form as an E statement, but because we have changed the predicate, it is not logically equivalent to an A statement. As you can see from Figure \ref{fig:obversion} this is true for all four forms of categorical statement. 
This in turn gives us four valid argument forms, which are shown in Figure \ref{fig:obversion_arguments}

\begin{figure}
\begin{mdframed}[style=mytablebox]
\begin{tabu}{p{.5\linewidth}p{.5\linewidth}}

\underline{Original Statement}

&

\underline{Obverse} \\

\begin{tikzpicture}
\def\firstcircle{(0,0) circle (1cm)}
\def\secondcircle{(0:1.33cm) circle (1cm)}
     \begin{scope}%[shift={(4cm,0cm)}]
        \begin{scope}[even odd rule]% first circle without the second
            \clip \secondcircle (-1,-1) rectangle (1,1);
        \fill[gray] \firstcircle;
        \end{scope}
        \draw \firstcircle node[outer sep=.75cm, above left] (s) {$S$};
        \draw \secondcircle node[outer sep=.75cm, above right] (p) {$P$};
    \end{scope} 
\end{tikzpicture}

&

\begin{tikzpicture}
\def\firstcircle{(0,0) circle (1cm)}
\def\secondcircle{(0:1.33cm) circle (1cm)}
     \begin{scope}[shift={(4cm,0cm)}]
        \begin{scope}[even odd rule]% first circle without the second
            \clip \secondcircle (-1,-1) rectangle (1,1);
        \fill[gray] \firstcircle;
        \end{scope}
        \draw \firstcircle node[outer sep=.75cm, above left] (s) {$S$};
        \draw \secondcircle node[outer sep=.75cm, above right] (p) {$P$};
    \end{scope} 
\end{tikzpicture}
\\

\textbf{A:} All $S$ are $P$

&

\textbf{E:} No $S$ are non-$P$

\\

\begin{tikzpicture}
\def\firstcircle{(0,0) circle (1cm)}
\def\secondcircle{(0:1.33cm) circle (1cm)}
\draw \firstcircle node[outer sep=.75cm, above left] (s) {$S$};
\draw \secondcircle node[outer sep=.75cm, above right] (p) {$P$};
    \begin{scope}
      \clip \firstcircle;
      \fill[gray] \secondcircle;
    \end{scope}
\end{tikzpicture}

&

\begin{tikzpicture}
\def\firstcircle{(0,0) circle (1cm)}
\def\secondcircle{(0:1.33cm) circle (1cm)}
\draw \firstcircle node[outer sep=.75cm, above left] (s) {$S$};
\draw \secondcircle node[outer sep=.75cm, above right] (p) {$P$};
    \begin{scope}
      \clip \firstcircle;
      \fill[gray] \secondcircle;
    \end{scope}
\end{tikzpicture}

\\

\textbf{E}: No $S$ are $P$

&
\textbf{A}: All $S$ are non-$P$


\\

\begin{tikzpicture}
\def\firstcircle{(0,0) circle (1cm)}
\def\secondcircle{(0:1.33cm) circle (1cm)}
\draw \firstcircle node[outer sep=.75cm, above left] (S) {$S$};
\node[xshift=.66cm] (x) {X};
\draw \secondcircle node[outer sep=.75cm, above right] (P) {$P$};
\end{tikzpicture}

&

\begin{tikzpicture}
\def\firstcircle{(0,0) circle (1cm)}
\def\secondcircle{(0:1.33cm) circle (1cm)}
\draw \firstcircle node[outer sep=.75cm, above left] (s) {$S$};
\node[xshift=.66cm] (x) {X};
\draw \secondcircle node[outer sep=.75cm, above right] (p) {$P$};
\end{tikzpicture}

\\

\textbf{I}: Some $S$ are $P$

&

\textbf{O}: Some $S$ are not non-$P$

\\

\begin{tikzpicture}
\def\firstcircle{(0,0) circle (1cm)}
\def\secondcircle{(0:1.33cm) circle (1cm)}
\draw \firstcircle node[outer sep=.75cm, above left] (s) {$S$};
\node (x) {X};
\draw \secondcircle node[outer sep=.75cm, above right] (p) {$P$};
\end{tikzpicture}

&


\begin{tikzpicture}
\def\firstcircle{(0,0) circle (1cm)}
\def\secondcircle{(0:1.33cm) circle (1cm)}
\draw \firstcircle node[outer sep=.75cm, above left] (s) {$S$};
\node (x) {X};
\draw \secondcircle node[outer sep=.75cm, above right] (p) {$P$};
\end{tikzpicture}


\\

\textbf{O}: Some $S$ are not $P$

&

\textbf{I}: Some $S$ are non-$P$

\end{tabu}
\end{mdframed}
\caption{Obversions of the Four Basic Forms}
\label{fig:obversion} 	
\end{figure}



\begin{figure} %new fig
\begin{mdframed}[style=mytablebox]
\begin{tabu}{p{.5\linewidth}p{.5\linewidth}}

\begin{earg*}
\item All $S$ are $P.$
\itemc No $S$ are non-$P$.
\end{earg*} 

&

\begin{earg*}
\item No $S$ are $P$.
\itemc All $S$ are non-$P$.
\end{earg*} 

\\

\begin{earg*}
\item Some $S$ are $P$.
\itemc Some $S$ are not non-$P$.
\end{earg*} 

&

\begin{earg*}
\item Some $S$ are not $P$.
\itemc Some $S$ are non-$P$.
\end{earg*}
 
\end{tabu}
\end{mdframed}
\caption{Valid argument forms by obversion} \label{fig:obversion_arguments}
\end{figure}

One further note on complements. We don't just use complements to describe sentences that come out of obversion and contraposition. We can also perform these operations on statements that already have complements in them. Consider the sentence ``Some $S$ are non-$P$.'' This is its Venn diagram. 

\begin{figure}[H]
\begin{center}
\begin{tikzpicture}
\def\firstcircle{(0,0) circle (1cm)}
\def\secondcircle{(0:1.33cm) circle (1cm)}
\draw \firstcircle node[outer sep=.75cm, above left] (s) {$S$};
\node (x) {X};
\draw \secondcircle node[outer sep=.75cm, above right] (p) {$P$};
\end{tikzpicture}
\captionsetup{singlelinecheck=on}
\caption*{Some $S$ are non-$P$}
\end{center} 
\end{figure}

How would we take the obverse of this statement? Step 1 is to change the quality, making it ``Some $S$ are not non-$P$.'' Now how do we take the complement of the predicate? We could write ``non-non-$P$,'' but if we think about it for a second, we'd realize that this is the same thing as $P$. So we can just write ``Some $S$ is not $P$.'' This is logically equivalent to the original statement, which is what we wanted.   

Taking the converse of ``Some $S$ are non-$P$'' also takes a moment of thought. We are supposed to reverse subject and predicate. But does that mean that the ``non-'' moves to the subject position along with the ``$P$''? Or does the ``non-'' now attach to the $S$? We saw that E and I statements kept their truth value after conversion, and we want this to still be true when the statements start out referring to the complement of some class. This means that the ``non-'' has to travel with the predicate, because ``Some $S$ are non-$P$'' will always have the same truth value as ``Some non-$P$ are $S$.'' Another way of thinking about this is that the ``non-'' is part of the name of the class that forms the predicate of ``Some $S$ are non-$P$.'' The statement is making a claim about a class, and that class happens to be defined as the complement of another class. So, the bottom line is when you take the converse of a statement where one of the terms is a complement, move the ``non-'' with that term. 

\subsection{Contraposition}

\newglossaryentry{contraposition}
{
name=contraposition,
description={The process of transforming a categorical statement by reversing subject and predicate and replacing them with their complements.}
}


\textsc{\gls{contraposition}} is a two-step process, like obversion, but it doesn't always lead to results that are logically equivalent to the original sentence. The contrapositive of a categorical sentence is the sentence that results from reversing subject and predicate and then replacing them with their complements. Thus ``All $S$ are $P$'' becomes ``All non-$P$ are non-$S$.'' 

Figure \ref{fig:contraposition} shows the corresponding Venn diagrams. In this case, the shading around the outside of the two circles in the contraposed form of E is meant to indicate that nothing can lie outside the two circles. Everything must be $S$ or $P$ or both. Like conversion, applying contraposition to two of the forms gives us statements that are logically equivalent to the original. This time, though, it is forms A and O that come through the process without changing their truth value. 

\begin{figure}
\begin{mdframed}[style=mytablebox]
\begin{tabu}{p{.5\linewidth}p{.5\linewidth}}

\underline{Original Statement}

&

\underline{Contrapositive} \\

\begin{tikzpicture}
\def\firstcircle{(0,0) circle (.66cm)}
\def\secondcircle{(0:.88cm) circle (.66cm)}
     \begin{scope}[shift={(4cm,0cm)}]
        \begin{scope}[even odd rule]% first circle without the second
            \clip \secondcircle (-.66,-.66) rectangle (.66,.66);
        \fill[gray] \firstcircle;
        \end{scope}
        \draw \firstcircle node[outer sep=.4cm, above left] (s) {$S$};
        \draw \secondcircle node[outer sep=.4cm, above right] (p) {$P$};
    \end{scope} 
\end{tikzpicture}

&

\begin{tikzpicture}
\def\firstcircle{(0,0) circle (.66cm)}
\def\secondcircle{(0:.88cm) circle (.66cm)}
     \begin{scope}[shift={(4cm,0cm)}]
        \begin{scope}[even odd rule]% first circle without the second
            \clip \secondcircle (-.66,-.66) rectangle (.66,.66);
        \fill[gray] \firstcircle;
        \end{scope}
        \draw \firstcircle node[outer sep=.4cm, above left] (s) {$S$};
        \draw \secondcircle node[outer sep=.4cm, above right] (p) {$P$};
    \end{scope} 
\end{tikzpicture}
\\

\textbf{A:} All $S$ are $P$

&

\textbf{A:} All non-$P$ are non-$S$

\\

\begin{tikzpicture}
\def\firstcircle{(0,0) circle (.66cm)}
\def\secondcircle{(0:.88cm) circle (.66cm)}
\draw \firstcircle node[outer sep=.4cm, above left] (s) {$S$};
\draw \secondcircle node[outer sep=.4cm, above right] (p) {$P$};
    \begin{scope}
      \clip \firstcircle;
      \fill[gray] \secondcircle;
    \end{scope}
\end{tikzpicture}

&

\begin{tikzpicture}
\def\firstcircle{(0,0) circle (.66cm)}
\def\secondcircle{(0:.88cm) circle (.66cm)}

\fill[gray] (-.88,-.88) rectangle (1.88,.88);
\fill[light-gray] 
    \firstcircle node[outer sep=.4cm, above left] (s) {$S$}
    \secondcircle node[outer sep=.4cm, above right] (p) {$P$};
\draw
    \firstcircle 
    \secondcircle ;
\end{tikzpicture}
\\

\textbf{E}: No $S$ are $P$

&
\textbf{E}: No non-$P$ are non-$S$


\\

\begin{tikzpicture}
\def\firstcircle{(0,0) circle (.66cm)}
\def\secondcircle{(0:.88cm) circle (.66cm)}
\draw \firstcircle node[outer sep=.4cm, above left] (s) {$S$};
\node[outer sep=.22cm, right] (x) {x};
\draw \secondcircle node[outer sep=.4cm, above right] (p) {$P$};
\end{tikzpicture}

&

\begin{tikzpicture}
\def\firstcircle{(0,0) circle (.66cm)}
\def\secondcircle{(0:.88cm) circle (.66cm)}
\draw \firstcircle node[outer sep=.4cm, above left] (s) {$S$};
\node[outer sep=2cm, right] (x) {x};
\draw \secondcircle node[outer sep=.4cm, above right] (p) {$P$};
\end{tikzpicture}

\\

\textbf{I}: Some $S$ are $P$

&

\textbf{I}: Some non-$P$ are non-$S$

\\

\begin{tikzpicture}
\def\firstcircle{(0,0) circle (.66cm)}
\def\secondcircle{(0:.88cm) circle (.66cm)}
\draw \firstcircle node[outer sep=.4cm, above left] (s) {$S$};
\node[outer sep=.3cm] (x) {x};
\draw \secondcircle node[outer sep=.4cm, above right] (p) {$P$};
\end{tikzpicture}

&


\begin{tikzpicture}
\def\firstcircle{(0,0) circle (.66cm)}
\def\secondcircle{(0:.88cm) circle (.66cm)}
\draw \firstcircle node[outer sep=.4cm, above left] (s) {$S$};
\node[outer sep=.3cm] (x) {x};
\draw \secondcircle node[outer sep=.4cm, above right] (p) {$P$};
\end{tikzpicture}


\\

\textbf{O}: Some $S$ are not $P$

&

\textbf{O}: Some non-$P$ are not non-$S$

\end{tabu}
\end{mdframed}
\caption{Contrapositions the Four Basic Forms}
\label{fig:contraposition} 	
\end{figure}

This then gives us two valid argument forms, shown in Figure \ref{fig:argument_contraposition}. If you have an argument with an A or O statement as its premise and the contraposition of that statement as the conclusion, you know it must be valid. Whenever the premise is true, the conclusion must be true, because the two statements are logically equivalent. On the other hand, if you had an E or an I statement as the premise, the truth of the conclusion is undetermined, so these arguments would not be valid. 


\begin{figure}
\begin{mdframed}[style=mytablebox]
\begin{tabu}{p{.5\linewidth}p{.5\linewidth}}

\begin{earg*}
\item All $S$ are $P$.
\itemc All non-$P$ are non-$S$.
\end{earg*} 


&


\begin{earg*}
\item Some $S$ are not $P$
\itemc Some non-$P$ are not non-$S$.
\end{earg*}

\end{tabu}
\end{mdframed}
\caption{Valid argument forms from contraposition} \label{fig:argument_contraposition}
\end{figure}

\subsection{Evaluating Short Arguments}
\label{sec:ESA}
So far we have seen eight valid forms of argument with one premise: two arguments that are valid by conversion, four that are valid by obversion, and two that are valid by contraposition. As we said, short arguments like these are sometimes called ``immediate inferences,'' because your brain just flits automatically from the truth of the premises to the truth of the conclusion. Now that we have identified these valid forms of inference, we can use this knowledge to see whether some of the arguments we encounter in ordinary language are valid. We can now tell in a few cases if our brain is right to flit so seamlessly from the premise to the conclusion.

In the real world, the inferences we make are messy and hard to classify. Much of the complexity of this issue is tackled in the parts of the complete version of this text that cover critical thinking. \label{ver_var} \nix{Part \ref{part:CT_and_informal_logic} of this text, on critical thinking.} Right now we are just going to deal with a limited subset of inferences: immediate inferences that might be based on conversion, obversion, or contraposition. Let's start start with the uncontroversial premise ``All dogs are mammals.'' Can we infer from this that all non-mammals are non-dogs? In canonical form, the argument would look like this. 

%\pagebreak[4] %Note the hard page break. Delete if things start to drift.

\begin{earg*}
\item All dogs are mammals
\itemc[.3] All non-mammals are non-dogs.
\end{earg*} 

Evaluating an immediate inference like this is a four step process. First, identify the subject and predicate classes. Second, draw the Venn diagram for the premise. Third, see if the Venn diagram shows that the conclusion must be true. If it must be, then the argument is valid. Finally, if the argument is valid, identify the process that makes it valid. (You can skip this step if the argument is invalid.)

For the argument above, the result of the first two steps would look like this:   

\begin{figure}[H]
\begin{center}
\begin{tikzpicture}
\def\firstcircle{(0,0) circle (.66cm)}
\def\secondcircle{(0:.88cm) circle (.66cm)}
     \begin{scope}[shift={(4cm,0cm)}]
        \begin{scope}[even odd rule]% first circle without the second
            \clip \secondcircle (-.66,-.66) rectangle (.66,.66);
        \fill[gray] \firstcircle;
        \end{scope}
        \draw \firstcircle node[outer sep=.4cm, above left] (s) {$S$};
        \draw \secondcircle node[outer sep=.4cm, above right] (p) {$P$};
    \end{scope} 
\end{tikzpicture}
\end{center}
\captionsetup{singlelinecheck=on}
\caption*{$S$: Dogs \\ $P$: Mammals}
\end{figure}

The Venn diagram for the premise shades out the possibility that there are dogs that aren't mammals. For step three, we ask, does this mean the conclusion must be true?  In this case, it does. The same shading implies that everything that is not a mammal must also not be a dog. In fact, the Venn diagram for the premise and the Venn diagram for the conclusion are the same. So the argument is valid. This means that we must go on to step four and identify the process that makes it valid. In this case, the conclusion is created by reversing subject and predicate and taking their complements, which means that this is a valid argument by contraposition.
   
Now, remember what it means for an argument to be valid. \label{valid_definition_reinforcement} As we said on page \pageref{def:valid}, an argument is valid if it is impossible for the premises to be true and the conclusion false. This means that we can have a valid argument with false premises, so long as it is the case that \emph{if} the premises were true, the conclusion would have to be true. So if the argument above is valid, then so is this one:

\begin{earg*}
\item All dogs are reptiles.
\itemc[.3] All non-reptiles are non-dogs.
\end{earg*}

The premise is now false: all dogs are not reptiles. However, \emph{if} all dogs were reptiles, then it would also have to be true that all non-reptiles are non-dogs. The Venn diagram works the same way.

\begin{figure}[H]
\begin{center}
\begin{tikzpicture}
\def\firstcircle{(0,0) circle (.66cm)}
\def\secondcircle{(0:.88cm) circle (.66cm)}
     \begin{scope}[shift={(4cm,0cm)}]
        \begin{scope}[even odd rule]% first circle without the second
            \clip \secondcircle (-.66,-.66) rectangle (.66,.66);
        \fill[gray] \firstcircle;
        \end{scope}
        \draw \firstcircle node[outer sep=.4cm, above left] (s) {$S$};
        \draw \secondcircle node[outer sep=.4cm, above right] (p) {$P$};
    \end{scope} 
\end{tikzpicture}
\end{center}
\captionsetup{singlelinecheck=on}
\caption*{$S$: Dogs \\ $P$: Reptiles}
\end{figure}

The Venn diagram for the premise still matches the Venn diagram for the conclusion. Only the labels have changed. The fact that this argument form remains true even with a false premise is just a variation on a theme we saw in 
Figure \ref{fig:valid_sound} when we saw a valid argument (with false premises) for the conclusion ``Socrates is a carrot.'' So arguments by transposition, just like any argument, can be valid even if they have false premises. The same is true for arguments by conversion and obversion. 

Arguments like these can also be invalid, even if they have true premises and a true conclusion. Remember that A statements are not logically equivalent to their converse. So this is an invalid argument with a true premise and a false conclusion:

\begin{earg*}
\item All dogs are mammals.
\itemc[.3] All mammals are dogs. 
\end{earg*}

Our Venn diagram test shows that this is invalid. Steps one and two give us this for the premise:
\begin{figure}[H]
\begin{center}
\begin{tikzpicture}
\def\firstcircle{(0,0) circle (.66cm)}
\def\secondcircle{(0:.88cm) circle (.66cm)}
     \begin{scope}[shift={(4cm,0cm)}]
        \begin{scope}[even odd rule]% first circle without the second
            \clip \secondcircle (-.66,-.66) rectangle (.66,.66);
        \fill[gray] \firstcircle;
        \end{scope}
        \draw \firstcircle node[outer sep=.4cm, above left] (s) {$S$};
        \draw \secondcircle node[outer sep=.4cm, above right] (p) {$P$};
    \end{scope} 
\end{tikzpicture}
\end{center}
\captionsetup{singlelinecheck=on}
\caption*{$S$: Dogs \\ $P$: Mammals}
\end{figure}

But this is the Venn diagram for the conclusion:

\begin{figure}[H]
\begin{center}
\begin{tikzpicture}
\def\firstcircle{(0,0) circle (.66cm)}
\def\secondcircle{(0:.88cm) circle (.66cm)}
     \begin{scope}[shift={(4cm,0cm)}]
        \begin{scope}[even odd rule]% first circle without the second
            \clip \firstcircle (-.66,-.66) rectangle (1.66,.88);
        \fill[gray] \secondcircle;
        \end{scope}
        \draw \firstcircle node[outer sep=.4cm, above left] (s) {$S$};
        \draw \secondcircle node[outer sep=.4cm, above right] (p) {$P$};
    \end{scope} 
\end{tikzpicture}
\end{center}
\captionsetup{singlelinecheck=on}
\caption*{$S$: Dogs \\ $P$: Mammals}
\end{figure}

This is an argument by conversion on an mood-A statement, which is invalid. The argument remains invalid, even if we substitute in a predicate where the conclusion happens to be true. For instance this argument is invalid.

\begin{earg*}
\item  All dogs are \textit{Canis familiaris}.
\itemc[.4] All \textit{Canis familiaris} are dogs. 
\end{earg*}

The Venn diagrams for the premise and conclusion of this argument will be just like the ones for the previous argument, just with different labels. So even though the argument has a true premise and a true conclusion, it is still invalid, because it is possible for an argument of this form to have a true premise and a false conclusion. This is an unreliable argument form that just happened, in this instance, not to lead to a false conclusion. This again is just a variation on a theme we saw in Chapter \ref{chap:basicevaluation}, in Figure \ref{fig:invalid_paris}, when we saw an invalid argument for the conclusion that Paris was in France.


%%%%%%%%%%%%%%%%%%%%% practice problems %%%%%%%%%%%%%

\practiceproblems
\problempart The first two columns in the table below give you a statement and a truth value for that statement. The next column gives an operation that can be performed on the statement in the first column, and the final two columns give the new statement and its truth value. 

The first row is completed, as an example, but after that there are blanks. In problems 1--5 you must fill in the new statement and its truth value, and in problems 6--10 you must fill in the operation and the final truth value. If the truth value of the resulting statement cannot be determined from the original one, write a ``?'' for ``undetermined.'' You can check your work with Venn diagrams, or by identifying the logical form of the original statement and seeing if it is one where the named operation changes the truth value. 

\begin{longtabu}{X[1,l,m]X[15,l,m]X[2,l,m]X[6,l,m]X[16,l,m]X[1,l,m]}
%{p{.01\linewidth}p{.2\linewidth}p{.15\linewidth}p{.15\linewidth}p{.2\linewidth}p{.1\linewidth}}
%{llllll} 
 & \underline{Given statement} & \underline{T/F} & \underline{Operation} & \underline{New Statement} & \underline{T/F} \\
  \rowcolor{light-gray}
 Ex. & All $S$ are $P$ & F & Conv. & All $P$ are $S$ & ? \\ 
 \endhead
 
1. & Some $S$ are $P$  & F  & Obv. & \answerblank{Some $S$ are not non-$P$}{\rule[-5pt]{2.5cm}{.4pt}}& \answerblank{F}{\rule[-5pt]{.5cm}{.4pt}}\\



2.& Some non-$S$ are $P$  & F& Conv. & \answerblank{Some $P$ are non-$S$}{\rule[-5pt]{2.5cm}{.4pt}}&\answerblank{F}{\rule[-5pt]{.5cm}{.4pt}}\\


3. & All $S$ are $P$  & F & Contrap. & \answerblank{All non-$P$ are non-$S$}{\rule[-5pt]{2.5cm}{.4pt}}& \answerblank{F}{\rule[-5pt]{.5cm}{.4pt}}\\

4. & Some $S$ are $P$  & F & Contrap. & \answerblank{Some non-$P$ are non-$S$}{\rule[-5pt]{2.5cm}{.4pt}}& \answerblank{?}{\rule[-5pt]{.5cm}{.4pt}}\\

5. & Some $S$ are non-$P$ & T & Obv. & \answerblank{Some $S$ are not $P$}{\rule[-5pt]{2.5cm}{.4pt}}& \answerblank{T}{\rule[-5pt]{.5cm}{.4pt}}\\

6.& All $S$ are non-$P$  & T & \answerblank{Contrap.}{\rule[-5pt]{1.5cm}{.4pt}}& All $P$ are non-$S$ & \answerblank{T}{\rule[-5pt]{.5cm}{.4pt}}\\

7.& Some non-$S$ are not P & T & \answerblank{Conv.}{\rule[-5pt]{1.5cm}{.4pt}}& Some $P$ are not non-$S$ & \answerblank{?}{\rule[-5pt]{.5cm}{.4pt}}\\

8. & Some $S$ are not P & F  & \answerblank{Conv.}{\rule[-5pt]{1.5cm}{.4pt}}& Some $P$ are not $S$ & \answerblank{?}{\rule[-5pt]{.5cm}{.4pt}}\\

9.& All non-$S$ are $P$ & T & \answerblank{Obv.}{\rule[-5pt]{1.5cm}{.4pt}}& No non-$S$ are non-$P$ & \answerblank{T}{\rule[-5pt]{.5cm}{.4pt}}\\

10. & No non-$S$ are non-$P$ & T & \answerblank{Obv.}{\rule[-5pt]{1.5cm}{.4pt}} & All non-$S$ are non-non-$P$ & \answerblank{T}{\rule[-5pt]{.5cm}{.4pt}}\\ 
  
\end{longtabu}

\problempart See the instructions for Part A.

\begin{longtabu}{X[1,l,m]X[15,l,m]X[2,l,m]X[6,l,m]X[16,l,m]X[1,l,m]}
%{p{.01\linewidth}p{.2\linewidth}p{.15\linewidth}p{.15\linewidth}p{.2\linewidth}p{.1\linewidth}}
%{llllll} 
 & \underline{Given statement} & \underline{T/F} & \underline{Operation} & \underline{New Statement} & \underline{T/F} \\

  

1. & All $S$ are $P$& T & Obv. & \nix{ } \rule[-5pt]{2.5cm}{.4pt}  & \nix{ } \rule[-5pt]{.5cm}{.4pt} \\ 

2.& Some $S$ are not non-$P$& T & Contrap. & \nix{ } \rule[-5pt]{2.5cm}{.4pt} &\nix{ } \rule[-5pt]{.5cm}{.4pt} \\


3. & No $S$ are non-$P$& T & Obv. & \nix{  } \rule[-5pt]{2.5cm}{.4pt} & \nix{ } \rule[-5pt]{.5cm}{.4pt} \\

4. & All non-$S$ are $P$& T & Obv. & \nix{ } \rule[-5pt]{2.5cm}{.4pt}& \nix{ } \rule[-5pt]{.5cm}{.4pt} \\

5. & Some non-$S$ are $P$& F & Contrap. & \nix{ } \rule[-5pt]{2.5cm}{.4pt}& \nix{ } \rule[-5pt]{.5cm}{.4pt} \\

6. & Some $S$ are $P$& F &  \nix{Obv.} \rule[-5pt]{1.5cm}{.4pt} & Some $S$ are not non-$P$  & \nix{F} \rule[-5pt]{.5cm}{.4pt} \\ 

7.& No non-$S$ are non-$P$& F &  \nix{Contrap.} \rule[-5pt]{1.5cm}{.4pt} & No $P$ are $S$  & \nix{?}\rule[-5pt]{.5cm}{.4pt} \\

8. & Some non-$S$ are non-$P$& T & \nix{Contrap.} \rule[-5pt]{1.5cm}{.4pt} & Some $P$ are $S$ & \nix{?} \rule[-5pt]{.5cm}{.4pt} \\

9.& All $S$ are $P$ & F & \nix{Obv.}\rule[-5pt]{1.5cm}{.4pt}  & No $S$ are non-$P$  & \nix{F} \rule[-5pt]{.5cm}{.4pt} \\

10. &  Some non-$S$ are not non-$P$ &  T & \nix{Obv.} \rule[-5pt]{1.5cm}{.4pt}  &Some non-$S$ are $P$   & \nix{T}  \rule[-5pt]{.5cm}{.4pt} \\
 \end{longtabu}

\noindent \problempart For each sentence, write the converse, obvserse, or contrapostive as directed.

\begin{longtabu}{p{.1\linewidth}p{.9\linewidth}}
\textbf{Example}: & Write the contrapositive of ``Some sentences are categorical.''\\
\textbf{Answer}: & Some non-categorical things are non-sentences. \\
\end{longtabu}

\begin{exercises}
\item Write the converse of ``No weeds are benign.''
\answer{\\No benign things are weeds}
 
\item Write the converse of ``Some minds are not closed.'' 
\answer{\\Some closed things are not minds}

\item Write the contraposition of ``Some dentists are underpaid.'' 
\answer{\\Some non-underpaid people are non-dentists}

\item Write the converse of ``All humor is good.'' 
\answer{\\All good things are humorous}

\item Write the contraposition of ``No organizations are self-sustaining.'' 
\answer{\\No non-self-sustaining things are non-organizations} 

\item Write the obverse of ``Some dogs have fleas.'' 
\answer{\\Some dogs are not non-flea-havers. \\
or Some dogs are not non-flea-having things.}

\item Write the converse of ``Some things that have fleas are dogs.'' 
\answer{\\Some dogs have fleas}

\item Write the obverse of ``No detectives are uniformed.'' 
\answer{\\All detectives are ununiformed. }

\item Write the converse of ``No monkeys are well-behaved.'' 
\answer{\\No well-behaved things are monkeys"}

\item Write the contraposition of ``No donkeys are obedient.'' 
\answer{\\No disobedient things are non-donkeys." }

\end{exercises}

\noindent \problempart For each sentence, write the converse, obvserse, or contrapostive as directed.
\begin{exercises} 
\item Write the converse of ``No supplies are limited.'' 
\item Write the obverse of ``No knives are toys.'' 
\item Write the contraposition of ``All logicians are rational.'' 
\item Write the obverse of ``All uniforms are clothing.'' 
\item Write the converse of ``All risks are negligible.'' 
\item Write the contraposition of ``No bestsellers are great works of literature.'' 
\item Write the obverse of ``Some descriptions are accurate.'' 
\item Write the contraposition of ``Some ties are not tacky.'' 
\item Write the obverse of ``All spies are concealed.'' 
\item Write the contraposition of ``No valleys are barren.'' 
\end{exercises}

\noindent \problempart Determine whether the following arguments are valid by drawing a Venn diagram for the premise. If they are valid, say whether they are valid by conversion, obversion, or contraposition.

\begin{longtabu}{p{.2\linewidth}p{.8\linewidth}}
\textbf{Example 1}: & All swans are white. Therefore, no swans are non-white.\\
\textbf{Answer}: & \\
&\noindent \begin{tikzpicture}
\def\firstcircle{(0,0) circle (.75cm)}
\def\secondcircle{(0:1cm) circle (.75cm)}
     \begin{scope}[shift={(4cm,0cm)}]
        \begin{scope}[even odd rule]% first circle without the second
            \clip \secondcircle (-1,-1) rectangle (1,1);
        \fill[gray] \firstcircle;
        \end{scope}
\draw \firstcircle node[outer sep=.66cm, above left] (s) {$S$};
\draw \secondcircle node[outer sep=.66cm, above right] (p) {$P$};
        \end{scope}
\end{tikzpicture}\\
& $S$: Swans \\
&$P$: White things. \\
& Valid, because the Venn diagram for the premise also makes the conclusion true.\\
& Obversion.\\
\end{longtabu}

\begin{longtabu}{p{.2\linewidth}p{.8\linewidth}}
\textbf{Example 2}: & Some dogs are pets. Therefore, some non-pets are non-dogs. \\
\textbf{Answer}: & \\
&\noindent \begin{tikzpicture}
\def\firstcircle{(0,0) circle (.75cm)}
\def\secondcircle{(0:1cm) circle (.75cm)}
     \begin{scope}[shift={(4cm,0cm)}]
        \begin{scope}[even odd rule]% first circle without the second
            \clip \secondcircle (-1,-1) rectangle (1,1);
        \fill[gray] \firstcircle;
        \end{scope}
\draw \firstcircle node[outer sep=.66cm, above left] (s) {$S$};
\draw \secondcircle node[outer sep=.66cm, above right] (p) {$P$};
        \end{scope}
\end{tikzpicture}\\
& $S$: Pets\\
&$P$: Dogs. \\
& Invalid, because the Venn diagram for the premise doesn't make the conclusion true.\\
\end{longtabu}

\begin{exercises}
\item Some smurfs are not blue. Therefore, some blue things are not smurfs.

\answer{
\begin{venns}
\someexistonesent
\drawsubsent
\drawpredsent
\end{venns}\\

S: Smurfs \\
P: Blue things\\
Invalid, because the Venn diagram for the premise does not make the conclusion true. }

\item All giraffes are majestic. Therefore, all non-majestic things are non-giraffes. 

\answer{
\begin{venns}
%\someexistonesent
\shadecomplementred{\subjectcircle}{\subjectsquare}{\predicatecircle}
\drawsubsent
\drawpredsent
\end{venns}\\
S: Giraffes\\
P: Majestic things\\
Valid, because the diagram for the premise is also the diagram for the conclusion.\\
Contraposition on a mood-A statement

}


\item Some roosters are not pets.	Therefore, some pets are not roosters.  

\answer{
\begin{venns}
\someexistonesent
%\shadecomplementred{\subjectcircle}{\subjectsquare}{\predicatecircle}
\drawsubsent
\drawpredsent
\end{venns}\\
S: Pets\\
P: Roosters\\
%Valid, because the diagram for the premise is also the diagram for the conclusion.
Invalid, because the Venn diagram for the premise does not make the conclusion true.
 }

\item No anesthesiologists are doctors. Therefore, no doctors are anesthesiologists.  

\answer{
\begin{venns}
%\someexistonesent
%\shadecomplementred{\subjectcircle}{\subjectsquare}{\predicatecircle}
\shadeintersectred{\subjectcircle}{\predicatecircle}
\drawsubsent
\drawpredsent
\end{venns}\\
S: Anesthesiologists\\
P: Soctors\\
Valid, because the diagram for the premise is also the diagram for the conclusion.
%Invalid, because the Venn diagram for the premise does not make the conclusion true.
Conversion on a mood-E statement. 
}

\item All penguins are flightless. Therefore, all flightless things are penguins. 

\answer{
\begin{venns}
%\someexistonesent
\shadecomplementred{\subjectcircle}{\subjectsquare}{\predicatecircle}
%\shadeintersectred{\subjectcircle}{\predicatecircle}
\drawsubsent
\drawpredsent
\end{venns}\\
S: Penguins\\
P: Flightless things \\
%Valid, because the diagram for the premise is also the diagram for the conclusion.
Invalid, because the Venn diagram for the premise does not make the conclusion true.
%Conversion.
}

\item No kisses are innocent. Therefore, no non-innocent things are non-kisses. 

\answer{
\begin{venns}
%\someexistonesent
%\shadecomplementred{\subjectcircle}{\subjectsquare}{\predicatecircle}
\shadeintersectred{\subjectcircle}{\predicatecircle}
\drawsubsent
\drawpredsent
\end{venns}\\
S: Kisses\\
P: Innocent things\\

%Invalid, because the Venn diagram for the premise does not make the conclusion true. 
Contraposition on a mood-E statement. 
}

\item All operas are sung. Therefore, all sung things are operas.

\answer{
\begin{venns}
%\someexistonesent
\shadecomplementred{\subjectcircle}{\subjectsquare}{\predicatecircle}
%\shadeintersectred{\subjectcircle}{\predicatecircle}
\drawsubsent
\drawpredsent
\end{venns}\\
S: Operas \\
P: Things that are sung\\
%Valid, because the diagram for the premise is also the diagram for the conclusion.
Invalid, because the Venn diagram for the premise does not make the conclusion true.
}
\item Some shopping malls are not abandoned. Therefore, some shopping malls are non-abandoned.    
\answer{
\begin{venns}
\someexistonesent
%\shadecomplementred{\subjectcircle}{\subjectsquare}{\predicatecircle}
%\shadeintersectred{\subjectcircle}{\predicatecircle}
\drawsubsent
\drawpredsent
\end{venns}\\
S: \\
P: \\
Valid, because the diagram for the premise is also the diagram for the conclusion.\\
%Invalid, because the Venn diagram for the premise does not make the conclusion true.
Obversion--always valid}

\item Some great-grandfathers are not deceased. Therefore, some non-deceased people are not non-great-grandfathers. 

\answer{
\begin{venns}
\someexistonesent
\drawsubsent
\drawpredsent
\end{venns}\\
S: Great-grandfathers\\
P: Deceased people\\
Valid, because the diagram for the premise is also the diagram for the conclusion. \\
%Invalid, because the Venn diagram for the premise does not make the conclusion true.
Contraposition on an O statement. 
}

\item Some boats are not seaworthy. Therefore, some boats are non-seaworthy.

\answer{
\begin{venns}
\someexistonesent
%\shadecomplementred{\subjectcircle}{\subjectsquare}{\predicatecircle}
%\shadeintersectred{\subjectcircle}{\predicatecircle}
\drawsubsent
\drawpredsent
\end{venns}\\
S: Boats\\
P: Seaworthy things\\
Valid, because the diagram for the premise is also the diagram for the conclusion.\\
%Invalid, because the Venn diagram for the premise does not make the conclusion true.
Obversion--always valid. 
}
\end{exercises}


\noindent \problempart Determine whether the following arguments are valid by drawing a Venn diagram for the premise. If they are valid, say whether they are valid by conversion, obversion, or contraposition.

\begin{exercises}
\item No platypuses are spies. Therefore, no non-platypuses are non-spies.     
\item All sunburns are painful. Therefore, all painful things are sunburns. 
\item All ghosts are friendly. Therefore, all non-friendly things are non-ghosts.  
\item Some philosophers are not logicians. Therefore, some philosophers are non-logicians. 
\item Some felines are lions. Therefore, some felines are not non-lions.  
\item No doubts are unreasonable. Therefore, no reasonable things are non-doubts.      
\item All Mondays are weekdays. Therefore, no Mondays are non-weekdays.  
\item Some colors are pastels. Therefore, some pastel things are colors.  
\item All dogs are cosmonauts. Therefore, all cosmonauts are dogs.     
\item All cobwebs are made of spider silk. 	Therefore, no cobwebs are made of non-spider silk.  
\end{exercises}

% ***********************************
% * The Traditional Square of Opposition *
% ***********************************

\section{The Traditional Square of Opposition}

We have seen that conversion, obversion, and contraposition allow us to identify some valid one-premise arguments. There are actually more we can find out there, but investigating them is a bit more complicated. The original investigation made by the Aristotelian philosophers made an assumption that logicians no longer make.  To help you understand all sides of the issue, we will begin by looking at things in the traditional Aristotelian fashion, and then in the next section move on to the modern way of looking at things. 

When Aristotle was first investigating these four kinds of categorical statements, he noticed they they conflicted with each other in different ways. If you are just thinking casually about it, you might say that ``No $S$ is $P$'' is somehow ``the opposite'' of ``All $S$ is $P$.'' But isn't the real ``opposite'' of ``All $S$ is $P$'' actually ``Some $S$ is not $P$''?  

Aristotle, in his book \textit{On Interpretation} (c. 350 \textsc{bce}/1\citeyear{Aristotle1984a}), notes that the real opposite of A is O, because one must always be true and the other false.  If we know that ``All dogs are mammals'' is true, then we know ``some dog is not a mammal'' is false. On the other hand, if ``All dogs are mammals'' is false then ``some dog is not a mammal'' must be true. Back on page \pageref{def:contradictory} we said that when two propositions must have opposite truth values they are called contradictories. Aristotle noted that A and O sentences are contradictory in this way. Forms E and I also form a contradictory pair. If ``Some dogs are mammals'' then ``No dogs are mammals'' is false, and if ``Some dogs are mammals'' is false, then ``No dogs are mammals'' is true.


\newglossaryentry{contraries}
{
name=contraries,
description={Two statements that can't both be true, but can both be false. A set two inconsistent sentences.}
}


Mood-A and mood-E statements are opposed to each other in a different way. Aristotle claimed that they can't both be true, but could both be false. Take the statements ``All dogs are strays'' and ``No dogs are strays.'' We know that they are both false, because some dogs are strays and others aren't. However, it is also clear that they could not both be true. When a pair of statements cannot both be true, but might both be false, the Aristotelian tradition says they are \textsc{\gls{contraries}}. \label{def:Contraries} Aristotle's idea of a pair of contraries is really just a specific case of a set of sentences that are \emph{inconsistent}, an idea that we looked at in Chapter \ref{chap:whatisformallogic}. (See page \ref{def:inconsistency})

\newglossaryentry{square of opposition}
{
name=square of opposition,
description={A way of representing the four basic propositions and the ways they relate to one another.}
}


These distinctions, plus a few other comments from Aristotle, were developed by his later followers into an idea that came to be known as the \textsc{\gls{square of opposition}} \label{def:Squareofopposition}. The square of opposition is simply the diagram you see in Figure \ref{fig:traditionalsquare}. It is a way of representing the four basic propositions and the ways they relate to one another.  As we said before, this way of picturing the proposition turned out to make a problematic assumption. To emphasize that this is no longer the way logicians view things, we will call this diagram the traditional square of opposition. 

\begin{figure}
\begin{mdframed}[style=mytablebox]
\begin{center}
\begin{tikzpicture}

\node[{font=\fontsize{45}{45}\selectfont\sf}, draw, regular polygon,regular polygon sides=8] (A) at (-4, 4) {A};
	\node [gray, xshift=-.75em, yshift=-.75em] (A-Under-True) at (A.south) {\large{T}}; 
	\node[gray, xshift=.75em, yshift=-.75em](A-Under-False) at (A.south) {\large{F}};
	\node[gray, xshift=.75em, yshift=.75em](A-Right-True) at (A.east) {\large{T}};
	\node[gray, xshift=.75em, yshift=-.6em](A-Right-False) at (A.east) {\large{F}};
	\node[gray, yshift=-1em, xshift=.2em](A-Corner-True) at (A.south east) {\large{T}};
	\node[gray, xshift=1em, yshift=-.2em](A-Corner-False) at (A.south east) {\large{F}};

\node[{font=\fontsize{45}{45}\selectfont\sf}, draw, regular polygon,regular polygon sides=8] (E) at ( 4, 4) {E};
	\node [xshift=-.75em, yshift=-.75em, gray] (E-Under-True) at (E.south) {\large{T}}; 
	\node[xshift=.75em, yshift=-.75em, gray](E-Under-False) at (E.south) {\large{F}};
	\node [xshift=-.75em, yshift=-.6em, gray] (E-Left-True) at (E.west) {\large{T}}; 
	\node[xshift=-.75em, yshift=.75em, gray](E-Left-False) at (E.west) {\large{F}};
	\node[gray, yshift=-1em, xshift=-.2em](E-Corner-True) at (E.south west) {\large{T}};
	\node[gray, xshift=-1em, yshift=-.2em](E-Corner-False) at (E.south west) {\large{F}};
\node[{font=\fontsize{45}{45}\selectfont\sf}, draw, regular polygon,regular polygon sides=8] (O) at ( 4, -4){O};
	\node [yshift=.75em, xshift=-.75em, gray] (O-Over-True) at (O.north) {\large{T}}; 
	\node[yshift=.75em, xshift=.75em, gray](O-Over-False) at (O.north) {\large{F}};
	\node [yshift=-.75em, xshift=-.75em, gray] (O-Left-True) at (O.west) {\large{T}}; 
	\node[yshift=.6em, xshift=-.75em, gray](O-Left-False) at (O.west) {\large{F}};
	\node[gray, yshift=1em, xshift=-.2em](O-Corner-True) at (O.north west) {\large{T}};
	\node[gray, xshift=-1em, yshift=.2em](O-Corner-False) at (O.north west) {\large{F}};

\node[{font=\fontsize{45}{45}\selectfont\sf}, draw, regular polygon,regular polygon sides=8] (I) at (-4,-4) {I};
	\node [yshift=.75em, xshift=-.75em, gray] (I-Over-True) at (I.north) {\large{T}}; 
	\node[yshift=.75em, xshift=.75em, gray](I-Over-False) at (I.north) {\large{F}};
	\node[yshift=.6em, xshift=.75em, gray](I-Right-True) at (I.east) {\large{T}};
	\node[yshift=-.75em, xshift=.75em, gray](I-Right-False) at (I.east) {\large{F}};
	\node[gray, yshift=1em, xshift=.2em](I-Corner-True) at (I.north east) {\large{T}};
	\node[gray, xshift=1em, yshift=.2em](I-Corner-False) at (I.north east) {\large{F}};

\draw [myarrow1] (A-Under-True) to node [xshift=-2.25em, rotate=90] {\color{black} Subalternation} node[xshift=-1em, rotate=90] {\color{gray}False up, true down}(I-Over-True); 
\draw [myarrow1] (I-Over-False) to (A-Under-False); 
\draw [myarrow1] (E-Under-True) to (O-Over-True); 
\draw [myarrow1] (O-Over-False) to node [xshift=2.25em, rotate=-90] {\color{black} Subalternation}node[xshift=1em, rotate=-90] {\color{gray}False up, true down} (E-Under-False); 
\draw [myarrow1, ->] (A-Right-True) to node [yshift=18pt] { \color{black} Contraries} (E-Left-False); 
\draw [myarrow1, <-] (A-Right-False) to node [yshift=22pt] {\small \color{gray} Not both true} (E-Left-True);
\draw [myarrow1, <-] (I-Right-True) to node [yshift=-3.5em] { \color{gray} Not both false} (O-Left-False); 
\draw [myarrow1, ->] (I-Right-False) to node [yshift=-1em] {\small \color{black} Subcontraries} (O-Left-True);

\draw [myarrow1, <->] (I-Corner-True) to (E-Corner-False);
\draw [myarrow1, <->] (I-Corner-False) to (E-Corner-True);
\draw [myarrow1, <->] (A-Corner-False) to (O-Corner-True);
\draw [myarrow1, <->] (A-Corner-True) to node [fill=light-gray, yshift=.6em]{ \color{black} Contradiction} node [fill=light-gray, yshift=-.8em] {\small \color{gray} One true, one false} (O-Corner-False); 


%[yshift=3em, xshift=-2em, rotate=-45]
%node [yshift=-3em, xshift=-2em, rotate=45] {\small \color{black} Contradiction}
\end{tikzpicture}
\end{center}
\end{mdframed}
\caption{The traditional square of opposition}
\label{fig:traditionalsquare}
\end{figure}

The traditional square of opposition begins by picturing a square with A, E, I, and O at the four corners. The lines between the corners then represent the ways that the kinds of propositions can be opposed to each other. The diagonal lines between A and O and between E and I represent contradiction. These are pairs of propositions where one has to be true and the other false. The line across the top represents contraries. These are propositions that Aristotle thought could not both be true, although they might both be false. 

In Figure \ref{fig:traditionalsquare}, we have actually drawn each relationship as a pair of lines, representing the kinds of inferences you can make in that relationship. Contraries cannot both be true. So we know that if one is true, the other must be false. This is represented by the two lines going from a T to an F. Notice that there aren't any lines here that point from an F to something else. This is because you can't infer anything about contrary statements if you just know that one is false. For the contradictory statements, on the other hand, we have drawn double-headed arrows. This is because we know both that the truth of one statement implies that the other is false and that the falsity of one statement implies the truth of the other. 

\newglossaryentry{subcontraries}
{
name=subcontraries,
description={Two categorical statements that cannot both be false, but might both be true.}
}

Contraries and contradictories just give us the diagonal lines and the top line of the square. There are still three other sides to investigate. Form I and form O are called \textsc{\gls{subcontraries}}. \label{defSubcontraries} In the traditional square of opposition, their situation is reversed from that of A and E. Statements of forms A and E cannot both be true, but they can both be false. Statements of forms I and O cannot both be false, but they can both be true. Consider the sentences ``Some people in the classroom are paying attention'' and ``Some people in the classroom are not paying attention.'' It is possible for them both to be true. Some people are paying attention and some aren't. But the two sentences couldn't both be false. That would mean that everyone in the room was neither paying attention nor not paying attention. But they have to be doing one or the other!

This means that there are two inferences we can make about subcontraries. We know that if I is false, O must be true, and vice versa. This is represented in Figure \ref{fig:traditionalsquare} by arrows going from Fs on one side to Ts on the other. This is reversed from the way things were on the top of the square with the contraries. Notice that this time there are no arrows going away from a T. This is because we can't infer anything about subcontraries if all we know is that one is true

\newglossaryentry{subalternation}
{
name=subalternation,
description={The relationship between a universal categorical statement and the particular statement with the same quality.}
}


The trickiest relationship is the one between universal statements and their corresponding particulars. We call this \textsc{\gls{subalternation}}. Both of the statements in these pairs could be true, or they could both be false. However, in the traditional square of opposition, if the universal statement is true, its corresponding particular statement must also be true. For instance, ``All dogs are mammals'' implies that some dogs are mammals. Also, if the particular statement is false, then the universal statement must also be false. Consider the statement ``Some dinosaurs had feathers.'' If that statement is false, if no dinosaurs had feathers, then ``All dinosaurs have feathers'' must also be false. Something like this seems to be true on the negative side of the diagram as well. If ``No dinosaurs have feathers'' is true, then you would think that ``some dinosaurs do not have feathers'' is true. Similarly, if ``some dinosaurs do not have feathers'' is false, then ``No dinosaurs have feathers'' cannot be true either. 

In our diagram for the traditional square of opposition, we represent subalternation by a downward arrow for truth and an upward arrow for falsity. We can infer something here if we know the top is true, or if we know the bottom is false. In other situations, there is nothing we can infer. 

Note, by the way, that the language of subalternation works a little differently than the other relationships. With contradiction, we say that each sentence is the ``contradictory'' of the other. The relationship is symmetrical. With subalternation, we say that the particular sentence is the ``subaltern'' of the universal one, but not the other way around.  

People started using diagrams like this as early as the second century \textsc{ce} to explain Aristotle's ideas in \textit{On Interpretation} (See Parsons \citeyear{Parsons1997}). Figure \ref{fig:apuleiussquare} shows one of the earliest surviving versions of the square of opposition, from a 9th century manuscript of a commentary on Aristotle attributed to the Roman writer Apuleius of Madaura. Although this particular manuscript dates from the 9th century, the commentary itself was written in the 2nd century, and copied by hand many times over before this one was made. Figure \ref{fig:majorsquare} shows a later illustration of the square, from a 16th century book by the Scottish philosopher and logician Johannes de Magistris.

\begin{figure}
\begin{mdframed}[style=mytableclearbox]
\begin{center}
\includegraphics*{img/apuleiussquare}
\end{center}
\end{mdframed}
\caption{One of the earliest surviving versions of the square of opposition, from a 9th century manuscript of a commentary by Apuleius of Madaura on Aristotle's \textit{On Interpretation}. Digital image from \protect\url{www.logicmuseum.com}, curated by Edward Buckner.}
\label{fig:apuleiussquare}
\end{figure}


\begin{figure}
\begin{mdframed}[style=mytableclearbox]
\begin{center}
\includegraphics*[scale=.5]{img/Johannesmagistris-square}
\end{center}
\end{mdframed}
\caption{A 16th century illustration of the square of opposition from \textit{Summulae Logicales} by Johannes de Magistris, digital image by Peter Damian and uploaded to Wikimedia Commons: \protect\url{tinyurl.com/kmzmvzn}. Public Domain-U.S. }
\label{fig:majorsquare}
\end{figure}

As with the processes of conversion, obversion, and contraposition, we can use the traditional square of opposition to evaluate arguments written in canonical form. It will help us here to introduce the phrase ``It is false that'' to some of our statements, so that we can make inferences from the truth of one proposition to the falsity of another. This, for instance, is a valid argument, because A and O statements are contradictories.


\begin{earg*}
\item All humans are mortal.
\itemc[.45] It is false that some human is not mortal. 
\end{earg*} 

The argument above is an immediate inference, like the arguments we saw in the previous section, because it only has one premise. It is also similar to those arguments in that the conclusion is actually logically equivalent to the premise. This will not be the case for all immediate inferences based on the square of opposition, however. This is a valid argument, based on the subaltern relationship, but the premise and the conclusion are not logically equivalent. 

\begin{earg*}
\item It is false that some humans are dinosaurs.
\itemc[.45]  It is false that all humans are dinosaurs.
\end{earg*}

%%%%%%%%%%%%%%%%%%%% Practice Problems %%%%%

\practiceproblems
\noindent \problempart For each pair of sentences say whether they are contradictories, contraries, subcontraries, or one is the subaltern of the other. 

\begin{longtabu}{p{.1\linewidth}p{.9\linewidth}}
\textbf{Example}: & Some peppers are spicy. \newline No peppers are spicy. \\
\textbf{Answer}: &Contradictory\\
\end{longtabu}

\begin{exercises}

\item No quotations are spurious. \nix{{\color{red}E}} \\
	Some quotations are not spurious. \nix{{\color{red}O}} \answer{\\The second is subaltern to the first.}

\item Some children are not picky eaters. \nix{{\color{red}O}} \\
	All children are picky eaters. \nix{{\color{red}A}}\answer{\\Contradictories}

\item Some joys are not fleeting. \nix{{\color{red}O}} \\
	Some joys are fleeting.  \nix{{\color{red}I}}\answer{\\Subcontraries}

\item All fires are hot. \nix{{\color{red}A}} \\
	Some fires are not hot.  \nix{{\color{red}O}}\answer{\\Contradictories}

\item Some diseases are not fatal. \nix{{\color{red}O}} \\
	No diseases are fatal.  \nix{{\color{red}E}}\answer{\\The first is the subaltern of the second}

\item Some planets are not habitable. \nix{{\color{red}O}} \\
	Some planets are habitable. \nix{{\color{red}I}}\answer{\\Subcontraries}

\item Some toys are plastic. \nix{{\color{red}I}} \\
	No toys are plastic.  \nix{{\color{red}E}} \answer{\\Contradictories}

\item No transfats are healthy. \nix{{\color{red}E}} \\
	All transfats are healthy.  \nix{{\color{red}A}}\answer{\\Contraries}

\item No superheroes are invincible. \nix{{\color{red}E}}  \\
	Some superheroes are invincible. \nix{\nix{{\color{red}I}}}\answer{\\Contradictories}

\item Some villains are deplorable. \nix{{\color{red}I}} \\ 
	Some villains are not deplorable. \nix{{\color{red}O}} \answer{\\Subcontraries}

\end{exercises}

\noindent \problempart For each pair of sentences say whether they are contradictories, contraries, subcontraries, or one is the subaltern of the other. 

\begin{exercises}
\item No pants are headgear. \\
	All pants are headgear.  
\item Some dietitians are not qualified. \\
	All dietitians are qualified.  
\item Some monkeys are curious. \\ 
	No monkeys are curious.  
\item All dolphins are intelligent. \\
	Some dolphins are intelligent.  
\item No manuscripts are accepted. \\
	All manuscripts are accepted. 
\item Some hijinks are wacky. \\ 
	No hijinks are wacky. 
\item All clowns are terrifying. \\
	No clowns are terrifying.  
\item No cupcakes are nutritious. \\
	Some cupcakes are not nutritious. 
\item ``Some kinds of love are mistaken for vision.'' --Lou Reed \\
	All kinds of love are mistaken for vision. 
\item All sharks are cartilaginous. \\
	No sharks are cartilaginous.
\end{exercises}

\noindent \problempart For each sentence write its contradictory, contrary, subcontrary, or the corresponding sentence in subalternation as directed.

\begin{longtabu}{p{.1\linewidth}p{.9\linewidth}}
\textbf{Example}: & Write the subcontrary of ``Some jellyfish sting.''\\
\textbf{Answer}: & Some jellyfish do not sting.\\
\end{longtabu}


\begin{exercises}
\item Write the contrary of ``No hashtags are symbols.'' \answer{(E) \\
All hastags are symbols. (A)}	

\item Write the contradictory of ``All elephants are social.'' \answer{(A) \\ Some elephants are not social. (O)}

\item Write the subcontrary of ``Some children are well behaved.'' \answer{(E)\\Some children are not well behaved (O)}

\item Write the contradictory of ``All eggplants are purple.'' \answer{(A) \\ Some eggplants are not purple (O)}
 
\item Write the sentence that ``Some guitars are electric'' is a subaltern of.  \answer{(I)\\All guitars are electric (A)}

 \item Write the contradictory of  ``Some arches are not crumbling.'' \answer{(O)\\All arches are crumbling (A) }
 
\item Write the contrary of ``No resolutions are unsatisfying.'' \answer{(E)\\ All resolutions are satisfying (A)}

 \item Write the contradictory of  ``All flags are flying.'' \answer{(A)\\Some flags are not flying (I) }
 
\item Write the subaltern of ``No pains are chronic.'' \answer{(E)\\Some pains are not chronic (O)}
 
\item Write the contradictory of  ``No puffins are mammals.'\answer{(E)\\Some puffins are mammals (I)}
\end{exercises}


\noindent \problempart For each sentence write its contradictory, contrary, subcontrary, or the corresponding sentence in subalternation as directed.

\begin{exercises}
\item Write the subaltern of ``No libraries are unfunded.'' 
\item Write the contrary of ``All hooks are sharp.''
\item Write the contradictory of ``Some tankers are not seaworthy.'' 
\item Write the sentence that ``Some positions are not tenable'' is the subaltern of.  
\item Write the contradictory of ``Some haircuts are unfortunate.'' 
\item Write the contradictory of ``No violins are worthless.'' 
\item Write the subcontrary of ``Some missiles are not nuclear.'' 
\item Write the contrary of ``All animals are lifeforms.'' 
\item Write the contradictory of ``All animals are lifeforms.'' 
\item Write the subaltern of ``All animals are lifeforms.'' 
\end{exercises}

\noindent \problempart Given a sentence and its truth value, evaluate the truth of a second sentence, according to the traditional square of opposition. If the truth value cannot be determined, just write ``undetermined.''

\begin{longtabu}{p{.1\linewidth}p{.9\linewidth}}
\textbf{Example}: & If ``Some $S$ are $P$'' is true, what is the truth value of ``Some $S$ are not $P$''?\\
\textbf{Answer}: & Undetermined\\
\end{longtabu}

\begin{exercises}
\item If ``Some $S$ are not $P$'' is true, what is the truth value of ``All $S$ are $P$''?
\item If ``Some $S$ are not $P$'' is false, what is the truth value of ``Some $S$ are $P$''?
\item If ``All $S$ are $P$'' is true, what is the truth value of ``No $S$ are $P$''? 
\item If ``Some $S$ are not $P$'' is false, what is the truth value of ``No $S$ are $P$''?  
\item If ``No $S$ are $P$'' is true, what is the truth value of ``Some $S$ are not $P$''? 
\item If ``Some $S$ are not $P$'' is true, what is the truth value of ``All $S$ are $P$''? 
\item If ``Some $S$ are $P$'' is true, what is the truth value of ``All $S$ are $P$''? 
\item If ``All $S$ are $P$'' is false, what is the truth value of ``Some $S$ are $P$''? 
\item If ``No $S$ are $P$'' is false, what is the truth value of ``All $S$ are $P$''?
\item If ``No $S$ are $P$'' is true, what is the truth value of ``Some $S$ are $P$''?
\end{exercises}


\noindent \problempart Given a sentence and its truth value, evaluate the truth of a second sentence, according to the traditional square of opposition. If the truth value cannot be determined, just write ``undetermined.''

\begin{exercises}
\item If ``Some $S$ are not $P$'' is true, what is the truth value of ``All $S$ are $P$''?
\answer{\\False. If a mood-O statement is true, the corresponding mood-A statement is false, because they are contradictories, which always have opposite truth values.}

\item If ``Some $S$ are not $P$'' is false, what is the truth value of ``Some $S$ are $P$''? 
\answer{\\If a mood-O statement is false, the corresponding mood-E statement must be true, because they are subcontraries, and subcontraries can't both be false.}

\item If ``All $S$ are $P$'' is true, what is the truth value of ``No $S$ are $P$''? 
\answer{\\ False. If a mood-A statement is true, the corresponding mood-E statement must be false, because they are contraries, and can't both be true.}

\item If ``Some $S$ are not $P$'' is false, what is the truth value of ``No $S$ are $P$''? 
\answer{\\False. If a mood-O statement is false, then the statement it is a subaltern of, a mood-E statement, must also be false.}

\item If ``No $S$ are $P$'' is true, what is the truth value of ``Some $S$ are not $P$''? 
\answer{\\ True. If a mood-E statement is true, it's subaltern, the mood-O statement, must also be true.}
 
\item If ``Some $S$ are not $P$'' is true, what is the truth value of ``All $S$ are $P$''? 
\answer{\\ False. If a mood-O statement is true, then it's contradictory, a mood-A statement, is false. }

\item If ``Some $S$ are $P$'' is true, what is the truth value of ``All $S$ are $P$''? 
\answer{\\ Undetermined. If a mood-I statement is true, we know nothing about the mood A statement it is a subaltern of.}

\item If ``All $S$ are $P$'' is false, what is the truth value of ``Some $S$ are $P$''? 
\answer{\\Undetermined. If a mood-A statement is false, its subaltern, a mood-I statement, can still be true.}

\item If ``No $S$ are $P$'' is false, what is the truth value of ``All $S$ are $P$''? 
\answer{\\Undetermined. Mood-E and mood-A statements are contraries, so they can't both be true, but they can both be false.}

\item If ``No $S$ are $P$'' is true, what is the truth value of ``Some $S$ are $P$''? 
\answer{\\False. If a mood-E statement is true, its contradictory, a mood-I statement, must be false.}

\end{exercises}

\noindent\problempart Evaluate the following arguments using the traditional square of opposition. If the argument is valid, say which relationship in the square of opposition makes it valid. 

\begin{longtabu}{p{.1\linewidth}p{.9\linewidth}}
\textbf{Example}: & No $S$ are $P$.Therefore, some $S$ are not $P$.\\
\textbf{Answer}: & Valid, because the conclusion is the subaltern of the premise.\\
\end{longtabu}


\begin{exercises}
\item No $S$ are $P$. Therefore, it is false that some $S$ are $P$.
\answer{\\Valid. If a mood-E statement is true, then its contradictory, the corresponding mood-E statement, must be false. }

\item It is false that no $S$ are $P$. Therefore, it is false that all $S$ are $P$.
\answer{\\Invalid. The statements are contraries, so they can both be false, but they don't have to be. \\

Another way to see this is to consider the case where $S$ stands for ``dogs'' and $P$ stands for ``mammals.'' 
\begin{earg*}
\item It is false that no dogs are mammals \hspace{1cm} $\Leftarrow$ True premise 
\itemc[.4] It is false that all dogs are mammals \hspace{1cm} $\Leftarrow$ False conclusion
\end{earg*}
}

\item All $S$ are $P$. Therefore, it is false that no $S$ are $P$.
\answer{\\ Valid. ``All $S$ are $P$'' and ``No $S$ are $P$'' are contraries. So the first one is true is true, then the second one is false. }

\item It is false that no $S$ are $P$. Therefore, it is false that some $S$ are not $P$.
\answer{\\Invalid. The ``Some $S$ are not $P$ is the subaltern of ``No $S$ are $P$. The first sentence can be false, while the second one is still true. Consider ``No dogs are brown'' and ``Some dogs are not brown.'' 
\begin{earg*}
\item It is false that no dogs are brown \hspace{1cm} $\Leftarrow$ True premise 
\itemc[.4] It is false that some dogs are not brown \hspace{1cm} $\Leftarrow$ False conclusion
\end{earg*}}

\item It is false that all $S$ are $P$. Therefore, some $S$ are not $P$.
\answer{ \\Valid. ``All $S$ are $P$'' and ``Some $S$ are not $P$'' are contradictory, so if the first one is false, the second one must be true. }

\item It is false that no $S$ are $P$. Therefore, some $S$ are $P$.
\answer{\\ Valid. ``No $S$ are $P$'' and ``Some $S$ are $P$'' are contradictory. Therefore if the first is false, the second must be true. }

\item It is false that all $S$ are $P$. Therefore, some $S$ are $P$.
\answer{ \\Invalid. If ``All $S$ are $P$'' is false, then ``Some $S$ are $P$'' could be false. Consider ``All dogs are reptiles.''
\begin{earg*}
\item It is false that all dogs are reptiles \hspace{1cm} $\Leftarrow$ True premise 
\itemc[.4] Some dogs are reptiles. \hspace{2.5cm} $\Leftarrow$ False conclusion. 
\end{earg*}    }

\item Some $S$ are $P$. Therefore, all $S$ are $P$.
\answer{\\Invalid \begin{earg*}
\item Some dogs are white \hspace{1cm} $\Leftarrow$ True premise 
\itemc[.4] All dogs are white. \hspace{2.5cm} $\Leftarrow$ False conclusion. 
\end{earg*}   }

\item It is false that some $S$ are $P$. Therefore, some $S$ are $P$.
\answer{Invalid, obviously}

\item It is false that some $S$ are not $P$. Therefore, it is false that no $S$ are $P$. 
\answer{\\Valid. If a mood-O statement is false, than the statement it is a subaltern of is also false.}
\end{exercises}


\noindent\problempart Evaluate the following arguments using the traditional square of opposition. If the argument is valid, say which relationship in the square of opposition makes it valid. 


\begin{exercises}
\item Some $S$ are not $P$. Therefore, it is false that some $S$ are $P$ 
\item It is false that some $S$ are not $P$. Therefore, some $S$ are $P$  
\item It is false that all $S$ are $P$. Therefore, some $S$ are not $P$.  
\item It is false that all $S$ are $P$. Therefore, no $S$ are $P$.
\item Some $S$ are $P$. Therefore, it is false that no $S$ are $P$.
\item Some $S$ are $P$. Therefore, it is false that some $S$ are not $P$
\item No $S$ are $P$. Therefore, it is false that all $S$ are $P$.
\item It is false that no $S$ are $P$. Therefore, all $S$ are $P$.
\item It is false that all $S$ are $P$. Therefore, it is false that some $S$ are $P$.
\item All $S$ are $P$. Therefore, some $S$ are $P$.
\end{exercises}

% ****************************************************
% * Existential Import and the Modern Square of Opposition    *
% ****************************************************

\section{Existential Import and the Modern Square of Opposition}
\label{sec:ExistentialImport}

The traditional square of opposition seems straightforward and fairly clever. Aristotle made an interesting distinction between contraries and contradictories, and subsequent logicians developed it into a nifty little diagram. So why did we have to keep saying things like ``Aristotle thought'' and ``according to the traditional square of opposition.'' What is wrong here?

The traditional square of opposition goes awry because it makes assumptions about the existence of the things being talked about. Remember that when we drew the Venn diagram for ``All $S$ are $P$,'' we shaded out the area of $S$ that did not overlap with $P$ to show that nothing could exist there. We pointed out, though, that we did not put a little x in the intersection between $S$ and $P$. Statements of the form A ruled out the existence of one kind of thing, but they did not assert the existence of another. The A proposition, ``All dogs are mammals,'' denies the existence of any dog that is not a mammal, but it does not assert the existence of some dog that is a mammal. But why not? Dogs obviously do exist.

The problem comes when you start to consider categorical statements about things that don't exist, for instance ``All unicorns have one horn.'' This seems like a true statement, but unicorns don't exist. Perhaps what we mean by ``All unicorns have one horn'' is that \emph{if} a unicorn existed, \emph{then} is would have one horn. But if we interpret the statement about unicorns that way, shouldn't we also interpret the statement about dogs that way? Really all we mean when we say ``All dogs are mammals'' is that if there were dogs, then they would be mammals. It takes an extra assertion to point out that dogs do, in fact, exist. 

\newglossaryentry{existential import}
{
name=existential import,
description={An aspect of the meaning of a statement that which is present if the statement can only be true when the objects it describes exist.}
}

\newglossaryentry{vacuous truth}
{
name=vacuous truth,
description={The kind of truth possessed by statements that do not have existential import and refer to objects that do not exist.}
}

The issue we are discussing here is called existential import. A sentence is said to have \textsc{\gls{existential import}} \label{def:Existential_import} if it asserts the existence of the things it is talking about. Figure \ref{fig:existential_import} shows the two ways you could draw Venn diagrams for an A statement, with the x, as in the traditional interpretation, and without, as in our interpretation. If you interpret A statements in the traditional way, they are always false when you are talking about things that don't exist. So, ``All unicorns have one horn'' is false in the traditional interpretation. On the other hand, in the modern interpretation all statements about things that don't exist are true. ``All unicorns have one horn'' simply asserts that there are no multi-horned unicorns, and this is true because there are no unicorns at all. We call this \textsc{\gls{vacuous truth}}. Something is vacuously true \label{def:Vacuous_truth} if it is true simply because it is about things that don't exist. Note that \emph{all} statements about nonexistent things become vacuously true if you assume they have no existential import, even a statement like ``All unicorns have more than one horn.'' A statement like this simply rules out the existence of unicorns with one horn or fewer, and these don't exist because unicorns don't exist. This is a complicated issue that will come up again starting in Chapter \ref{chap:SL} when we consider conditional statements. For now just assume that this makes sense because you can make up any stories you want about unicorns. 

\begin{figure}
\begin{mdframed}[style=mytablebox]
\begin{tabu}{X[1,c,m]X[1,c,m]} 

\begin{tikzpicture}
\def\firstcircle{(0,0) circle (.66cm)}
\def\secondcircle{(0:.88cm) circle (.66cm)}
     \begin{scope}[shift={(4cm,0cm)}]
        \begin{scope}[even odd rule]% first circle without the second
            \clip \secondcircle (-.66,-.66) rectangle (.66,.66);
        \fill[gray] \firstcircle;
        \end{scope}
        \draw \firstcircle node[outer sep=.4cm, above left] (s) {$S$};
        \draw \secondcircle node[outer sep=.4cm, above right] (p) {$P$};
    \end{scope} 
\end{tikzpicture}

&

\begin{tikzpicture}

\def\firstcircle{(0,0) circle (.66cm)}
\def\secondcircle{(0:.88cm) circle (.66cm)}
     \begin{scope}[shift={(4cm,0cm)}]
        \begin{scope}[even odd rule]% first circle without the second
            \clip \secondcircle (-.66,-.66) rectangle (.66,.66);
        \fill[gray] \firstcircle;
        \end{scope}
        \draw \firstcircle node[outer sep=.4cm, above left] (s) {$S$};
	\node[outer sep=.22cm, right] (x) {x};
        \draw \secondcircle node[outer sep=.4cm, above right] (p) {$P$};
    \end{scope} 
\end{tikzpicture}

\\

Without existential import 
(Modern).

&

With existential import 
(Traditional).  

\end{tabu}
\end{mdframed}
\caption{Interpretations of A: ``All $S$ are $P$.''}
\label{fig:existential_import}
\end{figure}

Any statement can be read with or without existential import, even the particular ones. Consider the statements ``Some unicorns are rainbow colored'' and ``Some unicorns are not rainbow colored.'' You can argue that both of these statements are true, in the sense that if unicorns existed, they could come in many colors. If you say these statements are true, however, you are assuming that particular statements do not have existential import. As Terence Parsons (\citeyear{Parsons1997}) points out, you can change the wording of particular categorical statements in English to make them seem like they do or do not have existential import. ``Some unicorns are not rainbow colored'' might have existential import, but ``not every unicorn is rainbow colored'' doesn't seem to.	

So what does this have to do with the square of opposition? A lot of the claims made in the traditional square of opposition depend on assumptions about which statements have existential import. For instance, Aristotle's claim that contrary statements cannot both be true requires that A statements have existential import. Think about the sentences ``All dragons breathe fire'' and ``no dragons breathe fire.'' If the first sentence has no existential import, then both sentences could actually be true. They are both ruling out the existence of certain kinds of dragons and are correct because no dragons exist. 

In fact, the entire traditional square of opposition falls apart if you assume that all four forms of a categorical statement have existential import. Parsons (\citeyear{Parsons1997}) shows how we can derive a contradiction in this situation. Consider the I statement ``Some dragons breathe fire.'' If you interpret it as having existential import, it is false, because dragons don't exist. But then its contradictory statement, the E statement ``No dragons breathe fire'' must be true. And if that statement is true, and has existential import, then its subaltern, ``Some dragon does not breathe fire'' is true. But if it has existential import, it can't be true, because dragons don't exist. In logic, the worst thing you can ever do is contradict yourself, but that is what we have just done. So we have to change the traditional square of opposition.

 The way some textbooks talk about the problem, you'd think that for two thousand years logicians were simply ignorant about the problem of existential import and thus woefully confused about the square of opposition, until finally George Boole wrote \textit{The Laws of Thought} (\citeyear{Boole1854}) and found the one true solution to the problem. In fact, there was an extensive discussion of existential import from the 12th to the 16th centuries, mostly under the heading of the ``supposition'' of a term. Very roughly, we can say that the supposition of a term is the way it refers to objects, or what we now call the ``denotation'' of the term (Read \citeyear{Read2002}). \nix{insert cross reference to chapter on definitions when available} So in ``All people are mortal'' the supposition of the subject term is all of the people out there in the world. Or, as the medievals sometimes put it, the subject term ``supposits'' all the people in the world.
 
At least some medieval thinkers had a theory of supposition that made the traditional square of opposition work. Terrance Parsons (\citeyear{Parsons1997}, \citeyear{Parsons2008})  has argued for the importance of one solution, found most clearly in the writings of William of Ockham. Under this theory, affirmative forms A and I had existential import, but the negative forms E and O did not. We would say that a statement has existential import if it would be false whenever the subject or predicate terms refer to things that don't exist. To put the matter more precisely, we would say that the statement would be false whenever the subject or predicate terms ``fail to refer.'' Linguistic philosophers these days prefer say that a term ``fails to refer'' rather than saying that it ``refers to something that doesn't exist,'' because referring to things that don't exist seems impossible.

In any case, Ockham describes the supposition of affirmative propositions the same way we would describe the reference of terms in those propositions. Again, if the proposition supposes the existence of something in the world, the medievals would say it ``supposits.''  Ockham says ``In affirmative propositions a term is always asserted to supposit for something. Thus, if it supposits for nothing the proposition is false'' (\citeyear{Ockham1343}/1974). On the other hand, failure to refer or to supposit actually supports the truth of negative propositions: ``in negative propositions the assertion is either that the term does not supposit for something or that it supposits for something of which the predicate is truly denied. Thus a negative proposition has two causes of truth'' (\citeyear{Ockham1343}/1974). 

So, for Ockham, affirmative statements about nonexistent objects are false. ``All unicorns have one horn'' and ``Some unicorns are rainbow colored'' are false, because there are no unicorns. Negative statements, on the other hand, are vacuously true. ``No unicorns are rainbow colored'' and ``No unicorns have one horn'' are both true. There are no rainbow colored unicorns out there, and no one horned unicorns out there, because there are no unicorns out there. The O statement ``Some unicorns are not rainbow colored'' is also vacuously true. This might be harder to see, but it helps to think of the statement as saying ``It is not the case that every unicorn is rainbow colored.'' 

This way of thinking about existential import leaves the traditional square of opposition intact, even in cases where you are referring to nonexistent objects. Contraries still cannot both be true when you are talking about nonexistent objects, because the A proposition will be false, and the E vacuously true. ``All dragons breathe fire'' is false, because dragons don't exist, and ``No dragons breathe fire'' is vacuously true for the same reason. Similarly, subcontraries cannot both be false when talking about dragons and whatnot, because the I will always be false and the O will always be true. You can go through the rest of the relationships and show that similar arguments hold. \label{proving_trad_square}
         
Boole proposed a different solution, which is now taken as the standard way to do things. Instead of looking at the division between positive and negative statements, Boole looked at the division between singular and universal propositions. The universal statements A and E do not have existential import, but the particular statements I and O do have existential import. Thus all particular statements about nonexistent things are false and all universal statements about nonexistent things are vacuously true. 

John Venn was building on the work of George Boole. His diagrams avoided the problems that Euler had by using a Boolean interpretation of mood-A statements, where they really just assert that something is impossible. In fact, the whole system of Venn diagrams embodies Boole's assumptions about existential import, as you can see in Figure \ref{fig:fourvenns}. The particular forms I and O have you draw an x, indicating that something exists. The other two forms just have us shade in regions to indicate that certain combinations of subject and predicate are impossible. Thus A and E statements like ``All dragons breathe fire'' or ``No dragons are friendly'' can be true, even though no dragons exist. 

Venn diagrams doesn't even have the capacity to represent Ockham's understanding of existential import. We can represent A statements as having existential import by adding an x, as we did on the right hand side of Figure \ref{fig:existential_import}. However, we have no way to represent the O form without existential import. We have to draw the x, indicating existence. We don't have a way of representing O form statements about nonexistent objects as vacuously true. 

The Boolean solution to the the question of existential import leaves us with a greatly restricted form of the square of opposition. Contrary statements are both vacuously true when you refer to nonexistent objects, because neither have existential import. Subcontrary statements are both false when you refer to nonexistent objects, because they do have existential import. Finally, the subalterns of vacuously true statements are false, while on the traditional square of opposition they had to be true. The only thing remaining from the traditional square of opposition is the relationship of contradiction, as you can see in Figure \ref{fig:modernsquare}.

\begin{figure}
\begin{mdframed}[style=mytablebox]
\begin{center}
\begin{tikzpicture}
\node[{font=\fontsize{35}{35}\selectfont\sf}, draw, regular polygon,regular polygon sides=8] (A) at (-3, 3) {A};
	\node[gray, yshift=-1em, xshift=.2em](A-Corner-True) at (A.south east) {\large{T}};
	\node[gray, xshift=1em, yshift=-.2em](A-Corner-False) at (A.south east) {\large{F}};

\node[{font=\fontsize{35}{35}\selectfont\sf}, draw, regular polygon,regular polygon sides=8] (E) at ( 3, 3) {E};
	\node[gray, yshift=-1em, xshift=-.2em](E-Corner-True) at (E.south west) {\large{T}};
	\node[gray, xshift=-1em, yshift=-.2em](E-Corner-False) at (E.south west) {\large{F}};
\node[{font=\fontsize{35}{35}\selectfont\sf}, draw, regular polygon,regular polygon sides=8] (O) at ( 3, -3){O};
	\node[gray, yshift=1em, xshift=-.2em](O-Corner-True) at (O.north west) {\large{T}};
	\node[gray, xshift=-1em, yshift=.2em](O-Corner-False) at (O.north west) {\large{F}};

\node[{font=\fontsize{35}{35}\selectfont\sf}, draw, regular polygon,regular polygon sides=8] (I) at (-3,-3) {I};
	\node[gray, yshift=1em, xshift=.2em](I-Corner-True) at (I.north east) {\large{T}};
	\node[gray, xshift=1em, yshift=.2em](I-Corner-False) at (I.north east) {\large{F}};

\draw [myarrow1, <->] (I-Corner-True) to (E-Corner-False);
\draw [myarrow1, <->] (I-Corner-False) to (E-Corner-True);
\draw [myarrow1, <->] (A-Corner-False) to (O-Corner-True);
\draw [myarrow1, <->] (A-Corner-True) to node [fill=light-gray, yshift=.6em]{ \color{black} Contradiction} node [fill=light-gray, yshift=-.8em] {\small \color{gray} One true, one false} (O-Corner-False); 


%[yshift=3em, xshift=-2em, rotate=-45]
%node [yshift=-3em, xshift=-2em, rotate=45] {\small \color{black} Contradiction}
\end{tikzpicture}
\end{center}
\end{mdframed}
\caption{The modern square of opposition}
\label{fig:modernsquare}
\end{figure}

%%%%%%%%%%%%%%%%%%%%%%%%%% Practice Problems 

\practiceproblems
\problempart Evaluate each of the following arguments twice. First, evaluate it using Ockham's theory of existential import, where positive statements have existential import and negative ones do not. If the argument is valid, state which relationship makes it valid (contradictories, contraries, etc.) Second, evaluate the argument using Boole's theory, where particular statements have existential import and universal statements do not. %If the premise of the argument is just a categorical statement, rather than a claim that a categorical statement is false, draw the Venn diagram for the premise. 

\begin{longtabu}{p{.15\linewidth}p{.75\linewidth}}
\textbf{Example 1}: & All $S$ are $P$. Therefore, it is false that no $S$ are $P$.   \\
\textbf{Answer}: & Ockham: Valid. Contraries.\\ 
& Boole: Invalid \\
%& \noindent \begin{tikzpicture}
%\def\firstcircle{(0,0) circle (.75cm)}
%\def\secondcircle{(0:1cm) circle (.75cm)}
%\begin{scope}[shift={(4cm,0cm)}]
%\begin{scope}[even odd rule]% first circle without the second
%\clip \secondcircle (-1,-1) rectangle (1,1);
%\fill[gray] \firstcircle;
%\end{scope}
%\draw \firstcircle node[outer sep=.66cm, above left] (s) {$S$};
%\draw \secondcircle node[outer sep=.66cm, above right] (p) {$P$};
%\end{scope}
%\end{tikzpicture}
%\\
%& The overlap between $S$ and $P$ is neither shaded out nor has an ``x'' in it, so ``No $S$ are $P$'' could either be true or false. \\
%\end{longtabu}
%\begin{longtabu}{p{.15\linewidth}p{.75\linewidth}}
\textbf{Example 2:} & It is false that all $S$ are $P$. Therefore, some $S$ are not $P$.\\
\textbf{Answer:} & Ockham: Valid. Contradictories \\
& Boole: Valid.
\end{longtabu}


\begin{exercises}

\item Some $S$ are $P$. Therefore all $S$ are $P$.	
\answer{\\ Ockham: Invalid\\\
Boole: Invalid \\
%\begin{venns}
%%\shadeintersectred{\subjectcircle}{\predicatecircle}
%%\shadecomplementred{\subjectcircle}{\subjectsquare}{\predicatecircle}
%%\someexistonesent
%\someexistfoursent
%\drawsubsent 
%\drawpredsent 
%\end{venns}
}	

\item  It is false that some $S$ are $P$. Therefore, it is false that all $S$ are $P$.
\answer{\\ Ockham: Valid. Subalternation.\\
Boole: Invalid. \\
%\begin{venns}
%\shadeintersectred{\subjectcircle}{\predicatecircle}
%\shadecomplementred{\subjectcircle}{\subjectsquare}{\predicatecircle}
%\someexistonesent
%\someexistfoursent
%\drawsubsent 
%\drawpredsent 
%\end{venns}
}	


\item  All $S$ are $P$. Therefore, some $S$ are not $P$.
\answer{\\Ockham: Invalid. \\
Boole: invalid \\ %\vspace{11pt}
%\begin{venns}
%%\shadeintersectred{\subjectcircle}{\predicatecircle}
%\shadecomplementred{\subjectcircle}{\subjectsquare}{\predicatecircle}
%%\someexistonesent
%%\someexistfoursent
%\drawsubsent 
%\drawpredsent 
%\end{venns}\\ \vspace{6pt}
%The shading from the premise means the conclusion is impossible.\\
}	



\item It is false that some $S$ are $P$. Therefore no $S$ are $P$.
\answer{\\ Ockham: Valid. Contradictories\\ 
Boole: Valid \\
%\begin{venns}
%\shadeintersectred{\subjectcircle}{\predicatecircle}
%\shadecomplementred{\subjectcircle}{\subjectsquare}{\predicatecircle}
%\someexistonesent
%\someexistfoursent
%\drawsubsent 
%\drawpredsent 
%\end{venns}
}	


\item No $S$ are $P$. Therefore, it is false that all $S$ are $P$.  
\answer{\\Ockham: Valid. Contraries.\\
Boole: Invalid \vspace{5pt}\\ 
\begin{venns}
\shadeintersectred{\subjectcircle}{\predicatecircle}
%\shadecomplementred{\subjectcircle}{\subjectsquare}{\predicatecircle}
%\someexistonesent
%\someexistfoursent
\drawsubsent 
\drawpredsent 
\end{venns} \vspace{5pt} \\
It is possible that all $S$ are $P$, in the case that no $S$ exist at all. We can see this in by drawing the diagram for the premise. It is still possible to fill in the part of $S$ that is in the complement of $P$. Note that in doing this, we wind up filling in all of $S$. This just emphasizes the fact that ``No $S$ are $P$'' and ``All $S$ are $P$'' are both true only when no $S$ exist at all. 
}	
 


\item Some $S$ are not $P$. Therefore, it is false that all $S$ are $P$. 		
\answer{\\Ockham: Valid. Contradictories\\
Boole: Valid.\vspace{5pt}\\
%\begin{venns}
%%\shadeintersectred{\subjectcircle}{\predicatecircle}
%%\shadecomplementred{\subjectcircle}{\subjectsquare}{\predicatecircle}
%\someexistonesent
%%\someexistfoursent
%\drawsubsent 
%\drawpredsent 
%\end{venns}\vspace{5pt}\\
%You cannot shade in the part of $S$ that is in the complement of $P$, because there is already an X there. 
}	

\item It is false that some $S$ are not $P$. Therefore, it is false that no $S$ are $P$.
\answer{\\Ockham: Valid, subalternation.\\\
Boole: Invalid.\\
}	
%  Note that for Ockham, the premise is never true for empty terms. 

\item It is false that some $S$ are $P$. Therefore, it is false that no $S$ are $P$. 	
\answer{\\Ockham: Invalid \\
Boole: Invalid \\
}	

\item It is false that some $S$ are $P$. Therefore, some $S$ are not $P$.		
\answer{\\Ockham: Valid. Subcontraries \\
Boole: Invalid. Both statements could be false if there are no $S$. \\
}	

\item Some $S$ are $P$. Therefore, it is false that some $S$ are not $P$  
\answer{\\Ockham: Invalid \\
Boole: Invalid \\
%\begin{venns}
%%\shadeintersectred{\subjectcircle}{\predicatecircle}
%%\shadecomplementred{\subjectcircle}{\subjectsquare}{\predicatecircle}
%%\someexistonesent
%\someexistfoursent
%\drawsubsent 
%\drawpredsent 
%\end{venns}\\ Just because there is an x in the overlap between $S$ and $P$ doesn't mean there isn't also an x in the part of $S$ that is in the complement of $P$.
}	



\end{exercises}

\problempart See the instructions for Part A. 
\begin{exercises}
\item Some $S$ are $P$. Therefore, it is false that no $S$ are $P$. \answer{\\ valid,\\ valid.} 
\item It is false that some $S$ are not $P$. Therefore, some $S$ are $P$		\answer{\\ Valid,\\invalid,}
\item Some $S$ are not $P$. Therefore, it is false that some $S$ are $P$   \answer{\\ invalid,\\ invalid}
\item Some $S$ are not $P$. Therefore, it is false that all $S$ are $P$	\answer{\\ Valid,\\ Valid}
\item All $S$ are $P$. Therefore, some $S$ are $P$    \answer{\\ Ockham: valid,\\ Boole: Invalid.}
\item Some $S$ are not $P$. Therefore, no $S$ are $P$		\answer{\\ invalid,\\ invalid. }
\item Some $S$ are $P$. Therefore, it is false that no $S$ are $P$ \answer{\\ invalid, \\invalid. }
\item It is false that all $S$ are $P$. Therefore, it is false that no $S$ are $P$  \answer{\\ invalid \\invalid }
\item No $S$ are $P$. Therefore some $S$ are not $P$    \answer{\\ valid, \\invalid.\\ Notice that the argument is valid for empty terms under Ockham's interpretation because the premise is never true. }
\item It is false that some $S$ are $P$. Therefore, no $S$ are $P$.			\answer{\\  valid,\\ valid,}
\end{exercises}

%%%%%%%%%%%%%%%%%%%%%%%%%%% A to E
% All $S$ are $P$ \\
% Therefore, no $S$ are $P$   %Invalid, invalid

% It is false that all $S$ are $P$ \\
% Therefore, no $S$ are $P$    %invalid invalid

% All $S$ are $P$ \\
% Therefore, it is false that no $S$ are $P$  % valid, invalid     				 example

% It is false that all $S$ are $P$ \\
% Therefore, it is false that no $S$ are $P$  %invalid invalid 

% one different 

%%%%%%%%%%%%%%%%%%%%%%%%%%% A to O

% All $S$ are $P$ \\
%	Therefore, some $S$ are not $P$		% invalid, invalid

%  always match

%%%%%%%%%%%%%%%%%%%%%%%%%%% A to I
% All $S$ are $P$ \\
% Therefore, some $S$ are $P$    % Ockham: valid, Boole: Invalid.			B

% It is false that all $S$ are $P$ \\
% Therefore, some $S$ are $P$    % invalid, invalid.

% All $S$ are $P$ \\
% Therefore, it is false that some $S$ are $P$   % invalid, invalid

% It is false that all $S$ are $P$ \\
% Therefore, it is false that some $S$ are $P$   % invalid, invalid. 

% one different 

%%%%%%%%%%%%%%%%%%%%%%%%%%% % E to A 
% No $S$ are $P$ \\
% Therefore, all $S$ are $P$   % invalid, invalid

%No $S$ are $P$ \\
% Therefore, it is false that all $S$ are $P$  %valid invalid. 				A

% one different

%%%%%%%%%%%%%%%%%%%%%%%%%%% % E to O

% No $S$ are $P$ \\
% Therefore some $S$ are not $P$    %valid, invalid. Notice that the argument is valid for empty terms under Ockham's interpretation because the premise is never true. 								B
% one different 

%%%%%%%%%%%%%%%%%%%%%%%%%%% % E to I 

% Some $S$ are $P$
% Therefore, it is false that no $S$ are $P$	%valid, valid

% It is false that some $S$ are $P$
% Therefore, no $S$ are $P$.			% valid, valid,

%Always match

%%%%%%%%%%%%%%%%%%%%%%%%%%% % O to I

% Some $S$ are not $P$
%	Therefore, no $S$ are $P$		%invalid, invalid. 

% It is false that some $S$ are not $P$
%	Therefore, it is false that no $S$ are $P$.		%Valid, invalid. Note that for Ockham, the premise is never true for empty terms. 				A

% one different 

%%%%%%%%%%%%%%%%%%%%%%%%%%% % O to A  

% Some $S$ are not $P$
%	Therefore, it is false that all $S$ are $P$	Valid, Valid

%Always match

%%%%%%%%%%%%%%%%%%%%%%%%%%% % O to E

% It is false that some $S$ are not $P$ \\
%	Therefore, some $S$ are $P$		%Valid, invalid, 			B

% Some $S$ are not $P$
%	Therefore, it is false that some $S$ are $P$   %invalid, invalid

% one different 

%%%%%%%%%%%%%%%%%%%%%%%%%%% % I to A

% Some $S$ are $P$
%	Therefore all $S$ are $P$.	% invalid invalid.

% It is false that some $S$ are $P$.   \\
%	Therefore it is false that all $S$ are $P$	%Valid, Invalid		A

% one different 

%%%%%%%%%%%%%%%%%%%%%%%%%%% % I to E

% Some $S$ are $P$
%	Therefore, it is false that no $S$ are $P$. %valid, valid

% It is false that some $S$ are $P$ \\
%	Therefore no $S$ are $P$.		% valid, valid


%It is false that some $S$ are $P$.
%	Therefore, it is false that no $S$ are $P$. 	%invalid, invalid. 

% Always match

%%%%%%%%%%%%%%%%%%%%%%%%%%% % I to O 

% It is false that some $S$ are $P$. \\
%	Therefore, some $S$ are not $P$.		%valid, invalid.		A

%  Some $S$ are $P$
%	It is false that some $S$ are not $P$

% One different, as above. 

\problempart
\begin{exercises}\item On page \pageref{proving_trad_square} above we started to sketch a proof that the traditional square of opposition works properly under Ockham's understanding of existential import. Finish the proof by showing that subalternation works properly for statements about nonexistent objects.
\item Use Venn diagrams to show that in the modern square of opposition, contradictory statements work as they did in the traditional square, but no other relationships does.  
\item Use Venn diagrams to show whether the arguments in part A are valid or invalid.
\item Use Venn diagrams to show whether the arguments in part B are valid or invalid.
\end{exercises}


%\setglossarysection{section}
%\printglossary[type=catstatements]

\section*{Key Terms}
\begin{multicols}{2}
\begin{sortedlist}
\sortitem{Quantified categorical statement}{}
\sortitem{Quantifier}{}
\sortitem{Subject class}{}
\sortitem{Predicate class}{}
\sortitem{Copula}{}
\sortitem{Truth value}{}
\sortitem{Mood-A statement}{}
\sortitem{Mood-E statement}{}
\sortitem{Mood-I statement}{}
\sortitem{Mood-O statement}{}
\sortitem{Quantity}{}
\sortitem{Quality}{}
\sortitem{Logically structured English}{}
\sortitem{Distribution}{}
\sortitem{Complement}{}
\sortitem{Converse}{}
\sortitem{Obverse}{}
\sortitem{Contraposition}{}
\sortitem{Contradictories}{}
\sortitem{Contraries}{}
\sortitem{Square of opposition}{}
\sortitem{Subcontraries}{}
\sortitem{Subalternation}{}
\sortitem{Existential import}{}
\sortitem{Vacuous truth}{}
\sortitem{Venn diagram}{}
\sortitem{Standard form categorical statement}{}
\sortitem{Universal}{}
\sortitem{Particular}{}
\sortitem{Affirmative}{}
\sortitem{Negative}{}
\end{sortedlist}
\end{multicols}

%\chapter{Categorical Syllogisms}
\label{chap:cat_syllogisms}
\markright{Ch. \ref{chap:cat_syllogisms}: Categorical Syllogisms}
\setlength{\parindent}{1em}

%\label{whole_syl_chap} %uncomment and typeset twice to print the whole chapter.

% ********************************
% * Standard Form, Mood and Figure  *
% ********************************

\section{Standard Form, Mood, and Figure}
\label{sec:form_mood_figure}
\newglossaryentry{categorical syllogism}
{
name=categorical syllogism,
description={An argument with two premises composed of categorical statements.}
}


So far we have just been looking at very short arguments using categorical statements. The arguments just had one premise and a conclusion that was often logically equivalent to the premise. For most of the history of logic in the West, however, the focus has been on arguments that are a step more complicated called \textsc{\glspl{categorical syllogism}}. A categorical syllogism is a two-premise argument composed of categorical statements. Aristotle began the study of this kind of argument in his book the \cite*{Aristotle:prior}. This work was refined over the centuries by many thinkers in the Pagan, Christian, Jewish, and Islamic traditions until it reached the form it is in today.

\newglossaryentry{Aristotelian syllogism}
{
name=Aristotelian syllogism,
description={An argument where each statement is in one of the moods A, E, I, or O, and which has exactly three terms, arranged so that any two pairs of statements will share one term.}
}


There are actually all kinds of two-premise arguments using categorical statements, but Aristotle only looked at arguments where each statement is in one of the moods A, E, I, or O. The arguments also had to have exactly three terms, arranged so that any two pairs of statements will share one term. Let's call a categorical syllogism that fits this more narrow description an \textsc{\gls{Aristotelian syllogism}} Here is a typical Aristotelian syllogism using only mood-A sentences:

\begin{earg}
\item[P$_1$:] All mammals are vertebrates.
\item[P$_2$:] All dogs are mammals.
\vspace{-.5em}
\item [] \rule{0.3\linewidth}{.5pt} 
\item[C:] All dogs are vertebrates. 
\end{earg} 
\label{AAA_arg}

\newglossaryentry{major term}
{
name=major term,
description={The term that is used as the predicate of the conclusion of an Aristotelian syllogism.}
}

\newglossaryentry{minor term}
{
name=minor term,
description={The term that is used as the subject of the conclusion of an Aristotelian syllogism.}
}

\newglossaryentry{middle term}
{
name=middle term,
description={The one term in an Aristotelian syllogism that does not appear in the conclusion.}
}

\newglossaryentry{major premise}
{
name=major premise,
description={The one premise in an Aristotelian syllogism that names the major term.}
}

\newglossaryentry{minor premise}
{
name=minor premise,
description={The one premise in an Aristotelian syllogism that names the minor term.}
}

Notice how the statements in this argument overlap each other. Each statement shares a term with the other two. Premise 2 shares its subject term with the conclusion and its predicate with Premise 1. Thus there are only three terms spread across the three statements. Aristotle dubbed these the major, middle, and minor premises, but there was initially some confusion about how to define them. In the 6th century, the Christian philosopher John Philoponus, drawing on the work of his pagan teacher Ammonius, decided to arbitrarily designate the \textsc{\gls{major term}} as the predicate of the conclusion, the \textsc{\gls{minor term}} as the subject of the conclusion, and the \textsc{\gls{middle term}} as the one term of the Aristotelian syllogism that does not appear in the conclusion. So in the argument above, the major term is ``vertebrate,'' the middle term is ``mammal,'' and the minor term is ``dog.'' We can also define the \textsc{\gls{major premise}} as the one premise in an Aristotelian syllogism that names the major term, and the \textsc{\gls{minor premise}} as the one premise that names the minor term. So in the argument above, Premise 1 is the major premise and Premise 2 is the minor premise. 

\newglossaryentry{standard form for an Aristotelian syllogism}
{
name=standard form for an Aristotelian syllogism,
description={An Aristotelian syllogism that has been put into logically structured English with the following criteria: (1) all of the individual statements are in standard form, (2) each instance of a term is in the same format and is used in the same sense, and (3) the major premise appears first, followed by the minor premise, and then the conclusion.}
} 

With these definitions in place, we can now define the \textsc{\gls{standard form for an Aristotelian syllogism}} in logically structured English. \label{standard_form_for_an_Aristotelian_syllogism} Recall that in Section  \ref{sec:QQDVD}, we started standardizing our language into something we called ``logically structured English'' in order to remove ambiguity and to make its logical structure clear. The first step was to define the standard form for a categorical statement, which we did on page \pageref{def:standard_form_cat_statement}. Now we do the same thing for an Aristotelian syllogism. We say that an Aristotelian syllogism is in standard form for logically structured English if and only if these criteria have been met: (1) all of the individual statements are in standard form, (2) each instance of a term is in the same format and is used in the same sense, (3) the major premise appears first, followed by the minor premise, and then the conclusion.

\newglossaryentry{syllogism mood}
{
name=syllogism mood,
description={The classification of an Aristotelian syllogism based on the moods of statements it contains. The mood is designated simply by listing the three letters for the moods of the statements in the argument, such as AAA, EAE, AII, etc.}
} 

Once we standardize things this way, we can actually catalog every possible form of an Aristotelian syllogism. To begin with, each of the three statements can take one of four forms: A, E, I, or O. This gives us $4 \times 4 \times 4,$ or 64 possibilities. These 64 possibilities are called the \textsc{\gls{syllogism mood}}, and we designate it just by writing the three letters of the moods of the statements that make it up. So the mood of the argument on page \pageref{AAA_arg} is simply AAA. 

In addition to varying the kind of statements we use in an Aristotelian syllogism, we can also vary the placement of the major, middle, and minor terms. There are four ways we can arrange them that fit the definition of an Aristotelian syllogism in standard form, shown in Table \ref{tab:four_figures}. Here $P$ stands for the major term, $S$ for the minor term, and $M$ for the middle. The thing to pay attention to is the placement of the middle terms. In figure 1, the middle terms form a line slanting down to the right. In figure 2, the middle terms are both pushed over to the right. In figure 3, they are pushed to the left, and in figure 4, they slant in the opposite direction from figure 1.


\begin{table}
\begin{mdframed}[style=mytablebox]
\begin{tabu}{p{.5\linewidth}p{.5\linewidth}}


\begin{earg}
\item[P$_1$:]  \textbf{M} \hspace{1em} $P$
\item[P$_2$:] $S$ \hspace{1em} \textbf{M}
\vspace{-.5em}
\item [] \rule{0.4\linewidth}{.5pt} 
\item[C:] \hspace{.3ex} $S$ \hspace{1em} $P$
\end{earg} 

&

\begin{earg}
\item[P$_1$:] $P$ \hspace{1em} \textbf{M}
\item[P$_2$:] $S$ \hspace{1em} \textbf{M}
\vspace{-.5em}
\item [] \rule{0.4\linewidth}{.5pt} 
\item[C:]  \hspace{.3ex} $S$ \hspace{1em} $P$
\end{earg} 

\\

\hspace{2em} Figure 1 

&

\hspace{2em} Figure 2

\\

\begin{earg}
\item[P$_1$:] \textbf{M} \hspace{1em} $P$
\item[P$_2$:] \textbf{M} \hspace{1em} $S$
\vspace{-.5em}
\item [] \rule{0.4\linewidth}{.5pt} 
\item[C:]  \hspace{.3ex}$S$ \hspace{1em} $P$
\end{earg} 

&

\begin{earg}
\item[P$_1$:] $P$ \hspace{1em} \textbf{M}
\item[P$_2$:] \textbf{M} \hspace{1em} $S$
\vspace{-.5em}
\item [] \rule{0.4\linewidth}{.5pt} 
\item[C:] \hspace{.3ex} $S$ \hspace{1em} $P$
\end{earg} 

\\

\hspace{2em} Figure 3

&

\hspace{2em} Figure 4

\end{tabu}
\end{mdframed}
\caption{The four figures of the Aristotelian syllogism}
\label{tab:four_figures}
\end{table}

The combination of 64 moods and 4 figures gives us a total of 256 possible Aristotelian syllogisms. We can name them by simply giving their mood and figure. So this is OAO-3:

\begin{earg}
\item[P$_1$:] Some $M$ are not $P$.
\item[P$_2$:] All $M$ are $S$.
\vspace{-.5em}
\item [] \rule{0.2\linewidth}{.5pt} 
\item[C:] Some $S$ are not $P$.
\end{earg} 

Syllogism OAO-3 is a valid argument. We will be able to prove this with Venn diagrams in the next section. For now just read it over and try to see intuitively why it is valid. Most of the 256 possible syllogisms, however, are not valid.  In fact, most of them, like IIE-2, are quite obviously invalid:

\begin{earg}
\item[P$_1$:] Some $P$ are $M.$
\item[P$_2$:] Some $S$ are $M.$
\vspace{-.5em}
\item [] \rule{0.2\linewidth}{.5pt} 
\item[C:] No $S$ are $P.$
\end{earg} 

Given an Aristotelian syllogism in ordinary English, we can transform it into standard form in logically structured English and identify its mood and figure. Consider the following: 

\begin{quotation}
\noindent No geckos are cats. I know this because all geckos are lizards, but cats aren't lizards.
\end{quotation}

The first step is to identify the conclusion, using the basic skills you acquired back in Chapter \ref{Chap:what_is_logic}. In this case, you can see that ``because'' is a premise indicator word, so the statement before it, ``No geckos are cats,'' must be the conclusion.

Step two is to identify the major, middle, and minor terms. Remember that the major term is the predicate of the conclusion, and the minor term is the subject. So here the major term is ``cats,'' the minor term is ``geckos.'' The leftover term, ``lizards,'' must then be the middle term. 

We show that we have identified the major, middle, and minor terms by writing a \gls{translation key}\label{def:translation_key}. As we saw on page \pageref{def:translation_key} a translation key is just a list that assigns English phrases or sentences to variable names. For categorical syllogism, this means matching the English phrases for the terms with the variables $S$, $M$, and $P$. 

\begin{ekey}
\item[$S$:] Geckos
\item[$M$:] Lizards
\item[$P$:] Cats
\end{ekey}

Step three is to write the argument in canonical form using variables for the terms. The last statement, ``cats aren't lizards,'' is the major premise, because it has the major term in it. We need to change it to standard form, however, before we substitute in the variables. So first we change it to ``No cats are lizards.'' Then we write ``No $S$ are $M$.'' For the minor premise and the conclusion we can just substitute in the variables, so we get this:
 
\begin{earg}
\item[P$_1$:] No $P$ are $M$.
\item[P$_2$:] All $S$ are $M$.
\vspace{-.5em}
\item [] \rule{0.15\linewidth}{.5pt} 
\item[C:] No $S$ are $P$. 
\end{earg} 

Step four is to identify mood and figure. We can see that this is figure 2, because the middle term is in the predicate of both premises. Looking at the form of the sentences tells us that this is EAE. 

%%%%%%%%%% Practice Problems %%%%%

\practiceproblems
\noindent\problempart Put the following English arguments into standard form in logically structured English using variables for the terms. Be sure to include a translation key. Then identify mood and figure of the argument.
\begin{longtabu}{{X[1,l,p]X[9,l,p]}} 
\textbf{Example}: &  No magical creatures are rainbow colored. Therefore, some unicorns are not rainbow colored, because all unicorns are magical creatures.
\end{longtabu}
\vspace{-22pt}
\begin{longtabu}{p{.05\linewidth}p{.45\linewidth}p{.3\linewidth}p{.1\linewidth}} 
  
\textbf{Answer}: & 
\vspace{-16pt}
\begin{ekey}
\item[$S$:] Unicorns
\item[$M$:] Magical Creatures
\item[$P$:] Things that are rainbow colored
\end{ekey}

&
\vspace{-16pt}
\begin{earg}
\item[P$_1$:] No $M$ are $P$.
\item[P$_2$:] All $S$ are $M$.
\vspace{-.5em}
\item [] \rule{0.5\linewidth}{.5pt} 
\item[C:] Some $S$ are not $P$.
\end{earg} 

&
%\vspace{-16pt}
EAO-1 

\end{longtabu} 

\begin{exercises} 
 
\item  All bunk beds are tall, and some beds are bunk beds. So, some beds are tall.  

\answer{
\begin{longtabu}{X[2,l,p]X[2,l,p]X[1,l,p]} 
\vspace{-16pt}
\begin{ekey}
\item[$S$:] Beds
\item[$M$:] Bunk beds.
\item[$P$:] Things that are tall
\end{ekey}

&
\vspace{-16pt}
\begin{earg*}
\item All $M$ are $P$ 
\item Some $S$ are $M$ 
\itemc Some $S$ are $P$
\end{earg*}

&

AII-1

\end{longtabu}

}


\item No fluffy things are tusked pachyderms, but all tusked pachyderms are elephants. Therefore some elephants are not fluffy.

\answer{
\begin{longtabu}{X[2,l,p]X[2,l,p]X[1,l,p]} 
\vspace{-16pt}
\begin{ekey}
\item[$S$:] Elephants 
\item[$M$:]  Tusked pachyderms
\item[$P$:]  Things that are fluffy
\end{ekey}

&
\vspace{-16pt}
\begin{earg*}
\item  No $P$ are $M$
\item  All $M$ are $S$
\itemc Some $S$ are not $P$
\end{earg*}

&

EAO-4. 
\end{longtabu}
}
 
\item Some strangers are not dangerous. After all, nothing that is dangerous is also a kitten. But some kittens are strangers. 

\answer{
Notice this is conclusion-first.
\begin{longtabu}{X[2,l,p]X[2,l,p]X[1,l,p]} 
\vspace{-16pt}
\begin{ekey}
\item[$S$:]  Strangers
\item[$M$:]  Kittens
\item[$P$:] Dangerous things
\end{ekey}

&
\vspace{-16pt}
\begin{earg*}
\item No $P$ are $M$ 
\item Some $M$ are $S$ 
\itemc Some $S$ are not $P$ 
\end{earg*}

&

EIO-4 
\end{longtabu}
} 
\item Some giant monsters are not things to be trifled with. This is because all kaiju are giant monsters and no kaiju is a creature to be trifled with.
\answer{
Notice that the major premise is given last here.

\begin{longtabu}{X[2,l,p]X[2,l,p]X[1,l,p]} 
\vspace{-16pt}
\begin{ekey}
\item[$S$:]  Giant monsters
\item[$M$:]  Kaiju
\item[$P$:]  Things to be trifled with
\end{ekey}

&
\vspace{-16pt}
\begin{earg*}
\item  No $M$ are $P$
\item  All $M$ are $S$
\itemc Some $S$ are not $P$ 
\end{earg*}

&

EAO-3 
\end{longtabu}
}

\item All parties are celebrations, because no celebrations are unhappy and no parties are unhappy.
 

\answer{
\begin{longtabu}{X[2,l,p]X[2,l,p]X[1,l,p]} 
\vspace{-16pt}
\begin{ekey}
\item[$S$:]  Parties
\item[$M$:]  Unhappy things
\item[$P$:]  Celebrations
\end{ekey}

&
\vspace{-16pt}
\begin{earg*}
\item  No $P$ are $M$
\item  No $S$ are $M$
\itemc  All $S$ are $P$
\end{earg*}

&

EEA-2. 
\end{longtabu}
}
   
    
\item Nothing that is deadly is safe. Therefore, some snakes are not deadly, because some snakes are safe.

\answer{
\begin{longtabu}{X[2,l,p]X[2,l,p]X[1,l,p]} 
\vspace{-16pt}
\begin{ekey}
\item[$S$:]  Snakes
\item[$M$:]  Safe things.
\item[$P$:]  Deadly things
\end{ekey}

&
\vspace{-16pt}
\begin{earg*}
\item  No $P$ are $M$
\item  Some $S$ are $M$
\itemc  Some $S$ are not $P$
\end{earg*}

&

EIO-2. 
\end{longtabu}
}

\item Godzilla is a character in a movie by the Toho company. This is because Godzilla is a kaiju, and some kaiju are Toho characters.         

\answer{
\begin{longtabu}{X[3,l,p]X[2,l,p]X[1,l,p]} 
\vspace{-16pt}
\begin{ekey}
\item[$S$:] Things identical to Godzilla 
\item[$M$:]  Kaiju
\item[$P$:]  Toho characters 
\end{ekey}

&
\vspace{-16pt}
\begin{earg*}
\item  Some $M$ are $P$
\item  All $S$ are $M$
\itemc All $S$ are $P$  
\end{earg*}

&

IAA-1. 
\end{longtabu}
}
                        
                                                                        
\item No tyrannosaurs are birds. Therefore some pteranodons are not tyrannosaurs, because no pteranodons are birds.  

\answer{
\begin{longtabu}{X[2,l,p]X[2,l,p]X[1,l,p]} 
\vspace{-16pt}
\begin{ekey}
\item[$S$:]  Pteranodons
\item[$M$:]  Birds
\item[$P$:]  Tyrannosaurs
\end{ekey}

&
\vspace{-16pt}
\begin{earg*}
\item No $P$ are $M$ 
\item  No $S$ are $M$
\itemc Some $S$ are not $P$ 
\end{earg*}

&

EEO-2 
\end{longtabu}
}

\item Every carnivorous animal is a college professor. Therefore all logicians are carnivores, because some college professors are not logicians.
 

\answer{
\begin{longtabu}{X[2,l,p]X[2,l,p]X[1,l,p]} 
\vspace{-16pt}
\begin{ekey}
\item[$S$:] Logicians 
\item[$M$:] College professors 
\item[$P$:]  Carnivores
\end{ekey}

&
\vspace{-16pt}
\begin{earg*}
\item  All $P$ are $M$
\item  Some $M$ are not $S$
\itemc All $S$ are $P$  
\end{earg*}

&

AOA-4 
\end{longtabu}
}
\item Some balderdash is chicanery. Therefore no hogwash is chicanery, because all hogwash is balderdash.  


\answer{
\begin{longtabu}{X[2,l,p]X[2,l,p]X[1,l,p]} 
\vspace{-16pt}
\begin{ekey}
\item[$S$:] Hogwash 
\item[$M$:] Balderdash 
\item[$P$:]  Chicanery
\end{ekey}

&
\vspace{-16pt}
\begin{earg*}
\item Some $M$ are $P$
\item All $S$ are $M$ 
\itemc No $S$ are $P$ 
\end{earg*}

&

IAE-1. 
\end{longtabu}
}
   
         
\end{exercises} 

\noindent\problempart Put the following English arguments into standard form in logically structured English using variables for the terms. Be sure to include a translation key. Then identify mood and figure of the argument.
 

\begin{exercises} 

\item No chairs are tables. But all tables are furniture. So some furniture are not chairs.

% P$_1$:  No $P$ are $M$
% P$_2$: All $M$ are $S$
%C: Some $S$ are not $P$

% EAO-1V Valid 
  
 
\item All dogs bark, and some things that bark are annoying. Therefore some dogs are annoying.

%P$_1$: Some things that bark are annoying things
%P$_2$: All dogs are things that bark
%C: Some dogs are annoying things
%IAI-I 
 
\item Some superheroes are not arrogant. This is because anyone who is arrogant is unpopular. But no superheroes are unpopular.

%P$_1$: All arrogant people are people who are unpopular
%P$_2$: No superheros are people who are unpopular
%C: Some superheros are not arrogant
% AEO-II Valid 
 
 
\item Some mornings are not free time. But no evenings are mornings. Therefore no evenings are free time.

%P$_1$: Some $M$ are not $P$
%P$_2$: No $S$ are $M$
%C: No $S$ are $P$
%OEE-I 
 
\item All veterinarians are doctors. Therefore some veterinarians are well trained, because all doctors are well trained.

%Valid AAI-I

\item No books are valueless. Therefore some books are not nonsense, because all nonsense is valueless.

%No $S$ are $M$
%All $P$ are $M$
%Some $S$ are not $P$


 \item No battleships are brightly colored, because no brightly colored things are lizards, and no battleships are lizards.
%P$_1$: No $P$ are $M$
%P$_2$: No $S$ are $M$
%C: No battleships are brightly colored. 
%EEE-II 

\item No octagons are curvilinear, because all circles are curvilinear and some octagons are circles.

%P$_1$: All $M$ are $P$
%P$_2$: Some $S$ are $M$
%C: No $S$ are $P$
%AIE-III 

   \item Some eggs do not come from chickens. You can tell, because no milk comes from chickens, but all eggs are milk.

%P$_1$: No $M$ are $P$
%P$_2$: All $S$ are $M$
%C: Some $S$ are not $P$
%EAO-1II 

\item Some ichthyosaurs are not eoraptors. Therefore some ichthyosaurs are mixosauruses, because some eoraptors are not mixosauruses. 
    
%P$_1$: All $M$ are $P$
%P$_2$: Some $S$ are not $M$
%C: Some $S$ are $P$
%AOI-III Invalid Conclusion Middle 
\end{exercises}

\noindent\problempart Put the following English arguments into standard form in logically structured English using variables for the terms. Be sure to include a translation key. Then identify mood and figure of the argument. Problems in this section are a little trickier than problems in Part A and Part B. 


\begin{exercises} 
\item  All spiders make thread, and anything that makes thread makes webs.  So for sure, all spiders make webs.

\item  Some children are not afraid to explore. For no one afraid to explore suffers from abandonment issues, and some children don't suffer from abandonment issues.


%Solution 
% C = Things that are children
% E = Things that are (people and are) afraid to explore
% S = Things that suffer from abandonment issues
%
% 1. No E are S.
% 2. Some C are not S.
%      --------------------
% 3. Some C are not E. 

% \includegraphics*[width=2.10in, height=1.73in, keepaspectratio=false]{img/image16}
% Invalid
 
\item  Every professional baseball player is a professional athlete, and no professional athlete is poor. No professional baseball player, thus, is poor.


\item  No horse contracts scrapie. So, because some animals contracting scrapie lose weight, there are horses that do not lose weight.

% H = Things that are horses.
% S = Things that are (animals and) contract scrapie.
% L = Things that lose weight.
%
% 1. No H are S.
% 2. Some S are L.
%----------------
% 3. Some H are not L.

% \includegraphics*[width=2.10in, height=1.73in, keepaspectratio=false]{img/image17}

%invalid 

\item  Since everyone in this room is enrolled in logic, and since everyone at the college is enrolled in logic, everyone in this room is attending the college.

\item   All arguments are attempts to convince, and some attempts to convince are denials of autonomy. Therefore, some arguments are denials of autonomy.

\item  No one who likes smoked eel is completely reliable. For, everyone who likes smoked eel is a person with odd characteristics, and no one with odd characteristics is completely reliable.

\item Breaking an addiction requires self-control, and nothing requiring self-control is easy. Thus, breaking an addiction is never easy.
 
\item Jack is an American soldier in Iraq, and some American soldiers in Iraq are unable to sleep much. Hence, Jack is unable to sleep much.

\item All smurfs are blue, and no smurfs are tall. Therefore some tall things are not blue.

%P$_1$: All $M$ are $P$
%P$_2$: No $M$ are $S$
%C: Some $S$ are not $P$
% AEO-III Invalid 


\end{exercises}

 \noindent\problempart Put the following English arguments into standard form in logically structured English using variables for the terms. Be sure to include a translation key. Then identify mood and figure of the argument. Problems in this section are a little trickier than problems in Part A and Part B. 


\begin{exercises}
\item All Old World monkeys are primates. Some Old World monkeys are baboons. Therefore some primates are baboons

% All $M$ are $P$
% Some $M$ are $S$
% Some $S$ are $P$.
% Datisi (AII-3)
 
\item All gardeners are schnarf. And all extraterrestrials are gardeners. Therefore all extraterrestrials are schnarf. 

% Barbara (AAA-I)
 
\item No corn chips are potato chips, but all corn chips are snacks, so no snacks are potato chips. 

%P$_1$: No $M$ are $P$
%P$_2$: All $M$ are $S$
%C: No $S$ are $P$
%EAE-III Invalid 
 
\item Everything in the attic is old and musty. Moreover, some pieces of furniture are old and musty. So, necessarily, some pieces of furniture are in the attic.

%$S$: Things that are pieces of furniture \newline
%					$M$: Things that are old and musty \newline
%					$P$: Things that are in the attic \newline 

 
\item Some offices are pleasant places to work. All friendly places to work are workplaces. Therefore some workplaces are offices. 
 
% Some $P$ are $M$ 
% All $M$ are $S$. 
% Therefore some $S$ are $P$.
%Dimatis (IAI-IV)
 
% S: Supertaxon
% M: Taxon
% P: Overlapping taxon

\item Some rabbits are not white, but some snowdrifts are white. Therefore some snowdrifts are not rabbits.  

%P$_1$: Some $P$ are not $M$
%P$_2$: Some $S$ are $M$
%C: Some $S$ are not $P$
%OIO-II Invalid Conclusion Middle 
 
\item No airplanes are submarines, and some submarines are u-boats, so some airplanes are not u-boats.

%No $P$ are $M$ and some $M$ are $S$, so some $S$ are not $P$.
% No $P$ are $M$ 
% Some $M$ are $S$
% So some $S$ are not $P$.
% S: Subtaxon of M. 
% M: Disjoint taxon
% P: Taxon
%Fresison (EIO-IV) 
 
\item All rules have exceptions, but no commands from God have exceptions. So no rules are commands from God. 

% All $P$ are $M$
% No $S$ are $M$
% No $S$ are $P$. 
% Camestres (AEE-II) 
 
\item All spies are liars, and some liars are not platypuses. Therefore some platypuses are spies.

%P$_1$: All $P$ are $M$
%	 P$_2$: Some $M$ are not $S$
%	 C: Some $S$ are $P$
%AOI-IV Invalid 
 
\item Some bacteria are not harmful, and some harmful things are lions. Therefore some bacteria are lions. 
%
%P$_1$: Some $P$ are not $M$
%	 P$_2$: Some $M$ are $S$
%	 C: Some $S$ are $P$
%OII-IV Invalid 
     
\end{exercises} 
 
\noindent\problempart Given the mood and figure, write out the full syllogism, using the term variables $S$, $M$, and $P$. 
\begin{longtabu}{p{.1\linewidth}p{.9\linewidth}} 
\textbf{Example}: &IAA-2 \\ 
\textbf{Answer}: & P$_1$: Some $P$ are $M$. \newline
P$_2$: All $S$ are $M$. 
\vskip -6pt
\rule{0.2\linewidth}{.5pt} \newline
C: All $S$ are $P$.
\end{longtabu} 

\begin{exercises} 
\begin{longtabu}{X[1]X[1]}
\item EEE-4 
\answer{
\begin{earg*}
\item No $P$ are $M$
\item No $M$ are $S$
\itemc No $S$ are $P$
\end{earg*}
}

&

\item EIE-1 
\answer{
\begin{earg*}
\item No $M$ are $P$
\item Some $S$ are $M$
\itemc No $S$ are $P$
\end{earg*}
}

\\[-15pt]

\item AII-1 
\answer{
\begin{earg*}
\item All $M$ are $P$
\item Some $S$ are $M$
\itemc Some $S$ are $P$
\end{earg*}
}

&

\item IIA-4 
\answer{
\begin{earg*}
\item Some $P$ are $M$
\item Some $M$ are $S$
\itemc All $S$ are $P$
\end{earg*}
}

\\[-15pt]

\item IOO-2 
\answer{
\begin{earg*}
\item Some $P$ are $M$
\item Some $S$ are not $M$
\itemc Some $S$ are not $P$
\end{earg*}
}

&

\item OEI-4 
\answer{
\begin{earg*}
\item Some $P$ are not $M$.
\item  No $M$ are $S$. 
\itemc Some $S$ are $P$.
\end{earg*}
}

\\[-15pt]

\item IIO-2 
\answer{
\begin{earg*}
\item Some $P$ are not $M$
\item Some $S$ are not $M$
\itemc Some $S$ are not $P$
\end{earg*}
}

&

\item OAI-1 
\answer{
\begin{earg*}
\item Some $M$ are $P$ 
\item All $S$ are $M$.
\itemc Some $S$ are $P$
\end{earg*}
}

\\[-15pt]

\item AAA-2 
\answer{
\begin{earg*}
\item All $P$ are $M$
\item All $S$ are $M$
\itemc All $S$ are $P$
\end{earg*}
}

&

\item IAA-3 
\answer{
\begin{earg*}
\item Some $M$ are $P$
\item All $M$ are $S$ 
\itemc All $S$ are $P$
\end{earg*}
}

\end{longtabu}
\end{exercises}

\noindent\problempart Given the mood and figure, write out the full syllogism, using the term variables $S$, $M$, and $P$ 


\begin{exercises} 
\begin{longtabu}{X[1]X[1]}
\item EIO-1  

&

\item AAI-4  

\\[-15pt]

\item IIO-4 

  &

\item AEA-4 

\\[-15pt]

\item AOE-3 

&

\item IAO-4 

\\[-15pt]

\item OAI-3 

&

\item IOE-2 

\\[-15pt]

\item IAE-2

  &

\item EAO-2 

\\[-15pt]

\end{longtabu}
\end{exercises}  

 % ********************* 
 % *     Testing Validity   *
 % *********************
            
\section{Testing Validity}
\label{sec:testing_validity}
We have seen that there are 256 possible categorical arguments that fit Aristotle's requirements. Most of them are not valid, and as you probably saw in the exercises, many don't even make sense. In this section, we will learn to use Venn diagrams to sort the good arguments from the bad. The method we will use will simply be an extension of what we did in the last chapter, except with three circles instead of two. 

\subsection{Venn Diagrams for Single Propositions}
In the previous chapter, we drew Venn diagrams with two circles for arguments that had had two terms. The circles partially overlapped, giving us four areas, each of which represented a way an individual could relate to the two classes. So area 1 represented things that were $S$ but not $P$, etc. 

\begin{center}
\begin{tikzpicture}
\def\firstcircle{(0,0) circle (1cm)}
\def\secondcircle{(0:1.33cm) circle (1cm)}
\draw \firstcircle node[outer sep=.75cm, above left] (s) {$S$} 
	node [xshift=-.25cm] (1) {1}
	node [xshift=.66cm] (2){2};
\draw \secondcircle node[outer sep=.75cm, above right] (p) {$P$}
	node [xshift=.25cm] (3) {3}
	node [xshift=1.4cm] (4){4};
\end{tikzpicture}
\end{center}

Now that we are considering arguments with three terms, we will need to draw three circles, and they need to overlap in a way that will let us represent the eight possible ways an individual can be inside or outside these three classes.

\begin{center}
\begin{tikzpicture}
\def\firstcircle{(0,0) circle (1.25cm)}
\def\secondcircle{(60:1.5cm) circle (1.25cm)}
\def\thirdcircle{(0:1.5cm) circle (1.25cm)}

\draw \firstcircle node[outer sep=1cm, below left] {$S$}
	node [xshift=-.25cm, yshift=-.25cm](1) {1}
	node [xshift=.75cm, yshift=-.25cm] (4){4}
	node [xshift=.15cm, yshift=.8cm](5){5}
	node [xshift=.75cm, yshift=.5cm] (7){7};
\draw \secondcircle node [outer sep=1.33cm, above] {$M$}
	node [yshift=.25cm](2) {2};
\draw \thirdcircle node [outer sep=1cm, below right] {$P$}
	node [xshift=.25cm, yshift=-.25cm](3) {3}
	node[xshift=-.15cm, yshift=.8cm](6){6}
	node[xshift=1.25cm, yshift=1.75cm](8){8};  
\end{tikzpicture}
\label{fig:three_term_venn_areas}
\end{center}

So in this diagram, area 1 represents the things that are $S$ but not $M$ or $P$, area 2 represents the things that are $M$ but not $S$ or $P$, etc. 

As before, we represent universal statements by filling in the area that the statement says cannot be occupied. The only difference is that now there are more possibilities. So, for instance, there are now four mood-A propositions that can occur in the two premises. The major premise can either be ``All $P$ are $M$'' or ``All $M$ are $P$,'' and the minor premise can be either ``All $S$ are $M$'' or ``All $M$ are $S$.'' The Venn diagrams for those four sentences are given in the top two rows of Figure \ref{fig:mood-U_venns}.

Similarly, there are four mood-E propositions that can occur in the premises of an Aristotelian syllogism: ``No $P$ are $M$,'' ``No $M$ are $P$,'' ``No $S$ are $M$,'' and ``No $M$ are $S$.'' And again, we diagram these by shading out overlap between the two relevant circles. In this case, however, the first two statements are equivalent by conversion (see page \ref{defConversion}), as are the second two. Thus we only have two diagrams to worry about. See the bottom of Figure \ref{fig:mood-U_venns}


\begin{figure}
\begin{mdframed}[style=mytablebox]
\begin{tabu}{X[1,c,m]X[1,c,m]}

\multicolumn{2}{c}{\Large{Mood-A Statements}} \\


\multicolumn{2}{c}{\large{\emph{Major Premise}}} \\

\begin{center}
\begin{tikzpicture}
\def\firstcircle{(0,0) circle (1.25cm)}
\def\secondcircle{(60:1.5cm) circle (1.25cm)}
\def\thirdcircle{(0:1.5cm) circle (1.25cm)}

\begin{scope}[even odd rule] % Shade P without M
\clip \secondcircle (-1.5,-1.5) rectangle (2.75,2);
\fill[gray] \thirdcircle;
\end{scope}

\draw \firstcircle node[outer sep=1cm, below left] {$S$};
\draw \secondcircle node [outer sep=1.33cm, above] {$M$};
\draw \thirdcircle node [outer sep=1cm, below right] {$P$};
\end{tikzpicture}
\end{center}

&

\begin{center}
\begin{tikzpicture}
\def\firstcircle{(0,0) circle (1.25cm)}
\def\secondcircle{(60:1.5cm) circle (1.25cm)}
\def\thirdcircle{(0:1.5cm) circle (1.25cm)}

\begin{scope}[even odd rule] % Shade M without P
\clip \thirdcircle (-1.5,-1.5) rectangle (2,2.75);
\fill[gray] \secondcircle;
\end{scope}

\draw \firstcircle node[outer sep=1cm, below left] {$S$};
\draw \secondcircle node [outer sep=1.33cm, above] {$M$};
\draw \thirdcircle node [outer sep=1cm, below right] {$P$};

\end{tikzpicture}
\end{center}

\\

``All $P$ are $M$.'' 

&

``All $M$ are $P$.'' 

\\
\multicolumn{2}{c}{\large{\emph{Minor Premise}}} \\

\begin{center}
\begin{tikzpicture}
\def\firstcircle{(0,0) circle (1.25cm)}
\def\secondcircle{(60:1.5cm) circle (1.25cm)}
\def\thirdcircle{(0:1.5cm) circle (1.25cm)}

\begin{scope}[even odd rule] % Shade M without S
\clip \secondcircle (-1.5,-1.5) rectangle (2.75,2);
\fill[gray] \firstcircle;
\end{scope}

\draw \firstcircle node[outer sep=1cm, below left] {$S$};
\draw \secondcircle node [outer sep=1.33cm, above] {$M$};
\draw \thirdcircle node [outer sep=1cm, below right] {$P$};

\end{tikzpicture}
\end{center}

&

\begin{center}
\begin{tikzpicture}
\def\firstcircle{(0,0) circle (1.25cm)}
\def\secondcircle{(60:1.5cm) circle (1.25cm)}
\def\thirdcircle{(0:1.5cm) circle (1.25cm)}

\begin{scope}[even odd rule] % Shade S without M
\clip \firstcircle (-1.5,-1.5) rectangle (2,2.75);
\fill[gray] \secondcircle;
\end{scope}

\draw \firstcircle node[outer sep=1cm, below left] {$S$};
\draw \secondcircle node [outer sep=1.33cm, above] {$M$};
\draw \thirdcircle node [outer sep=1cm, below right] {$P$};
\end{tikzpicture}
\end{center}

\\


``All $S$ are $M$.'' 
&
``All $M$ are $S$.'' 


\\
\\
\arrayrulecolor{gray}
\hline
\\

\multicolumn{2}{c}{\Large{Mood-E Statements}} \\

\large{\emph{Major Premise}}  & \large{\emph{Minor Premise}} \\

\begin{center}
\begin{tikzpicture}
\def\firstcircle{(0,0) circle (1.25cm)}
\def\secondcircle{(60:1.5cm) circle (1.25cm)}
\def\thirdcircle{(0:1.5cm) circle (1.25cm)}

\begin{scope} %shade overlap between P and M
\clip \thirdcircle;
\fill[gray] \secondcircle;
\end{scope}


\draw \firstcircle node[outer sep=1cm, below left] {$S$};
\draw \secondcircle node [outer sep=1.33cm, above] {$M$};
\draw \thirdcircle node [outer sep=1cm, below right] {$P$};
\end{tikzpicture}
\end{center}

&

\begin{center}
\begin{tikzpicture}
\def\firstcircle{(0,0) circle (1.25cm)}
\def\secondcircle{(60:1.5cm) circle (1.25cm)}
\def\thirdcircle{(0:1.5cm) circle (1.25cm)}

\begin{scope} %shade overlap between S and M
\clip \firstcircle;
\fill[gray] \secondcircle;
\end{scope}

\draw \firstcircle node[outer sep=1cm, below left] {$S$};
\draw \secondcircle node [outer sep=1.33cm, above] {$M$};
\draw \thirdcircle node [outer sep=1cm, below right] {$P$};
\end{tikzpicture}
\end{center}

\\


``No $M$ are $P$'' or 
``No $P$ are $M$.''

&

``No $M$ are $S$'' or ``No $S$ are $M$.''

\end{tabu}
\end{mdframed}
\caption{Venn diagrams for the eight universal statements that can occur in the premises.} \label{fig:mood-U_venns}
\end{figure}

Particular propositions are a bit trickier. Consider the statement ``Some $M$ are $P$.'' With a two circle diagram, you would just put an x in the overlap between the $M$ circle and the $P$ circle. But with the three circle diagram, there are now two places we can put it. It can go in either area 6 or area 7: 

\begin{center}
\begin{tikzpicture}
\def\firstcircle{(0,0) circle (1.25cm)}
\def\secondcircle{(60:1.5cm) circle (1.25cm)}
\def\thirdcircle{(0:1.5cm) circle (1.25cm)}

\draw \firstcircle node[outer sep=1cm, below left] {$S$}
	node [xshift=.75cm, yshift=.5cm] (7){x?};
\draw \secondcircle node [outer sep=1.25cm, above] {$M$};
\draw \thirdcircle node [outer sep=1cm, below right] {$P$}
	node[yshift=.9cm](6){x?};
\end{tikzpicture}
\end{center}

The solution here will be to put the x on the boundary between areas 6 and 7, to represent the fact that it could go in either location. 

\begin{center}
\begin{tikzpicture}
\def\firstcircle{(0,0) circle (1.25cm)}
\def\secondcircle{(60:1.5cm) circle (1.25cm)}
\def\thirdcircle{(0:1.5cm) circle (1.25cm)}
\draw \firstcircle node[outer sep=1cm, below left] {$S$}
	node [xshift=1.1cm, yshift=.6cm, fill=white] (7){\Large{x}};
\draw \secondcircle node [outer sep=1.25cm, above] {$M$};
\draw \thirdcircle node [outer sep=1cm, below right] {$P$};
\end{tikzpicture}
\end{center}

Sometimes, however, you won't have to draw the x on a border between two areas, because you will already know that one of those areas can't be occupied. Suppose, for instance, that you want to diagram ``Some $M$ are $P$,'' but you already know that all $M$ are $S$. You would diagram ``All $M$ are $S$'' like this:

\begin{center}
\begin{tikzpicture}
\def\firstcircle{(0,0) circle (1.25cm)}
\def\secondcircle{(60:1.5cm) circle (1.25cm)}
\def\thirdcircle{(0:1.5cm) circle (1.25cm)}

\begin{scope}[even odd rule] % Shade M without S
\clip \firstcircle (-2,-2) rectangle (2,2.6);
\fill[gray] \secondcircle;
\end{scope}

\draw \firstcircle node[outer sep=1cm, below left] {$S$};
\draw \secondcircle node [outer sep=1.25cm, above] {$M$};
\draw \thirdcircle node [outer sep=1cm, below right] {$P$};
\end{tikzpicture}
\end{center}

Then, when it comes time to add the x for ``Some $M$ are $P$,'' you know that it has to go in the exact center of the diagram:

\begin{center}
\begin{tikzpicture}
\def\firstcircle{(0,0) circle (1.25cm)}
\def\secondcircle{(60:1.5cm) circle (1.25cm)}
\def\thirdcircle{(0:1.5cm) circle (1.25cm)}

\begin{scope}[even odd rule] % Shade M without S
\clip \firstcircle (-2,-2) rectangle (2,2.6);
\fill[gray] \secondcircle;
\end{scope}

\draw \firstcircle node[outer sep=1cm, below left] {$S$}
	node [xshift=.75cm, yshift=.5cm, fill=white] (7){\Large{x}};
\draw \secondcircle node [outer sep=1.25cm, above] {$M$};
\draw \thirdcircle node [outer sep=1cm, below right] {$P$};
\end{tikzpicture}
\end{center}

The Venn diagrams for the particular premises that can appear in Aristotelian syllogisms are given in Figure \ref{fig:part_venns}. The figure assumes that you are just representing the individual premises, and don't know any other premises that would shade some regions out. Again, some of these premises are equivalent by conversion, and thus share a Venn diagram.

\begin{figure}
\begin{mdframed}[style=mytablebox]
\begin{tabu}{X[1,c,m]X[1,c,m]}

\multicolumn{2}{c}{\Large{Mood-I Statements}} \\

\large{\emph{Major Premise}}  & \large{\emph{Minor Premise}} \\

\begin{center}
\begin{tikzpicture}
\def\firstcircle{(0,0) circle (1.25cm)}
\def\secondcircle{(60:1.5cm) circle (1.25cm)}
\def\thirdcircle{(0:1.5cm) circle (1.25cm)}


\draw \firstcircle node[outer sep=1cm, below left] {$S$};
\draw \secondcircle node [outer sep=1.33cm, above] {$M$};
\draw \thirdcircle node [outer sep=1cm, below right] {$P$}
	node [xshift=-.5cm, yshift=.6cm, fill=light-gray] (7){\Large{x}};
\end{tikzpicture}
\end{center}

&

\begin{center}
\begin{tikzpicture}
\def\firstcircle{(0,0) circle (1.25cm)}
\def\secondcircle{(60:1.5cm) circle (1.25cm)}
\def\thirdcircle{(0:1.5cm) circle (1.25cm)}


\draw \firstcircle node[outer sep=1cm, below left] {$S$};
\draw \secondcircle node [outer sep=1.33cm, above] {$M$};
\draw \thirdcircle node [outer sep=1cm, below right] {$P$}
	node [xshift=-1.1cm, yshift=.6cm, fill=light-gray] (7){\Large{x}};
\end{tikzpicture}
\end{center}

\\


``Some $M$ are $P$'' or 
``Some $P$ are $M$.''

&

``Some $M$ are $S$'' or ``Some $S$ are $M$.''


\\
\\
\arrayrulecolor{gray}
\hline
\\


\multicolumn{2}{c}{\Large{Mood-O Statements}} \\


\multicolumn{2}{c}{\large{\emph{Major Premise}}} \\

\begin{center}
\begin{tikzpicture}
\def\firstcircle{(0,0) circle (1.25cm)}
\def\secondcircle{(60:1.5cm) circle (1.25cm)}
\def\thirdcircle{(0:1.5cm) circle (1.25cm)}

\draw \firstcircle node[outer sep=1cm, below left] {$S$};
\draw \secondcircle node [outer sep=1.33cm, above] {$M$};
\draw \thirdcircle node [outer sep=1cm, below right] {$P$}
	node [xshift=-.5cm, yshift=-.6cm, fill=light-gray] (7){\Large{x}};
\end{tikzpicture}
\end{center}

&

\begin{center}
\begin{tikzpicture}
\def\firstcircle{(0,0) circle (1.25cm)}
\def\secondcircle{(60:1.5cm) circle (1.25cm)}
\def\thirdcircle{(0:1.5cm) circle (1.25cm)}

\draw \firstcircle node[outer sep=1cm, below left] {$S$};
\draw \secondcircle node [outer sep=1.33cm, above] {$M$};
\draw \thirdcircle node [outer sep=1cm, below right] {$P$}
	node [xshift=-1.1cm, yshift=1.1cm, fill=light-gray] (7){\Large{x}};

\end{tikzpicture}
\end{center}

\\

``Some $P$ are not $M$.'' 

&

``Some $M$ are not $P$.'' 

\\
\multicolumn{2}{c}{\large{\emph{Minor Premise}}} \\

\begin{center}
\begin{tikzpicture}
\def\firstcircle{(0,0) circle (1.25cm)}
\def\secondcircle{(60:1.5cm) circle (1.25cm)}
\def\thirdcircle{(0:1.5cm) circle (1.25cm)}

\draw \firstcircle node[outer sep=1cm, below left] {$S$};
\draw \secondcircle node [outer sep=1.33cm, above] {$M$};
\draw \thirdcircle node [outer sep=1cm, below right] {$P$}
	node [xshift=-1cm, yshift=-.6cm, fill=light-gray] (7){\Large{x}};

\end{tikzpicture}
\end{center}

&

\begin{center}
\begin{tikzpicture}
\def\firstcircle{(0,0) circle (1.25cm)}
\def\secondcircle{(60:1.5cm) circle (1.25cm)}
\def\thirdcircle{(0:1.5cm) circle (1.25cm)}

\draw \firstcircle node[outer sep=1cm, below left] {$S$};
\draw \secondcircle node [outer sep=1.33cm, above] {$M$};
\draw \thirdcircle node [outer sep=1cm, below right] {$P$}
	node [xshift=-.1cm, yshift=1.2cm, fill=light-gray] (7){\Large{x}};


\end{tikzpicture}
\end{center}

\\


``Some $S$ are not $M$.'' 
&
``Some $M$ are not $S$.'' 



\end{tabu}
\end{mdframed}
\caption{Venn diagrams for the eight particular statements that can occur in the premises.} \label{fig:part_venns}
\end{figure}


\subsection{Venn Diagrams for Full Syllogisms}

In the last chapter, we used Venn diagrams to evaluate arguments with single premises. It turned out that when those arguments were valid, the conclusion was logically equivalent to the premise, so they had the exact same Venn diagram. This time we have two premises to diagram, and the conclusion won't be logically equivalent to either of them. Nevertheless we will find that for valid arguments, once we have diagrammed the two premises, we will also have diagrammed the conclusion.

First we need to specify a rule about the order to diagram the premises in: if one of the premises is universal and the other is particular, diagram the universal one first. This will allow us to narrow down the area where we need to put the x from the particular premise, as in the example above where we diagrammed ``Some $M$ are $P$'' assuming that we already knew that all $M$ are $S$. 

Let's start with a simple example, an argument with the form AAA-1. 

\begin{earg}
\item[P$_1$:] All $M$ are $P$.
\item[P$_2$:] All $S$ are $M.$
\vspace{-.5em}
\item [] \rule{0.3\linewidth}{.5pt} 
\item[C:] All $S$ are $P$.
\end{earg} 

Since both premises are universal, it doesn't matter what order we do them in. Let's do the major premise first. The major premise has us shade out the parts of the $M$ circle that don't overlap the $P$ circle, like this:

\begin{center}
\begin{tikzpicture}
\def\firstcircle{(0,0) circle (1cm)}
\def\secondcircle{(60:1.25cm) circle (1cm)}
\def\thirdcircle{(0:1.25cm) circle (1cm)}
        \begin{scope}[even odd rule]
            \clip \thirdcircle (-2,-2) rectangle (2,2.5);
        \fill[gray] \secondcircle;
        \end{scope}
\draw \firstcircle node[outer sep=.8cm, below left] {$S$};
\draw \secondcircle node [outer sep=1cm, above] {$M$};
\draw \thirdcircle node [outer sep=.8cm, below right] {$P$};
\end{tikzpicture}
\end{center}

The second premise, on the other hand, tells us that there is nothing in the $S$ circle that isn't also in the $M$ circle. We put that together with the first diagram, and we get this:

\begin{center}
\begin{tikzpicture}
\def\firstcircle{(0,0) circle (1cm)}
\def\secondcircle{(60:1.25cm) circle (1cm)}
\def\thirdcircle{(0:1.25cm) circle (1cm)}

        \begin{scope}[even odd rule]
            \clip \thirdcircle (-1,-1) rectangle (2,2.6);
        \fill[gray] \secondcircle;
        \end{scope}

	\begin{scope}[even odd rule]
            \clip \secondcircle (-1,-1) rectangle (2,2);
        \fill[gray] \firstcircle;
        \end{scope}

\draw \firstcircle node[outer sep=.8cm, below left] {$S$};
\draw \secondcircle node [outer sep=1cm, above] {$M$};
\draw \thirdcircle node [outer sep=.8cm, below right] {$P$};
\end{tikzpicture}
\end{center}

From this we can see that the conclusion must be true. All $S$ are $P$, because the only space left in $S$ is the area in the exact center, area 7.

Now let's look at an argument that is invalid. One of the interesting things about the syllogism AAA-1 is that if you change the figure, it ceases to be valid. Consider AAA-2.

\begin{earg}
\item[P$_1$:] All $P$ are $M$.
\item[P$_2$:] All $S$ are $M$.
\vspace{-.5em}
\item [] \rule{0.2\linewidth}{.5pt} 
\item[C:] All $S$ are $P$.
\end{earg}

Again, both premises are universal, so we can do them in any order, so we will do the major premise first. This time, the major premise tells us to shade out the part of $P$ that does not overlap $M$.

\begin{center}
\begin{tikzpicture}
\def\firstcircle{(0,0) circle (1cm)}
\def\secondcircle{(60:1.25cm) circle (1cm)}
\def\thirdcircle{(0:1.25cm) circle (1cm)}
        \begin{scope}[even odd rule]
            \clip \secondcircle (-1,-1) rectangle (2.5,2);
        \fill[gray] \thirdcircle;
        \end{scope}
\draw \firstcircle node[outer sep=.8cm, below left] {$S$};
\draw \secondcircle node [outer sep=1cm, above] {$M$};
\draw \thirdcircle node [outer sep=.8cm, below right] {$P$};
\end{tikzpicture}
\end{center}

The second premise adds the idea that all $S$ are $M$, which we diagram like this:

\begin{center}
\begin{tikzpicture}
\def\firstcircle{(0,0) circle (1cm)}
\def\secondcircle{(60:1.25cm) circle (1cm)}
\def\thirdcircle{(0:1.25cm) circle (1cm)}
        \begin{scope}[even odd rule]
            \clip \secondcircle (-1,-1) rectangle (2.5,2);
        \fill[gray] \thirdcircle;
        \end{scope}
        
        \begin{scope}[even odd rule]
            \clip \secondcircle (-1,-1) rectangle (1,1);
        \fill[gray] \firstcircle;
        \end{scope}
        
\draw \firstcircle node[outer sep=.8cm, below left] {$S$};
\draw \secondcircle node [outer sep=1cm, above] {$M$};
\draw \thirdcircle node [outer sep=.8cm, below right] {$P$};
\end{tikzpicture}
\end{center}

Now we ask if the diagram of the two premises also shows that the conclusion is true. Here the conclusion is that all $S$ are $P$. If this diagram had made this true, we would have shaded out all the parts of $S$ that do not overlap $P$. But we haven't done that. It is still possible for something to be in area 5. Therefore this argument is invalid. 

Now let's try an argument with a particular statement in the premises. Consider the argument IAI-1:

\begin{earg}
\item[P$_1$:] Some $M$ are $P$.
\item[P$_2$:] All $S$ are $M$.
\vspace{-.5em}
\item [] \rule{0.2\linewidth}{.5pt} 
\item[C:] Some $S$ are $P$.
\end{earg} 

Here, the second premise is universal, while the first is particular, so we begin by diagramming the universal premise.

\begin{center}
\begin{tikzpicture}
\def\firstcircle{(0,0) circle (1cm)}
\def\secondcircle{(60:1.25cm) circle (1cm)}
\def\thirdcircle{(0:1.25cm) circle (1cm)}

        \begin{scope}[even odd rule]
            \clip \secondcircle (-1,-1) rectangle (1,1);
        \fill[gray] \firstcircle;
        \end{scope}

\draw \firstcircle node[outer sep=.8cm, below left] {$S$};
\draw \secondcircle node [outer sep=1cm, above] {$M$};
\draw \thirdcircle node [outer sep=.8cm, below right] {$P$};
\end{tikzpicture}
\end{center}

Then we diagram the particular premise ``Some $M$ are $P$.'' This tells us that something is in the overlap between $M$ and $P$, but it doesn't tell us whether that thing is in the exact center of the diagram or in the area for things that are $M$ and $P$ but not $S$. Therefore, we place the x on the border between these two areas.

\begin{center}
\begin{tikzpicture}
\def\firstcircle{(0,0) circle (1cm)}
\def\secondcircle{(60:1.25cm) circle (1cm)}
\def\thirdcircle{(0:1.25cm) circle (1cm)}

        \begin{scope}[even odd rule]
            \clip \secondcircle (-1,-1) rectangle (1,1);
        \fill[gray] \firstcircle;
        \end{scope}

\draw \firstcircle node[outer sep=.8cm, below left] {$S$};
\draw \secondcircle node [outer sep=1cm, above] {$M$};
\draw \thirdcircle node [outer sep=.8cm, below right] {$P$}
	node[xshift=-.45cm, yshift=.5cm, fill=white](6){x};
\end{tikzpicture}
\end{center}

Now we can see that the argument is not valid. The conclusion asserts that something is in the overlap between $S$ and $P$. But the x we drew does not necessarily represent an object that exists in that overlap. There is something out there that could be in area 7, but it could just as easily be in area 6. The second premise doesn't help us, because it just rules out the existence of objects in areas 1 and 4. 

For a final example, let's look at a case of a valid argument with a particular statement in the premises. If we simply change the figure of the argument in the last example from 1 to 3, we get a valid argument. This is the argument IAI-3:


\begin{earg}
\item[P$_1$:] Some $M$ are $P$.
\item[P$_2$:] All $M$ are $S$.
\vspace{-.5em}
\item [] \rule{0.3\linewidth}{.5pt} 
\item[C:] Some $S$ are $P$.
\end{earg} 

Again, we begin with the universal premise. This time it tells us to shade out part of the $M$ circle. 

\begin{center}
\begin{tikzpicture}
\def\firstcircle{(0,0) circle (1cm)}
\def\secondcircle{(60:1.25cm) circle (1cm)}
\def\thirdcircle{(0:1.25cm) circle (1cm)}

        \begin{scope}[even odd rule]
            \clip \firstcircle (-1,-1) rectangle (2,2.1);
        \fill[gray] \secondcircle;
        \end{scope}

\draw \firstcircle node[outer sep=.8cm, below left] {$S$};
\draw \secondcircle node [outer sep=1cm, above] {$M$};
\draw \thirdcircle node [outer sep=.8cm, below right] {$P$};
\end{tikzpicture}
\end{center}

But now we fill in the parts of $M$ that don't overlap with $S$, we have to put the x in the exact center of the diagram.


\begin{center}
\begin{tikzpicture}
\def\firstcircle{(0,0) circle (1cm)}
\def\secondcircle{(60:1.25cm) circle (1cm)}
\def\thirdcircle{(0:1.25cm) circle (1cm)}

        \begin{scope}[even odd rule]
            \clip \firstcircle (-1,-1) rectangle (2,2.1);
        \fill[gray] \secondcircle;
        \end{scope}

\draw \firstcircle node[outer sep=.8cm, below left] {$S$};
\draw \secondcircle node [outer sep=1cm, above] {$M$};
\draw \thirdcircle node [outer sep=.8cm, below right] {$P$}
	node [xshift=-.6cm, yshift=.4cm] (7){\large{x}};
\end{tikzpicture}
\end{center}


And now this time we see that ``Some $S$ are $P$'' has to be true based on the premises, because the X has to be in area 7. So this argument is valid.                                 
                                               
\newglossaryentry{conditional validity}
{
name=conditional validity,
description={A kind if validity that Aristotelian syllogisms have if they are valid only given the assumption that the objects named by its terms actually exist.}
}

\newglossaryentry{unconditional validity}
{
name=unconditional validity,
description={A kind of validity that an Aristotelian syllogism has regardless of whether the objects named by its terms actually exist.}
}
                                                                                                    
Using this method, we can show that 15 of the 256 possible syllogisms are valid. Remember, however, that the Venn diagram method uses Boolean assumptions about existential import. If you make other assumptions about existential import, you will allow more valid syllogisms, as we will see in the next section. The additional syllogisms we will be able to prove valid in the next section will be said to have \textsc{\gls{conditional validity}} \label{def:Conditional_validity} because they are valid on the condition that the objects talked about in the universal statements actually exist. The 15 syllogisms that we can prove valid using the Venn diagram method have \textsc{\gls{unconditional validity}}. \label{def:Unconditional_validity} These syllogisms are given in Table \ref{tab:unconditionally_valid}. 

\begin{table}
\begin{mdframed}[style=mytablebox]
\begin{tabu}{X[1,c,m]X[1,c,m]X[1,c,m]X[1,c,m]}
\rowfont\bfseries
Figure 1 		& Figure 2 			& Figure 3 		& Figure 4 \\
%\endhead 
Barbara (AAA) 	& Camestres (AEE) 	& Disamis (IAI) 	& Calemes (AEE) \\
Celarent (EAE) 	& Cesare (EAE) 	& Bocardo (OAO)	& Dimatis (IAI) \\
Ferio (EIO)		& Festino (EIO) 	& Ferison (EIO) 	& Fresison (EIO) \\
Darii (AII)		& Baroco (AOO) 	& Datisi (AII) 	 & \\
\end{tabu}
\end{mdframed}
\caption{The 15 unconditionally valid syllogisms.}
\label{tab:unconditionally_valid}
% the version of the naming scheme used here is just the one from wikipedia. Hurley gives a version of the poem, but doesn't say which one it is. He also doesn't use the names as he goes along. Hurley's list differs from the wikipedia list on three names: Ferioque for Ferio, Camenes for Calemes, Dimaris for Dimatis.
\end{table}

The names on Table \ref{tab:unconditionally_valid} come from the Christian part of the Aristotelian tradition, where thinkers were writing in Latin. Students in that part of the tradition learned the valid forms by giving each one a female name. The vowels in the name represented the mood of the syllogism. So B\textbf{a}rb\textbf{a}r\textbf{a} has the mood AAA, Fr\textbf{e}s\textbf{i}s\textbf{o}n has the mood EIO, etc. The consonants in each name were also significant: they related to a process the Aristotelians were interested in called reduction, where arguments in the later figures were shown to be equivalent to arguments in the first figure, which was taken to be more self-evident. We won't worry about reduction in this textbook, however. The names of the valid syllogisms were often worked into a mnemonic poem. The oldest known version of the poem appears in a late 13th century book called \textit{Introduction to Logic} by William of Sherwood (\cite{Sherwood1275}). Figure \ref{fig:barbara,celarent} is an image of the oldest surviving manuscript of the poem, digitized by the Bibliothèque Nationale de France.

The columns in Table \ref{tab:unconditionally_valid} represent the four figures. Syllogisms with the same mood also appear in the same row. So the EIO sisters---Ferio, Festino, Ferison, and Fresison---fill up row 3.  Camestres and Calemes share row 1;  Celarent and Cesare share row 2; and Darii and Datisi share row 4.   

\begin{figure}[b]
\begin{mdframed}[style=mytableclearbox]
\includegraphics*[scale=.75]{img/barbara_celarent_darii_ferio} 
\end{mdframed}
\caption{The oldest surviving version of the ``Barbara, Celarent...'' poem, from William of Sherwood (\cite*{Sherwood1275}). The manuscript is held at the Biblioth\`eque Nationale de France, ms. Lat. 16617, \url{http://gallica.bnf.fr/ark:/12148/btv1b9066740r}}
\label{fig:barbara,celarent}
\end{figure}


%%%%%%%%%	 Practice problems %%%%%%%%%%%

\practiceproblems
\label{venn_proofs}
\problempart Use Venn diagrams to determine whether the following Aristotelian syllogisms are valid. You can check your answers against Table \ref{tab:unconditionally_valid}.

\begin{longtabu}{X[1,l,p]X[9,l,p]} 
\textbf{Example}: &All $P$ are $M$ and no $M$ are $S$. Therefore, no $S$ are $P$. \\ 
\textbf{Answer}: &Valid (Calemes, AEE-4) \\
&\begin{tikzpicture}
\def\firstcircle{(0,0) circle (.75cm)}
\def\secondcircle{(60:.75cm) circle (.75cm)}
\def\thirdcircle{(0:.75cm) circle (.75cm)}

	\begin{scope}[even odd rule] % Shade P without M
            \clip \secondcircle (-1,-1) rectangle (1.5,1.5);
        \fill[gray] \thirdcircle;
        \end{scope}

    \begin{scope} %shade overlap between S and M
      \clip \firstcircle;
      \fill[gray] \secondcircle;
    \end{scope}

\draw \firstcircle node[outer sep=.66cm, below left] {$S$};
\draw \secondcircle node [outer sep=.75cm, above] {$M$};
\draw \thirdcircle node [outer sep=.66cm, below right] {$P$};
\end{tikzpicture}\\ 
\end{longtabu} 

\begin{exercises}

\item Some $P$ are not $M$, and no $M$ are $S$. Therefore, some $S$ are $P$.
\answer{\\
\begin{venns}
\shadeintersectred{\middlecircle}{\subjectcircle}
\drawsubsyl
\drawmidsyl
\drawpredsyl
\someexistthreefour
\end{venns}
\\
OEI-4 Invalid 
}

\item All $M$ are $P$, and some $M$ are $S$. Therefore some $S$ are $P$. 
\answer{\\
\begin{venns}
\shadecomplementred{\middlecircle}{\middlesquare}{\predicatecircle}
\drawsubsyl
\drawmidsyl
\drawpredsyl
\someexistseven
\end{venns}
\\
Datisi (AII-3) Valid 
} 

\item No $P$ are $M$, and some $S$ are $M$. Therefore, some $S$ are not $P$.
\answer{\\
\begin{venns}
\shadeintersectred{\predicatecircle}{\middlecircle}
\drawsubsyl
\drawmidsyl
\drawpredsyl
\someexistfive
\end{venns}
\\
Festino (EIO-2) Valid 
} 
      
\item No $M$ are $P$, and all $S$ are $M$. Therefore, no $S$ are $P$.
\answer{\\
\begin{venns}
\shadeintersectred{\middlecircle}{\predicatecircle}
\shadecomplementred{\subjectcircle}{\subjectsquare}{\middlecircle}
\drawsubsyl
\drawmidsyl
\drawpredsyl
\end{venns}
\\
Celarent (EAE-1) Valid 
} 
\item All $M$ are $P$, and no $M$ are $S$. Therefore, all $S$ are $P$.
\answer{\\
\begin{venns}
\shadecomplementred{\middlecircle}{\middlesquare}{\predicatecircle}
\shadeintersectred{\middlecircle}{\subjectcircle}
\drawsubsyl
\drawmidsyl
\drawpredsyl
\end{venns}
\\
AEA-3 Invalid 
} 
\item Some $M$ are not $P$, and some $M$ are $S$. Therefore, all $S$ are $P$.
\answer{\\
\begin{venns}
\drawsubsyl
\drawmidsyl
\drawpredsyl
\someexisttwofive
\someexistfiveseven
\end{venns}
\\

 OIA-3 Invalid 
} 
 
\item No $P$ are $M$, and some $S$ are not $M$. Therefore, some $S$ are not $P$.
\answer{\\
\begin{venns}
\shadeintersectred{\predicatecircle}{\middlecircle}
\drawsubsyl
\drawmidsyl
\drawpredsyl
\someexistonefour
\end{venns}
\\
EOO-2 Invalid 
}
 
\item Some $P$ are $M$, and some $S$ are $M$. Therefore, no $S$ are $P$.
\answer{\\
\begin{venns}
\drawsubsyl
\drawmidsyl
\drawpredsyl
\someexistsixseven
\someexistfiveseven
\end{venns}
\\
IIE-2 Invalid 
}

\item No $P$ are $M$, and all $S$ are $M$. Therefore no $S$ are $P$.
\answer{\\
\begin{venns}
\shadeintersectred{\predicatecircle}{\middlecircle}
\shadecomplementred{\subjectcircle}{\subjectsquare}{\middlecircle}
\drawsubsyl
\drawmidsyl
\drawpredsyl
\end{venns}
\\
Cesare (EAE-2) Valid 
}  
\item No $M$ are $P$ and all $S$ are $M$. Therefore some $S$ are not $P$
\answer{\\
\begin{venns}
\shadeintersectred{\predicatecircle}{\middlecircle}
\shadecomplementred{\subjectcircle}{\subjectsquare}{\middlecircle}

\drawsubsyl
\drawmidsyl
\drawpredsyl
\end{venns}
\\
EAO-1 Invalid \\
As we will learn in the next section, this argument would be valid if we added the premise ``Some $S$ exist.'' As a result, we will say this form is ``conditionally valid''
}
\end{exercises}

\noindent \problempart Use Venn diagrams to determine whether the following Aristotelian syllogisms are valid. You can check your answers against Table \ref{tab:unconditionally_valid}.

\begin{exercises}

\item No $M$ are $P$, and some $M$ are $S$. Therefore some $S$ are not $P$.
\answer{\\Ferison (EIO-III)} 

\item Some $M$ are not $P$, and all $M$ are $S$. Therefore some $S$ are not $P$.
\answer{\\Bocardo (OAO-3)}
 
\item   No $M$ are $P$, and some $S$ are $M$. Therefore, some $S$ are not $P$.
 \answer{\\Ferio (EIO-I)}
 
\item All $M$ are $P$, and some $S$ are not $M$. Therefore, some $S$ are $P$.
\answer{\\AOI-I Invalid} 
 
\item Some $P$ are $M$, and all $M$ are $S$. Therefore some $S$ are $P$.
 \answer{\\Dimatis (IAI-IV)} 
 
\item Some $M$ are $P$, and all $S$ are $M$. Therefore, all $S$ are $P$.
\answer{\\IAA-I Invalid }
 
\item All $P$ are $M$, and all $M$ are $S$. Therefore, some $S$ are $P$.
\answer{\\AAI-IV Invalid }

 \item  All $P$ are $M$, and some $S$ are not $M$. Therefore, some $S$ are not $P$. 
 \answer{\\Baroco (AOO-II)}

\item No $P$ are $M$, and all $M$ are $S$. Therefore no $S$ are $P$.
\answer{\\EAE-IV Invalid} 
 
\item Some $P$ are $M$, and some $S$ are $M$. Therefore, some $S$ are $P$.
\answer{\\III-II Invalid} 
 \end{exercises}

\noindent\problempart The arguments below are missing conclusions. Use Venn diagrams to determine what conclusion can be drawn from the two premises. If no conclusion can be drawn, write ``No conclusion.'' \label{no_conclusion_set1}

\begin{longtabu}{p{.15\linewidth}p{.85\linewidth}} 
\textbf{Example 1}: & No $P$ are $M$ and some $M$ are not $S$. Therefore \underline{\hspace{2cm}} \\
\textbf{Answer}: & No conclusion\\
& 

\begin{venns}
\shadeintersect{\predicatecircle}{\middlecircle}
\someexisttwo
\drawsubsyl
\drawmidsyl
\drawpredsyl
\end{venns} 
\\
\textbf{Example 2}: & No $P$ are $M$ and  All $S$ are $M$. Therefore \underline{\hspace{2cm}} \\
\textbf{Answer}: &  No $S$ are $P$\\
& 

\begin{venns}
\shadeintersect{\predicatecircle}{\middlecircle}
\shadecomplement{\subjectcircle}{\subjectsquare}{\middlecircle}
\drawsubsyl
\drawmidsyl
\drawpredsyl
\end{venns}

\end{longtabu} 

\begin{exercises} 
 
\item All $M$ are $P$, and all $S$ are $M$. Therefore \underline{\hspace{2cm}}.
\answer{ \hspace{-2.4cm} All $S$ are $P$. \\
\begin{venns}
\shadecomplementred{\middlecircle}{\middlesquare}{\predicatecircle}
\shadecomplementred{\subjectcircle}{\subjectsquare}{\middlecircle}
\drawsubsyl
\drawmidsyl
\drawpredsyl
\end{venns}
\\ Barbara (AAA-1) 
} 

\item All $M$ are $P$, and some $M$ are $S$. Therefore \underline{\hspace{2cm}}. 
\answer{\hspace{-2.4cm} Some $S$ are $P$. \\
\begin{venns}
\shadecomplementred{\middlecircle}{\middlesquare}{\predicatecircle}
\drawsubsyl
\drawmidsyl
\drawpredsyl
\someexistseven
\end{venns}
\\Datisi (AII-3) 
} 

\item No $M$ are $P$ and some $S$ are not $M$. Therefore \underline{\hspace{2cm}}. 
\answer{ \hspace{-2.4cm} No conclusion \\
\begin{venns}
\shadeintersectred{\middlecircle}{\predicatecircle}
\drawsubsyl
\drawmidsyl
\drawpredsyl
\someexistonefour
\end{venns}
 \\ EO?-1 Invalid 
 }
\item Some $M$ are $P$, and some $S$ are $M$. Therefore \underline{\hspace{2cm}}. 
\answer{ \hspace{-2.4cm} No conclusion \\
\begin{venns}
\drawsubsyl
\drawmidsyl
\drawpredsyl
\someexistsixseven
\someexistfiveseven
\end{venns}
\\  II?-1 Invalid 
} 
\item Some $P$ are $M$, and all $M$ are $S$. Therefore \underline{\hspace{2cm}}. 
\answer{\hspace{-2.4cm} Some $S$ are $P$. \\
\begin{venns}
\shadecomplementred{\middlecircle}{\middlesquare}{\subjectcircle}
\drawsubsyl
\drawmidsyl
\drawpredsyl
\someexistseven
\end{venns}
\\ IAI-4 Valid (Dimatis) 
 }
\item All $P$ are $M$ and no $M$ are $S$. Therefore \underline{\hspace{2cm}} .
\answer{\hspace{-2.4cm} No $S$ are $P$. \\
\begin{venns}
\shadecomplementred{\predicatecircle}{\predicatesquare}{\middlecircle}
\shadeintersectred{\subjectcircle}{\middlecircle}
\drawsubsyl
\drawmidsyl
\drawpredsyl
\end{venns}
\\Calemes (AEE-4) 
}
 
\item No $M$ are $P$ and all $S$ are $M$. Therefore \underline{\hspace{2cm}} .
\answer{ \hspace{-2.4cm} No $S$ are $P$. \\
\begin{venns}
\shadeintersectred{\middlecircle}{\predicatecircle}
\shadecomplementred{\subjectcircle}{\subjectsquare}{\middlecircle}
\drawsubsyl
\drawmidsyl
\drawpredsyl
\end{venns}
\\ Celarent (EAE-1) 
} 

\item \label{itm:no_conclusion_set1_EEE} No $P$ are $M$, and no $M$ are $S$. Therefore \underline{\hspace{2cm}}.
\answer{\hspace{-2.4cm} No conclusion. \\
\begin{venns}
\shadeintersectred{\middlecircle}{\predicatecircle}
\shadeintersectred{\middlecircle}{\subjectcircle}
\drawsubsyl
\drawmidsyl
\drawpredsyl
\end{venns}
\\ EE?-4 Invalid  
 }
 
\item No $M$ are $P$, and some $S$ are $M$. Therefore \underline{\hspace{2cm}}.
\answer{\hspace{-2.4cm} Some $S$ are not $P$. \\
\begin{venns}
\shadeintersectred{\middlecircle}{\predicatecircle}
\drawsubsyl
\drawmidsyl
\drawpredsyl
\someexistfive
\end{venns}
\\ Valid Ferio (EIO-1) . 
}
\item Some $P$ are $M$, and some $S$ are $M$. Therefore \underline{\hspace{2cm}}. 
\answer{\hspace{-2.4cm} No conclusion. \\
\begin{venns}
\drawsubsyl
\drawmidsyl
\drawpredsyl
\someexistsixseven
\someexistfiveseven
\end{venns}
 \\II?-2 Invalid 
  }
  \end{exercises}

\noindent \problempart The arguments below are missing conclusions. Use Venn diagrams to determine what conclusion can be drawn from the two premises. If no conclusion can be drawn, write ``No conclusion.'' 

\begin{exercises} 
\item Some $P$ are not $M$, and all $M$ are $S$. Therefore, \underline{\hspace{2cm}}. 
 \answer{\\OAA-IV Invalid }
 
\item All $M$ are $P$, and some $S$ are $M$. Therefore \underline{\hspace{2cm}}. 
\answer{\\Darii (AII-1) (unconditionally valid) }
 
\item All $P$ are $M$, and some $S$ are not $M$. Therefore \underline{\hspace{2cm}}.
\answer{\\Baroco (AOO-II) (unconditionally valid)}
 
\item Some $P$ are $M$, and all $M$ are $S$. Therefore \underline{\hspace{2cm}}.
\answer{\\Dimatis (IAI-IV) (unconditionally valid) }
 
\item All $P$ are $M$, and some $M$ are not $S$. Therefore \underline{\hspace{2cm}}. 
\answer{\\AOA-IV Invalid}
 
\item No $M$ are $P$, and some $S$ are $M$. Therefore \underline{\hspace{2cm}}.
\answer{\\Ferio (EIO-I) (unconditionally valid)} 
 
\item No $P$ are $M$, and no $S$ are $M$. Therefore \underline{\hspace{2cm}}. 
 
\item Some $M$ are not $P$, and no $M$ are $S$. Therefore \underline{\hspace{2cm}}. 
 
\item No $P$ are $M$, and all $S$ are $M$. Therefore \underline{\hspace{2cm}}. 
\answer{\\EAA-II Invalid} 
  
\item No $M$ are $P$, and some $M$ are $S$. Therefore \underline{\hspace{2cm}}.
\answer{\\Ferison (EIO-III) (unconditionally valid)} 
 
\end{exercises}
    

\noindent\problempart
\label{ex_on_syllogism_patterns}
\answer{\\ The questions in this section were mostly meant to get you thinking about the issues we will be discussing in Section 5.4. Hopefully by looking for patterns in the 256 syllogisms, you can actually anticipate the rules we will demonstrate later on}

\begin{exercises}
\item Do you think there are any valid arguments in Aristotle's set of 256 syllogisms where both premises are particular? Why or why not? \label{itm:two_particulars} 
\answer{\\No. You can intuitively see why this is the case when you remember that two particular premises could just be telling you about the existence of one or two particular objects, and you aren't going to be able to infer any generalities or any information about other particular objects from those two. For a more formal proof of why this is so, see see the discussion of Derived Rule 1 on page \ref{derived_rule_1}.} 


\item Do you think there are any valid arguments in Aristotle's set of 256 syllogisms where both premises are negative? Why or why not?
\answer{\\No. You can see this by playing around with the Venn diagrams for negative statements, or by looking at the more formal proof of Rule 3 on page \ref{rule_3}.}

\item Can a valid argument have a negative statement in the conclusion, but only affirmative statements in the premises? Why or why not?
\answer{\\No. Again you can see this by playing around with Venn diagrams for negative statements, or by looking at the more formal proof of Rule 4 on page \ref{rule_4}}


\item Can a valid argument have an affirmative statement in the conclusion, but only one affirmative premise?  
\answer{\\No, for the reasons stated above.}

\item Can a valid argument have two universal premises and a particular conclusion? 
\answer{\\No, because the two universal statements do not have existential import, but the particular statement does. We will be talking more about this in the next section and in the proof of Rule 5 on page \ref{rule_5}.}
\end{exercises}

% **********************************************
% *   Existential Import and Conditionally Valid Forms   *
% **********************************************

\section{Existential Import and Conditionally Valid Forms}
\label{sec:conditionally_valid_forms}
In the last section, we mentioned that you can prove more syllogisms valid if you make different assumptions about existential import. Recall that a statement has existential import if, when you assert the statement, you are also asserting the existence of the things the statement talks about. (See page \pageref{def:Existential_import}.) So if you interpret a mood-A statement as having existential import, it not only asserts ``All $S$ is $P$,'' it also asserts ``$S$ exists.'' Thus the mood-A statement ``All unicorns have one horn'' is false, if it is taken to have existential import, because unicorns do not exist. It is probably  true, however, if you do not imagine the statement as having existential import. If anything is true of unicorns, it is that they would have one horn if they existed.

We saw in Section \ref{sec:ExistentialImport} that before Boole, Aristotelian thinkers had all sorts of opinions about existential import, or, as they put it, whether a term ``supposits.'' This generally led them to recognize additional syllogism forms as valid. You can see this pretty quickly if you just remember the traditional square of opposition. The traditional square allowed for many more valid immediate inferences than the modern square. It stands to reason that traditional ideas about existential import will also allow for more valid syllogisms. 

Our system of Venn diagrams can't represent all of the alternative ideas about existential import. For instance, it has no way of representing Ockham's belief that mood-O statements do \emph{not} have existential import. Nevertheless, it would be nice if we could expand our system of Venn diagrams to show that some syllogisms are valid if you make additional assumptions about existence. 

Consider the argument Barbari (AAI-1).

\begin{earg}
\item[P$_1$:] All $M$ are $P$.
\item[P$_2$:] All $S$ are $M$.
\vspace{-.5em}
\item [] \rule{0.2\linewidth}{.5pt} 
\item[C:] Some $S$ are $P$.
\end{earg} 
 
 
You won't find this argument in the list of unconditionally valid forms in Table \ref{tab:unconditionally_valid}. This is because under Boolean assumptions about existence it is not valid. The Venn diagram, which follows Boolean assumptions, shows this. 

\begin{center}
\begin{tikzpicture}
\def\firstcircle{(0,0) circle (1.25cm)}
\def\secondcircle{(60:1.5cm) circle (1.25cm)}
\def\thirdcircle{(0:1.5cm) circle (1.25cm)}

\begin{scope}[even odd rule] % Shade M without P
\clip \thirdcircle (-1,-1) rectangle (2,2.6);
\fill[gray] \secondcircle;
\end{scope}

\begin{scope}[even odd rule] % Shade S without M
\clip \secondcircle (-1.5,-1.5) rectangle (1.5,1.5);
\fill[gray] \firstcircle;
\end{scope}


\draw \firstcircle node[outer sep=1cm, below left] {$S$};
\draw \secondcircle node [outer sep=1.33cm, above] {$M$};
\draw \thirdcircle node [outer sep=1cm, below right] {$P$};
\end{tikzpicture}
\end{center}

This is essentially the same argument as Barbara, but the mood-A statement in the conclusion has been replaced by a mood-I statement. We can see from the diagram that the mood-A statement ``All $S$ are $P$'' is true. There is no place to put an $S$ other than in the overlap with $P$. But we don't actually know the mood-I statement ``Some $S$ is $P$,'' because we haven't drawn an x in that spot. Really, all we have shown is that \emph{if} an $S$ existed, it would be $P$. 

But by the traditional square of opposition (p. \pageref{fig:traditionalsquare}) we know that the mood-I statement is true. The traditional square, unlike the modern one, allows us to infer the truth of a particular statement given the truth of its corresponding universal statement. This is because the traditional square assumes that the universal statement has existential import. It is really two statements, ``All $S$ is $P$'' and ``Some $S$ exists.'' 

Because the mood-A statement is actually two statements on the traditional interpretation, we can represent it simply by adding an additional line to our argument. It is always legitimate to change an argument by making additional assumptions. The new argument won't have the exact same impact on the audience as the old argument. The audience will now have to accept an additional premise, but in this case all we are doing is making explicit an assumption that the Aristotelian audience was making anyway. The expanded argument will look like this:

 \begin{earg}
\item[P$_1$:] All $M$ are $P$.
\item[P$_2$:] All $S$ are $M$.
\item[P$_3$:] Some $S$ exists.*
\vspace{-.5em}
\item [] \rule{0.2\linewidth}{.5pt} 
\item[C:] Some $S$ are $P$
\end{earg} 
 
Here the asterisk indicates that we are looking at an implicit premise that has been made explicit.\iflabelexists{chap:incomplete_unclear_arguments}{This is similar to what we did in Chapter \ref{chap:incomplete_arguments} on adding warrants.}{} Now that we have an extra premise, we can add it to our Venn diagram. Since there is only one place for the $S$ to be, we know where to put our x. \label{CVFex1}

\begin{center}
\begin{tikzpicture}
\def\firstcircle{(0,0) circle (1.25cm)}
\def\secondcircle{(60:1.5cm) circle (1.25cm)}
\def\thirdcircle{(0:1.5cm) circle (1.25cm)}

\begin{scope}[even odd rule] % Shade M without P
\clip \thirdcircle (-1,-1) rectangle (2,2.6);
\fill[gray] \secondcircle;
\end{scope}

\begin{scope}[even odd rule] % Shade S without M
\clip \secondcircle (-1.5,-1.5) rectangle (1.5,1.5);
\fill[gray] \firstcircle;
\end{scope}


\draw \firstcircle node[outer sep=1cm, below left] {$S$};
\draw \secondcircle node [outer sep=1.33cm, above] {$M$};
\draw \thirdcircle node [outer sep=1cm, below right] {$P$}
	node[xshift=-.75cm, yshift=.5cm](7){\Large{x}};
\end{tikzpicture}
\end{center}


\newglossaryentry{critical term}
{
name=critical term,
description={the term that names things that must exist in order for a conditionally valid argument to be actually valid.}
}

\

In this argument $S$ is what we call the ``critical term.'' The \textsc{\gls{critical term}}\label{def:critical_term} is the term that names things that must exist in order for a conditionally valid argument to be actually valid. In this argument, the critical term was $S$, but sometimes it will be $M$ or $P$. 

We have used Venn diagrams to show that Barbari is valid once you include the additional premise. Using this method we can identify nine more forms, on top of the previous 15, that are valid if we add the right existence assumptions (Table \ref{tab:full_twentyfour}) 

\begin{table}
\small
\begin{mdframed}[style=mytablebox]
\begin{longtabu}{X[1,c,m]X[10,c,m]X[11,c,m]X[10,c,m]X[10,c,m]X[6,c,m]}
\rowfont\bfseries 
&	Figure 1 		& Figure 2 			& Figure 3 		& Figure 4 & Condition \\
\endhead 
&&&&\\

\parbox[t]{2mm}{\multirow{4}{*}{\rotatebox{90}{\hspace{-2em}Unconditional}}} 

& Barbara (AAA) 	& Camestres (AEE)  	&  Disamis (IAI) 	& Calemes (AEE) \\

& Celarent (EAE) 	& Cesare (EAE) 	& Bocardo (OAO)	& Dimatis (IAI) \\

& Ferio (EIO)		& Festino (EIO) 	& Ferison (EIO) 	& Fresison (EIO) \\

& Darii (AII)		& Baroco (AOO) 	& Datisi (AII) 	 & \\

&&&&\\

\hline

&&&&\\

\parbox[t]{2mm}{\multirow{4}{*}{\rotatebox{90}{\hspace{-1em}Conditional}}}
& Barbari (AAI)		& Camestros (AEO)	&				&	Calemos (AEO)	& $S$ exists \\

& Celaront (EAO)	& Cesaro (EAO)	&				&			& $S$ exists \\

&					&					& Felapton (EAO)	& Fesapo (EAO)	& $M$ exists \\

&					&					& Darapti (AAI)		&  			& $M$ exists \\

&					&					& 					& Bamalip (AAI) 			& $P$ exists \\

\end{longtabu}
\end{mdframed}
\caption{All 24 Valid Syllogisms}
\label{tab:full_twentyfour}
\end{table}

Thus we now have an expanded method for evaluating arguments using Venn diagrams. To evaluate an argument, we first use a Venn diagram to determine whether it is unconditionally valid. If it is, then we are done. If it is not, then we see if adding an existence assumption can make it conditionally valid. If we can add such an assumption, add it to the list of premises and put an x in the relevant part of the Venn diagram. If we cannot make the argument valid by including additional existence assumptions, we say it is completely invalid.

Let's run through a couple examples. Consider the argument EAO-3.

\begin{earg}
\item[P$_1$:] No $M$ are $P$.
\item[P$_2$:] All $M$ are $S$.
\vspace{-.5em}
\item [] \rule{0.2\linewidth}{.5pt} 
\item[C:] Some $S$ are not $P$.   
\end{earg} 

First we use the regular Venn digram method to see whether the argument is unconditionally valid.

\begin{center}
\begin{tikzpicture}
\def\firstcircle{(0,0) circle (1.25cm)}
\def\secondcircle{(60:1.5cm) circle (1.25cm)}
\def\thirdcircle{(0:1.5cm) circle (1.25cm)}

\begin{scope} %shade overlap between P and M
\clip \thirdcircle;
\fill[gray] \secondcircle;
\end{scope}

\begin{scope}[even odd rule] % Shade M without S
\clip \firstcircle (-1,-1) rectangle (2,2.75);
\fill[gray] \secondcircle;
\end{scope}


\draw \firstcircle node[outer sep=1cm, below left] {$S$};
\draw \secondcircle node [outer sep=1.33cm, above] {$M$};
\draw \thirdcircle node [outer sep=1cm, below right] {$P$};
\end{tikzpicture}
\end{center}

We can see from this that the argument is not valid. The conclusion says that some $S$ are not $P$, but we can't tell that from this diagram. There are three possible ways something could be $S$, and we don't know if any of them are occupied. 

Simply adding the premise $S$ exists won't help us, because we don't know whether to put the x in the overlap between $S$ and $M$, the overlap between $S$ and $P$, or in the area that is just $S$. Of course, we would want to put it in the overlap between $S$ and $M$, because that would mean that there is an $S$ that is not $P$. However, we can't justifying doing this simply based on the premise that $S$ exists. 

The premise that $P$ exists will definitely not help us. The $P$ would either go in the overlap between $S$ and $P$ or in the area that is only $P$. Neither of these would show ``Some $S$ is not $P$.''

The premise ``$M$ exists'' does the trick, however. If an $M$ exists, it has to also be $S$ but not $P$. And this is sufficient so show that some $S$ is not $P$. We can then add this additional premise to the argument to make it valid. \label{CVFex2}

\begin{earg}
\item[P$_1$:] No $M$ are $P$.
\item[P$_2$:] All $M$ are $S$.
\item[P$_3$:] $M$ exists.*
\vspace{-.5em}
\item [] \rule{0.2\linewidth}{.5pt} 
\item[C:] Some $S$ are not $P$.   
\end{earg} 

Checking it against Table \ref{tab:full_twentyfour}, we see that we were right: this is a conditionally valid argument named Felapton.

Now consider the argument EII-3:

\begin{earg} 
\item[P$_1$:] No $M$ are $P$.
\item[P$_2$:] Some $M$ are $S$.
\vspace{-.5em} 
\item [] \rule{0.2\linewidth}{.5pt} 
\item[C:] Some $S$ are $P$.
 \end{earg}

First we need to see if it is unconditionally valid. So we draw the Venn diagram. 

\begin{center}
\begin{tikzpicture}

\def\firstcircle{(0,0) circle (1.25cm)}
\def\secondcircle{(60:1.5cm) circle (1.25cm)}
\def\thirdcircle{(0:1.5cm) circle (1.25cm)}

\begin{scope} %shade overlap between P and M
\clip \thirdcircle;
\fill[gray] \secondcircle;
\end{scope}

\draw \firstcircle node[outer sep=1cm, below left] {$S$};
\draw \secondcircle node [outer sep=1.33cm, above] {$M$};
\draw \thirdcircle node [outer sep=1cm, below right] {$P$}
	node[xshift=-1.33cm, yshift=.75cm](5){\Large{x}};
\end{tikzpicture}
\end{center}

The conclusion says that some $S$ are $P$, but we obviously don't know this from the diagram above. There is no x in the overlap between $S$ and $P$. Part of that region is shaded out, but the rest could go either way.

What about conditional validity? Can we add an existence assumption that would make this valid? Well, the x we have already drawn lets us know that both $S$ and $M$ exist, so it won't help to add those premises. What about adding $P$? That won't help either. We could add the premise ``$P$ exists'' but we wouldn't know whether that $P$ is in the overlap between $S$ and $P$ or in the area to the right, which is just $P$. \label{CVFex3}

Therefore this argument is invalid. And when we check the argument against Table \ref{tab:full_twentyfour}, we see that it is not present. 

\practiceproblems
\noindent \problempart Use Venn diagrams to determine whether the following arguments are unconditionally valid, conditionally valid, or invalid. If they are conditionally valid, write out the premise you need to add. You can check your answers against Table \ref{tab:full_twentyfour}.

\begin{longtabu}{p{.1\linewidth}p{.9\linewidth}} 
\textbf{Example}: & All $M$ are $P$ and all $M$ are $S$. Therefore some $S$ are $P$ \\
\textbf{Answer}: & Added premise: P$_3$: $M$ exists. \\
& \begin{center}
\begin{tikzpicture}
\def\firstcircle{(0,0) circle (.75cm)}
\def\secondcircle{(60:.75cm) circle (.75cm)}
\def\thirdcircle{(0:.75cm) circle (.75cm)}

\begin{scope}[even odd rule] % Shade M without P
\clip \thirdcircle (-1,-1) rectangle (2,2.75);
\fill[gray] \secondcircle;
\end{scope}


\begin{scope}[even odd rule] % Shade M without S
\clip \firstcircle (-1,-1) rectangle (2,2.75);
\fill[gray] \secondcircle;
\end{scope}


\draw \firstcircle node[outer sep=.66cm, below left] {$S$};
\draw \secondcircle node [outer sep=.75cm, above] {$M$};
\draw \thirdcircle node [outer sep=.66cm, below right] {$P$}
	node[xshift=-.4cm, yshift=.25cm](5){{\Large{x}}};
\end{tikzpicture}
\end{center}
\\ &
Conditionally valid (Darapti, AAI-3)
\end{longtabu} 

 
\begin{exercises} 
 
\item No $P$ are $M$, and all $S$ are $M$, so some $S$ are not $P$.
\answer{ \\
\begin{venns}

\shadeintersectred{\predicatecircle}{\middlecircle}
\shadecomplementred{\subjectcircle}{\subjectsquare}{\middlecircle}

\drawsubsyl
\drawmidsyl
\drawpredsyl

\end{venns}

Additional Premise: $S$ exists. 

Cesaro (EAO-1I, conditionally valid)  
\vspace{6pt}}
 
\item Some $P$ are $M$, and some $S$ are $M$. Therefore all $S$ are $P$.
\answer{ \\
\begin{venns}
\drawsubsyl
\drawmidsyl
\drawpredsyl
\someexistsixseven
\someexistfiveseven
\end{venns}

IIA-II Invalid 
\vspace{6pt}}

\item No $M$ are $P$, and some $S$ are $M$. Therefore, some $S$ are not $P$.
\answer{ \\
\begin{venns}
\shadeintersectred{\middlecircle}{\predicatecircle}
\someexistfive
\drawsubsyl
\drawmidsyl
\drawpredsyl

\end{venns}

Ferio (EIO-I, unconditionally valid) 
\vspace{6pt} }

\item  No $M$ are $P$, and all $S$ are $M$. Therefore, some $S$ are not $P$.
\answer{ \\
\begin{venns}
\shadeintersectred{\middlecircle}{\predicatecircle}
\shadecomplementred{\subjectcircle}{\subjectsquare}{\middlecircle}

\drawsubsyl
\drawmidsyl
\drawpredsyl

\end{venns}

Needed premise: Some $S$ exist.

Celaront (EAO-1, Conditionally valid) 
\vspace{6pt} }

\item No $P$ are $M$, and some $M$ are $S$, so some $S$ are not $P$.  
\answer{  \\
\begin{venns}

\shadeintersectred{\predicatecircle}{\middlecircle}

\drawsubsyl
\drawmidsyl
\drawpredsyl
\someexistfive

\end{venns}
\\Fresison (EIO-IV, unconditionally valid) 
\vspace{6pt} }
 
\item All $P$ are $M$, and all $M$ are $S$, so some $S$ are $P$.
\answer{ \\
\begin{venns}
\shadecomplementred{\predicatecircle}{\predicatesquare}{\middlecircle}
\shadecomplementred{\middlecircle}{\middlesquare}{\subjectcircle}
\drawsubsyl
\drawmidsyl
\drawpredsyl
\end{venns}
\\Needed premise: $P$ exists. \\
Bamalip (AAI-IV, conditionally valid) 
\vspace{6pt}  }

\item Some $P$ are not $M$, and some $M$ are $S$. Therefore all $S$ are $P$.
\answer{ \\
\begin{venns}

\drawsubsyl
\drawmidsyl
\drawpredsyl
\someexistthreefour
\someexistfiveseven
\end{venns}
\\ OIA-IV Invalid 7
 \vspace{6pt}}

\item No $P$ are $M$, and all $M$ are $S$. Therefore some $S$ are not $P$. 
\answer{ \\
 \begin{venns}
\shadeintersectred{\predicatecircle}{\middlecircle}
\shadecomplementred{\middlecircle}{\middlesquare}{\subjectcircle}
\drawsubsyl
\drawmidsyl
\drawpredsyl
\end{venns}
\\Needed premise: $P$ exists. \\ Fesapo (AEO-IV, conditionally valid) 
\vspace{6pt}} 

\item All $M$ are $P$, and no $S$ are $M$. Therefore, some $S$ are $P$.
\answer{ \\
 \begin{venns}
\shadecomplementred{\middlecircle}{\middlesquare}{\predicatecircle}
\shadeintersectred{\middlecircle}{\subjectcircle}
\drawsubsyl
\drawmidsyl
\drawpredsyl
\end{venns}
\\ AEI-I Invalid 
}
\item Some $M$ are $P$, and some $S$ are not $M$. Therefore, some $S$ are $P$.
\answer{ \\
\begin{venns}

\drawsubsyl
\drawmidsyl
\drawpredsyl
\someexistsixseven
\someexistonefour
\end{venns}
\\ IOI-I Invalid 
}
\end{exercises}
    
\noindent \problempart Use Venn diagrams to determine whether the following arguments are unconditionally valid, conditionally valid, or invalid. If they are conditionally valid, write out the premise you need to add. You can check your answers against Table \ref{tab:full_twentyfour}.

\begin{exercises} 

\item No $M$ are $P$, and all $M$ are $S$. Therefore some $S$ are not $P$.
\answer{Felapton (EAO-1II)  }
   
\item Some $M$ are $P$, and all $M$ are $S$. Therefore some $S$ are $P$.
\answer{Disamis (IAI-III) (unconditionally valid) }
 
\item All $M$ are $P$, and some $M$ are $S$, so no $S$ are $P$.
\answer{AIE-III Invalid }
 
    
\item No $M$ are $P$, and some $M$ are $S$, so some $S$ are not $P$. 
\answer{Ferison (EIO-III) (unconditionally valid) }
 
 
\item Some $P$ are $M$, and some $S$ are not $M$, so no $S$ are $P$
\answer{IOE-II Invalid} 
 
\item All $M$ are $P$, and all $S$ are $M$, so some $S$ are $P$. 
\answer{Barbari (AAI-I) (conditionally valid) }
 
\item All $P$ are $M$, and no $M$ are $S$, so no $S$ are $P$. 
\answer{Calemes (AEE-IV) (unconditionally valid) }
 
\item No $P$ are $M$, and all $S$ are $M$, so some $S$ are not $P$.
\answer{Camestros (EAO-1I) }
 
\item All $P$ are $M$, and no $M$ are $S$, so some $S$ are not $P$.  
\answer{Calemos (AEO-IV) }
 
\item All $M$ are $P$, and some $S$ are $M$, so some $S$ are $P$. 
\answer{Darii (AII-1) (unconditionally valid) }
 
\end{exercises}


% *******************************************
% *                  Rules and Fallacies                            *
% *******************************************

\section{Rules and Fallacies}
\label{sec:rules_and_fallacies}
Did you do the exercises in Part \ref{ex_on_syllogism_patterns} of Section \ref{sec:testing_validity}? If you didn't, go back and think about those questions now, before reading any further. The problem set had five general questions like, ``Can a valid argument have only negative premises?'' The point of those questions was to get you to think about what patterns might exist among the 256 Aristotelian syllogisms, and how those patterns might single out 24 syllogisms as the only ones that can be valid. 

In this section, we are going to answer the questions in Part \ref{ex_on_syllogism_patterns} in detail by identifying rules that all valid syllogisms amongst the 256 Aristotelian syllogisms must obey. Seeing these rules will help you understand the \emph{structure} of this part of logic. We aren't just  assigning the labels ``valid'' and ``invalid'' to arguments randomly. Each of the rules we will identify is associated with a fallacy. If you violate the rule, you commit the fallacy. 

\iflabelexists{part:CT}{
The fallacies in this section are like the fallacies that are identified in the parts of the in Part \ref{part:CT_and_informal_logic}in that they represent mistakes in reasoning. If you make an inference that commits one of these fallacies, you have used reasoning incorrectly. However, unlike the fallacies we looked at in the Critical Thinking section, many of these fallacies are not even tempting. They are not ways the human mind goes naturally off the rails. They are just things that you shouldn't do.   
}{}


In the next subsection we are going to outline five basic rules and the fallacies that go with them, along with an addition rule/fallacy pair that can be derived from the initial five. All standard logic textbooks these days use some version of these rules, although they might divide them up differently. Some textbooks also include rules that we have built into our definition of an Aristotelian syllogism in standard form. For instance, other textbooks might have a rule here saying valid syllogisms can't have four terms, or have to use terms in the same way each time. All of this is built into our definitions of an Aristotelian syllogism and standard form for such a syllogism, so we don't need to discuss them here. 

\subsection{Six Rules and Fallacies}
\begin{quotation}
\begin{longtabu}{p{.1\linewidth}p{.9\linewidth}}
\textbf{Rule 1}: & The middle term in a valid Aristotelian syllogism must be distributed at least once. 
\end{longtabu}
\end{quotation}

\newglossaryentry{fallacy of the undistributed middle}
{
name=fallacy of the undistributed middle,
description={A fallacy committed in an Aristotelian syllogism where the middle term is not distributed in either premise.}
}

Consider these two arguments:

\begin{tabu}{p{.5\linewidth}p{.5\linewidth}}

\begin{earg}
\item[P$_1$:] All $M$ are $P$.
\item[P$_2$:] All $S$ are $M$.
\vspace{-.5em}
\item [] \rule{0.4\linewidth}{.5pt} 
\item[C:] All $S$ are $P$.
\end{earg}

&

\begin{earg}
\item[P$_1$:] All $P$ are $M$.
\item[P$_2$:] All $S$ are $M$.
\vspace{-.5em}
\item [] \rule{0.4\linewidth}{.5pt} 
\item[C:] All $S$ are $P$.
\end{earg}

\end{tabu}

The syllogism on the left (Barbara) is obviously valid, but if you change it to figure 2, you get the syllogism on the right, which is obviously invalid. What causes this change?


The premises in the second syllogism say that $S$ and $P$ are both parts of $M$, but they no longer tell us anything about the relationship between $S$ and $P$. To see why this is the case, we need to bring back a term we saw on page \pageref{def:Distribution}, distribution. A term is distributed in a statement if the statement makes a claim about every member of that class. So in ``All $M$ are $P$'' the term $M$ is distributed, because the statement tells us something about every single $M$. They are all also $P$. The term $P$ is not distributed in this sentence, however. We do not know anything about every single $P$. We know that $M$ is in $P$, but not vice versa. 

In general mood-A statements distribute the subject, but not the predicate. This means that when we reverse $P$ and $M$ in the first premise, we create an argument where $S$ and $P$ are distributed, but $M$ is not. This means that the argument is always going to be invalid. 

This short argument can show us that arguments with an undistributed middle are always invalid: The conclusion of an Aristotelian syllogism tries to say something about the relationship between $S$ and $P$. It does this using the relationship those two terms have to the third term $M$. But if $M$ is never distributed, then $S$ and $P$ can be different, unrelated parts of $M$. Therefore arguments with an undistributed middle are invalid. Syllogisms that violate this rule are said to commit the \textsc{\gls{fallacy of the undistributed middle}}. \label{def:undistributed_middle}

\begin{quotation}
\begin{tabu}{p{.1\linewidth}p{.9\linewidth}}
\textbf{Rule 2}: & If a term is distributed in the conclusion of a valid Aristotelian syllogism, then it must also be distributed in one of the premises.  
\end{tabu}
\end{quotation}

Suppose, instead of changing Barbara from a figure 1 to a figure 2 argument, we changed it to a figure 4 argument. This is what we'd get.

\begin{earg}
\item[P$_1$:] All $P$ are $M$.
\item[P$_2$:] All $M$ are $S$.
\vspace{-.5em}
\item [] \rule{0.2\linewidth}{.5pt} 
\item[C:] All $S$ are $P$.
\end{earg}

When we changed the argument from figure 1 to figure 2, it ceased to be valid because the middle became undistributed. But this time the middle is distributed in the second premise, and the argument still doesn't work. You can see this by filling in ``animals,'' ``mammals,'' and ``dogs,'' for $S$, $M$, and $P$. 

\begin{tabu}{p{.02\linewidth}p{.3\linewidth}p{.1\linewidth}p{.58\linewidth}}
P$_1:$ &All dogs are mammals. & & $\Leftarrow$ True \\
P$_2:$ &All mammals are animals. & & $\Leftarrow$ True \\ \cline{1-2}
C: &All animals are dogs. & & $\Leftarrow$ False
\end{tabu}

This version of the argument has true premises and a false conclusion, so you know the argument form must be invalid. \label{counter_example_method_instance} A valid argument form should never be able to take true premises and turn them into a false conclusion. What went wrong here?

The conclusion is a mood-A statement, which means it tries to say something about the entire subject class, namely, that it is completely contained by the predicate class. But that is not what these premises tell us. The premises tell us that the the subject class, animals, is actually the broadest class of the three, containing within it the classes of mammals and dogs.

As with the previous rule, the problem here is a matter of distribution. The conclusion has the subject class distributed. It wants to say something about the entire subject class, animals. But the premises do not have ``animals'' as a distributed class. Premise 1 distributes the class ``dogs'' and premise 2 distributes the class ``mammals.'' 

Here is another argument that makes a similar mistake:

\begin{earg}
\item[P$_1$:] All $M$ are $P$.
\item[P$_2$:] Some $S$ are not $M$.
\vspace{-.5em}
\item [] \rule{0.2\linewidth}{.5pt} 
\item[C:] Some $S$ are not $P$.
\end{earg}

This time the conclusion is a mood-O statement, so the predicate term is distributed. We are trying to say something about the entire class $P$. But again, the premises do not say something about the entire class $P$. $P$ is undistributed in the major premise. 

\newglossaryentry{fallacy of illicit process}
{
name=fallacy of illicit process,
description={A fallacy committed in an Aristotelian syllogism when a term is distributed in the conclusion but is not distributed in the corresponding premise. This fallacy is called ``illicit major'' or ``illicit minor'' depending on which term is not properly distributed. }
}


These examples illustrate rule 2: If a term is distributed in the conclusion, it must also be distributed in the corresponding premise. Arguments that violate this rule are said to commit the \textsc{\gls{fallacy of illicit process}}. \label{def:illicit_process} This fallacy has two versions, depending on which term is not distributed. If the subject term is the one that is not distributed, we say that the argument commits the fallacy of an illicit minor. If the predicate term isn't distributed, we say that the argument commits the fallacy of the illicit major. Some particularly silly arguments commit both.

The justification for this rule is easy enough to see. If the conclusion makes a claim about all of a class, but the premises only make a claim about some of the class, the conclusion clearly says more than what the premises justify. 

\begin{quotation}
\begin{tabu}{p{.1\linewidth}p{.9\linewidth}}
\textbf{Rule 3}: & A valid Aristotelian syllogism cannot have two negative premises.
\end{tabu}\label{rule_3}
\end{quotation}

\newglossaryentry{fallacy of exclusive premises}
{
name=fallacy of exclusive premises,
description={A fallacy committed in an Aristotelian syllogism where both premises are negative.}
}

Back in exercise set \ref{no_conclusion_set1} you were asked to determine what conclusion, if any, could be drawn from a given pair of premises. Some of the exercises involved arguments with two negative premises. Problem \ref{itm:no_conclusion_set1_EEE} went like this: ``No $P$ are $M$, and no $M$ are $S$, therefore \underline{\hspace{2cm}}.'' If you haven't done so already, try to find a conclusion about $S$ and $P$ that you can draw from this pair of premises. 

Hopefully you have convinced yourself that there is no conclusion to be drawn  from the premises above using standard Aristotelian format. No matter what mood you put the conclusion is, it will not follow from the premises. The same thing would be true of any syllogism with two negative premises. We could show this conclusively by running through the 16 possible combinations of negative premises and figures. A more intuitive proof of this rule goes like this: The conclusion of an Aristotelian syllogism must tell us about the relationship between subject and predicate. But if both premises are negative then the middle term must be disjoint, either entirely or partially, from the subject and predicate terms. An argument that breaks this rule is said to commit the \textsc{\gls{fallacy of exclusive premises}}. \label{def:exclusive_premises}


\begin{quotation}
\begin{tabu}{p{.1\linewidth}p{.9\linewidth}}
\textbf{Rule 4}: & A valid Aristotelian syllogism can have a negative conclusion if and only if it has exactly one negative premise.
\end{tabu} \label{rule_4}
\end{quotation}
\label{rule4}

\newglossaryentry{negative-affirmative fallacy}
{
name=negative-affirmative  fallacy,
description={A fallacy committed in an Aristotelian syllogism where the conclusion is negative but both of the premises are positive or the conclusion is affirmative but one or more of the premises is negative.}
}

Again, let's start with examples, and try to see what is wrong with them.

\begin{tabu}{p{.5\linewidth}p{.5\linewidth}}

\begin{earg}
\item[P$_1$:] All $M$ are $P$.
\item[P$_2$:] All $P$ are $M$.
\vspace{-.5em}
\item [] \rule{0.4\linewidth}{.5pt} 
\item[C:] Some $S$ are not $P$.
\end{earg}

&

\begin{earg}
\item[P$_1$:] No $P$ are $M$.
\item[P$_2$:] All $S$ are $M$.
\vspace{-.5em}
\item [] \rule{0.4\linewidth}{.5pt} 
\item[C:] All $S$ are $P$.
\end{earg}

\end{tabu}

These arguments are so obviously invalid, you might look at them and say, ``Sheesh, is there anything \emph{right} about them?'' Actually, these arguments obey all the rules we have seen so far. Look at the left hand argument. Premise 1 ensures that the middle term is distributed. The conclusion is mood O, which means the predicate is distributed, but $P$ is also distributed in the second premise. The argument does not have two negative premises. A similar check will show that the argument on the right also obeys the first three rules. 

 Actually, these arguments illustrate an important premise that is independent of the previous three. You can't draw a negative conclusion from two affirmative premises, and you cannot drawn an affirmative conclusion if there is a negative premise. Because the previous rule tells us that you can never have two negative premises, we can actually state this rule quite simply: an argument can have a negative conclusion if and only if it has exactly one negative premise. (The phrase ``if and only if'' will become important when we get to SL in Chapter \ref{chap:SL}. For now you can just note that ``if and only if'' means that the rule goes both ways. If you have a negative conclusion, then you must have one negative premise, and if you have one negative premise, you must have a negative conclusion.) 

To see why this rule is justified, you need to look at each part of it separately. First, consider the case with the affirmative conclusion. An affirmative conclusion tells us that some or all of $S$ is contained in $P$. The only way to show this is if some or all of $S$ is in $M$, and some or all of $M$ is in $P$. You need a complete chain of inclusion. Therefore if an argument has a negative premise, it cannot have an affirmative conclusion. 

On the other hand, if an argument has a negative conclusion, it is saying that $S$ and $P$ are at least partially separate. But if you have all affirmative premises you are never separating classes. Also, a valid argument cannot have two negative premises. Therefore, a valid argument with a negative conclusion must have exactly one negative premise. 

There is not a succinct name for the fallacy that goes with violating this rule, because this is not a mistake people commonly make. We will call it the \textsc{\gls{negative-affirmative fallacy}}. \label{def:negative-affirmative _fallacy}

\begin{quotation}
\begin{tabu}{p{.1\linewidth}p{.9\linewidth}}
\textbf{Rule 5}: & A valid Aristotelian syllogism cannot have two universal premises and a particular conclusion.
\end{tabu} \label{rule_5}
\end{quotation}


\newglossaryentry{existential fallacy}
{
name=existential fallacy,
description={A fallacy committed in an Aristotelian syllogism where the conclusion is particular but both premises are universal. }
}

This rule is a little different than the previous ones, because it really only applies if you take a Boolean approach to existential import. Consider Barbari, the sometimes maligned step-sister of Barbara:

\begin{earg}
\item[P$_1$:] All $M$ are $P$.
\item[P$_2$:] All $S$ are $M$.
\vspace{-.5em}
\item [] \rule{0.2\linewidth}{.5pt} 
\item[C:] Some $S$ are $P$.
\end{earg} 

This syllogism is not part of the core 15 valid syllogisms we identified with the Venn diagram method using Boolean assumptions about existential import. The premises never assert the existence of something, but the conclusion does. And this is something that is generally true under the Boolean interpretation. Universal statements never have existential import and particular statements always do. Therefore you cannot derive a particular statement from two universal statements. 

Some textbooks act as if the ancient Aristotelians simply overlooked this rule. They say things like ``the traditional account paid no attention to the problem of existential import'' which is simply false. As we have seen, the Latin part of the Aristotelian tradition engaged in an extensive discussion of the issue from the 12th to the 16th centuries, under the heading ``supposition of terms'' \citep{Read2002}. And at least some people, like William of Ockham, had consistent theories that show why syllogisms like Barbari were valid \citep{Parsons2008}. 

In this textbook, we handle the existential import of universal statements by adding a premise, where appropriate, which makes the existence assumption explicit. So Barbari should look like this.

\begin{earg}
\item[P$_1$:] All $M$ are $P$.
\item[P$_2$:] All $S$ are $M$.
\item[P$_3$:] Some $S$ exist.*
\vspace{-.5em}
\item [] \rule{0.2\linewidth}{.5pt} 
\item[C:] Some $S$ are $P$.
\end{earg} 

Adding this premise merely gives a separate line in the proof for an idea that Ockham said was already contained in premise 2. And if we make it a practice of adding existential premises to arguments like these, Rule 5 still holds true. You cannot conclude a particular statement from all universal premises. However in this case, we do have a particular premise, namely, P$_3$. So if we provide this reasonable accommodation, we can see that syllogisms like Barbari are perfectly good members of the valid syllogism family. We will say, however, that an argument like this that does not provide the extra premise commits the 
\textsc{\gls{existential fallacy}}. \label{def:existential_fallacy}


%%%%%%%%%%  Proving the Rules  %%%%%%%%%%%%%

\subsection{Proving the Rules}

For each rule, we have presented an argument that any syllogism that breaks that rule is invalid. It turns out that the reverse is also true. If a syllogism obeys all five of these rules, it must be valid. In other words, these rules are \emph{sufficient} to characterize validity for Aristotelian syllogisms. \nix{(For a discussion of the term ``sufficient'' see page [ref]).} It is good practice to actually walk through a proof that these five rules are sufficient for validity. After all, that sort of proof is what formal logic is really all about. The proof below follows Hurley (\cite*	{Hurley2014}).

Imagine we have a syllogism that obeys the five rules above. We need to show that it must be valid. There are four possibilities to consider: the conclusion is either mood A, mood E, mood I, or mood O. 

If the conclusion is in mood A, then we know that $S$ is distributed in the conclusion. If the syllogism obeys rules 1 and 2, then we know that $S$ and $M$ are distributed in the premises. Rule 4 tells us that both premises must be affirmative, so the premises can't be I or O. They can't be E, either, because E does not distribute any terms, and we know that terms are distributed in the premises. Therefore both premises are in mood A. Furthermore, we know that they are in the first figure, because they have to distribute $S$ and $M$. Therefore the syllogism is Barbara, which is valid. 

Now suppose the conclusion is in mood E. By rule 4, we have one negative and one affirmative premise. Because mood-E statements distribute both subject and predicate, rules 1 and 2 tell us that all three terms must be distributed in the premises. Therefore one premise must be E, because it will have to distribute two terms. Since E is negative, the other premise must be affirmative, and since it has to distribute a term, it can't be I. So we know one premise is A and the other E. If all the terms are distributed, this leaves us four possibilities: EAE-1, EAE-2, AEE-2, and AEE-4. These are the valid syllogisms Celarent, Cesare, Camestres, and Calemes.

Next up, consider the case where the conclusion is in mood I. By rule 4, it has two affirmative premises, and by rule 5 both premises cannot be universal. This means that one premise must be an affirmative particular statement, that is, mood I. But we also know that by rule 1 some premise must distribute the middle term. Since this can't be the mood-I premise, it must be the other premise, which then must be in mood A. Again we are reduced to four possibilities: AII-1,  AII-2, IAI-3, and IAI-4, which are the valid syllogisms Darii, Datisi, Disamis, and Dimatis.  

Finally, we need to consider the case where the conclusion is mood O. Rule 4 tells us that one premise must be negative and the other affirmative, and rule 5 tells us that they can't both be universal. Rules 1 and 2 tell us that $M$ and $P$ are distributed in the premises. This means that the premises can't both be particular, because then one would be I and one would be O, and only one term could be distributed. So one premise must be negative and the other affirmative, and one premise must be particular and the other universal. In other words, our premises must be a pair that goes across the diagonal of the square of opposition, either an A and an O or an E and an I. 

With the AO pair, there are two possibilities that distribute the right terms: OAO-3 and AOO-II. These are the valid syllogisms Bocardo and Baroco. With the EI pair, there are four possibilities, which are all valid. They are the EIO sisters: Ferio, Festino, Ferison, and Fresison. 

So there you have it. Those five rules completely characterize the possible valid Aristotelian syllogisms. Any other patterns you might notice among the valid syllogisms can be derived from these five rules. For instance, Problem \ref{itm:two_particulars} in exercise set \ref{ex_on_syllogism_patterns} of Section \ref{sec:testing_validity} asked if you could have a valid Aristotelian syllogism with two particular premises. If you did that problem, hopefully you saw that the answer was ``no.'' We could, in fact, make this one of our five rules above. But we don't need to. When we showed that these five rules were sufficient to characterize validity, we also showed that any other rule characterizing validity that we care to come up with can be derived from the rules we already set out. So, let's state the idea that a syllogism cannot have two particular premises as a rule, and show how it can be derived. This will be our statement of the rule:

 \begin{quotation}
\begin{tabu}{p{.25\linewidth}p{.75\linewidth}}
\textbf{Derived Rule 1}: &  A valid Aristotelian syllogism cannot have two particular premises. 
\end{tabu}\label{derived_rule_1}
\end{quotation}

\newglossaryentry{fallacy of particular premises}
{
name=fallacy of particular premises,
description={A fallacy committed in an Aristotelian syllogism where both premises are particular.}
}

And let's call the associated fallacy the \textsc{\gls{fallacy of particular premises}}. \label{def:particular_premises} To show that this rule can be derived from the previous five, it is sufficient to show that any syllogism that violates this rule will also violate one of the previous five rules. Thus there will always be a reason, independent of this rule, that can explain why that syllogism is false. 

So suppose we have a syllogism with two particular premises. If we want to avoid violating rule 1, we need to distribute the middle term, which means that both premises cannot be mood I, because mood-I statements don't distribute any term. We also know that both statements can't be mood O, because rule 3 says we can't have two negative premises. Therefore our syllogism has one premise that is I and one premise that is O. It thus has exactly one negative premise, and by rule 4, must have a negative conclusion, either an E or an O. But an argument with premises I and O can only have one term distributed: if the conclusion is mood O, then two terms are distributed; and if it is mood E then all three terms are distributed. Thus any syllogism that manages to avoid rules 1, 3, and 4 will fall victim to rule 2. Therefore any syllogism with two particular premises will violate one of the five basic rules. 

%%%%%%%%% Practice Problems  %%%%%%%%%

\practiceproblems
\noindent \problempart Determine whether the following arguments are valid by seeing if they violate any of the five basic rules. If they are invalid, list the rules they violate. If they are valid, name their form. For conditionally valid arguments, label them valid if the existential premise is given explicitly, and invalid if it is not. 

\begin{longtabu}{p{.15\linewidth}p{.9\linewidth}} 
\textbf{Example 1}: &All $M$ are $P$, and all $S$ are $M$. Therefore no $S$ are $P$.\\ 
\textbf{Answer}: & Invalid. It violates rule 2, because $P$ is distributed in the conclusion but not the premises, and rule 4, because it has a negative conclusion and two affirmative premises.\\ 
\end{longtabu} 

\begin{longtabu}{p{.15\linewidth}p{.9\linewidth}} 
\textbf{Example 2}: & No $P$ are $M$, and all $S$ are $M$. Therefore some $S$ are not $P$.\\ 
\textbf{Answer}: & Invalid. It violates rule 5 because it is missing the existential premise ``Some $S$ exist.''\\ 
\end{longtabu} 


\begin{exercises} 
 
\item Some $M$ are $P$, and some $M$ are $S$. Therefore, no $S$ are $P$.

\answer{ It violates Rule 1 (middle term must be distributed), Rule 2 (terms distributed in the conclusion must be distributed in the premises), Rule 4 negative conclusion if and only if exactly one negative premise), and Derived Rule 1 (At least one universal premise.)}   
      
\item Some $P$ are $M$, and some $M$ are not $S$. Therefore, all $S$ are $P$.

\answer{Invalid. It violates Rule 1 (middle term must be distributed), Rule 4 (negative conclusion if and only if exactly one negative premise) and Derived Rule 1 (At least one universal premise.)}    
   
\item All $P$ are $M$, and no $M$ are $S$. Therefore, no $S$ are $P$.

\answer{ Valid. Calemes (AEE-4.)

\begin{venns}
\shadecomplementred{\predicatecircle}{\predicatesquare}{\middlecircle}
\shadeintersectred{\middlecircle}{\subjectcircle}
\drawsubsyl
\drawmidsyl
\drawpredsyl
\end{venns}
}   

\item Some $P$ are not $M$, some $M$ are $S$. Therefore, all $S$ are $P$.

\answer{Invalid. It violates Rule 2 (terms distributed in the conclusion must be distributed in the premises) and Derived Rule 1 (At least one universal premise.)}   


\item No $M$ are $P$, and all $S$ are $M$. Also, some $S$ exist. Therefore some $S$ are not $P$.

\answer{ Valid. Celeront (EAO-1). Notice that this does not violate rule 5 because the premise ``Some $S$ exist'' was included.

\begin{venns}
\shadeintersectred{\middlecircle}{\predicatecircle}
\shadecomplementred{\subjectcircle}{\subjectsquare}{\middlecircle}

\drawsubsyl
\drawmidsyl
\drawpredsyl
\end{venns}
}


\item All $P$ are $M$, and no $S$ are $M$. Therefore some $S$ are not $P$.

\answer{Invalid. This time we violate rule 5, because the existential premise was not included. If it were included, it would be Camestros (AEO-2)}

\item Some $M$ are $P$, and all $M$ are $S$. Therefore some $S$ are $P$.  
\answer{Valid. Disamis (IAI-3)
\begin{venns}
\shadecomplementred{\middlecircle}{\middlesquare}{\subjectcircle}
\drawsubsyl
\drawmidsyl
\drawpredsyl
\someexistseven
\end{venns}
}
 
\item All $M$ are $P$, and all $S$ are $M$. Therefore some $S$ are not $P$.

\answer{Invalid. It violates Rule 2 (terms distributed in the conclusion must be distributed in the premises, Rule 4 (a negative conclusion requires exactly one negative premise), and Rule 5 (a particular conclusion must have at least one particular premise.)}
      
\item Some $M$ are not $P$, and all $S$ are $M$. Therefore, some $S$ are $P$.
 
\answer{Invalid. It violates Rule 1 (middle term must be distributed), and Rule 4 (a valid syllogism has a negative conclusion if and only if it has one negative premise.)} 
 
\item Some $P$ are $M$, and some $M$ are not $S$. Therefore some $S$ are not $P$.
 
\answer{Invalid. It violates Rule 2 (terms distributed in the conclusion must be distributed in the premises) and Derived Rule 1 (No two particular premises) } 
 
\end{exercises}
      
\noindent \problempart Determine whether the following arguments are valid by seeing if they violate any of the five basic rules. If they are invalid, list the rules they violate. If they are valid, name their form. For conditionally valid arguments, label them valid if the existential premise is given explicitly, and invalid if it is not. 

\begin{exercises} 
 
\item Some $M$ are not $P$, and no $S$ are $M$. Therefore, all $S$ are $P$.
 
\item No $M$ are $P$, and some $S$ are $M$. Therefore, some $S$ are not $P$.
 
\item All $P$ are $M$, and no $S$ are $M$. Therefore no $S$ are $P$. 
 
\item All $P$ are $M$, and all $M$ are $S$. Also, some $S$ exist. Therefore some $S$ are $P$. 
  
\item All $P$ are $M$, and no $S$ are $M$. Therefore some $S$ are not $P$. 
 
\item All $M$ are $P$, and no $M$ are $S$. Therefore, some $S$ are $P$.
 
\item No $P$ are $M$, and all $M$ are $S$. Therefore, some $S$ are not $P$.
 
\item Some $M$ are not $P$, and some $M$ are $S$. Therefore, some $S$ are not $P$.
  
\item Some $M$ are $P$, and all $M$ are $S$. Therefore, some $S$ are not $P$.
 
\item No $P$ are $M$, and no $M$ are $S$. Therefore no $S$ are $P$.
 
\end{exercises}

%
%  This is the opening of the conditional formatting tag for typesetting only part of this chapter. Everything from here to the close tag will be skipped unless the {whole_syl_chap} label at the
%  start of this chapter is uncommented.
%

\iflabelexists{whole_syl_chap}{



% ********************************************
% *	  Validity and the Counterexample Method         *
% ********************************************

\section{Validity and the Counterexample Method} 
\label{sec:counterexample}

Except for a brief discussion of logically structured English in section \ref{sec:form_mood_figure}, so far, we have only been evaluating arguments that use variables for the subject, middle, and predicate terms. Now, we will be looking in detail at issues that come up when we try to evaluate categorical arguments that come up in ordinary English. In this section we will consider how your knowledge of the real world terms discussed in an argument can distract you from evaluating the form of the argument itself. In this section we will consider difficulties in putting ordinary language arguments into logically structured English, so they can be analyzed using the techniques we have learned. 

Let's go back again to the definition of validity on page \pageref{def:valid}. (It is always good for beginning student to reinforce their understanding of validity). We did this in the last chapter in Section \ref{sec:ESA}, and we are doing it again now. \label{valid_definition_reinforcement} Validity is a fundamental concept in logic that can be confusing. A valid argument is not necessarily one with true premises or a true conclusion. An argument is valid if the premises \emph{would} make the conclusion true \emph{if} the premises were true.

This means that, as we have seen before, there can be valid arguments with false conclusions. Consider this argument:

\begin{quotation}\noindent No reptiles are chihuahuas. But all dogs are reptiles. Therefore, no dogs are chihuahuas. \end{quotation}

This seems silly, if only because the conclusion is false. We know that some dogs are chihuahuas. But the argument is still valid. In fact, it shares a form with an argument that makes perfect sense:

\begin{quotation}\noindent No reptiles are dogs, but all chameleons are reptiles. Therefore, no dogs are chameleons. \end{quotation} 

Both of these arguments have the form Celarent: 

\begin{earg}
\item[P$_1$:] No $M$ are $P$.
\item[P$_2$:] All $S$ are $M$.
\vspace{-.5em}
\item [] \rule{0.15\linewidth}{.5pt} 
\item[C:] No $S$ are $P$.
\end{earg} 

This form is valid, whether the subject and predicate term are dogs and chameleons, or dogs and chihuahuas, which you can see from this Venn diagram.

\begin{center}
\begin{tikzpicture}
\def\firstcircle{(0,0) circle (1cm)}
\def\secondcircle{(60:1.25cm) circle (1cm)}
\def\thirdcircle{(0:1.25cm) circle (1cm)}

\begin{scope} 
\clip \thirdcircle;
\fill[gray] \secondcircle;
\end{scope}

\begin{scope}[even odd rule] % Shade P without M
\clip \secondcircle (-1,-1) rectangle (2,2);
\fill[gray] \firstcircle;
\end{scope}

\draw \firstcircle node[outer sep=.8cm, below left] {$S$};
\draw \secondcircle node [outer sep=1cm, above] {$M$};
\draw \thirdcircle node [outer sep=.8cm, below right] {$P$};
\end{tikzpicture}
\end{center}

This means you can't assume an argument is invalid because it has a false conclusion. The reverse is also true. You can't assume an argument is valid just because it has a true conclusion. Consider this 

\begin{quotation}  All cats are animals, and some animals are dogs. Therefore no dogs are cats. \end{quotation}

Makes sense, right? Everything is true. But the argument isn't valid. The premises aren't making the conclusion true. Other arguments with the same form have true premises and a false conclusion. Like this one.

\begin{quotation}All chihuahuas are animals, and some animals are dogs. Therefore, no dogs are chihuahuas.\end{quotation}

Really, the arguments in both these passages have the same form: AEE-IV: 

\begin{earg}
\item[P$_1$:] All $P$ are $M$.
\item[P$_2$:] Some $M$ are $S$.
\vspace{-.5em}
\item [] \rule{0.2\linewidth}{.5pt} 
\item[C:] No $S$ are $P$.
\end{earg} 

This in an invalid form, and it remains invalid whether the $P$ stands for cats or chihuahuas. You can see this in the Venn diagram:

\begin{center}
\begin{tikzpicture}
\def\firstcircle{(0,0) circle (1cm)}
\def\secondcircle{(60:1.25cm) circle (1cm)}
\def\thirdcircle{(0:1.25cm) circle (1cm)}

\begin{scope}[even odd rule] % Shade P without M
\clip \secondcircle (-1,-1) rectangle (2.5,2);
\fill[gray] \thirdcircle;
\end{scope}

\draw \firstcircle node[outer sep=.8cm, below left] {$S$};
\draw \secondcircle node [outer sep=1cm, above] {$M$};
\draw \thirdcircle node [outer sep=.8cm, below right] {$P$}
	node[xshift=-.8cm, yshift=.45cm, fill=white](6){x};
\end{tikzpicture}
\end{center}

All these examples bring out an important fact about the kind of logic we are doing in this chapter and the last one: this is \emph{formal} logic. As we discussed on page \pageref{def:Formal_logic} formal logic is a way of making our investigation content neutral. By using variables for terms instead of noun phrases in English we can show that certain ways of arguing are good or bad regardless of the topic being argued about. \iflabelexists{part:formal_logic}{This method will be extended in Chapters \ref{chap:SL} and \ref{chap:QL}, when we introduce our full formal languages SL and QL. } 
{Parts \iflabelexists{part:cat_logic}{This method will be extended in Part \ref{part:sent_logic}, when we introduce the full formal language SL} 
{}}


The examples above also show us another way of proving that an argument given in English is invalid, called the counterexample method. As we have just seen, if you are given an argument in English with, say, false premises and a false conclusion, you cannot determine immediately whether the argument is valid. However, we can look at arguments that have the same form, and use them to see whether the argument is valid. If we can find an argument that has the exact same form as a given argument but has true premises and a false conclusion, then we know the argument is invalid. We just did that with the AEE-IV argument above. We were given an argument with true premises and a true conclusion involving cats, dogs, and animals. We were able to show this argument invalid by finding an argument with the same form that has true premises and a false conclusion, this time involving chihuahuas, dogs, and animals. 


\newglossaryentry{counterexample method}
{
name=counterexample method,
description={A method for determining whether an argument with ordinary English words for terms is valid. One consistently substitutes other English terms for the terms in the given argument to see whether one can find an argument with the same form that has true premises and a false conclusion.}
}

More precisely, we can define the \textsc{\gls{counterexample method}} \label{def:counter_example_method} as a method for determining if an argument with ordinary English words for terms is valid, where one consistently substitutes other English terms for the terms in the given argument to see if one can find an argument with the same form that has true premises and a false conclusion. Let's run through a couple more examples to see how this works. 

First consider this argument in English:

\begin{quotation}
\noindent  All tablet computers are computers. We know this because a computer is a kind of machine, and some machines are not tablet computers. 
\end{quotation}

Every statement in this argument is true, but it doesn't seem right. The premises don't really relate to the conclusion. That means you can probably find an argument with the same form that has true premises and a false conclusion. Let's start by putting the argument in canonical form. Notice that the English passage had the conclusion first. 

\begin{earg} 
\item[P$_1$:] All computers are machines.
\item[P$_2$:] Some machines are not tablet computers.
\vspace{-.5em} 
 \item [] \rule{0.4\linewidth}{.5pt} 
\item[C:] All tablet computers are computers.
 \end{earg}

Let's find substitutes for ``machines,'' ``computers,'' and ``tablet computers'' that will give us true premises and a false conclusion. It is easiest to work with really common sense categories, like ``dog'' and ``cat.'' It is also easiest to start with a false conclusion and then try to find a middle term that will give you true premises. ``All dogs are cats'' is a nice false conclusion to start with:  

\begin{earg} 
\item[P$_1$:] All cats are $M$.
\item[P$_2$:] Some $M$ are not dogs. 
\vspace{-.5em} 
 \item [] \rule{0.3\linewidth}{.5pt} 
\item[C:] All dogs are cats.
 \end{earg}

So what can we substitute for $M$ (which used to be ``machines'') that will make P$_1$ and P$_2$ true? ``Animals'' works fine.

\begin{earg} 
\item[P$_1$:] All cats are animals.
\item[P$_2$:] Some animals are not dogs. 
\vspace{-.5em} 
 \item [] \rule{0.3\linewidth}{.5pt} 
\item[C:] All dogs are cats.
 \end{earg}

There you have it: a counterexample that shows the argument invalid. Let's try another one. 

\begin{quotation}
Some diseases are viral, therefore some things caused by bacteria are not things that are caused by viruses, because all diseases are bacterial.
\end{quotation}

This will take a bit more unpacking. You can see from the indicator words that the conclusion is in the middle. We also have to fix ``viral'' and ``things that are caused by viruses'' so they match, and the same is true for ``bacterial'' and ``things that are caused by bacteria.'' Once we have the sentences in canonical form, the argument will look like this:

\begin{earg} 
\item[P$_1$:] Some diseases are things caused by viruses.
\item[P$_2$:] All diseases are things causes by bacteria.
\vspace{-.5em} 
 \item [] \rule{0.7\linewidth}{.5pt} 
\item[C:] Some things caused by bacteria are not things caused by viruses.
 \end{earg}

Once you manage to think through the complicated wording here, you can see that P$_1$ and the conclusion are true. Some diseases come from viruses, and not everything that comes from a bacteria comes from a virus. But P$_2$ is false. All diseases are not caused by bacteria. In fact, P$_1$ contradicts P$_2$. But none of this is enough to show the argument is invalid. To do that, we need to find an argument with the same form that has true premises and a false conclusion. 

Let's go back to the simple categories: ``dogs,'' ``animals,'' etc. We need a false conclusion. Let's go with ``Some dogs are not animals.'' 

\begin{earg} 
\item[P$_1$:] Some $M$ are dogs.
\item[P$_2$:] All $M$ are animals.
\vspace{-.5em} 
 \item [] \rule{0.3\linewidth}{.5pt} 
\item[C:] Some dogs are not animals.
 \end{earg}

We need a middle term that will make the premises true. It needs to be a class that is more general than ``dogs'' but more narrow than ``animals.'' ``Mammals'' is a good standby here.

\begin{earg} 
\item[P$_1$:] Some mammals are dogs.
\item[P$_2$:] All mammals are animals.
\vspace{-.5em} 
 \item [] \rule{0.3\linewidth}{.5pt} 
\item[C:] Some dogs are not animals.
 \end{earg}

And again we have it, a counterexample to the given syllogism. 

\practiceproblems

\noindent\problempart Each of the following arguments are invalid. Prove that they are invalid by providing a counterexample that has the same form, but true premises and a false conclusion. 
\begin{longtabu}{p{.1\linewidth}p{.9\linewidth}} 
\textbf{Example}: & No pants are shirts, and no shirts are dresses, so some pants are not dresses. \\ 
\textbf{Answer}: & No dogs are reptiles, and no reptiles are mammals, so some dogs are not mammals.   \\ 
\end{longtabu} 

%\begin{earg} 
%\item[P$_1$:] No $P$ are $M$
%\item[P$_2$:] No $M$ are $S$
%\vspace{-.5em} 
% \item [] \rule{0.6\linewidth}{.5pt} 
%\item[C:] Some $S$ are not $P$
% \end{earg}
% EEO-IV Invalid Conclusion First 
% 



\begin{exercises} 

\item Some mammals are animals, and some dogs are mammals. Therefore, all dogs are animals.  

\answer{
\begin{longtabu}{X[1,l,p]X[1,l,p]}
\begin{earg} 
\item[P$_1$:] Some $M$ are $P$
\item[P$_2$:] Some $S$ are $M$
\vspace{-.5em} 
 \item [] \rule{0.6\linewidth}{.5pt} 
\item[C:] All $S$ are $P$
 \end{earg}
 
& 
\begin{earg*}
\item  Some mammals are chihuahuas 
\item Some dogs are mammals
\itemc All dogs are chihuahuas
\end{earg*}
\end{longtabu}
}
 
\item Some evergreens are not spruces, and all spruces are trees. Therefore some trees are not evergreens.

\answer{
\begin{longtabu}{X[1,l,p]X[1,l,p]}

\begin{earg} 
\item[P$_1$:] Some $P$ are not $M$
\item[P$_2$:] All $M$ are $S$
\vspace{-.5em} 
 \item [] \rule{0.6\linewidth}{.5pt} 
\item[C:] Some $S$ are not $P$
 \end{earg}
&
 \begin{earg*}
\item Some plants are not spruces
\item All spruces are trees. 
\itemc Some trees are not plants.
\end{earg*}
\end{longtabu}
}

\item No wild animals are completely safe to be around, and some pigs are not completely safe to be around. Therefore some pigs are wild. 
\answer{
\begin{longtabu}{X[1,l,p]X[2,l,p]}

\begin{earg} 
\item[P$_1$:] No $P$ are $M$
\item[P$_2$:] Some $S$ are not $M$
\vspace{-.5em} 
 \item [] \rule{0.6\linewidth}{.5pt} 
\item[C:] Some $S$ are $P$
 \end{earg}
&
\begin{earg*} 
\item No explosives are completely safe to be around
\item Some pigs are not completely safe to be around. 
\itemc Some pigs are explosives. 
\end{earg*} 
\end{longtabu}
} 

\item Some sodas are clear, and some healthy drinks are clear. Therefore no sodas are healthy.
\answer{
\begin{longtabu}{X[1,l,p]X[1,l,p]}
\begin{earg} 
\item[P$_1$:] Some $P$ are $M$
\item[P$_2$:] Some $S$ are $M$
\vspace{-.5em} 
 \item [] \rule{0.6\linewidth}{.5pt} 
\item[C:] No $S$ are $P$
 \end{earg}
 & 
\begin{earg*}
\item Some beverages are clear
\item Some healthy drinks are clear.
\itemc No beverages are healthy. 
\end{earg*}
\end{longtabu}
}

\item No hamburgers are rocks, no igneous rocks are hamburgers. Therefore all igneous rocks are rocks.
\answer{
\begin{longtabu}{X[1,l,p]X[1,l,p]}

\begin{earg} 
\item[P$_1$:] No $M$ are $P$
\item[P$_2$:] No $S$ are $M$
\vspace{-.5em} 
 \item [] \rule{0.6\linewidth}{.5pt} 
\item[C:] All $S$ are $P$
 \end{earg}
&
\begin{earg*}
\item No hamburgers are cats
\item No dogs are hamburgers. 
\itemc All dogs are cats
\end{earg*}
\end{longtabu}
}

\item Some plants are not food, and some foods are vegetables. Therefore some plants are vegetables.
\answer{
\begin{longtabu}{X[1,l,p]X[1,l,p]}

\begin{earg} 
\item[P$_1$:] Some $P$ are not $M$
\item[P$_2$:] Some $M$ are $S$
\vspace{-.5em} 
 \item [] \rule{0.6\linewidth}{.5pt} 
\item[C:] Some $S$ are $P$
 \end{earg}
&
\begin{earg*}
\item Some plants are not food
\item Some foods are animal products. 
\itemc Some plants are animal products
\end{earg*}
\end{longtabu}
}
\item All apartment buildings are buildings, and some buildings are residential buildings. Therefore some residential buildings are not apartment buildings.
\answer{
\begin{longtabu}{X[1,l,p]X[1,l,p]}
\begin{earg} 
\item[P$_1$:] All $P$ are $M$
\item[P$_2$:] Some $M$ are $S$
\vspace{-.5em} 
 \item [] \rule{0.6\linewidth}{.5pt} 
\item[C:] Some $S$ are not $P$
 \end{earg}
 &

\begin{earg*}
\item All dogs are mammals
\item  Some mammals are animals
\itemc Some dogs are not animals. 
\end{earg*}

\end{longtabu}
}

\item No foods are games, and some games are video games. Therefore no video games are food.
\answer{
\begin{longtabu}{X[1,l,p]X[1,l,p]}

\begin{earg} 
\item[P$_1$:] No $P$ are $M$
\item[P$_2$:] Some $M$ are $S$
\vspace{-.5em} 
 \item [] \rule{0.6\linewidth}{.5pt} 
\item[C:] No $S$ are $P$
 \end{earg}

&
\begin{earg*}
\item No mammals are reptiles
\item Some reptiles are pets
\itemc No pets are mammals
\end{earg*}
\end{longtabu}
}
\item Some trucks are rentals, and no cars are trucks. Therefore some cars are rentals.
\answer{
\begin{longtabu}{X[1,l,p]X[1,l,p]}

\begin{earg} 
\item[P$_1$:] Some $M$ are $P$
\item[P$_2$:] No $S$ are $M$
\vspace{-.5em} 
 \item [] \rule{0.6\linewidth}{.5pt} 
\item[C:] Some $S$ are $P$
 \end{earg}

&
\begin{earg*}
\item Some plants are food
\item No cars are plants.
\itemc Some cars are food.  
\end{earg*}
\end{longtabu}
}

\item Some online things are not fun, and some fun things are not video games. Therefore, some video games are online.
\answer{
\begin{longtabu}{X[1,l,p]X[1,l,p]}

\begin{earg} 
\item[P$_1$:] Some $P$ are not $M$
\item[P$_2$:] Some $M$ are not $S$
\vspace{-.5em} 
 \item [] \rule{0.6\linewidth}{.5pt} 
\item[C:] Some $S$ are $P$
 \end{earg}
&
\begin{earg*}
\item Some foods are not fun
\item Some fun things are not video games.
\vspace{-1em} 
\itemc Some video games are food.
\end{earg*}
\end{longtabu}
}
\end{exercises} 

\noindent\problempart Each of the following arguments are invalid. Prove that they are invalid by providing a counterexample that has the same form, but true premises and a false conclusion. 

\begin{exercises}      
 

\item All cars are machines, and all vehicles are machines. Therefore all cars are vehicles.  
\answer{
\begin{earg} 
\item[P$_1$:] All $P$ are $M$
\item[P$_2$:] All $S$ are $M$
\vspace{-.5em} 
 \item [] \rule{0.6\linewidth}{.5pt} 
\item[C:] All $S$ are $P$
 \end{earg}
 AAA-II Invalid 
 
 All cars are machines and all trucks are machines. Therefore all cars are trucks.}
 

\item Some black animals are panthers. No panthers are sheep. Therefore some sheep are black. 

\answer{Some carnivorous animals are panthers. No panthers are sheep. Some sheep are carnivorous.
\begin{earg} 
\item[P$_1$:] Some $P$ are $M$
\item[P$_2$:] No $M$ are $S$
\vspace{-.5em} 
 \item [] \rule{0.6\linewidth}{.5pt} 
\item[C:] Some $S$ are $P$
 \end{earg}
 IEI-IV Invalid }


\item Some paper money is not counterfeit, and some counterfeit monies are quarters. Therefore no quarters are paper money.
\answer{
\begin{earg}
\item[P$_1$:] Some $P$ are not $M$
\item[P$_2$:] Some $M$ are $S$
\vspace{-.5em} 
 \item [] \rule{0.6\linewidth}{.5pt} 
\item[C:] No $S$ are $P$
 \end{earg}
 OIE-IV Invalid 
 
Some legal tender is not counterfeit, and some counterfeit monies are quarters. Therefore no quarters are legal tender.}
 
\item Some spacecraft are man-made, and no comets are man-made. Therefore some comets are not spacecraft.
\answer{
\begin{earg} 
\item[P$_1$:] Some $P$ are $M$
\item[P$_2$:] No $S$ are $M$
\vspace{-.5em} 
 \item [] \rule{0.6\linewidth}{.5pt} 
\item[C:] Some $S$ are not $P$
 \end{earg}
 IEO-II Invalid 
: Some things made of rock and ice are man-made, and no comets are man-made. Therefore some comets are not things made of rock and ice}
 
 
\item No planets are stars, and no planets are fusion reactions. Therefore all stars are fusion reactions.

\answer{No planets are starts and no planets are asteroids. Therefore all stars are asteroids. 
\begin{earg} 
\item[P$_1$:] No $M$ are $P$
\item[P$_2$:] No $M$ are $S$
\vspace{-.5em} 
 \item [] \rule{0.6\linewidth}{.5pt} 
\item[C:] All $S$ are $P$
 \end{earg}
 EEA-III Invalid }
 
\item All straight lines are lines, and no straight lines are curved lines. Therefore, all curved lines are lines.

\answer{All straight lines are lines, and no cows are curved lines. Therefore, all cows are lines.
\begin{earg} 
\item[P$_1$:] All $M$ are $P$
\item[P$_2$:] No $M$ are $S$
\vspace{-.5em} 
 \item [] \rule{0.6\linewidth}{.5pt} 
\item[C:] All $S$ are $P$
 \end{earg}
 AEA-III Invalid }
 
\item Some physical objects are not writing implements, and all pencils are writing implements. Therefore, all pencils are physical objects.

\answer{Some foods are not writing implements, and all pencils are writing implements. Therefore, all pencils are food.

\begin{earg} 
\item[P$_1$:] Some $P$ are not $M$
\item[P$_2$:] All $S$ are $M$
\vspace{-.5em} 
 \item [] \rule{0.6\linewidth}{.5pt} 
\item[C:] All $S$ are $P$
 \end{earg}
 OAA-II Invalid }
 
\item No traumatic memories are pleasant, but some memories are traumatic. Therefore, some memories are pleasant.

\answer{No chihuahuas are cats, but some dogs are chihuahuas. Therefore, some dogs are cats
\begin{earg} 
\item[P$_1$:] No $M$ are $P$
\item[P$_2$:] Some $S$ are $M$
\vspace{-.5em} 
 \item [] \rule{0.6\linewidth}{.5pt} 
\item[C:] Some $S$ are $P$
 \end{earg}
 EII-I Invalid }

\item Some farms are for-profit enterprises, and all munitions factories are for-profit enterprises. Therefore, no munitions factories are farms.

\answer{Some mammals are animals, and all dogs are animals, so no dogs are mammals 
\begin{earg} 
\item[P$_1$:] Some $P$ are $M$
\item[P$_2$:] All $S$ are $M$
\vspace{-.5em} 
 \item [] \rule{0.6\linewidth}{.5pt} 
\item[C:] No $S$ are $P$
 \end{earg}
 IAE-II Invalid }

\item No parasites are western brook lampreys, and all western brook lampreys are lampreys. Therefore some lampreys are parasitic. 

\answer{
\begin{earg} 
\item[P$_1$:] No $P$ are $M$
\item[P$_2$:] All $M$ are $S$
\vspace{-.5em} 
 \item [] \rule{0.6\linewidth}{.5pt} 
\item[C:] Some $S$ are $P$
 \end{earg}
 EAI-IV }
   
         
\end{exercises}

% *******************************************
% *               Ordinary Language Arguments             *
% *******************************************

\section{Ordinary Language Arguments}

In the last section we saw that arguments in ordinary language can throw us off because our knowledge of the truth or falsity of the statements in them can cloud our perception of the validity of the argument. Now we are going to look at ways we might need to transform ordinary language arguments in order to put them in standard form (see p. \pageref{standard_form_for_an_Aristotelian_syllogism}) and use our evaluative tools.  
%%%%%%%%%% Basic Transformations in to LSE  %%%%%%%%

\subsection{Basic Transformations into Logically Structured English}

Arguments are composed of statements, so all of the guidelines we discussed for putting ordinary English statements into logically structured English in Section \ref{sec:transformation} apply here. Also, as you saw in the exercises in Section \ref{sec:form_mood_figure}, we need to be aware that the conclusion of an ordinary language argument can occur anywhere in the passage. Consider this argument

\begin{quotation}
\noindent Every single penguin in the world is a bird. Therefore, at least one bird in the world is not a person, because not one person is a penguin.
\end{quotation}

The indicator words let us know the conclusion is the sentence in the middle of this passage, ``At least one bird in the world is not a person.'' The quantifier in this sentence is nonstandard. As we saw on page \pageref{subsec:nonstandard_quantifiers} ``at least one'' should be changed to ``some.'' That would give us ``Some birds are not people'' for the conclusion. Similar changes need to be made to the two premises. Thus the argument and translation key look like this: 

\begin{tabu}{{X[1,l,p]X[1,l,p]}}


\begin{ekey}
\item[$S$:] Birds
\item[$M$:] Penguins
\item[$P$:] People
\end{ekey}

&

\begin{earg}
\item[P$_1$:] No $P$ are $M$. 
\item[P$_2$:] All $M$ are $S$.
\vspace{-.5em}
\item [] \rule{0.4\linewidth}{.5pt} 
\item[C:] Some $S$ are not $P$.
\end{earg} 

\end{tabu}

Now that we have it in this form, we can see that it is the argument Fesapo (EAO-4), which is valid if you add the premise that some penguins exist.

Another issue that can come up in ordinary language arguments is the use of synonyms. Consider another argument about penguins:

\begin{quotation}
\noindent No Ad\'{e}lie penguins can fly, because all members of \emph{Pygoscelis adeliae} are members of the family Sphenisciformes, and no Sphenisciformes can fly.
\end{quotation}

You might think this argument can't be a proper syllogism, because it has four terms in it: Ad\'{e}lie penguins, \emph{Pygoscelis adeliae}, Sphenisciformes, and things that can fly. However, a quick trip to Wikipedia can confirm that \emph{Pygoscelis adeliae} is just the scientific name from the Ad\'{e}lie penguin, and for that matter ``Sphenisciformes'' are just penguins. So really, the argument just has three terms, and there is no problem representing the argument  like this:

\begin{tabu}{{X[1,l,p]X[1,l,p]}}


\begin{ekey}
\item[$S$:] Ad\'{e}lie penguins
\item[$M$:] Penguins
\item[$P$:] Things that can fly 
\end{ekey}

&

\begin{earg}
\item[P$_1$:]  No $M$ are $P$.
\item[P$_2$:] All $S$ are $M$.
\vspace{-.5em}
\item [] \rule{0.4\linewidth}{.5pt} 
\item[C:] No $S$ are $P$.
\end{earg} 

\end{tabu}

And this is our good friend Celarent (EAE-1). 

Generally speaking, arguments involving different names for the same person can work the same way as synonyms. Consider this argument:

\begin{quotation}
\noindent Bertrand Russell was a philosopher. We know this because Mr. Russell was a logician, and all logicians are philosophers. 
\end{quotation}

``Bertrand Russell'' and ``Mr. Russell'' are names for the same person, so we can represent them using the same variable. Back on page \pageref{subsec:singular_propositions} we learned that names for individuals need to be transformed into classes using phrases like ``objects identical with \ldots'' or ``people identical with \ldots.''  Thus the argument turns out to be Celarent's best friend, Barbara (AAA-I):

\begin{tabu}{{X[1.25,l,p]X[1,l,p]}}


\begin{ekey}
\item[$S$:] People identical with Bertrand Russell
\item[$M$:] Logicians
\item[$P$:] Philosophers 
\end{ekey}

&

\begin{earg}
\item[P$_1$:]  All $M$ are $P$.
\item[P$_2$:] All $S$ are $M$.
\vspace{-.5em}
\item [] \rule{0.4\linewidth}{.5pt} 
\item[C:] All $S$ are $P$.
\end{earg} 

\end{tabu}

Merging proper nouns into one variable doesn't always work the way merging synonyms does, however. Sometimes exchanging one name with another can change the truth value of a sentence, even when the two names refer to the same person. Any comic book fan will tell you that the sentence ``J. Jonah Jameson hates Spider-Man'' is true. Also, ``Peter Parker'' and ``Spider-Man'' refer to the same person. However ``J. Jonah Jameson hates Peter Parker'' is false. These situations are called ``intentional contexts.'' We won't be dealing with them in this textbook, but you should be aware that they exist. 

Another thing to be aware of when you assign variables to terms is that often you will need to find the right general category that will allow your terms to hook up with each other. \label{finding_general_terms} Consider this argument.

\begin{quotation}
Bertrand Russell did not believe in God. I know this because Mr. Russell was not a Christian, and all Christians believe in God
\end{quotation}

Here again we need to use one variable for both ``Bertrand Russell'' and ``Mr. Russell.'' But we also have to change the verb phrase ``believes in God'' into a predicate term (see page \pageref{subsec:nonstandard_verbs}). The main trick here is to transform the verb phrase ``believes in God'' and the singular terms referring to Bertrand Russell in a way that will match. The trick is to use the word ``people'' each time. ``Believes in God'' becomes ``people who believe in God,'' and the two names for Bertrand Russell need to become ``people identical with Bertrand Russell.'' By including the word ``people'' in each case, we ensure that the terms will be comparable. 
 
\begin{tabu}{{X[2,l,p]X[1,l,p]}}

\begin{ekey}
\item[$S$:] People identical to Bertrand Russell
\item[$M$:] Christians
\item[$P$:] People who believe in God
\end{ekey}

&

\begin{earg}
\item[P$_1$:] All $M$ are $P$.
\item[P$_2$:] No $S$ are $M$.
\vspace{-.5em}
\item [] \rule{0.5\linewidth}{.5pt} 
\item[C:] No $S$ are $P$.
\end{earg} 

\end{tabu}

The argument here is AEE-1, which is invalid. It violates rule 2, because $P$ is distributed in the conclusion, but not the premises. More informally, we can say that just because Russell wasn't a Christian doesn't mean he was an atheist or an agnostic. (In fact, Russell said that he was an atheist in some ways, and an agnostic in others, but this argument doesn't show that.) 

%%%%%%%%%%%% Complement classes, etc. %%%%

\subsection{Complement Classes, Obversion, and Contraposition}

In the last subsection, we saw that we can represent synonyms using the same variable, thus reducing the number of terms in an argument down to the three terms required for an Aristotelian syllogism. Something similar can be done when you have terms that are complements of other terms. Back on page \pageref{def:Complement} we said that the complement of a class was the class of things that are not in the original class. So the complement of the class ``penguins'' is everything that is not a penguin. Sometimes, if you have an argument with more than three terms, you can reduce the number of terms down to three, because some of the terms are complements of each other, like ``penguins'' and ``non-penguins.''

 Consider the following argument, which talks about the stoics, an ancient philosophical school and pioneers in logical thinking. \nix{[insert historical sidebar ref]}

\begin{quotation}
No stoics are non-logicians, and all non-philosophers are non-stoics. Therefore all logicians are philosophers.
\end{quotation}

This argument has six terms: stoics, logicians, philosophers, non-stoics, non-logicians, and non-philosophers. Since the second three are all complements of the first, we should be able to reduce the number of terms to three. Once we do this, we might find that the argument is valid---it looks like it might be Barbara. We can begin by assigning $S$, $M$, and $P$ to the terms ``stoics,'' ``logicians,'' and ``philosophers.'' We can then represent the other six terms as non-$S$, non-$M$, and non-$P$. 


\begin{tabu}{{X[1,l,p]X[1,l,p]}}

\begin{ekey}
\item[$S$:] Logicians
\item[$M$:] Stoics
\item[$P$:] Philosophers
\end{ekey}

&

\begin{earg}
\item[P$_1$:]  No $M$ are non-$P$.
\item[P$_2$:] All non-$S$ are non-$M$.
\vspace{-.5em}
\item [] \rule{0.6\linewidth}{.5pt} 
\item[C:] All $S$ are $P$.
\end{earg} 

\end{tabu}

The argument still has six terms, so we can't evaluate it as is. However, we can get rid of three of these terms. The secret is to apply two of the transformation tools we learned about in Section \ref{sec:conv_obv_cont}, obversion and contraposition. In that section, we saw that there were different ways you could alter categorical statements that would sometimes yield a logically equivalent statement. Two of those methods---obversion and contraposition---involved taking the complements of existing terms. In the cases where these transformations yield logically equivalent terms, we can use them to change the premises of an argument into statements that lack the terms we want to get rid of. 

In obversion, you change the quality of a sentence---switch it from affirmative to negative or vice versa---and then replace the predicate term with its complement. This process always yields a logically equivalent statement, so we can always use it. Applying obversion to the first premise of the argument above allows us to eliminate the term non-$P$.  

\begin{earg}
\item[P$_1$:]  All $M$ are $P$.
\item[P$_2$:] All non-$S$ are non-$M$.
\vspace{-.5em}
\item [] \rule{0.25\linewidth}{.5pt} 
\item[C:] All $S$ are $P$.
\end{earg} 

We can also use contraposition to eliminate terms, but here we must be more careful. Contraposition only yields logically equivalent statements when it is applied to mood-A or mood-O statements. Fortunately, Premise 2 is a mood-A statement, so we can change the argument to look like this:

\begin{earg}
\item[P$_1$:]  All $M$ are $P$.
\item[P$_2$:] All $M$ are $S$.
\vspace{-.5em}
\item [] \rule{0.2\linewidth}{.5pt} 
\item[C:] All $S$ are $P$.
\end{earg} 

And now the argument only has three terms, and we can evaluate it. As you can see it is not Barbara at all, but one of her unnamed evil step-sisters, AAA-3. Again, remember not to use contraposition on mood-E or mood-I statements. You can use obversion on any statement, but contraposition only works on mood A or mood O.  

Ordinary English has other prefixes besides ``non-'' to denote the complements of classes. ``Un-'' is the most common prefix, but English also uses ``in-,'' ``dis-,'' and ``a-.'' These can be handled the same way ``non-'' is. Consider,

\begin{quotation}
All dogs in the park must be leashed. Some unleashed pets in the park are cats. Therefore, some cats in the park are not dogs.
\end{quotation}

To make this argument work, we need to get the predicate of the major premise to match the subject of the minor premise. The first step is just turning the verb phrase ``must be leashed'' into a noun phrase that matches ``unleashed pets.'' We also should use the qualification ``in the park'' consistently across the terms. 

\begin{tabu}{{X[1,l,p]X[1,l,p]}}

\begin{ekey}
\item[$S$:] Cats in the park
\item[$M$:] Leashed pets in the park
\item[$P$:] Dogs in the park 
\end{ekey}

&

\begin{earg}
\item[P$_1$:]  All $P$ are $M$.
\item[P$_2$:] Some non-$M$ are $S$.
\vspace{-.5em}
\item [] \rule{0.6\linewidth}{.5pt} 
\item[C:] Some $S$ are not $P$.
\end{earg} 

\end{tabu}

Now the trick is to get the terms $M$ and non-$M$ to match. We could try to transform the second premise, but that won't do us any good. It is an E statement, so we can't use contraposition, and obversion won't change the subject term. 

Instead of changing the ``non-$M$'' to ``$M$'' in the second premise, we need to change the ``$M$'' to ``non-$M$'' in the first premise. We can do this using obversion, which works on statements in any mood.  

\begin{earg}
\item[P$_1$:]  No $P$ are non-$M$.
\item[P$_2$:] Some non-$M$ are $S$.
\vspace{-.5em}
\item [] \rule{0.2\linewidth}{.5pt} 
\item[C:] Some $S$ are not $P$.
\end{earg} 

And now we can see that the argument is Fresison (EIO-4), which is valid.


\practiceproblems

\noindent \problempart The following arguments are given with variables for the terms. Use obversion and contraposition to reduce the number of terms to three. State which operations you are using on which statements, and give the resulting syllogism in canonical form. Finally, use Venn diagrams to evaluate the argument.

\begin{longtabu}{p{.1\linewidth}p{.3\linewidth}p{.6\linewidth}}

\textbf{Example}: & \multicolumn{2}{p{.9\linewidth}}{No $M$ are $P$, and all non-$M$ are non-$S$. Therefore all $S$ are non-$P$.}\\

\textbf{Answer}: & \multicolumn{2}{p{.9\linewidth}}{Use contraposition on the minor premise and obversion on the conclusion to get the following} \\ 
 &
\begin{earg} 
\item[P$_1$:] No $M$ are $P$.
\item[P$_2$:] All $S$ are $M$.
\vspace{-.5em} 
 \item [] \rule{0.4\linewidth}{.5pt} 
\item[C:] No $S$ are $P$.
 \end{earg} 

Valid, Celartent (EAE-1)

&


\vspace{-.75cm}
\hspace{-1cm}

\begin{center}
\begin{tikzpicture}
\def\firstcircle{(0,0) circle (.75cm)}
\def\secondcircle{(60:.75cm) circle (.75cm)}
\def\thirdcircle{(0:.75cm) circle (.75cm)}

\begin{scope} 
\clip \thirdcircle;
\fill[gray] \secondcircle;
\end{scope}

\begin{scope}[even odd rule] % Shade P without M
\clip \secondcircle (-1,-1) rectangle (2,2);
\fill[gray] \firstcircle;
\end{scope}

\draw \firstcircle node[outer sep=.66cm, below left] {$S$};
\draw \secondcircle node [outer sep=.75cm, above] {$M$};
\draw \thirdcircle node [outer sep=.66cm, below right] {$P$};
\end{tikzpicture}
\end{center}
\end{longtabu} 


\begin{exercises} 
\item No $P$ are $M$, and all non-$S$ are non-$M$. Therefore some $S$ are $P$. 

\answer{

Use contraposition on the minor premise to get the following

\begin{longtabu}{X[1,l,m]X[1,l,m]} 

\begin{earg} 
\item[P$_1$:] No $P$ are $M$
\item[P$_2$:] All $M$ are $S$
\vspace{-.5em} 
 \item [] \rule{0.6\linewidth}{.5pt} 
\item[C:] Some $S$ are $P$
 \end{earg}
 EAI-IV Invalid 

&

\begin{venns}
\shadeintersectred{\middlecircle}{\predicatecircle}
\shadecomplementred{\middlecircle}{\middlesquare}{\subjectcircle}
\drawsubsyl
\drawmidsyl
\drawpredsyl
\end{venns}

\end{longtabu}
}


\item No $P$ are $M$, and all $S$ are $M$. Therefore all $S$ are non-$P$.
\answer{
Take the obverse of the conclusion to get:


\begin{longtabu}{X[1,l,m]X[1,l,m]} 

\begin{earg} 
\item[P$_1$:] No $P$ are $M$
\item[P$_2$:] All $S$ are $M$
\vspace{-.5em} 
 \item [] \rule{0.6\linewidth}{.5pt} 
\item[C:] No $S$ are $P$
 \end{earg} 
Cesare (EAE-II) (Valid) 


&
\begin{venns}
\shadeintersectred{\middlecircle}{\predicatecircle}
\shadecomplementred{\subjectcircle}{\subjectsquare}{\middlecircle}
\drawsubsyl
\drawmidsyl
\drawpredsyl
\end{venns}
\end{longtabu}

}
\item No $M$ are $P$, and some $S$ are $M$. Therefore some non-$P$ are not non-$S$.

\answer{
Take the contrapositive of the conclusion to get


\begin{longtabu}{X[1,l,m]X[1,l,m]} 

\begin{earg} 
\item[P$_1$:] No $M$ are $P$
\item[P$_2$:] Some $S$ are $M$
\vspace{-.5em} 
 \item [] \rule{0.6\linewidth}{.5pt} 
\item[C:] Some $S$ are not $P$
 \end{earg} 
Ferio (EIO-I) (valid) 
 
&
\begin{venns}
\shadeintersectred{\middlecircle}{\predicatecircle}
\someexistfive
\drawsubsyl
\drawmidsyl
\drawpredsyl
\end{venns}
\end{longtabu}
}
\item All non-$P$ are non-$M$, and some $S$ are $M$. Therefore all $S$ are $P$. 

\answer{
 Perform contraposition on the major premise to get

\begin{longtabu}{X[1,l,m]X[1,l,m]} 

\begin{earg} 
\item[P$_1$:] All $M$ are $P$
\item[P$_2$:] Some $S$ are $M$
\vspace{-.5em} 
 \item [] \rule{0.6\linewidth}{.5pt} 
\item[C:] All $S$ are $P$
 \end{earg}
 AIA-I Invalid 
 
&
\begin{venns}
\shadecomplementred{\middlecircle}{\middlesquare}{\predicatecircle}
\someexistseven
\drawsubsyl
\drawmidsyl
\drawpredsyl
\end{venns}
\end{longtabu}
}
\item All non-$M$ are non-$P$, and no $S$ are $M$. Therefore no $S$ are $P$. 
\answer{
Perform contraposition on the major premise to get

\begin{longtabu}{X[1,l,m]X[1,l,m]} 

\begin{earg} 
\item[P$_1$:] All $P$ are $M$
\item[P$_2$:] No $S$ are $M$
\vspace{-.5em} 
 \item [] \rule{0.6\linewidth}{.5pt} 
\item[C:] No $S$ are $P$
 \end{earg} 
Camestres (AEE-II) (valid) 
%
% 
&
\begin{venns}
\shadecomplementred{\predicatecircle}{\predicatesquare}{\middlecircle}
\shadeintersectred{\subjectcircle}{\middlecircle}
\drawsubsyl
\drawmidsyl
\drawpredsyl
\end{venns}
\end{longtabu}
}
\item All $P$ are $M$, and all $S$ are non-$M$. Also, some $S$ exist. Therefore, some $S$ are not $P$.
\answer{
Perform obversion on the minor premise to get

\begin{longtabu}{X[1,l,m]X[1,l,m]} 

\begin{earg*}
\item All $P$ are $M$
\item No $S$ are $M$
\item Some $S$ exist.
\itemc Some $S$ are not $P$
\end{earg*}

AEO-IV (conditionally valid) 

&
\begin{venns}
\shadecomplementred{\predicatecircle}{\predicatesquare}{\middlecircle}
\shadeintersectred{\subjectcircle}{\middlecircle}
\someexistone
\drawsubsyl
\drawmidsyl
\drawpredsyl
\end{venns}
\end{longtabu}
}
\item Some $P$ are not non-$M$. All non-$M$ are non-$S$. Some $S$ exist. Therefore some $S$ are $P$. 
\answer{
perform obversion on the major premise and contraposition on the minor premise  

\begin{longtabu}{X[1,l,m]X[1,l,m]} 

\begin{earg} 
\item[P$_1$:] Some $P$ are $M$
\item[P$_2$:] All $S$ are $M$
\item[P$_3$:] Some $S$ exist. 
\vspace{-.5em} 
 \item [] \rule{0.6\linewidth}{.5pt} 
\item[C:] Some $S$ are $P$
 \end{earg}
 IAI-II Invalid 

&
\begin{venns}
\SaMred
\drawsubsyl
\drawmidsyl
\drawpredsyl
\someexistsixseven
\someexistfiveseven
\end{venns}
\end{longtabu}
}

\item All non-$M$ are non-$P$, and all $S$ are non-$M$. Therefore no $S$ are $P$. 
\answer{
Perform contraposition on the major premise and obversion on the minor premise to get: 
\begin{longtabu}{X[1,l,m]X[1,l,m]} 

\begin{earg} 
\item[P$_1$:] All $P$ are $M$
\item[P$_2$:] No $S$ are $M$
\vspace{-.5em} 
 \item [] \rule{0.6\linewidth}{.5pt} 
\item[C:] No $S$ are $P$
 \end{earg} 
Camestres (AEE-II) (valid) 

% 
&
\begin{venns}
\PaMred
\SeMred
\drawsubsyl
\drawmidsyl
\drawpredsyl

\end{venns}
\end{longtabu}
}
\item All $P$ are $M$, and all $M$ are non-$S$. Therefore all $P$ are non-$S$

\answer{
Take the obverse of the minor premise and the conclusion to get:

\begin{longtabu}{X[1,l,m]X[1,l,m]} 
\begin{earg} 
\item[P$_1$:] All $P$ are $M$
\item[P$_2$:] No $M$ are $S$
\vspace{-.5em} 
 \item [] \rule{0.6\linewidth}{.5pt} 
\item[C:] No $S$ are $P$
 \end{earg} 
Calemes (AEE-IV) (valid) 
% 
&
\begin{venns}

\PaMred
\SeMred


\drawsubsyl
\drawmidsyl
\drawpredsyl

\end{venns}
\end{longtabu}
}

\item All $M$ are $P$, and some non-$M$ are not $S$. Therefore some $S$ are not non-$P$. 

 \answer{Convert the minor premise to get ``Some $S$ are not non-$M$.'' Then take the obverse to get ``Some $S$ are $M$.'' Then take the obverse of the conclusion to get:
 
\begin{longtabu}{X[1,l,m]X[1,l,m]} 

\begin{earg} 
\item[P$_1$:] All $M$ are $P$
\item[P$_2$:] Some $S$ are $M$
\vspace{-.5em} 
 \item [] \rule{0.6\linewidth}{.5pt} 
\item[C:] Some $S$ are $P$
 \end{earg} 
Darii (AII-1) (valid) 


%
&
\begin{venns}
\MaPred
\someexistseven
\drawsubsyl
\drawmidsyl
\drawpredsyl

\end{venns}
\end{longtabu}
}
\end{exercises}	
\noindent\problempart The following arguments are given with variables for the terms. Use obversion and contraposition to reduce the number of terms to three. State which operations you are using on which statements, and give the resulting syllogism in canonical form. Finally, use Venn diagrams to evaluate the argument.
\begin{exercises} 

\item All non-$M$ are non-$P$, and no $S$ are $M$. Therefore no $S$ are $P$. 

\answer{
\begin{earg} 
\item[P$_1$:] All $P$ are $M$
\item[P$_2$:] No $S$ are $M$
\vspace{-.5em} 
 \item [] \rule{0.6\linewidth}{.5pt} 
\item[C:] No $S$ are $P$
 \end{earg} 
Camestres (AEE-II) (valid) 
contraposition Major premise }

\item All non-$P$ are non-$M$, and all non-$M$ are non-$S$. Therefore all $S$ are $P$.

\answer{
\begin{earg} 
\item[P$_1$:] All $M$ are $P$
\item[P$_2$:] All $S$ are $M$
\vspace{-.5em} 
 \item [] \rule{0.6\linewidth}{.5pt} 
\item[C:] All $S$ are $P$
 \end{earg} 
   }
         
\item All non-$M$ are non-$P$, and no $M$ are $S$. Also, some $S$ exist. Therefore some $S$ are not $P$. 

\answer{
AEO-IV (conditionally valid) 
contraposition Major premise  
}
 
\item Some $M$ are $P$ and no $M$ are non-$S$. Therefore some $S$ are $P$. 

\answer{
\begin{earg} 
\item[P$_1$:] Some $M$ are $P$
\item[P$_2$:] All $M$ are $S$
\vspace{-.5em} 
 \item [] \rule{0.6\linewidth}{.5pt} 
\item[C:] Some $S$ are $P$
 \end{earg}
 IAI-III Invalid 
obverse Minor premise  
}
 
\item No $P$ are $M$, and all $M$ are non-$S$. Therefore no $S$ are $P$. 

\answer{
\begin{earg} 
\item[P$_1$:] No $P$ are $M$
\item[P$_2$:] No $M$ are $S$
\vspace{-.5em} 
 \item [] \rule{0.6\linewidth}{.5pt} 
\item[C:] No $S$ are $P$
 \end{earg}
 IAA-4 
obv Minor premise  
}
 
\item No $P$ are $M$, and all $S$ are $M$. Also some $S$ exist. Therefore some non-$P$ are not non-$S$. 
\answer{EAO-1I (conditionally valid) 
contraposition Conclusion } 
 
\item All $P$ are $M$, and no $S$ are $M$. Therefore all $S$ are non-$P$.
\answer{
\begin{earg} 
\item[P$_1$:] All $P$ are $M$
\item[P$_2$:] No $S$ are $M$
\vspace{-.5em} 
 \item [] \rule{0.6\linewidth}{.5pt} 
\item[C:] No $S$ are $P$
 \end{earg} 
Camestres (AEE-II) (valid) 
obverse Conclusion  
 obverse Minor premise  
} 
 
\item All $M$ are $P$, and all $M$ are $S$. Also, some $M$ exist. Therefore some $S$ are not non-$P$. 
\answer{
AAI-III (conditionally valid) 
obverse Minor premise  
 obverse Conclusion  }
 
\item All non-$P$ are non-$M$, and some $M$ are not $S$. Also some $M$ exist. Therefore, some non-$P$ are $S$. 

\answer{
\begin{earg} 
\item[P$_1$:] All $M$ are $P$
\item[P$_2$:] Some $M$ are not $S$
Some $M$ exist. 
\vspace{-.5em} 
 \item [] \rule{0.6\linewidth}{.5pt} 
\item[C:] Some $S$ are not $P$
 \end{earg}
 AOO-III Invalid 
 contraposition Conclusion  then obverse Conclusion  
 contraposition major premise
}
 
\item All non-$M$ are non-$P$, and all $M$ are non-$S$. Therefore all $S$ are non-$M$. 

\answer{
\begin{earg} 
\item[P$_1$:] All $P$ are $M$
\item[P$_2$:] No $M$ are $S$
\vspace{-.5em} 
 \item [] \rule{0.6\linewidth}{.5pt} 
\item[C:] No $S$ are $P$
 \end{earg} 
Calemes (AEE-IV) (valid) 
contraposition Major premise  
 obsert  minor premise  
 obverse Conclusion  
}
 
   
\end{exercises}



\noindent \problempart For each inference make a translation key and put the argument in standard form. If you use obversion or contraposition to reduce the number of the terms, make a note of it. Then construct a Venn diagram for it, and determine whether the inference is valid. 

\begin{longtabu}{p{.1\linewidth}p{.5\linewidth}p{.4\linewidth}}
\textbf{Example}: & \multicolumn{2}{p{.9\linewidth}}{No one who studies logic is completely stupid, and some philosophers are not non-logicians. Therefore some philosophers are not completely stupid.} \\
\\
\textbf{Answer}: & $S$: Philosophers \newline
					$M$: Logicians \newline
					$P$: People who are completely stupid \newline 
					Obversion on the minor premise

& 
\vspace{-16pt}
\begin{earg}
\item[P$_1$:] No $M$ are $P$.
\item[P$_2$:] Some $S$ are $M$.
\vspace{-.5em}
\item [] \rule{0.4\linewidth}{.5pt} 
\item[C:] Some $S$ are not $P$.
\end{earg} \\

& 
\vspace{-28pt}
\begin{center}
\begin{tikzpicture}
\def\firstcircle{(0,0) circle (.75cm)}
\def\secondcircle{(60:.75cm) circle (.75cm)}
\def\thirdcircle{(0:.75cm) circle (.75cm)}

\begin{scope} %shade overlap between S and M
\clip \thirdcircle;
\fill[gray] \secondcircle;
\end{scope}

\draw \secondcircle node [outer sep=.75cm, above] {$M$};
\draw \thirdcircle node [outer sep=.66cm, below right] {$P$};
\draw \firstcircle node[outer sep=.66cm, below left] {$S$}
	node [xshift=-.05cm, yshift=.45cm](5){\large{x}};
\end{tikzpicture}
\end{center}

& Ferio (EIO-1) \newline Unconditionally valid

\end{longtabu}

%No food is inedible. In fact, all food is edible. Therefore nothing this is inedible is edible. 

\begin{exercises}

\item All of Lorain County is outside of Cuyahoga County, but at least some of Cleveland is in Cuyahoga County. Therefore some of Lorain County is not in Cleveland. 

\answer{

\begin{longtabu}{X[1,l,p]X[1,l,p]}

\begin{ekey}
\item[$S$:] Places in Lorain County 
\item[$M$:] Places in Cuyahoga County
\item[$P$:] Places in Cleveland
\end{ekey}

&

\begin{earg} 
\item[P$_1$:] Some $P$ are $M$
\item[P$_2$:] No $S$ are $M$  %All $S$ are non-$M$
\vspace{-.5em} 
 \item [] \rule{0.6\linewidth}{.5pt} 
\item[C:] Some $S$ are not $P$
 \end{earg} 
IEO-II (Invalid) 

\end{longtabu}
}

\item All Civics are vehicles. After all, any automobile is a vehicle, and a Civic is a kind of car.  

\answer{
\begin{longtabu}{X[1,l,p]X[1,l,p]}

\begin{ekey}
\item[$S$:] Civics
\item[$M$:] Car (or automobile)
\item[$P$:] Vehicle. 
\end{ekey}

&

\begin{earg} 
\item[P$_1$:] All $M$ are $P$
\item[P$_2$:] All $S$ are $M$
\vspace{-.5em} 
 \item [] \rule{0.6\linewidth}{.5pt} 
\item[C:] All $S$ are $P$
 \end{earg} 
Barbara (AAA-I) (valid) 

\end{longtabu}
}

\item Cows are animals. After all, anything that is not an animal is also not a farm animal, and cows are farm animals. 

\answer{
\begin{longtabu}{X[1,l,p]X[1,l,p]}

\begin{ekey}
\item[$S$:] Cows
\item[$M$:] Farm animals
\item[$P$:] Animals
\end{ekey}
 Contrapose major premise.
&

\begin{earg*}
\item All $M$ are $P$
\item All $S$ are $M$
\itemc All $S$ are $P$
\end{earg*}


Barbara (AAA-1) (Valid.)

  \end{longtabu}
} 
         
\item Some digits are fingers, and if something is a digit, then it is a body part. Therefore some body parts are non-fingers. 

\answer{
\begin{longtabu}{X[1,l,p]X[1,l,p]}

\begin{ekey}
\item[$S$:] Body Parts
\item[$M$:] Digits
\item[$P$:] Fingers
\end{ekey}

Contrapose minor premise, obvert conclusion.

&


\begin{earg} 
\item[P$_1$:] Some $M$ are $P$
\item[P$_2$:] All $M$ are $S$
\vspace{-.5em} 
 \item [] \rule{0.6\linewidth}{.5pt} 
\item[C:] Some $S$ are not $P$
 \end{earg}
 IAO-III Invalid 

\end{longtabu}
}

\item Earth is a planet. Therefore Earth is not a moon, because no planets are moons. 

\answer{
\begin{longtabu}{X[1,l,p]X[1,l,p]}

\begin{ekey}
\item[$S$:] Things identical to the Earth
\item[$M$:] Planet
\item[$P$:] Moon
\end{ekey}

&

\begin{earg} 
\item[P$_1$:] No $M$ are $P$
\item[P$_2$:] All $S$ are $M$
\vspace{-.5em} 
 \item [] \rule{0.6\linewidth}{.5pt} 
\item[C:] No $S$ are $P$
 \end{earg} 
Celarent (EAE-I) (valid) 

\end{longtabu}
}

\item All trees are non-animals. Some trees are deciduous. Therefore some non-animals are not evergreen.

\answer{
\begin{longtabu}{X[1,l,p]X[1,l,p]}

\begin{ekey}
\item[$S$:] Deciduous trees.
\item[$M$:] Trees.
\item[$P$:] Animals.
\end{ekey}

Obvert major premise, convert evergreen to non-deciduous, contrapose conclusion

&

\begin{earg} 
\item[P$_1$:] No $P$ are $M$
\item[P$_2$:] Some $M$ are $S$
\vspace{-.5em} 
 \item [] \rule{0.6\linewidth}{.5pt} 
\item[C:] Some $S$ are not $P$
 \end{earg} 
Fresison (EIO-IV) (valid) 

\end{longtabu}
}

\item Some relatives are not blood relatives, and some in-laws are mothers-in-law. Therefore, some non-relatives are not non-mothers-in-law.

\answer{

\begin{longtabu}{X[1,l,p]X[1,l,p]}

\begin{ekey}
\item[$S$:] Mothers-in-law
\item[$M$:] In-laws
\item[$P$:] Relatives
\end{ekey}

In the major premise, change ``not blood relatives'' to ``in laws'' and then contrapose the conclusion. 

&

\begin{earg} 
\item[P$_1$:] Some $P$ are $M$
\item[P$_2$:] Some $M$ are $S$
\vspace{-.5em} 
 \item [] \rule{0.6\linewidth}{.5pt} 
\item[C:] Some $S$ are not $P$
 \end{earg}
 IIO-IV Invalid 

\end{longtabu}
}

\item Ludwig Wittgenstein was not English. Therefore Ludwig Wittgenstein was a philosopher, because some English people are philosophers. 

\answer{
\begin{longtabu}{X[2,l,p]X[1,l,p]}

\begin{ekey}
\item[$S$:] People identical to Ludwig Wittgenstein
\item[$M$:] Philosophers
\item[$P$:] People who are English
\end{ekey}

&

\begin{earg} 
\item[P$_1$:] Some $M$ are $P$
\item[P$_2$:] No $S$ are $M$
\vspace{-.5em} 
 \item [] \rule{0.6\linewidth}{.5pt} 
\item[C:] All $S$ are $P$
 \end{earg}
 IEA-I Invalid 
\end{longtabu}
}

\item Some liquids are non-alcoholic. This is because only liquids are drinks, and some non-alcoholic things are not non-drinks. 

\answer{
\begin{longtabu}{X[1,l,p]X[1,l,p]}

\begin{ekey}
\item[$S$:] Liquids
\item[$M$:] Drinks
\item[$P$:] Alcoholic liquids
\end{ekey}

Obvert the conclusion and take the contraposition of the minor premise

&



\begin{earg} 
\item[P$_1$:] Some $M$ are not $P$
\item[P$_2$:] All $M$ are $S$  
\vspace{-.5em} 
 \item [] \rule{0.6\linewidth}{.5pt} 
\item[C:] Some $S$ are not $P$
 \end{earg}
 OAO-III Bocardo, Valid.
\end{longtabu}
}

\item No advertisements are misleading. Therefore all pandering things are non-misleading, because no advertisements are pandering.

\answer{
\begin{longtabu}{X[1,l,p]X[1,l,p]}

\begin{ekey}
\item[$S$:] Things that pander.
\item[$M$:] Advertisements
\item[$P$:] Misleading things
\end{ekey}

Obvert the conclusion.

&


 
\begin{earg} 
\item[P$_1$:] No $M$ are $P$
\item[P$_2$:] No $M$ are $P$
\vspace{-.5em} 
 \item [] \rule{0.6\linewidth}{.5pt} 
\item[C:] No $S$ are $P$
 \end{earg}
 EEE-I Invalid 
\end{longtabu}
}


\end{exercises}
\noindent \problempart For each inference make a translation key and put the argument in standard form. Then construct a Venn diagram for it and determine whether the inference is valid. 


\begin{exercises} 

\item Some machines are likely to break, because some machines are elaborate, and nothing that is elaborate is unlikely to break. 

\answer{
\begin{earg}
\item[P$_1$:] All $M$ are $P$
\item[P$_2$:] Some $S$ are $M$
\vspace{-.5em} 
 \item [] \rule{0.6\linewidth}{.5pt} 
\item[C:] Some $S$ are $P$
 \end{earg} 
Darii (AII-1) (valid) 
 obvert major premise
}
 
\item Only mammals are dogs, and no mammals are reptiles. Also, \emph{Canis familiaris} really do exist. Therefore some dogs are not reptiles.


\answer{
\begin{earg}
\item[P$_1$:] All $P$ are $M$
\item[P$_2$:] No $M$ are $S$
\item[{\color{red}P$_3$:}] {\color{red}Some $S$ exist.}
\vspace{-.5em}
\item [] \rule{0.2\linewidth}{.5pt} 
\item[C:] Some $S$ are not $P$
\end{earg} 

AEO-IV (conditionally valid) 
 exclusive propositions 
 }


\item All of Ohio is in the United States. Therefore none of Ohio is in Canada, because no part of Canada is in the United States. 
\answer{Camestres (AEE-II) (unconditionally valid) }



\item All sneeps are snine. This is because all non-snine things are also non-sneeps, and all sneeps are snine. 

\answer{
\begin{ekey}
\item[$S$:] Snerps
\item[$M$:] Sneeps 
\item[$P$:] Snine things
\end{ekey}

\begin{earg} 
\item[P$_1$:] All $M$ are $P$
\item[P$_2$:] All $S$ are $M$
\vspace{-.5em} 
 \item [] \rule{0.6\linewidth}{.5pt} 
\item[C:] All $S$ are $P$
 \end{earg} 
Barbara (AAA-I) (valid) 
 implicit noun phrases 
}

\item No non-forks are pointy. Therefore, all pointy things are forks, because no pointy things are non-forks. 

\answer{
\begin{ekey}
\item[$S$:] Utensils.
\item[$M$:] Pointy things
\item[$P$:] Forks
\end{ekey}


\begin{earg}
\item[P$_1$:] All $M$ are $P$
\item[P$_2$:] Some $S$ are $M$
\vspace{-.5em} 
 \item [] \rule{0.6\linewidth}{.5pt} 
\item[C:] Some $S$ are $P$
 \end{earg} 
Darii (AII-1) (valid) 
 nonstandard verbs 
obverse Conclusion  
 obverse Minor premise  
}

\item Some arguments are not invalid. After all, anything that is persuasive is valid, and anything that is persuasive is also an argument. Furthermore, we know that some arguments exist.

\answer{
\begin{earg*}
\item All $M$ are $P$
\item All $M$ are $S$
\item Some $S$ exist. 
\itemc Some $S$ are $P$
\end{earg*}
AAI-III (conditionally valid) 
} 


\item Some things that sing don't sink in water. You can tell because some bricks do not sing, and all bricks float.
 
\answer{
\begin{earg} 
\item[P$_1$:] Some $M$ are not $P$
\item[P$_2$:] All $M$ are $S$
\vspace{-.5em} 
 \item [] \rule{0.6\linewidth}{.5pt} 
\item[C:] All $S$ are $P$
 \end{earg}
 OAA-III Invalid 
 
}


\item Crayons are not precision tools. Therefore all toys are non-precision tools, because some toys are crayons.

\answer{
\begin{earg} 
\item[P$_1$:] No $M$ are $P$
\item[P$_2$:] Some $S$ are $M$
\vspace{-.5em} 
 \item [] \rule{0.6\linewidth}{.5pt} 
\item[C:] No $S$ are $P$
 \end{earg}
 EIE-1 Invalid 
 obverse Conclusion  
 contraposition Major premise  
}


   
\item  Some things that aren't beliefs are not objectionable. This is because all things that are not well founded are non-beliefs, and all things that are well founded are unobjectionable. 

\answer{
\begin{ekey}
\item[$S$:] objectionable.
\item[$M$:]  Well-founded
\item[$P$:] Beliefs 
\end{ekey}

AEO-IV (conditionally valid) 

\begin{earg*}
\item All $P$ are $M$			
\item  No $M$ are $S$			
\itemc  Some $S$ are not $P$		
\end{earg*}
}
         
\item Some Kaiju are not characters from Toho studios. Therefore Godzilla is not a Kaiju, because Godzilla is a character from Toho studios.  

\answer{
\begin{earg} 
\item[P$_1$:] Some $P$ are not $M$
\item[P$_2$:] All $S$ are $M$
\vspace{-.5em} 
 \item [] \rule{0.6\linewidth}{.5pt} 
\item[C:] No $S$ are $P$
 \end{earg}
 OAE-II Invalid 
}

  
\end{exercises}


% *******************************************
% *       Enthymemes                                                *
% *******************************************

\section{Enthymemes}
\label{sec:enthymemes}
\newglossaryentry{enthymeme}
{
name=enthymeme,
description={An argument where a premise or conclusion has been left unstated.}
}


In the real world, arguments are often presented with pieces missing. The missing piece could be an important premise or sometimes even the conclusion.  Arguments with parts missing are called \textsc{\glspl{enthymeme}}. \label{def:enthymeme} We discuss them extensively in the chapter on incomplete arguments in the full version of this text. \label{ver_var} \nix{Chapter \ref{chap:incomplete_arguments}.} Here we will be dealing with them specifically as they are used with Aristotelian syllogisms. Because we are now dealing with more real-world arguments, we will return to our practice of giving the context in italics before passages (see page \pageref{context_marker}). Remember, that this context is not a part of the argument. It is just there to help interpret the passage. 

The simplest and most common reason one might leave out a part of an Aristotelian syllogism is brevity. Sometimes part of an argument is common knowledge or completely obvious from the context and simply doesn't need to be said. Consider this common inference that might pop into your head one afternoon:

\begin{quotation} \noindent \textit{Susan is working from home} Well, the dog is barking. I guess that means the mail is here.\end{quotation}

Here the missing premise is that dogs bark at letter carriers. But this isn't something Susan is going to think of consciously. It is completely common knowledge, and probably something Susan has a lot of first-hand experience with as a dog owner. So the passage above basically represents Susan's inference as it occurs to her. 

If in order to evaluate the argument, though, we need to make all of the reasoning explicit. This means putting the whole thing in standard form and then finding the missing premise. Once you do that, it becomes clear that this common, everyday inference is actually very complicated. The simplest way to put the argument in canonical form would simply be this:

\begin{earg}
\item[P:] The dog is barking.
\vspace{-.5em}
\item [] \rule{0.2\linewidth}{.5pt} 
\item[C:] The mail is here.
\end{earg} 

But if we are going to use the tools of Aristotelian logic to show that this is valid, we are going to need to convert these statements into categorical statements. The ``now'' functions here as a singular term, so we needed to use the awkward ``times identical with'' construction (see page \pageref{subsec:singular_propositions}). Also, we needed to find a general category to put things under, in this case ``times,'' to be sure that all the terms matched. (This is similar to the ``persons identical to Bertrand Russell example on page \pageref{finding_general_terms}.) 

\begin{earg}
\item[P:] All times identical with now are times when the dog is barking.
\vspace{-.5em}
\item [] \rule{0.7\linewidth}{.5pt} 
\item[C:] All times identical with now are times when the mail arrives.
\end{earg} 

Once we do this, we can see that the conclusion is a statement in mood A. The minor term is ``times identical with now'' and the major term is ``times when the mail arrives.'' The premise is a minor premise, because it has the minor term in it, and the middle term is ``times when the dog is barking.'' From this we can see that the missing premise is the major premise. When we add it, the full argument looks like this:

\begin{earg}
\item[P$_1$:] All times when the dog is barking are times when the mail arrives*
\item[P$_2$:] All times identical with now are times when the dog is barking.
\vspace{-.5em}
\item [] \rule{0.7\linewidth}{.5pt} 
\item[C:] All times identical with now are times when the mail arrives.
\end{earg} 

Remember that the asterisks means that an implicit premise has been made explicit. \iflabelexists{chap:incomplete_unclear_arguments}{(See Chapter \ref{chap:incomplete_arguments})}{}Once we have that in place, however, we can see that this argument is just a version of Barbara (AAA-1). The full argument is quite wordy, which shows how much sophisticated reasoning is going on in your brain when you make even the most everyday of inferences. 

Enthymemes can also be used to hide premises that are needed for the argument, but that the audience might not accept if their attention were drawn to them. Consider this argument:

\begin{quotation}\noindent \textit{A used car salesperson is showing you a Porsche Cayenne Hybrid SUV} And it is a hybrid electric car, so you know it gets good gas mileage. \end{quotation}

Here again we have an argument about a singular object---this car the salesperson is showing you. So when we represent the argument, we need to use the awkward ``things identical with'' construction.

\begin{earg}
\item[P:] All things identical with this vehicle are hybrid electric vehicles.
\vspace{-.5em}
\item [] \rule{0.7\linewidth}{.5pt} 
\item[C:] All things identical with this vehicle are vehicles that get good gas mileage.
\end{earg} 

Again, we have two mood-A statements. The minor term is ``things identical with this vehicle,'' the middle term is ``hybrid electric vehicles'' and the major term is ``vehicles that get good gas mileage.'' The missing premise must be the major premise ``All hybrid electric vehicles get good gas mileage.'' 

\begin{earg}
\item[P$_1$:] All hybrid electric vehicles are vehicles that get good gas mileage.*
\item[P$_2$:] All things identical with this vehicle are hybrid electric vehicles
\vspace{-.5em}
\item [] \rule{0.7\linewidth}{.5pt} 
\item[C:] All things identical with this vehicle are vehicles that get good gas mileage.
\end{earg} 

But wait, is this implicit premise even true? Actually it isn't---the Cayenne itself only gets 20 miles per gallon in the city and 24 on the highway---no wonder the salesperson kept this premise implicit! Compare the Cayenne's mileage to that of a regular gas-powered car like the Honda Fit, which gets 33 city and 41 highway. It is true that hybrids generally get better mileage than similarly sized cars that run on gasoline only, but that isn't the reasoning the salesperson is presenting here. Someone who was really all that concerned about mileage probably should not be in the market for a luxury SUV like the Cayenne.  

When filling in missing parts of an enthymeme it is important to remember the principle of charity. The principle of charity says that we should try to interpret other people in a way that makes reasoning coherent and their beliefs reasonable. \iflabelexists{chap:incomplete_unclear_arguments}{(See Chapter \ref{chap:incomplete_arguments}.)}{}With the example of the salesperson above, one might argue that we were not being as charitable as we could have been. Perhaps the salesperson didn't expect us to believe that \textit{all} hybrids get good gas mileage. Perhaps they only expected us to believe that \textit{most} hybrids get good gas mileage. 

If you interpret the argument this way, you are getting a more believable premise in exchange for a weaker inference. The original argument had the strongest possible inference---it was genuinely valid. If you change the major premise to something involving ``most,'' you will have an argument that is at best strong, not  valid. The premise is now believable, but the inference is fallible.  \iflabelexists{chap:incomplete_unclear_arguments}{(This kind of trade-off is discussed more extensively in Chapter \ref{chap:incomplete_arguments})}{} Also note that if you do decide to use ``most'' when filling in the missing premise, the argument becomes inductive, and thus is something that you would evaluate using the tools from the parts of the complete version of this textbook on inductive and scientific reasoning, \nix{Part \ref{part:inductive_scientific} of this textbook,} \label{ver_var} not the tools of Aristotelian logic. In this section we will only be looking at enthymemes where you fill in premises in a way that yields Aristotelian syllogisms. 

Sometimes enthymemes are missing a conclusion, rather than a premise. Consider this example:

\begin{tabu}{p{.1\linewidth}p{.7\linewidth}}
\multicolumn{2}{p{.8\linewidth}}{\textit{Annabeth, age 7, is pestering her mother Susan with strange questions.}}\\
\textbf{Annabeth}: & Mommy, do any dogs have forked tongues? \\
\textbf{Susan}:  & Well, dear, all dogs are mammals, and no mammals have forked tongues. So do you think any dogs have forked tongues?\\
\end{tabu}

Teachers and parents often replace the conclusion with a rhetorical question like this because it forces the audience to fill in the missing conclusion for themselves. The effect is going to be different for different audiences, however. A child might be pleased with herself for figuring out the answer with just a few hints. Often replacing the conclusion with a rhetorical question when talking to an adult can be demeaning, for instance, when someone glibly says ``you do the math,'' even when there is no math involved in the issue. 

In any case, the proper conclusion for the argument above is ``No dogs have forked tongues.'' The argument is Celarent (EAE-1), and in canonical form it looks like this: 

\begin{earg}
\item[P$_1$:] No mammals have forked tongues.
\item[P$_2$:] All dogs are mammals.
\vspace{-.5em}
\item [] \rule{0.4\linewidth}{.5pt} 
\item[C:] No dogs have forked tongues. 
\end{earg} 


Evaluating an enthymeme requires four steps. First, you need to figure out if the missing statement is a premise or a conclusion. As you might expect, indicator phrases will be helpful here. Words like ``because'' or ``therefore'' will mark one of the sentences you are given as a premise or a conclusion, and you can use this information to determine what hasn't been given. In fact, if the enthymeme contains any indicator words at all, you can be pretty confident that the missing statement is a premise. If the enthymeme contains a conclusion indicator word like ``therefore,'' you know that the statement after it is the conclusion. But even if the indicator word is a premise indicator word, you the missing statement is still likely to be a premise, because the premise indicator words are generally preceded by conclusions. Consider this following argument. 

\begin{quotation} \noindent\textit{Susan is now wondering about what reptilian characteristics some mammals do have} Well, I know some mammals have scales, because armadillos have scales. \end{quotation}

 Here the ``because'' lets us know that  ``Armadillos have scales'' is a premise, and ``some mammals have scales'' is the conclusion. Thus the missing statement is one of the premises. 

Once you have figured out what part of the syllogism is missing, the second step is to write the parts of the syllogism you do have in standard form. For this example, that is pretty easy, but you do need to provide the missing quantifier for the premise.

\begin{earg}
\item[P:] All armadillos have scales.
\vspace{-.5em}
\item [] \rule{0.3\linewidth}{.5pt} 
\item[C:] Some mammals have scales. 
\end{earg}

 Step three is to figure out what terms are in the missing statements. The two statements you have been given will contain a total of three terms, one of which will be repeated once in each statement. Since every term in an Aristotelian syllogism appears twice, we know that the two terms that only appear once must be the ones that appear in the missing statement. In the example above, the terms are ``mammals,'' ``armadillos,'' and ``things with scales.'' The term ``things with scales'' is the one that appears twice. So the missing statement will use the terms ``armadillos'' and ``mammals.''
   
Step four is to determine the mood of the missing statement and the overall figure of the syllogism. This will let you fill in everything else you need to know to complete the syllogism. The rules for a valid syllogism listed in Section \ref{sec:rules_and_fallacies} can be a big help here.

In the example we have been working with we know that the conclusion is ``Some mammals have scales.'' This means that the major term is ``things with scales'' and the minor term is ``mammals.'' The premise that we are given, ``All armadillos have scales'' must then be the major premise. So we know this much:

\begin{earg}
\item[P$_1$:] All armadillos have scales.
\item[P$_2$:] [something with armadillos and mammals]
\vspace{-.5em}
\item [] \rule{0.6\linewidth}{.5pt} 
\item[C:] Some mammals have scales. 
\end{earg}

We also know that the middle term is in the subject position of the major premise, which means that the whole syllogism is either figure 1 or figure 3. Because the major premise and the conclusion are both affirmative, we know by Rule 4 (page \pageref{rule4}) that the minor premise must be affirmative. That leaves us only mood-A and mood-I statements. \label{rule_use}

So there are four possibilities remaining: the argument is either AAI-1, AII-1, AAI-3 or AII-3. A quick glance at the Table \ref{tab:full_twentyfour} tells us that all of these are valid. (Although AAI-1 and AAI-3 are conditionally valid.) So any of these are correct answers. Let's go with AII-1 (Darii):


\begin{earg}
\item[P$_1$:] All armadillos have scales
\item[P$_2$:] Some mammals are armadillos.
\vspace{-.5em}
\item [] \rule{0.3\linewidth}{.5pt} 
\item[C:] Some mammals have scales. 
\end{earg}

Sometimes, when people give enthymemes, there is no way to fill in the missing premise that will make the argument valid. This can happen either because the person making the argument is confused or because they are being deceptive. Let's look at one example: 

\begin{quotation}\noindent\textit{Now Annabeth wants a pet armadillo} If an animal makes a good pet, then it must be cute. So armadillos must make good pets. \end{quotation}

There is no way to fill in Annabeth's reasoning here that can turn this into a valid syllogism. To see this, we need to first put it in standard form, which means converting the conditional ``if...then'' statement into a mood-A statement. 


\begin{earg}
\item[P:] All animals that make good pets are animals that are cute. 
\vspace{-.5em}
\item [] \rule{0.6\linewidth}{.5pt} 
\item[C:] All armadillos are animals that make good pets. 
\end{earg}

It seems like the missing premise here should be ``All armadillos are cute,''  but adding that premise doesn't give us a valid argument. 

\begin{earg}
\item[P$_1$:] All animals that make good pets are animals that are cute.
\item[P$_2$:] All armadillos are animals that are cute. 
\vspace{-.5em}
\item [] \rule{0.6\linewidth}{.5pt} 
\item[C:] All armadillos are animals that make good pets.   
\end{earg}

The argument is AAA-2, which has an undistributed middle. In fact, there is no way to get from P$_1$ to the conclusion with an Aristotelian categorical statement. The major term has the premise in the subject position, which means that the argument can only be figure 2 or 4. But a quick look at Table \ref{tab:full_twentyfour} lets us know that the only valid argument with a mood-A major premise and a mood-A conclusion is Barbara, which is figure 1. Nice try, kid. 

In other cases, the rules given in Section \ref{sec:rules_and_fallacies} can be used to determine whether an enthymeme with a missing conclusion can be valid. In the argument showing that some mammals have scales (p. \pageref{rule_use}), we used Rule 4 to show that the missing premise had to be affirmative. In an enthymeme with a missing conclusion, you might note that there are two negative premises, so that there actually is no valid conclusion to draw. In other situations you might be able to determine that the missing conclusion must be negative. The rules for valid syllogisms are your friends when working with enthymemes, because they allow you to consider broad categories of syllogisms, rather than having to work on a trial and error basis with the Venn diagram method.

\practiceproblems
\noindent \problempart Write each enthymeme below in standard form and supply the premise or conclusion that makes the argument valid, marking it with an asterisk. If no statement can make the argument valid, write ``invalid.''

\begin{longtabu}{p{.1\linewidth}p{.9\linewidth}}
\textbf{Example}: & Edinburgh is in Scotland, and no part of Scotland is sunny. \\
\textbf{Answer}: & 
\vspace{-16pt}
\begin{earg}
\item[P$_1$:] All places in Edinburgh are places in Scotland.
\item[P$_2$:] No places in Scotland are places that are sunny.
\vspace{-.5em}
\item [] \rule{0.6\linewidth}{.5pt} 
\item[C:] No places in Edinburgh are sunny.* 
\end{earg} 
\end{longtabu}


%Version of the example with context. 

%\begin{longtabu}{p{.1\linewidth}p{.1\linewidth}p{.8\linewidth}}
%\textbf{Example}: & \multicolumn{2}{p{.9\linewidth}}{\textit{Steve and Monica are considering where to go on their vacation.}} \\ 
%&\textbf{Steve}: &Is Edinburgh sunny? \\ 
%&\textbf{Monica}: &Well, Edinburgh is in Scotland, and no part of Scotland is sunny. Do you think it is sunny?\\ 
%\textbf{Answer}: & 
%\multicolumn{2}{p{.9\linewidth}}{
%\vspace{-16pt}
%\begin{earg}
%\item[P$_1$:] All places in Edinburgh are places in Scotland.
%\item[P$_2$:] No places in Scotland are places that are sunny.
%\vspace{-.5em}
%\item [] \rule{0.6\linewidth}{.5pt} 
%\item[C:] No places in Edinburgh are sunny.* 
%\end{earg} 
%}
%\end{longtabu} 

\begin{exercises} 
\item Dogs are mammals, which means that they aren't reptiles.

\answer{
\begin{earg*} 
\item No reptiles are mammals*
\item All dogs are mammals
\itemc[.3] No dogs are reptiles
 \end{earg*} 
Cesare (EAE-2)
}

\item Some pastas must be whole wheat, because some linguine is whole wheat.
\answer{
\begin{earg*} 
\item All linguine is pasta*
\item Some linguine is whole wheat
\itemc[.3] Some pastas are whole wheat
\end{earg*} 
Datisi (AII-3)
}


\item If you have ten dollars, you can see the movie. So therefore none of the kids will see the movie. 

\answer{
Not valid. \vspace{6pt}\\
If you change the if-then statement into the appropriate mood A statement, you get ``All people who have ten dollars can see the movie,'' but there is no way to get from there to the conclusion ``No kids will see the movie.''

\begin{earg*}
\item All people who have ten dollars are people who can see the movie
\item ?
\itemc No kids are people who can see the movie
\end{earg*}
A?E-1 or -3 \vspace{6pt}\\
To make the argument work, you would need to have ``All people who can see the movie are people who have ten dollars'' as the major premise. 
} 


\item Nothing divine is evil, so no gods are evil.

\answer{
\begin{earg*} 
\item All Gods are divine things*
\item No divine things are evil
 \itemc[.2] No Gods are evil
 \end{earg*} 
Calemes (AEE-IV) (valid) 
}

\item No logicians are ignorant of Aristotle, and some people who are ignorant of Aristotle are foolish.
\answer{
\begin{earg*} 
\item No logicians are people who are ignorant of Aristotle.
\item Some people who are ignorant of Aristotle are foolish people 
\itemc[.3] Some foolish people are not logicians*'
 \end{earg*} 
Fresison (EIO-IV) 
}


\item No holidays are work days, and some work days are not Mondays.

\answer{
Not valid \vspace{6pt}
\begin{earg*}
\item No holidays are work days.
\item Some work days are not Mondays.
\itemc ?
\end{earg*}
EO?-?
}

\item Some trees are not evergreens. Therefore some evergreens are not spruces.

\answer{
Not valid \vspace{6pt}
\begin{earg*} 
\item ?
\item Some trees are not evergreens
\itemc[.2] Some evergreens are not spruces
 \end{earg*} 
?OO-3 or -4
}

\item No jellyfish are vertebrates, but some animals that make good pets are vertebrates.
\answer{
\begin{earg*} 
\item No jellyfish are vertebrates
\item Some animals that make good pets are vertebrates 
\itemc[.2] Some animals that make good pets are not jellyfish* 
 \end{earg*} 
Festino (EIO-II)}

\item Some chairs are not houses, and all tables are houses.
\answer{
\begin{earg*} 
\item All tables are houses
\item Some chairs are not houses
\itemc[.4] Some chairs are not tables*
 \end{earg*} 
Baroco (AOO-2) \vspace{6pt} \\
This is a little tricky, because you have to switch the order of the two premises from the way they are given in the problem. If you don't, ``tables'' has to be the subject of the conclusion and ``chairs'' has to be the predicate, and there is nothing you can with the terms in that order. 
}


\item No doodlebugs are octofish. Therefore no thing-havers are octofish.

\answer{
\begin{earg*} 
\item No doodlebugs are octofish
\item All thing-havers are doodlebugs 
\itemc[.4] No thing-havers are octofish
 \end{earg*} 
Celarent (EAE-I) 
}




\end{exercises}

\noindent \problempart Write each enthymeme below in standard form and supply the premise or conclusion that makes the argument valid, marking it with an asterisk. If no statement can make the argument valid, write ``invalid.''

\begin{exercises}
\item No board games are online games, because all online games are video games.

%\begin{earg} 
%\item[P$_1$:] No $M$ are $P$
%\item[P$_2$:] All $S$ are $M$
%\vspace{-.5em} 
% \item [] \rule{0.6\linewidth}{.5pt} 
%\item[C:] No $S$ are $P$
% \end{earg} 
%Celarent (EAE-I) (valid) 
%nonstandard quantifiers 
%Delete Major premise 

\item No coins are paper money. Therefore no coins are two dollar bills.
%\begin{earg} 
%\item[P$_1$:] All $P$ are $M$
%\item[P$_2$:] No $S$ are $M$
%\vspace{-.5em} 
% \item [] \rule{0.6\linewidth}{.5pt} 
%\item[C:] No $S$ are $P$
% \end{earg} 
%Camestres (AEE-II) (valid) 
%exclusive propositions 
%Delete Major premise  


\item Some vegetables are peppers. Therefore some foods are peppers.

%\begin{earg} 
%\item[P$_1$:] Some $M$ are $P$
%\item[P$_2$:] All $M$ are $S$
%\vspace{-.5em} 
% \item [] \rule{0.6\linewidth}{.5pt} 
%\item[C:] Some $S$ are $P$
% \end{earg} 
%Disamis (IAI-III) (valid) 
%adverbs and pronouns 
%Delete Minor premise 


\item Everyone who fights for justice is a saint. Therefore no politicians are saints.
%
%\begin{earg} 
%\item[P$_1$:] All justice fighters are saints
%\item[P$_2$:] No politicians are justice fighters*
%\vspace{-.5em} 
% \item [] \rule{0.6\linewidth}{.5pt} 
%\item[C:] No politicians are saints
% \end{earg} 
% AEE-I----looks like celarent, but 

%invalid

\item Some children are not getting treats because no one who was naughty gets treats.

%\begin{earg} 
%\item[P$_1$:] No $M$ are $P$
%\item[P$_2$:] Some $S$ are $M$
%\vspace{-.5em} 
% \item [] \rule{0.6\linewidth}{.5pt} 
%\item[C:] Some $S$ are not $P$
% \end{earg} 
%Ferio (EIO-I) (valid)nonstandard quantifiers 
%Delete Minor premise 


\item On days when there is more than a foot of snow, school is canceled. Therefore some Mondays this winter school will be canceled. 

%\begin{earg} 
%\item[P$_1$:] All $M$ are $P$
%\item[P$_2$:] Some $S$ are $M$
%\vspace{-.5em} 
% \item [] \rule{0.6\linewidth}{.5pt} 
%\item[C:] Some $S$ are $P$
% \end{earg} 
%Darii (AII-1) (valid) conditional statements 
%Delete Minor premise 



\item All churches are religious institutions. Therefore, some churches are not Christian.
%\begin{earg} 
%\item[P$_1$:] Some $M$ are not $P$
%\item[P$_2$:] All $S$ are $M$
%\vspace{-.5em} 
% \item [] \rule{0.6\linewidth}{.5pt} 
%\item[C:] Some $S$ are not $P$
% \end{earg} 
%invalid

\item All rocks are food, and some vegetables are rocks.
%\begin{earg} 
%\item[P$_1$:] All $M$ are $P$
%\item[P$_2$:] Some $S$ are $M$
%\vspace{-.5em} 
% \item [] \rule{0.6\linewidth}{.5pt} 
%\item[C:] Some $S$ are $P$
% \end{earg} 
%Darii (AII-1) (valid) 
%the only 
%Delete Conclusion 



\item Some houses are not offices, and some residences are not offices.
%
%\begin{earg} 
%\item[P$_1$:] Some $P$ are not $M$
%\item[P$_2$:] All $M$ are $S$
%\vspace{-.5em} 
% \item [] \rule{0.6\linewidth}{.5pt} 
%\item[C:] Some $S$ are not $P$
% \end{earg} 

%invalid.


\item No snirt are hirt. Therefore some blorp are not snirt.
%\begin{earg} 
%\item[P$_1$:] No $P$ are $M$
%\item[P$_2$:] Some $S$ are $M$
%\vspace{-.5em} 
% \item [] \rule{0.6\linewidth}{.5pt} 
%\item[C:] Some $S$ are not $P$
% \end{earg} 
%Festino (EIO-II) (valid) 
%nonstandard verbs 
%Delete Minor premise  

\end{exercises}


% *******************************************
% *                      Sorites Categorical Arguments      *
% *******************************************

\section{Sorites Categorical Arguments }

The Aristotelian tradition mostly focused on syllogisms with two premises; however, there are fun things we can do with longer arguments. These were explored, among other places, in a 19th century textbook called \textit{Symbolic Logic} \citep{Dodgson1896}, by the mathematician and logician Charles Lutwidge Dodgson. (Dodgson is actually better known by his pen name, Lewis Carroll, under which he wrote the children's classics \textit{Alice's Adventures in Wonderland} and \textit{Through the Looking Glass}.)

\newglossaryentry{sorites categorical argument}
{
name=sorites categorical argument,
description={A categorical argument with more than two premises.}
}

Categorical arguments with more than two premises are called \textsc{\glspl{sorites categorical argument}}, \label{def:sorites_categorical_arguments} or just ``Sorites'' (pronounced ``sore-EYE-tease'') for short. The name comes from the Greek word ``Soros'' meaning ``heap.'' This kind of sorites should not be confused with the sorites paradoxes we talk about in the chapter on real world evaluation in the complete version of this text, \label{ver_var} \nix{Chapter \ref{chap:realworldevaluation},} which were arguments that exploited vague premises. 

Here is a simple example of a categorical sorites.

\begin{earg}
\item[P$_1$:] All carrots are vegetables.
\item[P$_2$:] No vegetables are houses. %No carrots are houses
\item[P$_3$:] All houses are buildings. 	
\vspace{-.5em}
\item [] \rule{0.3\linewidth}{.5pt} 
\item[C:] No carrots are buildings. 
\end{earg}
%Calemes-Celarent

We have more premises and terms than we do in the classic Aristotelian syllogism, but they are arranged in a way that is a natural extension of the Aristotelian syllogism. Every term appears twice, except for the major and minor terms of the conclusion. Each premise shares one term with each of the other premises. Really, this is just like a regular Aristotelian syllogism with more middle terms. 

Because the form of the argument above is an extension of the standard form of an Aristotelian syllogism, you can actually break it apart into two valid Aristotelian syllogisms if you supply the missing statements.


\begin{tikzpicture}

\node at (0,0)[text width=6cm, inner sep=1mm] {
\begin{earg}
\item[P$_1$:] All carrots are vegetables. 
\item[P$_2$:] No vegetables are houses.
\vspace{-.5em}
\item [] \rule{0.6\linewidth}{.5pt} 
\item[C:] No carrots are houses.*  
\end{earg}
};

\draw [myarrow1, ->] (1.5,-.75) .. controls (3.25, -.75) and (3.25, .5) .. (5.1, .4);

\node at (8,0)[text width=6cm] {
\begin{earg}
\item[P$_1$:] No carrots are houses.*
\item[P$_2$:] All houses are buildings.
\vspace{-.5em}
\item [] \rule{0.6\linewidth}{.5pt} 
\item[C:] No carrots are buildings.
\end{earg}
};

\end{tikzpicture}




The argument on the left takes the first two premises from the sorites argument and derives a conclusion from them that was not stated in the original argument. We then slide this conclusion over and use it as the major premise for the next argument, where it gets combined with the final premise and the conclusion of the sorites argument. The result is two valid arguments, Calemes and Celarent, which have been chained together. The conclusion of the Calemes argument is an intermediate conclusion, and the conclusion of the Celarent is the final conclusion. You can think of the sorites argument as a sort of abbreviation for this chain of valid arguments. 

\newglossaryentry{standard form for a sorites categorical argument}
{
name=standard form for a sorites categorical argument,
description={A sorites argument that has been put into logically structured English with the following criteria: (1) each statement in the argument is in standard form for a categorical statement in logically structured English, (2) each instance of a term is in the same format and is used in the same sense, (3) the major premise is first in the list of premises and the minor premise is last, and (4) the middle premises are arranged so that premises that share a term are adjacent to one another.}
}

The task for this section is to evaluate the validity of sorites arguments. To do this, we will need to put them in standard form, just as we did for ordinary Aristotelian arguments. We will define \textsc{\gls{standard form for a sorites categorical argument}} \label{def:standard_form_for_a_sorites_categorical_argument} as a sorites argument that has been put into logically structured English with the following criteria: (1) each statement in the argument is in standard form for a categorical statement in logically structured English, (2) each instance of a term is in the same format and is used in the same sense, (3) the major premise is first in the list of premises and the minor premise is last, and (4) the middle premises are arranged so that premises that share a term are adjacent to one another.

Some passages will take more work to get into standard form for sorites than others. We will just look at situations where you need to rearrange the order of the premises and fix the terms so that they match. We will also have an example using variables for terms. We will use the letters $A$, $B$, $C$, \ldots for the terms in sorites categorical arguments, rather than $S$, $M$, and $P$. 

\begin{earg}
\item[P$_1$:] All $D$ are $E$.
\item[P$_2$:] Some $C$ are $D$.
\item[P$_3$:] All $C$ are $B$.  
\item[P$_4$:] All $A$ are non-$B$.
\vspace{-.5em}
\item [] \rule{0.2\linewidth}{.5pt} 
\item[C:] Some $E$ are not $A$.    
\end{earg} 

In this argument, P$_3$ and P$_4$ don't match up, because one talks about $B$ and the other non-$B$. We can just use obversion on P$_4$:  

\begin{earg}
\item[P$_1$:] All $D$ are $E$.
\item[P$_2$:] Some $C$ are $D$.
\item[P$_3$:] All $C$ are $B$.  
\item[P$_4$:] No $A$ are $B$.
\vspace{-.5em}
\item [] \rule{0.2\linewidth}{.5pt} 
\item[C:] Some $E$ are not $A$.    
\end{earg} 

Now we need to put the premises in the proper order. The predicate for the conclusion is $A$, so the statement ``No $A$ are $B$'' needs to be the first premise. The statement containing $E$ needs to be the last premise, and the middle two premises need to be swapped so each premise will share a term with the statements on either side of it.

\begin{earg}
\item[P$_1$:] No $A$ are $B$.
\item[P$_2$:] All $C$ are $B$. % No C are A. Cesare (EAE-II)
\item[P$_3$:] Some $C$ are $D$.  % Some $D$ are not $A$ Ferison (EIO-III)
\item[P$_4$:] All $D$ are $E$.
\vspace{-.5em}
\item [] \rule{0.2\linewidth}{.5pt} 
\item[C:] Some $E$ are not $A$.  % Bocardo (OAO-3)  
\end{earg} 
\label{standard_forms_sorites_1}

In this example, the letters wind up in ordinary alphabetical order. Not every example will work this way.

One we have the arguments in standard form, our job is to evaluate them. We will look at three methods for doing this. The first will require you to break down the sorites into its component syllogisms and evaluate them separately. The second two will let you evaluate the argument all at once. 

\subsection{Checking Each Inference with Venn Diagrams}

The most thorough way to check a categorical sorites argument is to break it down into its component Aristotelian syllogisms and check each of those separately. Look at the argument on the previous page. We need to identify the implicit intermediate conclusions and write out each syllogism separately. The first two premises of this argument are ``No $A$ are $B$,'' and ``All $C$ are $B$.'' That's a mood-E statement and a mood-A statement, with both middle terms on the right. A glance at Table \ref{tab:full_twentyfour} lets us know that we can conclude a mood-E statement from this, ``No $C$ are $A$.'' So the first component argument is Cesare (EAE-II).

\begin{earg}
\item[P$_1$:] No $A$ are $B$.
\item[P$_2$:] All $C$ are $B$. % .
\vspace{-.5em}
\item [] \rule{0.2\linewidth}{.5pt} 
\item[C:] No $C$ are $A$.*
\end{earg}

The conclusion of this argument, ``No $C$ are $A$,'' is an implicit intermediate step, which means it is the major premise for the next component syllogism. The minor premise is P$_3$, ``Some $C$ are $D$.'' This gives us a mood-E and a mood-O statement. The new middle term is $C$. Again we can consult Table \ref{tab:full_twentyfour} to see that this sets up Ferison (EIO-3): 

\begin{earg}
\item[P$_1$:] No $C$ are $A$.*
\item[P$_2$:] Some $C$ are $D$.% .
\vspace{-.5em}
\item [] \rule{0.2\linewidth}{.5pt} 
\item[C:] Some $D$ are not $A$.* 
\end{earg}

The conclusion here becomes the major premise of the next component argument. The minor premise is P$_4$ of the original sorites. This means that the last step is Bocardo (OAO-3). 

\begin{earg}
\item[P$_1$:] Some $D$ are not $A$.* 
\item[P$_2$:] All $D$ are $E$.
\vspace{-.5em}
\item [] \rule{0.2\linewidth}{.5pt} 
\item[C:] Some $E$ are not $A$. 
\end{earg} 

To wrap things up, we can use three Venn diagrams to confirm that each step is valid. This step is needed to be sure that the terms appear in the correct position in each diagram. The three Venn diagrams are shown together with the corresponding arguments in Figure \ref{fig:sorites_example_1}. Notice that the major term is always represented by the lower right circle: it is the predicate of the conclusion of each component argument. The subject of the conclusion for each component argument then becomes the middle term for the next argument. 

\begin{figure}
\begin{mdframed}[style=mytableclearbox]
%\begin{center}
\begin{tikzpicture}%

%%%% Venn 1


\begin{scope}[xshift=-1cm]
\def\firstcircle{(0,0) circle (1cm)}
\def\secondcircle{(60:1.25cm) circle (1cm)}
\def\thirdcircle{(0:1.25cm) circle (1cm)}

\begin{scope} %shade overlap between S and M
\clip \thirdcircle;
\fill[gray] \secondcircle;
\end{scope}

\begin{scope}[even odd rule] % Shade P without M
\clip \secondcircle (-1,-1) rectangle (2,2);
\fill[gray] \firstcircle;
\end{scope}


\draw \firstcircle node[outer sep=.8cm, below left] {$C$};
\draw \secondcircle node [outer sep=1cm, above] {$B$};
\draw \thirdcircle node [outer sep=.8cm, below right] {$A$};

\end{scope}

%%%% Venn 2

\begin{scope}[shift={(4.25cm,0cm)}]
\def\firstcircle{(0,0) circle (1cm)}
\def\secondcircle{(60:1.25cm) circle (1cm)}
\def\thirdcircle{(0:1.25cm) circle (1cm)}

\begin{scope} %shade overlap between P and M
\clip \thirdcircle;
\fill[gray] \secondcircle;
\end{scope}


\draw \firstcircle node[outer sep=.8cm, below left] {$D$};
\draw \secondcircle node [outer sep=1cm, above] {$C$};
\draw \thirdcircle node [outer sep=.8cm, below right] {$A$}
	node[xshift=-1.15cm, yshift=.66cm](5){\Large{x}};
\end{scope}

%%%% Venn 3

\begin{scope}[shift={(9.5cm,0cm)}]

\def\firstcircle{(0,0) circle (1cm)}
\def\secondcircle{(60:1.25cm) circle (1cm)}
\def\thirdcircle{(0:1.25cm) circle (1cm)}

\begin{scope}[even odd rule] % Shade P without M
\clip \firstcircle (-1,-1) rectangle (2,2.5);
\fill[gray] \secondcircle;
\end{scope}


\draw \firstcircle node[outer sep=.8cm, below left] {$E$};
\draw \secondcircle node [outer sep=1cm, above] {$D$};
\draw \thirdcircle node [outer sep=.8cm, below right] {$A$}
	node[xshift=-1.15cm, yshift=.66cm](5){\Large{x}};


\end{scope}

%%%%%%%%%%%% Arg 1

 \node at (0,-2.65) [text width=4.5cm, outer sep=1mm] (Cesare){
\begin{earg}
\item[P$_1$:] No $A$ are $B$.
\item[P$_2$:] All $C$ are $B$. % .
\vspace{-.5em}
\item [] \rule{0.7\linewidth}{.5pt} 
\item[C:] No $C$ are $A$.
\end{earg}
};

%%%%%%%%%%%% Arg 2


\node at (4.75,-2.5)[text width=4.5cm, outer sep=1mm] (Ferison){
\begin{earg}
\item[P$_1$:] No $C$ are $A$.
\item[P$_2$:] Some $C$ are $D$.% .
\vspace{-.5em}
\item [] \rule{0.9\linewidth}{.5pt} 
\item[C:] Some $D$ are not $A$. 
\end{earg}
};


%%%%%%%%%%%% Arg 3

\node at (10,-2.5)[text width=4.5cm, outer sep=1mm] (Bocardo){
\begin{earg}
\item[P$_1$:] Some $D$ are not $A$. 
\item[P$_2$:] All $D$ are $E$.
\vspace{-.5em}
\item [] \rule{0.9\linewidth}{.5pt} 
\item[C:] Some $E$ are not $A$. 
\end{earg} 
};

\draw [myarrow1, ->] (1, -3.3) .. controls (2.125, -3.3) and (2.125, -2.15) .. (3.15, -2.15);

\draw [myarrow1, ->] (6.6, -3.3) .. controls (7.5, -3.3) and (7.5, -2.15) .. (8.4, -2.15);

\node at (0,-4){\textbf{Cesare (EAE-II})}; 	
\node at (5,-4) {\textbf{Fresison (EIO-IV)}};  
\node at (10,-4) {\textbf{Bocardo (OAO-3)}}; 


%\filldraw [red] (0,0) circle (.1cm);
%\filldraw [red] (0,-2.5) circle (.1cm); 
%\filldraw [red] (5.5,-2.5) circle (.1cm);
%\filldraw [red] (11,-2.5) circle (.1cm);


\end{tikzpicture}
%\end{center}
\end{mdframed}
\caption{Example of a valid sorites broken down into its component arguments}
\label{fig:sorites_example_1}
\end{figure}

This example turned out to be valid, but not every sorites you will be asked to evaluate will work that way. Consider this example, given in standard form. 


\begin{earg} 
\item[P$_1$:] All $B$ are $A$.
\item[P$_2$:] All $C$ are $B$.  %All $C$ are $A$ %Barbara (AAA-I) (valid) 
\item[P$_3$:] Some $C$ are not $D$.
\vspace{-.5em} 
 \item [] \rule{0.2\linewidth}{.5pt} 
\item[C:] No $D$ are $A$.
 \end{earg} 

If you look at the first two premises, you can easily see that they entail ``All $C$ are $A$.'' The first component argument is a simple Barbara. But if you plug that conclusion into the next component argument, the result is invalid. 

\begin{center}
\begin{tikzpicture}

\node at (0,0)[text width=4.5cm, outer sep=1mm] (Barbara){ 
\begin{earg}
\item[P$_1$:] All $B$ are $A$.
\item[P$_2$:] All $C$ are $B$.
\vspace{-.5em}
\item [] \rule{0.6\linewidth}{.5pt} 
\item[C:] All $C$ are $A$.* 
\end{earg} 

};

\node at (-.1, -1.5) {\textbf{Barbara (AAA-I)}};

\draw [myarrow1, ->] (1.1, -.55) .. controls (2.4, -.5) and (2.4, .5) .. (3.3, .45);
%
%\filldraw [red] (0,0) circle (.1cm);
%\filldraw [red] (2.4, -.5) circle (.1cm); 
%\filldraw [red] (2.4, .5)  circle (.1cm);

\node at (5,.1)[text width=4.5cm, outer sep=1mm] (AOE-III){ 
\begin{earg} 
\item[P$_1$:] All $C$ are $A$.*
\item[P$_2$:] Some $C$ are not $D$.
\vspace{-.5em} 
\item [] \rule{0.6\linewidth}{.5pt} 
\item[C:] No $D$ are $A$.
\end{earg}
%AOE-III Invalid
};

\node at (4.9, -1.5) {\textbf{AOE-III (Invalid)}};

\end{tikzpicture}
\end{center}

You will also encounter situations where two early premises in a sorites don't lead to any valid conclusion, so there is simply no way to fill in the missing steps. Consider this:

\begin{earg}
\item[P$_1$:] No $D$ are $B$.
\item[P$_2$:] No $D$ are $C$.
\item[P$_3$:] All $A$ are $C$.
\vspace{-.5em}
\item [] \rule{0.2\linewidth}{.5pt} 
\item[C:] All $A$ are $B$.
\end{earg} 

Here the first two premises are negative, so we know by Rule 3 that there is no conclusion we can validly draw here. Thus there is nothing to plug in as the major premise for the second component argument. 

\begin{center}
\begin{tikzpicture}

\node at (0,0)[text width=4.5cm, outer sep=1mm] (Barbara){ 
\begin{earg}
\item[P$_1$:] No $D$ are $B$.
\item[P$_2$:] No $D$ are $C$.
\vspace{-.5em}
\item [] \rule{0.6\linewidth}{.5pt} 
\item[C:] ??? 
\end{earg} 

};

\node at (-.1, -1.5) {\textbf{Incomplete}};

\draw [myarrow1, ->] (1.1, -.55) .. controls (2.4, -.5) and (2.4, .5) .. (3.3, .45);
%
%\filldraw [red] (0,0) circle (.1cm);
%\filldraw [red] (2.4, -.5) circle (.1cm); 
%\filldraw [red] (2.4, .5)  circle (.1cm);

\node at (5,.1)[text width=4.5cm, outer sep=1mm] (AOE-III){ 
\begin{earg} 
\item[P$_1$:] ???
\item[P$_2$:]  All $A$ are $C$.
\vspace{-.5em} 
\item [] \rule{0.6\linewidth}{.5pt} 
\item[C:]  All $A$ are $B$.
\end{earg}
%AOE-III Invalid
};

\node at (4.9, -1.5) {\textbf{Incomplete}};

\end{tikzpicture}
\end{center}

Actually, once you note that the first two premises violate Rule 3 for Aristotelian syllogisms, there really isn't a need to break down the argument further. This brings us to the second method for evaluating sorites categorical arguments. 

\subsection{Checking the Whole Sorites Using Rules}

Rather than filling in all the missing intermediate conclusions, we can see directly whether a sorites is valid by applying the rules we went over in Section  \ref{sec:rules_and_fallacies}. To spare you from flipping back and forth too much in the text, let's repeat the five basic rules here, modified so that they work with the longer arguments we are discussing. 

\begin{quotation}
\begin{tabu}{p{.1\linewidth}p{.9\linewidth}}
\textbf{Rule 1}: & Each middle term in a valid categorical sorites argument must be distributed at least once. \\
\textbf{Rule 2}: & If a term is distributed in the conclusion of a valid categorical sorites argument, then it must also be distributed in one of the premises. \\ 
\textbf{Rule 3}: & A valid Aristotelian syllogism cannot have two or more negative premises. \\
\textbf{Rule 4}: & A valid Aristotelian syllogism can have a negative conclusion if and only if it has exactly one negative premise.\\
\textbf{Rule 5}: & A valid Aristotelian syllogism with only universal premises cannot have a particular conclusion.
\end{tabu}
\end{quotation}

Here ``conclusion'' means the final conclusion of the whole sorites, not any of the implicit intermediate conclusions. The middle terms are the ones that appear in two premises. Also, as was the case with Aristotelian syllogisms, if Rule 5 is broken, the argument can be fixed by adding an appropriate existential premise. We will look at how to do that shortly. 

To evaluate an sorites argument using the rules, it helps not only to have the argument in standard form, but also to mark each term as distributed or undistributed. Remember that a term is distributed in a statement if the statement is making a claim about that whole class. (See our original discussion on page \pageref{def:Distribution}.) Table \ref{tab:distribution_reminder} is a reminder of which kinds of sentence distribute which terms. Basically, universal statements distribute the subject, and negative statements distribute the predicate. In the example below, distributed terms are marked with a superscript D. 

\begin{earg} 
\item[P$_1$:] No $B$\textsuperscript{D} are $A$\textsuperscript{D}.
\item[P$_2$:] All $C$\textsuperscript{D} are $B$. %No $C$ are $A$, Celarent (EAE-I) 
\item[P$_3$:] Some $C$ are $D$.  % All $D$ are $A$ EIA-III, invalid or Some $D$ are not $A$, which is valid, but it leaves $D$ undistributed in the next syllogism.
\item[P$_4$:] All $E$\textsuperscript{D} are $D$.
\vspace{-.5em} 
 \item [] \rule{0.2\linewidth}{.5pt} 
\item[C:] All $E$\textsuperscript{D} are $A$.  % Barbara 
 \end{earg} 

\begin{table}
\begin{mdframed}[style=mytablehalfbox, userdefinedwidth=.6\textwidth]
\begin{tabu}{p{.1\linewidth}p{.4\linewidth}p{.5\linewidth}}
 \underline{Mood} & \underline{Form} & \underline{Terms Distributed} \\ 
A & All $S$ are $P$ & S\\
E & No $S$ are $P$ &  S and P\\
I & Some $S$ are $P$ & None\\
O &Some $S$ are not $P$ & P \\
\end{tabu}
\end{mdframed}
\caption{Moods and distribution}\label{tab:distribution_reminder}
\end{table}

To check the above argument using the rules we just run through the rules in order, and see if they are all followed. Rule 1 says that every middle term must be distributed in at least one premise. The middle terms here are B, C, and D. We can quickly check to see that B is distributed in premise 1, C is distributed in premise 2, but D is never distributed. You can check this by trying to break the argument down into component syllogisms, as we did in the first section.

Sometimes the argument you are given will satisfy the two distribution rules but fail to satisfy other rules. Consider this example.

\begin{earg} 
\item[P$_1$:] No $C$\textsuperscript{D} are $B$\textsuperscript{D}.
\item[P$_2$:] Some $A$ are $B$. %Some $A$ are not $C$, Festino (EIO-II) (valid) 
\item[P$_3$:] No $A$\textsuperscript{D} are $D$\textsuperscript{D}. 
\vspace{-.5em} 
 \item [] \rule{0.2\linewidth}{.5pt} 
\item[C:] Some $C$ are $D$. % OEI-III
 \end{earg} 

Here the two middle terms are $A$ and $B$, which are distributed in the third and first premises, respectively, so the first rule is satisfied. No terms are distributed in the conclusion, so neither need to be distributed in the premises, and the second rule is satisfied. This argument fails Rule 3, however, because it contains two negative premises. Again, you can confirm the results of this method using the results of the previous method. 

The final case we need to look at is an argument that fails Rule 5, and thus needs an existential premise added. 

\begin{earg}
\item[P$_1$:] No $A$\textsuperscript{D} are $D$\textsuperscript{D}. 
\item[P$_2$:] All $C$\textsuperscript{D} are $A$.  %  No $C$ are $D$ Celarent, EAE-I
\item[P$_3$:] All $C$\textsuperscript{D} are $B$. 
\vspace{-.5em}
\item [] \rule{0.3\linewidth}{.5pt} 
\item[C:] Some $B$ are not $D$\textsuperscript{D}. %Felapton (EAO-1II)
\end{earg} 

Rules 1--4 are satisfied: the middle terms are $A$ and $C$, which are both distributed; $D$ is distributed in the conclusion and in P$_1$; there is exactly one negative premise and the conclusion is also negative. However, Rule 5 is not satisfied, because all the premises are universal, but the conclusion is particular. This means that for the argument to be valid, we will need to add an existential premise. 

But what existential premise should we add? Remember that the term that must exist in order for a conditionally valid argument to be actually valid is called the critical term (see page \pageref{def:critical_term}. When looking at conditionally valid arguments in Section \ref{sec:conditionally_valid_forms}, we basically just found the critical term by trial and error. For instance, when we used Venn diagrams to show that Felapton was conditionally valid (p. \pageref{CVFex2}), we saw that adding either the premise ``Some $S$ exist'' or the premise ``Some $P$ exist'' would not help, but adding ``Some $M$ exist'' would help. On the other hand, when considering the argument EII-3 (p. \pageref{CVFex3}), we saw that there were no existential premises that we could add that would help. However, as the number of terms we have to deal with increases, the trial and error method becomes less appealing. On top of that, our previous trial and error method was guided by the Venn diagrams for the syllogisms, and we haven't even looked yet at how to do Venn diagrams for sorites arguments. 


\newglossaryentry{superfluous distribution rule}
{
name=superfluous distribution rule,
description={A rule that says that in a conditionally valid argument, the critical term will be the one that is distributed more times in the premises than is necessary to satisfy Rules 1 and 2.}
}

The trick to finding the critical term in sorites arguments is called the superfluous distribution rule. The \textsc{\gls{superfluous distribution rule}} \label{def:superfluous_distribution_rule} says that in a conditionally valid argument, the critical term will be the one that is distributed more times in the premises than is necessary to satisfy Rules 1 and 2. In the argument above, the middle term $C$ is distributed in two premises, but it only needs to be distributed once to satisfy Rule 1. So we can make the argument above valid by adding the premise ``Some $C$ exist'':


\begin{earg}
\item[P$_1$:] No $A$\textsuperscript{D} are $D$\textsuperscript{D}. 
\item[P$_2$:] All $C$\textsuperscript{D} are $A$.  %  No $C$ are $D$ Celarent, EAE-I
\item[P$_3$:] All $C$\textsuperscript{D} are $B$. 
\item[P$_4$:] Some $C$ exist.
\vspace{-.5em}
\item [] \rule{0.3\linewidth}{.5pt} 
\item[C:] Some $B$ are not $D$\textsuperscript{D}. %Felapton (EAO-1II)
\end{earg} 

You can confirm that this is a valid argument now using the method of filling in the missing intermediate conclusions described in the previous subsection. You will find that this argument actually consists of a Celarent (EAE-1) and a Felapton (EAO-3). The latter is conditionally valid and has the critical term as its middle term.  

\subsection{A Venn Diagram with More Than Three Terms}

The last method we will consider is the most advanced but also the most fun. Your instructor may decide to skip this section, given that the prior two techniques are sufficient to evaluate sorites arguments. The Venn diagrams we have been making for this section have had three circles to represent three terms. The arguments we are dealing with have four or more terms, so if we want to represent them all at once, we will need a diagram with more shapes. To see how to set this up we need to remember the principles behind the original design of the diagram. 

Recall that John Venn's original insight was that we should always begin by arranging the shapes so that every possible combination of the terms was represented (see pages \pageref{fig:two_circle_venn} and \pageref{fig:three_term_venn_areas}). If we have two terms, we can use circles arranged like this,

\begin{center}
\begin{tikzpicture}
\def\firstcircle{(0,0) circle (1cm)}
\def\secondcircle{(0:1.33cm) circle (1cm)}
\draw \firstcircle node[outer sep=.75cm, above left] (s) {$S$} 
	node [xshift=-.25cm] (1) {1}
	node [xshift=.66cm] (2){2};
\draw \secondcircle node[outer sep=.75cm, above right] (p) {$P$}
	node [xshift=.25cm] (3) {3}
	node [xshift=1.4cm] (4){4};
\end{tikzpicture}
\end{center}

\noindent where area 1 represents things that are $S$ but not $P$, area 2 represents things that are both  $S$ and $P$, area 3 represents things that are only $P$, and area 4 represents things that are neither $S$ nor $P$.

Three terms give us eight possibilities to represent, so we draw the diagram like this:

\begin{center}
\begin{tikzpicture}
\def\firstcircle{(0,0) circle (1.25cm)}
\def\secondcircle{(60:1.5cm) circle (1.25cm)}
\def\thirdcircle{(0:1.5cm) circle (1.25cm)}

\draw \firstcircle node[outer sep=1cm, below left] {$S$}
	node [xshift=-.25cm, yshift=-.25cm](1) {1}
	node [xshift=.75cm, yshift=-.25cm] (4){4}
	node [xshift=.15cm, yshift=.8cm](5){5}
	node [xshift=.75cm, yshift=.5cm] (7){7};
\draw \secondcircle node [outer sep=1.33cm, above] {$M$}
	node [yshift=.25cm](2) {2};
\draw \thirdcircle node [outer sep=1cm, below right] {$P$}
	node [xshift=.25cm, yshift=-.25cm](3) {3}
	node[xshift=-.15cm, yshift=.8cm](6){6}
	node[xshift=1.25cm, yshift=1.75cm](8){8};  
\end{tikzpicture}
\end{center}

A four-term sorites will have 16 possible combinations of terms. To represent all of these, we will need to stretch out our circles into ellipses.   
\begin{center}
\begin{tikzpicture}
%definitions
\def\firstellip{(1.6, .4) ellipse [x radius=3cm, y radius=1.5cm,rotate=50]}
\def\secondellip{(0.3, 1.4cm) ellipse [x radius=3cm, y radius=1.5cm, rotate=50]} 
\def\thirdellip{(-0.3, 1.4cm) ellipse [x radius=3cm, y radius=1.5cm, rotate=-50]} 
\def\fourthellip{(-1.6, .4) ellipse [x radius=3cm, y radius=1.5cm, rotate=-50]} 

\draw \firstellip node [outer sep=.8cm, above right, xshift=1.1cm, yshift=1.6cm] {$A$};
\draw \secondellip node [outer sep=.8cm, above right, xshift=1.1cm, yshift=1.6cm]{$B$};
\draw \thirdellip node [outer sep=.8cm, above left, xshift=-1.1cm, yshift=1.6cm]{$C$};
\draw \fourthellip node [outer sep=.8cm, above left, xshift=-1.1cm, yshift=1.6cm] {$D$}
	node at (-3,2) {1}
	node at (-1.8,3.2) {2}
	node at (1.8,3.2) {3}
	node at (3,2) {4}
	node at (1.4,-.6) {5}
	node at (0,-1.5) {6}	
	node at (-1.4,-.6) {7}
	node at (-2,2) {8}
	node at (0,2) {9}	
	node at (2,2) {10}	
	node at (.45,-.8) {\footnotesize{11}}
	node at (-.45,-.8) {\footnotesize{12}}
	node at (-.8,.8) {13}
	node at (.8,.8) {14}
	node at (0,0) {15}
	node at (3,-1.5) {16};
\end{tikzpicture}  
\end{center}

The diagram is complex, and it takes some practice to learn to work with it. However, the basic methods for using it are the same as with the three-term diagram. Consider this argument, taken from Lewis Carroll's logic textbook \citep{Dodgson1896}.

\begin{earg}
\item[P$_1$:] Nobody who is despised can manage a crocodile.
\item[P$_2$:] Illogical persons are despised.
\item[P$_3$:] Babies are illogical.
\vspace{-.5em}
\item [] \rule{0.4\linewidth}{.5pt} 
\item[C:] No babies can manage a crocodile.
\end{earg} 

If we set the major term ``Persons who can manage a crocodile'' as $A$, the minor term ``babies'' as $D$, and the two middle terms as $B$ and $C$, we get this for the Venn diagram.

\begin{tabu}{X[1,c,m]X[1,c,m]}

\begin{ekey}
\item[$A$:] Persons who can manage a crocodile
\item[$B$:] Persons who are despised
\item[$C$:] Illogical persons
\item[$D$:] Babies
\end{ekey}

&

\begin{tikzpicture}[scale=.75,  every node/.style={scale=0.75}]
%definitions
\def\firstellip{(1.6, .4) ellipse [x radius=3cm, y radius=1.5cm,rotate=50]}
\def\secondellip{(0.3, 1.4cm) ellipse [x radius=3cm, y radius=1.5cm, rotate=50]} 
\def\thirdellip{(-0.3, 1.4cm) ellipse [x radius=3cm, y radius=1.5cm, rotate=-50]} 
\def\fourthellip{(-1.6, .4) ellipse [x radius=3cm, y radius=1.5cm, rotate=-50]} 

%fills
\begin{scope} %P1
\clip \firstellip; 
\filldraw[gray, opacity=.5] \secondellip;
\end{scope}

\begin{scope}[even odd rule] 
\clip \secondellip (-3,-1.55) rectangle (2,4);
\fill[gray, opacity=.5] \thirdellip;
\end{scope}

\begin{scope}[even odd rule] 
\clip \thirdellip (-5,-3) rectangle (4,5);
\fill[gray, opacity=.5] \fourthellip;
\end{scope}

%shapes

\draw \firstellip node [outer sep=.8cm, above right, xshift=1.1cm, yshift=1.6cm] {$A$};
\draw \secondellip node [outer sep=.8cm, above right, xshift=1.1cm, yshift=1.6cm]{$B$};
\draw \thirdellip node [outer sep=.8cm, above left, xshift=-1.1cm, yshift=1.6cm]{$C$};
\draw \fourthellip node [outer sep=.8cm, above left, xshift=-1.1cm, yshift=1.6cm] {$D$};
\end{tikzpicture}  

\end{tabu}

%%%%%%%%%%%%%%%%%%%%%Practice problems

\practiceproblems

\noindent\problempart \label{venn_set1} Rewrite the following arguments in standard form, reducing the number of terms if necessary. Then supply the missing intermediate conclusions and evaluate the arguments with Venn diagrams.

\begin{longtabu}{{X[1,l,p]X[9,l,p]}}
\textbf{Example}: & 
\vspace{-16pt}
\begin{earg} 
\item[P$_1$:] No $D$ are non-$B$.
\item[P$_2$:] All $B$ are $A$. 
\item[P$_3$:] No $A$ are $C$. 
\vspace{-.5em} 
\item [] \rule{0.2\linewidth}{.5pt} 
\item[C:] Some $D$ are not $C$. 
\end{earg}
\\
\textbf{Answer}: & Standard form:  \begin{earg} 
\item[P$_1$:] No $A$ are $C$.
\item[P$_2$:] All $B$ are $A$. % No $C$ are $A$
\item[P$_3$:] All $D$ are $B$.
\vspace{-.5em} 
\item [] \rule{0.2\linewidth}{.5pt} 
\item[C:] Some $D$ are not $C$. 
\end{earg}\\

&

\begin{tikzpicture}%
%%%% Venn 1
\begin{scope}
\def\firstcircle{(0,0) circle (.75cm)}
\def\secondcircle{(60:.75cm) circle (.75cm)}
\def\thirdcircle{(0:.75cm) circle (.75cm)}

\begin{scope} %shade overlap between S and M
\clip \thirdcircle;
\fill[gray] \secondcircle;
\end{scope}

\begin{scope}[even odd rule] % Shade P without M
\clip \secondcircle (-1,-1) rectangle (2,2);
\fill[gray] \firstcircle;
\end{scope}

\draw \firstcircle node[outer sep=.66cm, below left] {$B$};
\draw \secondcircle node [outer sep=.75cm, above] {$A$};
\draw \thirdcircle node [outer sep=.66cm, below right] {$C$};
\end{scope}


%%%% Venn 2
\begin{scope}[shift={(4.25cm,0cm)}]
\def\firstcircle{(0,0) circle (.75cm)}
\def\secondcircle{(60:.75cm) circle (.75cm)}
\def\thirdcircle{(0:.75cm) circle (.75cm)}

\begin{scope} %shade overlap between S and M
\clip \thirdcircle;
\fill[gray] \secondcircle;
\end{scope}

\begin{scope}[even odd rule] % Shade P without M
\clip \secondcircle (-1,-1) rectangle (2,2);
\fill[gray] \firstcircle;
\end{scope}

\draw \firstcircle node[outer sep=.66cm, below left] {$D$};
\draw \secondcircle node [outer sep=.75cm, above] {$B$};
\draw \thirdcircle node [outer sep=.66cm, below right] {$C$};
\end{scope}

%%%%%%%%%%%% Arg 1
 \node at (0,-2.65) [text width=4.5cm, outer sep=1mm] (Cesare){
\begin{earg}
\item[P$_1$:] No $A$ are $C$.
\item[P$_2$:] All $B$ are $A$. 
\vspace{-.5em}
\item [] \rule{0.7\linewidth}{.5pt} 
\item[C:] No $B$ are $C$.
\end{earg}
};

%%%%%%%%%%%% Arg 2
\node at (5.25,-2.5)[text width=4.5cm, outer sep=1mm] (Ferison){
\begin{earg}
\item[P$_1$:] No $B$ are $C$.
\item[P$_2$:] All $D$ are $B$.
\vspace{-.5em}
\item [] \rule{0.9\linewidth}{.5pt} 
\item[C:] Some $D$ are not $C$.
\end{earg}
};

\draw [myarrow1, ->] (1, -3.3) .. controls (2.125, -3.3) and (2.125, -2.15) .. (3.15, -2.15);

\node at (0,-4){\textbf{Celarent (EAE-I)}}; 	
\node at (5,-4) {\textbf{Celaront (EAO-1)}};  

\end{tikzpicture}

\\
& Conditionally valid. It also needs the premise ``Some $D$ exist.''

\end{longtabu}

\begin{exercises}

\begin{longtabu}{X[1,p,m]X[1,p,m]} 

\item \begin{earg} 
\item[P$_1$:] All $B$ are $A$.
\item[P$_2$:] Some $C$ are $D$. 
\item[P$_3$:] No $D$ are $A$.
\vspace{-.5em} 
 \item [] \rule{0.4\linewidth}{.5pt} 
\item[C:] Some $C$ are not $B$.
 \end{earg} 

%\begin{earg} 
%\item[P$_1$:] All $B$ are $A$
%\item[P$_2$:] No $D$ are $A$ %No $D$ are $B$
%\item[P$_3$:] Some $C$ are $D$
%\vspace{-.5em} 
% \item [] \rule{0.2\linewidth}{.5pt} 
%\item[C:] Some $C$ are not $B$
% \end{earg} 

% A --> B
% B --> A
% C --> D
% D --> C

&

\item \begin{earg} 
\item[P$_1$:] Some $B$ are $C$.
\item[P$_2$:] All $D$ are $A$. 
\item[P$_3$:] Some $C$ are $D$.
\vspace{-.5em} 
 \item [] \rule{0.4\linewidth}{.5pt} 
\item[C:] All $A$ are $B$.
 \end{earg}

%\begin{earg} 
%\item[P$_1$:] Some $B$ are $C$
%\item[P$_2$:] Some $C$ are $D$
%\item[P$_3$:] All $D$ are $A$
%\vspace{-.5em} 
% \item [] \rule{0.2\linewidth}{.5pt} 
%\item[C:] All $A$ are $B$
% \end{earg}

% A --> B
% B --> C
% C --> D
% D --> A

\\

\item \begin{earg} 
\item[P$_1$:] All $B$ are $C$.  
\item[P$_2$:] Some $B$ are $A$. %Some $B$ are not $D$
\item[P$_3$:] All $D$ are non-$A$. %becomes No $D$ are $A$
\vspace{-.5em} 
 \item [] \rule{0.4\linewidth}{.5pt} 
\item[C:] Some $C$ are not $D$.
 \end{earg}
 
% \begin{earg} 
%\item[P$_1$:] All $A$ are non-$B$ %becomes No $A$ are $B$
%\item[P$_2$:] Some $C$ are $B$ %Some $C$ are not $A$
%\item[P$_3$:] All $C$ are $D$ 
%\vspace{-.5em} 
% \item [] \rule{0.2\linewidth}{.5pt} 
%\item[C:] Some $D$ are not $A$.
% \end{earg}

% A --> D
% B --> A
% C --> B
% D --> C
   
%Festino (EIO-II), Bocardo (OAO-3)

&

\item \begin{earg} 
\item[P$_1$:] No $A$ are $B$.
\item[P$_2$:] No $C$ are $B$. % No $B$ are $E$
\item[P$_3$:] All $D$ are $A$.
\item[P$_4$:] All $E$ are $C$. 
\vspace{-.5em} 
 \item [] \rule{0.4\linewidth}{.5pt} 
\item[C:] No $D$ are $E$. 
 \end{earg} 

%\begin{earg} 
%\item[P$_1$:] All $E$ are $C$
%\item[P$_2$:] No $C$ are $B$ % No $B$ are $E$
%\item[P$_3$:] No $A$ are $B$
%\item[P$_4$:] All $D$ are $A$
%\vspace{-.5em} 
% \item [] \rule{0.2\linewidth}{.5pt} 
%\item[C:] No $D$ are $E$. 
% \end{earg} 

% A --> E
% B --> C
% C --> B
% D --> A
% E --> D

\\

\item \begin{earg} 
\item[P$_1$:] All $A$ are $E$.
\item[P$_2$:] All $E$ are $D$. % All $E$ are $C$  
\item[P$_3$:] All $A$ are $B$.
\item[P$_4$:] All $D$ are $C$. 
\vspace{-.5em} 
 \item [] \rule{0.4\linewidth}{.5pt} 
\item[C:] All $B$ are $C$.
 \end{earg} 

% A --> C
% B --> D
% C --> E
% D --> A
% E --> B

%\item \begin{earg} 
%\item[P$_1$:] All $D$ are $C$
%\item[P$_2$:] All $E$ are $D$ % All $E$ are $C$  
%\item[P$_3$:] All $A$ are $E$
%\item[P$_4$:] All $A$ are $B$
%\vspace{-.5em} 
% \item [] \rule{0.2\linewidth}{.5pt} 
%\item[C:] All $B$ are $C$.
% \end{earg} 

&

\item \begin{earg} 
\item[P$_1$:] All $A$ are $E$. % All $A$ are $D$  
\item[P$_2$:] All $E$ are $D$. 
\item[P$_3$:] Some $B$ are $C$.
\item[P$_4$:] All $B$ are $A$.
\vspace{-.5em} 
 \item [] \rule{0.4\linewidth}{.5pt} 
\item[C:] Some $C$ are $D$.
 \end{earg} 

%\item \begin{earg} 
%\item[P$_1$:] All $E$ are $D$
%\item[P$_2$:] All $A$ are $E$ % All $A$ are $D$  
%\item[P$_3$:] All $B$ are $A$
%\item[P$_4$:] Some $B$ are $C$
%\vspace{-.5em} 
% \item [] \rule{0.2\linewidth}{.5pt} 
%\item[C:] Some $C$ are $D$.
% \end{earg} 

% A --> D
% B --> E
% C --> A
% D --> B
% E --> C

\\

\item \begin{earg} 
\item[P$_1$:] All $A$ are $B$.
\item[P$_2$:] No $C$ are $E$. 
\item[P$_3$:] No $D$ are non-$A$.	%Some $A$ are not $C$
\item[P$_4$:] Some $E$ are $D$. %Some $D$ are not $C$
\vspace{-.5em} 
 \item [] \rule{0.4\linewidth}{.5pt} 
\item[C:] Some $B$ are not $C$.
 \end{earg} 

%\begin{earg} 
%\item[P$_1$:] No $C$ are $E$
%\item[P$_2$:] Some $E$ are $D$ %Some $D$ are not $C$
%\item[P$_3$:] No $D$ are non-$A$	%Some $A$ are not $C$
%\item[P$_4$:] All $A$ are $B$
%\vspace{-.5em} 
% \item [] \rule{0.6\linewidth}{.5pt} 
%\item[C:] Some $B$ are not $C$
% \end{earg} 

% A --> C
% B --> E
% C --> D
% D --> A
% E --> B

&

\item \begin{earg} 
\item[P$_1$:] All $D$ are $A$. % Some $A$ are not $E$
\item[P$_2$:] All $A$ are $C$.
\item[P$_3$:] All $E$ are $B$. 
\item[P$_4$:] Some $D$ are not $B$. %  Some $D$ are not $E$
\vspace{-.5em} 
 \item [] \rule{0.4\linewidth}{.5pt} 
\item[C:] Some $C$ are not $E$.
 \end{earg} 

%\begin{earg} 
%\item[P$_1$:] All $E$ are $B$
%\item[P$_2$:] Some $D$ are not $B$ %  Some $D$ are not $E$
%\item[P$_3$:] All $D$ are $A$ % Some $A$ are not $E$
%\item[P$_4$:] All $A$ are $C$
%\vspace{-.5em} 
% \item [] \rule{0.2\linewidth}{.5pt} 
%\item[C:] Some $C$ are not $E$
% \end{earg} 

% A --> E
% B --> B
% C --> D
% D --> A
% E --> C

\\

\item \begin{earg} 
\item[P$_1$:] All $E$ are $D$. %  Some $D$ are not $F$ 
\item[P$_2$:] No $E$ are $F$.
\item[P$_3$:] All $B$ are $A$. 
\item[P$_4$:] All $C$ are $B$.  
\item[P$_5$:] All $D$ are $C$.  
\vspace{-.5em} 
 \item [] \rule{0.4\linewidth}{.5pt} 
\item[C:] Some $A$ are not $F$.
 \end{earg} 

 % A --> F
% B --> E
% C --> D
% D --> C
% E --> B
% F --> A
% 5 term conditionally valid.  
&

\item \begin{earg} 
\item[P$_1$:] All $B$ are $A$. 
\item[P$_2$:] Some $C$ are $E$. %  Some $C$ are $F$
\item[P$_3$:] No $A$ are $D$.
\item[P$_4$:] All $C$ are $B$.%   %
\item[P$_5$:] All $E$ are $F$. 
\vspace{-.5em} 
\item [] \rule{0.4\linewidth}{.5pt} 
\item[C:] Some $D$ are $F$.
 \end{earg} 
%5 premises term reduction invalid 2
% A --> F
% B --> E
% C --> C
% D -->B
% E --> A
% F --> D 

\end{longtabu}

\end{exercises}

\noindent\problempart \label{venn_set2} Rewrite the following arguments in standard form, reducing the number of terms if necessary. Then supply the missing intermediate conclusions and evaluate the arguments with Venn diagrams.

\begin{exercises}
\begin{longtabu}{X[1,p,m]X[1,p,m]} 
\item \begin{earg}
\item[P$_1$:] No $D$ are $C$.
\item[P$_2$:] No $A$ are $B$. 
\item[P$_3$:] Some $D$ are $A$ %Some $D$ are not $B$
\vspace{-.5em}
\item [] \rule{0.4\linewidth}{.5pt} 
\item[C:] Some $C$ are not $B$.
\end{earg} 
%3 premise, invalid

% * P --> B
% * M --> A
% * M2 --> D
% * S --> C

& 
\item \begin{earg}
\item[P$_1$:] All $C$ are $B$.
\item[P$_2$:] No $B$ are $D$. %  No $D$ are $C$
\item[P$_3$:] All $A$ are $D$.
\vspace{-.5em}
\item [] \rule{0.4\linewidth}{.5pt} 
\item[C:] Some $A$ are not $C$.
\end{earg} 
%3 premise, conditionally valid--EA family

% * P -->  C
% * M --> B
% * M2 --> D
% * S --> A

\\
\item \begin{earg}
\item[P$_1$:] Some $M$ are not $P$.
\item[P$_2$:] No $M$ are non-$M2$. % Some M2 are not $P$
\item[P$_3$:] No $M2$ are non-$S$.
\vspace{-.5em}
\item [] \rule{0.4\linewidth}{.5pt} 
\item[C:] Some $S$ are not $P$.
\end{earg} 
%3 premise, reduce terms

% * P --> D
% * M --> C
% * M2 --> B
% * S --> A

&
\item \begin{earg}
\item[P$_1$:] All $A$ are $B$. %some $A$ are $C$
\item[P$_2$:] All $E$ are $A$.
\item[P$_3$:] All $D$ are $C$.
\item[P$_4$:] All $D$ are $B$. % Some $B$ are $C$
\vspace{-.5em}
\item [] \rule{0.4\linewidth}{.5pt} 
\item[C:] Some $E$ are $C$.
\end{earg} 
%4 premise, conditionally valid--AI family

% * P --> C
% * M --> D
% * M2 --> B
% * M3 --> A
% * S --> E 
\\
\item \begin{earg}
\item[P$_1$:] Some $E$ are $C$. % Some $C$ are $D$
\item[P$_2$:] Some $A$ are $C$.
\item[P$_3$:] All $B$ are $D$.
\item[P$_4$:] All $E$ are $B$.	%All $E$ are $D$
\vspace{-.5em}
\item [] \rule{0.4\linewidth}{.5pt} 
\item[C:] Some $A$ are $D$. 
\end{earg} 
%4 premise

% * P --> D
% * M --> B
% * M2 --> E
% * M3 --> C
% * S --> A

&
\item \begin{earg}
\item[P$_1$:] No $B$ are $C$.
\item[P$_2$:] Some $B$ are not $A$. % Some $B$ are not $E$ 
\item[P$_3$:] All $C$ are $D$.
\item[P$_4$:] All $E$ are $A$.
\vspace{-.5em}
\item [] \rule{0.4\linewidth}{.5pt} 
\item[C:] Some $D$ are not $E$. 
\end{earg} 
%4 premise, invalid

% * P --> E
% * M --> A
% * M2 --> B
% * M3 --> C
% * S --> D

\\
\item \begin{earg}
\item[P$_1$:] All $D$ are $A$.
\item[P$_2$:] All $C$ are $D$.
\item[P$_3$:] No $E$ are $B$.
\item[P$_4$:] No $B$ are $A$.
\vspace{-.5em}
\item [] \rule{0.4\linewidth}{.5pt} 
\item[C:] No $C$ are $E$. 
\end{earg} 
%4 premise, invalid

% * P --> E
% * M --> B
% * M2 --> A
% * M3 --> D
% * S --> C

&
\item \begin{earg}
\item[P$_1$:] No $A$ are non-$D$.
\item[P$_2$:] No $D$ are $C$. %No $C$ are $A$
\item[P$_3$:] All $B$ are $C$. %No $B$ are $A$
\item[P$_4$:] All non-$B$ are non-$E$
\vspace{-.5em}
\item [] \rule{0.4\linewidth}{.5pt} 
\item[C:] No $E$ are $A$.
\end{earg} 
%4 premise, reduce terms

% * P --> A
% * M --> D
% * M2 --> C
% * M3 --> B
% * S --> E

\\
\item \begin{earg}
\item[P$_1$:] No $E$ are $F$.
\item[P$_2$:] All $D$ are $B$. % All $D$ are $A$
\item[P$_3$:] All $C$ are $E$.
\item[P$_4$:] Some $D$ are not $F$.
\item[P$_5$:] All $B$ are $A$.
\vspace{-.5em}
\item [] \rule{0.4\linewidth}{.5pt} 
\item[C:] Some $C$ are not $A$.
\end{earg} 
%5 premise, invalid

% * P --> A
% * M --> B
% * M2 --> D
% * M3 --> F
% * M4 --> E
% * S --> C

&
\item \begin{earg}
\item[P$_1$:] No $B$ are $D$.
\item[P$_2$:] All $E$ are $A$.
\item[P$_3$:] All $F$ are $C$. %All $F$ are $A$
\item[P$_4$:] All $C$ are $E$. % All $C$ are $A$
\item[P$_5$:] All $D$ are $F$. % All $D$ are $A$
\vspace{-.5em}
\item [] \rule{0.4\linewidth}{.5pt} 
\item[C:] No $B$ are $A$. 
\end{earg} 
%5 premise, reduce terms

% * P --> A
% * M --> E
% * M2 --> C
% * M3 --> F
% * M4 --> D
% * S --> B

\end{longtabu}
\end{exercises}

\noindent\problempart Do the exercises in Part \ref{venn_set1}, but skip writing out the missing intermediate conclusions, and evaluate them using the rules method instead of the Venn diagram method.

\begin{longtabu}{p{.1\linewidth}p{.9\linewidth}}

\textbf{Example}: & \vspace{-16pt} \begin{earg} 
\item[P$_1$:] No $D$ are non-$B$.
\item[P$_2$:] All $B$ are $A$. 
\item[P$_3$:] No $A$ are $C$. 
\vspace{-.5em} 
\item [] \rule{0.3\linewidth}{.5pt} 
\item[C:] Some $D$ are not $C$. 
\end{earg} \\
\textbf{Answer}: & \vspace{-16pt} \begin{earg} 
\item[P$_1$:] No $A$\textsuperscript{D} are $C$\textsuperscript{D}.
\item[P$_2$:] All $B$\textsuperscript{D} are $A$. % No $C$ are $A$
\item[P$_3$:] All $D$\textsuperscript{D} are $B$.
\vspace{-.5em} 
\item [] \rule{0.3\linewidth}{.5pt} 
\item[C:] Some $D$ are not $C$\textsuperscript{D}. 
\end{earg}
\\ & 
\textbf{Rule 1:} Pass. Middle terms $A$ and $B$ are distributed in P$_1$ and P$_2$. \newline
\textbf{Rule 2:} Pass. The conclusion distributes $C$, which is also distributed in  P$_1$.\newline
\textbf{Rule 3:} Pass. There is one negative premise, P$_1$. \newline
\textbf{Rule 4:} Pass. The conclusion is negative, and there is exactly one negative premise.\newline
\textbf{Rule 5:} Fail. The premises are all universal and the conclusion is particular. \\
& Conditionally valid. It needs the premise ``Some $D$ exist.''
\end{longtabu}


\noindent\problempart Do the exercises in Part \ref{venn_set2}, but skip writing out the missing intermediate conclusions, and evaluate them using the rules method instead of the Venn diagram method.


\noindent\problempart  Rewrite the following arguments in standard form, reducing terms if necessary, and then prove that they are valid using a single, four-term Venn diagram.

\begin{longtabu}{p{.1\linewidth}p{.8\linewidth}}

\textbf{Example}: & \vspace{-16pt} \begin{earg} 
\item[P$_1$:] No $D$ are non-$B$.
\item[P$_2$:] All $B$ are $A$. 
\item[P$_3$:] No $A$ are $C$. 
\vspace{-.5em} 
\item [] \rule{0.3\linewidth}{.5pt} 
\item[C:] Some $D$ are not $C$. 
\end{earg} \\
\textbf{Answer}: & \vspace{-16pt} \begin{earg} 
\item[P$_1$:] No $A$ are $C$.
\item[P$_2$:] All $B$ are $A$. % No $C$ are $A$
\item[P$_3$:] All $D$ are $B$.
\vspace{-.5em} 
\item [] \rule{0.3\linewidth}{.5pt} 
\item[C:] Some $D$ are not $C$.
\end{earg}
\\ & 

\begin{tikzpicture}

%definitions
\def\firstellip{(1.6, .4) ellipse [x radius=3cm, y radius=1.5cm,rotate=50]}
%
\def\secondellip{(0.3, 1.4cm) ellipse [x radius=3cm, y radius=1.5cm, rotate=50]} 

\def\thirdellip{(-0.3, 1.4cm) ellipse [x radius=3cm, y radius=1.5cm, rotate=-50]} 

\def\fourthellip{(-1.6, .4) ellipse [x radius=3cm, y radius=1.5cm, rotate=-50]} 

\def\firstrect{[rotate around={50:(.85, -2.9)}](.85, -2.9) rectangle ++(6.05, 3.05)}

\def\secondrect{[rotate around={50:(-.45, -1.85)}](-.45, -1.85) rectangle ++(6.05, 3)}

\def\thirdrect{[rotate around={-50:(.45, -1.85)}](.45, -1.85) rectangle ++(-6.05, 3)}

\def\fourthrect{[rotate around={-50:(-.85, -2.9)}](-.85, -2.9) rectangle ++(-6.05, 3.05)}

\def\bounding{(-5,-3) rectangle (5,4)}

%fills

\begin{scope} %P1
\clip \secondellip; 
\filldraw[gray] \fourthellip;
\end{scope}
%
\begin{scope}[even odd rule] 
\clip \secondellip \thirdrect;
\fill[gray] \thirdellip;
\end{scope}

\begin{scope}[even odd rule] 
\clip \thirdellip \firstrect;
\fill[gray] \firstellip;
\end{scope}

%shapes

\draw \firstellip node [outer sep=.8cm, above right, xshift=1.1cm, yshift=1.6cm] {$D$};

\draw \secondellip node [outer sep=.8cm, above right, xshift=1.1cm, yshift=1.6cm]{$A$};

\draw \thirdellip node [outer sep=.8cm, above left, xshift=-1.1cm, yshift=1.6cm]{$B$};

\draw \fourthellip node [outer sep=.8cm, above left, xshift=-1.1cm, yshift=1.6cm] {$C$};
\end{tikzpicture}  

Conditionally valid. If you had a premise that told you to write an x in the $D$ ellipse, it would clearly fall outside of $C$.

\end{longtabu}


\begin{exercises}
\begin{longtabu}{X[1,p,m]X[1,p,m]} 
\item \begin{earg}
\item[P$_1$:] No $D$ are $B$. 
\item[P$_2$:] All $A$ are $C$.
\item[P$_3$:] All $C$ are $B$.
\vspace{-.5em}
\item [] \rule{0.6\linewidth}{.5pt} 
\item[C:] No $D$ are $A$.
\end{earg} 

% Invalid
% * P --> A
% * M1 --> C 
% * M2 --> B
% * S --> D

&
\item\begin{earg}
\item[P$_1$:] All $A$ are $B$.
\item[P$_2$:] No $D$ are $C$.
\item[P$_3$:] All $C$ are $A$. % All C are B
\vspace{-.5em}
\item [] \rule{0.3\linewidth}{.5pt} 
\item[C:] No $D$ are $B$.
\end{earg}

% Valid
% * P --> B
% * M1 --> A
% * M2 --> C
% * S --> D

\\
\item\begin{earg}
\item[P$_1$:] All $B$ are $D$. %Some $B$ are not $A$
\item[P$_2$:] All $B$ are $C$.
\item[P$_3$:] No $A$ are $D$.
\vspace{-.5em}
\item [] \rule{0.3\linewidth}{.5pt} 
\item[C:] Some $C$ are not $A$.
\end{earg}

% conditionally valid
% * P --> A
% * M1 --> D
% * M2 --> B
% * S --> C

&
\item\begin{earg}
\item[P$_1$:] No $A$ are $C$.
\item[P$_2$:] All $B$ are $D$.
\item[P$_3$:] All $C$ are $B$. %All C are D
\vspace{-.5em}
\item [] \rule{0.6\linewidth}{.5pt} 
\item[C:] No $A$ are $D$.
\end{earg}

% Valid, reduce terms
% * P --> D
% * M1 --> B
% * M2 --> C
% * S --> A

\\
\item\begin{earg}
\item[P$_1$:] All $D$ are $A$.
\item[P$_2$:] Some $C$ are $B$.
\item[P$_3$:] Some $B$ are not $D$.
\vspace{-.5em}
\item [] \rule{0.6\linewidth}{.5pt} 
\item[C:] Some non-$C$ are not non-$A$.
\end{earg}

% Invalid, reduce terms
% * P --> C
% * M1 --> B 
% * M2 --> D
% * S --> A

&
\end{longtabu}
\end{exercises}

\noindent\problempart  Rewrite the following arguments in standard form, reducing terms if necessary, and then prove that they are valid using a single, four-term Venn diagram.
            
\begin{exercises}
\begin{longtabu}{X[1,p,m]X[1,p,m]} 
\item \begin{earg}
\item[P$_1$:] No $A$ are $C$.
\item[P$_2$:] All $B$ are $D$.
\item[P$_3$:] Some $B$ are $A$.
\vspace{-.5em}
\item [] \rule{0.6\linewidth}{.5pt} 
\item[C:] Some $S$ are not $C$.
\end{earg} 
%
%% Valid
%% * P --> C
%% * M1 --> A
%% * M2 --> B
%% * S --> D
%
&
\item\begin{earg}
\item[P$_1$:] No $P$ are $M$.
\item[P$_2$:] All $M$ are $M2$.
\item[P$_3$:] No $M2$ are $S$.
\vspace{-.5em}
\item [] \rule{0.6\linewidth}{.5pt} 
\item[C:] No $S$ are $P$.
\end{earg}
%
%% Invalid
%% * P --> B
%% * M1 --> A
%% * M2 --> C
%% * S --> D
%
\\
\item\begin{earg}
\item[P$_1$:] All $A$ are $C$.
\item[P$_2$:] All $B$ are $C$.
\item[P$_3$:] All $A$ are $D$.
\vspace{-.5em}
\item [] \rule{0.6\linewidth}{.5pt} 
\item[C:] Some $D$ are $B$.
\end{earg}

% Conditionally  valid
% * P --> B
% * M1 --> C
% * M2 --> A
% * S --> D

&
\item\begin{earg}
\item[P$_1$:] Some non-$D$ are not non-$C$.
\item[P$_2$:] All $D$ are $B$.
\item[P$_3$:] No $B$ are $A$.
\vspace{-.5em}
\item [] \rule{0.6\linewidth}{.5pt} 
\item[C:] Some $A$ are not $C$.
\end{earg}
% Invalid, reduce terms
% * P --> C
% * M1 --> D
% * M2 --> B
% * S --> A
\\
\item\begin{earg}
\item[P$_1$:] No $C$ are non-$B$.
\item[P$_2$:] Some $D$ are $A$.
\item[P$_3$:] All $D$ are $C$. %Some $C$ are $A$
\vspace{-.5em}
\item [] \rule{0.6\linewidth}{.5pt} 
\item[C:] Some $B$ are not $A$.
\end{earg}
% Valid, reduce terms
% * P --> A
% * M1 --> D
% * M2 --> C
% * S --> B

\end{longtabu}
\end{exercises}

\noindent\problempart Rewrite the following arguments in standard form using variables and a translation key. Then evaluate using any method you want. The example problem, along with exercises \ref{itm:sons}, \ref{itm:ducks},  \ref{itm:Auk}, and \ref{itm:rainbow}, come from Lewis Carroll's logic textbook \citep{Dodgson1896}. Other exercises are just Lewis Carroll themed.

\begin{longtabu}{p{.1\linewidth}p{.5\linewidth}p{.4\linewidth}}

\textbf{Example}: & \multicolumn{2}{p{.9\linewidth}}{My saucepans are the only things I have that are made of tin. I find all your presents very useful, but none of my saucepans are of the slightest use. Therefore, none of your presents are made of tin.} \\
\\
\textbf{Answer}:&
\vspace{-.5cm}
\begin{ekey}
\item[$A$:]Things of mine made from tin
\item[$B$:] Saucepans
\item[$C$:] Useful things 
\item[$D$:] Presents from you
\end{ekey}

& 
\vspace{-.5cm}
\begin{earg}
\item[P$_1$:] All $A$ are $B$.
\item[P$_2$:] No $B$ are $C$.
\item[P$_3$:] All $D$ are $C$. 
\vspace{-.5em}
\item [] \rule{0.6\linewidth}{.5pt} 
\item[C:] No $D$ are $A$.
\end{earg}
\end{longtabu}
\vspace{-.75cm}
\begin{center}
\begin{tikzpicture}
\begin{scope}
% venns
\begin{scope}
\def\firstcircle{(0,0) circle (.75cm)}
\def\secondcircle{(60:.75cm) circle (.75cm)}
\def\thirdcircle{(0:.75cm) circle (.75cm)}

\begin{scope} 
\clip \thirdcircle;
\fill[gray] \secondcircle;
\end{scope}

\begin{scope}[even odd rule] % Shade P without M
\clip \secondcircle (-1,-1) rectangle (2,2);
\fill[gray] \firstcircle;
\end{scope}

\draw \firstcircle node[outer sep=.66cm, below left] {$A$};
\draw \secondcircle node [outer sep=.75cm, above] {$B$};
\draw \thirdcircle node [outer sep=.66cm, below right] {$C$};

\end{scope}

\begin{scope}[xshift=4cm]

\def\firstcircle{(0,0) circle (.75cm)}
\def\secondcircle{(60:.75cm) circle (.75cm)}
\def\thirdcircle{(0:.75cm) circle (.75cm)}

\begin{scope} %shade overlap between S and M
\clip \thirdcircle;
\fill[gray] \secondcircle;
\end{scope}

\begin{scope}[even odd rule] % Shade P without M
\clip \secondcircle (-1,-1) rectangle (2,2);
\fill[gray] \firstcircle;
\end{scope}

\draw \secondcircle node [outer sep=.75cm, above] {$C$};
\draw \thirdcircle node [outer sep=.66cm, below right] {$A$};
\draw \firstcircle node[outer sep=.66cm, below left] {$D$};

\end{scope}

%% args

\node at (0,-2.5)[text width=4.5cm, outer sep=1mm] (Celarent){ 
\begin{earg}
\item[P$_1$:] All $A$ are $B$.
\item[P$_2$:] No $B$ are $C$.
\vspace{-.5em}
\item [] \rule{0.6\linewidth}{.5pt} 
\item[C:] No $A$ are $C$.* 
\end{earg} 

};

\node at (-.1, -3.9) {\textbf{Celarent (EAE-I)}};

\draw [myarrow1, ->] (1.1, -3.05) .. controls (2.4, -3) and (2.4, -2) .. (3.3, -1.95);
%
%\filldraw [red] (0,0) circle (.1cm);
%\filldraw [red] (2.4, -.5) circle (.1cm); 
%\filldraw [red] (2.4, .5)  circle (.1cm);

\node at (5,-2.3)[text width=4.5cm, outer sep=1mm] { 
\begin{earg} 
\item[P$_1$:] No $A$ are $C$.*
\item[P$_2$:] All $D$ are $C$.
\vspace{-.5em} 
\item [] \rule{0.6\linewidth}{.5pt} 
\item[C:] No $D$ are $A$.
\end{earg}
%Cesare (EAE-II)
};

\node at (4.9, -3.9) {\textbf{Cesare (EAE-II)}};
\end{scope}
\end{tikzpicture}
\end{center}

\begin{exercises}

\item All metal things are solid, and some of those metal things are also chairs. All chairs are furniture. Therefore some solid things are furniture. 

%Three premise valid


%\begin{earg} 
%\item[P$_1$:] All $M1$ are $P$
%\item[P$_2$:] Some $M2$ are $M1$
%\item[P$_3$:] All $M2$ are $S$
%\vspace{-.5em} 
% \item [] \rule{0.6\linewidth}{.5pt} 
%\item[C:] Some $S$ are $P$
% \end{earg}


\item \label{itm:sons} Every one who is sane can do Logic. No lunatics are fit to serve on a jury. None of \textit{your} sons can do Logic. Therefore, none of your sons can serve on a jury.

%three premise valid

\item All hat-wearers are heaven-sent. This is because all platypuses are heaven-sent, all secret agents are platypuses, and some secret agents wear hats. 

%\begin{earg} 
%\item[P$_1$:] All platypuses are heaven-sent
%\item[P$_2$:] All secret agents are platypuses     %All $C$ are $A$
%\item[P$_3$:] Some secret agents wear hats
%\vspace{-.5em} 
% \item [] \rule{0.6\linewidth}{.5pt} 
%\item[C:] All hat-wearers are heaven-sent
% \end{earg}


\item \label{itm:ducks} No ducks waltz. No officers ever decline to waltz. All my poultry are ducks. Therefore, none of my poultry are officers.



%\begin{earg}
%\item[P$_1$:] All my poultry are ducks
%\item[P$_2$:] No ducks waltz 
%\item[P$_3$:] All officers waltz 
%\vspace{-.5em}
%\item [] \rule{0.6\linewidth}{.5pt} 
%\item[C:] None of my poultry are officers
%\end{earg} 

\item Some things are uffish, but nothing that is uffish is vorpal. Therefore some uffish things are not swords, because everything that is vorpal goes snicker-snack, and everything that goes snicker-snack is a sword. 

%\begin{ekey}
%\item[$A$:] Swords. 
%\item[$B$:] Things that go snicker-snack
%\item[$C$:] Vorpal things.
%\item[$D$:] Uffish things
%\end{ekey}



%\begin{earg}
%\item[P$_1$:] Everything that goes snicker-snack is a sword.
%\item[P$_2$:] Everything that is vorpal goes snicker-snack  
%& Everything that is vorpal is a sword
%\item[P$_3$:] Nothing uffish is vorpal
%\item[{\color{red}P$_4$:}] {\color{red}Some uffish things exist.}
%\vspace{-.5em}
%\item [] \rule{0.2\linewidth}{.5pt} 
%\item[C:] Some uffish things are not swords
%\end{earg} 

%\begin{earg}
%\item[P$_1$:] All $B$ are $A$.
%\item[P$_2$:] All $C$ are $B$
%\item[P$_3$:] No $D$ are $C$
%\item[{\color{red}P$_4$:}] {\color{red}$D$ exists.}
%\vspace{-.5em}
%\item [] \rule{0.2\linewidth}{.5pt} 
%\item[C:] Some $D$ are not $A$
%\end{earg} 

\item No animals are plants. Some animals are mammals. Some mammals are dogs. All dachshunds are dogs. Therefore, no dachshunds are plants.


%\begin{earg}
%\item[P$_1$:] No $B$ are $A$
%\item[P$_2$:] Some $B$ are $C$.    % Therefore, some $C$ are not $A$. 
%\item[P$_3$:] Some $C$ are $D$     
%\item[P$_4$:] All $E$ are $D$
%\vspace{-.5em}
%\item [] \rule{0.6\linewidth}{.5pt} 
%\item[C:]  No $E$ are $A$. 
%\end{earg} 


\item \label{itm:Auk} Things sold in the street are of no great value. Nothing but rubbish can be had for a song. Eggs of the Great Auk are very valuable. It is only what is sold in the streets that is really rubbish. Therefore the eggs of the Great Auk cannot be had for a song.

%\begin{earg}
%\item[P$_1$:] Nothing but rubbish can be had for a song
%\item[P$_2$:] It is only what is sold in the streets that is really rubbish.
%\item[P$_3$:] Things sold in the street are of no great value; 
%\item[P$_3$:] Eggs of the Great Auk are very valuable
%\vspace{-.5em}
%\item [] \rule{0.6\linewidth}{.5pt} 
%\item[C:] The eggs of the Great Auk cannot be had for a song
%\end{earg} 

%\begin{ekey}
%\item[$A$:] Things you can have for a song
%\item[$B$:] Rubbish
%\item[$C$:] Valuable things
%\item[$D$:] Things sold in the street
%\item[$E$:] Eggs of the Great Auk
%\end{ekey}

%\begin{earg}
%\item[P$_1$:] All $A$ are $B$
%\item[P$_2$:] All $B$ are $C$
%\item[P$_3$:] No $C$ are $D$; 
%\item[P$_3$:] All $E$ are $D$
%\vspace{-.5em}
%\item [] \rule{0.6\linewidth}{.5pt} 
%\item[C:] No E are A.
%\end{earg} 

\item All life forms are physical objects, and all bacteria are life forms. Also, all \textit{E. coli} are bacteria, so all rod shaped things are physical objects, because all \textit{E. coli} are rod-shaped. 
   
% four premise invalid 
 
%\begin{earg}
%\item[P$_1$:] All $B$ are $A$
%\item[P$_2$:] All $C$ are $B$
%\item[P$_3$:] All $D$ are $C$; 
%\item[P$_4$:] All $D$ are $E$
%\vspace{-.5em}
%\item [] \rule{0.6\linewidth}{.5pt} 
%\item[C:] All E are A.
%\end{earg}  




\item Some playing cards shout ``Off with his head!'' All playing cards play croquet. Everyone who plays croquet uses hedgehogs for balls. Everyone who uses hedgehogs for balls uses flamingos for mallets. The queen of hearts uses flamingos for mallets. Therefore some cards identical to the queen of hearts shout ``Off with his head!''

%Playing cards
%``shout off with his head'!'
%Plays croquet 
%Uses hedgehogs for balls 
%uses flamingos for mallets
%Is identical to the queen of hearts

%\begin{earg} 
%\item[P$_1$:] Some $B$ are $A$
%\item[P$_2$:] All $B$ are $C$     %Some $C$ are $A$ 	Disamis (IAI-III) (valid) 
% \item[P$_3$:] All $C$ are $D$ 	% Some $D$ are $A$ %		Disamis (IAI-III) (valid) 
%\item[P$_4$:] All $D$ are $E$ % Some $E$ are $A$ Disamis (IAI-III) (valid) 
%\item[P$_5$:] All $F$ are $E$ 
% \item [] \rule{0.6\linewidth}{.5pt} 
%\item[C:] Some $F$ are $A$ % IAI-I (invalid) 
% \end{earg} 

%\begin{earg} 
%\item[P$_1$:] Some $B$ are $A$
%\item[P$_2$:] All $B$ are $C$
%\vspace{-.5em} 
% \item [] \rule{0.6\linewidth}{.5pt} 
%\item[IC$_1$:] Some $C$ are $A$
% \end{earg} Disamis (IAI-III) (valid) 

%\begin{earg} 
%\item[IC$_1$:] Some $C$ are $A$
%\item[P$_3$:] All $C$ are $D$
%\vspace{-.5em} 
% \item [] \rule{0.6\linewidth}{.5pt} 
%\item[IC$_2$:] Some $D$ are $A$
% \end{earg} Disamis (IAI-III) (valid) 

%\begin{earg} 
%\item[IC$_2$:] Some $D$ are $A$
%\item[P$_4$:] All $D$ are $E$
%\vspace{-.5em} 
% \item [] \rule{0.6\linewidth}{.5pt} 
%\item[IC$_3$:] Some $E$ are $A$
% \end{earg} Disamis (IAI-III) (valid) 

%\begin{earg} 
%\item[IC$_3$:] Some $E$ are $A$
%\item[P$_5$:] All $F$ are $E$ 
% \item [] \rule{0.6\linewidth}{.5pt} 
%\item[C:] Some $F$ are $A$
% \end{earg} IAI-I (invalid) 



\item \label{itm:rainbow} I despise anything that cannot be used as a bridge. Everything, that is worth writing an ode to, would be a welcome gift to me. A rainbow will not bear the weight of a wheelbarrow.  Whatever can be used as a bridge will bear the weight of a wheel-barrow. I would not take, as a gift, a thing that I despise. Therefore a rainbow is not worth writing an ode to. 

%\begin{earg}
%\item[P$_1$:] Everything, that is worth writing an ode to, would be a welcome gift to me
%\item[P$_2$:] I would not take, as a gift, a thing that I despise. 
%\item[P$_3$:] I despise anything that cannot be used as a bridge;
%\item[P$_4$:] Whatever can be used as a bridge will bear the weight of a wheel-barrow; 
%\item[P$_5$:] A rainbow will not bear the weight of a wheelbarrow; 
%\vspace{-.5em}
%\item [] \rule{0.6\linewidth}{.5pt} 
%\item[C:] A rainbow is not worth writing an ode to. 
%\end{earg} 
\end{exercises}

\noindent\problempart Rewrite the following arguments in standard form using variables and a translation key, and supply intermediate conclusions. Then evaluate using any method you want. Exercises \ref{itm:talkers}, \ref{itm:hedgehogs}, \ref{itm:mermaids}, \ref{itm:books}, and \ref{itm: poems} come from Lewis Carroll's logic textbook \citep{Dodgson1896}. Other exercises are just Lewis Carroll themed.

\begin{exercises}

\item All buildings are habitable places, and all houses are residences. Therefore all houses are buildings because all residences are habitable places. 

%\begin{earg}
%\item[P$_1$:] All $A$ are $B$
%\item[P$_2$:] All $C$ are $B$		%All $C$ are $A$
%\item[P$_3$:] All $D$ are $C$
%\vspace{-.5em}
%\item [] \rule{0.6\linewidth}{.5pt} 
%\item[C:]  All $D$ are $A$
%\end{earg} 


\item Not all cephalopods are moralizing, but cuttlefish are. All pompous squid are cephalopods. Therefore some pompous squid are not cuttlefish. 

%
%\begin{earg}
%\item[P$_1$:] All $A$ are $B$
%\item[P$_2$:] Some $C$ are not $B$		%Some $C$ are not $A$
%\item[P$_3$:] All $D$ are $C$
%\vspace{-.5em}
%\item [] \rule{0.6\linewidth}{.5pt} 
%\item[C:] Some $D$ are not $A$
%\end{earg} 

%AEE-II, EAO-1


\item \label{itm:talkers} Showy talkers think too much of themselves; No really well-informed people are bad company; and people who think too much of themselves are not good company. Therefore no showy talkers are really well informed. %[three premise] LC 


%\begin{earg}
%\item[P$_1$:] No really well-informed people are bad company; 
%\item[P$_2$:] People who think too much of themselves are not good company.
%\item[P$_3$:] Showy talkers think too much of themselves; 
%\vspace{-.5em}
%\item [] \rule{0.6\linewidth}{.5pt} 
%\item[C:] No one who thinks too much of themselves is really well informed.
%\end{earg} 

%\begin{ekey}
%\item[$A$:] Well informed people.
%\item[$B$:] Good company
%\item[$C$:] People who think too much of themselves.
%\item[$D$:] Showy talkers
%\end{ekey}

%\begin{earg}
%\item[P$_1$:] All $A$ are $B$    %No $A$ are non-$B$
%\item[P$_2$:] No $C$ are $B$.    %No $C$ are $A$ camestres
%\item[P$_3$:] All $C$ are $D$; 
%\vspace{-.5em}
%\item [] \rule{0.6\linewidth}{.5pt} 
%\item[C:] No $D$ are $A$
%\end{earg} 



\item All white animals are late, and some late animals are rabbits. Also, all animals taking a watch out of their waistcoat pocket are white. Therefore, some rabbits are taking a watch out of their waistcoat pocket.


%$A$: animals taking a watch out of their waistcoat pocket. 
%B: White animals
%C: Late animals
%D: Rabbits


%\begin{earg}
%\item[P$_1$:] All animals taking a watch out of their waist-coat pocket are white
%\item[P$_2$:] All white animals are late 			All $M2$ are $P$
%\item[P$_3$:] Some late animals are rabbits
%\vspace{-.5em}
%\item [] \rule{0.6\linewidth}{.5pt} 
%\item[C:] Some rabbits are taking a watch out of their waist-coat pocket.
%\end{earg} 


%\begin{earg}
%\item[P$_1$:] All A are B
%\item[P$_2$:] All B are C 			All A$ are C
%\item[P$_3$:] Some C are D
%\vspace{-.5em}
%\item [] \rule{0.6\linewidth}{.5pt} 
%\item[C:] Some D are A.
%\end{earg} 

\item \label{itm:hedgehogs} No one takes in the \textit{Times} unless he is well-educated. But no hedge-hogs can read, and those who cannot read are not well-educated. Therefore no hedge-hog takes in the \textit{Times}.

\item The Red Queen is a chess piece. Among the things that have to run faster and faster just to stay in the same place is the Red Queen. All lost children have to run faster and faster just to stay in the same place. Alice is a lost child. Therefore, Alice is one of the chess pieces. 

%\begin{earg} 
%\item[P$_1$:] All $B$ are $A$
%\item[P$_2$:] Some $C$ are $B$  %Some $C$ are $A$  Darii (AII-1) (valid) 
%\item[P$_3$:] All $D$ are $C$ 
%\item[P$_4$:] All $E$ are $D$
%\vspace{-.5em} 
% \item [] \rule{0.6\linewidth}{.5pt} 
%\item[C:] Some $E$ are $A$
% \end{earg} 

\item \label{itm:mermaids} None of the unnoticed things met with at sea are mermaids. Things entered in the log as met with at sea are sure to be worth remembering. I have never met with anything worth remembering, when on a voyage. Things met with at sea, that are noticed, are sure to be recorded in the log. Therefore, I have never come across a mermaid at sea. %[four premise] LC 


%\begin{ekey}
%\item[$A$:] Mermaids
%\item[$B$:] Things noticed 
%\item[$C$:] Things recorded in the log
%\item[$D$:] Things worth remembering.
%\item[$E$:] Things met by me.
%\end{ekey}



%\begin{earg}
%\item[P$_1$:] All $A$ are $B$   %Starts are ``No $A$ are non-$B$''
%\item[P$_2$:] All $A$ are $C$
%\item[P$_3$:] All $C$ are $D$
%\item[P$_3$:] No $E$ are $D$
%\vspace{-.5em}
%\item [] \rule{0.6\linewidth}{.5pt} 
%\item[C:] No $E$ are $A$
%\end{earg} 

\item \label{itm:plum-pudding} A plum-pudding that is not really solid is mere porridge. Every plum-pudding served at my table has been boiled in a cloth. A plum-pudding that is mere porridge is indistinguishable from soup. No plum puddings are really solid except what are served at my table. Therefore no plum-pudding that has not been boiled in cloth can be distinguished from soup. %[four premise] LC

\item \label{itm:books} The only books in this library that I do \textit{not} recommend for reading are unhealthy in tone. All the bound books are well-written. All the romances are healthy in tone, and I do not recommend you read any of the unbound books. Therefore all the romances in this library are well-written. 

\item \label{itm: poems} No interesting poems are unpopular among people of real taste. No modern poetry is free from affectation. All \textit{your} poems are on the subject of soap bubbles. No affected poetry is popular among people of real taste. No ancient poem is on the subject of soap-bubbles. Therefore all of \textit{your} poems are uninteresting. %[five premise] LC

\end{exercises}
                                    
\noindent\problempart
\begin{exercises}
\item Suppose you wanted to represent five terms using a Venn diagram. How would you arrange the ellipses? 
\item Can you figure out a principle for continually adding shapes to a Venn diagram that will always allow you to represent every combination of terms?
\end{exercises}

%
% Below is the closing tag for typesetting only part of the chapter. Everything up to here to the close tag will be skipped unless the {whole_syl_chap} label at the start of this chapter file is 
%  uncommented.

}{}


\section*{Key Terms}
\begin{sortedlist}
\sortitem {Categorical syllogism}{}
\sortitem {Aristotelian syllogism}{}
\sortitem {Conditional validity}{}
\sortitem {Major premise}{}
\sortitem {Major term}{}
\sortitem {Middle term}{}
\sortitem {Minor premise}{}
\sortitem {Minor term}{}
\sortitem {Syllogism mood}{}
\sortitem {Standard form for an Aristotelian syllogism}{}
\sortitem {Translation key}{}
\sortitem {Unconditional validity}{}
\sortitem{Critical term}{}
\sortitem {Fallacy of exclusive premises}{}
\sortitem {Fallacy of illicit process}{}
\sortitem {Fallacy of particular premises}{}
\sortitem {Fallacy of the undistributed middle}{}
\sortitem {Negative-affirmative fallacy}{}
\iflabelexists{def:counter_example_method}{\sortitem {Counterexample method}{}}{}  
\iflabelexists{def:enthymeme}{\sortitem{Enthymeme}{}}{}
\iflabelexists{def:sorites_categorical_arguments}{\sortitem{Sorites categorical arguments}{}}{}
\iflabelexists{def:standard_form_for_a_sorites_categorical_argument}{\sortitem{Standard form for a sorites categorical argument}{}}{}

\end{sortedlist}
 %Label for typesetting full chapter is at the start of the file. Uncomment to get the whole thing. 
%\part{Sentential Logic} \label{part:sent_logic}
%\chapter{Sentential Logic}
\markright{Chap. \ref{chap:SL}: Sentential Logic}
\label{chap:SL}
\setlength{\parindent}{1em}

This chapter introduces a logical language called SL. It is a version of \emph{sentential logic}, because the basic units of the language will represent statements, and a statement is usually given by a complete sentence in English.




% ******************************************
%  * Section 6.1  Sentence Letters                          *
% ******************************************

\section{Sentence Letters}


\newglossaryentry{sentence letter}
{
name=sentence letter,
description={A single capital letter, used in SL to represent a basic sentence.}
}

\newglossaryentry{symbolization key}
{
name=symbolization key,
description={A list that shows which English sentences are represented by which sentence letters in SL. It is also called a dictionary.}
}

In SL, capital letters, called \textsc{\glspl{sentence letter}} \label{def:sentence_letter} are used to represent simple statements. Considered only as a symbol of SL, the letter $A$ could mean any statement. So when translating from English into SL, it is important to provide a symbolization key, or dictionary. The \textsc{\gls{symbolization key}} \label{def:symbolization_key} provides an English language sentence for each sentence letter used in the symbolization.

Consider this argument:
\begin{earg}
\item[1.] There is an apple on the desk.
\item[2.] If there is an apple on the desk, then Jenny made it to class.
\item[] \textcolor{white}{.}\sout{\hspace{.8\linewidth}}\textcolor{white}{.} 
\item[$\therefore$] Jenny made it to class.
\end{earg}
This is obviously a valid argument in English. In symbolizing it, we want to preserve the structure of the argument that makes it valid.
What happens if we replace each sentence with a letter? Our symbolization key would look like this:
\begin{ekey}
\item[A:]There is an apple on the desk.
\item[B:]If there is an apple on the desk, then Jenny made it to class.
\item[C:]Jenny made it to class.
\end{ekey}
We would then symbolize the argument in this way:
\begin{earg}
\item[1.] $A$
\item[2.] $B$
\item[] \textcolor{white}{.}\sout{\hspace{.05\linewidth}}\textcolor{white}{.} 
\item[$\therefore$] $C$
\end{earg}
There is no necessary connection between some sentence $A$, which could be any statement, and some other sentences $B$ and $C$, which could also be anything.
The structure of the argument has been completely lost in this translation.

The important thing about the argument is that the second premise is not merely \emph{any} statement, logically divorced from the other statement in the argument. The second premise contains the first premise and the conclusion \emph{as parts}. Our symbolization key for the argument only needs to include meanings for $A$ and $C$, and we can build the second premise from those pieces. So we symbolize the argument this way:
\begin{earg}
\item[1.] $A$
\item[2.] If $A$, then $C$.
\item[] \textcolor{white}{.}\sout{\hspace{.2\linewidth}}\textcolor{white}{.} 
\item[$\therefore$] $C$
\end{earg}
This preserves the structure of the argument that makes it valid, but it still makes use of the English expression ``If$\ldots$ then$\ldots$.'' Although we ultimately want to replace all of the English expressions with logical notation, this is a good start.

\newglossaryentry{atomic sentence}
{
name=atomic sentence,
description={A sentence that does not have any sentences as proper parts.}
}


The individual sentence letters in SL are called atomic sentences, because they are the basic building blocks out of which more complex sentences can be built. We can identify atomic sentences in English as well. An \textsc{\gls{atomic sentence}} \label{def:atomic_sentence} is one that cannot be broken into parts that are themselves sentences. ``There is an apple on the desk'' is an atomic sentence in English, because you can't find any proper part of it that forms a complete sentence. For instance ``an apple on the desk'' is a noun phrase, not a complete sentence. Similarly ``on the desk'' is a prepositional phrase, and not a sentence, and ``is an'' is not any kind of phrase at all. This is what you will find no matter how you divide ``There is an apple on the desk.'' On the other hand you can find two proper parts of ``If there is an apple on the desk, then Jenny made it to class'' that are complete sentences: ``There is an apple on the desk'' and ``Jenny made it to class.'' As a general rule, we will want to use atomic sentences in SL (that is, the sentence letters) to represent atomic sentences in English. Otherwise, we will lose some of the logical structure of the English sentence, as we have just seen. 

%Atomic sentences go together to make complex sentences in much the same way that physical atoms go together to make molecules. Physical atoms were originally called `atoms' because chemists thought that they were irreducible. Chemists were wrong, and physical atoms can be split.

%It is important to remember that a symbolization key only gives the meaning of atomic sentences for purposes of translating a specific argument.

There are only 26 letters of the alphabet, but there is no logical limit to the number of atomic sentences. We can use the same letter to symbolize different atomic sentences by adding a subscript, a small number written after the letter. We could have a symbolization key that looks like this:
\begin{ekey}
\item[A$_1$:] The apple is under the armoire.
\item[A$_2$:] Arguments in SL always contain atomic sentences.
\item[A$_3$:] Adam Ant is taking an airplane from Anchorage to Albany.
\item[$\vdots$]
\item[A$_{294}$:] Alliteration angers otherwise affable astronauts.
\end{ekey}
Keep in mind that each of these is a different sentence letter. When there are subscripts in the symbolization key, it is important to keep track of them.


% ******************************************
%  * Sentential Connectives		                          *
% ******************************************

\section{Sentential Connectives}

\newglossaryentry{sentential connective}
{
name=sentential connective,
description={A logical operator in SL used to combine sentence letters into larger sentences.}
}

Logical connectives are used to build complex sentences from atomic components. In SL, our logical connectives are called \textsc{\glspl{sentential connective}} \label{def:sentential_connective} because they connect sentence letters. There are five sentential connectives in SL. This table summarizes them, and they are explained below.

\begin{table}[h]
\center
\begin{tabular}{|c|c|c|}
\hline
Symbol&What it is called&What it means\\
\hline
\enot&negation&``It is not the case that$\ldots$''\\
\eand&conjunction&``Both $\ldots$\ and $\ldots$''\\
\eor&disjunction&``Either $\ldots$\ or $\ldots$''\\
\eif&conditional&``If $\ldots$\ then $\ldots$''\\
\eiff&biconditional&``$\ldots$ if and only if $\ldots$''\\
\hline
\end{tabular}
\end{table}

%%%%%%%%%%%%%%%%%% 2.2.1 Negation

\subsection{Negation}
Consider how we might symbolize these sentences:
\begin{earg}
\item[\ex{not1}] Mary is in Barcelona.
\item[\ex{not2}] Mary is not in Barcelona.
\item[\ex{not3}] Mary is somewhere other than Barcelona.
\end{earg}

In order to symbolize sentence \ref{not1}, we will need one sentence letter. We can provide a symbolization key:

\begin{ekey}
\item[B:]Mary is in Barcelona.
\end{ekey}

Note that here we are giving $B$ a different interpretation than we did in the previous section. The symbolization key only specifies what $B$ means \emph{in a specific context}. It is vital that we continue to use this meaning of $B$ so long as we are talking about Mary and Barcelona. Later, when we are symbolizing different sentences, we can write a new symbolization key and use $B$ to mean something else.

\newglossaryentry{negation}
{
name=negation,
description={The symbol \enot, used to represent words and phrases that function like the English word ``not''.}
}

Now, sentence \ref{not1} is simply $B$. Sentence \ref{not2} is obviously related to sentence \ref{not1}: it is basically \ref{not1} with a ``not'' added. We could put the sentence partly our symbolic language by writing ``Not $B$.'' This means we do not want to introduce a different sentence letter for \ref{not2}. We just need a new symbol for the ``not'' part. Let's use the symbol `\enot,' which we will call \textsc{\gls{negation}}. \label{def:negation} Now we can translate `Not $B$' to $\enot B$. 

Sentence \ref{not3} is about whether or not Mary is in Barcelona, but it does not contain the word ``not.'' Nevertheless, it is obviously logically equivalent to sentence \ref{not2}. They both mean: It is not the case that Mary is in Barcelona. As such, we can translate both sentence \ref{not2} and sentence \ref{not3} as $\enot B$.

\factoidbox{
A sentence can be symbolized as $\enot\script{A}$ if it can be paraphrased in English as ``It is not the case that \script{A}.''
}


Consider these further examples:
\begin{earg}
\item[\ex{not4}] The widget can be replaced if it breaks.
\item[\ex{not5}] The widget is irreplaceable.
\item[\ex{not5b}] The widget is not irreplaceable.
\end{earg}


If we let $R$ mean ``The widget is replaceable'', then sentence \ref{not4} can be translated as $R$.

What about sentence \ref{not5}? Saying the widget is irreplaceable means that it is not the case that the widget is replaceable. So even though sentence \ref{not5} is not negative in English, we symbolize it using negation as $\enot{R}$.

Sentence \ref{not5b} can be paraphrased as ``It is not the case that the widget is irreplaceable.'' Using negation twice, we translate this as $\enot\enot R$. The two negations in a row each work as negations, so the sentence means ``It is not the case that it is not the case that $R$.'' If you think about the sentence in English, it is logically equivalent to sentence \ref{not4}. So when we define logical equivalence in SL, we will make sure that $R$ and $\enot\enot R$ are logically equivalent.

More examples:
\begin{earg}
\item[\ex{not6}] Elliott is happy.
\item[\ex{not7}] Elliott is unhappy.
\end{earg}


If we let $H$ mean ``Elliot is happy'', then we can symbolize sentence \ref{not6} as $H$.

However, it would be a mistake to symbolize sentence \ref{not7} as $\enot{H}$. If Elliott is unhappy, then he is not happy---but sentence \ref{not7} does not mean the same thing as ``It is not the case that Elliott is happy.'' It could be that he is not happy but that he is not unhappy either. Perhaps he is somewhere between the two. In order to symbolize sentence \ref{not7}, we would need a new sentence letter.

For any sentence \script{A}: If \script{A} is true, then \enot\script{A} is false. If \enot\script{A} is true, then \script{A} is false. Using T for true and F for false, we can summarize this in a \emph{characteristic truth table} for negation:
\begin{center}
\begin{tabular}{c|c}
\script{A} & \enot\script{A}\\
\hline
T & F\\
F & T 
\end{tabular}
\end{center}
We will discuss truth tables at greater length in the next chapter.

%%%%%%%%%%%%%%%%%% 2.2.2 Conjunction

\subsection{Conjunction}
Consider these sentences:
\begin{earg}
\item[\ex{and1}]Adam is athletic.
\item[\ex{and2}]Barbara is athletic.
\item[\ex{and3}]Adam is athletic, and Barbara is also athletic.
\end{earg}

We will need separate sentence letters for \ref{and1} and \ref{and2}, so we define this symbolization key:
\begin{ekey}
\item[A:] Adam is athletic.
\item[B:] Barbara is athletic.
\end{ekey}

Sentence \ref{and1} can be symbolized as $A$.

Sentence \ref{and2} can be symbolized as $B$.

\newglossaryentry{conjunction}
{
name=conjunction,
description={The symbol \eand, used to represent words and phrases that function like the English word ``and.''}
}

\newglossaryentry{conjunct}
{
name=conjunct,
description={A sentences joined to another by a conjunction.}
}

Sentence \ref{and3} can be paraphrased as ``$A$ and $B$.'' In order to fully symbolize this sentence, we need another symbol. We will use \eand. We translate ``$A$ and $B$'' as $A\eand B$. The logical connective \eand is called the \textsc{\gls{conjunction}}, \label{def:conjunction} and $A$ and $B$ are each called \textsc{\glspl{conjunct}}. \label{def:conjunct}
 

Notice that we make no attempt to symbolize ``also'' in sentence \ref{and3}. Words like ``both'' and ``also'' function to draw our attention to the fact that two things are being conjoined. They are not doing any further logical work, so we do not need to represent them in SL.

Some more examples:
\begin{earg}
\item[\ex{and4}]Barbara is athletic and energetic.
\item[\ex{and5}]Barbara and Adam are both athletic.
\item[\ex{and6}]Although Barbara is energetic, she is not athletic.
\item[\ex{and7}]Barbara is athletic, but Adam is more athletic than she is.
\end{earg}

Sentence \ref{and4} is obviously a conjunction. The sentence says two things about Barbara, so in English it is permissible to refer to Barbara only once. It might be tempting to try this when translating the argument: Since $B$ means ``Barbara is athletic'', one might paraphrase the sentences as ``$B$ and energetic.'' This would be a mistake. Once we translate part of a sentence as $B$, any further structure is lost. $B$ is an atomic sentence; it is nothing more than true or false. Conversely, ``energetic'' is not a sentence; on its own it is neither true nor false. We should instead paraphrase the sentence as ``$B$ and Barbara is energetic.'' Now we need to add a sentence letter to the symbolization key. Let $E$ mean ``Barbara is energetic.'' Now the sentence can be translated as $B \eand E$.

\factoidbox{
A sentence can be symbolized as $\script{A} \eand \script{B}$ if it can be paraphrased in English as `Both \script{A}, and \script{B}.' Each of the conjuncts must be a sentence.
}

Sentence \ref{and5} says one thing about two different subjects. It says of both Barbara and Adam that they are athletic, and in English we use the word ``athletic'' only once. In translating to SL, it is important to realize that the sentence can be paraphrased as, ``Barbara is athletic, and Adam is athletic.'' This translates as $B \eand A$.

Sentence \ref{and6} is a bit more complicated. The word ``although'' sets up a contrast between the first part of the sentence and the second part. Nevertheless, the sentence says both that Barbara is energetic and that she is not athletic. In order to make each of the conjuncts an atomic sentence, we need to replace ``she'' with ``Barbara.''

So we can paraphrase sentence \ref{and6} as, ``\emph{Both} Barbara is energetic, \emph{and} Barbara is not athletic.'' The second conjunct contains a negation, so we paraphrase further: ``\emph{Both} Barbara is energetic \emph{and} \emph{it is not the case that} Barbara is athletic.'' This translates as $E \eand \enot B$.

Sentence \ref{and7} contains a similar contrastive structure. It is irrelevant for the purpose of translating to SL, so we can paraphrase the sentence as ``\emph{Both} Barbara is athletic, \emph{and} Adam is more athletic than Barbara.'' (Notice that we once again replace the pronoun ``she'' with her name.) How should we translate the second conjunct? We already have the sentence letter $A$ which is about Adam's being athletic and $B$ which is about Barbara's being athletic, but neither is about one of them being more athletic than the other. We need a new sentence letter. Let $R$ mean ``Adam is more athletic than Barbara.'' Now the sentence translates as $B \eand R$.

\factoidbox{Sentences that can be paraphrased ``\script{A}, but \script{B}'' or ``Although \script{A}, \script{B}'' are best symbolized using conjunction  \script{A} \eand \script{B}.}

It is important to keep in mind that the sentence letters $A$, $B$, and $R$ are atomic sentences. Considered as symbols of SL, they have no meaning beyond being true or false. We have used them to symbolize different English language sentences that are all about people being athletic, but this similarity is completely lost when we translate to SL. No formal language can capture all the structure of the English language, but as long as this structure is not important to the argument there is nothing lost by leaving it out.

For any sentences \script{A} and \script{B}, \script{A} \eand \script{B} is true if and only if both \script{A} and \script{B} are true. We can summarize this in the {characteristic truth table} for conjunction:
\begin{center}
\begin{tabular}{c|c|c}
\script{A} & \script{B} & \script{A} \eand \script{B}\\
\hline
T & T & T\\
T & F & F\\
F & T & F\\
F & F & F
\end{tabular}
\end{center}

Conjunction is symmetrical because we can swap the conjuncts without changing the truth value of the sentence. Regardless of what \script{A} and \script{B} are, \script{A}\eand\script{B} is logically equivalent to \script{B} \eand \script{A}.


%%%%%%%%%%%%%%%%%%%% 2.2.3 disjunction

\subsection{Disjunction}
Consider these sentences:
\begin{earg}
\item[\ex{or1}]Either Denison will play golf with me, or he will watch movies.
\item[\ex{or2}]Either Denison or Ellery will play golf with me. 
\end{earg}

For these sentences we can use this symbolization key:

\begin{ekey}
\item[D:] Denison will play golf with me.
\item[E:] Ellery will play golf with me.
\item[M:] Denison will watch movies.
\end{ekey}

\newglossaryentry{disjunction}
{
name=disjunction,
description={The symbol \eor, used to represent words and phrases that function like the English word ``or'' in its inclusive sense.}
}

\newglossaryentry{disjunct}
{
name=disjunct,
description={A sentences joined to another by a disjunction.}
}



Sentence \ref{or1} is ``Either $D$ or $M$.'' To fully symbolize this, we introduce a new symbol. The sentence becomes $D \eor M$. The \eor connective is called \textsc{\gls{disjunction}}, \label{def:disjunction} and $D$ and $M$ are called \textsc{\glspl{disjunct}}. \label{def:disjunct}

Sentence \ref{or2} is only slightly more complicated. There are two subjects, but the English sentence only gives the verb once. In translating, we can paraphrase it as ``Either Denison will play golf with me, or Ellery will play golf with me.'' Now it obviously translates as $D \eor E$.


\factoidbox{
A sentence can be symbolized as $\script{A}\eor\script{B}$ if it can be paraphrased in English as ``Either \script{A} or \script{B}.'' Each of the disjuncts must be a sentence.
}

Sometimes in English, the word ``or'' excludes the possibility that both disjuncts are true. This is called an \define{exclusive or}.  An \emph{exclusive or} is clearly intended when a restaurant menu says, ``Entrees come with either soup or salad.'' You may have soup; you may have salad; but, if you want \emph{both} soup \emph{and} salad, then you will have to pay extra.

At other times, the word ``or'' allows for the possibility that both disjuncts might be true. This is probably the case with sentence \ref{or2}, above. I might play with Denison, with Ellery, or with both Denison and Ellery. Sentence \ref{or2} merely says that I will play with \emph{at least} one of them. This is called an \define{inclusive or}.

The symbol $\eor$ represents an \emph{inclusive or}. So $D \eor E$ is true if $D$ is true, if $E$ is true, or if both $D$ and $E$ are true. It is false only if both $D$ and $E$ are false. We can summarize this with the {characteristic truth table} for disjunction:

\begin{center}
\begin{tabular}{c|c|c}
\script{A} & \script{B} & \script{A} \eor \script{B} \\
\hline
T & T & T\\
T & F & T\\
F & T & T\\
F & F & F
\end{tabular}
\end{center}

Like conjunction, disjunction is symmetrical. \script{A} \eor \script{B} is logically equivalent to \script{B} \eor \script{A}.


These sentences are somewhat more complicated:

\begin{earg}
\item[\ex{or3}] Either you will not have soup, or you will not have salad.
\item[\ex{or4}] You will have neither soup nor salad.
\item[\ex{or.xor}] You get either soup or salad, but not both.
\end{earg}

We let $S_1$ mean that you get soup and $S_2$ mean that you get salad.

Sentence \ref{or3} can be paraphrased in this way: ``Either \emph{it is not the case that} you get soup, or \emph{it is not the case that} you get salad.'' Translating this requires both disjunction and negation. It becomes $\enot S_1 \eor \enot S_2$.

Sentence \ref{or4} also requires negation. It can be paraphrased as, ``\emph{It is not the case that} either you get soup or you get salad.'' We need some way of indicating that the negation does not just negate the right or left disjunct, but rather negates the entire disjunction. In order to do this, we put parentheses around the disjunction: ``It is not the case that $(S_1 \eor S_2)$.'' This becomes simply $\enot (S_1 \eor S_2)$.

Notice that the parentheses are doing important work here. The sentence $\enot S_1 \eor S_2$ would mean ``Either you will not have soup, or you will have salad.''

Sentence \ref{or.xor} is an \emph{exclusive or}. Although $\eor$ is an inclusive or, we can symbolize an exclusive or in {SL}. We just need more than one connective to do it. We can break the sentence into two parts. The first part says that you get one or the other. We translate this as $(S_1 \eor S_2)$. The second part says that you do not get both. We can paraphrase this as ``It is not the case both that you get soup and that you get salad.'' Using both negation and conjunction, we translate this as $\enot(S_1 \eand S_2)$. Now we just need to put the two parts together. As we saw above, ``but'' can usually be translated as a conjunction. Sentence \ref{or.xor} can thus be translated as $(S_1 \eor S_2) \eand \enot(S_1 \eand S_2)$.

%%%%%%%%%%%%%%%%% 2.2.4 conditional

\subsection{Conditional}
For the following sentences, let $R$ mean ``You will cut the red wire'' and $B$ mean ``The bomb will explode.''

\begin{earg}
\item[\ex{if1}] If you cut the red wire, then the bomb will explode.
\item[\ex{if2}] The bomb will explode only if you cut the red wire.
\end{earg}

\newglossaryentry{conditional}
{
name=conditional,
description={The symbol \eif, used to represent words and phrases that function like the English phrase ``if \ldots then.''}
}

\newglossaryentry{antecedent}
{
name=antecedent,
description={The sentence to the left of a conditional..}
}


\newglossaryentry{consequent}
{
name=consequent,
description={The sentence to the right of a conditional.}
}


Sentence \ref{if1} can be translated partially as ``If $R$, then $B$.'' We will use the symbol $\eif$ to represent logical {entailment}. Sentence \ref{if1} then becomes $R\eif B$. The connective is called a \textsc{\gls{conditional}}. \label{def:conditional} The sentence on the left-hand side of the conditional ($R$ in this example) is called the \textsc{\gls{antecedent}}. \label{def:antecedent}
 The sentence on the right-hand side ($B$) is called the \textsc{\gls{consequent}}. \label{def:consequent} 

Sentence \ref{if2} is also a conditional. Since the word ``if'' appears in the second half of the sentence, it might be tempting to symbolize this in the same way as sentence \ref{if1}. That would be a mistake.

The conditional $R\eif B$ says that \emph{if} $R$ were true, \emph{then} $B$ would also be true. It does not say that you cutting the red wire is the \emph{only} way that the bomb could explode. Someone else might cut the wire, or the bomb might be on a timer. The sentence $R\eif B$ does not say anything about 
what to expect if $R$ is false. Sentence \ref{if2} is different. It says that the only conditions under which the bomb will explode involve you having cut the red wire; i.e., if the bomb explodes, then you must have cut the wire. As such, sentence \ref{if2} should be symbolized as $B \eif R$.

It is important to remember that the connective $\eif$ says only that, if the antecedent is true, then the consequent is true. It says nothing about the \emph{causal} connection between the two events. Translating sentence \ref{if2} as $B \eif R$ does not mean that the bomb exploding would somehow have caused you cutting the wire. Both sentence \ref{if1} and \ref{if2} suggest that, if you cut the red wire, you cutting the red wire would be the cause of the bomb exploding. They differ on the \emph{logical} connection. If sentence \ref{if2} were true, then an explosion would tell us---those of us safely away from the bomb---that you had cut the red wire. Without an explosion, sentence \ref{if2} tells us nothing.

\factoidbox{
The paraphrased sentence ``\script{A} only if \script{B}'' is logically equivalent to ``If \script{A}, then \script{B}.''
}

% Could discuss necessary and sufficient conditions here.

``If \script{A}, then \script{B}'' means that if \script{A} is true, then so is \script{B}. So we know that if the antecedent \script{A} is true but the consequent \script{B} is false, then the conditional ``If \script{A} then \script{B}'' is false. What is the truth value of ``If \script{A}, then \script{B}'' under other circumstances? Suppose, for instance, that the antecedent \script{A} happened to be false. ``If \script{A}, then \script{B}'' would then not tell us anything about the actual truth value of the consequent \script{B}, and it is unclear what the truth value of ``If \script{A}, then \script{B}'' would be.

In English, the truth of conditionals often depends on what \emph{would} be the case if the antecedent \emph{were true}---even if, as a matter of fact, the antecedent is false. This poses a problem for translating conditionals into SL.  Considered as sentences of SL, $R$ and $B$ in the above examples have nothing intrinsic to do with each other. In order to consider what the world would be like if $R$ were true, we would need to analyze what $R$ says about the world. Since $R$ is an atomic symbol of SL, however, there is no further structure to be analyzed. When we replace a sentence with a sentence letter, we consider it merely as some atomic sentence that might be true or false.

In order to translate conditionals into SL, we will not try to capture all the subtleties of the English language ``If$\ldots$, then$\ldots$.'' Instead, the symbol $\eif$ will be what logicians call a material conditional. This means that when \script{A} is false, the conditional \script{A} \eif \script{B} is automatically true, regardless of the truth value of \script{B}. If both \script{A} and \script{B} are true, then the conditional \script{A} \eif \script{B} is true.


In short, \script{A} \eif \script{B} is false if and only if \script{A} is true and \script{B} is false. We can summarize this with a characteristic truth table for the conditional.

\begin{center}
\begin{tabular}{c|c|c}
\script{A} & \script{B} & \script{A}\eif\script{B}\\
\hline
T & T & T\\
T & F & F\\
F & T & T\\
F & F & T
\end{tabular}
\end{center}

The conditional is asymmetrical. You cannot swap the antecedent and consequent without changing the meaning of the sentence, because \script{A} \eif \script{B} and \script{B} \eif \script{A} are not logically equivalent.

%\begin{earg}
%\item[\ex{if3}] Everytime a bell rings, an angel earns its wings.
%\item[\ex{if4}] Bombs always explode when you cut the red wire.
%\end{earg}

Not all sentences of the form ``If$\ldots$, then$\ldots$'' are conditionals. Consider this sentence:

\begin{earg}
\item[\ex{if5}] If anyone wants to see me, then I will be on the porch.
\end{earg}

When I say this, it means that I will be on the porch, regardless of whether anyone wants to see me or not---but if someone did want to see me, then they should look for me there. If we let $P$ mean ``I will be on the porch,'' then sentence \ref{if5} can be translated simply as $P$.

%%%%%%%%%%%%%%%% 6.2.5 Biconditional

\subsection{Biconditional}
Consider these sentences:
\begin{earg}
\item[\ex{iff1}] The figure on the board is a triangle only if it has exactly three sides.
\item[\ex{iff2}] The figure on the board is a triangle if it has exactly three sides.
\item[\ex{iff3}] The figure on the board is a triangle if and only if it has exactly three sides.
\end{earg}

Let $T$ mean ``The figure is a triangle'' and $S$ mean ``The figure has three sides.''

Sentence \ref{iff1}, for reasons discussed above, can be translated as $T\eif S$.

Sentence \ref{iff2} is importantly different. It can be paraphrased as ``If the figure has three sides, then it is a triangle.'' So it can be translated as $S\eif T$.

\newglossaryentry{biconditional}
{
name=biconditional,
description={The symbol \eiff, used to represent words and phrases that function like the English phrase ``if and only if.''}
}


Sentence \ref{iff3} says that $T$ is true \emph{if and only if} $S$ is true; we can infer $S$ from $T$, and we can infer $T$ from $S$. This is called a \textsc{\gls{biconditional}}, \label{def:biconditional} because it entails the two conditionals $S\eif T$ and $T \eif S$. We will use \eiff  to represent the biconditional; sentence \ref{iff3} can be translated as $S \eiff T$.

We could abide without a new symbol for the biconditional. Since sentence \ref{iff3} means ``$T \eif S$ and $S\eif T$,'' we could translate it as $(T \eif S)\eand(S\eif T)$. We would need parentheses to indicate that $(T \eif S)$ and $(S\eif T)$ are separate conjuncts; the expression $T \eif S\eand S\eif T$ would be ambiguous.

Because we could always write $(\script{A}\eif\script{B})\eand(\script{B}\eif\script{A})$ instead of $\script{A}\eiff\script{B}$, we do not strictly speaking \emph{need} to introduce a new symbol for the biconditional. Nevertheless, logical languages usually have such a symbol. SL will have one, which makes it easier to translate phrases like ``if and only if.''

\script{A} \eiff \script{B} is true if and only if \script{A} and \script{B} have the same truth value. \label{defBiconditional}This is the characteristic truth table for the biconditional:

\begin{center}
\begin{tabular}{c|c|c}
\script{A} & \script{B} & \script{A} \eiff \script{B}\\
\hline
T & T & T\\
T & F & F\\
F & T & F\\
F & F & T
\end{tabular}
\end{center}

% ******************************************
%  * 		Other Symbolization	                          *
% ******************************************

\section{Other Symbolization}
We have now introduced all of the connectives of SL. We can use them together to translate many kinds of sentences. Consider these examples of sentences that use the English-language connective ``unless'':

\begin{earg}
\item[\ex{unless1}] Unless you wear a jacket, you will catch cold. 
\item[\ex{unless2}] You will catch cold unless you wear a jacket. 
\end{earg}

Let $J$ mean ``You will wear a jacket'' and let $D$ mean ``You will catch a cold.''

We can paraphrase sentence \ref{unless1} as ``Unless $J$, $D$.'' This means that if you do not wear a jacket, then you will catch cold; with this in mind, we might translate it as $\enot J \eif D$. It also means that if you do not catch a cold, then you must have worn a jacket; with this in mind, we might translate it as $\enot D \eif J$.

Which of these is the correct translation of sentence \ref{unless1}? Both translations are correct, because the two translations are logically equivalent in SL.

Sentence \ref{unless2}, in English, is logically equivalent to sentence \ref{unless1}. It can be translated as either $\enot J \eif D$ or $\enot D \eif J$.

When symbolizing sentences like sentence \ref{unless1} and sentence \ref{unless2}, it is easy to get turned around. Since the conditional is not symmetric, it would be wrong to translate either sentence as $J \eif \enot D$. Fortunately, there are other logically equivalent expressions. Both sentences mean that you will wear a jacket or---if you do not wear a jacket---then you will catch a cold. So we can translate them as $J \eor D$. (You might worry that the ``or'' here should be an \emph{exclusive or}. However, the sentences do not exclude the possibility that you might \emph{both} wear a jacket \emph{and} catch a cold; jackets do not protect you from all the possible ways that you might catch a cold.)


\factoidbox{
If a sentence can be paraphrased as ``Unless \script{A}, \script{B},'' then it can be symbolized as $\script{A}\eor\script{B}$.
}




% ******************************************
%  * 		Recursive Syntax for  SL                    *
% ******************************************


\section{Recursive Syntax for SL} % I reworked this section to focus on the idea of recursive syntax

The previous two sections gave you a rough, informal sense of how to create sentences in SL. If I give you an English sentence like ``Grass is either green or brown,'' you should be able to write a corresponding sentence in SL: ``$A \eor B$.'' In this section we want to give a more precise definition of a sentence in SL.  When we defined sentences in English, we did so using the concept of truth: Sentences were units of language that can be true or false. In SL, it is possible to define what counts as a sentence without talking about truth. Instead, we can just talk about the structure of the sentence. This is one respect in which a formal language like SL is more precise than a natural language like English.

\newglossaryentry{syntax}
{
name=syntax,
description={The structure of a bit of language, considered without reference to truth, falsity, or meaning.}
}

\newglossaryentry{semantics}
{
name=semantics,
description={The meaning of a bit of language is its meaning, including truth and falsity.}
}

The structure of a sentence in SL considered without reference to truth or falsity is called its syntax. More generally \textsc{\gls{syntax}} \label{def:syntax} refers to the study of the properties of language that are there even when you don't consider meaning. Whether a sentence is true or false is considered part of its meaning. In this chapter, we will be giving a purely syntactical definition of a sentence in SL.  The contrasting term is \textsc{\gls{semantics}} \label{def:semantics} the study of aspects of language that relate to meaning, including truth and falsity. (The word ``semantics'' comes from the Greek word for ``mark'')

\newglossaryentry{object language}
{
name=object language,
description={A language that is constructed and studied by logicians. In this textbook, the object languages are SL and QL.}
}

\newglossaryentry{metalanguage}
{
name=metalanguage,
description={The language logicians use to talk about the object language. In this textbook, the metalanguage is English, supplemented by certain symbols like metavariables and technical terms like ``valid.''}
}

If we are going to define a sentence in SL just using syntax, we will need to carefully distinguish SL from the language that we use to talk about SL. When you create an artificial language like SL, the language that you are creating is called the \textsc{\gls{object language}}. \label{def:object_language} The language that we use to talk about the object language is called the \textsc{\gls{metalanguage}}. \label{def:metalanguage} Imagine building a house. The object language is like the house itself. It is the thing we are building. While you are building a house, you might put up scaffolding around it. The scaffolding isn't part of the the house. You just use it to build the house. The metalanguage is like the scaffolding. 

The object language in this chapter is SL. For the most part, we can build this language just by talking about it in ordinary English. However we will also have to build some special scaffolding that is not a part of SL, but will help us build SL. Our metalanguage will thus be ordinary English plus this scaffolding.

\newglossaryentry{metavariables}
{
name=metavariables,
description={A variable in the metalanguage that can represent any sentence in the object language.}
}



%rob: Paragraph on metavariables added.
An important part of the scaffolding are the \textsc{\gls{metavariables}} \label{def:metavariables} These are the fancy script letters we have been using in the characteristic truth tables for the connectives: \script{A}, \script{B}, \script{C}, etc. These are letters that can refer to any sentence in SL. They can represent sentences like $P$ or $Q$, or they can represent longer sentences, like $(((A \eor B) \eand G) \eif (P \eiff Q))$. Just as the sentence letters $A$, $B$, etc. are variables that range over any English sentence, the metavariables \script{A}, \script{B}, etc. are variables that range over any sentence in SL, including the sentence letters $A$, $B$, etc. 

As we said, in this chapter we will give a syntactic definition for ``sentence of SL.'' The definition itself will be given in mathematical English, the metalanguage.

\begin{center}
\begin{tabular}{|c|c|}
\hline
sentence letters & $A,B,C,\ldots,Z$\\
with subscripts, as needed & $A_1, B_1,Z_1,A_2,A_{25},J_{375},\ldots$\\
\hline
connectives & \enot,\eand,\eor,\eif,\eiff\\
\hline
parentheses&( , )\\
\hline
\end{tabular}
\end{center}

Most random combinations of these symbols will not count as sentences in SL. Any random connection of these symbols will just be called a ``string'' or ``expression'' Random strings only become meaningful sentences when the are structured according to the rules of syntax. We saw from the earlier two sections that individual sentence letters,  like $A$ and $G_{13}$ counted as sentences. We also saw that we can put these sentences together using connectives so that  $\enot A$ and $\enot G_{13}$ is a sentence.  The problem is, we can't simply list all the different sentences we can put together this way, because there are infinitely many of them. Instead, we will define a sentence in SL by specifying the process by which they are constructed.

Consider negation: Given any sentence \script{A} of SL, $\enot\script{A}$ is a sentence of SL. It is important here that \script{A} is not the sentence letter $A$. Rather, it is a metavariable: part of the metalanguage, not the object language. Since \script{A} is not a symbol of SL, $\enot\script{A}$ is not an expression of SL. Instead, it is an expression of the metalanguage that allows us to talk about infinitely many expressions of SL: all of the expressions that start with the negation symbol. 


\newglossaryentry{sentence of SL}
{
name=sentence of SL,
description={A string of symbols in SL that can be built up using according to the recursive rules given on page }
}



We can say similar things for each of the other connectives. For instance, if \script{A} and \script{B} are sentences of SL, then $(\script{A}\eand\script{B})$ is a sentence of SL. Providing clauses like this for all of the connectives, we arrive at the following formal definition for a \textsc{\gls{sentence of SL}}: \label{def:sentence_of_SL}

\begin{enumerate}
\item Every atomic sentence is a sentence.
\item If \script{A} is a sentence, then $\enot\script{A}$ is a sentence of SL.
\item If \script{A} and \script{B} are sentences, then $(\script{A}\eand\script{B})$ is a sentence.
\item If \script{A} and \script{B} are sentences, then $(\script{A}\eor\script{B})$ is a sentence.
\item If \script{A} and \script{B} are sentences, then $(\script{A}\eif\script{B})$ is a sentence.
\item If \script{A} and \script{B} are sentences, then $(\script{A}\eiff\script{B})$ is a sentence.
\item All and only sentences of SL can be generated by applications of these rules.
\end{enumerate}

We can apply this definition to see whether an arbitrary string is a sentence. Suppose we want to know whether or not $\enot \enot \enot D$ is a sentence of SL. Looking at the second clause of the definition, we know that $\enot \enot \enot D$ is a sentence \emph{if} $\enot \enot D$ is a sentence. So now we need to ask whether or not $\enot \enot D$ is a sentence. Again looking at the second clause of the definition, $\enot \enot D$ is a sentence \emph{if} $\enot D$ is. Again, $\enot D$ is a sentence \emph{if} $D$ is a sentence. Now $D$ is a sentence letter, an atomic sentence of SL, so we know that $D$ is a sentence by the first clause of the definition. So for a compound formula like $\enot \enot \enot D$, we must apply the definition repeatedly. Eventually we arrive at the atomic sentences from which the sentence is built up.

\newglossaryentry{recursive definition}
{
name=recursive definition,
description={A definition that defines a term by identifying base class and rules for extending that class.}
}

Definitions like this are called recursive. \textsc{\Glspl{recursive definition}} \label{def:recursive_definition} begin with some specifiable base elements and define ways to indefinitely compound the base elements. Just as the recursive definition allows complex sentences to be built up from simple parts, you can use it to decompose sentences into their simpler parts. To determine whether or not something meets the definition, you may have to refer back to the definition many times.

\newglossaryentry{scope}
{
name=scope,
description={The sentences that are joined by a connective. These are the sentences the connective was applied to when the sentence was assembled using a recursive definition.}
}

When you use a connective to build a longer sentence from shorter ones, the shorter sentences are said to be in the \textsc{\gls{scope}} \label{def:scope} of the connective. So in the sentence $(A \eand B) \eif C$, the scope of the connective $\eif$ includes $(A \eand B)$ and C. In the sentence $\enot(A \eand B)$ the scope of the $\enot$ is $(A \eand B)$. On the other hand, in the sentence $\enot A \eand B$ the scope of the $\enot$ is just A.

\newglossaryentry{main connective}
{
name=main connective,
description={The last connective that you add when you assemble a sentence using the recursive definition.}
}

The last connective that you add when you assemble a sentence using the recursive definition is the \textsc{\gls{main connective}} \label{def:main_connective} of that sentence. For example: The main logical operator of $\enot (E \eor (F \eif G))$ is negation, \enot. The main logical operator of $(\enot E \eor (F \eif G))$ is disjunction, \eor. The main connective of any sentence will have all the rest of the sentence in its scope.

%The recursive structure of sentences in SL will be important when we consider the circumstances under which a particular sentence would be true or false. The sentence $\enot \enot \enot D$ is true if and only if the sentence $\enot \enot D$ is false, and so on through the structure of the sentence until we arrive at the atomic components: $\enot \enot \enot D$ is true if and only if the atomic sentence $D$ is false. We will return to this point in the next chapter.
%restore when you restore the recursive part of chap. 3.

\subsection{Notational conventions}
\label{SLconventions}
A sentence like $(Q \eand R)$ must be surrounded by parentheses, because we might apply the definition again to use this as part of a more complicated sentence. If we negate $(Q \eand R)$, we get $\enot(Q \eand R)$. If we just had $Q \eand R$ without the parentheses and put a negation in front of it, we would have $\enot Q \eand R$. It is most natural to read this as meaning the same thing as $(\enot Q \eand R)$, something very different than $\enot(Q\eand R)$. The sentence $\enot(Q \eand R)$ means that it is not the case that both $Q$ and $R$ are true; $Q$ might be false or $R$ might be false, but the sentence does not tell us which. The sentence $(\enot Q \eand R)$ means specifically that $Q$ is false and that $R$ is true. As such, parentheses are crucial to the meaning of the sentence.

So, strictly speaking, $Q \eand R$ without parentheses is \emph{not} a sentence of SL. When using SL, however, we will often be able to relax the precise definition so as to make things easier for ourselves. We will do this in several ways.

First,  we understand that $Q \eand R$ means the same thing as $(Q \eand R)$. As a matter of convention, we can leave off parentheses that occur \emph{around the entire sentence}.

Second, it can sometimes be confusing to look at long sentences with many nested pairs of parentheses. We adopt the convention of using square brackets [ and ] in place of parentheses. There is no logical difference between $(P\eor Q)$ and $[P\eor Q]$, for example. The unwieldy sentence
$$(((H \eif I) \eor (I \eif H)) \eand (J \eor K))$$
could be written in this way:
$$\bigl[(H \eif I) \eor (I \eif H)\bigr] \eand (J \eor K)$$


Third, we will sometimes want to translate the conjunction of three or more sentences. For the sentence ``Alice, Bob, and Candice all went to the party,'' suppose we let $A$ mean ``Alice went,'' $B$ mean ``Bob went,'' and $C$ mean ``Candice went.'' The definition only allows us to form a conjunction out of two sentences, so we can translate it as $(A \eand B) \eand C$ or as $A \eand (B \eand C)$. There is no reason to distinguish between these, since the two translations are logically equivalent. There is no logical difference between the first, in which $(A \eand B)$ is conjoined with $C$, and the second, in which $A$ is conjoined with $(B \eand C)$.  So we might as well just write $A \eand B \eand C$. As a matter of convention, we can leave out parentheses when we conjoin three or more sentences.

Fourth, a similar situation arises with multiple disjunctions. ``Either Alice, Bob, or Candice went to the party'' can be translated as $(A \eor B) \eor C$ or as $A \eor (B \eor C)$. Since these two translations are logically equivalent, we may write $A \eor B \eor C$.

These latter two conventions only apply to multiple conjunctions or multiple  disjunctions. If a series of connectives includes both disjunctions and conjunctions, then the parentheses are essential; as with $(A \eand B) \eor C$ and $A \eand (B \eor C)$. The parentheses are also required if there is a series of conditionals or biconditionals; as with $(A \eif B) \eif C$ and $A \eiff (B \eiff C)$.

We have adopted these four rules as notational conventions, not as changes to the definition of a sentence. Strictly speaking, $A \eor B \eor C$ is still not a sentence. Instead, it is a kind of shorthand. We write it for the sake of convenience, but we really mean the sentence $(A \eor (B \eor C))$.

If we had given a different definition for a sentence, then these could count as sentences. We might have written rule 3 in this way: ``If \script{A}, \script{B}, $\ldots$ \script{Z} are sentences, then $(\script{A}\eand\script{B}\eand\ldots\eand\script{Z})$, is a sentence .'' This would make it easier to translate some English sentences, but would have the cost of making our formal language more complicated. We would have to keep the complex definition in mind when we develop truth tables and a proof system. We want a logical language that is expressively simple and allows us to translate easily from English, but we also want a formally simple language. Adopting notational conventions is a compromise between these two desires.


\practiceproblems
\setlength{\parindent}{0em}

\problempart Using the symbolization key given, translate each English-language sentence into SL.
\label{pr.monkeysuits}
\begin{ekey}
\item[M:] Those creatures are men in suits. 
\item[C:] Those creatures are chimpanzees. 
\item[G:] Those creatures are gorillas.
\end{ekey}
\begin{earg}
\item Those creatures are not men in suits. % {\color{red}$\enot M$} \vspace{1ex}
\item Those creatures are men in suits, or they are not. % {\color{red}$M \eor \enot M$} \vspace{1ex}
\item Those creatures are either gorillas or chimpanzees. % {\color{red}$G \eor C$} \vspace{1ex}
\item Those creatures are not gorillas, but they are not chimpanzees either. % {\color{red}$\enot G \eand \enot C$} \vspace{1ex}
\item Those creatures cannot be both gorillas and men in suits. % {\color{red}$\enot(G \eand M)$} \vspace{1ex}
\item If those creatures are not gorillas, then they are men in suits % {\color{red} $\enot G \eif M$} \vspace{1ex}
\item Those creatures are men in suits only if they are not gorillas. % {\color{red} $M \eif \enot G$} \vspace{1ex}
\item Those creatures are chimpanzees if and only if they are not gorillas. % {\color{red} $C \eiff G$} \vspace{1ex}
\item Those creatures are neither gorillas nor chimpanzees. % {\color{red} $~(G \eor C).$ See p.34, sentence 19, and p. 156} \vspace{1ex}
\item Unless those creatures are men in suits, they are either chimpanzees or they are gorillas. % {\color{red}$M \eor (C \eor G)$} \vspace{1ex}
\end{earg}

%If					X
%only if				X
%if and only if		X
%but				X
%unless				X
%not both      		X
%neither nor			X

%added and changed problems to get a better distribution of kinds of problems. 

\problempart Using the symbolization key given, translate each English-language sentence into SL.
\begin{ekey}
\item[A:] Mister Ace was murdered.
\item[B:] The butler did it.
\item[C:] The cook did it.
\item[D:] The Duchess is lying.
\item[E:] Mister Edge was murdered.
\item[F:] The murder weapon was a frying pan.
\end{ekey}
\begin{earg}
\item Either Mister Ace or Mister Edge was murdered. % {\color{red} $A \eor E$}  \vspace{1ex}
\item If Mister Ace was murdered, then the cook did it. % {\color{red} $A \eif C$} \vspace{1ex}
\item If Mister Edge was murdered, then the cook did not do it. % {\color{red} $E \eif \enot C} \vspace{1ex}
\item Either the butler did it, or the Duchess is lying. % {\color{red} $B \eor D$} \vspace{1ex}
\item The cook did it only if the Duchess is lying. % {\color{red} $C \eif D$} \vspace{1ex}
\item If the murder weapon was a frying pan, then the culprit must have been the cook. % {\color{red} $F \eif C$} \vspace{1ex}
\item If the murder weapon was not a frying pan, then the culprit was neither the cook nor the butler. % {\color{red} $\enot F \eif \enot(C \or B) \vspace{1ex}
\item Mister Ace was murdered if and only if Mister Edge was not murdered. % {\color{red} $A \eiff \enot E$} \vspace{1ex}
\item The Duchess is lying, unless it was Mister Edge who was murdered. % {\color{red} $D \eor A$} \vspace{1ex}
\item Mister Ace was murdered, but not with a frying pan. % {\color{red} $A \eand \enot F$} \vspace{1ex}
\item The butler and the cook did not both do it. % {\color{red} $\enot(B \enad C)$} \vspace{1ex}
\item Of course the Duchess is lying! % {\color{red}$D$} \vspace{1ex}
\end{earg}

%If  			x
%only if             x
%if and only if   x
%but			x
%unless			x
%not both		x
%neither nor		x

%changed problems to get a better distribution of kinds of problems. 


\problempart Using the symbolization key given, translate each English-language sentence into SL.
\label{pr.avacareer}
\begin{ekey}
\item[E$_1$:] Ava is an electrician.
\item[E$_2$:] Harrison is an electrician.
\item[F$_1$:] Ava is a firefighter.
\item[F$_2$:] Harrison is a firefighter.
\item[S$_1$:] Ava is satisfied with her career.
\item[S$_2$:] Harrison is satisfied with his career.
\end{ekey}
\begin{earg}
\item Ava and Harrison are both electricians. %{\color{red} $E_1 \eand E_2$} \vspace{1ex}
\item If Ava is a firefighter, then she is satisfied with her career. %{\color{red} $F_1 \eif S_1$}  \vspace{1ex}
\item Ava is a firefighter, unless she is an electrician. %{\color{red} $F_1 \eor E_1$  \vspace{1ex}
\item Harrison is an unsatisfied electrician. %{\color{red} $E_2 \eand \enot S_2$}  \vspace{1ex}
\item Neither Ava nor Harrison is an electrician. %{\color{red} $\enot(E_1 \eor E_2)$}  \vspace{1ex}
\item Both Ava and Harrison are electricians, but neither of them find it satisfying. %{\color{red} $(E_1 \eand E_2) \eand \enot (S_1 \eor S_2)$} \vspace{1ex}
\item Harrison is satisfied only if he is a firefighter. %{\color{red} $S_2 \eif F_2$} \vspace{1ex}
\item If Ava is not an electrician, then neither is Harrison, but if she is, then he is too. %{\color{red} $(\enot E_1 \eif \enot E_2) \eand (E_1 \eif E_2)$} \vspace{1ex}
\item Ava is satisfied with her career if and only if Harrison is not satisfied with his. %{\color{red} $S_1 \eiff \enot S_2$} \vspace{1ex}
\item If Harrison is both an electrician and a firefighter, then he must be satisfied with his work. %{\color{red} $(E_2 \eand F_2) \eif S_2$} \vspace{1ex}
\item It cannot be that Harrison is both an electrician and a firefighter. %{\color{red} $\enot (E_2 \eand F_2)$} \vspace{1ex}
\item Harrison and Ava are both firefighters if and only if neither of them is an electrician. %{\color{red} $(F_1 \eand F_2) \eiff \enot (E_1 \eor E_2)$} \vspace{1ex}
\end{earg}

%If					x	
%only if				x
%if and only if		x
%but				x
%unless				x
%not both			x
%neither nor			x

\problempart Using the symbolization key given, translate each English-language sentence into SL.
\label{pr.jazzinstruments}
\begin{ekey}
\item[J$_1$:] John Coltrane played tenor sax.
\item[J$_2$:] John Coltrane played soprano sax.
\item[J$_3$:] John Coltrane played tuba
\item[M$_1$:] Miles Davis played trumpet
\item[M$_2$:]Miles Davis played tuba
\end{ekey}

\begin{earg}
\item John Coltrane played tenor and soprano sax. %{\color{red} $J_1 \eand J_2$} \vspace{1ex}
\item Neither Miles Davis nor John Coltrane played tuba. %{\color{red} $\enot(M_2 \eor J_3)$ or $\enot M_2 \eand \enot J_3$} \vspace{1ex}
\item John Coltrane did not play both tenor sax and tuba.  %{\color{red} $\enot(J_1 \eand J_3)$ or $\enot J_1 \eor \enotJ_3$} \vspace{1ex}
\item John Coltrane did not play tenor sax unless he also played soprano sax. %{\color{red} $\enot J_1 \eor J_2$} \vspace{1ex}
\item John Coltrane did not play tuba, but Miles Davis did. %{\color{red} $\enotJ_3 \eand M_2$} \vspace{1ex}
\item Miles Davis played trumpet only if he also played tuba. %{\color{red} $M_1 \eiff M_2$} \vspace{1ex}
\item If Miles Davis played trumpet, then John Coltrane played at least one of these three instruments: tenor sax, soprano sax, or tuba. %{\color{red} $M_1 \eif (J_1 \eor (J_2 \eor J_3))&} \vspace{1ex}
\item If John Coltrane played tuba then Miles Davis played neither trumpet nor tuba. %{\color{red} $J_3 \eif \enot(M_1 \eor M_2)$ or $J_3 \eif (\enot M_1 \eand \enot M_2)$  } \vspace{1ex}
\item Miles Davis and John Coltrane both played tuba if and only if Coltrane did not play tenor sax and Miles Davis did not play trumpet. %{\color{red} $(J_3 \eand M_2) \eiff \enotJ_1 & \enot M_1)$ or $(J_3 \eand M_2) \eiff \enot (J_1 \eor M_1)$} \vspace{1ex}
\end{earg}
%If					x					
%only if				x		
%if and only if		x
%but				x
%unless				x
%not both			x
%neither nor			x


\problempart
\label{pr.spies}
Give a symbolization key and symbolize the following sentences in SL. \\
%\begin{ekey}
%\item[A:] Alice is a spy
%\item [B:] Bob is a spy
%\item [C:] The code has been broken
%\item [D:] The German embassy is in an uproar
%\end{ekey}
\begin{earg}
\item Alice and Bob are both spies. % {\color{red}$A \eand B$ \vspace{1ex}}
\item If either Alice or Bob is a spy, then the code has been broken. %{\color{red} $(A \eor B) \eif C$ \vspace{1ex}}
\item If neither Alice nor Bob is a spy, then the code remains unbroken.%{\color{red}$\enot(A \eor B) \eif \enot C$ \vspace{1ex}}
\item The German embassy will be in an uproar, unless someone has broken the code.% {\color{red}$D \eor C$ \vspace{1ex}}
\item Either the code has been broken or it has not, but the German embassy will be in an uproar regardless. %{\color{red}$(C \eor \enot C) \eand D$ \vspace{1ex}}
\item Either Alice or Bob is a spy, but not both. %{\color{red}$(A \eor B) \eand \enot (A \eand B)$}
\end{earg}

%If
%only if
%if and only if
%but
%unless
%not both
%neither nor

\problempart Give a symbolization key and symbolize the following sentences in SL.
%\begin{ekey}
%\item[A:] Gregor plays first base
%\item[B:] The team will lose
%\item[C:] There is a miracle
%\item[D:] Gregor's mom will bake cookies.
%\end{ekey}

\begin{earg}
\item If Gregor plays first base, then the team will lose. %{\color{red} $A \eif B$ \vspace{1ex}}
\item The team will lose unless there is a miracle. %{\color{red}$B \eor C$ \vspace{1ex}}
\item The team will either lose or it won't, but Gregor will play first base regardless. % {\color{red}$(B \eor \enot B) \eand A$ \vspace{1ex}}
\item Gregor's mom will bake cookies if and only if Gregor plays first base.% {\color{red}$C \eiff A$ \vspace{1ex}}
\item If there is a miracle, then Gregor's mom will not bake cookies. %{\color{red} $C \eif \enot D$}
\end{earg}


\problempart
For each argument, write a symbolization key and translate the argument as well as possible into SL.
\begin{earg}
\item If Dorothy plays the piano in the morning, then Roger wakes up cranky. Dorothy plays piano in the morning unless she is distracted. So if Roger does not wake up cranky, then Dorothy must be distracted.
%{\color{red}
%\begin{ekey}
%\item[A:] Dorothy plays the piano in the morning
%\item[B:] Roger wakes up cranky
%\item[C:] Dorothy is distracted
%\end{ekey}

%begin{\earg}
%\item[1.] $A \ief B$
%\item[2.] $A \eor C$
%\item[$$\therefore$$] $\enot B \eif C$
%}
\item It will either rain or snow on Tuesday. If it rains, Neville will be sad. If it snows, Neville will be cold. Therefore, Neville will either be sad or cold on Tuesday.

%{\color{red}
%\begin{ekey}
%\item[A:]  It will rain on Tuesday
%\item[B:]  It will snow on Tuesday
%\item[C:]  Neville will be sad
%\item[D:]  Neville will be cold
%\end{ekey}

%\begin{earg}
%\item[1.]  $A \eor B$
%\item[2.]  $A \eif C$
%\item[3.] $B \eif D$
%\item[$$\therefore$$]  $C \eor D$
%\end{earg}
%}

\item If Zoog remembered to do his chores, then things are clean but not neat. If he forgot, then things are neat but not clean. Therefore, things are either neat or clean---but not both.
\end{earg}

%{\color{red}
%\begin{ekey}
%\item[A:] Zoog remembered to do his chores. 
%\item[B:] Things are clean 
%\item[C:] Things are neat %\end{ekey}

%\begin{earg}
%\item[1.]  $A \eif (B \eand \enot C)$
%\item[2.]  $\enot A \eif (\enot B \eand C)$
%\item[$\therefore$]  $(B \eor C) \eand \enot (B \eand C)$ 
%\end{earg}
%}

\problempart
For each argument, write a symbolization key and translate the argument as well as possible into SL. The part of the passage in italics is there to provide context for the argument, and doesn't need to be symbolized.
\begin{earg}
\item It is going to rain soon. I know because my leg is hurting, and my leg hurts if it’s going to rain. 

%{\color{red}
%\begin{ekey}
%\item[A:]  
%\item[B:]  
%\item[C:]  %\end{ekey}

%begin{\earg}
%\item[1.]  
%\item[2.]  
%\item[$\therefore$]  
%}

\item  \emph{Spider-man tries to figure out the bad guy’s plan.} If Doctor Octopus gets the uranium, he will blackmail the city. I am certain of this because if Doctor Octopus gets the uranium, he can make a dirty bomb, and if he can make a dirty bomb, he will blackmail the city.

%{\color{red}
%\begin{ekey}
%\item[A:]  
%\item[B:]  
%\item[C:]  %\end{ekey}

%begin{\earg}
%\item[1.]  
%\item[2.]  
%\item[$\therefore$]  
%}

\item \emph{A westerner tries to predict the policies of the Chinese government.} If the Chinese government cannot solve the water shortages in Beijing, they will have to move the capital. They don’t want to move the capital. Therefore they must solve the water shortage. But the only way to solve the water shortage is to divert almost all the water from the Yangzi river northward. Therefore the Chinese government will go with the project to divert water from the south to the north.       



%{\color{red}
%\begin{ekey}
%\item[A:]  
%\item[B:]  
%\item[C:]  %\end{ekey}

%begin{\earg}
%\item[1.]  
%\item[2.]  
%\item[$\therefore$]  
%}

\end{earg}




\problempart
\begin{earg}
\item Are there any sentences of SL that contain no sentence letters? Why or why not? % \vspace{1ex} \\ {\color{red} No, because the rules for creating sentences begin with sentence letters and then apply connectives and more sentence letters. There is no way to remove the sentence letters that you start with.} \vspace{2ex} 
\item In the chapter, we symbolized an \emph{exclusive or} using \eor, \eand, and \enot. How could you translate an \emph{exclusive or} using only two connectives? Is there any way to translate an \emph{exclusive or} using only one kind of connective? %\vspace{1ex} \\ {\color{red} The exclusive or (sometimes written xor) is true whenever the two sides of it have opposite truth values. This is the reverse of what the biconditional does. Thus you can represent the xor like this: \enot(A \eiff B). You can't get rid of any more connectives, though. If you had a single connective in the sentence, it would have to be equivalent on its own to the exclusive or, and none of our connectives work like that. Some systems do introduce a separate symbol for the exclusive or, often a plus sign: +.}
\end{earg}




%\solutions
%\problempart
%\label{pr.wiffSL}
%For each of the following: (a) Is it a wff of SL? (b) Is it a sentence of SL, allowing for notational conventions?
%\begin{earg}
%\item $(A)$
%\item $J_{374} \eor \enot J_{374}$
%\item $\enot \enot \enot \enot F$
%\item $\enot \eand S$
%\item $(G \eand \enot G)$
%\item $\script{A} \eif \script{A}$
%\item $(A \eif (A \eand \enot F)) \eor (D \eiff E)$
%\item $[(Z \eiff S) \eif W] \eand [J \eor X]$
%\item $(F \eiff \enot D \eif J) \eor (C \eand D)$
%\end{earg}


%%%%    Key term list
\section*{Key Terms}
\begin{multicols}{2}
\begin{sortedlist}
\sortitem{Sentence letter}{} 	
\sortitem{Symbolization key}{} 	
\sortitem{Atomic sentence}{}
\sortitem{Sentential connective}{}
\sortitem{Negation}{}
\sortitem{Conjunction}{}
\sortitem{Conjunct}{}
\sortitem{Disjunction}{}
\sortitem{Disjunct}{}
\sortitem{Conditional}{}
\sortitem{Antecedent}{}
\sortitem{Consequent}{}
\sortitem{Biconditional}{}
\sortitem{Syntax}{}
\sortitem{Semantics}{}
\sortitem{Object language}{}
\sortitem{Metalanguage}{}
\sortitem{Metavariables}{}
\sortitem{Sentence of SL}{}
\sortitem{Main connective}{}
\sortitem{Recursive definition}{}
\sortitem{Scope}{}
\end{sortedlist}
\end{multicols}





	
%\chapter{Truth Tables}
\label{chap:truth_tables}
\markright{Chap \ref{chap:truth_tables}: Truth Tables}

% The long comment below is material from the old semantics chapter that I am gradually folding in to this chapter.

This chapter introduces a way of evaluating sentences and arguments of SL called the truth table method. As we shall see, the truth table method is \emph{semantic} because it involves one aspect of the meaning of sentences, whether those sentences are true or false. As we saw on page \pageref{def:semantics}, semantics is the study of aspects of language related to meaning, including truth and falsity. Although it can be laborious, the truth table method is a purely mechanical procedure that requires no intuition or special insight. \iflabelexists{part:quant_logic}{When we get to Chapter \ref{chap:semantics_for_ql},we will provide a parallel semantic method for QL; however, this method will not be purely mechanical.}{}


% *********************************************
% *   Basic Concepts								*
% *********************************************

\section{Basic Concepts}

\newglossaryentry{logical constant}
{
name=logical constant,
description={A symbol whose meaning is fixed by a formal language. Sometimes these are just called ``logical symbols.'' They are contrasted with \textsc{non-logical symbols}.}
}

\newglossaryentry{nonlogical symbol}
{
name=nonlogical symbol,
description={A symbol whose meaning is not fixed by a formal language.}
}



In the previous chapter, we said that a formal language is built from two kinds of elements: logical constants and nonlogical symbols. The \textsc{\glspl{logical constant}}\label{def:logical_constant} have their meaning fixed by the formal language, while the \textsc{\glspl{nonlogical symbol}} \label{def:nonlogical_symbol} get their meaning in the symbolization key. The logical constants in SL are the sentential connectives and the parentheses, while the nonlogical symbols are the sentence letters. 

\newglossaryentry{interpretation}
{
name=interpretation,
description={A correspondence between nonlogical symbols of the object language and elements of some other language or logical structure.}
}

When we assign meaning to the nonlogical symbols of a language using a dictionary, we say we are giving an ``interpretation'' of the language. More formally an \textsc{\gls{interpretation}\label{def:interpretation}} of a language is a correspondence between elements of the object language and elements of some other language or logical structure. The symbolization keys we defined in Chapter \ref{chap:SL} (p. \pageref{def:translation_key}) are one sort of interpretation. Fancier languages will have more complicated kinds of interpretations.

\newglossaryentry{truth value}
{
  name=truth value,
  description={The status of a statement with relationship to truth. For  this textbook, this means the status of a statement as true or false}
}

The truth table method will also involve giving an interpretation of sentences, but they will be much simpler than the translation keys we used in Chapter \ref{chap:SL}. We will not be concerned with what the individual sentence letters mean. We will only care whether they are true or false. In other words, our interpretations will assign \glspl{truth value} to the sentence letters. (See page \pageref{def:Truth_value}.)

\newglossaryentry{truth-functional connective}
{
name=truth-functional connective,
description={an operator that builds larger sentences out of smaller ones and fixes the truth value of the resulting sentence based only on the truth value of the component sentences.}
}

We can get away with only worrying about the truth values of sentence letters because of the way that the meaning of larger sentences is generated by the meaning of their parts. Any larger sentence of SL is composed of atomic sentences with sentential connectives. The truth value of the compound sentence depends only on the truth value of the atomic sentences that it comprises. In order to know the truth value of $D\eiff E$, for instance, you only need to know the truth value of $D$ and the truth value of $E$. Connectives that work in this way are called truth functional. More technically, we define a \textsc{\gls{truth-functional connective}} \label{def:truth-functional_connective}as an operator that builds larger sentences out of smaller ones, and fixes the truth value of the resulting sentence based only on the truth value of the component sentences. 

\newglossaryentry{truth assignment}
{
name=truth assignment,
description={A function that maps the sentence letters in SL onto truth values.}
}

Because all of the logical symbols in SL are truth functional, we can study the the semantics of SL looking only at truth and falsity. If we want to know about the truth of the sentence $A \eand B$, the only thing we need to know is whether $A$ and $B$ are true. It doesn't actually matter what else they mean. So if $A$ is false, then $A \eand B$ is false no matter what false sentence $A$ is used to represent. It could be ``I am the Pope'' or ``Pi is equal to 3.19.'' The larger sentence $A \eand B$ is still false. So to give an interpretation of sentences in SL, all we need to do is create a truth assignment. A \textsc{\gls{truth assignment}} \label{def:truth_assignment} is a function that maps the sentence letters in SL onto our two truth values. In other words, we just need to assign Ts and Fs to all our sentence letters.

It is worth knowing that most languages are not built only out of truth functional connectives. In English, it is possible to form a new sentence from any simpler sentence \script{X} by saying ``It is possible that \script{X}.'' The truth value of this new sentence does not depend directly on the truth value of \script{X}. Even if \script{X} is false, perhaps in some sense \script{X} \emph{could} have been true---then the new sentence would be true. Some formal languages, called \emph{modal logics}, have an operator for possibility. In a modal logic, we could translate ``It is possible that \script{X}'' as {\large $\diamond$}\script{X}. However, the ability to translate sentences like these comes at a cost: The {\large $\diamond$} operator is not truth-functional, and so modal logics are not amenable to truth tables.

% *********************************************
% *   Complete Truth Tables							*
% *********************************************
\section{Complete Truth Tables}

In the last chapter we introduced the characteristic truth tables for the different connectives. To put them all in one place, the truth tables for the connectives of SL are repeated in Table \ref{table.CharacteristicTTs}. On the left is the truth table for negation, and on the right is the truth table for the other four connectives. Notice that the truth table for the negation is shorter than the other table. This is because there is only one metavariable here, \script{A}, which can either be true or false. The other connectives involve two metavariables, which give us four possibilities of true and false. The columns to the left of the double line in these tables are called the reference columns. They just specify the truth values of the individual sentence letters. Each row of the table assigns truth values to all the variables. Each row is thus a truth assignment---a kind of interpretation---for that sentence. Because the full table gives all the possible truth assignments for the sentence, it gives all the possible interpretations of it. 


\begin{table}
\begin{mdframed}[style=mytableclearbox]
\begin{center}
\begin{longtabu}{cccc|c||c|c|c|c}
\multicolumn{1}{r||}{\script{A}}&\enot\script{A}&	&	\script{A} & \script{B} & \script{A}\eand\script{B} & \script{A}\eor\script{B} & \script{A}\eif\script{B} & \script{A}\eiff\script{B}\\
\cline{1-2} \cline{4-9}
\multicolumn{1}{r||}{T}	&	F	&	&	T & T & T & T & T & T\\
\multicolumn{1}{r||}{F}	&	T	&	&	T & F & F & T & F & F\\
	&		&	&	F & T & F & T & T & F\\
	&		&	&	F & F & F & F & T & T
\end{longtabu}
\end{center}
\end{mdframed}
\caption{The characteristic truth tables for the connectives of SL.}
\label{table.CharacteristicTTs}
\end{table}

The truth table of sentences that contain only one connective is given by the characteristic truth table for that connective. So the truth table for the sentence $P \eand Q$ looks just like the characteristic truth table for \eand, with the sentence letters $P$ and $Q$ substituted in. The truth tables for more complicated sentences can simply be built up out of the truth tables for these basic sentences. Consider the sentence $(H\eand I)\eif H$. This sentence has two sentence letters, so we can represent all the possible truth assignments using a four line truth table. We can start by writing out all the possible combinations of true and false for $H$ and $I$ in the reference columns. We then copy the truth values for the sentence letters and write them underneath the letters in the sentence.

\begin{center}
\tabulinesep=.5ex
\begin{tabu}{c|c||@{\TTon}*{5}{c}@{\TToff}}
$H$&$I$&$(H$&\eand&$I)$&\eif&$H$\\
\hline
 T & T & T & & T & & T\\
 T & F & T & & F & & T\\
 F & T & F & & T & & F\\
 F & F & F & & F & & F
\end{tabu}
\end{center}

Now consider just one part of the sentence above, the subsentence $H\eand I$. This is a conjunction \script{A}\eand\script{B} with $H$ as \script{A} and with $I$ as \script{B}. $H$ and $I$ are both true on the first row. Since a conjunction is true when both conjuncts are true, we write a T underneath the conjunction symbol. We continue for the other three rows and get this:

\begin{center}
\begin{tabu}{c|c||ccccc}%{c|c||@{\TTon}*{5}{c}@{\TToff}}
\multicolumn{1}{r}{} &\multicolumn{1}{r}{} & \multicolumn{3}{c}{\script{A} \eand  \script{B}} & & \\
\multicolumn{1}{r}{} &\multicolumn{1}{r}{} & \multicolumn{3}{c}{\downbracefill} & & \\
$H$	&	$I$	&	$(H$	&\eand	&	$I)$	&	\eif	&	$H$\\
\hline
 T & T & T & \TTbf{T} & T & & T\\
 T & F & T & \TTbf{F} & F & & T\\
 F & T & F & \TTbf{F} & T & & F\\
 F & F & F & \TTbf{F} & F & & F
\end{tabu}
\end{center}

Next we need to fill in the final column under the conditional. The conditional is the main connective of the sentence, so the whole sentence is of the form $\script{A}\eif\script{B}$ with $(H \eand I)$ as \script{A} and with $H$ as \script{B}. So to fill the final column, we just need to look at the characteristic truth table for the conditional. For the first row, the sentence $(H \eand I)$ is true and the sentence $H$ is also true. The truth table for he conditional tells us this means that the whole sentence is true. Filling out the rest of the column gives us this:

\begin{center}
\begin{tabu}{c|c||ccccc}%{c|c||@{\TTon}*{5}{c}@{\TToff}}
\multicolumn{1}{r}{} &\multicolumn{1}{r}{} & \multicolumn{3}{c}{\script{A}}		& \eif 					&	\script{B} \\
\multicolumn{1}{r}{} &\multicolumn{1}{r}{} & \multicolumn{3}{c}{\downbracefill}	& \downbracefill	&	\downbracefill \\
$H$&$I$&$(H$&\eand&$I)$&\eif&$H$\\
\hline
 T & T & T  & {T} & T &\TTbf{T} & T\\
 T & F & T & {F} &  F &\TTbf{T} & T\\
 F & T & F & {F} & T &\TTbf{T} & F\\
 F & F & F & {F} & F &\TTbf{T} & F
\end{tabu}
\end{center}

The column of Ts underneath the conditional tells us that the sentence $(H \eand I)\eif H$ is true regardless of the truth values of $H$ and $I$. They can be true or false in any combination, and the compound sentence still comes out true. It is crucial that we have considered all of the possible combinations. If we only had a two-line truth table, we could not be sure that the sentence was not false for some other combination of truth values.

In this example, the script letters over the table have just been there to indicate how the columns get filled in. We won't need them in the final product. Also, the reference columns are redundant with the columns under the individual sentence letters, so we can eliminate those as well. Most of the time, when you see truth tables, we will just write them out this way:
\begin{center}
\begin{tabu}{ccccc}
$(H$	&	\eand	&	$I)$	& \eif	\tikz[overlay, shift={(-1ex,-27pt)}, gray] \draw (0pt,0pt) ellipse (2ex and 44pt);			&$H$\\
\hline
T 		& 	{T} 	& 	T 		& T 	& T\\
T 		& 	{F} 	& 	F 		& T 	& T\\
F 		& 	{F} 	&	T 		& T 	& F\\
F 		& 	{F} 	& 	F 		& T 	& F
\end{tabu}
\end{center}
\label{tautology3.1} 

%\begin{tikzpicture}[remember picture, overlay]
%\end{tikzpicture}


%\markcells[gray]{1 em}{66 pt}

The truth value of the sentence on each row is just the column underneath the \emph{main connective} (see p. \pageref{def:main_connective}) of the sentence, in this case, the column underneath the conditional.

\newglossaryentry{complete truth table}
{
name=complete truth table,
description={A table that gives all the possible interpretations for a sentence or set of sentences in SL.}
}

A \textsc{\gls{complete truth table}} \label{def:complete_truth_table} is a table that gives all the possible interpretations for a sentence or set of sentences in SL. It has a row for each possible assignment of T and F to all of the sentence letters. The size of the complete truth table depends on the number of different sentence letters in the table. A sentence that contains only one sentence letter requires only two rows, as in the characteristic truth table for negation. This is true even if the same letter is repeated many times, as in this sentence: $$[(C\eiff C) \eif C] \eand \enot(C \eif C).$$ The complete truth table requires only two lines because there are only two possibilities: $C$ can be true, or it can be false. A single sentence letter can never be marked both T and F on the same row. The truth table for this sentence looks like this:
\begin{center}
\begin{tabu}{cccccccccc}%{c@{\TTon}*{13}{c}@{\TToff}}
[($C$	&\eiff	&	$C)$	&	\eif	&	$C]$	&	\eand	\tikz[overlay, shift={(-1ex,-12pt)}, gray] \draw (0pt,0pt) ellipse (2ex and 27pt);		&\enot	&	$(C$	&	\eif	&	$C)$\\
\hline
	T 	&  T  	& 	T 		&  T  		& 	T 		&	F	&  F		& T 		&  T 		& T   \\
	F 	&  T  	& 	F		&  F  		&	 F 		&	F	&  F		& F 		&  T  		& F  
\end{tabu}
\end{center}
\label{contradiction3.1}
Looking at the column underneath the main connective, we see that the sentence is false on both rows of the table; i.e., it is false regardless of whether $C$ is true or false.

A sentence that contains two sentence letters requires four lines for a complete truth table, as we saw above in the table for $(H \eand I)\eif I$.

A sentence that contains three sentence letters requires eight lines, as in this example. Here the reference columns are included so you can see how to arrange the truth values for the individual sentence letters so that all the possibilities are covered.

\begin{center}
\begin{tabu}{c|c|c|@{\TTon}*{5}{c}@{\TToff}}
$M$	&	$N$	&	$P$	&	$M$	&	\eand	\tikz[overlay, shift={(-1.25ex,-52pt)}, gray] \draw (0pt,0pt) ellipse (2ex and 66pt);			&	$(N$	&	\eor	&	$P)$\\
\hline
%           M        &     N   v   P
T		& T 		& T 		& T 		& T & T & T & T\\
T 		& T 		& F 		& T 		& T & T & T & F\\
T 		& F 		& T 		& T 		& T & F & T & T\\
T 		& F 		& F 		& T 		& F & F & F & F\\
F 		& T 		& T 		& F 		& F & T & T & T\\
F 		& T 		& F 		& F 		& F & T & T & F\\
F 		& F 		& T 		& F 		& F & F & T & T\\
F 		& F 		& F 		& F 		& F & F & F & F
\end{tabu}
\end{center}
\label{contingentsentence3.1}
From this table, we know that the sentence $M\eand(N\eor P)$ might be true or false, depending on the truth values of $M$, $N$, and $P$.

A complete truth table for a sentence that contains four different sentence letters requires 16 lines. For five letters, 32 lines are required. For six letters, 64 lines, and so on. To be perfectly general: If a complete truth table has $n$ different sentence letters, then it must have $2^n$ rows.

By convention, the reference columns are filled in with the right most row alternating Ts and Fs. The next column over alternates sets of two Ts and two Fs. For the third column from the right, you have sets of four Ts and four Fs. This continues until you reach the leftmost column, which will always have the top have all Ts and the bottom half all Fs. This convention is completely arbitrary. There are other ways to be sure that all the possible combinations are covered, but everything is easier if we all stick to the same pattern.

%%%%%%%%%%%%%%%%% practice problems


\practiceproblems
\noindent\noindent\problempart Identify the main connective in the each sentence.

\begin{longtabu}{p{.1\linewidth}p{.9\linewidth}}
\textbf{Example}: & $(A \eif C) \eand \enot D$ \\
\textbf{Answer}: & $(A \eif C) \circled[gray, shape=circle]{\eand} \enot D$\\
\end{longtabu}



\begin{exercises}

\item \iflabelexists{showanswers}{$\circled[red, shape=circle]{\enot}(A \eor \enot B)$}{$\enot(A \eor \enot B) $}

\item \iflabelexists{showanswers}{$\enot(A \eor \enot B) \circled[red, shape=circle]{\eor} \enot(A \eand D)$}	{$\enot(A \eor \enot B) \eor \enot(A \eand D)$}
	
\item \iflabelexists{showanswers}{$[\enot(A \eor \enot B) \eor \enot (A \eand D)] \circled[red, shape=circle]{\eif} E$}{$[ \enot(A \eor \enot B) \eor \enot (A \eand D)] \eif E$}	

\item \iflabelexists{showanswers}{$[(A \eif B) \eand C]$ \circled[red, shape=circle]{\eiff} $[A \eor (B \eand C)]$}{$[(A \eif B) \eand C] \eiff [A \eor (B \eand C)]$ }  

\item \iflabelexists{showanswers}{\circled[red, shape=circle]{\enot} $\enot \enot [A \eor (B \eand (C \eor D))]$}{$\enot \enot \enot [A \eor (B \eand (C \eor D))] $} 
\end{exercises}

\noindent\problempart Identify the main connective in the each sentence.
\begin{exercises}

\item $[(A \eiff B) \eand C] \eif D$  %$[(A \eiff B) \eand C] $\framebox[1.1\width]{\eif}$ D$ 

\item $[(D \eand (E \eand F)) \eor G] \eiff  \enot [A \eif (C \eor G)] $ %$[(D \eand (E \eand F)) \eor G] $\framebox[1.1\width]{\eiff}$  \enot [A \eif (C \eor G)] $

\item $\enot (\enot Z \eor \enot H) $ %\framebox[1.1\width]{\enot}  $(\enot Z \eor \enot H) $

\item $(\enot (P \eand S) \eiff G) \eand Y $ %$(\enot (P \eand S) \eiff G) $\framebox[1.1\width]{\eand} $ Y $

\item $(A \eand (B \eif C)) \eor \enot D	$  %$(A \eand (B \eif C)) $\framebox[1.1\width]{\eor} $\enot D	$ 

\end{exercises}

\noindent\problempart Assume A, B, and C are true and X, Y, and Z are false and evaluate the truth of the each sentence by writing a one-line truth table.

\begin{longtabu}{p{.1\textwidth}p{.01\textwidth}p{.01\textwidth}p{.01\textwidth}p{.01\textwidth}p{.01\textwidth}p{.01\textwidth}p{.01\textwidth}p{.01\textwidth}p{.01\textwidth}}
\textbf{Example}: & \multicolumn{9}{p{.9\textwidth}}{$(A \eand \enot X) \eiff (B \eor Y)$ }\\
\textbf{Answer}: & (A &\eand &\enot& X)& \eiff	\tikz[overlay, shift={(-1ex,-6pt)}, gray] \draw (0pt,0pt) ellipse (2ex and 18pt); & (B& \eor& Y)&\\
\cline{2-9} 
& T  &    T    &  T    &  F&	T	&	 T&	T   & F&\\
\tabuphantomline
\end{longtabu}

\begin{exercises}
\item $\enot ((A \eand B) \eif X) $

\answer{
\begin{tabu}{c c c c c c}
\enot \tikz[overlay, shift={(-1ex,-6pt)}, red] \draw (0pt,0pt) ellipse (2ex and 18pt);	 &((A &	\eand&	B)&	\eif&	X) \\
\cline{1-6}
 T    &	T&	T&	T&	F&		F \\
\end{tabu}
}

\item $(Y \eor Z) \eiff	 (\enot X \eiff B)$

\answer{
\begin{tabu}{cccccccc}
(Y	&\eor &	Z)	& \eiff \tikz[overlay, shift={(-1ex,-6pt)}, red] \draw (0pt,0pt) ellipse (2ex and 18pt);	&(\enot	&X	&\eiff	&B)\\	
\cline{1-8}
F &	F &	F	&	F	&	T &	F &  	T  &	T\\
\end{tabu}
}

\item $[(X \eif A) \eor (A \eif X)] \eand Y$

\answer{
\begin{tabu}{ccccccccc}
[(X &\eif& A)& \eor& (A& \eif& X)] &\eand	\tikz[overlay, shift={(-1ex,-6pt)}, red] \draw (0pt,0pt) ellipse (2ex and 18pt);	& Y	\\
\cline{1-9}
F&  T&  T&   T&  T&  F& F &   F&  F\\
\end{tabu}
}

\item $(X  \eif  A) \eor  (A \eif X)$

\answer{
\begin{tabu}{ccccccc}
(X & \eif & A) &\eor \tikz[overlay, shift={(-1ex,-6pt)}, red] \draw (0pt,0pt) ellipse (2ex and 18pt);	& (A & \eif& X)\\	
\cline{1-7}
F&  T&  T&   T&  T&  F& F\\   
\end{tabu}
}

\item $[A \eand (Y \eand Z)] \eor A $

\answer{
\begin{tabu}{ccccccc}
[A& \eand &(Y &\eand &Z)]& \eor	\tikz[overlay, shift={(-1ex,-6pt)}, red] \draw (0pt,0pt) ellipse (2ex and 18pt);	& A \\
\cline{1-7}
T&  F&  F&  F& F&   T& T\\
\end{tabu}
}
\end{exercises}


\noindent\problempart
Assume A, B, and C are true and X, Y, and Z are false and evaluate the truth of the each sentence by writing a one-line truth table.. 


\begin{exercises}
\item $\enot  \enot  (\enot  \enot  \enot A  \eor  X) $

%\begin{tabular}{c|c|ccccccc}
%\cline{2-2}
%1. &	\enot & \enot & (\enot & \enot & \enot &A & \eor & X) \\	
%&F	&T	& F& T& F& T & F & F \\
%\cline{2-2}
%\end{tabular}
%\vspace{1em}

\item $(A \eif B) \eif X$	

%\begin{tabular}{cccc|c|c}
%\cline{5-5}
%2. &	(A& \eif& B)& \eif& X	\\
%&T &T&T&F&F\\
%\cline{5-5}
%\end{tabular}

\item $((A \eor B) \eand (C \eiff X)) \eor Y$	

%\begin{tabular}{cccccccc|c|c}
%\cline{9-9}
%3.&	((A &\eor& B)& \eand& (C& \eiff& X))& \eor& Y	\\
%&T&T&T&F&T&F&F&F&F\\
%\cline{9-9}
%\end{tabular}

\item $(A \eif 	B)	\eor 	(X 	\eand 	(Y 	\eand 	Z))$	

%\begin{tabular}{cccc|c|ccccc}
% \cline{5-5}
%4.&	(A&	\eif &	B)&	\eor &	(X &	\eand &	(Y &	\eand &	Z)) \\
%&	T &	T &	T &	T &	F &	F &	F &	F &	F \\
%\cline{5-5}
%\end{tabular}

\item $((A  	\eor 	X) \eif Y) 	\eand B $

%\begin{tabular}{cccccc|c|c}
%\cline{7-7}
%5.&	((A  &	\eor &	X) &	\eif &	Y) &	\eand &	B \\
%&	T &	T &	F &	F &	F &	F &	T \\
%\cline{7-7}
%\end{tabular}

\end{exercises}

\noindent\problempart Write complete truth tables for the following sentences and mark the column that represents the possible truth values for the whole sentence.

\begin{longtabu}{p{.1\linewidth}p{.9\linewidth}}
\textbf{Example}: & $D \eif (D \eand (\enot F \eor F))$ \\
\textbf{Answer}: & \vspace{-8pt} \begin{tabular}[t]{cccccccc}
D 	&\eif 	\tikz[overlay, shift={(-1ex,-20pt)}, gray] \draw (0pt,0pt) ellipse (2ex and 33pt);	&(D 	&\eand 	& (\enot	& F 	& \eor 	&  F))\\
\cline{1-8}
T	&	T	&	T	&	T		&	F	  	&	T	&	T		& T	\\
T	&	T	&	T	&	T		&	T	  	&	F	&	T		& F	\\
F	&	T	&	F	&	F		&	F	  	&	T	&	T		& T	\\
F	&	T	&	F	&	F		&	T	  	&	F	&	T		& F 	\\
\end{tabular}\\
\end{longtabu}



\begin{exercises}

\item $\enot (S \eiff (P \eif S))$

\answer{
\begin{longtabu}{cccccc}
\enot \tikz[overlay, shift={(-1ex,-27pt)}, red] \draw (0pt,0pt) ellipse (2ex and 44pt);	&	(S 	&	\eiff	&	(P 	&	\eif	&	S))	\\ 
\cline{1-6}
F 		&	T	&	T	&	T	&	T	&	T	\\
F 		&	T	&	T	&	F	&	T	&	T	\\
F 		&	F	&	T	&	T	&	F	&	F	\\
T 		&	F	&	F	&	F	&	T	&	F	\\
\end{longtabu}
}

 \item $\enot [(X \eand Y) \eor (X \eor Y)]$

\answer{
\begin{longtabu}{cccccccc}
\enot	\tikz[overlay, shift={(-1ex,-27pt)}, red] \draw (0pt,0pt) ellipse (2ex and 44pt);	&	 [(X 	&	\eand& 	Y) 	&	\eor 	&	(X 	&	\eor 	&	Y)] \\
\cline{1-8}
F	&	T	&	T	&	T	&	T	&	T	&	T	&	T	\\
F	&	T	&	F	&	F	&	T	&	T	&	T	&	F	\\
F	&	F	&	F	&	T	&	T	&	F	&	T	&	T	\\
T	&	F	&	F	&	F	&	F	&	F	&	F	&	F	\\
\end{longtabu}
}

\item $(A \eif B) \eiff (\enot B\eiff \enot A)$

\answer{
\begin{longtabu}{ccccccccc}
(A 	&	\eif	&	B)	&	 \eiff 	\tikz[overlay, shift={(-1ex,-27pt)}, red] \draw (0pt,0pt) ellipse (2ex and 44pt);	&	(\enot&	B 	&	\eiff 	&	 \enot 	& 	 A) \\
\cline{1-9}
T	&	T	&	T	&	T		&	F	 &	T	&	T	&	F		&	T	\\	
T	&	F	&	F	&	T		&	T	 &	F	&	F	&	F		&	T	\\
F	&	T	&	T	&	F		&	F	 &	T	&	F	&	T		&	F	\\
F	&	T	&	F	&	T		&	T	 &	F	&	T	&	T		&	F	\\
\end{longtabu}
}

\item $[C \eiff (D \eor E)] \eand \enot C$

\answer{
\begin{longtabu}{cccccccc}
[C 	&	\eiff 	&	(D 	&	\eor 	&	E)] 	&	\eand 	\tikz[overlay, shift={(-1ex,-52pt)}, red] \draw (0pt,0pt) ellipse (2ex and 77pt);	&	 \enot 	&	 C \\
\cline{1-8}
T	&	T	&	T	&	T	&	T	&	F		&	F		&	T	\\
T	&	T	&	T	&	T	&	F	&	F		&	F		&	T	\\
T	&	T	&	F	&	T	&	T	&	F		&	F		&	T	\\
T	&	F	&	F	&	F	&	F	&	F		&	F		&	T	\\
F	&	F	&	T	&	T	&	T	&	F		&	T		&	F	\\
F	&	F	&	T	&	T	&	F	&	F		&	T		&	F	\\
F	&	F	&	F	&	T	&	T	&	F		&	T		&	F	\\
F	&	T	&	F	&	F	&	F	&	T		&	T		&	F	\\
\end{longtabu}
}

\item $\enot(G \eand (B \eand H)) \eiff (G \eor (B \eor H))$

\answer{
\begin{longtabu}{cccccccccccc}
\enot&	(G 	&\eand &	(B 	&	 \eand 	&	 H))	&	\eiff \tikz[overlay, shift={(-1ex,-52pt)}, red] \draw (0pt,0pt) ellipse (2ex and 77pt); 	&	(G 	& \eor 	& (B 	& \eor	& H))	\\
\cline{1-12}
F	   &	T	&	  T &	T	&	T		&	T	&	F	&	T	&	T	&	T	&	T	&	T	\\
T	   &	T	&	  F &	T	&	F		&	F	&	T	&	T	&	T	&	T	&	T	&	F	\\	
T	   &	T	&	 F  &	F	&	F		&	T	&	T	&	T	&	T	&	F	&	T	&	T	\\
T	   &	T	&	 F  &	F	&	F		&	F	&	T	&	T	&	T	&	F	&	F	&	F	\\
T	   &	F	&	F   &	T	&	T		&	T	&	T	&	F	&	T	&	T	&	T	&	T	\\
T	   &	F	&	F   &	T	&	F		&	F	&	T	&	F	&	T	&	T	&	T	&	F	\\
T	   &	F	&	F   &	F	&	F		&	T	&	T	&	F	&	T	&	F	&	T	&	T	\\
T	   &	F	&	F   &	F	&	F		&	F	&	F	&	F	&	F	&	F	&	F	&	F	\\
\end{longtabu}
}

\end{exercises}

\noindent\problempart Write complete truth tables for the following sentences and mark the column that represents the possible truth values for the whole sentence.

\begin{exercises}

\item	$(D \eand \enot D) \eif G $

%\vspace{1em}

%\begin{tabular}{ccccc|c|c}
%\cline{6-6}
%1.	&	(D 	&	 \eand 	& 	 \enot	&	 D) 	&	 \eif 	&	 G \\
%	&	T	&	F		&	F		&	T	&	T	&	T	\\
%	&	T	&	F		&	F		&	T	&	T	&	F	\\
%	&	F	&	F		&	T		&	F	&	T	&	T	\\
%	&	F	&	F		&	T		&	F	&	T	&	F	\\
%\cline{6-6}
%\end{tabular}
%\vspace{1em}


\item	$(\enot P \eor \enot M) \eiff M $

%\begin{tabular}{cccccc|c|c}
%\cline{7-7}
%2.	&	(\enot 	&	P 	&	\eor 	&	\enot 	& 	 M) 	& 	\eiff 	&	 M \\
%	&	F		&	T	&	F	&	F		&	T	&	T	&	T	\\
%	&	F		&	T	&	T	&	T		&	F	&	F	&	F	\\
%	&	T		&	F	&	T	&	F		&	T	&	T	&	T	\\
%	&	T		&	F	&	T	&	T		&	F	&	T	&	F	\\
%\cline{7-7}
%\end{tabular}
%\vspace{1em}



\item	$\enot \enot (\enot A \eand \enot B)  $

%\begin{tabular}{c|c|cccccc}
%\cline{2-2}
%3.	&	\enot		&	 \enot 	&	(\enot 	& 	 A 	& \eand 	& 	\enot 	&	 B)  \\
%	&	F		&	T		&	F		&	T	&	F	&	F		&	T	\\
%	&	F		&	T		&	F		&	T	&	F	&	T		&	F	\\
%	&	F		&	T		&	T		&	F	&	F	&	F		&	T	\\
%	&	T		&	F		&	T		&	F	&	T	&	T		&	F	\\
%\cline{2-2}
%\end{tabular}
%\vspace{1em}



\item 	$[(D \eand R) \eif I] \eif \enot(D \eor R) $

%\begin{tabular}{cccccc|c|cccc}
%\cline{7-7}
%4.	&	[(D 	& 	 \eand 	& 	 R)	& 	\eif 	&	I] 	&	\eif 	&	 \enot 	&	(D 	&	 \eor 	& R) \\
%	&	T	&	T		&	T	&	T	&	T	&	F	&	F		&	T	&	T		&T	\\
%	&	T	&	T		&	T	&	F	&	F	&	T	&	F		&	T	&	T		&T	\\
%	&	T	&	F		&	F	&	T	&	T	&	F	&	F		&	T	&	T		&F	\\
%	&	T	&	F		&	F	&	T	&	F	&	F	&	F		&	T	&	T		&F	\\
%	&	F	&	F		&	T	&	T	&	T	&	F	&	F		&	F	&	T		&T	\\
%	&	F	&	F		&	T	&	T	&	F	&	F	&	F		&	F	&	T		&T	\\
%	&	F	&	F		&	F	&	T	&	T	&	T	&	T		&	F	&	F		&F	\\
%	&	F	&	F		&	F	&	T	&	F	&	T	&	T		&	F	&	F		&F	\\
%\cline{7-7}
%\end{tabular}
%	
%\vspace{1em}


\item	$\enot [(D \eiff O) \eiff A] \eif (\enot D \eand O) $

%\begin{tabular}{ccccccc|c|cccc}
%\cline{8-8}
%5.	&	\enot 	&	[(D 	&	\eiff 	&	O) 	&	\eiff 	&	 A]	& 	\eif 	 &	(\enot 	& 	D 	 & 	 \eand &O) \\ 
%	&	F		&	T	&	T	&	T	&	T	&	T	&	T	&	F		&	T	&	F	&T	\\
%	&	T		&	T	&	T	&	T	&	F	&	F	&	F	&	F		&	T	&	F	&T	\\
%	&	T		&	T	&	F	&	F	&	F	&	T	&	F	&	F		&	T	&	F	&F	\\
%	&	F		&	T	&	F	&	F	&	T	&	F	&	T	&	F		&	T	&	F	&F	\\
%	&	T		&	F	&	F	&	T	&	F	&	T	&	T	&	T		&	F	&	T	&T	\\
%	&	F		&	F	&	F	&	T	&	T	&	F	&	T	&	T		&	F	&	T	&T	\\
%	&	F		&	F	&	T	&	F	&	T	&	T	&	T	&	T		&	F	&	F	&F	\\
%	&	T		&	F	&	T	&	F	&	F	&	F	&	T	&	T		&	F	&	F	&F	\\
%\cline{8-8}
%\end{tabular}
%\vspace{1em}


\end{exercises}


% *********************************************
% *   Using Truth Tables								*
% *********************************************

\section{Using Truth Tables}

Because truth table show all the possible interpretations of a sentence or set of sentences we can use them to explore the logical properties we first introduced in Chapter \ref{chap:whatisformallogic}.

\subsection{Tautologies, contradictions, and contingent sentences}
We defined a tautology as a statement that must be true as a matter of logic, no matter how the world is (p. \pageref{def:tautology}). A statement like ``Either it is raining or it is not raining'' is always true, no matter what the weather is like outside. Something similar goes on in truth tables. With a complete truth table, we consider all of the ways that the world might be. Each line of the truth table corresponds to a way the world might be. This means that if the sentence is true on every line of a complete truth table, then it is true as a matter of logic, regardless of what the world is like.

We can use this fact to create a test for whether a sentence is a tautology: if the column under the main connective of a sentence is a T on every row, the sentence is a tautology. We already have seen an example of this. On page \pageref{tautology3.1} that the sentence $(H \eand I)\eif H$ had only T's under its main connective, so it is a tautology.

Not every tautology in English will correspond to a tautology in SL. The sentence ``All bachelors are unmarried'' is a tautology in English, but we cannot represent it as a tautology in SL, because it just translates as a single sentence letter, like $B$. On the other hand, if something is a tautology in SL, it will also be a tautology in English. No matter how you translate $A \eor \enot A$, if you translate the $A$s consistently, the statement will be a tautology. 

\newglossaryentry{semantic tautology in SL}
{
name=semantic tautology in SL,
description={A statement that has only Ts in the column under the main connective of its complete truth table.}
}

\label{semantic_definitions_in_SL}
Rather than thinking of complete truth tables as an imperfect test for the English notion of a tautology, we can define a separate notion of a tautology in SL based on truth tables. A statement is a \textsc{\gls{semantic tautology in SL}} \label{def:semantic_tautology_in_sl} if and only if the column  under the main connective in the complete truth table for the sentence contains only Ts. This is the semantic definition of a tautology in SL, because it uses truth tables. Later we will create a separate, syntactic definition and show that it is equivalent to the semantic definition. We will be doing the same thing for all the concepts defined in this section. 

\newglossaryentry{semantic contradiction in SL}
{
name=semantic contradiction in SL,
description={A statement that has only Fs in the column under the main connective of its complete truth table.}
}

We defined a contradiction as a sentence that is false no matter how the world is (p. \pageref{def:contradiction}). This means we can define a \textsc{\gls{semantic contradiction in SL}} \label{def:semantic_contradiction_in_sl} as a sentence that has only Fs in the column under them main connective of its complete truth table. We saw on page \pageref{contradiction3.1} that the sentence $[(C\eiff C) \eif C] \eand \enot(C \eif C)$ was a contradiction in this sense. As with the definition of a semantic tautology, this is a semantic definition because it uses truth tables. 
		
\newglossaryentry{semantically contingent in SL}
{
name=semantically contingent in SL,
description={A property held by a sentence in SL if and only if the complete truth table for that sentence has both Ts and Fs under its main connective.}
}

Finally, a sentence is contingent if it is sometimes true and sometimes false (p. \pageref{def:contingent_statement}). Similarly, a sentence is \textsc{\gls{semantically contingent in SL}} \label{def:semantically_contingent_in_sl} if and only if its complete truth table for has both Ts and Fs under the main connective. We saw on page \pageref{contingentsentence3.1} that the sentence $M \eand (N \eor P)$ was contingent.

\subsection{Logical equivalence}

\newglossaryentry{semantically logically equivalent in SL}
{
name=semantically logically equivalent in SL,
description={A property held by pairs of sentences if and only if the complete truth table for those sentences has identical columns under the two main connectives.}
}

Two sentences are logically equivalent in English if they have the same truth value as a matter of logic (p. \pageref{def:logical_equivalence}). Once again, we can use truth tables to define a similar property in SL: Two sentences are \textsc{\gls{semantically logically equivalent in SL}} \label{def:semantically_logically_equivalent_in_sl} if they have the same truth value on every row of a complete truth table.

Consider the sentences $\enot(A \eor B)$ and $\enot A \eand \enot B$. Are they logically equivalent? To find out, we construct a truth table.
\begin{center}
\begin{tabu}{ccccc|cccccc}
\enot	\tikz[overlay, shift={(-1ex,-30pt)}, gray] \draw (0pt,0pt) ellipse (2ex and 44pt);		&	$(A$	&	\eor	&	$B)$	&	&	&	\enot	&	$A$	&	\eand	\tikz[overlay, shift={(-1ex,-30pt)}, gray] \draw (0pt,0pt) ellipse (2ex and 44pt); &	\enot	&	$B$\\
\hline
F	& 	T 		& T 		& T 		& 	&	&	F & T & F & F & T\\
F 	&	T 		& T 		& F 		& 	&	&	F & T & F & T & F\\
F 	& 	F 		& T		& T 		& 	&	&	T & F & F & F & T\\
T 	& 	F 		& F 		& F 		& 	&	&	T & F & T & T & F
\end{tabu}
\end{center}
Look at the columns for the main connectives; negation for the first sentence, conjunction for the second. On the first three rows, both are F. On the final row, both are T. Since they match on every row, the two sentences are logically equivalent.

\subsection{Consistency}

\newglossaryentry{semantically consistent in SL}
{
name=semantically consistent in SL,
description={A property held by sets of sentences if and only if the complete truth table for that set contains one line on which all the sentences are true}
}

A set of sentences in English is consistent if it is logically possible for them all to be true at once (p. \pageref{def:inconsistency}).
This means that a sentence is \textsc{\gls{semantically consistent in SL}} \label{def:semantically_consistent_in_sl} if and only if there is at least one line of a complete truth table on which all of the sentences are true. It is semantically inconsistent otherwise.

Consider the three sentences $A \eif B$, $B \eif C$ and $C \eif A$. Since we are considering them as a set, we will put curly braces around them, as is done in set theory: \{$A \eif B, B \eif C, C \eif A$\}. The conditionals in this set form a little loop, but it is possible for all the sentences to be true at the same time, as this truth table shows.

\begin{longtabu}{cccc|ccccc|cccc}
A	&	\eif	&	B	&	&	&	B	&	\eif	&	C	&	&	&	C	&	\eif	&	A	\\
\cline{1-13}
T	\tikz[overlay, shift={(100pt,1ex)}, gray] \draw (0pt,0pt) ellipse (132pt and 2ex); &	T		&	T	&	&	&	T	&	T		&	T	&	&	&	T	&		T	&	T	\\
T	&	T		&	T	&	&	&	T	&	F		&	F	&	&	&	F	&		T	&	T	\\
T	&	F		&	F	&	&	&	F	&	T		&	T	&	&	&	T	&		T	&	T	\\
T	&	F		&	F	&	&	&	F	&	T		&	F	&	&	&	F	&		T	&	T	\\
F	&	T		&	T	&	&	&	T	&	T		&	T	&	&	&	T	&		F	&	F	\\
F	&	T		&	T	&	&	&	T	&	F		&	F	&	&	&	F	&		T	&	F	\\
F	&	T		&	F	&	&	&	F	&	T		&	T	&	&	&	T	&		F	&	F	\\
F	\tikz[overlay, shift={(100pt,1ex)}, gray] \draw (0pt,0pt) ellipse (132pt and 2ex); &	T		&	F	&	&	&	F	&	T		&	F	&	&	&	F	&		T	&	F	\\			
\end{longtabu}

\subsection{Validity}

\newglossaryentry{semantically valid in SL}
{
name=semantically valid in SL,
description={A property held by arguments if and only if the complete truth table for the argument contains no rows where the premises are all true and the conclusion false.}
}

Logic is the study of argument, so the most important use of truth tables is to test the validity of arguments. An argument in English is valid if it is logically impossible for the premises to be true and for the conclusion to be false at the same time (p. \pageref{def:valid}). So we can define an argument as \textsc{\gls{semantically valid in SL}} \label{def:semantically_valid_in_sl} if there is no row of a complete truth table on which the premises are all marked ``T'' and the conclusion is marked ``F.'' An argument is invalid if there is such a row.

Consider this argument:
\begin{earg}
\item[1.] $\enot L \eif (J \eor L)$
\item[2.] $\enot L$
\item[] \textcolor{white}{.}\sout{\hspace{.2\linewidth}} \textcolor{white}{.} 
\item[$\therefore$] $J$
\end{earg}
Is it valid? To find out, we construct a truth table.

\begin{center}
\tabulinesep=.5ex
\begin{longtabu}{c|c|@{\TTon}*{6}{c}@{\TToff}|@{\TTon}*{2}{c}@{\TToff}|@{\TTon}c@{\TToff}}
$J$&$L$&\enot&$L$&\eif \tikz[overlay, shift={(-1.25ex,-24pt)}, gray] \draw (0pt,0pt) ellipse (2ex and 36pt);&$(J$&\eor&$L)$&\enot\tikz[overlay, shift={(-1ex,-24pt)}, gray] \draw (0pt,0pt) ellipse (2ex and 36pt);&L&J\tikz[overlay, shift={(-.75ex,-24pt)}, gray] \draw (0pt,0pt) ellipse (2ex and 36pt);\\
\hline
%J   L   -   L      ->     (J   v   L)
 T & T & F & T & T & T & T & T & F & T & T\\
 T & F & T & F & T & T & T & F & T & F & T\\
 F & T & F & T & T & F & T & T & F & T & F\\
 F & F & T & F & F & F & F & F & T & F & F
\end{longtabu}
\end{center}

Yes, the argument is valid.
The only row on which both the premises are T is the second row, and on that row the conclusion is also T.

In Chapters 1 and 2 we used the three dots $\therefore$ to represent an inference in English. We used this symbol to represent any kind of inference. The truth table method gives us a more specific notion of a valid inference. We will call this semantic entailment and represent it using a new symbol, $\sdtstile{}{},$ called the ``double turnstile.'' \label{defDoubleTurnstile} The $\sdtstile{}{}$ is like the $\therefore$, except for arguments verified by truth tables. When you use the double turnstile, you write the premises as a set, using curly brackets, \{ and \}, which mathematicians use in set theory. The argument above would be written  $ \{ \enot L \eif (J \eor L), \enot L \} \sdtstile{}{} J$.

More formally, we can define the double turnstile this way: $ \{ \script{A_1}\ldots \script{A_n} \} \sdtstile{}{} \script{B} $ if and only if there is no truth value assignment for which \script{A_1}\ldots \script{A_n} are true and \script{B} is false. Put differently, it means that \script{B} is true for any and all truth value assignments for which \script{A_1}\ldots \script{A_n} are true.

We can also use the double turnstile to represent other logical notions. Since a tautology is always true, it is like the conclusion of a valid argument with no premises. The string $\sdtstile{}{}\script{C}$ means that \script{C} is true for all truth value assignments. This is equivalent to saying that the sentence is entailed by anything. We can represent logical equivalence by writing the double turnstile in both directions: $\script{A} \ndststile{}{} \hspace{.5em} \sdtstile{}{} \script{B}$ For instance, if we want to point out that the sentence $A \eand B$ is equivalent to $B \eand A$ we would write this: $A \eand B \ndststile{}{} \hspace{.5em} \sdtstile{}{} B \eand A$. 

%%%%%%%%%%%%%%%%%%Practice Problems

\practiceproblems

If you want additional practice, you can construct truth tables for any of the sentences and arguments in the exercises for the previous chapter.


\noindent\problempart Determine whether each sentence is a tautology, a contradiction, or a contingent sentence, using a complete truth table.

\begin{longtabu}{p{.1\linewidth}p{.9\linewidth}}
\textbf{Example}: & $(A \eif B) \eor (B \eif A)$ \\
\textbf{Answer}: & \vspace{-8pt}\begin{tabular}[t]{cccccccc} 
	 (A 	 	 & 	 \eif 	& 	 B) 	 	 & 	 \eor \tikz[overlay, shift={(-.75ex,-24pt)}, gray] \draw (0pt,0pt) ellipse (2ex and 36pt);	 & 	(B 	 	 & 	 \eif	 	 	 & 	 A)	 	 & 	 Tautology\\ 
\cline{1-7}
 T 	 	 & 	 T 		& 	T 	 	 & 	 T 		 & 	 T 	 	 & 	 T 	 	 & 	T 	 	 & 	 \\ 
 T 	 	 & 	 F 		& 	F 	 	 & 	 T 	 	 & 	 F 	 	 & 	 T 	 	 & 	T 	 	 & 	  \\ 
 F 	 	 & 	 T 		& 	T 	 	 & 	 T 	 	 & 	 T 	 	 & 	 F 	 	 & 	F 	 	 & 	 \\ 
 F 	 	 & 	 T		& 	F 	 	 & 	 T 	 	 & 	 F 	 	 & 	 T 	 	 & 	F 	 	 & 	 \\ 

\end{tabular}\\
\end{longtabu}

\begin{exercises}
\item $A \eif A$ 

\answer{
\begin{longtabu}{ccccc}
A 	&\eif \tikz[overlay, shift={(-1ex,-12pt)}, red] \draw (0pt,0pt) ellipse (2ex and 24pt);	& A & Tautology\\
\cline{1-3}
T		&	T	& T	 &			\\
F		&	T	& F	 &			\\
\end{longtabu}
}


\item $C \eif\enot C$ 

\answer{
\begin{longtabu}{ccccc}

C 	& \eif \tikz[overlay, shift={(-1ex,-12pt)}, red] \draw (0pt,0pt) ellipse (2ex and 24pt);	& \enot 	& C 	& Contingent \\
\cline{1-4}
T	&	F	&	F	& 	T	&			\\
F	&	T	&	T	& 	F	&	\\

\end{longtabu}
}

\item $(A \eiff B) \eiff \enot(A\eiff \enot B)$ %tautology

\answer{
\begin{longtabu}{cccccccccc}
(A 	& \eiff 	& B) 	& \eiff \tikz[overlay, shift={(-1ex,-30pt)}, red] \draw (0pt,0pt) ellipse (2ex and 44pt);	& \enot 	& (A 	& \eiff 	& \enot 	& B) 	& Tautology \\
\cline{1-9}
T	&	T	&	T 	&	T	&	T	&	T	&	F	&	F	& 	T	&	\\	
T	&	F	&	F	&	T	&	 F	&	T	&	T	&	T	& 	F	&	\\
F	&	F	&	T	&	T	&	 F	&	F	&	T	&	F	& 	T	&	\\
F	&	T	&	F	&	T	&	 T	&	F	&	F	&	T	& 	F	&	\\
\end{longtabu}
}


\item $(A \eand B) \eif (B \eor A)$  %taut

\answer{
    \begin{longtabu}{cccccccc} 
(A  	 	 & 	 \eand  	  & 	 B)  & 	 \eif  \tikz[overlay, shift={(-1ex,-30pt)}, red] \draw (0pt,0pt) ellipse (2ex and 44pt);	 & 	 (B 	 	 & 	 \eor  	 & 	 A)   	 	 & 	 Tautology\\ 
\cline{1-7} 
T 	 	 & 	 T 	 	 & 	 T 	& 	 T 	 	 & 	 T 	 	 & 	 T 	 	 & 	T 	 	 & 	   \\ 
T 	 	 & 	 F 	 	 & 	 F 	& 	 T 	 	 & 	 F 	 	 & 	 T 	 	 & 	T 	 	 & 	   \\ 
F 	 	 & 	 F 	 	 & 	 T 	& 	 T 	 	 & 	 T 	 	 & 	 T 	 	 & 	F 	 	 & 	   \\ 
F 	 	 & 	 F 	 	 & 	 F 	& 	 T 	 	 & 	 F 	 	 & 	 F 	 	 & 	F 	 	 & 	   \\ 
\end{longtabu}
}


\item $[(\enot A \eor A) \eor B] \eif B$ %taut.

\answer{
 \begin{longtabu}{ccccccccc} 
[(\enot  	  & 	 A  	 	 & 	 \eor  	 & 	A)	  	 & 	\eor	 	 & 	B]   	 & 	 \eif	\tikz[overlay, shift={(-1ex,-30pt)}, red] \draw (0pt,0pt) ellipse (2ex and 44pt);	 & 	 B 	 	 & 	 Contingent sentence \\ 
\cline{1-8} 
F 	 	 & 	 T 	 	 & 	 T 	 	 & 	 T 	 	 & 	 T 	 	 & 	 T 	 	 & 	 T 	 	 & 	 T 	 	 & 	 \\ 
F 	 	 & 	 T 	 	 & 	 T 	 	 & 	 T 	 	 & 	 T 	 	 & 	 F 	 	 & 	 F 	 	 & 	 F 	 	 & 	 \\ 
T 	 	 & 	 F 	 	 & 	 T 	 	 & 	 F 	 	 & 	 T 	 	 & 	 T 	 	 & 	 T 	 	 & 	 T 	 	 & 	 \\ 
T 	 	 & 	 F 	 	 & 	 T 	 	 & 	 F 	 	 & 	 T 	 	 & 	 F 	 	 & 	 F 	 	 & 	 F 	 	 & 	 \\ 

\end{longtabu}
}

\item $[(A \eor B) \eand \enot A] \eand (B \eif A)$ %Contradiction. 

\answer{
\begin{longtabu}{ccccccccccc} 
[(A  	 &	\eor  & 	 B) 	 & 	 \eand 	& 	\enot & 	 A] 	 	 & 	 \eand \tikz[overlay, shift={(-1ex,-30pt)}, red] \draw (0pt,0pt) ellipse (2ex and 44pt); & (B 	 	 & 	 \eif  	 & 	 A) 	 	& Contradiction. \\ 
\cline{1-10} 
T 	 & 	 T 	 & 	T 	 & 	 F 	 	 & 	 F 	  & 	 T 	 	 & 	 F 	  & 	 T 	 	 & 	 T 	 	 & 	 T 	 	 & 	  \\ 
T 	 & 	 T 	 & 	F 	 & 	 F 	 	 & 	 F 	 & 	 T 	 	 & 	 F 	 & 	 F 	 	 & 	 T 	 	 & 	 T 	 	 & 	  \\ 
F 	 & 	 T 	 & 	T 	 & 	 T 	 	 & 	 T 	 & 	 F 	 	 & 	 F 	 & 	 T 	 	 & 	 F 	 	 & 	 F 	 	 & 	 \\ 
F 	 & 	 F 	 & 	F 	 & 	 F 	 	 & 	 T 	 & 	 F 	 	 & 	 F 	 & 	 F 	 	 & 	 T 	 	 & 	 F 	 	 & 	 \\ 
\end{longtabu}
}
\end{exercises}

\noindent\problempart Determine whether each sentence is a tautology, a contradiction, or a contingent sentence, using a complete truth table.
\begin{exercises}
\item $\enot B \eand B$ \vspace{.5ex}%contra


\item $\enot D \eor D$ \vspace{.5ex}%taut


\item $(A\eand B) \eor (B\eand A)$\vspace{.5ex} %contingent


\item $\enot[A \eif (B \eif A)]$\vspace{.5ex} %contra


\item $A \eiff [A \eif (B \eand \enot B)]$ \vspace{.5ex}%contra


\item $[(A \eand B) \eiff B] \eif (A \eif B)$ \vspace{.5ex}% contingent. 

\end{exercises}



\noindent\problempart \label{pr.TT.equiv} Determine whether each the following statements are equivalent using complete truth tables. If the two sentences really are logically equivalent, write "Logically equivalent." Otherwise write, "Not logically equivalent." 

\begin{longtabu}{p{.1\linewidth}p{.9\linewidth}}
\textbf{Example}: & $A \eor B  \ndststile{}{} \hspace{.5em} \sdtstile{}{} \enot A \eif B $\\
\textbf{Answer}: & \vspace{-8pt}\begin{tabular}[t]{ccccccccc}
A	&	\eor \tikz[overlay, shift={(-.75ex,-24pt)}, gray] \draw (0pt,0pt) ellipse (2ex and 36pt);	&	B	&	&	\enot	&	A	&	\eif \tikz[overlay, shift={(-.75ex,-24pt)}, gray] \draw (0pt,0pt) ellipse (2ex and 36pt);&	B	&	Logically Equivalent\\ 
\cline{1-3} \cline{5-7}
T	&	T		&	T	&	&		F	&	T	&	T		&	T	&	\\
T	&	T		&	F	&	&		F	&	T	&	T		&	F	&	\\
F	&	T		&	T	&	&		T	&	F	&	T		&	T	&	\\
F	&	F		&	F	&	&		T	&	F	&	F		&	F	&	\\
\end{tabular}\\
\end{longtabu}


\begin{exercises}
\item $A\ndststile{}{} \hspace{.5em} \sdtstile{}{} \enot A$\vspace{.5ex} %No

\answer{
\begin{longtabu}{ccccc} 
A 	 \tikz[overlay, shift={(-1.25ex,-24pt)}, red] \draw (0pt,0pt) ellipse (2ex and 36pt);	 & 	  	 	 & 	 \enot \tikz[overlay, shift={(-.75ex,-24pt)}, red] \draw (0pt,0pt) ellipse (2ex and 36pt);	 	 & 	 A 	 	 & 	 Not logically equivalent \\ 
\cline{1-1} \cline{3-4}
T 	 	 & 	   	 	 & 	 F 	 	 & 	 T 	 	 & 	  \\ 
T 	 	 & 	   	 	 & 	 F 	 	 & 	 T 	 	 & 	  \\ 
F 	 	 & 	   	 	 & 	 T 	 	 & 	 F 	 	 & 	  \\ 
F 	 	 & 	   	 	 & 	 T 	 	 & 	 F 	 	 & 	  \\ 
\end{longtabu}
}

\item $A \eand \enot A\ndststile{}{} \hspace{.5em} \sdtstile{}{} \enot B \eiff B$\vspace{.5ex} %Yes

\answer{
\begin{longtabu}{cccccccccc} 

A	 & 	 	\eand \tikz[overlay, shift={(-1.25ex,-24pt)}, red] \draw (0pt,0pt) ellipse (2ex and 36pt);	 & 	 \enot	  & 	 A	 	 & 	 	 	 & 	 \enot	 & 	 B 	 	& 	\eiff \tikz[overlay, shift={(-1.25ex,-24pt)}, red] \draw (0pt,0pt) ellipse (2ex and 36pt);	 & 	 B 	 & 	 Logically equivalent \\ 
\cline{1-4} \cline{6-9} 
T 	 	 & 	 F 	 	 & 	 F 	 	 & 	 T 	 	 & 	  	 	 & 	 F 	 	 & 	 T 	 	 & 	 F 	 	 & 	 T 	 	 & 	  \\ 
T 	 	 & 	 F 	 	 & 	 F 	 	 & 	 T 	 	 & 	  	 	 & 	 T 	 	 & 	 F 	 	 & 	 F 	 	 & 	 F 	 	 & 	  \\ 
F 	 	 & 	 F 	 	 & 	 T 	 	 & 	 F 	 	 & 	  	 	 & 	 F 	 	 & 	 T 	 	 & 	 F 	 	 & 	 T 	 	 & 	  \\ 
F 	 	 & 	 F 	 	 & 	 T 	 	 & 	 F 	 	 & 	  	 	 & 	 T 	 	 & 	 F 	 	 & 	 F 	 	 & 	 F 	 	 & 	  \\ 
\end{longtabu}
}

\item $[(A \eor B) \eor C]\ndststile{}{} \hspace{.5em} \sdtstile{}{} [A \eor (B \eor C)]$\vspace{.5ex} %Yes

\answer{
\begin{longtabu}{cccccccccccc} 
(A		 & 	 \eor	 & 	 	B) 	 & 	\eor	\tikz[overlay, shift={(-1ex,-52pt)}, red] \draw (0pt,0pt) ellipse (2ex and 77pt); 	 	 & 	 C	 	 & 	 	 	 & A 	 	 & 	\eor	\tikz[overlay, shift={(-1ex,-52pt)}, red] \draw (0pt,0pt) ellipse (2ex and 77pt); 	 	 & 	(B 	 	 & 	 \eor 	 & 	C) 	 	 & 	Logically equivalent  \\ 
\cline{1-5} \cline{7-11} 
T	 	 & 	 T	 	 & 	 	T 	 & 	 T	 	 & 	T 	 	 & 	 	 	 & 	 T	 	 & 	T 	 	 & 	 T	 	 & 	 T	 	 & 	 T	 	 & 	  \\ 
T	 	 & 	 T	 	 & 	 	 T	 & 	 T	 	 & 	 F	 	 & 	 	 	 & 	 T	 	 & 	 T	 	 & 	 T	 	 & 	 T	 	 & 	 F	 	 & 	  \\ 
T	 	 & 	 T	 	 & 	 	 F	 & 	 T	 	 & 	 T	 	 & 	 	 	 & 	 T	 	 & 	 T	 	 & 	 F	 	 & 	 T	 	 & 	 T	 	 & 	  \\ 
T	 	 & 	 T	 	 & 	 	 F	 & 	 T	 	 & 	 F	 	 & 	 	 	 & 	 T	 	 & 	 T	 	 & 	 F	 	 & 	 T	 	 & 	 F	 	 & 	  \\ 
F	 	 & 	 T	 	 & 	 	 T	 & 	 T	 	 & 	 T	 	 & 	 	 	 & 	 F	 	 & 	 T	 	 & 	 T	 	 & 	 T	 	 & 	 T	 	 & 	  \\ 
F	 	 & 	 T	 	 & 	 	 T	 & 	 T	 	 & 	 F	 	 & 	 	 	 & 	 F	 	 & 	 T	 	 & 	 T	 	 & 	 T	 	 & 	 F	 	 & 	  \\ 
F	 	 & 	 F	 	 & 	 	 F	 & 	 T	 	 & 	 T	 	 & 	 	 	 & 	 F	 	 & 	 T	 	 & 	 F	 	 & 	 F	 	 & 	 T	 	 & 	  \\ 
F	 	 & 	 F	 	 & 	 	 F	 & 	 F	 	 & 	 F	 	 & 	 	 	 & 	 F	 	 & 	 F	 	 & 	 F	 	 & 	 F	 	 & 	 F	 	 & 	  \\ 

\end{longtabu}
}

\item $A \eor (B \eand C)\ndststile{}{} \hspace{.5em} \sdtstile{}{} (A \eor B) \eand (A \eor C)$\vspace{.5ex} %Equivalent

\answer{
\begin{longtabu}{cccccccccccccc} 
A	 & 	 \eor \tikz[overlay, shift={(-1ex,-52pt)}, red] \draw (0pt,0pt) ellipse (2ex and 77pt); 		 & 	(B 	 	 & 	 \eand 	 & 	 C)	 	 & 	 	 	 & 	 (A	 	 & 	 	\eor	 & 	 	B) 	 & 	 \eand \tikz[overlay, shift={(-1ex,-52pt)}, red] \draw (0pt,0pt) ellipse (2ex and 77pt); 		 & 	 (A	 	 & 	 \eor 	 & 	 C) 	& Logically equivalent\\ 
\cline{1-13} 
T	 & 	 T	 	 & 	 T	 	 & 	 	T 	 & 	T 	 	 & 	 	 	 & 	 T	 	 & 	 T	 	 & 	 	T 	 & 	 T	 	 & 	 T	 	 & 	T 	 	 & 	 T 	 & \\ 
T	 & 	 T	 	 & 	 T	 	 & 	 	 F	 & 	 F	 	 & 	 	 	 & 	 T	 	 & 	 T	 	 & 	 	 T	 & 	 T	 	 & 	 T	 	 & 	 T	 	 & 	 F 	 &  \\ 
T	 & 	 T	 	 & 	 F	 	 & 	 F	 	 & 	 T	 	 & 	 	 	 & 	 T	 	 & 	 T	 	 & 	 F	 	 & 	 T	 	 & 	 T	 	 & 	 T	 	 & 	 T 	 &  \\ 
T	 & 	 T	 	 & 	 F	 	 & 	 F	 	 & 	 F	 	 & 	 	 	 & 	 T	 	 & 	 T	 	 & 	 F	 	 & 	 T	 	 & 	 T	 	 & 	 T	 	 & 	 F 	 &   \\ 
F	 & 	 T	 	 & 	 T	 	 & 	 T	 	 & 	 T	 	 & 	 	 	 & 	 F	 	 & 	 T	 	 & 	 T	 	 & 	 T	 	 & 	 F	 	 & 	 T	 	 & 	 T 	 &  \\ 
F	 & 	 F	 	 & 	 T	 	 & 	 F	 	 & 	 F	 	 & 	 	 	 & 	 F	 	 & 	 T	 	 & 	 T	 	 & 	 F	 	 & 	 F	 	 & 	 F	 	 & 	 F 	 &   \\ 
F	 & 	 F	 	 & 	 F	 	 & 	 F	 	 & 	 T	 	 & 	 	 	 & 	 F	 	 & 	 F	 	 & 	 F	 	 & 	 F	 	 & 	 F	 	 & 	 T	 	 & 	 T 	 &  \\ 
F	 & 	 F	 	 & 	 F	 	 & 	 F	 	 & 	 F	 	 & 	 	 	 & 	 F	 	 & 	 F	 	 & 	 F	 	 & 	 	F 	 & 	 F	 	 & 	 F	 	 & 	F 	 &    \\ 
\end{longtabu}
}

\item $[A \eand (A \eor B)] \eif B\ndststile{}{} \hspace{.5em} \sdtstile{}{} A \eif B$\vspace{.5ex} %Equivalent. 

\answer{
\begin{longtabu}{cccccccccccc} 
[A	 & 	\eand 	 & 	 (A	 	 & 	 \eor	 & 	B)] 	 	 & \eif 	
\tikz[overlay, shift={(-.75ex,-24pt)}, red] \draw (0pt,0pt) ellipse (2ex and 36pt); 	 & 	 	B 	 & 	 	 	 & 	 A	 	 & \eif 	
\tikz[overlay, shift={(-.75ex,-24pt)}, red ] \draw (0pt,0pt) ellipse (2ex and 36pt); 	 & 	 	B 	 & 	Logically equivalent  \\ 
\cline{1-7} \cline{9-11}
T	  & 	 	T 	 & 	 T	 	 & 	 T	 	 & 	T 	 	 & 	 T	 	 & 	 T	 	 & 	 	 	 & 	 T	 	 & 	T 	 	 & 	T 	 	 & 	  \\ 
T	  & 	 	T 	 & 	 T	 	 & 	 T	 	 & 	 F	 	 & 	 F	 	 & 	 F	 	 & 	 	 	 & 	 T	 	 & 	 F	 	 & 	 F	 	 & 	  \\ 
F	  & 	 	 F	 & 	 F	 	 & 	 T	 	 & 	 T	 	 & 	 T	 	 & 	 T	 	 & 	 	 	 & 	 F	 	 & 	 T	 	 & 	 T	 	 & 	  \\ 
F	  & 	 	 F	 & 	 F	 	 & 	 F	 	 & 	 F	 	 & 	 T	 	 & 	 F	 	 & 	 	 	 & 	 F	 	 & 	 T	 	 & 	 F	 	 & 	  \\ 
\end{longtabu}
}

\end{exercises}


\noindent\problempart
\label{pr.TT.equiv}
Determine whether each the following statements of equivalence are true or false using complete truth tables. If the two sentences really are logically equivalent, write "Logically equivalent." Otherwise write, "Not logically equivalent." 
\begin{exercises}
\item $A\eif A\ndststile{}{} \hspace{.5em} \sdtstile{}{} A \eiff A$ \vspace{.5ex}%No
\item $\enot(A \eif B)\ndststile{}{} \hspace{.5em} \sdtstile{}{} \enot A \eif \enot B$\vspace{.5ex} %No
\item $A \eor B\ndststile{}{} \hspace{.5em} \sdtstile{}{} \enot A \eif B$ \vspace{.5ex}%equivalent. 
\item$(A \eif B) \eif C\ndststile{}{} \hspace{.5em} \sdtstile{}{} A \eif (B \eif C)$\vspace{.5ex} %not equivalent.
\item $A \eiff (B \eiff C)\ndststile{}{} \hspace{.5em} \sdtstile{}{} A \eand (B \eand C)$ \vspace{.5ex}%not equivalent. 
\end{exercises}


\noindent\problempart \label{pr.TT.consistent} Determine whether each set of sentences is consistent or inconsistent using a complete truth table. 

\begin{longtabu}{p{.1\linewidth}p{.9\linewidth}}
\textbf{Example}: & \{$\enot(A \eor B)$, $\enot A \eor B$, $A \eor \enot B$\}\\
\textbf{Answer}: & \begin{tabular}[t]{ccccccccccccccc}
\enot	&	(A	&	\eor	&	B),	&	&	\enot	&	A	&	\eor	&	B,	&	&	A	&	\eor	&	\enot	&	B	&	Consistent \\	
\cline{1-4}	\cline{6-9}	\cline{11-14}
F		&	T	&	T		&	T	&	&		F	&	T	&	T		&	T	&	&	T	&	T		&	F		&	T	&\\
F		&	T	&	T		&	F	&	&		F	&	T	&	F		&	F	&	&	T	&	T		&	T		&	F	&\\
F		&	F	&	T		&	T	&	&		T	&	F	&	T		&	T	&	&	F	&	F		&	F		&	T	&\\
\textbf{T}	\tikz[overlay, shift={(120pt,1ex)}, gray] \draw (0pt,0pt) ellipse (154pt and 2ex);	&	F	&	F		&	F	&	&		T	&	F	&	\textbf{T}		&	F	&	&	F	&	\textbf{T}		&	T		&	F	&\\		
\end{tabular}\\
\end{longtabu}


\begin{exercises}
\item \{$A \eand \enot B$, $\enot(A \eif B)$, $B \eif A$\}\vspace{.5ex} %Consistent

\answer{
\begin{longtabu}{cccccccccccccc} 
A 					 & \eand 		&  \enot & B & & \enot  		& 	 (A	  & 	 \eif	 	 & 	 B)		 & 	 & 	 B	 	 & 	\eif 	 	 & 	A 	 	 & 	 Consistent \\ 
\cline{1-4} \cline{6-9}\cline{11-13} 
T 					 & 	 F	 		&  F	 & T & & F	 		& 	 T	  & 	 T	 	 & 	T 	 	 & 	 & 	 T	 	 & 	 T	 	 & T	 	 	&	  \\ 
T \tikz[overlay, shift={(110pt,1ex)}, red] \draw (0pt,0pt) ellipse (143pt and 2ex); & 	{\color{black}\textbf{T}}	 & T	 & F & & {\color{black}\textbf{T}}	 & 	 T	 & 	 F	 	 & 	 F	 	 & 	 & 	 F	 	 & 	 {\color{black}\textbf{T}}	 	 & 	 T	 	 & 	  \\ 
F	 				 & 	 F	 & 	 F	 & T & 	& 	 F	 & 	 F	 & 	 T	 	 & 	 T	 	 & 	  & 	 T	 	 & 	 F	 	 & 	 F	 	 & 	  \\ 
F	  				& 	 F	 & 	 T	 & 	F&  & 	 F	 & 	 F	 & 	 T	 	 & 	 F	 	 & 	  & 	 F	 	 & 	 T	 	 & 	 F	 	 & 	  \\ 
\end{longtabu}
}
\item \{$A \eor B$, $A \eif \enot A$, $B \eif \enot B\}$ \vspace{.5ex}%inconsistent. 

\answer{
\begin{longtabu}{cccccccccccccc} 
A	 & \eor 	 & B 	 & 	 	 & A 	 & \eif 	 & 	\enot & A 	 & 	 	 & B 	 & \eif 	 & \enot	 & 	B 	 & 	Inconsistent \\ 
\cline{1-3}\cline{5-8} \cline{10-13}
T	 & 	 T	 &T  	 & 	 	 & T	 & 	 F	 & 	F 	 & T 	 & 	 	 & 	T 	 & 	F 	 & 	 F	 & 	T 	 & 	 \\ 
T	& 	 T	 & F 	 & 	 	 & 	T 	 & 	 F	 & 	 F	 & 	 T	 & 	 	 & 	F 	 & 	 T	 & 	 T	 & 	 F	 & 	 \\ 
F	& 	 T	 & 	 T	 & 	 	 & 	F 	 & 	 T	 & 	 T	 & 	F 	 & 	 	 & 	 T	 & 	 F	 & 	 F	 & 	 T	 & 	 \\ 
F	& 	 F	 & 	 F	 & 	 	 & 	 F	 & 	 T	 & 	 T	 & 	 F	 & 	 	 & 	 F	 & 	 T	 & 	 T	 & 	 F	 & 	 \\ 
\end{longtabu}
}

\item \{$\enot(\enot A \eor B) $, $A \eif \enot C$, $A \eif (B \eif C)\}$\vspace{.5ex} %Inconsistent

\answer{
\begin{longtabu}{ccccccccccccccccc} 
\enot & (\enot & A & \eor &B) &  &A  & \eif 	 &\enot 	 &C & 	 & A &\eif 	& (B 	 &\eif 	& C)	 &Consistent \\ 
\cline{1-5}\cline{7-10} \cline{12-16} 
F 	& 	F	 & 	T & T	 & T & 	  & T & F	 & 	 F&T 	 & 	 &T & T	 & T	 &T 	 &T 	 & \\ 
F	& 	F	 & 	T & T	 & T & 	  & T & T	 & 	 T& F	 & 	 &T & F	 & T	 & F	 &F 	 & \\ 
T & 	F 	& 	T & F	 & F & 	  & T & F	 & 	 F& T	 & 	 &T & T	 & F	 & T	 &T 	 & \\ 
\color{black}\textbf{T}	\tikz[overlay, shift={(140pt,1ex)}, red] \draw (0pt,0pt) ellipse (179pt and 2ex);	&  F	 & 	T & F	 & 	F &  & 	T & {\color{black}\textbf{T}}	 & 	 T&F 	& 	 &T & {\color{black}\textbf{T}}	 & F	 & T	&  F 	 & \\ 
 F	& 	T	 & 	F & T	 & 	T &  & 	F & T	 & 	 F& T	 & 	 &F	 & F	 & T	 & T	 &T 	 & \\ 
 F	& 	 T	& 	F & T	 & 	T &  & 	F & T	 & 	T & F & 	 &F	 & T	 & T	 &F 	 &F 	 & \\ 
 F	& 	 T	& 	F & T	 & 	F &  & 	F & T	 & 	F & T	 & 	 &F	 & T	 & F	 & T	 &T 	 & \\ 
 F	& 	 T	& 	F & T	 & 	F &  & 	F & T	 & 	T & F	 & 	 &F	 & T	 & F	 & T	 &F 	 & \\ 
\end{longtabu}
}


\item \{$A \eif B$, $A \eand \enot B$\}\vspace{.5ex} %Inconsistent

\answer{
\begin{longtabu}{ccccccccc}
A	&	\eif	 &	B,	&	&	A	&	\eand	&	\enot	&	B	&	Inconsistent\\
\cline{1-3} \cline{5-8}
T	&	T		&	T	&	&	T	&		F	&		F	&	T	&	\\
T	&	F		&	F	&	&	T	&		T	&		T	&	F	&	\\
F	&	T		&	T	&	&	F	&		F	&		F	&	T	&	\\
F	&	T		&	F	&	&	F	&		F	&		T	&	F	&	\\
\end{longtabu}
}


\item \{$A \eif (B \eif C)$, $(A \eif B) \eif C$, $A \eif C$\}\vspace{.5ex} % consistent. 

\answer{
\begin{longtabu}{cccccccccccccccc}
A	&	\eif	&	(B	&	\eif	&	 C)	&	&	(A	&	\eif	&	B)	&	\eif	&C	&	&	A	&	\eif	&	C	&	Consistent\\
\cline{1-5} \cline{7-11} \cline{13-15}
T	&	T		&	T	&	T		&	T	&	&	T	&	T		&	T	&	T		&	T	&	&	T	&	T		&	T	&	\\
T	&	F		&	T	&	F		&	F	&	&	T	&	T		&	T	&	F		&	F	&	&	T	&	F		&	F	&	\\
T	&	T		&	F	&	T		&	T	&	&	T	&	F		&	F	&	T		&	T	&	&	T	&	T		&	T	&	\\
T	&	T		&	F	&	T		&	F	&	&	T	&	F		&	F	&	T		&	F	&	&	T	&	F		&	F	&	\\
F	&	T		&	T	&	T		&	T	&	&	F	&	T		&	T	&	T		&	T	&	&	F	&	T		&	T	&	\\
F	&	T		&	T	&	F		&	F	&	&	F	&	T		&	T	&	T		&	F	&	&	F	&	T		&	F	&	\\
F	&	T		&	F	&	T		&	T	&	&	F	&	T		&	F	&	T		&	T	&	&	F	&	T		&	T	&	\\
F	&	{\color{black}\textbf{T}}	\tikz[overlay, shift={(125pt,1ex)}, red] \draw (0pt,0pt) ellipse (165pt and 2ex);	&	F	&	T		&	F	&	&	F	&	T		&	F	&	{\color{black}\textbf{T}}		&	F	&	&	F	&	{\color{black}\textbf{T}}		&	T	&	\\
\end{longtabu}
}

\end{exercises}

\noindent\problempart
\label{pr.TT.consistent}
Determine whether each set of sentences is consistent or inconsistent, using a complete truth table. 
\begin{exercises}
\item \{$\enot B$, $A \eif B$, $A$\} \vspace{.5ex}%inconsistent.
\item \{$\enot(A \eor B)$, $A \eiff B$, $B \eif A$\}\vspace{.5ex} %Consistent
\item \{ $A \eor B$, $\enot B$, $\enot B \eif \enot A$\}\vspace{.5ex} %Inconsistent
\item \{$A \eiff B$, $\enot B \eor \enot A$, $A \eif B$\}\vspace{.5ex} %consistent. 
\item \{$(A \eor B) \eor C$, $\enot A \eor \enot B$, $\enot C \eor \enot B$\}\vspace{.5ex} %consistent
\end{exercises}




\noindent\problempart \label{pr.TT.valid} Determine whether each argument is valid or invalid, using a complete truth table. 

\begin{longtabu}{p{.1\linewidth}p{.9\linewidth}}
\textbf{Example}: & $A \eor B$, $C \eif A$, $C \eif B \sdtstile{}{} C$   \\
\textbf{Answer}: & \begin{tabular}[t]{cccccccccccccc}
A	&	\eor	&	B	&	&	C	&	\eif	&	A	&	&	C	&	\eif	&	B	&	&	C	&	Invalid	\\
\cline{1-3} \cline{5-7}	\cline{9-11}	\cline{13-13}
T	&	T		&	T	&	&	T	&	T		&	T	&	&	T	&	T		&	T	&	&	T	&	\\

T \tikz[overlay, shift={(100pt,1ex)}, gray] \draw (0pt,0pt) ellipse (132pt and 2ex);	&	\textbf{T}		&	T	&	&	F	&	\textbf{T}		&	T	&	&	F	&	\textbf{T}		&	T	&	&	\textbf{F}	&	\\

T	&	T		&	F	&	&	T	&	T		&	T	&	&	T	&	F		&	F	&	&	T	&	\\
T	&	T		&	F	&	&	F	&	T		&	T	&	&	F	&	T		&	F	&	&	F	&	\\
F	&	T		&	T	&	&	T	&	F		&	F	&	&	T	&	T		&	T	&	&	T	&	\\
F	&	T		&	T	&	&	F	&	T		&	F	&	&	F	&	T		&	T	&	&	F	&	\\
F	&	F		&	F	&	&	T	&	F		&	F	&	&	T	&	F		&	F	&	&	T	&	\\
F	&	F		&	F	&	&	F	&	T		&	F	&	&	F	&	T		&	F	&	&	F	&	\\						
\end{tabular}\\
\end{longtabu}


\begin{exercises}
\item $A\eif A \sdtstile{}{} A$ \vspace{.5ex}%invalid

\answer{
\begin{longtabu}{cccccc}
A	&	\eif	&	A	&	&	A	&	Invalid \\	
\cline{1-3}	\cline{5-5}
T	&	T		&	T	&	&	T	&		\\

F \tikz[overlay, shift={(33pt,1ex)}, red] \draw (0pt,0pt) ellipse (66pt and 2ex);	&	{\color{black}\textbf{T}}		&	F	&	&	{\color{black}\textbf{F}}	&		\\
\end{longtabu}
}

\item $A\eif B$, $B \sdtstile{}{} A$ %invalid
\answer{
\begin{longtabu}{cccccccc}
A	&	\eif	&	B	&	&	B	&	&	A	&	Invalid \\
\cline{1-3} \cline{5-5}	\cline{7-7}
T	&	T		&	T	&	&	T	&	&	T	&	\\
T	&	F		&	F	&	&	F	&	&	T	&	\\
F \tikz[overlay, shift={(44pt,1ex)}, red] \draw (0pt,0pt) ellipse (88pt and 2ex);	&	{\color{black}\textbf{T}}		&	T	&	&	{\color{black}\textbf{T}}	&	&	{\color{black}\textbf{F}}	&	\\
F	&	T		&	F	&	&	F	&	&	F	&	\\
\end{longtabu}
}

\item $A\eiff B$, $B\eiff C \sdtstile{}{}A\eiff C$ %valid

\answer{
\begin{longtabu}{cccccccccccc}
A	&	\eiff	&	B	&	&	B	&	\eiff	&	C	&	&	A	&	\eiff	&	C 	&	Valid \\
\cline{1-3} \cline{5-7} \cline{9-11}
T	&	T		&	T	&	&	T	&	T		&	T	&	&	T	&	T		&	T	&	\\
T	&	T		&	T	&	&	T	&	F		&	F	&	&	T	&	F		&	F	&	\\	
T	&	F		&	F	&	&	F	&	F		&	T	&	&	T	&	T		&	T	&	\\
T	&	F		&	F	&	&	F	&	T		&	F	&	&	T	&	F		&	F	&	\\
F	&	F		&	T	&	&	T	&	T		&	T	&	&	F	&	F		&	T	&	\\
F	&	F		&	T	&	&	T	&	F		&	F	&	&	F	&	T		&	F	&	\\	
F	&	T		&	F	&	&	F	&	F		&	T	&	&	F	&	F		&	T	&	\\
F	&	T		&	F	&	&	F	&	T		&	F	&	&	F	&	T		&	F	&	\\
\end{longtabu}
}

\item $A \eif B$, $A \eif C\sdtstile{}{}B \eif C$ %invalid. 

\answer{
\begin{longtabu}{cccccccccccc}
A	&	\eif	&	B	&	&	A	&	\eif	&	C	&	&	B	&	\eif	&	C	&	Invalid \\
\cline{1-3} \cline{5-7} \cline{9-11}
T	&	T		&	T	&	&	T	&	T		&	T	&	&	T	&	T		&	T	&	\\
T	&	T		&	T	&	&	T	&	F		&	F	&	&	T	&	F		&	F	&	\\
T	&	F		&	F	&	&	T	&	T		&	T	&	&	F	&	T		&	T	&	\\
T	&	F		&	F	&	&	T	&	F		&	F	&	&	F	&	T		&	F	&	\\
F	&	T		&	T	&	&	F	&	T		&	T	&	&	T	&	T		&	T	&	\\	

F \tikz[overlay, shift={(100pt,1ex)}, red] \draw (0pt,0pt) ellipse (132pt and 2ex);	&	{\color{black}\textbf{T}}		&	T	&	&	F	&	{\color{black}\textbf{T}}		&	F	&	&	T	&	{\color{black}\textbf{F}}		&	F	&	\\

F	&	T		&	F	&	&	F	&	T		&	T	&	&	F	&	T		&	T	&	\\
F	&	T		&	F	&	&	F	&	T		&	F	&	&	F	&	T		&	F	&	\\		
\end{longtabu}
}

\item $A \eif B$, $B \eif A\sdtstile{}{}A \eiff B$ %valid. 

\answer{
\begin{longtabu}{cccccccccccc}
A	&	\eif	&	B	&	&	B	&	\eif	&	A	&	&	A	&	\eiff	&	B	&	Valid	\\
\cline{1-3} \cline{5-7} \cline{9-11}
T	&	T		&	T	&	&	T	&	T		&	T	&	&	T	&	T		&	T	&	\\
T	&	F		&	F	&	&	F	&	T		&	T	&	&	T	&	F		&	F	&	\\
F	&	T		&	T	&	&	T	&	F		&	F	&	&	F	&	F		&	T	&	\\
F	&	T		&	F	&	&	F	&	T		&	F	&	&	F	&	T		&	F	&	\\		
\end{longtabu}
}

\end{exercises}

\noindent\problempart
\label{pr.TT.valid}
Determine whether each argument is valid or invalid, using a complete truth table. 
\begin{exercises}
\item $A\eor\bigl[A\eif(A\eiff A)\bigr] \sdtstile{}{} A $\vspace{.5ex}%invalid
\item $A\eor B$, $B\eor C$, $\enot B \sdtstile{}{}A \eand C$\vspace{.5ex} %valid
\item $A \eif B$, $\enot A\sdtstile{}{}\enot B$ \vspace{.5ex}%invalid
\item $A$, $B\sdtstile{}{}\enot(A\eif \enot B)$ \vspace{.5ex}%valid
\item $\enot(A \eand B)$, $A \eor B$, $A \eiff B\sdtstile{}{}C$ \vspace{.5ex}%valid 
\end{exercises}


% *********************************************
% *   Partial Truth Tables							*
% *********************************************
\section{Partial Truth Tables}

In order to show that a sentence is a tautology, we need to show that it is T on every row. So we need a complete truth table. To show that a sentence is \emph{not} a tautology, however, we only need one line: a line on which the sentence is F. Therefore, in order to show that something is not a tautology, it is enough to provide a one-line \emph{partial truth table}---regardless of how many sentence letters the sentence might have in it.

Consider, for example, the sentence $(U \eand T) \eif (S \eand W)$. We want to show that it is \emph{not} a tautology by providing a partial truth table. To begin, we fill in F for the entire sentence, the reverse of how we started when we were doing complete truth tables.

\begin{center}
\begin{tabu}{c|c|c|c|@{\TTon}*{7}{c}@{\TToff}}
$S$&$T$&$U$&$W$&$(U$&\eand&$T)$&\eif  \tikz[overlay, shift={(-1ex,-6pt)}, gray] \draw (0pt,0pt) ellipse (2ex and 18pt);  &$(S$&\eand&$W)$\\
\hline
   &   &   &   &    &    &    &F&    &    &   
\end{tabu}
\end{center}

 The main connective of the sentence is a conditional. In order for the conditional to be false, the antecedent must be true (T) and the consequent must be false (F). So we fill these in on the table:

\begin{center}
\begin{tabu}{c|c|c|c|@{\TTon}*{7}{c}@{\TToff}}
$S$&$T$&$U$&$W$&$(U$&\eand&$T)$&\eif  \tikz[overlay, shift={(-1ex,-6pt)}, gray] \draw (0pt,0pt) ellipse (2ex and 18pt);  &$(S$&\eand&$W)$\\
\hline
   &   &   &   &    &  T  &    &F&    &   F &   
\end{tabu}
\end{center}

In order for the $(U\eand T)$ to be true, both $U$ and $T$ must be true.

\begin{center}
\begin{tabu}{c|c|c|c|@{\TTon}*{7}{c}@{\TToff}}
$S$&$T$&$U$&$W$&$(U$&\eand&$T)$&\eif  \tikz[overlay, shift={(-1ex,-6pt)}, gray] \draw (0pt,0pt) ellipse (2ex and 18pt);  &$(S$&\eand&$W)$\\
\hline
   & T & T &   &  T &  T  & T  &F&    &   F &   
\end{tabu}
\end{center}

Now we just need to make $(S\eand W)$ false. To do this, we need to make at least one of $S$ and $W$ false. We can make both $S$ and $W$ false if we want. All that matters is that the whole sentence turns out false on this line. Making an arbitrary decision, we finish the table in this way:

\begin{center}
\begin{tabu}{c|c|c|c|@{\TTon}*{7}{c}@{\TToff}}
$S$&$T$&$U$&$W$&$(U$&\eand&$T)$&\eif  \tikz[overlay, shift={(-1ex,-6pt)}, gray] \draw (0pt,0pt) ellipse (2ex and 18pt);  &$(S$&\eand&$W)$\\
\hline
 F & T & T & F &  T &  T  & T  &F&  F &   F & F  
\end{tabu}
\end{center}

Showing that something is a contradiction requires a complete truth table. Showing that something is \emph{not} a contradiction requires only a one-line partial truth table, where the sentence is true on that one line.

A sentence is contingent if it is neither a tautology nor a contradiction. So showing that a sentence is contingent requires a \emph{two-line} partial truth table: The sentence must be true on one line and false on the other. For example, we can show that the sentence above is contingent with this truth table:
\begin{center}
\begin{tabu}{c|c|c|c|@{\TTon}*{7}{c}@{\TToff}}
$S$&$T$&$U$&$W$&$(U$&\eand&$T)$&\eif  \tikz[overlay, shift={(-1ex,-12pt)}, gray] \draw (0pt,0pt) ellipse (2ex and 24pt);  &$(S$&\eand&$W)$\\
\hline
 F & T & T & F &  T &  T  & T  &F&  F &   F & F\\
 F & T & F & F &  F &  F  & T  &T&  F &   F & F
\end{tabu}
\end{center}
Note that there are many combinations of truth values that would have made the sentence true, so there are many ways we could have written the second line.

Showing that a sentence is \emph{not} contingent requires providing a complete truth table, because it requires showing that the sentence is a tautology or that it is a contradiction.  If you do not know whether a particular sentence is contingent, then you do not know whether you will need a complete or partial truth table. You can always start working on a complete truth table. If you complete rows that show the sentence is contingent, then you can stop. If not, then complete the truth table. Even though two carefully selected rows will show that a contingent sentence is contingent, there is nothing wrong with filling in more rows.

Showing that two sentences are logically equivalent requires providing a complete truth table. Showing that two sentences are \emph{not} logically equivalent requires only a one-line partial truth table: Make the table so that one sentence is true and the other false.

Showing that a set of sentences is consistent requires providing one row of a truth table on which all of the sentences are true. The rest of the table is irrelevant, so a one-line partial truth table will do. Showing that a set of sentences is inconsistent, on the other hand, requires a complete truth table: You must show that on every row of the table at least one of the sentences is false.

Showing that an argument is valid requires a complete truth table. Showing that an argument is \emph{invalid} only requires providing a one-line truth table: If you can produce a line on which the premises are all true and the conclusion is false, then the argument is invalid.

\begin{table}
\begin{mdframed}[style=mytablebox]
\begin{center}
\begin{tabu}{X[1,l,b] X[1,l,b] X[1,l,b]}
\tabulinesep=1ex
\underline{Property}	& Truth table required \newline \underline{to show presence}		&	Truth table required \newline \underline{to show absence} \\ 
being a tautology		& complete 													 		& one-line partial \\ 
being a contradiction 	& complete 													 		& one-line partial \\ 
contingency				& two-line partial 													& complete truth \\ 
equivalence				& complete 													 		& one-line partial\\ 
consistency				& one-line partial 													& complete \\ 
validity					& complete 															& one-line partial \\ 
\end{tabu}
\end{center}
\end{mdframed}
\caption{Complete or partial truth tables to test for different properties}
\label{table.CompleteVsPartial}
\end{table}

Table \ref{table.CompleteVsPartial} summarizes when a complete truth table is required and when a partial truth table will do. 

%\section{The material conditional}
%\label{MaterialConditional}

%The material conditional has some odd properties. For one thing, it does not require that the antecedent and consequent are related in any way.

%contradiction in the antecedent

%tautology in the consequent


%%%%%%%%%%%%%%%%% practice problems 

\practiceproblems
\noindent\problempart \label{pr.TT.TTorC} Determine whether each sentence is a tautology, a contradiction, or a contingent sentence. Justify your answer with a complete or partial truth table where appropriate.


\begin{exercises}
\item  $A \eif \enot A$ \vspace{.5ex}							

\answer{

\begin{longtabu}{cccc}
A&\eif \tikz[overlay, shift={(-1ex,-12pt)}, red] \draw (0pt,0pt) ellipse (2ex and 24pt);&\enot&A\\\hline
T&F&F&T\\
F&T&T&F
\end{longtabu}
Contingent	 \vspace{6pt}
}
%	T letter, 2 connectives

\item $A \eif (A \eand (A \eor B))$ \vspace{.5ex}	

\answer{


\begin{longtabu}{ccc@{}ccc@{}ccc@{}c@{}c}
A&\eif \tikz[overlay, shift={(-1ex,-30pt)}, red] \draw (0pt,0pt) ellipse (2ex and 44pt);
 &(&A&\eand&(&A&\eor&B&)&)\\\hline
T&T&&T&T&&T&T&T&&\\
T&T&&T&T&&T&T&F&&\\
F&T&&F&F&&F&T&T&&\\
F&T&&F&F&&F&F&F&&
\end{longtabu}


Tautology \vspace{6pt}
}
%			2 letters, 3 connectives

\item $(A \eif B) \eiff (B \eif A)$ 	\vspace{.5ex}				%

\answer{
 
\begin{longtabu}{ccccccc}
(A	&	\eif 	&	B) 	&	\eiff \tikz[overlay, shift={(-1ex,-12pt)}, red] \draw (0pt,0pt) ellipse (2ex and 24pt);	&	(B 	&	\eif 	&	A) \\ \hline
T	&	T		&	T	&	T		&	T	&	T		&	T	\\
T	&	F		&	F	&	F		&	F	&	T		&	T	\\
\end{longtabu}
Contingent \vspace{6pt}
}
%		2 letters, 3 connectives

\item $A \eif \enot(A \eand (A \eor B)) $	\vspace{.5ex}	

\answer{
 
\begin{longtabu}{cccccccc}
A	&	\eif	&	 \enot	&	(A	&	\eand	&	(A	&	\eor	&	B)) \\ \hline
	&			&			&		&			&		&			&	\\
	&			&			&		&			&		&			&	\\				
\end{longtabu}
}


%
% 2 letters, 4 connectives

\item $\enot B \eif [(\enot A \eand A) \eor B]$\vspace{.5ex} 

\answer{
 
\begin{longtabu}{ccccccc}

\end{longtabu}
}

%{\color{red}
%$
%\begin{array}{cc|cccc@{}c@{}cccc@{}ccc@{}c}
%a&b&\enot&b&\rightarrow&(&(&\enot&a&\eand&a&)&\eor&b&)\\\hline
%T&T&F&T&\mathbf{T}&&&F&T&F&T&&T&T&\\
%T&F&T&F&\mathbf{F}&&&F&T&F&T&&F&F&\\
%F&T&F&T&\mathbf{T}&&&T&F&F&F&&T&T&\\
%F&F&T&F&\mathbf{F}&&&T&F&F&F&&F&F&
%\end{array}
%$
%Contingent	 \vspace{6pt}
%
%}
%	2 letters, 5 connectives

\item $\enot(A \eor B) \eiff (\enot A \eand \enot B)$ \vspace{.5ex}

\answer{
 
\begin{longtabu}{ccccccc}

\end{longtabu}
}

%{\color{red}
%$
%\begin{array}{cc|cc@{}ccc@{}ccc@{}ccccc@{}c}
%a&b&\enot&(&a&\eor&b&)&\leftrightarrow&(&\enot&a&\eand&\enot&b&)\\\hline
%T&T&F&&T&T&T&&\mathbf{T}&&F&T&F&F&T&\\
%T&F&F&&T&T&F&&\mathbf{T}&&F&T&F&T&F&\\
%F&T&F&&F&T&T&&\mathbf{T}&&T&F&F&F&T&\\
%F&F&T&&F&F&F&&\mathbf{T}&&T&F&T&T&F&
%\end{array}
%$
%
%Tautology \vspace{6pt}
%}
%2 letters, 6 connectives

\item $[(A \eand B) \eand C] \eif B$\vspace{.5ex}		

\answer{
 
\begin{longtabu}{ccccccc}

\end{longtabu}
}
					
%
%{\color{red}
%$
%\begin{array}{ccc|c@{}c@{}ccc@{}ccc@{}ccc}
%a&b&c&(&(&a&\eand&b&)&\eand&c&)&\rightarrow&b\\\hline
%T&T&T&&&T&T&T&&T&T&&\mathbf{T}&T\\
%T&T&F&&&T&T&T&&F&F&&\mathbf{T}&T\\
%T&F&T&&&T&F&F&&F&T&&\mathbf{T}&F\\
%T&F&F&&&T&F&F&&F&F&&\mathbf{T}&F\\
%F&T&T&&&F&F&T&&F&T&&\mathbf{T}&T\\
%F&T&F&&&F&F&T&&F&F&&\mathbf{T}&T\\
%F&F&T&&&F&F&F&&F&T&&\mathbf{T}&F\\
%F&F&F&&&F&F&F&&F&F&&\mathbf{T}&F
%\end{array}
%$
%
%Tautology \vspace{6pt}
%}
%
%3 letters, 3 connectives

\item $\enot\bigl[(C\eor A) \eor B\bigr]$\vspace{.5ex} 						

\answer{
 
\begin{longtabu}{ccccccc}

\end{longtabu}
}


%
%{\color{red}
%$
%\begin{array}{ccc|cc@{}c@{}ccc@{}ccc@{}c}
%a&b&c&\enot&(&(&c&\eor&a&)&\eor&b&)\\\hline
%T&T&T&\mathbf{F}&&&T&T&T&&T&T&\\
%T&T&F&\mathbf{F}&&&F&T&T&&T&T&\\
%T&F&T&\mathbf{F}&&&T&T&T&&T&F&\\
%T&F&F&\mathbf{F}&&&F&T&T&&T&F&\\
%F&T&T&\mathbf{F}&&&T&T&F&&T&T&\\
%F&T&F&\mathbf{F}&&&F&F&F&&T&T&\\
%F&F&T&\mathbf{F}&&&T&T&F&&T&F&\\
%F&F&F&\mathbf{T}&&&F&F&F&&F&F&
%\end{array}
%$
%
%Contingent \vspace{6pt}
%
%}
%	 	3 letters, 3 connectives

\item $\bigl[(A\eand B) \eand\enot(A\eand B)\bigr] \eand C$ \vspace{.5ex}	


\answer{
 
\begin{longtabu}{ccccccc}

\end{longtabu}
}

%
%{\color{red}
%$
%\begin{array}{ccc|c@{}c@{}ccc@{}cccc@{}ccc@{}c@{}ccc}
%a&b&c&(&(&a&\eand&b&)&\eand&\enot&(&a&\eand&b&)&)&\eand&c\\\hline
%T&T&T&&&T&T&T&&F&F&&T&T&T&&&\mathbf{F}&T\\
%T&T&F&&&T&T&T&&F&F&&T&T&T&&&\mathbf{F}&F\\
%T&F&T&&&T&F&F&&F&T&&T&F&F&&&\mathbf{F}&T\\
%T&F&F&&&T&F&F&&F&T&&T&F&F&&&\mathbf{F}&F\\
%F&T&T&&&F&F&T&&F&T&&F&F&T&&&\mathbf{F}&T\\
%F&T&F&&&F&F&T&&F&T&&F&F&T&&&\mathbf{F}&F\\
%F&F&T&&&F&F&F&&F&T&&F&F&F&&&\mathbf{F}&T\\
%F&F&F&&&F&F&F&&F&T&&F&F&F&&&\mathbf{F}&F
%\end{array}
%$
%
%Contradiction \vspace{6pt}
%
%}
%
%% 	3 letters, 5 connectives
%
\item $(A \eand B) ]\eif[(A \eand C) \eor (B \eand D)]$ \vspace{.5ex}		


\answer{
 
\begin{longtabu}{ccccccc}

\end{longtabu}
}


%
%{\color{red}
%$
%\begin{array}{cccc|c@{}c@{}ccc@{}c@{}ccc@{}c@{}ccc@{}ccc@{}ccc@{}c@{}c}
%a&b&c&d&(&(&a&\eand&b&)&)&\eif&(&(&a&\eand&c&)&\eor&(&b&\eand&d&)&)\\\hline
%T&T&T&T&&&T&T&T&&&\mathbf{T}&&&T&T&T&&T&&T&T&T&&\\
%T&T&F&F&&&T&T&T&&&\mathbf{F}&&&T&F&F&&F&&T&F&F&&\\
%\end{array}
%$
%
%Contingent \vspace{6pt}
%}
%
%	4 letters, 5 connectives
\end{exercises}

\noindent\problempart
\label{pr.TT.TTorC}
Determine whether each sentence is a tautology, a contradiction, or a contingent sentence. Justify your answer with a complete or partial truth table where appropriate.
\begin{exercises}
\item  $\enot (A \eor A)$\vspace{.5ex}							%	Contradiction		1 letter, 2 connectives
\item $(A \eif B) \eor (B \eif A)$\vspace{.5ex}					%	Tautology			2 letters, 2 connectives
\item $[(A \eif B) \eif A] \eif A$\vspace{.5ex}					%	Tautology			2 letters, 3 connectives
\item $\enot[( A \eif B) \eor (B \eif A)]$\vspace{.5ex}			%	Contradiction		2 letters, 4 connectives
\item $(A \eand B) \eor (A \eor B)$\vspace{.5ex} 				%	Contingent		2 letters, 5 connectives
\item $\enot(A\eand B) \eiff A$\vspace{.5ex} 					%contingent			2 letters, 3 connectives
\item $A\eif(B\eor C)$\vspace{.5ex} 							%contingent			3 letters, 2 connectives
\item $(A \eand\enot A) \eif (B \eor C)$\vspace{.5ex} 			%tautology			3 letters, 4 connectives 
\item $(B\eand D) \eiff [A \eiff(A \eor C)]$\vspace{.5ex}			%contingent			4 letters, 4 connectives
\item $\enot[(A \eif B) \eor (C \eif D)]$\vspace{.5ex} 			% Contingent. 		4 letters, 4 connectives
\end{exercises}


\noindent\problempart
\label{pr.TT.equiv}
Determine whether each the following statements of equivalence are true or false using complete truth tables. If the two sentences really are logically equivalent, write "Logically equivalent." Otherwise write, "Not logically equivalent." 
\begin{exercises}
\item $A\ndststile{}{} \hspace{.5em} \sdtstile{}{}\enot A$\vspace{.5ex} 											%No		1 letter, 1 connective, matching
\item $A\eif A\ndststile{}{} \hspace{.5em} \sdtstile{}{}A \eiff A$\vspace{.5ex} 									%No		1 letter, 1 connectives, matching	
\item $A	\eand (B	\eand C)\ndststile{}{} \hspace{.5em} \sdtstile{}{} A \eand \enot A$\vspace{.5ex}  					%No		2 letters, 4 connectives, matching
\item $A \eand \enot A\ndststile{}{} \hspace{.5em} \sdtstile{}{}\enot B \eiff B$ \vspace{.5ex}						%Yes		2 letters, 4 connectives, matching
\item $\enot(A \eif B)\ndststile{}{} \hspace{.5em} \sdtstile{}{}\enot A \eif \enot B$\vspace{.5ex}					%No		2 letters, 5 connectives, matching
\item $A \eiff B\ndststile{}{} \hspace{.5em} \sdtstile{}{}\enot[(A \eif B) \eif \enot (B\eif A)]$\vspace{.5ex}			%yes		2 letters, 6 connectives, matching
\item $(A \eand B) \eif (\enot A \eor \enot B)\ndststile{}{} \hspace{.5em} \sdtstile{}{} \enot(A\eand B)$\vspace{.5ex}	%Yes		2 letters, 7 connectives, matching
\item $[(A \eor B) \eor C]\ndststile{}{} \hspace{.5em} \sdtstile{}{}[A \eor (B \eor C)]$\vspace{.5ex} 				%Yes		3 letters, 4 connectives, matching
\item $ (Z \eand (\enot R \eif O))\ndststile{}{} \hspace{.5em} \sdtstile{}{} \enot (R \eif \enot O) $\vspace{.5ex}		%			3 letters, 6 connectives, not matching	

\end{exercises}

\noindent\problempart
Determine whether each the following statements of equivalence are true or false using complete truth tables. If the two sentences really are logically equivalent, write "Logically equivalent." Otherwise write, "Not logically equivalent." 
\begin{exercises}
\item $A\ndststile{}{} \hspace{.5em} \sdtstile{}{}A \eor A$\vspace{.5ex} 												%Yes		1 letter, 1 connective, matching
\item $A\ndststile{}{} \hspace{.5em} \sdtstile{}{}A \eand A$\vspace{.5ex} 												%Yes		1 letter, 1 connective, matching
\item $A \eor \enot B\ndststile{}{} \hspace{.5em} \sdtstile{}{}A\eif B$\vspace{.5ex} 									%No		2 letters, 3 connectives, matching
\item $(A \eif B)\ndststile{}{} \hspace{.5em} \sdtstile{}{}(\enot B \eif \enot A)$\vspace{.5ex} 							%Yes		2 letters, 4 connectives, matching
\item $\enot(A \eand B)\ndststile{}{} \hspace{.5em} \sdtstile{}{}\enot A \eor \enot B$ \vspace{.5ex}						%Yes		2 letters, 5 connectives, matching
\item $ ((U \eif (X \eor X)) \eor U) \ndststile{}{} \hspace{.5em} \sdtstile{}{} \enot (X \eand (X \eand U)) $\vspace{.5ex}	% 			2 letters, 6 connectives, matching
\item $ ((C \eand (N \eiff C)) \eiff C) \ndststile{}{} \hspace{.5em} \sdtstile{}{} (\enot \enot \enot N \eif C) $\vspace{.5ex}	% 			2 letters, 7 connectives, matching
\item $[(A \eor B) \eand C]\ndststile{}{} \hspace{.5em} \sdtstile{}{}[A \eor (B \eand C)]$\vspace{.5ex} 					%No		3 letters, 4 connectives, matching
\item $((L \eand C) \eand I)\ndststile{}{} \hspace{.5em} \sdtstile{}{}L \eor C$\vspace{.5ex}								%No		3 letters, not matching	
\end{exercises}


\noindent\problempart
\label{pr.TT.consistent}
Determine whether each set of sentences is consistent or inconsistent. Justify your answer with a complete or partial truth table where appropriate.
\begin{exercises}
\item \{$A\eif A$, $\enot A \eif \enot A$, $A\eand A$, $A\eor A$\} \vspace{.5ex}%consistent
\item \{$A \eif \enot A$, $\enot A \eif A$\}\vspace{.5ex}%inconsistent. 
\item \{$A\eor B$, $A\eif C$, $B\eif C$\}\vspace{.5ex} %consistent
\item \{$A \eor B$, $A \eif C$, $B \eif C$, $\enot C$\}\vspace{.5ex} %	Inconsistent
\item \{$B\eand(C\eor A)$, $A\eif B$, $\enot(B\eor C)$\}\vspace{.5ex}  %inconsistent
\item \{$(A \eiff B) \eif B$,  $B \eif \enot (A \eiff B)$, $A \eor B$\} \vspace{.5ex} %	Consistent
\item \{$A\eiff(B\eor C)$, $C\eif \enot A$, $A\eif \enot B$\}\vspace{.5ex} %consistent
\item  \{$A \eiff B$,  $\enot B \eor \enot A$,  $A \eif  B$\} \vspace{.5ex}% Consistent
\item \{$A \eiff B$, $A \eif C$, $B \eif D$, $\enot(C \eor D)$\}\vspace{.5ex} %consitent
\item \{$\enot (A \eand \enot B)$,  $B \eif \enot A$, $\enot B$ \} \vspace{.5ex} %Consistent
\end{exercises}

\noindent\problempart
\label{pr.TT.consistent}
Determine whether each set of sentences is consistent or inconsistent. Justify your answer with a complete or partial truth table where appropriate.
\begin{exercises}
\item \{$A \eand B$, $C\eif \enot B$, $C$\} \vspace{.5ex}%inconsistent
\item \{$A\eif B$, $B\eif C$, $A$, $\enot C$\}\vspace{.5ex} %inconsistent
\item \{$A \eor B$, $B\eor C$, $C\eif \enot A$\}\vspace{.5ex} %consistent
\item \{$A$, $B$, $C$, $\enot D$, $\enot E$, $F$\}\vspace{.5ex} %consistent
\item \{$A \eand (B \eor C)$, $\enot(A \eand C)$, $\enot(B \eand C)$\} \vspace{.5ex}%consistent
\item \{$A \eif B$, $B \eif C$, $\enot(A \eif C)$\} \vspace{.5ex} %inconsistent

%\begin{tabular}{ccc|ccc|ccccc}
%A 	&\eif	&B 	&B 	&\eif 	&C 	&\enot 	&(A 	&\eif 	&C) 	&Inconsistent\\
%\cline{1-10} 
%T 	&T 	&T 	&T 	&T 	&T 	&F 		&T 	&T 	&T 	& \\
%T&T&T&T&F&F&T&T&F&F&\\
%T&F&F&F&T&T&F&T&T&T&\\
%T&F&F&F&T&F&T&T&F&F&\\
%F&T&T&T&T&T&F&F&T&T&\\
%F&T&T&T&F&F&F&F&T&F&\\
%F&T&F&F&T&T&F&F&T&T&\\
%F&T&F&F&T&F&F&F&T&F&\\
%\end{tabular}

\end{exercises}


\noindent\problempart Determine whether each argument is valid or invalid. Justify your answer with a complete or partial truth table where appropriate.
\label{pr.TT.valid} 
\begin{exercises}

\item $A\eif(A\eand\enot A)\sdtstile{}{}\enot A$% valid

\answer{
 \begin{longtabu}{ccccccc|cc}
A	&	\eif	&	(A	&	\eand	&	\enot	&	A)	&	&	\enot	&	A	\\ \hline
T	&	F		&	T	&	F		&	F		&	T	&	&		F	&	T\\ 
F	&	T		&	F	&	F		&	T		&	F	&	&		T	&	F\\ 
\end{longtabu}
Valid
}



\item $A \eor B$, $A \eif B$, $B \eif A \sdtstile{}{} A \eiff B$  % Valid

\answer{
 
\begin{longtabu}{cccc|cccc|cccc|ccc}
A	&	\eor 	&	B	&		&	 A	&	\eif	&	B	&		&	B	&	\eif	&	A	&		&	A	&	\eiff	&	B\\ \hline
T	&			&	T	&		&	T	&			&	T	&		&	T	&			&	T	&		&	T	&			&	T	\\
T	&			&	F	&		&	T	&			&	F	&		&	F	&			&	T	&		&	T	&			&	F	\\
F	&			&	T	&		&	F	&			&	T	&		&	T	&			&	F	&		&	F	&			&	T	\\
F	&			&	F	&		&	F	&			&	F	&		&	F	&			&	F	&		&	F	&			&	F	\\			
\end{longtabu}
}


\item $A\eor(B\eif A)\sdtstile{}{}\enot A \eif \enot B$ %valid

\answer{
 
\begin{longtabu}{ccccccc}

\end{longtabu}
}


\item $A \eor B$, $A \eif B$, $ B \eif A \sdtstile{}{} A \eand B$ %valid

\answer{
 
\begin{longtabu}{ccccccc}

\end{longtabu}
}


\item $(B\eand A)\eif C$, $(C\eand A)\eif B\sdtstile{}{}(C\eand B)\eif A$ % invalid

\answer{
 
\begin{longtabu}{ccccccc}

\end{longtabu}
}


\item $\enot (\enot A \eor \enot B)$, $A \eif \enot C \sdtstile{}{} A \eif (B \eif C)$ % invalid.

\answer{
 
\begin{longtabu}{ccccccc}

\end{longtabu}
}


\item $A \eand (B \eif C)$, $\enot C \eand (\enot B \eif \enot A)\sdtstile{}{}C \eand \enot C$ % valid

\answer{
 
\begin{longtabu}{ccccccc}

\end{longtabu}
}


\item $A \eand B$, $\enot A \eif \enot C$, $B \eif \enot D \sdtstile{}{} A \eor B$ % Invalid

\answer{
 
\begin{longtabu}{ccccccc}

\end{longtabu}
}


\item $A \eif B\sdtstile{}{}(A \eand B) \eor (\enot A \eand \enot B)$ % invalid

\answer{
 
\begin{longtabu}{ccccccc}

\end{longtabu}
}


\item $\enot A \eif B$,$ \enot B \eif C $,$ \enot C \eif A \sdtstile{}{} \enot A \eif (\enot B \eor \enot C) $% Invalid

\answer{
 
\begin{longtabu}{ccccccc}

\end{longtabu}
}


\end{exercises}

\noindent\problempart Determine whether each argument is valid or invalid. Justify your answer with a complete or partial truth table where appropriate.
\label{pr.TT.valid} 
\begin{exercises}
\item $A\eiff\enot(B\eiff A)\sdtstile{}{}A$ % invalid

\answer{
 
\begin{longtabu}{ccccccc}

\end{longtabu}
}


\item $A\eor B$, $B\eor C$, $\enot A\sdtstile{}{}B \eand C$ % invalid

\answer{
 
\begin{longtabu}{ccccccc}

\end{longtabu}
}


\item $A \eif C$, $E \eif (D \eor B)$, $B \eif \enot D\sdtstile{}{}(A \eor C) \eor (B \eif (E \eand D))$ % invalid

\answer{
 
\begin{longtabu}{ccccccccccccccccccccc}

\end{longtabu}
}


\item $A \eor B$, $C \eif A$, $C \eif B\sdtstile{}{}A \eif (B \eif C)$ % invalid

\answer{
 
\begin{longtabu}{ccccccc}

\end{longtabu}
}


\item $A \eif B$, $\enot B \eor A\sdtstile{}{}A \eiff B$ % valid

\answer{
 
\begin{longtabu}{ccccccc}

\end{longtabu}
}


\end{exercises}

\noindent\problempart
\label{pr.TT.concepts}
Answer each of the questions below and justify your answer.
\begin{exercises}
\item Suppose that \script{A} and \script{B} are logically equivalent. What can you say about $\script{A}\eiff\script{B}$?
%\script{A} and \script{B} have the same truth value on every line of a complete truth table, so $\script{A}\eiff\script{B}$ is true on every line. It is a tautology.
\item Suppose that $(\script{A}\eand\script{B})\eif\script{C}$ is contingent. What can you say about the argument ``\script{A}, \script{B}, $\therefore$\ \script{C}''?
%The sentence is false on some line of a complete truth table. On that line, \script{A} and \script{B} are true and \script{C} is false. So the argument is invalid.
\item Suppose that $\{\script{A},\script{B}, \script{C}\}$ is inconsistent. What can you say about $(\script{A}\eand\script{B}\eand\script{C})$?
%Since there is no line of a complete truth table on which all three sentences are true, the conjunction is false on every line. So it is a contradiction.
\item Suppose that \script{A} is a contradiction. What can you say about the argument \{\script{A}, \script{B}\} $\sdtstile{}{}$  \script{C}?
%Since \script{A} is false on every line of a complete truth table, there is no line on which \script{A} and \script{B} are true and \script{C} is false. So the argument is valid.
\item Suppose that \script{C} is a tautology. What can you say about the argument \{\script{A}, \script{B}\} $\sdtstile{}{}$ \script{C}''?
%Since \script{C} is true on every line of a complete truth table, there is no line on which \script{A} and \script{B} are true and \script{C} is false. So the argument is valid.
%\item Suppose that \script{A} and \script{B} are logically equivalent. What can you say about $(\script{A}\eor\script{B})$?
%Not much. $(\script{A}\eor\script{B})$ is a tautology if \script{A} and \script{B} are tautologies; it is a contradiction if they are contradictions; it is contingent if they are contingent.
\item Suppose that \script{A} and \script{B} are \emph{not} logically equivalent. What can you say about $(\script{A}\eor\script{B})$?
%\script{A} and \script{B} have different truth values on at least one line of a complete truth table, and $(\script{A}\eor\script{B})$ will be true on that line. On other lines, it might be true or false. So $(\script{A}\eor\script{B})$ is either a tautology or it is contingent; it is \emph{not} a contradiction.
\end{exercises}

% *********************************************
% *   Expressive Completeness	      						*
% *********************************************

\section{Expressive Completeness}
\label{sec:expressive_completeness}

We could leave the biconditional (\eiff) out of the language. If we did that, we could still write ``$A\eiff B$'' so as to make sentences easier to read, but that would be shorthand for $(A\eif B) \eand (B\eif A)$. The resulting language would be formally equivalent to SL, since $A\eiff B$ and $(A\eif B) \eand (B\eif A)$ are logically equivalent in SL. If we valued formal simplicity over expressive richness, we could replace more of the connectives with notational conventions and still have a language equivalent to SL. 

There are a number of equivalent languages with only two connectives. You could do logic with only the negation and the material conditional. Alternately you could just have the negation and the disjunction. You will be asked to prove that these things are true in the last problem set. You could even have a language with only one connective, if you designed the connective right. The \emph{Sheffer stroke} is a logical connective with the following characteristic truth table:
\begin{center}
\begin{tabular}{c|c|c}
\script{A} & \script{B} & \script{A}$|$\script{B}\\
\hline
T & T & F\\
T & F & T\\
F & T & T\\
F & F & T
\end{tabular}
\end{center}
The Sheffer stroke has the unique property that it is the only connective you need to have a complete system of logic. You will be asked to prove that this is true in the last problem set also.  


%\fix{Summary of test conditions}

\practiceproblems
\noindent\problempart
\begin{exercises}
\item In section \ref{sec:expressive_completeness}, we said that you could have a language that only used the negation and the material conditional. Prove that this is true by writing sentences that are logically equivalent to each of the following using only parentheses, sentence letters, negation (\enot), and the material conditional (\eif).
\begin{enumerate}
\item $A\eor B$
%$\enot A \eif B$
\item $A\eand B$
%$\enot(A \eif \enot B)$
\item $A\eiff B$
%$\enot [(A\eif B) \eif \enot(B\eif A)]$
\end{enumerate}

\item We also said in section 3.5 that you could have a language which used only the negation and the disjunction. Show this: Using only parentheses, sentence letters, negation (\enot), and disjunction (\eor), write sentences that are logically equivalent to each of the following.
\begin{enumerate}

\item $A \eand B$
%$\enot(\enot A \eor \enot B)$
\item $A \eif B$
%$\enot A \eor B$
\item $A \eiff B$
%$\enot(\enot A \eor \enot B) \eor \enot(A \eor B)$
\end{enumerate}

\item Write a sentence using the connectives of SL that is logically equivalent to $(A|B)$.
\item Every sentence written using a connective of SL can be rewritten as a logically equivalent sentence using one or more Sheffer strokes. Using only the Sheffer stroke, write sentences that are equivalent to each of the following. 
%...
\begin{enumerate}
\setcounter{eargnum}{\arabic{OLDeargnum}}
\item $\enot A$
\item $(A\eand B)$
\item $(A\eor B)$
\item $(A\eif B)$
\item $(A\eiff B)$
\end{enumerate}
\end{exercises}


%%%% A recursive definition of truth in SL

%[go back and explicitly mark the section on a recursive definition of a sentence in SL as optional and then rework it so it is parallel to this passage.]

%In the optional later sections of the chapter, we gave a recursive definition of what it meant to be a sentence in SL. We will also end this chapter with a recursive definition that summarizes the material in the chapter. In this case, we are going to give a recursive definition of truth in SL. [sources: magnus's original treatment. Hodges in the logic handbook. Tarski. Be sure to motivate this and explain its relationship to a recursive definition of a truth table.]



%%%Below here is Magnus's original version of recursive definition of truth in SL  


%
%Formally, what we want is a function that assigns a 1 or 0 to each of the sentences of SL. We can interpret this function as a definition of truth for SL if it assigns 1 to all of the true sentences of SL and 0 to all of the false sentences of SL. Call this function ``$v$'' (for ``valuation''). We want $v$ to a be a function such that for any sentence \script{A}, $v(\script{A})=1$ if \script{A} is true and $v(\script{A})=0$ if \script{A} is false.
%
%Recall that the recursive definition of a wff for SL had two stages: The first step said that atomic sentences (solitary sentence letters) are wffs. The second stage allowed for wffs to be constructed out of more basic wffs. There were clauses of the definition for all of the sentential connectives. For example, if \script{A} is a wff, then \enot\script{A} is a wff.
%
%Our strategy for defining the truth function, $v$, will also be in two steps. The first step will handle truth for atomic sentences; the second step will handle truth for compound sentences.
%
%
%\subsection{Truth in SL}
%How can we define truth for an atomic sentence of SL? Consider, for example, the sentence $M$. Without an interpretation, we cannot say whether $M$ is true or false. It might mean anything. If we use $M$ to symbolize ``The moon orbits the Earth'', then $M$ is true. If we use $M$ to symbolize ``The moon is a giant turnip'', then $M$ is false.
%
%Moreover, the way you would discover whether or not $M$ is true depends on what $M$ means. If $M$ means ``It is Monday,'' then you would need to check a calendar. If $M$ means ``Jupiter's moon Io has significant volcanic activity,'' then you would need to check an astronomy text---and astronomers know because they sent satellites to observe Io.
%
%When we give a symbolization key for SL, we provide an {interpretation} of the sentence letters that we use. The key gives an English language sentence for each sentence letter that we use. In this way, the interpretation specifies what each of the sentence letters \emph{means}. However, this is not enough to determine whether or not that sentence is true. The sentences about the moon, for instance, require that you know some rudimentary astronomy. Imagine a small child who became convinced that the moon is a giant turnip. She could understand what the sentence ``The moon is a giant turnip'' means, but mistakenly think that it was true.
%
%Consider another example: If $M$ means ``It is morning now'', then whether it is true or not depends on when you are reading this. I know what the sentence means, but---since I do not know when you will be reading this---I do not know whether it is true or false.
%
%So an interpretation alone does not determine whether a sentence is true or false. Truth or falsity depends also on what the world is like. If $M$ meant ``The moon is a giant turnip'' and the real moon were a giant turnip, then $M$ would be true. To put the point in a general way, truth or falsity is determined by an interpretation \emph{plus} a way that the world is.
%
%\begin{center}
%INTERPRETATION + STATE OF THE WORLD $\Longrightarrow$ TRUTH/FALSITY
%\end{center}
%
%In providing a logical definition of truth, we will not be able to give an account of how an atomic sentence is made true or false by the world. Instead, we will introduce a \emph{truth value assignment}. Formally, this will be a function that tells us the truth value of all the atomic sentences. Call this function ``$a$'' (for ``assignment''). We define $a$ for all sentence letters \script{P}, such that
%\begin{displaymath}
%a(\script{P}) =
%\left\{
%	\begin{array}{ll}
%	1 & \mbox{if \script{P} is true},\\
%	0 & \mbox{otherwise.}
%	\end{array}
%\right.
%\end{displaymath}
%This means that $a$ takes any sentence of SL and assigns it either a one or a zero; one if the sentence is true, zero if the sentence is false. The details of the function $a$ are determined by the meaning of the sentence letters together with the state of the world. If $D$ means ``It is dark outside'', then $a(D)=1$ at night or during a heavy storm, while $a(D)=0$ on a clear day.
%
%You can think of $a$ as being like a row of a truth table. Whereas a truth table row assigns a truth value to a few atomic sentences, the truth value assignment assigns a value to every atomic sentence of SL. There are infinitely many sentence letters, and the truth value assignment gives a value to each of them. When constructing a truth table, we only care about sentence letters that affect the truth value of sentences that interest us. As such, we ignore the rest. Strictly speaking, every row of a truth table gives a \emph{partial} truth value assignment.
%
%It is important to note that the truth value assignment, $a$, is not part of the language SL. Rather, it is part of the mathematical machinery that we are using to describe SL. It encodes which atomic sentences are true and which are false.
%
%
%We now define the truth function, $v$, using the same recursive structure that we used to define a wff of SL.
%
%\begin{enumerate}
%\item If \script{A} is a sentence letter, then $v(\script{A})=a(\script{A})$.
%%\setcounter{Example}{\arabic{enumi}}\end{enumerate}
%%...
%% Break out of the {enumerate} environment to say something about what is
%% going on. Using \setcounter in this way preserves the numbering, so
%% that the list can resume after the comments.
%
%%This is a mathematical equals sign, not the identity predicate we defined for QL.
%
%% Resume the {enumerate} environment and restore the counter.
%%...
%%\begin{enumerate}\setcounter{enumi}{\arabic{Example}}
%\item If \script{A} is ${\enot}\script{B}$ for some sentence \script{B}, then
%\begin{displaymath}v(\script{A}) =
%	\left\{\begin{array}{ll}
%	1 & \mbox{if $v(\script{B}) = 0$},\\
%	0 & \mbox{otherwise.}
%	\end{array}\right.
%\end{displaymath}
%
%\item If \script{A} is $(\script{B}\eand\script{C})$ for some sentences \script{B,C}, then
%\begin{displaymath}v(\script{A}) =
%	\left\{\begin{array}{ll}
%	1 & \mbox{if $v(\script{B}) = 1$ and $v(\script{C}) = 1$,}\\
%	0 & \mbox{otherwise.}
%	\end{array}\right.
%\end{displaymath}
%\setcounter{Example}{\arabic{enumi}}\end{enumerate}
%%...
%
%It might seem as if this definition is circular, because it uses the word ``and'' in trying to define ``and.'' Notice, however, that this is not a definition of the English word ``and''; it is a definition of truth for sentences of SL containing the logical symbol ``\eand.'' We define truth for object language sentences containing the symbol ``\eand'' using the metalanguage word ``and.'' There is nothing circular about that.
%
%%...
%\begin{enumerate}\setcounter{enumi}{\arabic{Example}}
%\item If \script{A} is $(\script{B}\eor\script{C})$ for some sentences \script{B,C}, then
%\begin{displaymath}v(\script{A}) =
%	\left\{\begin{array}{ll}
%	0 & \mbox{if $v(\script{B}) = 0$ and $v(\script{C}) = 0$,}\\
%	1 & \mbox{otherwise.}
%	\end{array}\right.
%\end{displaymath}
%%\setcounter{Example}{\arabic{enumi}}\end{enumerate}
%%...
%%Notice that this defines truth for sentences containing the symbol ``\eor''' using the word ``and.''
%%...
%%\begin{enumerate}\setcounter{enumi}{\arabic{Example}}
%\item If \script{A} is $(\script{B}\eif\script{C})$ for some sentences \script{B,C}, then
%\begin{displaymath}v(\script{A}) =
%	\left\{\begin{array}{ll}
%	0 & \mbox{if $v(\script{B}) = 1$ and $v(\script{C}) = 0$,}\\
%	1 & \mbox{otherwise.}
%	\end{array}\right.
%\end{displaymath}
%
%\item If \script{A} is $(\script{B}\eiff\script{C})$ for some sentences \script{B,C}, then
%\begin{displaymath}v(\script{A}) =
%	\left\{\begin{array}{ll}
%	1 & \mbox{if $v(\script{B}) = v(\script{C})$},\\
%	0 & \mbox{otherwise.}
%	\end{array}\right.
%\end{displaymath}
%\end{enumerate}
%
%Since the definition of $v$ has the same structure as the definition of a wff, we know that $v$ assigns a value to \emph{every} wff of SL. Since the sentences of SL and the wffs of SL are the same, this means that $v$ returns the truth value of every sentence of SL.
%
%Truth in SL is always truth \emph{relative to} some truth value assignment, because the definition of truth for SL does not say whether a given sentence is true or false. Rather, it says how the truth of that sentence relates to a truth value assignment.
%

%%%%%%%%%%%%%%%%%%%%%%%%		 Key Terms

\section*{Key Terms}
\begin{multicols}{2}
\begin{sortedlist}
\sortitem{Semantically contingent in SL}{}
\sortitem{Semantically logically equivalent in SL}{}
\sortitem{Semantically consistent in SL}{}
\sortitem{Semantically valid in SL}{}
\sortitem{Semantic contradiction in SL}{}
\sortitem{Semantic tautology in SL}{}
\sortitem{Complete truth table}{}
\sortitem{Truth assignment}{}
\sortitem{Truth-functional connective}{} 
\sortitem{Nonlogical symbol}{}
\sortitem{Logical constant}{}
\sortitem{Interpretation}{}
\end{sortedlist}
\end{multicols}









\chapter{Proofs in Sentential Logic}
\label{chap:proofsinSL}
\markright{Chap. \ref{chap:proofsinSL}: Proofs in SL}
\setlength{\parindent}{1em}

\newcounter{theorem}
\setcounter{theorem}{1}

%\label{whole_slproof_chap} %uncomment and typeset twice to print the whole chapter.


%rob: This chapter is based on the original Chapter 6. I took all the material on proof in SL and moved it earlier in the book so the students would have a chance to start doing derivations earlier. I have also expanded the opening material that wasn't in a Chapter section into its own Chapter section explaining the basic idea of a proof. 

% *******************************************
% *		Substitution Instances and Proofs			   *	
% *******************************************

\section{Substitution Instances and Proofs}

% rob: Changed opening to add big picture stuff. What we did last chapter, what we will do this chapter. The ability to use deduction as an important mental skill

In the last chapter, we introduced the truth table method, which allowed us to check to see if various logical properties were present, such as whether a statement is a tautology or whether an argument is valid. The method in that chapter was semantic, because it relied on the meaning of symbols, specifically, whether they were interpreted as true or false. The nice thing about that method was that it was completely mechanical. If you just followed the rules like a robot, you would eventually get the right answer. You didn't need any special insight and there were no tough decisions to make. The downside to this method was that the tables quickly became way too long. It just isn't practical to make a 32 line table every time you have to deal with five different sentence letters. 

In this chapter, we are going to introduce a new method for checking for validity and other logical properties. This time our method is going to be purely syntactic. We won't be at all concerned with what our symbols mean. We are just going to look at the way they are arranged. Our method here will be called a system of natural deduction. When you use a system of natural deduction, you won't do it mechanically. You will need to understand the logical structure of the argument and employ your insight. This is actually one of the reasons people like systems of natural deduction. They let us represent the logical structure of arguments in a way we can understand. Learning to represent and manipulate arguments this way is a core mental skill, used in fields like mathematics and computer programming. 

Consider two arguments in SL:
\begin{quotation}
\begin{tabu}{X[1,p,m]X[1,p,m]}
\textbf{Argument A} & \textbf{Argument B} \\
\begin{earg*}
\item $P \eor Q$
\item  $\enot P$
\itemc[.2] Q
\end{earg*}
&

\begin{earg*}
\item $P \eif Q$
\item $P$
\itemc[.2] Q
\end{earg*}

\end{tabu}
\end{quotation}

These are both valid arguments. Go ahead and prove that for yourself by constructing the four-line truth tables. These particular valid arguments are examples of important kinds of arguments that are given special names. Argument A is an example of a kind of argument traditionally called \emph{disjunctive syllogism}. In the system of proof we will develop later in the chapter, it will be given a newer name, \emph{disjunction elimination} (\eor-E). Given a disjunction and the negation of one of the disjuncts, the other disjunct follows as a valid consequence. Argument B makes use of a different valid form: Given a conditional and its antecedent, the consequent follows as a valid consequence. This is traditionally called \emph{modus ponens}. In our system it will be called \emph{conditional elimination} (\eif-E). 

Both of the arguments above remain valid even if we substitute different sentence letters. You don't even need to run the truth tables again to see that these arguments are valid: 
\begin{quotation}
\begin{tabu}{X[1,p,m]X[1,p,m]}
\textbf{Argument A*} & \textbf{Argument B*} \\
\begin{earg*}
\item $A \eor B$
\item $\enot A$
\itemc[.2] B
\end{earg*}

&

\begin{earg*}
\item $A \eif B$
\item $A$
\itemc[.2] B
\end{earg*}
\end{tabu}
\end{quotation}

Replacing $P$ with $A$ and $Q$ with $B$ changes nothing (so long as we are sure to replace \emph{every} $P$ with an $A$ and every $Q$ with a $B$). What's more interesting is that we can replace the individual sentence letters in Argument A and Argument B with longer sentences in SL and the arguments will still be valid, as long as we do the substitutions consistently. Here are two more perfectly valid instances of disjunction and conditional elimination. 
\begin{quotation}
\begin{tabu}{X[1,p,m]X[1,p,m]}
\textbf{Argument A**} & \textbf{Argument B**} \\
\begin{earg*}
\item  $(C \eand D) \eor (E \eor F)$
\item  $\enot (C \eand D)$
\itemc[.2] $E \eor F$
\end{earg*}

&

\begin{earg*}
\item $(G \eif H) \eif (I \eor J)$
\item $(G \eif H)$
\itemc[.2] $I \eor J$
\end{earg*}
\end{tabu}
\end{quotation}
Again, you can check these using truth tables, although the 16 line truth tables begin to get tiresome. All of these arguments are what we call \emph{substitution instances} of the same two logical forms. We call them that because you get them by replacing the sentence letters with other sentences, either sentence letters or longer sentences in SL. A substitution instance cannot change the sentential connectives of a sentence, however. The sentential connectives are what make the \emph{logical form} of the sentence. We can write these logical forms using fancy script letters.

\begin{quotation}
\begin{tabu}{X[1,p,m]X[1,p,m]}
\textbf{Disjunction Elimination} \newline (Disjunctive Syllogism) &
\textbf{Conditional Elimination} \newline (Modus Ponens) \\


\begin{earg*}
\item $\script{A} \eor \script{B}$
\item $\enot \script{A}$
\itemc[.2] \script{B}
\end{earg*}

&

\begin{earg*}
\item  $\script{A} \eif \script{B}$
\item  $\script{A}$
\itemc[.2] \script{B}
\end{earg*}
\end{tabu}
\end{quotation}

As we explained in Chapter \ref{chap:SL}, the fancy script letters are \emph{metavariables}.  They are a part of our metalanguage and can refer to single sentence letters like $P$ or longer sentences like $A \eiff (B \eand (C \eor D))$. 

\newglossaryentry{sentence form}
{
name=sentence form,
description={A sentence in SL that contains one or more metavariables in place of sentence letters.}
}



\newglossaryentry{substitution instance}
{
name=substitution instance,
description={A sentence that is created by consistently substituting sentences for one or more of the metavariables in a sentence form..}
}


\newglossaryentry{argument form}
{
name=argument form,
description={An argument that includes one or more sentence forms.}
}


\newglossaryentry{substitution instance of an argument form}
{
name=substitution instance of an argument form,
description={An argument obtained by consistently replacing the sentence forms in the argument form with their substitution instances..}
}



Formally, we can define a \textsc{\gls{sentence form}}\label{def:sentence_form} as a sentence in SL that contains one or more metavariables in place of sentence letters. A \textsc{\gls{substitution instance}}\label{def:substitution_instance} of that sentence form is then a sentence created by consistently substituting sentences for one or more of the metavariables in the sentence form. Here ``consistently substituting'' means replacing all instances of the metavariable with the same sentence. You cannot replace instances of the same metavariable with different sentences, or leave a metavariable as it is, if you have replaced other metavariables of that same type. An \textsc{\gls{argument form}}\label{def:argument_form}
 is an argument that includes one or more sentence forms, and a \textsc{\gls{substitution instance of an argument form}}\label{def:substitution instance_of_an_argument_form} of the argument form is the argument obtained by consistently replacing the sentence forms in the argument form with their substitution instances.

Once we start identifying valid argument forms like this, we have a new way of showing that longer arguments are valid. Truth tables are fun, but doing the 1028 line truth table for an argument with 10 sentence letters would be tedious. Worse, we would never be sure we hadn't made a little mistake in all those Ts and Fs. Part of the problem is that we have no way of knowing  \emph{why} the argument is valid. The table gives you very little insight into how the premises work together. 

The aim of a \emph{proof system} is to show that particular arguments are valid in a way that allows us to understand the reasoning involved in the argument. Instead of representing all the premises and the conclusion in one table, we break the argument up into steps. Each step is a basic argument form of the sort we saw above, like disjunctive syllogism or modus ponens. Suppose we are given the premises $\enot L \eif (J \eor L)$ and $\enot L$ and wanted to show $J$. We can break this up into two smaller arguments, each of which is a substitution inference of a form we know is correct.

\begin{quotation}
\begin{tabu}{X[1,p,m]X[1,p,m]}
\textbf{Argument 1} & \textbf{Argument 2} \\
\begin{earg*}
\item $\enot L \eif (J \eor L)$
\item $\enot L$
\itemc[.2] $J \eor L$
\end{earg*}

&

\begin{earg*}
\item $J \eor L$
\item $\enot L$
\itemc[.2] $J$
\end{earg*}
\end{tabu}
\end{quotation}

The first argument is a substitution instance of modus ponens and the second is a substitution instance of disjunctive syllogism, so we know they are both valid. Notice also that the conclusion of the first argument is the first premise of the second, and the second premise is the same in both arguments. Together, these arguments are enough to get us from $\enot L \eif (J \eor L)$ and $\enot L$ to $J$.

These two arguments take up a lot of space, though. To complete our proof system, we need a system for showing clearly how simple steps can combine to get us from premises to conclusions. The system we will use in this book was devised by the American logician Frederic Brenton Fitch (1908--1987). We begin by writing our premises on numbered lines with a bar on the left and a little bar underneath to represent the end of the premises. Then we write ``Want'' on the side followed by the conclusion we are trying to reach. If we wanted to write out arguments 1 and 2 above, we would begin like this.

\begin{proof}
	\hypo{1}{\enot L \eif (J \eor\ L)}
	\hypo{2}{\enot L} \by{Want: $J$}{}			
\end{proof}

We then add the steps leading to the conclusion below the horizontal line, each time explaining off to the right why we are allowed to write the new line. This explanation consists of citing a rule and the prior lines the rule is applied to. In the example we have been working with we would begin like this

\begin{proof}
	\hypo{1}{\enot L \eif (J \eor L)}
	\hypo{2}{\enot L} \by{Want: $J$}{}
	\have{3}{J \eor L} \ce{1, 2}
\end{proof}

and then go like this

\begin{proof}
	\hypo{1}{\enot L \eif (J \eor L)}
	\hypo{2}{\enot L} \by{Want: $J$}{}
	\have{3}{J \eor L} \ce {1, 2}
	\have{4}{J} \oe{2, 3}
\end{proof}

\newglossaryentry{proof}
{
name=proof,
description={A sequence of sentences, where the first sentences of the sequence are assumptions, and all sentences after the assumptions follow from sentences earlier in the sequence according to the rules of derivation.}
}



The little chart above is a \emph{proof} that $J$ follows from $\enot L \eif (J \eor L)$ and $\enot L$. We will also call proofs like this \emph{derivations}. Formally, a \textsc{\gls{proof}}\label{def:proof} is a sequence of sentences. The first sentences of the sequence are assumptions; these are the premises of the argument. Every sentence later in the sequence follows from earlier sentences by one of the rules of proof. The final sentence of the sequence is the conclusion of the argument.

\iflabelexists{chap:proofsinQL}{In the remainder of this chapter, we will develop a system for proving sentences in SL. Later, in Chapter \ref{chap:proofsinQL}, this will be expanded to cover Quantified Logic (QL). First, though, you should practice identifying substitution instances of sentences and longer rules.}{} 

%I added exercises for identifying substitution inferences, because many students need practice with this really basic form of pattern recognition. 

%%%%%  PRACTICE PROBLEMS %%%%%%%%%%%%%

\practiceproblems
\noindent\problempart For each problem, a sentence form is given in metavariables. Identify which of the sentences after it are legitimate substitution instances of that form. 

\begin{exercises}
\begin{longtabu}{X[1,p,m]X[1,p,m]} 

\item $\script{A} \eand \script{B}$: 
	\begin{enumerate}[label=\alph*.]
	\item $P \eor Q$
	\iflabelexists{showanswers}{{\color{red}\item [\circled{\emph{\color{red}{b.}}}]$(A \eif B) \eand C$}}{\item $(A \eif B) \eand C$}
	\iflabelexists{showanswers}{{\color{red}\item [\circled{\emph{\color{red}{c.}}}] \begin{flushleft}$[(A \eand B) \eif (B \eand A)] \linebreak \eand (\enot A \eand \enot B)$\end{flushleft}}}{\item \begin{flushleft}$[(A \eand B) \eif (B \eand A)] \linebreak \eand (\enot A \eand \enot B)$\end{flushleft}}
	\iflabelexists{showanswers}{{\color{red}\item [\circled{\emph{\color{red}{d.}}}]$[((A \eand B) \eand C) \eand D] \eand F$}}{\item $[((A \eand B) \eand C) \eand D] \eand F$ } 
	\item[e.] $(A \eand B) \eif C$
	\end{enumerate}

&

\item $\enot(\script{P} \eand \script{Q})$
	\begin{enumerate}[label=\alph*.]
	\iflabelexists{showanswers}{{\color{red}\item [\circled{\emph{\color{red}{a.}}}]$\enot(A \eand B)$}}{\item $\enot(A \eand B)$}
	\iflabelexists{showanswers}{{\color{red}\item [\circled{\emph{\color{red}{b.}}}]$\enot(A \eand A)$}}{\item $\enot(A \eand A)$}
	\item[c.] $\enot A \eand B$
	\iflabelexists{showanswers}{{\color{red}\item [\circled{\emph{\color{red}{d.}}}]\begin{flushleft}$\enot((\enot A \eand B) \eand (B \eand \enot A))$\end{flushleft}}}{\item[d.] \begin{flushleft}$\enot((\enot A \eand B) \eand (B \eand \enot A))$\end{flushleft}}
	\item[e.] $\enot(A \eif B)$
	\end{enumerate}


\\

\item $\enot \script{A}$
	\begin{enumerate}[label=\alph*.]
	\item $\enot A \eif B$
	\iflabelexists{showanswers}{{\color{red}\item [\circled{\emph{\color{red}{b.}}}]$\enot (A \eif B)$}}{\item $\enot (A \eif B)$}
	\iflabelexists{showanswers}{{\color{red}\item [\circled{\emph{\color{red}{c.}}}]$\enot[(G \eif (H \eor I)) \eif G]$}}{\item $\enot[(G \eif (H \eor I)) \eif G]$}
	\item[d.] $\enot G \eand (\enot B \eand \enot H)$
	\iflabelexists{showanswers}{{\color{red}\item [\circled{\emph{\color{red}{e.}}}]$\enot(G \eand (B \eand H))$}}{\item[e.] $\enot(G \eand (B \eand H))$}
	\end{enumerate}
&

\item $\enot \script{A} \eif  \script{B}$
	\begin{enumerate}[label=\alph*.]
	\item $\enot A \eand B$
	\iflabelexists{showanswers}{{\color{red}\item [\circled{\emph{\color{red}{b.}}}]$\enot B \eif A$}}{\item $\enot B \eif A$}
	\iflabelexists{showanswers}{{\color{red}\item [\circled{\emph{\color{red}{c.}}}]$\enot(X \eand Y) \eif (Z \eor B)$}}{\item $\enot(X \eand Y) \eif (Z \eor B)$}
	\item[d.] $\enot(A \eif B)$
	\item[e.] $A \eif \enot B$
	\end{enumerate}
\\

\item $\enot \script{A} \eiff \enot \script{Z}$
	\begin{enumerate}[label=\alph*.]
	\item $\enot (P \eiff Q)$
	\iflabelexists{showanswers}{{\color{red}\item [\circled{\emph{\color{red}{b.}}}]$\enot(P \eiff Q) \eiff \enot (Q \eiff P)$}}{\item $\enot(P \eiff Q) \eiff \enot (Q \eiff P)$}
	\item[c.] $\enot H \eif \enot G$
	\item[d.] $\enot (A \eand B) \eiff C$
	\iflabelexists{showanswers}{{\color{red}\item [\circled{\emph{\color{red}{e.}}}]\begin{flushleft} $\enot [\enot (P \eiff Q) \eiff R] \eiff \enot S$ \end{flushleft}}}{\item[e.] \begin{flushleft} $\enot [\enot (P \eiff Q) \eiff R] \eiff \enot S$ \end{flushleft}}
	\end{enumerate}

&

\item $(\script{A} \eand \script{B}) \eor \script{C}$
	\begin{enumerate}[label=\alph*.]
	\item $(P \eor Q) \eand R$
	\iflabelexists{showanswers}{{\color{red}\item [\circled{\emph{\color{red}{b.}}}]$(\enot M \eand \enot D) \eor C$}}{\item $(\enot M \eand \enot D) \eor C$}
	\item[c.] $(D \eand R) \eand (I \eor D)$
	\item[d.] $[(D \eif O) \eor A] \eand D$
	\iflabelexists{showanswers}{{\color{red}\item [\circled{\emph{\color{red}{e.}}}]$[(A \eand B) \eand C] \eor (D \eor A)$}}{\item[e.] $[(A \eand B) \eand C] \eor (D \eor A)$}
	\end{enumerate}
%\factoidbox{B, E}


\\

\item $(\script{A} \eand \script{B}) \eor \script{A}$							
	\begin{flushleft}
	\begin{enumerate}[label=\alph*.]
	\item$((C \eif D) \eand E) \eor A$
	\iflabelexists{showanswers}{{\color{red}\item [\circled{\emph{\color{red}{b.}}}]$(A \eand A) \eor A$}}{\item$(A \eand A) \eor A$}
	\iflabelexists{showanswers}{{\color{red}\item [\circled{\emph{\color{red}{c.}}}]$((C \eif D) \eand E) \eor (C \eif D)$}}{\item$((C \eif D) \eand E) \eor (C \eif D)$}
	\iflabelexists{showanswers}{{\color{red}\item [\circled{\emph{\color{red}{d.}}}]$((G \eand B) \eand (Q \eor R)) \eor (G \eand B)$}}{\item$((G \eand B) \eand (Q \eor R)) \eor (G \eand B)$}
	\item[e.]$(P \eor Q) \eand P$
	\end{enumerate}
	\end{flushleft}

&
\item $\script{P} \eif (\script{P} \eif \script{Q})$
	\begin{flushleft}
	\begin{enumerate}[label=\alph*.]
	\item $A \eif (B \eif C)$
	\iflabelexists{showanswers}{{\color{red}\item [\circled{\emph{\color{red}{b.}}}]$(A \eand B) \eif [(A \eand B) \eif C]$}}{\item $(A \eand B) \eif [(A \eand B) \eif C]$}
	\iflabelexists{showanswers}{{\color{red}\item [\circled{\emph{\color{red}{c.}}}]$(G \eif B) \eif [(G \eif B) \eif (G \eif B)]$}}{\item $(G \eif B) \eif [(G \eif B) \eif (G \eif B)]$}
	\iflabelexists{showanswers}{{\color{red}\item [\circled{\emph{\color{red}{d.}}}]$M \eif [M \eif (D \eand (C \eand M))]$}}{\item $M \eif [M \eif (D \eand (C \eand M))]$}
	\item[e.] $(S \eor O) \eif [(O \eor S) \eif A]$
	\end{enumerate}
	\end{flushleft}



\\
\item $\enot \script{A} \eor (\script{B} \eand \enot \script{B})$
	\begin{flushleft}
	\begin{enumerate}[label=\alph*.]
	\item $\enot P \eor (Q \eand \enot P)$
	\iflabelexists{showanswers}{{\color{red}\item [\circled{\emph{\color{red}{b.}}}]$\enot A \eor (A \eand \enot A)$}}{\item $\enot A \eor (A \eand \enot A)$}
	\item[c.] $(P \eif Q) \eor [(P \eif Q) \eand \enot R]$
	\item[d.] $\enot E \eand (F \eand \enot F)$
	\iflabelexists{showanswers}{{\color{red}\item [\circled{\emph{\color{red}{e.}}}]$\enot G \eor [(H \eif G) \eand \enot (H \eif G)]$}}{\item[e.] $\enot G \eor [(H \eif G) \eand \enot (H \eif G)]$}
	\end{enumerate}
	\end{flushleft}

&


\item	$(\script{P} \eor \script{Q}) \eif \enot(\script{P} \eand \script{Q})$
\begin{flushleft} 	
\begin{enumerate}[label=\alph*.]
	\item	$A \eif \enot B$
	\iflabelexists{showanswers}{{\color{red}\item [\circled{\emph{\color{red}{b.}}}]$(A \eor B) \eif \enot(A \eand B)$}}{\item	$(A \eor B) \eif \enot(A \eand B)$}
	\iflabelexists{showanswers}{{\color{red}\item [\circled{\emph{\color{red}{c.}}}]$(A \eor A) \eif \enot(A \eand A)$}}{\item	$(A \eor A) \eif \enot(A \eand A)$}
	\iflabelexists{showanswers}{{\color{red}\item [\circled{\emph{\color{red}{d.}}}]$[(A \eand B) \eor (D \eif E)] \eif $ \linebreak[4]$ \enot[(A \eand B) \eand (D \eif E)]$}}{\item $[(A \eand B) \eor (D \eif E)] \eif $ \linebreak[4]$ \enot[(A \eand B) \eand (D \eif E)]$}
	\item[e.]	$(A \eand B) \eif \enot(A \eor B)$
	\end{enumerate}
\end{flushleft} 

\end{longtabu}
\end{exercises}
\noindent\problempart For each problem, a sentence form is given in sentence variables. Identify which of the sentences after it are legitimate substitution instances of that form. 

\begin{exercises}
\begin{longtabu}{p{2.5in}p{2.5in}}

\item $ \script{P} \eand \script{P} $ 
\begin{flushleft} 	
\begin{enumerate}[label=\alph*.]
\item 	$A \eand B$
\item 	$D \eor D$
\item 	$Z \eand Z$
\item 	$(Z \eor B) \eand (Z \eand B)$
\item 	$(Z \eor B) \eand (Z \eor B)$
\end{enumerate}
\end{flushleft}
%\begin{flushleft} 	
%\begin{enumerate}[label=\alph*.]
%\item 	$A \eand B$
%\item 	$D \eor D$
%\item 	\framebox{$Z \eand Z$}
%\item 	$(Z \eor B) \eand (Z \eand B)$
%\item 	\framebox{$(Z \eor B) \eand (Z \eor B)$}
%\end{enumerate}
%\end{flushleft}
&
\item $ \script{O} \eand (\script{N} \eand \script{N}) $ 
\begin{flushleft} 	
\begin{enumerate}[label=\alph*.]
\item 	$A \eand (B \eand C)$
\item 	$A \eand (A \eand B)$
\item 	$(A \eand B) \eand B$
\item 	$A \eand (B \eand B)$
\item 	$(C\eif D) \eand (Q \eand Q)$
\end{enumerate}
\end{flushleft}
%\begin{flushleft} 	
%\begin{enumerate}[label=\alph*.]
%\item 	$A \eand (B \eand C)$
%\item 	$A \eand (A \eand B)$
%\item 	$(A \eand B) \eand B$
%\item 	\framebox{$A \eand (B \eand B)$}
%\item 	\framebox{$(C\eif D) \eand (Q \eand Q)$}
%\end{enumerate}
%\end{flushleft}
\\ 
\item $ \script{H} \eif \script{Z} $ 
\begin{flushleft} 	
\begin{enumerate}[label=\alph*.]
\item 	$E \eif E$
\item 	$G \eif H$
\item 	$G \eif (I \eif K)$
\item 	$[(I \eif K) \eif G] \eif A$
\item 	$G \eand (I \eif K)$
\end{enumerate}
\end{flushleft}
%\begin{flushleft} 	
%\begin{enumerate}[label=\alph*.]
%\item 	\framebox{$E \eif E$}
%\item 	\framebox{$G \eif H$}
%\item 	\framebox{$G \eif (I \eif K)$}
%\item 	\framebox{$[(I \eif K) \eif G] \eif A$}
%\item 	$G \eand (I \eif K)$
%\end{enumerate}
%\end{flushleft}
&
\item $ \enot \script{H} \eand \script{C} $ 
\begin{flushleft} 	
\begin{enumerate}[label=\alph*.]
\item 	$H \eand C$
\item 	$\enot (H \eand C)$
\item 	$\enot Q \eand R$
\item 	$R \eand \enot Q$
\item 	$\enot (X \eiff Y) \eand (Y \eif Z)$
\end{enumerate}
\end{flushleft}
%\begin{flushleft} 	
%\begin{enumerate}[label=\alph*.]
%\item 	$H \eand C$
%\item 	$\enot (H \eand C)$
%\item 	\framebox{$\enot Q \eand R}$
%\item 	$R \eand \enot Q$
%\item 	\framebox{$\enot (X \eiff Y) \eand (Y \eif Z)$}
%\end{enumerate}
%\end{flushleft}
\\
\item $ \enot (\script{G} \eiff \script{M}) $ 
\begin{flushleft} 	
\begin{enumerate}[label=\alph*.]
\item 	$\enot (K \eiff K) $
\item 	$\enot K \eiff K$
\item 	$\enot ((I \eiff K) \eiff (S \eand S)) $
\item 	$\enot (H \eif (I \eor J)$
\item 	$\enot ((H \eor F)  \eiff (Z \eif D) ) $
\end{enumerate}
\end{flushleft}
&
\item $ (\script{I} \eif \script{W}) \eor \script{W} $ 
\begin{flushleft} 	
\begin{enumerate}[label=\alph*.]
\item 	$(D \eor E) \eif E$
\item 	$(D \eif E) \eor E$
\item 	$ D \eif (E \eor E)	$
\item 	$ ((W \eand L) \eif L) \eor W$
\item 	$((W \eand L) \eif J) \eor J$
\end{enumerate}
\end{flushleft}

\\

\item $ \script{M} \eor (\script{A} \eor \script{A}) $ 
\begin{flushleft} 	
\begin{enumerate}[label=\alph*.]
\item 	$ A \eor (A \eor A) 			$
\item 	$ (A \eor A) \eor A			$
\item 	$ C \eor (C \eor D)			$
\item 	$ (R \eif K) \eor ((D \eand G) \eor (D \eand G)) 			$
\item 	$ (P \eand P)  \eor ((\enot H \eand C) \eor (\enot H \eand C)) 			$
\end{enumerate}
\end{flushleft}
&
\item $ \script{A} \eif \enot (\script{G} \eand \script{G}) $ 
\begin{flushleft} 	
\begin{enumerate}[label=\alph*.]
\item 	$B \eiff \enot (G \eand G) 			$
\item 	$O \eif \enot (R \eand D) 			$
\item 	$(H \eif Z) \eif (\enot D	\eand D)		$
\item 	$ (O \eand (N \eand N))  \eif \enot (F \eand F)			$
\item 	$\enot D \eand \enot( (J \eif J) \eand (O \eiff O) $ 
\end{enumerate}
\end{flushleft}
\\
\item $ \enot ((\script{K} \eif \script{K}) \eor \script{K}) \eand \script{G} $ 
\begin{flushleft} 	
\begin{enumerate}[label=\alph*.]
\item 	$\enot (D \eif D) (\eor D \eand L)	 			$
\item 	$ \enot (D \eif (D \eor (D \eand L))				$
\item 	$ \enot ((D \eif D) \eor D) \eand L			$
\item 	$((\enot K \eif \enot K) \eor K) \eand L 			$
\item 	$ \enot ((D \eif D) \eor D) \eand ((D \eif D) \eor D)			$
\end{enumerate}
\end{flushleft}
&
\item $ (\script{B} \eiff (\script{N} \eiff \script{N})) \eor \script{N} $ 
\begin{flushleft} 	
\begin{enumerate}[label=\alph*.]
\item 	$(B \eiff (N \eiff (N \eand N))) \eor N  			$ %nope
\item 	$((E \eand T) \eiff (V \eiff V )) \eor V  			$  %yup
\item 	$ (B \eiff (N \eand N)) \eor B			$  %nope
\item 	$A \eiff (N \eiff (N \eor N)))			$ %nope
\item 	$((X \eiff N) \eiff N) \eor N			$ %nope
\end{enumerate}
\end{flushleft}
\end{longtabu}
\end{exercises}

\noindent\problempart Use the following symbolization key in the gray bubble to create substitution instances of the sentences below.

\begin{mdframed}[style=mytablebox] 
\begin{longtabu}{X[.5]X[.5]X[1]X[1]X[1]} 
$\script{A}: B$ 	& 	$\script{B}: \enot C$  	& $\script{C}: A \eif B$ &
$\script{D}:\enot (B \eand C)$  & $\script{E}: D \eiff E$
\end{longtabu}
\end{mdframed}

\begin{exercises}
\begin{longtabu}{X[1,l,m]X[1,p,m]} 
\item $\enot( \script{A} \eiff \script{B})$ 
\answer{$\enot(B \eiff  \enot{C})$}
&

\item $(\script{B} \eif \script{C}) \eand \script{D}$ 


\answer{$(\enot{C} \eif (A \eif B)) \eand \enot(B \eand C)$}
\\
\item $\script{D} \eif (\script{B} \eand \enot \script{B}) $ 


\answer{$\enot(B \eand C) \eif (\enot{C} \eand \enot \enot{C}) $}
&
\item $\enot \enot (\script{C} \eor \script{E})$ 


\answer{$\enot \enot ((A \eif B) \eor (D \eiff E))$}
\\
\item $\enot \script{C} \eiff (\enot \enot \script{D} \eand \script{E})$ 


\answer{$\enot (A \eif B) \eiff (\enot \enot \enot(B \eand C) \eand (D \eiff E))$}

\end{longtabu}
\end{exercises}



\noindent\problempart Use the following symbolization key in the gray bubble to create substitution instances of the sentences below.

\begin{mdframed}[style=mytablebox] 
\begin{longtabu}{X[1]X[1]X[1]X[.5]X[.5]} 
$\script{A}: I \eor (I \eiff V)  $ 
&	$\script{B}: C \eiff V$ 
&	$\script{C}: L \eif X$  
&	$\script{D}: V$  
&	$\script{E}: U$ 
\end{longtabu}
\end{mdframed}

\begin{exercises}
\begin{longtabu}{X[1,l,m]X[1,p,m]} 
\item $\enot \script{A} \eif \enot \script B$ 
&
\item $\enot(\script{B} \eand \script{D})$ 
\\
\item $(\script{A} \eif \script{A}) \eor (\script{C} \eif \script{A})$ 
&
\item $[(\script{A} \eif \script{B}) \eif \script{A}] \eif \script{A}$ 
\\
\item $\script{A} \eand (\script{B} \eand (\script{C} \eand (\script{D} \eand \script{E})))$
&\\
\end{longtabu}
\end{exercises}

%%%%%%%%%%%%%%%%%% Part E


\noindent\problempart \label{sec4.1partC} Decide whether the following are examples of $\eif$E (modus ponens).

\begin{exercises}
\begin{longtabu}{X[1,p,m]X[1,p,m]X[1,p,m]} 

\item \begin{earg*}
\item $A \eif B$ 
\item $B \eif C$ 
\itemc[.3] $A \eif C$
\end{earg*}
\answer{\framebox{Not MP}}
	
&

\item \begin{earg*}	
\item$P \eand Q$ 
\item 	$P$ 
\itemc[.3] 	 $Q$
\end{earg*}

\answer{\framebox{Not MP}}
	
&
\item \begin{earg*}	
\item $P \eif Q$ 
\itemc[.3] 	$Q$
\end{earg*}
\answer{\framebox{Not MP}}

\\
\item \begin{earg*}	
\item $D \eif E$ 
\item 	$E$ 
\itemc[.3] 	$D$
\end{earg*}
\answer{\framebox{Not MP}}

&

\item \begin{earg*}
\item $(P \eand Q) \eif (Q \eand V)$
\item 	$P \eand Q$
\itemc[.3] 	 $Q \eand V$
\end{earg*}
\answer{\framebox{MP}}
\end{longtabu}
\end{exercises}
	

\noindent\problempart \label{sec4.1partC} Decide whether the following are examples of $\eif$E (modus ponens).

\begin{exercises}
\begin{longtabu}{X[1]X[1]} 
\item \begin{earg*}
\item	$C \eif D$  
\itemc[.3] 	 $C$
\end{earg*}
%\frame{Not MP}\\
	
&

\item \begin{earg*}
\item $(C \eand L) \eif (E \eor C)$ 
\item $C \eand L$ 
\itemc[.3] 	  $E \eor C$
\end{earg*}
%\framebox{MP}
	
\\
\item \begin{earg*}
\item  $\enot A \eif B$ 
\item $\enot B$ 
\itemc[.3] 	 $B$
\end{earg*}
	%\framebox{Not MP}\\
&

\item \begin{earg*}
\item	$X \eif \enot Y$ 
\item  	$\enot Y$ 
\itemc[.3] 	 $\therefore$\ $X$
\end{earg*}
%\framebox{Not MP}\\
\\
\item \begin{earg*}
\item $G \eif H$ 
\item  $\enot H$ 
\itemc[.3] 	  $\enot G$
\end{earg*}
%\framebox{Not MP}\\

\end{longtabu}
\end{exercises}


%%%% part G

\noindent\problempart Decide whether the following are examples of \eor-E (disjunctive syllogism). 

\begin{exercises}
\begin{longtabu}{X[1]X[1]} 
\item \begin{earg*}
\item $(A \eif B) \eor (X \eif Y)$  
\item $\enot A$  
\itemc[.3]  $X \eif Y$
\end{earg*}

\answer{\framebox{Not DS}}

&	

\item \begin{earg*}
\item $[(S \eor T) \eor U] \eor V$  
\item $\enot[(S \eor T) \eor U]$  
\itemc[.3] $V$
\end{earg*}
\answer{\framebox{DS}}

\\
\item \begin{earg*}
\item $P \eor Q$  
\item $P$  
\itemc[.3] \enot $Q$
\end{earg*}
\answer{\framebox{Not DS}}

&
\item \begin{earg*}
\item $\enot (A \eor B)$  
\item $\enot A$  
\itemc[.3] $B$
\end{earg*}
\answer{\framebox{Not DS}}
\\

\item \begin{earg*}
\item $(P \eor Q) \eor R$  
\itemc[.3]  $R$
\answer{\framebox{Not DS}}
\end{earg*}

\end{longtabu}
\end{exercises}

\noindent\problempart Decide whether the following are examples of \eor-E (disjunctive syllogism).

\begin{exercises}
\begin{longtabu}{X[1]X[1]} 
\item \begin{earg*} 
\item $(C \eand D) \eor E$  
\item $(C \eand D)$  
\itemc[.3] $E$
\end{earg*}
%\framebox{Not DS}\\
&

\item \begin{earg*} 
\item $(P \eor Q) \eif R$  
\item $\enot(P \eor Q)$  
\itemc[.3] $R$
\end{earg*}
%\framebox{Not DS}\\
\\

\item \begin{earg*} 
\item  $X \eor (Y \eif Z)$  
\item $\enot X$  
\itemc[.3] $Y \eif Z$
\end{earg*}
%\framebox{DS}\\

&
\item \begin{earg*} 
\item $(P \eor Q) \eor R$  
\item  $\enot P$  
\itemc[.3] $Q$
\end{earg*}
%\framebox{Not DS}\\

\\
\item \begin{earg*} 
\item $A \eor (B \eor C)$  
\item $\enot A$   
\itemc[.3]  $B \eor C$	
\end{earg*}
%\framebox{DS}\\
\end{longtabu}
\end{exercises}




% *******************************************
% *				Basic Rules for Sentential Logic	   *	
% *******************************************

\section{Basic Rules for Sentential Logic}
\setlength{\parindent}{1em}
%rob: I removed indirect and conditional proof from this section, so that they would have practice just doing direct proofs before they moved on to the fancy stuff. 

In designing a proof system, we could just start with disjunctive syllogism and modus ponens. Whenever we discovered a valid argument that could not be proved with rules we already had, we could introduce new rules. Proceeding in this way, we would have an unsystematic grab bag of rules. We might accidentally add some strange rules, and we would surely end up with more rules than we need.

Instead, we will develop what is called a \define{system of natural deduction}. In a natural deduction system, there will be two rules for each logical operator: an introduction, and an elimination rule. The introduction rule will allow us to prove a sentence that has the operator you are ``introducing'' as its main connective. The elimination rule will allow us to prove something given a sentence that has the operator we are ``eliminating'' as the main logical operator.

In addition to the rules for each logical operator, we will also have a reiteration rule. If you already have shown something in the course of a proof, the reiteration rule allows you to repeat it on a new line. We can define the rule of reiteration like this

Reiteration (R)
\begin{proof}
	\have[m]{a}{\script{A}}
	\have[n]{b}{\script{A}} \by{R}{a}
\end{proof}

This diagram shows how you can add lines to a proof using the rule of reiteration. As before, the script letters represent sentences of any length. The upper line shows the sentence that comes earlier in the proof, and the bottom line shows the new sentence you are allowed to write and how you justify it. The reiteration rule above is justified by one line, the line that you are reiterating. So the ``R $m$'' on line 2 of the proof means that the line is justified by the reiteration rule (R) applied to line $m$. The letters $m$ and $n$ are variables, not real line numbers. In a real proof, they might be lines 5 and 7, or lines 1 and 2, or whatever. When we define the rule, however, we use variables to underscore the point that the rule may be applied to any line that is already in the proof. 

Obviously, the reiteration rule will not allow us to show anything \emph{new}. For that, we will need more rules. The remainder of this section will give six basic introduction and elimination rules. This will be enough to do some basic proofs in SL. Sections 4.3 through 4.5 will explain introduction rules involved in fancier kinds of derivation called conditional proof and indirect proof. The remaining sections of this chapter will develop our system of natural deduction further and give you tips for playing in it.

All of the rules introduced in this chapter are summarized starting on p.\pageref{ProofRules}.

\subsection{Conjunction}

Think for a moment: What would you need to show in order to prove $E \eand F$?

Of course, you could show $E \eand F$ by proving $E$ and separately proving $F$. This holds even if the two conjuncts are not atomic sentences. If you can prove $[(A \eor J) \eif V]$ and  $[(V \eif L) \eiff (F \eor N)]$, then you have effectively proved $[(A \eor J) \eif V] \eand [(V \eif L) \eiff (F \eor N)].$
So this will be our conjunction introduction rule, which we abbreviate {\eand}I:

\begin{multicols}{2}

\begin{proof}
	\have[m]{a}{\script{A}}
	\have[n]{b}{\script{B}}
	\have[\ ]{c}{\script{A}\eand\script{B}} \ai{a, b}
\end{proof}

\begin{proof}
	\have[m]{a}{\script{A}}
	\have[n]{b}{\script{B}}
	\have[\ ]{c}{\script{B}\eand\script{A}} \ai{a, b}
\end{proof}

\end{multicols}

A line of proof must be justified by some rule, and here we have ``{\eand}I $m$, $n$.'' This means: Conjunction introduction applied to line $m$ and line $n$. Again, these are variables, not real line numbers; $m$ is some line and $n$ is some other line. If you have $K$ on line 8 and $L$ on line 15, you can prove $(K\eand L)$ at some later point in the proof with the justification ``{\eand}I 8, 15.'' 

We have written two versions of the rule to indicate that you can write the conjuncts in any order. Even though $K$ occurs before $L$ in the proof, you can derive $(L \eand K)$ from them using the right-hand version {\eand}I. You do not need to mark this in any special way in the proof.

Now, consider the elimination rule for conjunction. What are you entitled to conclude from a sentence like $E \eand F$? Surely, you are entitled to conclude $E$; if $E \eand F$ were true, then $E$ would be true. Similarly, you are entitled to conclude $F$. This will be our conjunction elimination rule, which we abbreviate {\eand}E:

\begin{multicols}{2}
\begin{proof}
	\have[m]{ab}{\script{A}\eand\script{B}}
	\have[\ ]{a}{\script{A}} \ae{ab}
\end{proof}

\begin{proof}
	\have[m]{ab}{\script{A}\eand\script{B}}
	\have[\ ]{a}{\script{B}} \ae{ab}
\end{proof}
\end{multicols}

When you have a conjunction on some line of a proof, you can use {\eand}E to derive either of the conjuncts. Again, we have written two versions of the rule to indicate that it can be applied to either side of the conjunction. The {\eand}E rule requires only one sentence, so we write one line number as the justification for applying it. For example, both of these moves are acceptable in derivations. 

\begin{multicols}{2}
\begin{proof}
\have[4]{4}{A \eand (B \eor C)}
\have[5]{5}{A} \ae{4}
\end{proof}

\begin{proof}
\have[10]{10}{A \eand (B \eor C)}
\have[\ldots]{...}{\ldots}
\have[15]{15}{(B \eor C)} \by {\eand E}{10}
\end{proof}
\end{multicols}
Some textbooks will only let you use \eand E on one side of a conjunction. They then make you \emph{prove} that it works for the other side. We won't do this, because it is a pain in the neck. 

Even with just these two rules, we can provide some proofs. Consider this argument.
\begin{earg}
\item[] $[(A\eor B)\eif(C\eor D)] \eand [(E \eor F) \eif (G\eor H)]$
\item[$\therefore$] $[(E \eor F) \eif (G\eor H)] \eand [(A\eor B)\eif(C\eor D)]$
\end{earg}
The main logical operator in both the premise and conclusion is a conjunction. Since the conjunction is symmetric, the argument is obviously valid. In order to provide a proof, we begin by writing down the premise. After the premises, we draw a horizontal line---everything below this line must be justified by a rule of proof. So the beginning of the proof looks like this:

\begin{proof}
	\hypo{ab}{{[}(A\eor B)\eif(C\eor D){]} \eand {[}(E \eor F) \eif (G\eor H){]}}
\end{proof}

From the premise, we can get each of the conjuncts by {\eand}E. The proof now looks like this:

\begin{proof}
	\hypo{ab}{{[}(A\eor B)\eif(C\eor D){]} \eand {[}(E \eor F) \eif (G\eor H){]}}
	\have{a}{{[}(A\eor B)\eif(C\eor D){]}} \ae{ab}
	\have{b}{{[}(E \eor F) \eif (G\eor H){]}} \ae{ab}
\end{proof}

The rule {\eand}I requires that we have each of the conjuncts available somewhere in the proof. They can be separated from one another, and they can appear in any order. So by applying the {\eand}I rule to lines 3 and 2, we arrive at the desired conclusion. The finished proof looks like this:

\begin{proof}
	\hypo{ab}{{[}(A\eor B)\eif(C\eor D){]} \eand {[}(E \eor F) \eif (G\eor H){]}}

	\have{a}{{[}(A\eor B)\eif(C\eor D){]}} \ae{ab}
	\have{b}{{[}(E \eor F) \eif (G\eor H){]}} \ae{ab}
	\have{ba}{{[}(E \eor F) \eif (G\eor H){]} \eand {[}(A\eor B)\eif(C\eor D){]}} \ai{b,a}
\end{proof}

This proof is trivial, but it shows how we can use rules of proof together to demonstrate the validity of an argument form. Also: Using a truth table to show that this argument is valid would have required a staggering 256 lines, since there are eight sentence letters in the argument.

%When we defined a wff, we did not allow for conjunctions with more than two conjuncts. If we had done so, then we could define a more general version of the rules of proof for conjunction.


\subsection{Disjunction}
If $M$ were true, then $M \eor N$ would also be true. So the disjunction introduction rule ({\eor}I) allows us to derive a disjunction if we have one of the two disjuncts:

\begin{multicols}{2}

\begin{proof}
	\have[m]{a}{\script{A}}
	\have[\ ]{ab}{\script{A}\eor\script{B}}\oi{a}
\end{proof}

\begin{proof}
	\have[m]{a}{\script{A}}
	\have[\ ]{ab}{\script{B}\eor\script{A}}\oi{a}
\end{proof}

\end{multicols}

Like the rule of conjunction elimination, this rule can be applied two ways. Also notice that \script{B} can be \emph{any} sentence whatsoever. So the following is a legitimate proof:

\begin{proof}
	\hypo{m}{M}
	\have{mmm}{M \eor ([(A\eiff B) \eif (C \eand D)] \eiff [E \eand F])}\oi{m}
\end{proof}

This might seem odd. How can we prove a sentence that includes $A$, $B$, and the rest, from the simple sentence $M$---which has nothing to do with the other letters? The secret here is to remember that all the new letters are on just one side of a disjunction, and nothing on that side of the disjunction has to be true. As long as $M$ is true, we can add whatever we want after a disjunction and the whole thing will continue to be true. 

Now consider the disjunction elimination rule. What can you conclude from $M \eor N$? You cannot conclude $M$. It might be $M$'s truth that makes $M \eor N$ true, as in the example above, but it might not. From $M \eor N$ alone, you cannot conclude anything about either $M$ or $N$ specifically. If you also knew that $N$ was false, however, then you would be able to conclude $M$.

\begin{multicols}{2}
\begin{proof}
	\have[m]{ab}{\script{A}\eor\script{B}}
	\have[n]{nb}{\enot\script{B}}
	\have[\ ]{a}{\script{A}} \oe{ab,nb}
\end{proof}

\begin{proof}
	\have[m]{ab}{\script{A}\eor\script{B}}
	\have[n]{na}{\enot\script{A}}
	\have[\ ]{b}{\script{B}} \oe{ab,nb}
\end{proof}
\end{multicols}

We've seen this rule before: it is just disjunctive syllogism. Now that we are using a system of natural deduction, we are going to make it our rule for disjunction elimination ({\eor}E). Once again, the rule works on both sides of the sentential connective. 

\subsection{Conditionals and biconditionals}

The rule for conditional introduction is complicated because it requires a whole new kind of proof, called conditional proof. We will deal with this in the next section. For now, we will only use the rule of conditional elimination.

Nothing follows from $M\eif N$ alone, but if we have both $M \eif N$ and $M$, then we can conclude $N$. This is another rule we've seen before: modus ponens. It now enters our system of natural deduction as the conditional elimination rule ({\eif}E).

\begin{proof}
	\have[m]{ab}{\script{A}\eif\script{B}}
	\have[n]{a}{\script{A}}
	\have[\ ]{b}{\script{B}} \ce{ab,a}
\end{proof}

Biconditional elimination ({\eiff}E) will be a double-barreled version of conditional elimination. If you have the left-hand subsentence of the biconditional, you can derive the right-hand subsentence. If you have the right-hand subsentence, you can derive the left-hand subsentence. This is the rule:

\begin{multicols}{2}
\begin{proof}
	\have[m]{ab}{\script{A}\eiff\script{B}}
	\have[n]{a}{\script{A}}
	\have[\ ]{b}{\script{B}} \be{ab,a}
\end{proof}

\begin{proof}
	\have[m]{ab}{\script{A}\eiff\script{B}}
	\have[n]{a}{\script{B}}
	\have[\ ]{b}{\script{A}} \be{ab,a}
\end{proof}
\end{multicols}

\subsection{Invalid argument forms}

%rob: I added this brief subsection to make clear what was going to happen in the first problem part, and to re-emphasize the idea of invalid arguments

In section 4.1, in the last two problem parts  (p. \pageref{sec4.1partC}), we saw that sometimes an argument looks like a legitimate substitution instance of a valid argument form, but really isn't.  For instance, the problem set C asked you to identify instances of modus ponens. Below I'm giving you two of the answers.
 
\begin{multicols}{2}
(5) Modus ponens
	\begin{earg}
	\item[1.] $(C \eand L) \eif (E \eor C)$
	\item[2.] $C \eand L$
	\item[] \textcolor{white}{.}\sout{\hspace{.5\linewidth}} \textcolor{white}{.} 
	\item[$\therefore$] $E \eor C$
	\end{earg}
(7) \emph{Not} modus ponens.
	\begin{earg} 
	\item[1.] $D \eif E$
	\item[2.] $E$
\item[] \textcolor{white}{.}\sout{\hspace{.2\linewidth}} \textcolor{white}{.} 
	\item[$\therefore$] $D$
	\end{earg}
\end{multicols}
The argument on the left is an example of a valid argument, because it is an instance of modus ponens, while the argument on the right is an example of an invalid argument, because it is not an example of modus ponens. (We originally defined the terms valid and invalid on p. \pageref{def:valid}). Arguments like the one on the right, which try to trick you into thinking that they are instances of valid arguments, are called \define{deductive fallacies}. The argument on the right is specifically called the fallacy of \define{affirming the consequent}. In the system of natural deduction we are using in this textbook, modus ponens has been renamed ``conditional elimination,'' but it still works the same way. So you will need to be on the lookout for deductive fallacies like affirming the consequent as you construct proofs. 

\subsection{Notation}

The rules we have learned in this chapter give us enough to start doing some basic derivations in SL. This will allow us to prove things syntactically which would have been too cumbersome to prove using the semantic method of truth tables. We now need to introduce a few more symbols to be clear about what methods of proof we are using. 

In Chapter 1, we used the three dots $\therefore$ to indicate generally that one thing followed from another. In chapter 3 we introduced the double turnstile, $\sdtstile{}{}$, to indicate that one statement could be proven some others using truth tables. Now we are going to use a single turnstile, $\sststile{}{}$, to indicate that we can derive a statement from a bunch of premises, using the system of natural deduction we have begun to introduce in this section. Thus we will write $\{\script{A}, \script{B}, \script{C}\} \sststile{}{} \script{D}$, to indicate that there is a derivation going from the premises \script{A}, \script{B}, and \script{C} to the conclusion \script{D}. Note that these are metavariables, so I could be talking about any sentences in SL.

The single turnstile will work the same way the double turnstile did. So, in addition to the uses of the single turnstile above we can write  $\sststile{}{} \script{A}$ to indicate that \script{A} can be proven a tautology using syntactic methods. We can write $\script{A}\nsststile{}{} \hspace{.5em}  \sststile{}{}\script{B}$ to say that \script{A} and \script{B} can be proven logically equivalent using these derivations. You will learn how to do these later things at the end of the chapter. In the meantime, we need to practice our basic rules of derivation.

%%%%%%%%%% PRACTICE PROBLEMS %%%%%%
\practiceproblems
\noindent\problempart Some of the following arguments are legitimate instances of our six basic inference rules. The others are either invalid arguments or valid arguments that are still illegitimate because they would take multiple steps using our basic inference rules. For those that are legitimate, mark the rule that they are instances of. Mark those that are not ``Not a single inference.'' 

\begin{exercises}
\begin{longtabu}{X[1]X[1]} 

\item %1
	\begin{earg*}
	\item $R \eor S$ 
\itemc[.3] $S$
	\end{earg*}
\answer{\factoidbox{Not a single inference}}

&

\item %2
	\begin{earg*}
	\item $(A \eif B) \eor (B \eif A)$
	\item $A \eif B$
\itemc[.3] $B \eif A$
	\end{earg*}

\answer{\factoidbox{Not a single inference}}

\\

\item %3
	\begin{earg*}
	\item $P \eand (Q \eor R)$
\itemc[.3] $R$
	\end{earg*}
\answer{\factoidbox{Not a single inference}}

&
\item %4
	\begin{earg*}
	\item $P \eand (Q \eand R)$
\itemc[.3] $P$
	\end{earg*}
\answer{		\factoidbox{\eand-Elimination}}

\\
\item %5
	\begin{earg*}
	\item  $A$
\itemc[.3] $P \eand (Q \eif A)$
	\end{earg*}
\answer{\factoidbox{Not a single inference}}

&	
	
\item %6
	\begin{earg*}
	\item  $A$
	\item  $B \eand C$
\itemc[.3] $(A \eand B) \eand C$
	\end{earg*}
\answer{\factoidbox{\begin{flushleft}Not a single inference. You need the associativity of \eand to infer this.\end{flushleft}}}
\\
\item %7
	\begin{earg*}
	\item $(X \eand Y) \eiff (Z \eand W)$
	\item $Z \eand W$
\itemc[.3] $X \eand Y$
	\end{earg*}
\answer{	\factoidbox{\eiff-Elimination}}
&
\item %8
	\begin{earg*}
	\item $((L \eif M) \eif N) \eif O$
	\item $L$
\itemc[.3] $M$
	\end{earg*}
\answer{	\factoidbox{Not a single inference}}

\end{longtabu}
\end{exercises}


\noindent\problempart Some of the following arguments are legitimate instances of our six basic inference rules. The others are either invalid arguments or valid arguments that are still illegitimate because they would take multiple steps using our basic inference rules. For those that are legitimate, mark the rule that they are instances of. Mark those that are not ``Not a single inference.''

\begin{exercises} \vspace{-.5cm}
\begin{longtabu}{X[1]X[1]} 

\item %1
	\begin{earg*}
	\item  $A \eand B$
 
	\itemc[.3]$A$ 	
	\end{earg*}
%\factoidbox{\eand-Elimination}
&

\item %2
	\begin{earg*}
	\item $A \eif (B \eand (C \eor D))$
	\item $A$
 
	\itemc[.3]$B \eand (C \eor D)$
	\end{earg*}
%\factoidbox{\eif-Elimination}
\\

\item %3
	\begin{earg*}
	\item $P \eand (Q \eand R)$
 
	\itemc[.3]$R$
	\end{earg*}
%	\factoidbox{\begin{flushleft}Not a single inference. You need two uses of \eand-elim. to do this\end{flushleft}}
&

\item %4
	\begin{earg*}
	\item  $P$
 
	\itemc[.3] $P \eor [A \eand (B \eiff C)]$
	\end{earg*}
%	\factoidbox{\eor-Introduction}
\\

\item %5
	\begin{earg*}
	\item  $M$
	\item  $D \eand C$
 
	\itemc[.3] $M \eand (D \eand C)$
	\end{earg*}
%	\factoidbox{\eand-Introduction}
&

\item %6
	\begin{earg*}
	\item $(X \eand Y) \eif (Z \eand W)$
	\item $Z \eand W$
 
	\itemc[.3]$X \eand Y$
	\end{earg*}
%	\factoidbox{Not a single inference}
\\

\item %7
	\begin{earg*}
	\item $(X \eand Y) \eif (Z \eand W)$
	\item $\enot (X \eand Y)$
 
	\itemc[.3]$\enot(Z \eand W)$
	\end{earg*}
%	\factoidbox{Not a single inference}
&

\item %8
	\begin{earg*}
	\item $((L \eif M) \eif N) \eif O$
	\item $(L \eif M) \eif N$
 
	\itemc[.3]$O$
	\end{earg*}
%	\factoidbox{\eif-Elimination}

\end{longtabu}
\end{exercises}

\vspace{-8pt}

\noindent\problempart \label{pr.justifySLproof} Fill in the missing pieces in the following proofs. Some are missing the justification column on the right. Some are missing the left column that contains the actual steps, and some are missing lines from both columns.
%rob: problem one was in the original problem section at the end of Chapter 6. 

\begin{exercises}
\vspace{-.5cm}
\begin{longtabu}{X[1.4]X[1]} 

\item \textcolor{white}{.}  
\vspace{-16pt}
\begin{proof}
	\hypo{1}{W \eif \enot B}
	\hypo{2}{A \eand W}
	\hypo{3}{B \eor (J \eand K)} \by{Want: $K$}{}
	\have{4}{W}{} \iflabelexists{showanswers}{\by{\color{red}\eand E,}{2}}{}
	\have{5}{\enot B} {} \iflabelexists{showanswers}{\by{\color{red}\eif E,} {1,4}}{}
	\have{6}{J \eand K} {} \iflabelexists{showanswers}{\by{\color{red}\eor E}{3,5}}{}
	\have{7}{K}{} \iflabelexists{showanswers}{\by{\color{red}\eand E}{6}}{}
	\end{proof}

&

\item \textcolor{white}{.} 
\vspace{-16pt}

	\begin{proof}
	\hypo{1}{W \eand B}
	\hypo{2}{E \eand Z} \by{Want: $W \eand Z$}{}
	\have{3}{\iflabelexists{showanswers}{\color{red}W}{}} \by{\eand E}{1}
	\have{4}{\iflabelexists{showanswers}{\color{red}Z}{}} \by{\eand E}{2}
	\have{5}{W \eand Z} \iflabelexists{showanswers}{\by{\color{red}\eand I}{3, 4}}{}
	\end{proof}



\\

\item \textcolor{white}{.} 
\vspace{-16pt}
	\begin{proof}
	\hypo{1}{(A \eand B) \eand C} \by{Want: $A \eand (B \eand C)$}{}
	\have{2}{A \eand B} \iflabelexists{showanswers}{\by{\color{red}\eand E}{1}}{}	
	\have{3}{C} \iflabelexists{showanswers}{\by{\color{red}\eand E}{1}}{}
	\have{4}{A} \iflabelexists{showanswers}{\by{\color{red}\eand E}{2}}{}
	\have{5}{B} \iflabelexists{showanswers}{\by{\color{red}\eand E}{2}}{}
	\have{6}{B \eand C} \iflabelexists{showanswers}{\by{\color{red}\eand I}{3,5}}{}
	\have{7}{A \eand (B \eand C)} \iflabelexists{showanswers}{\by{\color{red}\eand I}{4,6}}{}
	\end{proof}

&

\item \textcolor{white}{.}  
\vspace{-16pt}
\begin{proof}
	\hypo{1}{(\enot A \eand B) \eif C}
	\hypo{2}{\enot A}
	\hypo{3}{A \eor B} \by{Want: $C$}{}
	\have{4}{\iflabelexists{showanswers}{\color{red}B}{}} \by{\eor E}{2, 3}
	\have{5}{\iflabelexists{showanswers}{\color{red}\enot A \eand B}{}} \by{\eand I}{2, 4}
	\have{6}{\iflabelexists{showanswers}{\color{red}C}{}} \by{\eif E}{1, 5}
	\end{proof} 


%\iflabelexists{showanswers}{\by{\color{red}foo}{bar}}{}
%\iflabelexists{showanswers}{\color{red}Foo}{}
\\
\end{longtabu}


\vspace{-1cm}
\item \textcolor{white}{.}  
\vspace{-16pt}
	\begin{proof}
	\hypo{1}{\enot A \eand (\enot B \eand C)}
	\hypo{2}{C \eif (D \eand (B \eor E))}
	\hypo{3}{(E \eand \enot A) \eif F}	\by{Want: $D \eand F$}{} 
	\have{4}{\iflabelexists{showanswers}{\color{red}\enot A}{}} \by{\eand E}{1} 
	\have{5}{\enot B \eand C} \iflabelexists{showanswers}{\by{\color{red}\eand E}{1}}{} %
	\have{6}{\iflabelexists{showanswers}{\color{red}\enot B}{}} \by{\eand E}{5}
	\have{7}{C} \iflabelexists{showanswers}{\by{\color{red}\eand E}{5}}{} %
	\have{8}{\iflabelexists{showanswers}{\color{red}D \eand (B \eor E)}{}} \by{\eif E}{2, 7} 
	\have{9}{D}\iflabelexists{showanswers}{\by{\color{red}\eand E}{8}}{} %
	\have{10}{\iflabelexists{showanswers}{\color{red}B \eor E}{}} \by{\eand E}{8} 
	\have{11}{E} \iflabelexists{showanswers}{\by{\color{red}\eor  E}{6, 10}}{}
	\have[12]{12}{\iflabelexists{showanswers}{\color{red}E \eand \enot A}{}} \by{\eand I}{4, 11} 
	\have[13]{13}{F} \iflabelexists{showanswers}{\by{\color{red}\eif E}{3, 12}}{}
	\have[14]{14}{\iflabelexists{showanswers}{\color{red}D \eand F}{}} \by{\eand I}{9, 13} 
	\end{proof}

\end{exercises}



\noindent\problempart \label{pr.justifySLproof} Fill in the missing pieces in the following proofs. Some are missing the justification column on the right. Some are missing the left column that contains the actual steps, and some are missing lines from both columns.

\begin{exercises}
\begin{longtabu}{X[1]X[1]} 

\item \textcolor{white}{.}  
\vspace{-16pt}
	\begin{proof}
	\hypo{1}{A \eand \enot B}
	\hypo{2}{A \eif \enot C}
	\hypo{3}{B \eor (C \eor D)}	 \by{Want: $D$}{}
	\have{4}{} \by{\eand E}{1}
	\have{5}{} \by{\eand E}{1}
	\have{6}{} \by{\eif E}{2, 4}
	\have{7}{} \by{\eor E}{3, 5}
	\have{8}{} \by{\eor E}{6,7}
	\end{proof}
&
\item \textcolor{white}{.}  
\vspace{-16pt}
	\begin{proof}
	\hypo{1}{W \eor V}
	\hypo{2}{I \eand (\enot Z \eif \enot W)}
	\hypo{3}{I \eif \enot Z} \by{Want: $I \eand V$}{}
	\have{4}{} \ae{2}
	\have{5}{} \ae{2}
	\have{6}{\enot Z} \by{}{}
	\have{7}{} \by{\eif E}{5, 6}
	\have{8}{V} \by{}{}
	\have{9}{} \ai{4,8}
	\end{proof}
\\
\item \textcolor{white}{.}  
\vspace{-16pt}
		
\begin{proof}
\hypo{1}{\enot P \eand S) \eiff S}
\hypo{2}{S \eand (P \eor Q)} \by{Want: Q}{}
\have{3}{S} \nix{\by{\eand E}{2}}
\have{4}{P \eor Q} \nix{\by{\eand E}{2}}
\have{5}{\enot P \eand S} \nix{\by{\eiff E}{1, 3}}
\have{6}{\enot P} \nix{\by{\eand E}{5}}
\have{7}{Q} \nix{\by{\eor E}{4, 6}}
\end{proof}

&
\item \textcolor{white}{.}  
\vspace{-16pt}
\begin{proof}
\hypo{1}{C \eif (A \eif B)}
\hypo{2}{D \eor C}
\hypo{3}{\enot D} \by{Want: A \eif B}{}
\have{4}{\nix{C}} \by{\eor E}{2, 3}
\have{5}{\nix{A \eif B}} \by{\eif E}{1, 4}
\end{proof}
\\

\item \textcolor{white}{.}  
\vspace{-16pt}

	\begin{proof}
	\hypo{1}{X \eand (Y \eand Z)} \by{Want: $(X \eor A) \eand [(Y \eor B) \eand (Z \eand C)]$} {}
	\have{2}{} \ae{1}
	\have{3}{} \ae{1}
	\have{4}{} \ae{3}
	\have{5}{} \ae{3}
	\have{6}{} \oi{2}
	\have{7}{} \oi{4}
	\have{8}{} \oi{5}
	\have{9}{} \by{\eand I}{7,8}
	\have{10}{} \ai{6,9}
	\end{proof}

\end{longtabu}
\end{exercises}

%%%%%PART E

\noindent\problempart Derive the following.

\begin{enumerate}[label=(\arabic*)]
\item \{$A \eif B, A\} \sststile{}{} A \eand B$

\answer{
	\begin{proof}
	\hypo{1}{A \eif B}
	\hypo{2}{A} \by{Want: A \eand B}{}
	\have{3}{B} \by{\eif E}{1,2}
	\have{4}{A \eand B} \ai{2,3}
	\end{proof}
}
\item \{$A \eiff D, C, [(A \eiff D) \eand C] \eif (C \eiff B)\} \sststile{}{} B$

\answer{
\begin{proof}
\hypo{1}{A \eiff D}
\hypo{2}{C}
\hypo{3}{((A \eiff D) \eand C) \eif (C \eiff B)} \by{Want: B}{} 
\have{4}{(A \eiff D) \eand C} \by{\eand I}{1, 2}
\have{5}{C \eiff B} \by{\eif E}{3, 4}
\have{6}{B} \by{\eiff E}{2, 5}
\end{proof}
}

\item \{$A \eiff B, B \eiff C, C \eif D, A\} \sststile{}{} D$

\answer{
	\begin{proof}
	\hypo{1}{A \eiff B}
	\hypo{2}{B \eiff C}
	\hypo{3}{C \eiff D} 
	\hypo{4}{A}	\by{Want: D}{}
	\have{5}{B} \be{1, 4}
	\have{6}{C} \be{2, 5}
	\have{7}{D} \be{3, 6}
	\end{proof}
}

\item $\{(A \eif \enot B) \eand A, B \eor C\} \sststile{}{} C$

\answer{
\begin{proof}
\hypo{1}{(A \eif \enot B) \eand A}
\hypo{2}{B \eor C} \by{Want: C}{}
\have{3}{A \eif \enot B} \by{\eand E}{1}
\have{4}{A}\by{\eand E}{1}
\have{5}{\enot B} \by{\eif E}{3, 4}
\have{6}{C} \by{\eor E}{2, 5}
\end{proof} 
}

\item $\{(A \eif B) \eor (C \eif (D \eand E)), \enot (A \eif B), C\} \sststile{}{} D$

\answer{
	\begin{proof}
	\hypo{1}{(A \eif B) \eor (C \eif (D \eand E))} 
	\hypo{2}{\enot (A \eif B)}
	\hypo{3}{C} \by{Want: D}{}
	\have{4}{C \eif (D \eand E)} \oe{1, 3}
	\have{5}{D \eand E} \ce{3, 4}
	\have{6}{D} \ae{5}
	\end{proof}
}

\item $\{C \eor (B \eand  A),  \enot C\} \sststile{}{} A \eor A$		%requires \eorI

\answer{
\begin{proof}
\hypo{1}{C \eor (B \eand A)}
\hypo{2}{\enot C} \by{Want: A \eor A}{}
\have{3}{B \eand A} \oe{1, 2}
\have{4}{A} \ae{3}
\have{5}{A \eor A} \oe{4}
\end{proof}
}

\item $\{A \eor B, \enot A, \enot B\} \sststile{}{} C$				%\eorIE trick

\answer{
	\begin{proof}
	\hypo{1}{A \eor B}
	\hypo{2}{\enot A}
	\hypo{3}{\enot B} \by{Want: C}{}
	\have{4}{B} \oe{1, 2}
	\have{5}{B \eor C} \oi{4}
	\have{6}{C} \oe{3, 5}
	\end{proof}
}
\end{enumerate}

%%%%PART F

\noindent\problempart Derive the following.
\begin{enumerate}[label=(\arabic*)]

\item $\{A \eand B, B \eif C\} \sststile{}{} A \eand (B \eand C) $ %1

%\begin{proof}
%\hypo{1}{A \eand B}
%\hypo{2}{B \eif C}	\by{Want: A \eand (B \eand C)}{}
%\have{3}{A} \ae{1}
%\have{4}{B} \ae{1}
%\have{5}{C} \ce{2, 4}
%\have{6}{B \eand C} \ai{4, 5}
%\have{7}{A \eand (B \eand C)} \ai{3, 6}
%\end{proof}

\item $\{(P \eor R) \eand (S \eor R), \enot R \eand Q\} \sststile{}{} P \eand (Q \eor R)$		%2
\item $\{(X \eand Y) \eif Z, X \eand W, W \eif Y\} \sststile{}{} Z$ 	%3

%\begin{proof}
%\hypo{1}{(X \eand Y) \eif Z}
%\hypo{2}{X \eand W}
%\hypo{3}{W \eif Y} \by{Want: Z}{}
%\have{4}{X} \ae{2}
%\have{5}{W} \ae{2}
%\have{6}{Y} \ce{3, 5}
%\have{7}{X \eand Y} \ai{4, 6}
%\have{8}{Z} \ce{1, 7}
%\end{proof}

\item $\{A \eor  (B \eor  G), A \eor  (B \eor  H), \enot A \eand \enot B\} \sststile{}{} G \eand H $		%4

\item $\{P \eand (Q \eand \enot R), R \eor T\} \sststile{}{} T \eor S$ 		%requires \eorI
\item $\{((A \eif D) \eor B) \eor C, \enot C, \enot B, A\} \sststile{}{} D$
\item $\{A \eor \enot\enot B, \enot B \eor \enot C, C \eor A, \enot A\} \sststile{}{}D		$			%\eorIE trick
\end{enumerate}

%%%%%%%% PART G
\noindent\problempart Derive the following.
\begin{enumerate}[label=(\arabic*)]

\item $H \eand A \sststile{}{} A \eand H	$

\answer{
\begin{proof}
\hypo{1}{H \eand A} \by{Want: A \eand H}{}
\have{2}{H} \by{\eand E}{1}
\have{3}{A} \by{\eand E}{1}
\have{4}{A \eand H} \by{\eand I}{2, 3}
\end{proof}
}

\item $\{{P \eor Q, D \eif E, \enot P \eand D} \} \sststile{}{} E \eand Q$

\answer{
\begin{proof}
\hypo{1}{P \eor Q}
\hypo{2}{D \eif E}
\hypo{3}{~P \eand D}  \by{Want: E \eand Q}{}
\have{4}{~P} \by{\eand E}{3}
\have{5}{D} \by{\eand E 3}{}
\have{6}{Q} \by{\eor E}{1, 4}
\have{7}{E	} \by{\eif E}{2, 5}
\have{8}{E \eand Q} \by{\eand E}{6, 7}
\end{proof}
}

\item $\{\enot A \eif (A \eor \enot C), \enot A, \enot C \eiff D \} \sststile{}{} D$

\answer{
\begin{proof}
\hypo{1}{~A \eif (A \eor ~C)}
\hypo{2}{~A}
\hypo{3}{~C \eiff D} \by{Want: D}{}
\have{4}{A \eor ~C} \by{\eif E}{1, 2}
\have{5}{~C} \by{\eor E} {2, 4}
\have{6}{D} \by{\eiff E}{3, 5}
\end{proof}
}

\item $\{\enot A \eand C, A \eor B, (B \eand C) \eif (D \eand E) \} \sststile{}{} D$

\answer{
\begin{proof}
\hypo{1}{~A \eand C}
\hypo{2}{A \eor B}
\hypo{3}{(B \eand C) \eif (D \eand E)} \by{Want: D}{}
\have{4}{~A} \by{\eand E}{1}
\have{5}{C} \by{\eand E}{1}
\have{6}{B} \by{\eor E}{2, 4}
\have{7}{B \eand C} \by{\eand E}{5, 6}
\have{8}{D \eand E} \by{\eif E}{3, 7}
\have{9}{D}\by{\eand E}{8}
\end{proof}
}

\item $\{A \eif (B \eif (C \eif D)), A \eand (B \eand C) \} \sststile{}{} D$

\answer{
\begin{proof}
\hypo{1}{A \eif (B \eif (C \eif D))}
\hypo{2}{A \eand (B \eand C)} \by{Want: D}{}
\have{3}{A} \by{\eand E}{2}
\have{4}{B \eand C} \by{\eand E}{2}
\have{5}{B} \by{\eand E}{4}
\have{6}{C} \by{\eand E}{4}
\have{7}{B \eif (C \eif D)} \by{\eif E}{1, 3}
\have{8}{C \eif D} \by{\eif E}{5, 7}
\have{9}{D} \by{\eif E}{6, 8}
\end{proof}
}


\item $\{E \eor F, F \eor G, \enot F\} \sststile{}{} E \eand G$

\answer{
\begin{proof}
\hypo{1}{E \eor F}
\hypo{2}{F\eor G}
\hypo{3}{\enot F} \by{Want: $E \eand G$}{}
\have{4}{E} \by{\eor E}{1, 3}
\have{5}{G} \by{\eor E}{2, 3}
\have{6}{E \eand G} \by{\eand I}{4, 5}
\end{proof}
}


\item $\{X \eand (Z \eor Y), \enot Z, Y \eif \enot X\} \sststile{}{} A$  %\eorIE trick

\answer{
\begin{proof}
\hypo{1}{X \eand (Z \eor Y)}
\hypo{2}{\enot Z}
\hypo{3}{Y \eif \enot X} \by{Want: $A$}{}
\have{4}{X} \by{\eand E}{1}
\have{5}{Z \eor Y} \by{\eand E}{1}
\have{6}{Y} \by{\eor E}{2, 5}
\have{7}{\enot X} \by{\eif E}{3, 6}
\have{8}{X \eor A} \by{\eor I}{4}
\have{9}{A} \by{\eor E}{7, 8}
\end{proof}
}

\end{enumerate}

%%%%%%PART H


\noindent\problempart Derive the following.
\begin{enumerate}[label=(\arabic*)]

\item $\{P \eiff (Q \eiff R)$,$ P$,$ P \eif R\} \sststile{}{} Q$

\item $\{A \eif (B \eif C), A, B\} \sststile{}{}C$
\item $\{(X \eor A) \eif \enot Y, Y \eor (Z \eand Q), X\} \sststile{}{}Z	$
\item $\{A \eand (B \eand C), A \eand D, B \eand E\} \sststile{}{}D \eand (E \eand C)		$
\item $\{A \eand (B \eor \enot C), \enot B \eand (C \eor E), E \eif D \} \sststile{}{} D$

%1.             A & (B ˅ ~C)
%2.             ~B & (C ˅ E)
%3.             E → D                                    Want: D
%4.             A                                                             &-E 1      
%5.             B ˅ ~C                                   &-E 1
%6.             ~B                                                           &-E 2
%7.             C ˅ E                                      &-E 2
%8              ~C                                                            ˅E 5, 6
%9.             E                                                               ˅E 7, 8
%10.           D                                                             →E 3,10

\item $\{A \eif B, B \eif C, C \eif A, B, \enot A\} \sststile{}{}D	$  %\eorIE trick
\item $\{\enot A \eand B, A \eor P, A \eor Q, B \eif R \} \sststile{}{} P \eand (Q\eand R)$


\end{enumerate}

\noindent\problempart Translate the following arguments into SL and then show that they are valid. Be sure to write out your dictionary. 
\begin{enumerate}[label=(\arabic*)]
\item If Professor Plum did it, he did it with the rope in the kitchen. Either Professor Plum or Miss Scarlett did it, and it wasn't Miss Scarlett. Therefore the murder was in the kitchen.  
%rob: note to self: problem 1 taken from test 4 SP08.

\answer{
A: Professor Plum did it \\
B: The murder was committed in the kitchen \\
C: The murder was committed with the rope\\
D: Miss Scarlett did it


\begin{proof}
\hypo{1}{A \eif (B \eand C)}
\hypo{2}{(A \eor D) \eand \enot D} \by{Want: B}{}
\have{3}{A \eor D} \ae{2}
\have{4}{\enot D} \ae{2}
\have{5}{A} \oe{3, 4}
\have{6}{B \eand C} \ce{1, 5}
\have{7}{B} \ae{6}
\end{proof}
}

\item If you are going to replace the bathtub, you might as well redo the whole bathroom. If you redo the whole bathroom, you will have to replace all the plumbing on the north side of the house. You will spend a lot of money on this project if and only if you replace the plumbing on the north side of the house. You are definitely going to replace the bathtub. Therefore you will spend a lot of money on this project. 

\answer{
A: You are going to replace the bathtub \\
B: You redo the whole bathroom.  \\
C: You replace all the plumbing on the north side of the house. \\
D: You will spend a lot of money on this project  \\ 


\begin{proof}
\hypo{1}{A \eif B}
\hypo{2}{B \eif C}
\hypo{3}{C \eiff D}
\hypo{4}{A} \by{Want: D}{}
\have{5}{B} \by{\eif E}{1, 4}
\have{6}{C} \by{\eif E}{2, 5}
\have{7}{D}  \by{\eiff E}{3, 6}
\end{proof}
}

\end{enumerate}

\noindent\problempart
Translate the following arguments into SL and then show that they are valid. Be sure to write out your dictionary. 
\begin{enumerate}[label=(\arabic*)]

\item Either Caroline is happy, or Joey is happy, but not both. If Joey teases Caroline, she is not happy. Joey is teasing Caroline. Therefore, Joey is happy.

%A: Caroline is happy. \hspace{.25in}
%B: Joey is happy.\hspace{.25in}
%C: Joey teases Caroline. \\
%
%\begin{proof}
%\hypo{1}{(A \eor B) \eand \enot(A \eand B)}
%\hypo{2}{C \eif \enot A}
%\hypo{3}{C} \by{Want: B}{}
%\have{4}{A \eor B} \ae{1}
%\have{5}{\enot A} \ce{2, 3}
%\have{6}{B} \oe{4,5}
%\end{proof}

\item Either grass is green or one of two other things: the sky is blue or snow is white. If my lawn is brown, the sky is gray, and if the sky is gray, it is not blue. If my lawn is brown, then grass is not green, and on top of that my lawn is brown. Therefore snow is white.
%rob: note to self: replace this with a better problem sometime.

\end{enumerate}

% *******************************************
% *			Conditional Proof					   *	
% *******************************************
\section{Conditional Proof}
\setlength{\parindent}{1em}
%I separated this out from the first section. 

So far we have introduced introduction and elimination rules for the conjunction and disjunction, and elimination rules for the conditional and biconditional, but we have no introduction rules for conditionals and biconditionals, and no rules at all for negations. That's because these other rules require fancy kinds of derivations that involve putting proofs inside proofs. In this section, we will look at one of these kinds of proof, called conditional proof.

%rob: added a transition paragraph.

\subsection{Conditional introduction}
Consider this argument:
\begin{earg*}
\item $R \eor F$
\itemc[.15] $\enot R \eif F$
\end{earg*}
The argument is valid. You can use the truth table to check it. Unfortunately, we don't have a way to prove it in our syntactic system of derivation. To help us see what our rule for conditional introduction should be, we can try to figure out what new rule would let us prove this obviously true argument.

Let's start the proof in the usual way, like this:

\begin{proof}
	\hypo{rf}{R \eor F} \by{Want: \enot R \eif F}{}
\end{proof}

If we had $\enot R$ as a further premise, we could derive $F$ by the {\eor}E rule. But sadly, we do not have $\enot R$ as a premise, and we can't derive it directly from the premise we do have---so we cannot simply prove $F$. What we will do instead is start a \emph{subproof}, a proof within the main proof. When we start a subproof, we draw another vertical line to indicate that we are no longer in the main proof. Then we write in an assumption for the subproof. This can be anything we want. Here, it will be helpful to assume $\enot R$. Our proof now looks like this:

\begin{proof}
	\hypo{rf}{R \eor F}\by{Want: \enot R \eif F}{}
	\open
		\hypo{nr}{\enot R}\by{Assumption for CD, Want: F}{}
	\close
\end{proof}

It is important to notice that we are not claiming to have proved $\enot R$. We do not need to write in any justification for the assumption line of a subproof. You can think of the subproof as posing the question: What could we show \emph{if} $\enot R$ were true? For one thing, we can derive $F$. To make this completely clear, I have annotated line 2 ``Assumption for CD,'' to indicate that this is an additional assumption we are making because we are using conditional derivation (CD). I have also added ``Want: F'' because that is what we will want to show during the subderivation. In the future I won't always include all this information in the annotation. But for now we will use it to be completely clear on what we will be doing.

So now let's go ahead and show F in the subderivation. 

\begin{proof}
	\hypo{rf}{R \eor F}\by{Want: \enot R \eif F}{}
	\open
		\hypo{nr}{\enot R}\by{Assumption for CD, Want: F}{}
		\have{f}{F}\oe{rf, nr}
	\close
\end{proof}

This has shown that \emph{if} we had $\enot R$ as a premise, \emph{then} we could prove $F$. In effect, we have proven $\enot R \eif F$. So the conditional introduction rule ({\eif}I) will allow us to close the subproof and derive $\enot R \eif F$ in the main proof. Our final proof looks like this:

\begin{proof}
	\hypo{rf}{R \eor F}\by{Want: \enot R \eif F}{}
	\open
		\hypo{nr}{\enot R}\by{Assumption for CD, Want: F}{}
		\have{f}{F}\oe{rf, nr}
	\close
	\have{nrf}{\enot R \eif F}\ci{nr-f}
\end{proof}

Notice that the justification for applying the {\eif}I rule is the entire subproof. Usually that will be more than just two lines.

%rob, I added a paragraph explaining the precise rules that govern subproofs and folded some later material into that paragraph

Now that we have that example, let's lay out more precisely the rules for subproofs and then give the formal schemes for the rule of conditional and biconditional introduction. 

\begin{enumerate}[leftmargin=1.5cm]
\item[\define{Rule 1}] You can start a subproof on any line, except the last one, and introduce any assumptions with that subproof.
\item[\define{Rule 2}] All subproofs must be closed by the time the proof is over.
\item[\define{Rule 3}] Subproofs may closed at any time. Once closed, they can be used to justify \eif I, \eiff I, \enot E, and \enot I.
\item[\define{Rule 4}] Nested subproofs must be closed before the outer subproof is closed.
\item[\define{Rule 5}] Once the subproof is closed, lines in the subproof cannot be used in later justifications.
\end{enumerate}

Rule 1 gives you great power. You can assume anything you want, at any time. But with great power, comes great responsibility, and rules 2--5 explain what your responsibilities are. Making an assumption creates the burden of starting a subproof, and subproofs must end before the proof is done. (That's why we can't start a subproof on the last line.) Closing a subproof is called \emph{discharging} the assumptions of that subproof. So we can summarize your responsibilities this way: You cannot complete a proof until you have discharged all of the assumptions introduced in subproofs. Once the assumptions are discharged, you can use the whole subproof as a justification, but not the individual lines. So you need to know going into the subproof what you are going to use it for once you get out. As in so many parts of life, you need an exit strategy.  

With those rules for subproofs in mind, the {\eif}I rule looks like this:

\begin{proof}
	\open
		\hypo[m]{a}{\script{A}} \by{want \script{B}}{}
		\have[n]{b}{\script{B}}
	\close
	\have[\ ]{ab}{\script{A}\eif\script{B}}\ci{a-b}
\end{proof}

You still might think this gives us too much power. In logic, the ultimate sign you have too much power is that given any premise \script{A} you can prove any conclusion \script{B}. Fortunately, our rules for subproofs don't let us do this. Imagine a proof that looks like this:

\begin{proof}
	\hypo{a}{\script{A}}
	\open
		\hypo{b1}{\script{B}}
\end{proof}

It may seem as if a proof like this will let you reach any conclusion \script{B} from any premise \script{A}. But this is not the case. By rule 2, in order to complete a proof, you must close all of the subproofs, and we haven't done that. A subproof is only closed when the vertical line for that subproof ends. To put it another way, you  can't end a proof and still have two vertical lines going. 

You still might think this system gives you too much power. Maybe we can try closing the subproof and writing \script{B} in the main proof, like this 

\begin{proof}
	\hypo{a}{\script{A}}
	\open
		\hypo{b1}{\script{B}}
		\have{b2}{\script{B}} \by{R}{b1}
	\close
	\have{b}{\script B} \by{R}{b2}
\end{proof}

But this is wrong, too. By rule 5, once you close a subproof, you cannot refer back to individual lines inside it.

Of course, it is legitimate to do this:

\begin{proof}
	\hypo{a}{\script{A}}
	\open
		\hypo{b1}{\script{B}}
		\have{b2}{\script{B}} \by{R}{b1}
	\close
	\have{bb}{\script{B}\eif\script{B}} \ci{b1-b2}
\end{proof}

This should not seem so strange, though. Since \script{B}\eif\script{B} is a tautology, no particular premises should be required to validly derive it. (Indeed, as we will see, a tautology follows from any premises.)

When we introduce a subproof, we typically write what we want to derive in the right column, just like we did in the first example in this section. This is just so that we do not forget why we started the subproof if it goes on for five or ten lines. There is no ``want'' rule. It is a note to ourselves and not formally part of the proof.

Having an exit strategy when you launch a subproof is crucial. Even if you discharge an assumption properly, you might wind up with a final line that doesn't do you any good. In order to derive a conditional by {\eif}I, for instance, you must assume the antecedent of the conditional in a subproof. The last line of the subproof must be the consequent of the conditional, and the whole conditional is the first line after the end of the subproof. Pick your assumptions so that you wind up with a conditional that you actually need. It is always permissible to close a subproof and discharge its assumptions, but it will not be helpful to do so until you get what you want.

%This is also moved from the conditional section

Now that we have the rule for conditional introduction, consider this argument:
\label{proofHS}
\begin{earg*}
\item $P \eif Q$
\item $Q \eif R$
\itemc[.15] $P \eif R$
\end{earg*}
We begin the proof by writing the two premises as assumptions. Since the main logical operator in the conclusion is a conditional, we can expect to use the {\eif}I rule. For that, we need a subproof---so we write in the antecedent of the conditional as an assumption of a subproof:

\begin{proof}
	\hypo{pq}{P \eif Q}
	\hypo{qr}{Q \eif R}
	\open
		\hypo{p}{P}
	\close
\end{proof}

We made $P$ available by assuming it in a subproof, allowing us to use {\eif}E on the first premise. This gives us $Q$, which allows us to use {\eif}E on the second premise. Having derived  $R$, we close the subproof. By assuming $P$ we were able to prove $R$, so we apply the {\eif}I rule and finish the proof.

\label{HSproof}
\begin{proof}
	\hypo{pq}{P \eif Q}
	\hypo{qr}{Q \eif R}
	\open
		\hypo{p}{P}\by{want $R$}{}
		\have{q}{Q}\ce{pq,p}
		\have{r}{R}\ce{qr,q}
	\close
	\have{pr}{P \eif R}\ci{p-r}
\end{proof}


\subsection{Biconditional introduction}

Just as the rule for biconditional elimination was a double-headed version of conditional elimination, our rule for biconditional introduction is a double-headed version of conditional introduction. In order to derive $W \eiff X$, for instance, you must be able to prove $X$ by assuming $W$ \emph{and} prove $W$ by assuming $X$. The biconditional introduction rule ({\eiff}I) requires two subproofs. The subproofs can come in any order, and the second subproof does not need to come immediately after the first---but schematically, the rule works like this:

\begin{proof}
	\open
		\hypo[m]{a1}{\script{A}} \by{want \script{B}}{}
		\have[n]{b1}{\script{B}}
	\close
	\open
		\hypo[p]{b2}{\script{B}} \by{want \script{A}}{}
		\have[q]{a2}{\script{A}}
	\close
	\have[\ ]{ab}{\script{A}\eiff\script{B}}\bi{a1-b1,b2-a2}
\end{proof}

We will call any proof that uses subproofs and either \eif I or \eiff I \define{conditional proof}. By contrast, the first kind of proof you learned, where you only use the six basic rules, will be called \define{direct proof}. In section 4.3 we will learn the third and final  kind of proof \emph{indirect proof}. But for now you should practice conditional proof. 

%%%%%%Practice Problems %%%%%%%%%%%%%%%
\practiceproblems
\noindent\problempart Fill in the blanks in the following proofs. Be sure to include the ``Want'' line for each subproof.  %!@#$

\begin{exercises}
\setlength\itemsep{1cm}
\item  \textcolor{white}{.} % $\{\enot P \eif (Q \eor R), P \eor \enot Q\} \sststile{}{} \enot P \eif R$
\vspace{-16pt}
\begin{proof}
\hypo{1}{\enot P \eif (Q \eor R)}
\hypo{2}{P \eor \enot Q}  \by{Want: $\enot P \eif R$}{}
\open
\hypo{3}{\enot P} \by{Want: \iflabelexists{showanswers}{\color{red} R}{}}{} 
\have{4}{Q \eor R}   \iflabelexists{showanswers}{\by{\color{red} \eif E}{1, 3}}{}
\have{5}{\iflabelexists{showanswers}{\color{red}\enot Q}{}} \oe{2, 3}
\have{6}{R}  \iflabelexists{showanswers}{\by{\color{red} \eor E}{2, 3}}{}
\close
\have{7}{\enot P \eif R} \iflabelexists{showanswers}{\by{\color{red} \eif I}{3-6}}{}
\end{proof}

%\begin{proof}
%\hypo{1}{\enot P \eif (Q \eor R)}
%\hypo{2}{P \eor \enot Q}  \by{Want: \enot P \eif R}{}
%\open
%\hypo{3}{\enot P} \by{Want: R}{}
%\have{4}{Q \eor R} \ce{1, 3}
%\have{5}{\enot Q} \oe{2, 3}
%\have{6}{R} \oe{4,5}
%\close
%\have{7}{\enot P \eif R} \ci{3-6}
%\end{proof}

\item  \textcolor{white}{.} \\ % $\{\enot P \eif (Q \eor R), P \eor \enot Q\} \sststile{}{} \enot P \eif R$
\vspace{-16pt}
\begin{proof}
\hypo{1}{A \eor B}
\hypo{2}{B \eif (B \eif \enot A)} \by{Want: \enot A \eiff B}{}
	\open
	\hypo{3}{\enot A} \by{Want: \iflabelexists{showanswers}{\color{red}B}{}}{}
	\have{4}{\iflabelexists{showanswers}{\color{red}B}{}} \by{\eor E}{1, 3}
	\close
	\open
	\hypo{5}{B} \by{Want: \iflabelexists{showanswers}{\color{red}\enot A}{}}{}
	\have{6}{B \eif \enot A} \iflabelexists{showanswers}{ \by{\color{red}\eif E}{2, 3}}{} % 
	\have{7}{\iflabelexists{showanswers}{\color{red}\enot A}{}} \by{\eif E}{5, 6}
	\close
\have{8}{\enot A \eiff B} \iflabelexists{showanswers}{\by{\color{red}\eiff I}{3-4, 5-7}}{}
\end{proof}
\end{exercises}

\pagebreak
 
\noindent\problempart Fill in the blanks in the following proofs. Be sure to include the ``Want'' line for each subproof. 

\begin{exercises}

\item \textcolor{white}{.} \\ % $\{\enot P \eif (Q \eor R), P \eor \enot Q\} \sststile{}{} \enot P \eif R$
\vspace{-16pt}
\begin{proof}
\hypo{1}{B \eif \enot D}
\hypo{2}{A \eif (D \eor C)}  \by{Want: $A \eif (B \eif C)$}{}
\open
\hypo{3}{A} \by{ }{}
\open 
\hypo{4}{} \by{Want: C}{}
\have{5}{} \ce{2, 3}
\have{6}{} \ce{1, 4}
\have{7}{} \oe{5, 6}
\close
\have{8}{} \ci{4-7}
\close
\have{9}{} \ci{3-8}
\end{proof}

%\item $B \eif \enot D, A \eif (D \eor C) $\therefore$ A \eif (B \eif C)$
%\begin{proof}
%\hypo{1}{B \eif \enot D}
%\hypo{2}{A \eif (D \eor C)}  \by{Want: A \eif (B \eif C)}{}
%\open
%\hypo{3}{A} \by{Want: B \eif C}{}
%\open 
%\hypo{4}{B} \by{Want: C}{}
%\have{5}{D \eor C} \ce{2, 3}
%\have{6}{\enot D} \ce{1, 4}
%\have{7}{C} \oe{5, 6}
%\close
%\have{8}{B \eif C} \ci{4-7}
%\close
%\have{9}{A \eif (B \eif C)} \ce{3-8}
%\end{proof}


\item \textcolor{white}{.} \\ 
\vspace{-16pt}

\begin{proof}
\hypo{1}{(G \eor H) \eif (S \eand T)}
\hypo{2}{(T \eor U) \eif (C \eand D)}	\by{Want: $G \eif C$}{}
	\open
	\hypo{3}{\nix{G}} \by{Want: C}{}
	\have{4}{G \eor H} \nix{\by{\eor I}{3}}
	\have{5}{\nix{S \eor T}} \by{\eif E}{1, 4}
	\have{6}{T} \nix{\by{\eand E}{5}}
	\have{7}{\nix{T \eor U}} \by{\eor I}{6}
	\have{8}{C \eand D} \nix{\by{\eif E}{2, 7}}
	\have{9}{\nix{C}}	\by{\eand E}{8}
	\close
\have{10}{\nix{G \eif C}} \by{\eif I}{3-9}
\end{proof}
\end{exercises}

\noindent\problempart Derive the following 
\begin{enumerate}[label=(\arabic*)]

\item	$\{S \eor Q, Q \eif P \}\sststile{}{} \enot S \eif P  $ %Basic conditional									

\answer{
\begin{proof}
\hypo{1}{S \eor Q}
\hypo{2}{Q \eif P} \by{Want: $\enot S \eif P$}{}
\open
\hypo{3}{\enot S} \by{Want: P}{}
\have{4}{Q} \oe{1, 3}
\have{5}{P} \ce{2, 4}
\close
\have{6}{\enot S \eif P} \by{\eif E}{3-5}
\end{proof}
}


\item	$\{A \eif C, B \eif D\}\sststile{}{}  (A \eand B) \eif (C \eand D)$ %Basic conditional					


\answer{
\begin{proof}
\hypo{1}{A \eif C}	
\hypo{2}{B \eif D} \by{Want: $(A \eand B) \eif (C \eand D)$}{}
\open
\hypo{3}{A \eand B} \by{Want: C \eand D}{}
\have{4}{A} \ae{3}
\have{5}{B} \ae{3}
\have{6}{C} \ce{1, 4}
\have{7}{D} \ce{2, 5}
\have{8}{C \eand D} \ai{6, 7}
\close
\have{9}{(A \eand B) \eif (C \eand D)} \by{\eif I}{3-8}
\end{proof}
}


\item $\{K\eand L\} \sststile{}{} K\eiff L$ %Basic biconditional
%originally Chapter 6, part B, number 1
\answer{
\begin{proof}
\hypo{1}{K \eand L} \by{want: $K \eiff L$}{}
\open
\hypo{2}{K} \by{Want: L}{}
\have{3}{L} \ae{1}
\close
\open
\hypo{4}{L} \by{Want: K}{}
\have{5}{K} \ae{1}
\close
\have{6}{K \eiff L} \by{\eiff E}{4-5, 6-7}
\end{proof}
}

\item $\{A\eif (B\eif C)\} \sststile{}{} (A\eand B)\eif C$ % Basic conditional, tempted to do the wrong thing
%originally Chapter 6, part B, number 2

\answer{
\begin{proof}
\hypo{1}{A\eif (B\eif C)} \by{$(A\eand B)\eif C$}{}
\open
\hypo{2}{A \eand B} \by{Want: C}{}
\have{3}{A} \ae{2}
\have{4}{B} \ae{2}
\have{5}{B \eif C} \ce{1, 3}
\have{6}{C} \ce{4, 5}
\close
\have{7}{(A \eand B) \eif C} \by{\eif I}{2--6}
\end{proof}
}

\item $\{A\eiff B, B\eiff C\} \sststile{}{} A\eiff C$ %Basic biconditional
%originally Chapter 6, part C, number 4

\answer{
\begin{proof}
\hypo{1}{A \eiff B}
\hypo{2}{B \eiff C} \by{Want: $A \eiff C$}{}
\open
\hypo{3}{A} \by{Want: C}{}
\have{4}{B} \by{\eiff E}{1, 3}
\have{5}{C} \by{\eiff E}{2, 4}
\close
\open
\hypo{6}{C} \by{Want: A}{}
\have{7}{B} \by{\eiff E}{2, 6}
\have{8}{A} \by{\eiff E}{1, 7}
\close
\have{9}{A \eiff C} \by{\eiff I}{3--5, 6--8}
\end{proof}
}

\item $\{P \eif (Q \eif R)\} \sststile{}{} Q \eif (P \eif R)$ %two subproofs

\answer{
\begin{proof}
\hypo{1}{P \eif (Q \eif R)} \by{Want: $Q \eif (P \eif R)$}{}
\open
\hypo{2}{Q} \by{Want: $P \eif R$}{}
\open
\hypo{3}{P} \by{Want: $R$}{}
\have{4}{Q \eif R} \ce{1, 3}
\have{5}{R} \ce{2, 4}
\close
\have{6}{P \eif R} \by{\eif I}{3-5}
\close 
\have{7}{Q \eif (P \eif R)} \by{\eif I}{2-6}
\end{proof}
}


\item $\{\enot A, (B \eand C) \eif D\} \sststile{}{}(A \eor B) \eif (C \eif D)$ %two subproofs  Modified from KMR T107 p. 82.

\answer{
\begin{proof}

\hypo{1}{\enot A}
\hypo{2}{(B \eand C) \eif D} \by{Want: (A \eor B) \eif (C \eif D}{}
\open
\hypo{3}{A \eor B} \by{Want: C \eif D}{}
\open
\hypo{4}{C}  \by{Want: D}{}
\have{5}{B} \oe{1, 4}
\have{6}{B \eand C} \ai{4, 5}
\have{7}{D} \ce{2, 6}
\close
\have{8}{C \eif D} \ci{4-7}
\close
\have{9}{(A \eor B) \eif (C \eif D)} \ci{3-8}
\end{proof}
}





\end{enumerate}	

\noindent\problempart Derive the following 
\begin{enumerate}[label=(\arabic*)]

\item $\{X \eiff (A \eand B), B \eiff Y, B \eif A\} \sststile{}{} X \eiff Y$

\answer{
\begin{proof}
\hypo{1}{X \eiff (A \eand B)}
\hypo{2}{B \eiff Y}
\hypo{3}{B \eif A} \by{Want: X \eiff Y}{} 
\open
\hypo{4}{X} \by{Want: Y}{}
\have{5}{A \eand B} \by{\eiff E}{1, 4}
\have{6}{B} \ae{5}
\have{7}{Y} \by{\eiff E}{2, 6}
\close
\open
\hypo{8}{Y} \by{Want: X}{}
\have{9}{B} \by{\eiff E}{2, 8}
\have{10}{A} \ce{3, 9}
\have{11}{A \eand B} \ai{9, 10}
\have{12}{X} \by{\eiff E}{1, 12}
\close
\have{13}{X \eiff Y} \by{\eiff I}{4-7, 8-12} 
\end{proof}
}

\item $\{B \eif \enot E, A \eif \enot D, D \eor (E \eor R), (R \eand A) \eif C\} \sststile{}{} A \eif (B \eif C)$ 

\answer{
\begin{proof}
\hypo{}{B \eif \enot E}
\hypo{}{A \eif \enot D}
\hypo{}{D \eor (E \eor R)}
\hypo{}{(R \eand A) \eif C}	\by{Want: A \eif (B \eif C)}{} 
\open
\hypo{}{A}	\by{Want: B \eif C}{}
\open
\hypo{}{B}	\by{Want: C}{}
\have{}{\enot E} \by{	\eif E 1, 6}{}
\have{}{\enot D} \by{\eif E 2, 5}{}
\have{}{E \eor R} \by{\eor E 3, 7}{}
\have{}{R} \by{\eor E 7, 9}{}
\have{}{R \eand} \by{\eand I 5, 10}{}
\have{}{C	} \by{\eif E 4, 11}{}
\close
\have{}{B \eif C} \by{\eif I 6-12}{}
\close
\have{}{A \eif (B \eif C)} \by{\eif I 5-13}{}
\end{proof}
}


\item $\{\enot W \eand \enot E, Q \eiff D\} \sststile{}{} (W \eor Q) \eiff (E \eor D)$

\answer{
\begin{proof}
\hypo{1}{\enot W \eand \eand E} 
\hypo{2}{Q \eiff D} \by{Want: (W \eor Q) \eiff (E \eor D)}{}
\have{3}{\enot W} \ae{1}
\have{4}{\enot E} \ae{1}
\open
\hypo{5}{W \eor Q} \by{Want: E \eor D}{}
\have{6}{Q} \oe{3, 5}
\have{7}{D} \by{\eiff E}{2, 6}
\have{8}{E \eor D} \oi{7}
\close
\open
\hypo{9}{E \eor D} \by{Want: W \eor Q}{}
\have{10}{D} \oe{4, 9}
\have{11}{Q} \by{\eiff E}{2, 10}
\have{12}{W \eor Q} \oi{11}
\close
\have{13}{(W \eor Q) \eiff (E \eor D)} \ci{5-8, 9-12}
\end{proof}
}


\item $\{(A \eand B) \eiff D, D \eiff (X \eand Y), C \eiff Z\} \sststile{}{} A \eand (B \eand C) \eiff X \eand (Y \eand Z)$ %long biconditional

\answer{
\begin{proof}
\hypo{1}{(A \eand B) \eiff D}
\hypo{2}{D \eiff (X \eand Y)}
\hypo{3}{C \eiff Z} \by{A \eand (B \eand C) \eiff X \eand (Y \eand Z)}{}
	\open
	\hypo{4}{A \eand (B \eand C)} \by{Want: X \eand (Y \eand Z)}{}
	\have{5}{A} \ae{4}
	\have{6}{B \eand C} \ae{4}
	\have{7}{B} \ae{6}
	\have{8}{C} \ae{6}
	\have{9}{Z} \by{\eiff-E}{3, 8}
	\have{10}{A \eand B} \ai{5, 7}
	\have{11}{D} \by{\eiff-E}{1, 10}
	\have{12}{X \eand Y}  \by{\eiff-E}{2, 11}
	\have{13}{X} \ae{12}
	\have{14}{Y} \ae{12}
	\have{15}{Y \eand Z} \ai{13, 14}
	\have{16}{X \eand (Y \eand Z)} \ai{13, 15}
	\close
	
	\open
	\hypo{17}{X \eand (Y \eand Z)} \by{Want: A \eand (B \eand C)}{} 
	\have{18}{X} \ae{17}
	\have{19}{Y \eand Z} \ae{17}
	\have{20}{Y} \ae{19}
	\have{21}{Z} \ae{19}
	\have{22}{C}  \by{\eiff-E}{3, 22}
	\have{23}{X \eand Y} \ai{18, 20}
	\have{24}{D}  \by{\eiff-E}{2, 23}
	\have{25}{A \eand B}  \by{\eiff-E}{1, 24}
	\have{26}{A} \ae{25}
	\have{27}{B} \ae{25}
	\have{28}{B \eand C} \ai{22, 27}
	\have{29}{A \eand (B \eand C)} \ai{26, 28}
	\close
\have{30}{A \eand (B \eand C) \eiff (X \eand (Y \eand Z)} \by{\eiff I, 4-16, 17-29}{}
\end{proof}
}


\end{enumerate}

%\noindent\problempart 
%Translate the following arguments in to SL and then show that they are valid. Be sure to write out your dictionary. 


% *******************************************
% *				Indirect Proof					   *	
% *******************************************

\section{Indirect Proof}
\label{sec:indirect_proof}

%signposting paragraph added

The last two rules we need to discuss are negation introduction (\enot I)  and negation elimination (\enot E). As with the rules of conditional and biconditional introduction, we have put off explaining the rules, because they require launching subproofs. In the case of negation introduction and elimination, these subproofs are designed to let us perform a special kind of derivation classically known as \emph{reductio ad absurdum}, or simply \emph{reductio}. 

%rob: changed the example to something less exact but more familiar
A \emph{reductio} in logic is a variation on a tactic we use in ordinary arguments all the time. In arguments we often stop to imagine, for a second, that what our opponent is saying is true, and then realize that it has unacceptable consequences. In so-called ``slippery slope'' arguments or ``arguments from consequences,'' we claim that doing one thing will will lead us to doing another thing which would be horrible. For instance, you might argue that legalizing physician assisted suicide for some patients might lead to the involuntary termination of lots of other sick people. These arguments are typically not very good, but they have a basic pattern whcih we can make rigorous  in our logical system. These arguments say ``if my opponent wins, all hell will break loose.'' In logic the equivalent of all hell breaking loose is asserting a contradiction. The worst thing you can do in logic is contradict yourself. The equivalent of our opponent being right in logic would be that a sentence we are trying to prove true turns out to be false (or alternately, that a sentence we are trying to prove false turns out to be true.) So, in developing the rules for reductio ad absurdum, we need to find a way to say ``if this sentence were false (or true), we would have to assert a contradiction.'' 

%The example from Magnus's version
%
%Here is a simple mathematical argument in English for the conclusion that there is no largest number:
%\begin{earg}
%\item[] Assume there \emph{is} some greatest natural number. Call it $A$.
%\item[] That number plus one is also a natural number.
%\item[] Obviously, $A+1 > A$.
%\item[] So there is a natural number greater than $A$.
%\item[] This is impossible, since $A$ is assumed to be the greatest natural number.
%\item[$\therefore$] There is no greatest natural number.
%\end{earg}
%This argument form is traditionally called a \emph{reductio}. Its full Latin name is \emph{reductio ad absurdum}, which means ``reduction to absurdity.'' In a reductio, we assume something for the sake of argument---for example, that there is a greatest natural number. Then we show that the assumption leads to two contradictory sentences---for example, that $A$ is the greatest natural number and that it is not. In this way, we show that the original assumption must have been false.

In our system of natural deduction, this kind of proof will be known as \define{indirect proof}.The basic rules for negation will allow for arguments like this. If we assume something and show that it leads to contradictory sentences, then we have proven the negation of the assumption. This is the negation introduction ({\enot}I) rule:

\begin{multicols}{2}

\begin{proof}
\open
	\hypo[m]{na}{\script{A}}\by{for reductio}{}
	\have[n]{b}{\script{B}}
	\have{nb}{\enot\script{B}}
\close
\have{a}[\ ]{\enot\script{A}}\ni{na-nb}
\end{proof}

\begin{proof}
\open
	\hypo[m]{na}{\script{A}}\by{for reductio}{}
	\have[n]{b}{\enot\script{B}}
	\have{nb}{\script{B}}
\close
\have{a}[\ ]{\enot\script{A}}\ni{na-nb}
\end{proof}

\end{multicols}


For the rule to apply, the last two lines of the subproof must be an explicit contradiction: either the second sentence is the direct negation of the first, or vice versa. We write ``for reductio'' as a note to ourselves, a reminder of why we started the subproof. It is not formally part of the proof, and you can leave it out if you find it distracting.

To see how the rule works, suppose we want to prove a law of double negation $A$ $\therefore$ $\enot \enot A$
\label{DN1}
%doublenegation

\begin{proof}
\hypo{1}{A} \by{Want: \enot \enot A}{}
	\open
	\hypo{2}{\enot A} \by{for reductio}{}
	\have{3}{A} \by{R}{1}
	\have{4}{\enot A} \by{R}{2}
	\close
\have{5}{\enot \enot A} \ni{2-4}
\end{proof}

The {\enot}E rule will work in much the same way. If we assume \enot\script{A} and show that it leads to a contradiction, we have effectively proven \script{A}. So the rule looks like this:

\begin{multicols}{2}
\begin{proof}
\open
	\hypo[m]{na}{\enot\script{A}}\by{for reductio}{}
	\have[n]{b}{\script{B}}
	\have{nb}{\enot\script{B}}
\close
\have{a}[\ ]{\script{A}}\ne{na-nb}
\end{proof}


\begin{proof}
\open
	\hypo[m]{na}{\enot\script{A}}\by{for reductio}{}
	\have[n]{b}{\enot\script{B}}
	\have{nb}{\script{B}}
\close
\have{a}[\ ]{\script{A}}\ne{na-nb}

\end{proof}
\end{multicols}

Here is a simple example of negation elimination at work. We can show $L \eiff \enot O, L \eor \enot O \therefore L$ by assuming \enot L, deriving a contradiction, and then using \enot E.

\begin{proof}
\hypo{1}{L \eiff \enot O}
\hypo{2}{L \eor \enot O}\by{Want: $L$}{}
\open
	\hypo{3}{\enot L}\by{for reductio}{}
	\have{4}{\enot O} \oe{2, 3}
	\have{5}{L} \be{1, 4}
	\have{6}{\enot L} \by{R}{3}
\close
\have{7}{L}\ne{3-6}
\end{proof}

With the addition of \enot E and \enot I, our system of natural deduction is complete. We can now prove that any valid argument is actually valid. This is really where the fun begins. 

One important bit of strategy. Sometimes, you will launch a subproof right away by assuming the negation of the conclusion to the whole argument. Other times, you will use a subproof to get a piece of the conclusion you want, or some stepping stone to the conclusion you want. Here's a simple example. Suppose you were asked to show that this argument is valid: $\enot(A \eor B) \therefore \enot A \eand \enot B$. (The argument, by the way, is part of DeMorgan's Laws, some very useful equivalences which we will see more of later on.)

You need to set up the proof like this.

\begin{proof}
\hypo{1}{\enot(A \eor B)} \by{Want $\enot A \eand \enot B$}{}
\end{proof}

Since you are trying to show $\enot A \eand \enot B$, you could open a subproof with $\enot(\enot A \eand \enot B$) and try to derive a contradiction, but there is an easier way to do things. Since you are tying to prove a conjunction, you can set out to prove each conjunct separately. Each conjunct, then, would get its own reductio. Let's start by assuming $A$ in order to show $\enot A$

\begin{proof}
\hypo{1}{\enot(A \eor B)} \by{Want $\enot A \eand \enot B$}{}
	\open
	\hypo{2}{A}	\by{for reductio}{}
	\have{3}{A \eor B} \oi{2}
	\have{4}{\enot (A \eor B)} \by{R}{1}
	\close
\have{5}{\enot A} \ni{2-4}
\end{proof}

%\pagebreak[4]

We can then finish the proof by showing $\enot B$ and putting it together with $\enot A$ and conjunction introduction. 

% Demorgan's negated disjunction to conjunction of negations.
\label{DeM1}
\begin{proof}
\hypo{1}{\enot(A \eor B)} \by{Want $\enot A \eand \enot B$}{}
	\open
	\hypo{2}{A}	\by{for reductio}{}
	\have{3}{A \eor B} \oi{2}
	\have{4}{\enot (A \eor B)} \by{R}{1}
	\close
\have{5}{\enot A} \ni{2-4}
	\open
	\hypo{6}{B}	\by{for reductio}{}
	\have{7}{A \eor B} \oi{6}
	\have{8}{\enot (A \eor B)} \by{R}{1}
	\close
\have{10}{\enot B} \ni{7-9}
\have{11}{\enot A \eand \enot B} \ai{6,10}
\end{proof}


%%%%%%%%Practice problems  %%%%%%%%%%%%%%%%%  !@#$ 
 	
\practiceproblems

\noindent\problempart Fill in the blanks in the following proofs.

\begin{multicols}{2}
\begin{enumerate}[label=(\arabic*)]

\item \textcolor{white}{.} \\ 
\vspace{-16pt}
%DeMorgans, conjunction of negations to negated disjunction
\label{DeM2}
\begin{proof}
\hypo{1}{\enot A \eand \enot B} \by{Want: \iflabelexists{showanswers}{\color{red}$\enot (A \eor B)$}{}}{}
\have{2}{\enot A} \iflabelexists{showanswers}{ \by{\color{red}\eand E}{1}}{}
\have{3}{\enot B} \iflabelexists{showanswers}{ \by{\color{red}\eand E}{1}}{}
\open
	\hypo{4}{A \eor B} \by{for reductio}{}
	\have{5}{B} \iflabelexists{showanswers}{ \by{\color{red}\eor E}{2, 4}}{}
	\have{6}{\enot B} \iflabelexists{showanswers}{ \by{\color{red}R}{3}}{}
\close
\have{7}{\enot(A \eor B)} \iflabelexists{showanswers}{\by{\color{red} \enot I}{4-6}}{}
\end{proof}

\vspace{5cm}

\item \textcolor{white}{.} \\ 
\vspace{-16pt}
%Demorgans disjunction of negations to a negated conjunction.
\label{DeM3}
\begin{proof}
\hypo{1}{\enot A \eor \enot B} \by{Want: $\enot(A \eand B)$}{}
	\open
	\hypo{2}{A \eand B} \by{for reductio}{}
	\have{3}{\iflabelexists{showanswers}{\color{red}A}{}} \ae{2}	
	\have{4}{\iflabelexists{showanswers}{\color{red}B}{}} \ae{2}
		\open
		\hypo{5}{\iflabelexists{showanswers}{\color{red} \enot A}{}}\by{for reductio}{}
		\have{6}{\iflabelexists{showanswers}{\color{red}A}{}} \by{R}{3}
		\have{7}{\iflabelexists{showanswers}{\color{red}\enot A}{}} \by{R}{5}
		\close
	\have{8}{\enot \enot A} \iflabelexists{showanswers}{ \by{\color{red}\enot I}{5-7}}{} 
	\have{9}{B} \iflabelexists{showanswers}{ \by{\color{red}R}{4}}{} 
	\have{10}{\enot B} \iflabelexists{showanswers}{ \by{\color{red}\eor E}{1, 8}}{} 
	\close
\have{11}{\enot(A \eand B)} \iflabelexists{showanswers}{ \by{\color{red}\enot I}{2-10}}{} 
\end{proof}

%Demorgans disjunction of negations to a negated conjunction.
%\begin{proof}
%\have{1}{\enot A \eor \enot B} \by{Want: \enot(A \eand B)}{}
%	\open
%	\hypo{2}{A \eand B} \by{for reductio}{}
%	\have{3}{A} \ae{2}	
%	\have{4}{B} \ae{2}
%		\open
%		\hypo{5}{\enot A} \by{for reductio}{}
%		\have{6}{A} \by{R}{3}
%		\have{7}{\enot A} \by{R}{5}
%		\close
%	\have{8}{\enot \enot A} \ni{5-7}
%	\have{9}{B} \by{R}{4}
%	\have{10}{\enot B} \oe{1, 8}
%	\close
%\have{11}{\enot(A \eand B} \ni{2-10}
%\end{proof}
\end{enumerate}
\end{multicols}

\noindent\problempart Fill in the blanks in the following proofs.

\begin{enumerate}[label=(\arabic*)]
\item \textcolor{white}{.}  
\vspace{-20pt} %$P \eif Q \therefore \enot P \eor Q$
%conditional disjunction, from conditional to disjunction
\begin{proof}
\hypo{1}{P \eif Q} \by{Want: $\enot P \eor Q$}{}
	\open
	\hypo{2}{\hspace{1cm}} \by{for reductio}{}
		\open
		\hypo{3}{\hspace{1cm}} \by{for reductio}{}
		\have{4}{} \ce{1, 3}
		\have{5}{} \oi{4}
		\have{6}{} \by{R}{2}
		\close
	\have{7}{\enot P} \ni{3-6}
	\have{8}{ } \oi{7}
	\have{9}{ } \by{R}{2}
	\close
\have{10}{\enot P \eor Q} \ne{2-9}			
\end{proof}

%\begin{proof}
%\hypo{1}{P \eif Q} \by{Want: \enot P \eor Q}{}
%	\open
%	\hypo{2}{\enot(\enot P \eor Q)} \by{for reductio}{}
%		\open
%		\hypo{3}{P} \by{for reductio}{}
%		\have{4}{Q} \ce{1, 3}
%		\have{5}{\enot P \eor Q} \oi{4}
%		\have{6}{\enot(\enot P \eor Q} \by{R}{2}
%		\close
%	\have{7}{\enot P} \ni{3-6}
%	\have{8}{\enot P \eor Q} \oi{7}
%	\have{9}{\enot(\enot P \eor Q)} \by{R}{2}
%	\close
%\have{10}{\enot P \eor Q} \ne{2-9}			
%\end{proof}


\item \textcolor{white}{.}  
\vspace{-18pt} %$(X\eand Y)\eor(X\eand Z)$, $\enot(X\eand D)$, $D\eor M$ $\therefore$ $M$

\begin{proof}
\hypo{1}{(X\eand Y)\eor(X\eand Z)}
\hypo{2}{\enot(X\eand D)}
\hypo{3}{D \eor M} \by{Want: M}{}
	\open
	\hypo{4}{\hspace{1cm}} \by{for reductio}{}
	\have{5}{D} \oe{ }
		\open
		\hypo{6}{\hspace{1cm}} \by{for reductio}{}
		\have{7}{\enot X \eor \enot Y} 
			\open
			\hypo{8}{\hspace{1cm}}	\by{for reductio}{}
			\have{9}{X}	
			\have{10}{\enot X}	
			\close
		\have{11}{\enot(X \eand Y)} \ni{8-10}
		\have{12}{} \oe{1, 11}
		\have{13}{} \ae{12}
		\have{14}{} \by{R}{6}
		\close
	\have{15}{X} 
	\have{16}{X \eand D} 
	\have{17}{\enot (X \eand D)} 
	\close
\have{18}{M} \ne{4-17}
\end{proof}

%\begin{proof}
%\have{1}{(X\eand Y)\eor(X\eand Z)}
%\have{2}{\enot(X\eand D)}
%\have{3}{D \eor M} \by{Want: M}{}
%	\open
%	\hypo{4}{\enot M} \by{for reductio}{}
%	\hypo{5}{D} \oe{3, 4}
%		\open
%		\hypo{6}{\enot X} \by{for reductio}{}
%		\have{7}{\enot X \eor \enot Y} \oi{6}
%			\open
%			\hypo{8}{X \eand Y}	\by{for reductio}{}
%			\have{9}{X}	\ae{8}
%			\have{10}{\enot X}	\by{R}{6}
%			\close
%		\have{11}{\enot(X \eand Y)} \ni{8-10}
%		\have{12}{X \eand Z} \oe{1, 11}
%		\have{13}{X} \ae{12}
%		\have{14}{\enot X} \by{R}{6}
%		\close
%	\have{15}{X} \ne{6-14}
%	\have{16}{X \eand D} \ai{5, 15}
%	\have{17}{\enot (X \eand D)} \by{R}{2}
%	\close
%\have{18}{M} \ne{4-17}
%\end{proof}

\end{enumerate}


%%%%%%%%%%%%%%%%%%%%%%%   Part C: Derive the following using indirect derivation %%%%%
\noindent\problempart Derive the following using indirect derivation. You may also have to use conditional derivation.
\begin{enumerate}[label=(\arabic*)]

\item $\enot \enot A  \sststile{}{}  A$
%Double negation removing negations
\label{DN2}

\answer{
\begin{proof}
\hypo{1}{\enot \enot A} \by{Want: $A$}{}
	\open
	\hypo{2}{\enot A} \by{for reductio}{}
	\have{3}{\enot \enot A} \by{R}{1}
	\have{4}{\enot A} \by{R}{2}
	\close
\have{5}{A} \ne{2-4}
\end{proof}
}

\item $\{A \eif B, \enot B\} \sststile{}{} \enot A$
%modus tollens
\label{ModusTollens}

\answer{
\begin{proof}
	\hypo{1}{A \eif B}
	\hypo{2}{\enot B} \by{Want: $\enot A$}{}
		\open
		\hypo{3}{A} \by{for reductio}{}
		\have{4}{B} \ce{1, 3}
		\have{5}{\enot B} \by{R}{2}
		\close
	\have{6}{\enot A} \ni{3-5}
\end{proof}
}


\item $\enot(A \eand B) \sststile{}{} \enot A \eor \enot B$
%DeMorgan's negated conjunction to disjunction of negations

\answer{
\begin{proof}
\hypo{1}{\enot(A \eand B)} \by{Want: \enot A \eor \enot B}{}
	\open
	\hypo{2}{\enot(\enot A \eor \enot B)} \by{for reductio}{}
		\open
		\hypo{3}{\enot A} \by{for reductio}{}
		\have{4}{\enot A \eor  \enot B} \oi{3}
		\have{5}{\enot(\enot A \eor \enot B)} \by{R}{2}
		\close
	\have{6}{A} \ne{3-5}
		\open
		\hypo{7}{\enot B} \by{for reductio}{}
		\have{8}{\enot A \eor \enot B} \oi{7}
		\have{9}{\enot(\enot A \eor \enot B)} \by{R}{2}
		\close
	\have{10}{B} \ne{7-9}
	\have{11}{A \eand B} \ai{6, 10}
	\have{12}{\enot(A \eand B)} \by{R}{1}
	\close
\have{13}{\enot A \eor \enot B} \ne{2-12}
\end{proof}
}

\item $\{(A \eif B) \}\sststile{}{} (A \eif \enot B) \eif \enot A$

\answer{
\begin{proof}
 \hypo{}{A \eif B} \by{Want: (A \eif \enot B) \eif \enot A}{}
\open
\hypo{}{A \eif \enot B} \by{Want: \enot A}{}
\open
\hypo{}{A} \by{Want: A contradiction}{}
\have{}{B} \by{\eif E}{1, 3}
\have{}{\enot B} \by{\eif E}{2, 3}
\close
\have{}{\enot A} \by{\enot I}{3-5}
\close
\have{}{(A \eif \enot B) \eif \enot A} \by{\eif I}{2-6}
\end{proof}
}



\item $\{\enot F\eif G, F\eif H\} \sststile{}{} G\eor H$

\answer{
\begin{proof}
\hypo{1}{\enot F\eif G}
\hypo{2}{F\eif H} \by{Want: $G\eor H$}{}
	\open
	\hypo{3}{\enot (G \eor H)} \by{for reductio}{}
		\open
		\hypo{4}{F} \by{for reductio}{}
		\have{5}{H} \ce{2, 4}
		\have{6}{G \eor H} \oi{5}
		\have{7}{\enot(G \eor H)} \by{R}{3}
		\close
	\have{8}{\enot F} \ni{4-7}
	\have{9}{G} \ce{1,8}
	\have{10}{G \eor H} \oi{9}
	\have{11}{\enot (G \eor H)} \by{R}{3}
	\close
\have{12}{G \eor H} \ne{3-11}
\end{proof}
}

\item	$\{(T \eand K) \eor (C \eand E), E \eif \enot C\} \sststile{}{}  T \eand K$

\answer{
There are two solutions. In one, you look at the want line to figure out the assumption for the subproof. In the other, you think of another think you might want, and assume the negation of that.

\begin{proof}
\hypo{1.}{(T \eand K) \eor (C \eand E)}
\hypo{2.}{E \eif \enot C} 			\by{Want: T \eand K}{}
\open
\hypo{3.}{\enot (T \eand K)} \by{For Reductio}{}
\have{4.}{C \eand E}	 \by{\eor E}{1, 4}
\have{5.}{E} \by{\eand E}{4}
\have{6.}{C} \by{\eand E}{5}
\have{7.}{\enot C} \by{\enot E}{2, 6}
\close
\have{8.}{T \eand K}	\by{\enot E}{3-7}
\end{proof}

\begin{proof}
\hypo{1.}{(T \eand K) \eor (C \eand E)}
\hypo{2.}{E \eif \enot C} \by{Want: T \eand K}{}
\open
\have{3.}{C \eand  E}	\by{For Reductio}{}
\have{4.}{E}\by{\eand E}{3}
\have{5.}{C}\by{\eand E}{4}
\have{6.}{\enot C}\by{\eif E}{2, 4}
\close
\have{7.}{\enot (C \eand E)}\by{\enot I}{3-4}
\have{8.}{T \eand K}\by{\eor E}{1, 7}
\end{proof}
}


\item $A \eif (\enot B \eor \enot C) \sststile{}{} A \eif \enot (B \eand C)$
%
\answer{
\begin{proof}
\hypo{1}{A \eif (\enot B \eor \enot C)} \by{Want: $A \eif \enot (B \eand C)$}{}
	\open
	\hypo{2}{A}	\by{Want: \enot (B \eand C)}{}
	\have{3}{\enot B \eor \enot C} \ce{1,2}		
		\open
		\hypo{4}{B \eand C} \by{for reductio}{}
		\have{5}{B} \ae{3}
		\have{6}{C} \ae{3}
			\open
			\hypo{7}{\enot B} \by{for reductio}{}
			\have{8}{B} \by{R}{4}
			\have{9}{\enot B} \by{R}{6}
			\close
		\have{10}{\enot \enot B} \ni{6-8}
		\have{11}{\enot C} \oe{3, 10}
		\have{12}{C} \by{R}{5}
		\close
	\have{13}{\enot (B \eand C)} \ni{3-11}
	\close
\have{14}{A \eif \enot (B \eand C)} \ci{2-12} 
\end{proof}
}

\end{enumerate}

%%%%%%%%%%%%%%%%%%%%%%%   Part D: Derive the following using indirect derivation %%%%%
\noindent\problempart Derive the following using indirect derivation. You may also have to use conditional derivation.
\label{derivation_set_with_const_d}
\begin{enumerate}[label=(\arabic*)]

\item $\{P \eif Q, P \eif \enot Q\} \sststile{}{} \enot P$

%\begin{proof}
%\hypo{1}{P \eif Q}
%\hypo{2}{P \eif \enot Q} \by{Want: \enot P}{}
%	\open
%	\hypo{3}{P} \by{for reductio}{}
%	\have{4}{Q} \ce{1, 3}
%	\have{5}{\enot Q} \ce{2, 3}
%	\close
%\have{6}{\enot P} \ni{3-5}
%\end{proof}

\item $(C\eand D)\eor E \sststile{}{} E\eor D$

%\begin{proof}
%\hypo{1}{(C\eand D)\eor E} \by{Want: $E \eor D$}{}
%	\open
%	\hypo{2}{\enot (E \eor D)} \by{for reductio}{}
%		\open
%		\hypo{3}{E} \by{for reductio}{}
%		\have{4}{E \eor D} \oi{3}
%		\have{5}{\enot (E \eor D)} \by{R}{2}
%		\close
%	\have{6}{\enot E} \ni{3-5}
%	\have{7}{C \eand D} \oe{1, 6}
%	\have{8}{D} \ae{7}
%	\have{9}{E \eor D} \oi{8}
%	\have{10}{\enot (E \eor D)}	\by{R}{2}
%	\close
%\have{11}{E \eor D} \ne{2-10}
%\end{proof}

\item $M\eor(N\eif M) \sststile{}{} \enot M \eif \enot N$ \label{DeM4}

%\begin{proof}
%\hypo{1}{M\eor(N\eif M)} \by{want: \enot M \eif \enot N}{}
%\open
%\hypo{2}{\enot M} \by{want: \enot N}{}
%\have{3}{N \eif M}
%\open
%\hypo{4}{N} \by{want: M and \enot M}{}
%\have{5}{M} \ce{3, 4}
%\have{6}{\enot M} \by{R}{2}
%\close
%\have{7}{\enot N} \by{\enot I}{4--6}
%\close
%\have{8}{\enot M \eif \enot N} \ci{2--7}
%\end{proof}



\item \label{itm:const_d} \{$A \eor B, A \eif C, B \eif C\} \sststile{}{} C$

%\begin{proof}
%\hypo{1}{A \eor B}
%\hypo{2}{A \eif C}
%\hypo{3}{B \eif C}  \by{Want: C}{}
%	\open
%	\hypo{4}{\enot C} \by{for reductio}{}
%		\open
%		\hypo{5}{\enot A} \by{for reductio}{}
%		\have{6}{B} \oe{1,5}
%		\have{7}{C} \ce{3, 6}
%		\have{8}{\enot C} \by{R}{7}
%		\close
%	\have{9}{A} \ne{5-8}
%	\have{10}{C} \ce{2, 9}
%	\have{11}{\enot C} \by{R}{4}
%	\close
%\have{12}{C} \ne{4-11}
%\end{proof}

\item	$A \eif (B \eor (C \eor D))  \sststile{}{} \enot[A \eand (\enot B \eand (\enot C \eand \enot D))] $

%1.	A → (B ˅ (C ˅ D)) 			Want: ~[A & (~B & (~C &~D))]
%2.		A & (~B & (~C &~D))	For reductio
%3.		A							&E 2
%4.		B ˅ (C ˅ D)				→E 1, 3
%5.		~B & (~C &~D)			&E 2
%6.		~B							&E 5
%7.		C ˅ D						˅E 4, 6
%8.		~C & ~D					&E 5
%9.		~C							&E 8
%10.		D							˅E 7, 9
%11.		~D							&E 8
%12.	~[A & (~B & (~C &~D))]	~I 2–11	
%





\end{enumerate}

%\noindent\problempart 
%Translate the following arguments in to SL and then show that they are valid. Be sure to write out your dictionary. 

%
%  This is the opening of the conditional formatting tag for typesetting only part of this chapter. Everything from here to the close tag will be skipped unless the {whole_slproof_chap} label at the
%  start of this chapter is uncommented.
%

\iflabelexists{whole_slproof_chap}{

% *******************************************
% *		Tautologies and Equivalences				   *	
% *******************************************

\section{Tautologies and Equivalences}


So far all we've looked at is whether conclusions follow validly from sets of premises. However, as we saw in the chapter on truth tables, there are other logical properties we want to investigate: whether a statement is a tautology, a contradiction or a contingent statement, whether two statements are equivalent, and whether sets of sentences are consistent. In this section, we will look at using derivations to test for two properties which will be important in later sections, logical equivalence and being a tautology.


\newglossaryentry{syntactically logically equivalent in SL}
{
name=syntactically logically equivalent in SL,
description={A property held by pairs of statements in SL if and only if there is a derivation which takes you from each one to the other one.}
}

We can say that two statements are \textsc{\gls{syntactically logically equivalent in SL}} \label{def:syntactically_logically_equivalent_in_sl} if you can derive each of them from the other. We can symbolize this the same way we symbolized semantic equivalence. When we introduced the double turnstile (p. \pageref{defDoubleTurnstile}), we said we would write the symbol facing both directions to indicate that two sentences were semantically equivalent, like this: $A \eand B \ndststile{}{} \hspace{.5em} \sdtstile{}{} B \eand A$. We can do the same thing with the single turnstile for syntactic equivalence, like this:  $A \eand B \nsststile{}{} \hspace{.5em} \sststile{}{} B \eand A$. 

For an example of how we can show two sentences to be syntactically equivalent, consider the sentences $P \eif (Q \eif R)$ and $(P \eif Q) \eif (P \eif R)$. \label{theorem_DistributionOfImplicationOverImplication} To prove these logically equivalent using derivations, we simply use derivations to prove the equivalence one way, from P \eif (Q \eif R) to (P \eif Q) \eif (P \eif R). And then we prove it going the other way, from (P \eif Q) \eif (P \eif R) to P \eif (Q \eif R). We set up the proof going left to right like this: 

\begin{proof}
\hypo{1}{P \eif (Q \eif R)}	\by{Want: (P \eif Q) \eif (P \eif R)}{}
\end{proof}

Since our want line is a conditional, we can set this up as a conditional proof. Once we set up the conditional proof, we also have a conditional in next want line, which means that we can put a conditional proof inside a conditional proof, like this.

\begin{proof}
\hypo{1}{P \eif (Q \eif R)}	\by{Want: (P \eif Q) \eif (P \eif R)}{}
	\open
	\hypo{2}{P \eif Q}	\by{Want: P \eif R}{}
		\open
		\hypo{3}{P}	\by{Want: R}{}
\end{proof}

The completed proof for the equivalence going in one direction will look like this.

\begin{proof}
\hypo{1}{P \eif (Q \eif R)}	\by{Want: (P \eif Q) \eif (P \eif R)}{}
	\open
	\hypo{2}{P \eif Q}	\by{Want: P \eif R}{}
		\open
		\hypo{3}{P}	\by{Want: R}{}
		\have{4}{Q \eif R} \by{\eif E}{1, 3}
		\have{5}{Q}	\by{\eif E}{2, 3}
		\have{6}{R} \by{\eif E}{4, 5}
		\close
	\have{7}{P \eif R} \by{\eif I}{3-6}
	\close
\have{8}{(P \eif Q) \eif (P \eif R)} \by{\eif I}{2-7}
\end{proof}

This shows that $P \eif (Q \eif R) \sststile{}{} (P \eif Q) \eif (P \eif R)$. In order to show $P \eif (Q \eif R) \nsststile{}{} \hspace{.5em} \sststile{}{} (P \eif Q) \eif (P \eif R)$, we need to prove the equivalence going the other direction. That proof will look like this:

\begin{proof}
\hypo{1}{(P \eif Q) \eif (P \eif R)}	\by{Want: P \eif (Q \eif R)}{}
	\open
	\hypo{2}{P} \by{Want: Q \eif R}{}
		\open
		\hypo{3}{Q} \by{Want: R}{}
			\open
			\hypo{4}{P} \by{Want: Q}{}
			\have{5}{Q} \by{R}{3}
			\close
		\have{6}{P \eif Q} \by{\eif I}{4-5}
		\have{7}{P \eif R} \by{\eif E}{1, 6}
		\have{8}{R} \by{\eif E}{2, 7}
		\close	
\have{9}{Q \eif R} \by{\eif I}{3-8}
\close
\have{10}{P \eif (Q \eif R)} \by{\eif I}{2-9}
\end{proof}
You might think it is strange that we assume $P$ twice in this proof, but that is the way we have to do it. When we assume $P$ on line 2, our goal is to prove $P \eif {Q \eif R}$. Before we can finish that proof, we also need to know that $P \eif Q$. This requires a different subproof. 

These two proofs show that $P \eif (Q \eif R)$ and $(P \eif Q) \eif (P \eif R)$ are equivalent, so we can write $P \eif (Q \eif R) \nsststile{}{} \hspace{.5em} \sststile{}{} (P \eif Q) \eif (P \eif R)$. 

\newglossaryentry{syntactic tautology in SL}
{
name=syntactic tautology in SL,
description={A statement in SL that can be derived without any premises}
}

We can also prove that a sentence is a tautology using a derivation. A tautology is something that must be true as a matter of logic. If we want to put this in syntactic terms, we would say that \textsc{\gls{syntactic tautology in SL}} \label{def:syntactic_tautology_in_sl} is a statement that can be derived without any premises, because its truth doesn't depend on anything else. Now that we have all of our rules for starting and ending subproofs, we can actually do this. Rather than listing any premises, we simply start a subproof at the beginning of the derivation. The rest of the proof can work only using premises assumed for the purposes of subproofs. By the end of the proof, you have discharged all these assumptions, and are left knowing a tautological statement without relying on any leftover premises. Consider this proof of the law of noncontradiction: $\enot(G \eand \enot G)$. \label{theorem_Noncontradiction}

\begin{proof}
	\open
		\hypo{gng}{G\eand \enot G}\by{for reductio}{}
		\have{g}{G}\ae{gng}
		\have{ng}{\enot G}\ae{gng}
	\close
	\have{ngng}{\enot(G \eand \enot G)}\ni{gng-ng}
\end{proof}

This statement simply says that any sentence G cannot be both true and not true at the same time. We prove it by imagining what would happen if G were actually both true and not true, and then pointing out that we already have our contradiction. 

In the previous chapter, we expressed the fact that something could be proven a tautology using truth tables by writing the double turnstile in front of it. The law of noncontradiction above could have been proven using truth tables, so we could write: $\sdtstile{}{} \enot(G \eand \enot G)$ In this chapter, we will use the single turnstile the same way, to indicate that a sentence can be proven to be a tautology using a derivation. Thus the above proof entitles us to write $\sststile{}{} \enot(G \eand \enot G)$.  


\practiceproblems
 	

\noindent\problempart
Prove each of the following equivalences
\begin{enumerate}[label=(\arabic*)]

\item $J \nsststile{}{} \hspace{.5em} \sststile{}{} J\eor (L\eand\enot L)$
%\vspace{5pt}
%$ \sststile{}{}$
%\vspace{5pt}
%\begin{proof}
%		\hypo{1}{J} \by{Want: J \eor(L \eand \enot L)}{}
%		\have{2}{J \eor (L \eand \enot L)} \oi{1}
%\end{proof}
%\vspace{5pt}
%$\nsststile{}{}$
%\vspace{5pt}
%\begin{proof}	
%		\hypo{1}{J \eor (L \eand \enot L)} \by{Want: J}{}
%			\open
%			\hypo{2}{L \eand \enot L} \by{for reductio}{}
%			\have{3}{L} \ae{4}
%			\have{4}{\enot L} \ae{4}
%			\close
%		\have{5}{\enot(L \eand \enot L)} \ni{4-6}
%		\have{6}{J} \oe{3, 7}
%\end{proof}


\item $P \eif (Q \eif R) \nsststile{}{} \hspace{.5em} \sststile{}{} Q \eif (P \eif R)$

%Modified from KMM T107 p. 82.

%\vspace{5pt}
%$ \sststile{}{}$
%\vspace{5pt}
%
%\begin{proof}
%\hypo{1.}{P \eif (Q \eif R)} \by{Want: Q \eif (P \eif R)}{}
%\open
%\hypo{2.}{Q} \by{Want: P \eif R}{}
%\open
%\hypo{3.}{P} \by{Want: R}{}
%\have{4}{Q \eif R } \by{\eif E}{1,3}
%\have{5.}{R} \by{ \eif E }{2, 4}
%\close
%\have{6}{P \eif R } \by{\eif I}{3-5}
%\close
%\have{7.}{Q \eif (P \eif R)} \by{\eif I}{2-6}
%\end{proof}
%
%\vspace{5pt}
%$\nsststile{}{}$
%\vspace{5pt}
%
%
%\begin{proof}
%\hypo{1.}{Q \eif (P \eif R)} \by{Want: P \eif (Q \eif R)}{}
%\open
%\hypo{2.}{P} \by{Want: Q \eif R}{}
%\open
%\hypo{3.}{Q} \by{Want: R}{}
%\have{4}{P \eif R } \by{\eif E}{1,3}
%\have{5.}{R} \by{ \eif E }{2, 4}
%\close
%\have{6}{Q \eif R } \by{\eif I}{3-5}
%\close
%\have{7.}{P \eif (Q \eif R)} \by{\eif I}{2-6}
%\end{proof}

\item $P \eif \enot P \nsststile{}{} \hspace{.5em} \sststile{}{}  \enot P $ %(KMM T115, p. 111)

%\vspace{5pt}
%$ \sststile{}{}$
%\vspace{5pt}
%
%
%\begin{proof}
%\hypo{1}{P \eif \enot P} \by{Want:  \enot P}{}
%	\open
%	\hypo{2}{P} \by{Want: A contradiction}{}
%	\have{3}{\enot P} \by{\eif E}{1, 2}
%	\have{4}{P} \by{R}{2}
%	\close
%\have{5}{P} \by{\enot E}{2-4}
%\end{proof}
%
%\vspace{5pt}
%$\nsststile{}{}$
%\vspace{5pt}
%
%\begin{proof}
%\hypo{1}{ \enot P } \by{Want: P \eif \enot P}{}
%	\open
%	\hypo{2}{P} \by{Want: \enot P}{}
%	\have{3}{\enot P} \by{R}{1}
%	\close
%\have{4}{P \eif \enot P} \by{\eif I}{2-3}
%\end{proof}
%

\item $\enot (P \eiff Q) \nsststile{}{} \hspace{.5em} \sststile{}{} (P \eiff \enot Q) $ %(KMM T90 p. 110)

%\vspace{5pt}
%$ \sststile{}{}$
%\vspace{5pt}
%
%\begin{proof}
%\hypo{1.}{\enot (P \eiff Q)} \by{Want: $P \eiff \enot Q$}{}
%	\open
%	\hypo{2.}{P} \by{Want: $\enot Q$}{}
%		\open
%		\hypo{3.}{Q}	\by{Want: A contradiction}{}
%			\open
%			\hypo{4.}{P} \by{Want: Q}{}
%			\have{5.}{Q} \by{R}{3}
%			\close
%			\open
%			\hypo{6.}{Q} \by{Want: P}{}
%			\have{7.}{P} \by{R}{2}
%			\close
%		\have{8.}{P \eiff Q} \by{\eiff I}{4-5, 6-7}
%		\have{9.}{\enot(P \eiff Q)} \by{R}{1}
%		\close
%	\have{10.}{\enot Q} \by{\enot I}{2-9}
%	\close
%	\open
%	\hypo{a}{\enot Q} \by{Want: P}{}
%		\open
%		\hypo{b}{\enot P} \by{Want: A contradiction}{}
%			\open
%			\hypo{c}{P} \by{Want: Q}{}
%			\have{d}{P \eor Q} \by{\eor I}{13}
%			\have{e}{Q} \by{\eor E}{12, 14}
%			\close
%			\open
%			\hypo{f}{Q} \by{Want: P}{}
%			\have{g}{Q \eor P} \by{\eor I}{f}
%			\have{h}{P} \by{\eor E}{f, g}
%			\close
%		\have{i}{P \eiff Q} \by{\eiff I}{c-e, f-h}
%		\have{j}{\enot (P \eiff Q)} \by{R}{b}
%		\close
%	\have{21.}{P} \by{\enot I}{a-j}
%	\close
%\have{22.}{P \eiff \enot Q} \by{\eiff I}{}
%\end{proof}
%
%\vspace{5pt}
%$\nsststile{}{}$
%\vspace{5pt}
%
%\begin{proof}
%\hypo{1}{P \eiff \enot Q} \by {Want: \enot (P \eiff Q)}{}
%	\open
%	\hypo{2}{P \eiff Q} \by{Want: A contradiction}{}
%		\open
%		\hypo{3}{Q} \by{Want: A contradiction}{}
%		\have{4}{P} \by{\eiff E}{2, 3}
%		\have{5}{\enot Q} \by{\eiff E}{1, 4}
%		\have{6}{Q} \by{R}{3}
%		\close
%	\have{7}{\enot Q} \by{\enot I}{3-6}
%	\have{8}{P} \by{\eiff}{1, 7}
%	\have{9}{Q}\by{\eiff}{2, 8}
%	\have{10}{\enot Q} \by{R}{7}
%	\close
%\have{11}{\enot (P \eiff Q)} \by{\enot I}{2-10}
%\end{proof}
%\vspace{15pt}




\end{enumerate}

\noindent\problempart
Prove each of the following equivalences
\begin{enumerate}[label=(\arabic*)]

\item $(P \eif R) \eand (Q \eif R) \nsststile{}{} \hspace{.5em} \sststile{}{}(P \eor Q) \eif R $ %(KMM T50 p.109)
\item $(P \eif (Q \eor R)) \nsststile{}{} \hspace{.5em} \sststile{}{} (P \eif Q) \eor (P \eif R)$ %(KMM T55 p.109)
\item $(P \eiff Q)  \nsststile{}{} \hspace{.5em} \sststile{}{} \enot P \eiff \enot Q		$ %(KMM T96 p.110)
\end{enumerate}

%\item $P \eif Q \nsststile{}{} \hspace{.5em} \sststile{}{}(R \eor P) \eif (R \eor Q)$ %(KMM T56 p.109)
% ^ removed because it doesn't work right to left. Check to see if this is really in KRR.

\noindent\problempart
Prove each of the following tautologies
\begin{enumerate}[label=(\arabic*)]

\item $\sststile{}{} O \eif O$		%KMM T1, p.41
%
%	\begin{proof}
%
%	\open
%	\hypo{1}{O}\by{Want: O}{}
%	\have{2}{O}\by{R}{1}
%	\close
%	\have{3}{O \eif O} \ci{1-2}
%
%	\end{proof}

\item $\sststile{}{} N \eor \enot N$ \label{theorem_ExcludedMiddle}

%	\begin{proof}
%
%	\open
%	\hypo{1}{\enot (N \eor \enot N)} \by{for reductio}{}
%	\open
%	\hypo{2}{N} \by{for reductio}{}
%	\have{3}{N \eor \enot N} \oi{2}
%	\have{4}{\enot (N \eor \enot N)} \by{R}{1}
%	\close
%	\have{5}{\enot N} \ni{2-4}
%	\have{6}{N \eor \enot N} \oi{5}
%	\have{7}{\enot (N \eor \enot N)} \by{R}{1}
%	\close
%	\have{8}{N \eor \enot N} \ne{2-7}
%
%	\end{proof}

\item $\sststile{}{} \enot(A \eif \enot C) \eif (A \eif C)$

%	\begin{proof}
%	
%		\open
%		\hypo{1}{\enot(A \eif \enot C)} \by{Want: A \eif C}{}
%			\open
%			\hypo{2}{A} \by{Want: C}{}
%				\open
%				\hypo{3}{\enot C} \by{for reductio}{}
%					\open
%					\hypo{4}{A} \by{Want: \enot C}{}
%					\have{5}{\enot C} \by{R}{3}
%					\close
%				\have{6}{A \eif \enot C} \ci{4-5}
%				\have{7}{\enot (A \eif \enot C)} \by{R}{1}
%				\close
%			\have{8}{C} \ne{3-7}
%			\close
%		\have{9}{A \eif C} \ci{2-9}
%		\close
%	\have{10}{\enot(A \eif \enot C) \eif (A \eif C)} \ci{1-9}
%	
%	\end{proof}

\item $\sststile{}{} P \eiff (P \eor (Q \eand P))$ 

%
%1.		P				Want: P  (Q & P)
%2.		P  	(Q & P)		I 1
%3.		P  (Q & P)		Want P
%4			~P			For reductio
%5.			Q & P		E 3, 4
%6.			P			&E 5
%8.			~P			R 4
%9.		P				~E 4
%10.	P ↔ (P  (Q & P)	I 1¬–2, 3–9
%
%Appears in Hurley 10, 404 replace ASAP

\end{enumerate}


\noindent\problempart
Prove each of the following tautologies
\begin{enumerate}[label=(\arabic*)]
\item $\sststile{}{} (B \eif \enot B) \eiff \enot B$

%1.		B \eif ~B		Want: ~B
%2.			B			For reductio
%3.			~B			\eif E 1, 2
%4.			B			R2
%5.		~B				~I 2–4
%6.		~B				Want: B \eif ~B
%7.			B			Want: ~B
%8.			~B			R6
%9.		B \eif ~B
%10	(B \eif ~B) \eiff ~B

\item $\sststile{}{} (P \eif [P \eif Q]) \eif (P \eif Q)$ %(KMM T9 p. 42)

\item $\sststile{}{} (P \eor \enot P) \eand (Q \eiff Q) $ %(KMM T119 p.111)

\item $\sststile{}{} (P \eand \enot P) \eor  (Q \eiff Q)$%(KMM T120 p.111)

\end{enumerate}

% *******************************************
% *					Derived Rules				   *	
% *******************************************

\section{Derived Rules}
\setlength{\parindent}{1em}

%rob: new opening paragraph, more ambitions for this section.

Now that we have our five rules for introduction and our five rules for elimination, plus the rule of reiteration, our system is complete. If an argument is valid, and you can symbolize that argument in SL, you can \emph{prove} that the argument is valid using a derivation. (We will say a bit more about this in section \ref{sec:rules_of_rep}.) Now that our system is complete, we can really begin to play around with it and explore the exciting  logical world it creates.

There's an exciting logical world created by these eleven rules? Yes, yes there is. You can begin to see this by noticing that there are a lot of other interesting rules that we could have used for our introduction and elimination rules, but didn't. In many textbooks, the system of natural deduction has a disjunction elimination rule that works like this:

\begin{proof}
	\have[m]{ab}{\script{A}\eor\script{B}}
	\have[n]{ac}{\script{A}\eif\script{C}}
	\have[o]{bc}{\script{B}\eif\script{C}}
	\have[\ ]{c}{\script{C}} \by{${\eor}\ast$}{ab,ac,bc}
\end{proof}

You might think our system is incomplete because it lacks this alternative rule of disjunction elimination. Yet this is not the case. If you can do a proof with this rule, you can do a proof with the basic rules of the natural deduction system. You actually proved this rule in problem \ref{itm:const_d} of part \ref{derivation_set_with_const_d} in the exercises for section \ref{sec:indirect_proof}. Furthermore, once you have a proof of this rule, you can use it inside other proofs whenever you think you would need a rule like $\eor \ast$. Simply use the proof you gave in the last homework as a sort of recipe for generating a new series of steps to get you to a line saying $\script{C} \eor \script{D}$

But adding lines to a proof using this recipe all the time would be a pain in the neck. What's worse, there are dozens of interesting possible rules out there, which we could have used for our introduction and elimination rules, and which we now find ourselves replacing with recipes like the one above. 

Fortunately our basic set of introduction and elimination rules, plus reiteration, was meant to be expanded on. That's part of the game we are playing here. The first system of deduction created in the Western tradition was the system of geometry created by Euclid (c 300 BCE). Euclid's \emph{Elements}  began with 10 basic laws, along with definitions of terms like ``point,'' ``line,'' and ``plane.'' He then went on to prove hundreds of different theorems about geometry, and each time he proved a theorem he could use that theorem to help him prove later theorems. 

We can do the same thing in our system of natural deduction. What we need is a rule that will allow us to make up new rules. The new rules we add to the system will be called \define{derived rules}. Our ten rules for adding and eliminating connectives are then the \define{axioms} of SL. Now here is our rule for adding rules. 

{\narrower \narrower
 
\bf{Rule of Derived Theorem Introduction:} \rm Given a derivation in SL of some argument $A_1$ \ldots $A_n \sststile{}{} B$, create the rule $\script{A}_1$ \ldots $\script{A}_n \sststile{}{} \script{B}$ and assign a name to it of the form ``$T_n$'', to be read ``theorem n.'' Now given a derivation of some theorem $T_m$, where $n < m$, if $\script{A_1}$ \ldots $\script{A_n}$ occur as earlier lines $x_1$ \ldots $x_n$ in a proof, one may infer \script{B}, and justify it ``$T_n$, $x_1$ \ldots $x_n$'', so long as none of lines $x_1$ \ldots $x_n$ are in a closed subproof.
\par
}


Let's make our rule $\eor \ast$ above our first theorem. The proof of $T_1$ is derived simply from the recipe above.

{\narrower
\bf $T_\arabic{theorem} $ (Constructive Dilemma, CD): \rm $ \{ \script{A} \eor \script {B}, \script{A}\eif\script{C}, \script{B}\eif\script{C} \} \sststile{}{} 	\script{C}$
\addtocounter{theorem}{1}
\par
}

Proof:

\begin{proof}
	\hypo{1}{A\eor B}
	\hypo{2}{A\eif C}
	\hypo{3}{B \eif C} \by{want: C}{}
	\open
		\hypo{4}{\enot{C}}\by{for reductio}{}
			\open
			\hypo{5}{\enot A} \by{for reductio}{}
			\have{6}{B} \oe{1, 5}
			\have{7}{C}\ci{3, 6}
			\have{8}{\enot C} \by{R}{4}
			\close
		\have{9}{A}\ne{5-8}
		\have{11}{C} \ce{2, 10}
		\have{12}{\enot C} \by{R}{4}
		\close
	\have{13}{C} \ne{4-13}		
\end{proof} 



Informally, we will refer to $T_1$ as ``Constructive Dilemma'' or by the abbreviation ``CD.'' Most theorems will have names and easy abbreviations like this. We will generally use the abbreviations to refer to the proofs when we use them in derivations, because they are easier to remember. 

Several other important theorems have already appeared as examples or in homework problems. We'll talk about most of them in the next section, when we discuss rules of replacement. In the meantime, there is one important one we need to introduce now

{\narrower
$\mathbf T_\arabic{theorem}$  \bf (Modus Tollens, MT): \rm $\{ \script{A} \eif \script{B}, \enot \script{B} \} \sststile{}{} \hspace{.25em} \enot \script{A}$
\addtocounter{theorem}{1}
\par}

Proof: See page \pageref{ModusTollens}

Now that we have some theorems, let's close by looking at how they can be used in a proof. 


$\mathbf T_\arabic{theorem}$ \bf (Destructive Dilemma, DD): \rm $ \{ \script{A} \eif \script{B}, \script{A} \eif \script{C}, \enot \script{B} \eor \enot \script{C} \} \sststile{}{} \hspace{.25em} \enot \script{A}$
\addtocounter{theorem}{1}

\begin{proof}
\hypo{1}{A \eif B}
\hypo{2}{A \eif C}
\hypo{3}{\enot B \eor \enot C} \by{Want: \enot A}{}
	\open
	\hypo{4}{A} \by{for reductio}{}
	\have{5}{B} \ce{1, 4}
		\open
		\hypo{6}{\enot B} \by{For reductio}{}
		\have{7}{B} \by{R}{5}
		\have{8}{\enot B} \by{R}{6}
		\close
	\have{9}{\enot \enot B} \ni{6-8}
	\have{10}{\enot C} \oe{3, 9}
	\have{11}{A}	\by{R}{4}
	\have{12}{\enot A} \by{MT}{2, 10}
	\close
\have{13}{\enot A} \ni{4-12}
\end{proof}


%%%%%%%%%%%%%%%%%          Practice problems %%%%%%

\practiceproblems
 

\noindent\problempart
Prove the following theorems

\begin{enumerate}[label=(\arabic*)]

\item $T_\arabic{theorem}$ (Hypothetical Syllogism, HS): $ \{ \script{A} \eif \script{B}, \script{B} \eif \script{C} \} \sststile{}{} \hspace{.25em} \script{A} \eif \script{C}$
\addtocounter{theorem}{1}


%\begin{proof}
%\hypo{1}{A \eif B}
%\hypo{2}{B \eif C} \by{Want: A \eif C}{}
%	\open
%	\hypo{3}{A} \by{Want: C}{}
%	\have{4}{B} \ce{1, 3}
%	\have{5}{C} \ce{2, 4}
%	\close
%\have{6}{A \eif C} \ci{3-5}
%\end{proof}

\item $T_\arabic{theorem}$ (Idempotence of \eor, Idem\eor): $  \script{A} \eor \script{A}  \sststile{}{} \script{A} $
\addtocounter{theorem}{1}

\item $T_\arabic{theorem}$ (Idempotence of \eand, Idem\eand): $  \script{A} \sststile{}{} \script{A} \eand \script{A} $
\addtocounter{theorem}{1}

%\begin{proof}
%\hypo{1}{A} \by{Want A \eand A}{}
%\have{2}{A} \by{R}{1}
%\have{3}{A \eand A} \ai{1, 2}
%\end{proof}

\item $ T_\arabic{theorem}$ (Weakening, WK): \rm $\script{A} \sststile{}{} \script{B} \eif \script{A}$ \\
\addtocounter{theorem}{1}

\end{enumerate}

%\item $\enot\enot\enot\enot G$, $G$

\noindent\problempart
Provide proofs using both axioms and derived rules to show each of the following.
\begin{enumerate}[label=(\arabic*)]
\item \{$M \eand (\enot N \eif \enot M) \} \sststile{}{} (N \eand M) \eor \enot M$

%\begin{proof}
%\hypo{1}{M \eand (\enot N \eif \enot M)} \by{Want: $(N \eand M) \eor \enot M$}{}
%\have{2}{M} \ae{1}
%\have{3}{\enot N \eif \enot M} \ae{1}
%	\open
%	\hypo{4}{\enot N} \by{For reductio}{}
%	\have{5}{M} \by{R}{2}
%	\have{6}{\enot M} \by{\eif E}{3, 4}
%	\close
%\have{7}{N} \ne{4-6}
%\have{8}{N \eand M} \ai{2, 7}
%\have{9}{(N \eand M) \eor M} \oi{8}
%\end{proof}



\item \{$C\eif(E\eand G)$, $\enot C \eif G$\} $\sststile{}{}$ $G$
\item \{$(Z\eand K)\eiff(Y\eand M)$, $D\eand(D\eif M)$\} $\sststile{}{}$ $Y\eif Z$

%\begin{proof}
%\hypo{1}{(Z\eand K)\eiff(Y\eand M)}
%\hypo{2}{D\eand(D\eif M)} \by{Want: Y \eif Z}{}
%\have[3]{3}{D} \ae{2}
%\have[4]{4}{D \eif M} \ae{2}
%\have[5]{5}{M} \ce{3-4}
%	\open
%	\hypo[6]{6}{Y} \by{Want: Z}{}
%	\have[7]{7}{Y \eand M} \ai{5, 6}
%	\have[8]{8}{Z \eand K} \by{\eiff E, 1, 7}{}
%	\have[9]{9}{Z} \ae{8}
%	\close
%\have[10]{10}{Y \eif Z} \ci{6-9}
%\end{proof}



\item \{$(W \eor X) \eor (Y \eor Z)$, $X\eif Y$, $\enot Z$\} $\sststile{}{}$ $W\eor Y$
\item \{$(B \eif C) \eand (C \eif D), (B \eif D) \eif A $ \}$ \sststile{}{}$ $A$

%\begin{proof}
%\hypo{1}{(B \eif C) \eand (C \eif D)}
%\hypo{2}{(B \eif D) \eif A} \by{Want: A}{}\
%\have{3}{B \eif C}\by{\eand E}{1}
%\have{4}{C \eif D} \by{\eand E}{1}
%\have{5}{B \eif D} \by{HS}{3, 5}
%\have{6}{A} \by{\eif E}{2, 5}
%\end{proof}


\end{enumerate}

\noindent\problempart
\begin{enumerate}[label=(\arabic*)]

\item If you know that $\script{A}\sststile{}{}\script{B}$, what can you say about $(\script{A}\eand\script{C})\sststile{}{}\script{B}$? Explain your answer.

%1) It is valid. If you know that \script{A} on its own implies \script{B}, then a proof of $(\script{A}\eand\script{C})\sststile{}{}\script{B}$ would only require \eand E and then the proof that $\script{A}\sststile{}{}\script{B}$


\item If you know that $\script{A}\sststile{}{}\script{B}$, what can you say about $(\script{A}\eor\script{C})\sststile{}{}\script{B}$? Explain your answer.
\end{enumerate}





% *******************************************
% *				Rules of Replacement			   *	
% *******************************************
\section{Rules of Replacement}
\label{sec:rules_of_rep}
\setlength{\parindent}{1em}



Very often in a derivation, you have probably been tempted to apply a rule to a part of a line. For instance, if you knew $F\eif(G\eand H)$ and wanted $F\eif G$, you would be tempted to apply \eand E to just the $G \eand H$ part of $F \eif (G \eand H)$. But, of course you aren't allowed to do that. We will now introduce some new derived rules where you can do that. These are called \define{rules of replacement}, because they can be used to replace part of a sentence with a logically equivalent expression. What makes the rules of replacement different from other derived rules is that they draw on only one previous line and are symmetrical, so that you can reverse premise and conclusion and still have a valid argument. Some of the most simple examples are Theorems $8-10$, the rules of commutativity for \eand, \eor, and \eiff. 

{\narrower


$\mathbf T_\arabic{theorem}$  \bf (Commutativity of \eand, Comm\eand): \rm $(\script{A}\eand\script{B}) \nsststile{}{} \hspace{.25em} \sststile{}{}  (\script{B}\eand\script{A})$\\ 
\addtocounter{theorem}{1}
$\mathbf T_\arabic{theorem}$  \bf (Commutativity of \eor, Comm\eor): \rm $(\script{A}\eor\script{B}) \nsststile{}{} \hspace{.25em} \sststile{}{} (\script{B}\eor\script{A})$\\
\addtocounter{theorem}{1}
$\mathbf T_{\arabic{theorem}}$  \bf (Commutativity of \eiff, Comm\eiff): \rm $(\script{A}\eiff\script{B}) \nsststile{}{} \hspace{.25em} \sststile{}{} (\script{B}\eiff\script{A})$
\addtocounter{theorem}{1}

\par}


You will be asked to prove these in the homework. In the meantime, let's see an example of how they work in a proof. Suppose you wanted to prove $(M \eor P) \eif (P \eand M)$, $\therefore$\ $(P \eor M) \eif (M \eand P)$ You could do it using only the basic rules, but it will be long and inconvenient. With the Comm rules, we can provide a proof easily:

\begin{proof}
	\hypo{1}{(M \eor P) \eif (P \eand M)}
	\have{2}{(P \eor M) \eif (P \eand M)}\by{Comm\eand}{1}
	\have{n}{(P \eor M) \eif (M \eand P)}\by{Comm\eor}{2}
\end{proof}

Formally, we can put our rule for deploying rules of replacement like this

{\narrower \narrower
 
\noindent\bf Inserting rules of replacement: \rm Given a theorem T of the form $\script{A} \nsststile{}{} \hspace{.5em} \sststile{}{} \script B$ and a line in a derivation \script{C} which contains in it a sentence \script{D}, where \script{D} is a substitution instance of either \script{A} or \script{B}, replace \script{D} with the equivalent substitution instance of the other side of theorem T. 
\setlength{\parindent}{1em}

\par}

\setlength{\parindent}{1em}

Here are some other important theorems that can act as rules of replacement. Some are theorems we have already proved, while you will be asked to prove others in the homework.

{\narrower

$\mathbf T_{\arabic{theorem}}$ \bf (Double Negation, DN): \rm $\script{A} \nsststile{}{} \hspace{.5em} \sststile{}{} \hspace{.25em} \enot \enot \script{A}$
\addtocounter{theorem}{1}
\par}
Proof: See pages \pageref{DN1} and \pageref{DN2}. 

{\narrower

$\mathbf T_{\arabic{theorem}}$: \rm $\enot(\script{A} \eor \script{B}) \nsststile{}{} \hspace{.5em} \sststile{}{} \hspace{.25em} \enot \script{A} \eand \enot \script{B}$
\addtocounter{theorem}{1}
\par}
Proof: See page \pageref{DeM1}

{\narrower
$\mathbf T_{\arabic{theorem}}$: \rm $\enot (\script{A} \eand \script{B}) \nsststile{}{} \hspace{.5em} \sststile{}{} \hspace{.25em} \enot \script{A} \eor \enot \script{B}$
\addtocounter{theorem}{1}
\par}

Proof: See pages \pageref{DeM3} and \pageref{DeM4}.



$ T_{12}$  and $T_{13}$ are collectively known as \define{DeMorgan's Laws}, and we will use the abbreviation DeM to refer to either of them in proofs.


{\narrower

$ \mathbf T_{\arabic{theorem}}$: \rm $(\script{A}\eif\script{B}) \nsststile{}{} \hspace{.5em} \sststile{}{} \hspace{.25em} (\enot\script{A}\eor\script{B})$ 
\addtocounter{theorem}{1}


$ \mathbf T_{\arabic{theorem}}$: \rm $(\script{A}\eor\script{B}) \nsststile{}{} \hspace{.5em} \sststile{}{} \hspace{.25em} (\enot\script{A}\eif\script{B})$  
\addtocounter{theorem}{1}

\par}

$ T_{14}$ and $T_{15}$ are collectively known as the rule of Material Conditional (MC). You will prove them in the homework. 

 
$ \mathbf T_{\arabic{theorem}}$ \bf (Biconditional Exportation, ex): \rm $\script{A}\eiff \script{B} \nsststile{}{} \hspace{.5em} \sststile{}{} (\script{A} \eif \script{B})\eand(\script{B}\eif \script{\script{A}})$ 
\addtocounter{theorem}{1}
\setlength{\parindent}{1em}

Proof: See the homework.

{\narrower
$ \mathbf T_{\arabic{theorem}}$ \bf (Transposition, trans): \rm $\script{A}\eif \script{B} \nsststile{}{} \hspace{.5em} \sststile{}{} \hspace{.25em} \enot \script{B} \eif \enot \script{A}$ \addtocounter{theorem}{1}
\par}

Proof: See the homework.


To see how much these theorems can help us, consider this argument: $$\enot(P \eif Q) \sststile{}{} P \eand \enot Q$$

As always, we could prove this argument using only the basic rules. With rules of replacement, though, the proof is much simpler:

 

\begin{proof}
	\hypo{1}{\enot(P \eif Q)}
	\have{2}{\enot(\enot P \eor Q)}\by{MC}{1}
	\have{3}{\enot\enot P \eand \enot Q}\by{DeM}{2}
	\have{4}{P \eand \enot Q}\by{DN}{3}
\end{proof}

%Although they don't do it in the book, I've been in the habit of writing $(\script{A}\eand\script{B}\eand\script{C})$ and dropping the inner pair of parentheses. This is fine. If we'd wanted to, we could have defined the basic rules in a more general way:

%\begin{proof}
%	\have[n]{a1}{\script{A}_1}
%	\have{2}{\script{A}_2}
%	\have[\vdots]{1}{\vdots}
%	\have[n]{an}{\script{A}_n}
%	\have[\ ]{aaa}{\script{A}_1~\eand\ldots\eand~\script{A}_n} \ai{}
%\end{proof}

%\bigskip
%\begin{proof}
%	\have{3}{\script{A}_1~\eand\ldots\eand~\script{A}_n}
%	\have{1}{\script{A}_i} \ae{}
%\end{proof}

%\bigskip
%\begin{proof}
%	\have{1}{\script{A}}
%	\have{3}{\script{A}\eor\script{B}_1\eor\script{B}_2\ldots\eor\script{B}_n} \ai{}
%\end{proof}

%We don't need these extended versions, since for any given n we could prove them as a derived rule.




%%%%%%%%%%%%%          Practice problems %%%%%%%%%
 

\practiceproblems
\noindent\problempart
Prove $T_{8}$ through $T_{10}$. You may use $T_{1}$ through $T_7$ in your proofs.

\noindent\problempart
Prove $T_{11}$ through $T_{17}$. You may use $T_{1}$ through $T_{12}$ in your proofs.




% *******************************************
% *				Proof Strategy					   *	
% *******************************************


\section{Proof Strategy}
\setlength{\parindent}{1em}
There is no simple recipe for proofs, and there is no substitute for practice. Here, though, are some rules of thumb and strategies to keep in mind.

\emph{Work backwards from what you want.}
The ultimate goal is to derive the conclusion. Look at the conclusion and ask what the introduction rule is for its main logical operator. This gives you an idea of what should happen \emph{just before} the last line of the proof. Then you can treat this line as if it were your goal. Ask what you could do to derive this new goal. For example: If your conclusion is a conditional $\script{A}\eif\script{B}$, plan to use the {\eif}I rule. This requires starting a subproof in which you assume \script{A}. In the subproof, you want to derive \script{B}. Similarly, if your conclusion is a biconditional, $\script{A} \eiff \script{B}$, plan on using {\eiff}I and be prepared to launch two subproofs. If you are trying to prove a single sentence letter or a negated single sentence letter, you might plan on using indirect proof. 

%Rob: I removed QL examples here and put ih more SL examples

\emph{Work forwards from what you have.}
When you are starting a proof, look at the premises; later, look at the sentences that you have derived so far. Think about the elimination rules for the main operators of these sentences. These will tell you what your options are. For example: If you have $A \eand B$ use \eand E to get $A$ and $B$ separately. If you have $A \eor B$ see if you can find the negation of either $A$ or $B$ and use \eor E.

\emph{Repeat as necessary.} Once you have decided how you might be able to get to the conclusion, ask what you might be able to do with the premises. Then consider the target sentences again and ask how you might reach them.  Remember, a long proof is formally just a number of short proofs linked together, so you can fill the gap by alternately working back from the conclusion and forward from the premises.

%Rob: I deleted a sentence from ``work forward from what you have'' that didn't make sense and then merged other material in the the ``repeat as necessary'' section.. 

\emph{Change what you are looking at.} Replacement rules can often make your life easier. If a proof seems impossible, try out some different substitutions.For example: It is often difficult to prove a disjunction using the basic rules. If you want to show $\script{A}\eor\script{B}$, it is often easier to show $\enot\script{A}\eif\script{B}$ and use the MC rule. Some replacement rules should become second nature. If you see a negated disjunction, for instance, you should immediately think of DeMorgan's rule.

\emph{When all else fails, try indirect proof.} If you cannot find a way to show something directly, try assuming its negation. Remember that most proofs can be done either indirectly or directly. One way might be easier---or perhaps one sparks your imagination more than the other---but either one is formally legitimate.

%Rob: I changed the way the advice is phrased to match the slogan I repeat in class.

%\emph{Persist.} Try different things. If one approach fails, then try something else.
% Rob: deleted ``persist'' in favor of other advice.

\emph{Take a break} If you are completely stuck, put down your pen and paper, get up from your computer, and do something completely different for a while. Walk the dog. Do the dishes. Take a shower. I find it especially helpful to do something physically active. Doing other desk work or watching TV doesn't have the same effect. When you come back to the problem, everything will seem clearer. Of course, if you are in a testing situation, taking a break to walk around might not be advisible. Instead, switch to another problem.

A lot of times, when you are stuck, your mind keeps trying the same solution again and again, even though you know it won't work. ``If I only knew $Q \eif R$,'' you say to yourself, ``it would all work. Why can't I derive $Q \eif R$!'' If you go away from a problem and then come back, you might not be as focused on That One Thing that you were sure you needed, and you can find a different approach.

	
%I added the section on taking a break


%%%%%%%Practice problems %%%%%%%%%

\practiceproblems
 
\noindent\problempart
 
Show the following theorems are valid. Feel free to use $T_{1}$ through $T_{17}$

\begin{enumerate}[label=(\arabic*)]
\item $ T_{\arabic{theorem}}$ (Associativity of \eand, Ass\eand): \rm $(\script{A} \eand \script{B}) \eand \script{C} \nsststile{}{} \hspace{.25em} \sststile{}{} \script{A} \eand (\script{B} \eand \script{C})$ \\ \addtocounter{theorem}{1}
\item $ T_{\arabic{theorem}}$  (Associativity of \eor, Ass\eor): \rm $(\script{A} \eor \script{B}) \eor \script{C} \nsststile{}{} \hspace{.25em} \sststile{}{} \script{A} \eor (\script{B} \eor \script{C})$ 	\\ \addtocounter{theorem}{1}
\item $ T_{\arabic{theorem}}$  (Associativity of \eiff, Ass\eiff): \rm $(\script{A} \eiff \script{B}) \eiff \script{C} \nsststile{}{} \hspace{.25em} \sststile{}{} \script{A} \eiff (\script{B} \eiff \script{C})$ 	\addtocounter{theorem}{1}
\end{enumerate}


% *******************************************
% *			Soundness and completeness			   *	
% *******************************************

%I merged sections 6.7, 6.8, and 6.9 and restricted the material to SL to create this section. 
\section{Soundness and completeness}
\label{sec:soundness_and_completeness}
In section 4.6, we saw that we could use derivations to test for the same concepts we used truth tables to test for. Not only could we use derivations to prove that an argument is valid, we could also use them to test if a statement is a tautology or a pair of statements are equivalent. We also started using the single turnstile the same way we used the double turnstile. If we could prove that \script{A} was a tautology with a truth table, we wrote $\sdtstile{}{}\script{A}$, and if we could prove it using a derivation, we wrote $\sststile{}{}\script{A}.$ 

You may have wondered at that point if the two kinds of turnstiles always worked the same way. If you can show that \script{A} is a tautology using truth tables, can you also always show that it is true using a derivation? Is the reverse true? Are these things also true for tautologies and pairs of equivalent sentences? As it turns out, the answer to all these questions and many more like them is yes. We can show this by defining all these concepts separately and then proving them equivalent. That is, we imagine that we actually have two notions of validity, $valid_{\models}$ and  $valid_{\vdash}$ and then show that the two concepts always work the same way. 

\newglossaryentry{syntactic contradiction in SL}
{
name=syntactic contradiction in SL,
description={A statement in SL whose negation can be derived without any premises.}
}


   
\newglossaryentry{syntactically contingent in SL}
{
name=syntactically contingent in SL,
description={A property held by a statement in SL if and only if it is not a syntactic tautology or a syntactic contradiction.}
}




To begin with, we need to define all of our logical concepts separately for truth tables and derivations. A lot of this work has already been done. We handled all of the truth table definitions in Chapter \ref{chap:truth_tables}. We have also already given syntactic definitions for a tautologies and pairs of logically equivalent sentences. The other definitions follow naturally. For most logical properties we can devise a test using derivations, and those that we cannot test for directly can be defined in terms of the concepts that we can define.

For instance, we defined a syntactic tautology as a statement that can be derived without any premises (p. \pageref{def:syntactic_tautology_in_sl}). Since the negation of a contradiction is a tautology, we can define a \textsc{\gls{syntactic contradiction in SL}} \label{def:syntactic_contradiction_in_sl} as a sentence whose negation can be derived without any premises. The syntactic definition of a contingent sentence is a little different. We don't have any practical, finite method for proving that a sentence is contingent using derivations, the way we did using truth tables. So we have to content ourselves with defining ``contingent sentence'' negatively. A sentence is \textsc{\gls{syntactically contingent in SL}} \label{def:syntactically_contingent_in_sl} if it is not a syntactic tautology or contradiction. 
 
\newglossaryentry{syntactically inconsistent in SL}
{
name=syntactically inconsistent in SL,
description={A property held by sets of sentences in SL if and only if one can derive a contradiction from them.}
}

\newglossaryentry{syntactically consistent in SL}
{
name=syntactically consistent in SL,
description={A property held by sets of sentences in SL if and only if they are not syntactically inconsistent.}
}

A set of sentences is \textsc{\gls{syntactically inconsistent in SL}} \label{def:syntactically_inconsistent_ in_sl} if and only if one can derive a contradiction from them. Consistency, on the other hand, is like contingency, in that we do not have a practical finite method to test for it directly. So again, we have to define a term negatively. A set of set of sentences is \textsc{\gls{syntactically consistent in SL}} \label{def:syntactically consistent in SL} if and only if they are not syntactically inconsistent.
    
\newglossaryentry{syntactically valid in SL}
{
name=syntactically valid in SL,
description={A property held by arguments in SL if and only if there is a derivation that goes from the premises to the conclusion.}
}

Finally, an argument is \textsc{\gls{syntactically valid in SL}} \label{def:syntactically_valid_in_SL} if and only if there is a derivation of it. All of these definitions are given in Table \ref{table:truth_tables_or_derivations}.


\begin{sidewaystable}
\begin{mdframed}[style=mytablebox]
\tabulinesep=1ex
\begin{tabu}{X[.5,c,m] ||X[1,l,m] |X[1,l,m]}
\bf{Concept} 		&	\bf{Truth table (semantic) definition} 	&	\bf{Derivation (syntactic) definition} \\ \hline \hline

Tautology  &	A statement whose truth table only has Ts under the main connective & A statement that can be derived without any premises.	 \\ \hline
 
Contradiction		&	A statement whose truth table only has Fs under the main connective  &	A statement whose negation can be derived without any premises\\ \hline

Contingent sentence	&	A statement whose truth table contains both Ts and Fs under the main connective & A statement that is not a syntactic tautology or contradiction \\ \hline

Equivalent sentences &	The columns under the main connectives are identical.& The statements can be derived from each other	\\ \hline

Inconsistent sentences	&	Sentences which do not have a single line in their truth table where they are all true.	& Sentences which one can derive a contradiction from \\ \hline

Consistent sentences	&	Sentences which have at least one line in their truth table where they are all true. & Sentences which are no inconsistent	\\ \hline

Valid argument		&	An argument whose truth table has no lines where there are all Ts under main connectives for the premises and an F under the main connective for the conclusion.  & An argument where can derive the conclusion from the premises	\\ 
\end{tabu}
\end{mdframed}
\caption{Two ways to define logical concepts.}
\label{table:truth_tables_or_derivations}
\end{sidewaystable}

All of our concepts have now been defined both semantically and syntactically. How can we prove that these definitions always work the same way? A full proof here goes well beyond the scope of this book. However, we can sketch what it would be like. We will focus on showing the two notions of validity to be equivalent.  From that the other concepts will follow quickly. The proof will have to go in two directions. First we will have to show that things which are syntactically valid will also be semantically valid. In other words, everything that we can prove using derivations could also be proven using truth tables. Put symbolically, we want to show that $valid_{\vdash}$ implies $valid_{\models}$. Afterwards, we will need to show things in the other directions,  $valid_{\models}$ implies $valid_{\vdash}$

\newglossaryentry{soundness}
{
name=soundness,
description={A property held by logical systems if and only if $\sststile{}{}$ implies $\sdtstile{}{}$}
}

This argument from $\sststile{}{}$ to $\sdtstile{}{}$ is the problem of \textsc{\gls{soundness}}. \label{def:soundness} A proof system is \define{sound} if there are no derivations of arguments that can be shown invalid by truth tables. \label{def_Soundness} Demonstrating that the proof system is sound would require showing that \emph{any} possible proof is the proof of a valid argument. It would not be enough simply to succeed when trying to prove many valid arguments and to fail when trying to prove invalid ones.

The proof that we will sketch depends on the fact that we initially defined a sentence of SL using a recursive definition (see p. \pageref{def:recursive_definition}). We could have also used recursive definitions to define a proper proof in SL and a proper truth table. \nix{Later this will be a truth assignment}(Although we didn't.) If we had these definitions, we could then use a \emph{recursive proof} to show the soundness of SL. A recursive proof works the same way a recursive definition does.With the recursive definition, we identified a group of base elements that were stipulated to be examples of the thing we were trying to define. In the case of a well formed formula, the base class was the set of sentence letters A, B, C \ldots{}. We just announced that these were sentences. The second step of a recursive definition is to say that anything that is built up from your base class using certain rules also counts as an example of the thing you are defining. In the case of a definition of a sentence, the rules corresponded to the five sentential connectives (see p. \pageref{def:sentence_of_SL}). Once you have established a recursive definition, you can use that definition to show that all the members of the class you have defined have a certain property. You simply prove that the property is true of the members of the base class, and then you prove that the rules for extending the base class don't change the property. This is what it means to give a recursive proof.

Even though we don't have a recursive definition of a proof in SL, we can sketch how a recursive proof of the soundness of SL would go. Imagine a base class of one-line proofs, one for each of our eleven rules of inference. The members of this class would look like this $\{\script{A}, \script{B}\} \sststile{}{} \script{A} \eand \script{B}$; $\script{A} \eand \script{B} \sststile{}{}\script{A}$; $\{\script{A} \eor \script{B}, \enot\script{A}\} \sststile{}{} \script{B}$ \ldots{} etc. Since some rules have a couple different forms, we would have to have add some members to this base class, for instance $\script{A} \eand \script{B} \sststile{}{} \script{B}$ Notice that these are all statements in the metalanguage. The proof that SL is sound is not a part of SL, because SL does not have the power to talk about itself. 

You can use truth tables to prove to yourself that each of these one-line proofs in this base class is $valid_{\models}$. For instance the proof $\{\script{A}, \script{B}\} \sststile{}{} \script{A} \eand \script{B}$ corresponds to a truth table that shows $\{\script{A}, \script{B}\} \sdtstile{}{} \script{A} \eand \script{B}$ This establishes the first part of our recursive proof. 

The next step is to show that adding lines to any proof will never change a $valid_{\models}$ proof into an $invalid_{\models}$ one. We would need to this for each of our eleven basic rules of inference. So, for instance, for \eand{I} we need to show that for any proof $\script{A}_{1} \ldots{} \script{A}_{n} \sststile{}{} \script {B}$ adding a line where we use \eand{I} to infer $\script{C} \eand \script{D}$, where $\script{C} \eand \script{D}$ can be legitimately inferred from $\{\script{A}_{1} \ldots{} \script{A}_{n}, \script {B}\}$, would not change a valid proof into an invalid proof. But wait, if we can legitimately derive $\script{C} \eand \script{D}$ from these premises, then $\script{C} and \script{D}$ must be already available in the proof. They are either members of  $\{\script{A}_{1} \ldots{} \script{A}_{n}, \script {B}\}$ or can be legitimately derived from them. As such, any truth table line in which the premises are true must be a truth table line in which \script{C} and \script{D} are true. According to the characteristic truth table for \eand, this means that \script{C}\eand\script{D} is also true on that line. Therefore, \script{C}\eand\script{D} validly follows from the premises. This means that using the {\eand}E rule to extend a valid proof produces another valid proof.

In order to show that the proof system is sound, we would need to show this for the other inference rules. Since the derived rules are consequences of the basic rules, it would suffice to provide similar arguments for the 11 other basic rules. This tedious exercise falls beyond the scope of this book.

So we have shown that $\script{A} \sststile{}{} \script{B}$ implies $\script{A} \sdtstile{}{}\script{B}.$ What about the other direction, that is why think that \emph{every} argument that can be shown valid using truth tables can also be proven using a derivation. 

\newglossaryentry{completeness}
{
name=completeness,
description={A property held by logical systems if and only if $\sdtstile{}{}$ implies $\sststile{}{}$}
}

This is the problem of completeness. A proof system has the property of  \textsc{\gls{completeness}} \label{def:completeness} if and only if there is a derivation of every semantically valid argument. Proving that a system is complete is generally harder than proving that it is sound. Proving that a system is sound amounts to showing that all of the rules of your proof system work the way they are supposed to. Showing that a system is complete means showing that you have included \emph{all} the rules you need, that you haven't left any out. Showing this is beyond the scope of this book. The important point is that, happily, the proof system for SL is both sound and complete. This is not the case for all proof systems and all formal languages. Because it is true of SL, we can choose to give proofs or give truth tables---whichever is easier for the task at hand.

Now that we know that the truth table method is interchangeable with the method of derivation, you can chose which method you want to use for any given problem. Students often prefer to use truth tables, because a person can produce them purely mechanically, and that seems `easier'. However, we have already seen that truth tables become impossibly large after just a few sentence letters. On the other hand, there are a couple situations where using derivations simply isn't possible. We syntactically defined a contingent sentence as a sentence that couldn't be proven to be a tautology or a contradiction. There is no practical way to prove this kind of negative statement. We will never know if there isn't some proof out there that a statement is a contradiction and we just haven't found it yet. We have nothing to do in this situation but resort to truth tables. Similarly, we can use derivations to prove two sentences equivalent, but what if we want to prove that they are \emph{not} equivalent? We have no way of proving that we will never find the relevant proof. So we have to fall back on truth tables again.

Table \ref{table.ProofOrModel} summarizes when it is best to give proofs and when it is best to give truth tables. 

\begin{table}
\tabulinesep=1ex
\begin{mdframed}[style=mytablebox]
\begin{tabu}{X[.5,l,b] X[1,l,b] X[1,l,b]}
\underline{Property}		& \underline{To prove it present} 	&	\underline{To prove it absent} \\ 
Being a tautology 			& Derive the statement  						& Find the false line in the truth table for the sentence \\ 
Being a contradiction 		&  Derive the negation of the statement  		 & Find the true line in the truth table for the sentence\\ 
Contingency			 		& Find a false line and a true line in the truth table for the statement & Prove the statement or its negation\\ 
Equivalence 					& Derive each statement from the other 		 & Find a line in the truth tables for the statements where they have different values\\ 
Consistency	 				& Find a line in truth table for the sentence where they all are true & Derive a contradiction from the sentences\\ 
Validity		 				& Derive the conclusion form the premises & Find a line in the truth table where the premises are true and the conclusion false. \\ 
\end{tabu}
\end{mdframed}
\caption{When to provide a truth table and when to provide a proof.}
\label{table.ProofOrModel}
\end{table}



\practiceproblems
\noindent\problempart Use either a derivation or a truth table for each of the following. 
\begin{enumerate}[label=(\arabic*)]
\item Show that $A \eif [((B \eand C) \eor D) \eif A]$ is a tautology.
\item Show that $A \eif (A \eif B)$ is not a tautology
\item Show that the sentence $A \eif \enot{A}$ is not a contradiction.
\item Show that the sentence $A \eiff \enot A$ is a contradiction. 
\item Show that the sentence $ \enot (W \eif (J \eor J)) $ is contingent
\item Show that the sentence $ \enot(X \eor (Y \eor Z)) \eor (X \eor (Y \eor Z))$ is not contingent
 \item Show that the sentence $B \eif \enot S$ is equivalent to the sentence $\enot \enot B \eif \enot S$
\item Show that the sentence $ \enot (X \eor O) $ is not equivalent to the sentence $X \eand O$
\item Show that the set $\{\enot(A \eor B), C, C \eif A\}$ is inconsistent.
\item Show that the set \{\enot(A \eor B), \enot{B}, B \eif A\} is consistent
\item Show that $\enot(A \eor (B \eor C)) $ \therefore $ \enot{C}$ is valid.
\item Show that $\enot(A \eand (B \eor C))$ \therefore $ \enot{C}$ is invalid. 
\end{enumerate}


\noindent\problempart Use either a derivation or a truth table for each of the following. 
\begin{enumerate}[label=(\arabic*)]
\item Show that $A \eif (B \eif A)$ is a tautology
\item Show that $\enot (((N \eiff Q) \eor Q) \eor N)$ is not a tautology
\item Show that $ Z \eor (\enot Z \eiff Z) $ is contingent
\item show that $ (L \eiff ((N \eif N) \eif L)) \eor H $ is not contingent
\item Show that $ (A \eiff A) \eand (B \eand \enot B)$ is a contradiction
\item Show that $ (B \eiff (C \eor B)) $ is not a contradiction.
\item Show that $ ((\enot X \eiff X) \eor X) $ is equivalent to $X$
\item Show that $F \eand (K \eand R) $ is not equivalent to $ (F \eiff (K \eiff R)) $
\item Show that the set \{$ \enot (W \eif W)$, $(W \eiff W) \eand W$, $E \eor (W \eif \enot (E \eand W))$\} is inconsistent.
\item Show that the set  \{$\enot R \eor C $, $(C \eand R) \eif \not R$, $(\enot (R \eor R) \eif R) $\} is consistent.
\item Show that $\enot \enot (C \eiff \enot C), ((G \eor C) \eor G) \therefore ((G \eif C) \eand G) $ is valid.
\item Show that $ \enot \enot L,  (C \eif \enot L) \eif C) \therefore \enot C$ is invalid. 
\end{enumerate}


%\noindent\problempart
%Show that each of the following is provably inconsistent.
%\begin{earg}
%\item \{$Sa\eif Tm$, $Tm \eif Sa$, $Tm \eand \enot Sa$\}
%\end{earg}
%convert last item to something in SL

%
% Below is the closing tag for typesetting only part of the chapter. Everything up to here to the close tag will be skipped unless the {whole_slproof_chap} label at the start of this chapter file is 
%  uncommented.

}{}



\section*{Key Terms}
\begin{multicols}{2}
\begin{sortedlist}
\sortitem{sentence form}{}

\sortitem{substitution instance}{}

\sortitem{argument form}{}

\sortitem{substitution instance of an argument form}{}

\sortitem{proof}{}

\iflabelexists{def:syntactically_logically_equivalent_in_sl}{\sortitem{Syntactically logically equivalent in SL}{}}{}

\iflabelexists{def:syntactic_tautology_in_sl}{\sortitem{Syntactic tautology in SL}{}}{}

\iflabelexists{syntactic contradiction in SL}{\sortitem{Syntactic contradiction in SL}{}}{}

\iflabelexists{def:syntactically_contingent_in_sl}{\sortitem{Syntactically contingent in SL}{}}{}

\iflabelexists{def:syntactically_inconsistent_ in_sl}{\sortitem{Syntactically inconsistent in SL}{}}{}

\iflabelexists{def:syntactically consistent in SL}{\sortitem{Syntactically consistent in SL}{}}{}

\iflabelexists{def:syntactically_valid_in_SL}{\sortitem{Syntactically valid in SL}{}}{}

\iflabelexists{def:soundness}{\sortitem{Soundness}{}}{}

\iflabelexists{def:completeness}{\sortitem{Completeness}{}}{} 	

\end{sortedlist}
\end{multicols}







%Label for typesetting full chapter is at the start of the file. Uncomment to get the whole thing. 
%\part{Quantificational Logic} \label{part:quant_logic}
%\include{tex/ch09-predicate}
%\chapter{Semantics for Quantified Logic}
\label{chap:semantics_for_ql}
\markright{Chap \ref{chap:semantics_for_ql}: Semantics for QL}
\setlength{\parindent}{1em}


% *****************************
% *		Creating Models in QL      *
% ****************************

\section{Creating models in Quantified Logic}

In Chapter 3 we developed the truth table system, a semantic method for evaluating sentences and arguments in SL. The method was called semantic, because it involved establishing meanings for all the non-logical parts of a sentence, in this case, the sentence letters. These were called non-logical because their meaning was not fixed by the rules of SL. The logical part of the language was the system of connectives and parentheses. As it turned out, the logical parts of our language were truth functional: the meaning of the larger sentences built up with the connectives and parentheses was simply a function of the truth of the smaller parts. Because the system of sentential connectives and parentheses is truth functional, our semantic system only needed to look at one aspect of the meaning of the sentence letters, their truth value. Thus an interpretation of a sentence in SL simply turned out to be a truth assignment on the sentence letters. That's why you spent Chapter 3 merrily writing Ts and Fs under sentence letters. Each little row of Ts and Fs represented one way of assigning meaning to a sentence or argument, and all the little rows in the truth table represented all the possible ways of assigning meaning to the sentence or argument (at least as far as truth was concerned.) 

In this chapter we are going to develop similar methods for QL. Because QL is a more complicated system than SL developing semantics for QL will be much more complicated, and there will be greater limitations for what we can do with these methods.

\subsection{Basic Models}

The first thing we need to do is to find an equivalent in QL of a little row of Ts and Fs beneath a sentence in SL. We called this little row of Ts and Fs an interpretation, because it gave one aspect of the meaning of the parts of the sentence, namely their truth values. To come up with an interpretation of a sentence in QL, we will need to look at more than the truth or falsity of sentences, though. We have, after all, broken open the atomic sentence to look at its subatomic parts, and now we will need to assign meaning to these parts. The way to start is to look at the symbolization keys we created when we were translating sentences in and out of QL. The symbolization key contained a universe of discourse (UD), a meaning for each predicate, and an object picked out by each term. For example:
\begin{ekey}
\item[UD:] comic book characters
\item[Fx:] $x$ fights crime.
\item[b:] Batman
\item[w:] Bruce Wayne
\end{ekey}

Given this symbolization key, we can translate sentences like this

\begin{earg}
\item[] $Fb$: Batman fights crime.
\item[] $Fw$: Bruce Wayne fights crime.
\item[] $Fb \eif Fw$: If Batman fights crime, then Bruce Wayne fights crime.  
\end{earg}
This symbolization key, along with some basic knowledge of superhero stories, gives us enough information to figure out that the sentence $Fb$ is true. (Notice that the sentence $Fb$ is not true \emph{just because} of the interpretation. The way we interpret the sentence, plus the way the world is, makes the sentence true.) 

This is the information we need to develop an interpretation of sentences in QL. We are not just assigning truth values anymore. We need to get deeper into meaning here, and look at the \emph{reference} of the parts of the sentence. \define{Reference} is an aspect of meaning that deals with the way bits of language pick out or identify objects in the real world. For our purposes the real world is the universe of discourse, the set of objects we are talking about. To explain how a singular term like $b$ refers, we need only point out the member of the UD it refers to, in this case, Batman. In more technical terms, Batman is the \emph{referent} of B. For our purposes it is enough to define a referent like this: the \define{referent} of a term in QL is the unique object picked out by that term. 

(Notice that here to explain what $b$ refers to, I have been forced to simply use another singular term in another language, the name ``Batman.'' To really pick out the object referred to, I would need to draw your attention to something in the actual world, for instance by holding up a comic book and pointing at the character on the cover.)

We want to use similar methods to talk about the meaning of predicates. To do this we will talk about the \emph{extension} of a predicate. The \define{extension} of a predicate is the set of objects in the UD that the predicate applies to. So if $b$ is in the extension of the predicate $F$ then the sentence $Fb$ is true. 

(Identifying the extension of a predicate like $F$ again forces us to rely on another language, English. We can say that the extension of $F$ includes Batman, Superman, Green Lantern, etc. The situation is worse because $F$ has an indefinitely large extension, so we have relied on the English predicate ``fights crime'' to describe it.) 

All this means that we are able to talk about the meaning of sentences in QL entirely in terms of sets. We use curly brackets \{ and \} to denote sets. The members of the set can be listed in any order, separated by commas. The fact that sets can be in any order is important, because it means that \{foo, bar\} and \{bar, foo\} are the same set. It is possible to have a set with no members in it. This is called the \define{empty set}. The empty set is sometimes written as \{\}, but usually it is written as the single symbol $\emptyset$.

We are now able to give the equivalent of a line in a truth table for QL. An interpretation in QL will be called a \emph{model}. A \define{model} of sentences or arguments in QL consists of a set that is the universe of discourse, individual members of that set that are the referents of the singular terms in the sentences or arguments, and subsets of the universe of discourse which are the extensions of the predicates used in the sentences or arguments. 

To see how this works imagine I have a bunch of sentences in QL, which include the predicate $H$ and the singular term $f$. Now consider this symbolization key:
\begin{ekey}
\item[UD:]People who played as part of the Three Stooges
\item[Hx:]$x$ had head hair.
\item[f:] Mister Fine
\end{ekey}

What is the model that corresponds to this interpretation? There were six people who played as part of the Three Stooges over the years, so the UD will have six members: Larry Fine, Moe Howard, Curly Howard, Shemp Howard, Joe Besser, and Curly Joe DeRita. Curly, Joe, and Curly Joe were the only completely bald stooges. The result is this model:
\begin{partialmodel}
	UD & \{Larry, Curly, Moe, Shemp, Joe, Curly Joe\}\\
	\extension{H} & \{Larry, Moe, Shemp\}\\
	\referent{f} & Larry
\end{partialmodel}

You do not need to know anything about the Three Stooges in order to evaluate whether sentences are true or false in this \emph{model}. $Hf$ is true, since the referent of $f$ (Larry) is in the extension of $H$. Both $\exists x Hx$ and $\exists x \enot Hx$ are true, since there is at least one member of the UD that is in the extension of $H$ and at least one member that is not in the extension of $H$. In this way, the model captures all of the formal significance of the interpretation.

\subsection{Models for multiplace predicates}

Now consider this interpretation:
\begin{ekey}
\item{UD:} whole numbers less than 10
\item{Ex:} $x$ is even.
\item{Nx:} $x$ is negative.
\item{Lxy:} $x$ is less than $y$.
\item{Txyz:} $x$ times $y$ equals $z$.
\end{ekey}
What is the model that goes with this interpretation?
The UD is the set $\{1,2,3,4,5,6,7,8,9\}$.

The extension of a one-place predicate like $E$ or $N$ is just the subset of the UD of which the predicate is true. The extension of $E$ is the subset $\{2,4,6,8\}$. There are many even numbers besides these four, but these are the only members of the UD that are even. There are no negative numbers in the UD, so $N$ has an empty extension; i.e. $\extension{N}=\emptyset$.

The extension of a two-place predicate like $L$ is somewhat vexing. It seems as if the extension of $L$ ought to contain 1, since 1 is less than all the other numbers; it ought to contain 2, since 2 is less than all of the other numbers besides 1; and so on. Every member of the UD besides 9 is less than some member of the UD. What would happen if we just wrote $\extension{L}=\{1,2,3,4,5,6,7,8\}$?

The problem is that sets can be written in any order, so this would be the same as writing $\extension{L}=\{8,7,6,5,4,3,2,1\}$. This does not tell us which of the members of the set are less than which other members.

We need some way of showing that 1 is less than 8 but that 8 is not less than 1. The solution is to have the extension of $L$ consist of pairs of numbers. An \define{ordered pair} is like a set with two members, except that the order \emph{does} matter. We write ordered pairs with angle brackets $<$ and $>$. The ordered pair \mbox{$<$foo, bar$>$} is different than the ordered pair \mbox{$<$bar, foo$>$}. The extension of $L$ is a collection of ordered pairs, all of the pairs of numbers in the UD such that the first number is less than the second. Writing this out completely:
\begin{quote}
$\extension{L}=$ \{$<$1,2$>$, $<$1,3$>$, $<$1,4$>$, $<$1,5$>$, $<$1,6$>$, $<$1,7$>$, $<$1,8$>$, $<$1,9$>$,
$<$2,3$>$, $<$2,4$>$, $<$2,5$>$, $<$2,6$>$, $<$2,7$>$, $<$2,8$>$, $<$2,9$>$,
$<$3,4$>$, $<$3,5$>$, $<$3,6$>$, $<$3,7$>$, $<$3,8$>$, $<$3,9$>$,
$<$4,5$>$, $<$4,6$>$, $<$4,7$>$, $<$4,8$>$, $<$4,9$>$,
$<$5,6$>$, $<$5,7$>$, $<$5,8$>$, $<$5,9$>$,
$<$6,7$>$, $<$6,8$>$, $<$6,9$>$,
$<$7,8$>$, $<$7,9$>$,
$<$8,9$>$%
\}
\end{quote}

Three-place predicates will work similarly; the extension of a three-place predicate is a set of ordered triples where the predicate is true of those three things \emph{in that order}. So the extension of $T$ in this model will contain ordered triples like $<$2,4,8$>$, because $2\times 4 = 8$.

Generally, the extension of an n-place predicate is a set of all ordered n-tuples ${<}a_1, a_2,\ldots, a_n{>}$ such that $a_1$--$a_n$ are members of the UD and the predicate is true of $a_1$--$a_n$ in that order.


\subsection{Models for identity}
Identity is a special predicate of QL. We write it a bit differently than other two-place predicates: $x=y$ instead of $Ixy$. We also do not need to include it in a symbolization key. The sentence $x=y$ always means ``$x$ is identical to $y$,''  and it cannot be interpreted to mean anything else. In the same way, when you construct a model, you do not get to pick and choose which ordered pairs go into the extension of the identity predicate. It always contains just the ordered pair of each object in the UD with itself.

The sentence $\forall x Ixx$, which contains an ordinary two-place predicate, is contingent. Whether it is true for an interpretation depends on how you interpret $I$, and whether it is true in a model depends on the extension of $I$.

The sentence $\forall x\ x=x$ is a tautology. The extension of identity will always make it true.

Notice that although identity always has the same interpretation, it does not always have the same extension. The extension of identity depends on the UD. If the UD in a model is the set \{Doug\}, then $\extension{=}$ in that model is \{$<$Doug, Doug$>$\}. If the UD is the set \{Doug, Omar\}, then $\extension{=}$ in that model is \{$<$Doug, Doug$>$, $<$Omar, Omar$>$\}. And so on.

If the referent of two constants is the same, then anything which is true of one is true of the other. For example, if $\referent{a}=\referent{b}$, then $Aa\eiff Ab$, $Ba\eiff Bb$, $Ca\eiff Cb$, $Rca\eiff Rcb$, $\forall x Rxa\eiff \forall x Rxb$, and so on for any two sentences containing $a$ and $b$. In metaphysics, this is called principle of the indiscernibility of identicals

In our system, the reverse of this principle is not true.\label{model.nonidentity} It is possible that anything which is true of $a$ is also true of $b$, yet for $a$ and $b$ still to have different referents. This may seem puzzling, but it is easy to construct a model that shows this. Consider this model:
\begin{partialmodel}
UD & \{Rosencrantz, Guildenstern\}\\
\referent{a} & Rosencrantz\\
\referent{b} & Guildenstern\\
for all predicates \script{P}, \extension{\script{P}} & $\emptyset$\\
\extension{=} & \{$<$Rosencrantz, Rosencrantz$>$,\\
\multicolumn{2}{r}{$<$Guildenstern, Guildenstern$>$\}}
\end{partialmodel}
This specifies an extension for every predicate of QL: All the infinitely-many predicates are empty. This means that both $Aa$ and $Ab$ are false, and they are equivalent; both $Ba$ and $Bb$ are false; and so on for any two sentences that contain $a$ and $b$. Yet $a$ and $b$ refer to different things. We have written out the extension of identity to make this clear: The ordered pair $<\referent{a},\referent{b}>$ is not in it. In this model, $a=b$ is false and $a\neq b$ is true.

\practiceproblems


\problempart
\label{pr.TorF1}
Determine whether each sentence is true or false in the model given.
\begin{partialmodel}
UD & \{Corwin, Benedict\}\\
\extension{A} & \{Corwin, Benedict\}\\
\extension{B} & \{Benedict\}\\
\extension{N} & $\emptyset$\\
\referent{c} & Corwin
\end{partialmodel}
\begin{earg}
\item $Bc$
\item $Ac \eiff \enot Nc$
\item $Nc \eif (Ac \eor Bc)$
\item $\forall x Ax$
\item $\forall x \enot Bx$
\item $\exists x(Ax \eand Bx)$
\item $\exists x(Ax \eif Nx)$
\item $\forall x(Nx \eor \enot Nx)$
\item $\exists x Bx \eif \forall x Ax$
\end{earg}



\problempart
\label{pr.TorF2}
Determine whether each sentence is true or false in the model given.
\begin{partialmodel}
UD & \{Waylan, Willy, Johnny\}\\
\extension{H} & \{Waylan, Willy, Johnny\}\\
\extension{W} & \{Waylan, Willy\}\\
\extension{R} & \{$<$Waylan, Willy$>$,$<$Willy, Johnny$>$,$<$Johnny, Waylan$>$\}\\
\referent{m} & Johnny
\end{partialmodel}
\begin{earg}
\item $\exists x(Rxm \eand Rmx)$
\item $\forall x(Rxm \eor Rmx)$
\item $\forall x(Hx \eiff Wx)$
\item $\forall x(Rxm \eif Wx)$
\item $\forall x\bigl[Wx \eif(Hx \eand Wx)\bigr]$
\item $\exists x Rxx$
\item $\exists x\exists y Rxy$
\item $\forall x \forall y Rxy$
\item $\forall x \forall y (Rxy \eor Ryx)$
\item $\forall x \forall y \forall z\bigl[(Rxy \eand Ryz) \eif Rxz\bigr]$
\end{earg}

\problempart
\label{pr.TorF3}
Determine whether each sentence is true or false in the model given.
\begin{partialmodel}
	UD			& \{Lemmy, Courtney, Eddy\}\\
	\extension{G}	& \{Lemmy, Courtney, Eddy\}\\
	\extension{H}	& \{Courtney\}\\
	\extension{M}	& \{Lemmy, Eddy\}\\
	\referent{c}	& Courtney\\
	\referent{e}	& Eddy
\end{partialmodel}
\begin{earg}
\item $Hc$
\item $He$
\item $Mc \eor Me$
\item $Gc \eor \enot Gc$
\item $Mc \eif Gc$
\item $\exists x Hx$
\item $\forall x Hx$
\item $\exists x \enot Mx$
\item $\exists x(Hx \eand Gx)$
\item $\exists x(Mx \eand Gx)$
\item $\forall x(Hx \eor Mx)$
\item $\exists x Hx \eand \exists x Mx$
\item $\forall x(Hx \eiff \enot Mx)$
\item $\exists x Gx \eand \exists x \enot Gx$
\item $\forall x\exists y(Gx \eand Hy)$
\end{earg}


\problempart
\label{pr.InterpretationToModel}
Write out the model that corresponds to the interpretation given.
\begin{ekey}
\item{UD:} natural numbers from 10 to 13
\item{Ox:} $x$ is odd. 
\item{Sx:} $x$ is less than 7.
\item{Tx:} $x$ is a two-digit number.
\item{Ux:} $x$ is thought to be unlucky.
\item{Nxy:} $x$ is the next number after $y$.
\end{ekey}


% *****************************
% *		Working with Models.       *
% ****************************



\section{Working with Models}
\label{sec.UsingModels}

Working with models is in some ways like working with truth tables and in some ways not. With truth tables, we could conclusively show that a sentence was a tautology or a contradiction, because the truth table would always have a finite number of lines. We cannot, however, use models to show that a sentence is a tautology or a contradiction, because there are infinitely many ways to model a sentence, and and no single way to talk about all of them, the way we talked about all the lines in a truth table. One thing we can do is show conclusively that a sentence is neither a tautology nor a contradiction, and is instead contingent. A contingent sentence will have at least one model where it is false and one model where it is true.

As we shall see, this pattern plays itself out with the other logical properties we have covered. Because there are infinitely many ways to model a sentence, we cannot construct models to prove two sentences equivalent, but we can use them to show that two sentences are not equivalent. We cannot use models to show that a set of sentences is inconsistent, but we can use models to show that a set of sentences is consistent. Finally, we cannot construct a finite number of models to show that an argument is valid, but we can use one to show that an argument is invalid. 

We will use the double turnstile symbol for QL much as we did for SL. ``$\script{A}\sdtstile{}{} \script{B}$''' means that an argument from \script{A} to \script{B} is semantically valid.  $\sdtstile{}{} \script{A}$ means that \script{A} is a semantic tautology. $\script{A} \ndststile{}{} \hspace{.5em} \sdtstile{}{} \script{B}$ means \script{A} and \script{B} are semantically equivalent. 

In \fullref{semantic_definitions_in_SL} we stipulated semantic definitions for various logical concepts in SL that matched our truth table method for determining these concepts. So a sentence was said to be a tautology in SL if the column under its main connective contained only Ts. This was an alternative to saying the the truth table method was an imprecise way of getting at the ordinary language versions of these concepts. We will do something similar for the semantic definitions of logical notions in QL. 

\begin{enumerate}
\item A \define{semantic tautology in QL} is a sentence \script{A} that is true in every model; i.e.,  $\sdtstile{}{} \script{A}$.

\item A \define{semantic contradiction in QL} is a sentence \script{A} that is false in every model; i.e., $\sdtstile{}{} \enot\script{A}$.

\item A sentence is \define{semantically contingent in QL} if and only if it is neither a tautology nor a contradiction.

\item Two sentences \script{A} and \script{B} are \define{semantically equivalent in QL} if and only if they have the same truth value in every model.

\item The set $\{\script{A}_1,\script{A}_2,\script{A}_3,\cdots\}$ is \define{semantically consistent in QL} if and only if there is at least one model in which all of the sentences are true. The set is \define{semantically inconsistent in QL} if and if only there is no such model.

\item An argument `` $\script{P}_1, \script{P}_2, \cdots$, \therefore\ \script{C} '' is \define{semantically valid in QL} if and only if there is no model in which all of the premises are true and the conclusion is false; i.e., $\{\script{P}_1,\script{P}_2,\cdots\}\sdtstile{}{}\script{C}$. It is \define{semantically invalid in QL} otherwise.
\end{enumerate}


\subsection{Models to show contingency}

Suppose we want to show that $\forall xAxx \eif Bd$ is \emph{not} a tautology. This requires showing that the sentence is not true in every model; i.e., that it is false in some model. If we can provide just one model in which the sentence false, then we will have shown that the sentence is not a tautology.

What would such a model look like? In order for $\forall xAxx \eif Bd$ to be false, the antecedent ($\forall x Axx$) must be true, and the consequent ($Bd$) must be false.

To construct such a model, we start with a UD. It will be easier to specify extensions for predicates if we have a small UD, so start with a UD that has just one member. Formally, this single member might be anything, so let's just call it $\alpha.$ 

We want $\forall x Axx$ to be true, so we want all members of the UD to be paired with themselves in the extension of $A$; this means that the extension of $A$ must be \{$<\alpha,\alpha>$\}.

We want $Bd$ to be false, so the referent of $d$ must not be in the extension of $B$. We give $B$ an empty extension.

Since $\alpha$ is the only member of the UD, it must be the referent of $d$. The model we have constructed looks like this:
\begin{partialmodel}
	UD			& \{$\alpha$\}\\
	\extension{A} 	& \{$<\alpha, \alpha>$\}\\
	\extension{B}	& $\emptyset$\\
	\referent{d}	& $\alpha$
\end{partialmodel}

Strictly speaking, a model specifies an extension for \emph{every} predicate of QL and a referent for \emph{every} constant. As such, it is generally impossible to write down a complete model. That would require writing down infinitely many extensions and infinitely many referents. However, we do not need to consider every predicate in order to show that there are models in which $\forall xAxx \eif Bd$ is false. Predicates like $H$ and constants like $f_{13}$ make no difference to the truth or falsity of this sentence. It is enough to specify extensions for $A$ and $B$ and a referent for $d$, as we have done. This provides a \emph{partial model} in which the sentence is false.

Perhaps you are wondering: What is $\alpha$? What does the predicate $A$ mean in English? The partial model could correspond to an interpretation like this one:
\begin{ekey}
\item[UD:] Paris
\item[$Axy$:] $x$ is in the same country as $y$.
\item[$Bx$:] $x$ was founded in the 20th century.
\item[$d$:] the City of Lights
\end{ekey}

However, we don't have to say that this model corresponds to any particular interpretation of the sentence in English in order to know that the sentence $\forall xAxx \eif Bd$  is not a tautology. We could have made our one-object universe of discourse contain only Mahatma Gandhi, or a pebble on a beach in Africa, or the number 2. As long as the predicate and terms were given the right reference, the sentence would come out false. Thus in the future we can evalaute sentences and arguments using UDs with arbitrarily named elements, like $\alpha$, $\beta$, $\gamma$, etc. 

We use the same method to show that $\forall xAxx \eif Bd$ is not a contradiction. We need only specify a model in which $\forall xAxx \eif Bd$ is true; i.e., a model in which either $\forall x Axx$ is false or $Bd$ is true. Here is one such partial model:
\begin{partialmodel}
	UD			& \{$\alpha$\}\\
	\extension{A} 	& \{$<\alpha,\alpha>$\}\\
	\extension{B}	& \{$\alpha$\}\\
	\referent{d}	& $\alpha$
\end{partialmodel}

We have now shown that $\forall xAxx \eif Bd$ is neither a tautology nor a contradiction. By the definition of ``contingent in QL,'' this means that $\forall xAxx \eif Bd$ is contingent. In general, showing that a sentence is contingent will require two models: one in which the sentence is true and another in which the sentence is false.

Notice, however, that we cannot show that a sentence actually is a tautology or a contradiction using one or two models this way. For a sentence to be a tautology, it must be true in every possible model. Similarly, if a sentence is a contradiction, it is false in all possible models. But there are infinitely many possible models for any sentence, and we don't have any tools in this text that will let us reason about all of them at once. 

\subsection{Models to show non-equivalence}

Suppose we want to show that $\forall x Sx$ and $\exists x Sx$ are \emph{not} logically equivalent. We need to construct a model in which the two sentences have different truth values; we want one of them to be true and the other to be false. We start by specifying a UD. Again, we make the UD small so that we can specify extensions easily. We will need at least two members. Let the UD be \{$\alpha$, $\beta$\}. (If we chose a UD with only one member, the two sentences would end up with the same truth value. In order to see why, try constructing some partial models with one-member UDs.)

We can make $\exists x Sx$ true by including something in the extension of $S$, and we can make $\forall x Sx$ false by leaving something out of the extension of $S$. It does not matter which one we include and which one we leave out. Making $\alpha$ the only $S$, we get a partial model that looks like this:
\begin{partialmodel}
	UD			& \{$\alpha$, $\beta$\}\\
	\extension{S}	& \{$\alpha$\}
\end{partialmodel}
This partial model shows that the two sentences are \emph{not} logically equivalent.

Notice, though, that we cannot show that two sentences \emph{are} logically equivalent by simply producing a model. If we claim that two sentences are logically equivalent, we are once again making a claim about every possible model. 

\subsection{Models to show consistency}

Suppose I wanted to show that the set of sentences $\{\exists x Fx, \exists x \enot Fx, \forall x Gx\}$ is consistent. For this to be the case, we need at least one model where all three sentences are true. In this case that means having one object in our UD that is $F$ and one that is not $F$, and we need both of them to be $G$. This would do the trick.

\begin{partialmodel}
	UD				 & \{$\alpha$, $\beta$\}\\
	\extension{F}		& \{$\alpha$\} \\
	\extension{G}		& \{$\alpha$, $\beta$\} 
\end{partialmodel}

We cannot show a set of sentences to be inconsistent this way, because as before, that would mean creating an infinite number of models. 

\subsection{Models to show invalidity}

Back on p.~\pageref{surgeon3correct}, we said that this argument would be invalid in QL:
\begin{earg}
\item[] $(Rc \eand K_1c) \eand Tc$
\item[\therefore] $Tc \eand K_2c$
\end{earg}
In order to show that it is invalid, we need to show that there is some model in which the premises are true and the conclusion is false. We can construct such a model deliberately. Here is one way to do it:
\begin{partialmodel}
	UD			& \{$\alpha$\}\\
	\extension{T}	& \{$\alpha$\}\\
	\extension{K_1}	& \{$\alpha$\}\\
	\extension{K_2}	& $\emptyset$\\
	\extension{R}	& \{$\alpha$\}\\
	\referent{c}	& $\alpha$
\end{partialmodel}

As you have probably guessed, we cannot show an argument valid with models so simply.



%\begin{table}[t]
%\caption{It is relatively easy to answer a question if you can do it by constructing a model or two. It is much harder if you need to reason about all possible models. This table shows when constructing models is enough.}
%\label{table.ModelOrArgument}
%\begin{center}
%\begin{tabular*}{\textwidth}[t]{p{10em}p{10em}p{10em}}
%& {\centerline{YES}} & {\centerline{NO}}\\
%\cline{3-3}
%
%Is \script{A} a tautology? & {show that \script{A} must be true in any model} & \tablefbox{\emph{construct a model} in which \script{A} is false}\\
%\cline{3-3}
%
%Is \script{A} a contradiction? &  {show that \script{A} must be false in any model} & \tablefbox{\emph{construct a model} in which \script{A} is true}\\
%\cline{2-3}
%
%Is \script{A} contingent? & \tablefbox{\emph{construct two models}, one in which \script{A} is true and another in which \script{A} is false}\vline & {either show that \script{A} is a tautology or show that \script{A} is a contradiction}\\
%\cline{2-3}
%
%Are \script{A} and \script{B} equivalent? & {show that \script{A} and \script{B} must have the same truth value in any model} & \tablefbox{\emph{construct a model} in which \script{A} and \script{B} have different truth values}\\
%\cline{2-3}
%
%Is the set \model{A} consistent? & \tablefbox{\emph{construct a model} in which all the sentences in \model{A} are true} & {show that the sentences could not all be true in any model}\\
%\cline{2-3}
%
%Is the argument \mbox{`\script{P}, \therefore\ \script{C}'} valid? & {show that any model in which \script{P} is true must be a model in which \script{C} is true} & \tablefbox{\emph{construct a model} in which \script{P} is true and \script{C} is false}\\
%\cline{3-3}
%\end{tabular*}
%\end{center}
%\end{table}
%
%
%
%
%
%
%
%\subsection{Satisfaction}
%
%The formula $Px$ says, roughly, that $x$ is one of the $P$s. This cannot be quite right, however, because $x$ is a variable and not a constant. It does not name any particular member of the UD. Instead, its meaning in a sentence is determined by the quantifier that binds it. The variable $x$ must stand-in for every member of the UD in the sentence $\forall xPx$, but it only needs to stand-in for one member in $\exists xPx$. Since we want the definition of satisfaction to cover $Px$ without any quantifier whatsoever, we will start by saying how to interpret a free variable like the $x$ in $Px$.
%
%We do this by introducing a \emph{variable assignment}. Formally, this is a function that matches up each variable with a member of the UD. Call this function `a.' (The `a' is for `assignment', but this is not the same as the truth value assignment that we used in defining truth for SL.)
%
%The formula $Px$ is satisfied in a model \model{M} by a variable assignment $a$ if and only if $a(x)$, the object that $a$ assigns to $x$, is in the the extension of P in \model{M}.
%
%When is $\forall x Px$ satisfied? It is not enough if $Px$ is satisfied in \model{M} by $a$, because that just means that $a(x)$ is in \extension{P}. $\forall x Px$ requires that every other member of the UD be in \extension{P} as well.
%
%So we need another bit of technical notation: For any member $\Omega$ of the UD and any variable \script{x}, let $a[\Omega|\script{x}]$ be the variable assignment that assigns $\Omega$ to \script{x} but agrees with $a$ in all other respects. We have used $\Omega$, the Greek letter Omega, to underscore the fact that it is some member of the UD and not some symbol of QL. Suppose, for example, that the UD is presidents of the United States. The function $a[\mbox{Grover Cleveland}|x]$ assigns Grover Cleveland to the variable $x$, regardless of what $a$ assigns to $x$; for any other variable, $a[\mbox{Grover Cleveland}|x]$ agrees with $a$.
%
%We can now say concisely that $\forall x Px$ is satisfied in a model \model{M} by a variable assignment $a$ if and only if, for every object $\Omega$ in the UD of \model{M}, $Px$ is satisfied in \model{M} by $a[\Omega|x]$.
%
%You may worry that this is circular, because it gives the satisfaction conditions for the sentence $\forall x Px$ using the phrase `for every object.' However, it is important to remember the difference between a logical symbol like `$\forall$' and an English language word like `every.' The word is part of the metalanguage that we use in defining satisfaction conditions for object language sentences that contain the symbol.
%
%We can now give a general definition of satisfaction, extending from the cases we have already discussed. We define a function $s$ (for `satisfaction') in a model \model{M} such that for any wff \script{A} and variable assignment $a$, $s(\script{A}, a)=1$ if \script{A} is satisfied in \model{M} by $a$; otherwise $s(\script{A}, a)=0$.
%
%\begin{enumerate}
%\item If \script{A} is an atomic wff of the form $\script{P}\script{t}_1\ldots\script{t}_n$ and $\Omega_i$ is the object picked out by $t_i$, then
%\begin{displaymath}s(\script{A}, a) =
%	\left\{\begin{array}{ll}
%	1 & \mbox{if ${<}\Omega_1\ldots\Omega_n{>}$ is in \extension{\script{P}} in \model{M}},\\
%	0 & \mbox{otherwise.}
%	\end{array}\right.
%\end{displaymath}
%
%For each term $t_i$: If $t_i$ is a constant, then $\Omega_i = \referent{t_i}$. If $t_i$ is a variable, then $\Omega_i = a(t_i)$.
%
%\item If \script{A} is ${\enot}\script{B}$ for some wff \script{B}, then
%\begin{displaymath}s(\script{A}, a) =
%	\left\{\begin{array}{ll}
%	1 & \mbox{if $s(\script{B}, a) = 0$},\\
%	0 & \mbox{otherwise.}
%	\end{array}\right.
%\end{displaymath}
%
%\item If \script{A} is $(\script{B}\eand\script{C})$ for some wffs \script{B,C}, then
%\begin{displaymath}s(\script{A}, a) =
%	\left\{\begin{array}{ll}
%	1 & \mbox{if $s(\script{B}, a) = 1$ and $s(\script{C}, a) = 1$,}\\
%	0 & \mbox{otherwise.}
%	\end{array}\right.
%\end{displaymath}
%
%\item If \script{A} is $(\script{B}\eor\script{C})$ for some wffs \script{B,C}, then
%\begin{displaymath}s(\script{A}, a) =
%	\left\{\begin{array}{ll}
%	0 & \mbox{if $s(\script{B}, a) = 0$  and $s(\script{C}, a) = 0$,}\\
%	1 & \mbox{otherwise.}
%	\end{array}\right.
%\end{displaymath}
%
%\item If \script{A} is $(\script{B}\eif\script{C})$ for some wffs \script{B,C}, then
%\begin{displaymath}s(\script{A}, a) =
%	\left\{\begin{array}{ll}
%	0 & \mbox{if $s(\script{B}, a) = 1$ and $s(\script{C}, a) = 0$,}\\
%	1 & \mbox{otherwise.}
%	\end{array}\right.
%\end{displaymath}
%
%\item If \script{A} is $(\script{B}\eiff\script{C})$ for some sentences \script{B,C}, then
%\begin{displaymath}s(\script{A}, a) =
%	\left\{\begin{array}{ll}
%	1 & \mbox{if $s(\script{B}, a) = s(\script{C}, a)$},\\
%	0 & \mbox{otherwise.}
%	\end{array}\right.
%\end{displaymath}
%
%\item If \script{A} is $\forall\script{x} \script{B}$ for some wff \script{B} and some variable \script{x}, then
%\begin{displaymath}s(\script{A}, a) =
%	\left\{\begin{array}{ll}
%	1 & \mbox{if $s(\script{B}, a[\Omega|\script{x}])=1$ for every member $\Omega$ of the UD},\\
%	0 & \mbox{otherwise.}
%	\end{array}\right.
%\end{displaymath}
%
%\item If \script{A} is $\exists\script{x} \script{B}$ for some wff \script{B} and some variable \script{x}, then
%\begin{displaymath}s(\script{A}, a) =
%	\left\{\begin{array}{ll}
%	1 & \mbox{if $s(\script{B}, a[\Omega|\script{x}])=1$ for at least one member $\Omega$ of the UD},\\
%	0 & \mbox{otherwise.}
%	\end{array}\right.
%\end{displaymath}
%\end{enumerate}
% 
%This definition follows the same structure as the definition of a wff for QL, so we know that every wff of QL will be covered by this definition. For a model \model{M} and a variable assignment $a$, any wff will either be satisfied or not. No wffs are left out or assigned conflicting values.




%\section*{Summary of definitions}
%\begin{itemize}
%\item A \define{tautology in QL} is a sentence \script{A} that is true in every model; i.e.,  $\models\script{A}$.

%\item A \define{contradiction in QL} is a sentence \script{A} that is false in every model; i.e., $\models\enot\script{A}$.

%\item A sentence is \define{contingent in QL} if and only if it is neither a tautology nor a contradiction.

%\item An argument `` $\script{P}_1, \script{P}_2, \cdots$, \therefore\ \script{C} '' is \define{valid in QL} if and only if there is no model in which all of the premises are true and the conclusion is false; i.e., $\{\script{P}_1,\script{P}_2,\cdots\}\models\script{C}$. It is \define{invalid in QL} otherwise.

%\item Two sentences \script{A} and \script{B} are \define{logically equivalent in SL} if and only if both $\script{A}\models\script{B}$ and $\script{B}\models\script{A}$.

%\item The set $\{\script{A}_1,\script{A}_2,\script{A}_3,\cdots\}$ is \define{consistent in QL} if and only if there is at least one model in which all of the sentences are true. The set is \define{inconsistent in QL} if and if only there is no such model.
%\end{itemize}
\practiceproblems


\problempart
\label{pr.Contingent}
Show that each of the following is contingent.
\begin{earg}
\item \leftsolutions\ $Da \eand Db$
\item \leftsolutions\ $\exists x Txh$
\item \leftsolutions\ $Pm \eand \enot\forall x Px$
\item $\forall z Jz \eiff \exists y Jy$
\item $\forall x (Wxmn \eor \exists yLxy)$
\item $\exists x (Gx \eif \forall y My)$
\end{earg}


\problempart
\label{pr.NotEquiv}
Show that the following pairs of sentences are not logically equivalent.
\begin{earg}
\item $Ja$, $Ka$
\item $\exists x Jx$, $Jm$
\item $\forall x Rxx$, $\exists x Rxx$
\item $\exists x Px \eif Qc$, $\exists x (Px \eif Qc)$
\item $\forall x(Px \eif \enot Qx)$, $\exists x(Px \eand \enot Qx)$
\item $\exists x(Px \eand Qx)$, $\exists x(Px \eif Qx)$
\item $\forall x(Px\eif Qx)$, $\forall x(Px \eand Qx)$
\item $\forall x\exists y Rxy$, $\exists x\forall y Rxy$
\item $\forall x\exists y Rxy$, $\forall x\exists y Ryx$
\end{earg}



\problempart
Show that the following sets of sentences are consistent.
\begin{earg}
\item \{Ma, \enot Na, Pa, \enot Qa\}
\item \{$Lee$, $Lef$, $\enot Lfe$, $\enot Lff$\}
\item \{$\enot (Ma \eand \exists x Ax)$, $Ma \eor Fa$, $\forall x(Fx \eif Ax)$\}
\item \{$Ma \eor Mb$, $Ma \eif \forall x \enot Mx$\}
\item \{$\forall y Gy$, $\forall x (Gx \eif Hx)$, $\exists y \enot Iy$\}
\item \{$\exists x(Bx \eor Ax)$, $\forall x \enot Cx$, $\forall x\bigl[(Ax \eand Bx) \eif Cx\bigr]$\}
\item \{$\exists x Xx$, $\exists x Yx$, $\forall x(Xx \eiff \enot Yx)$\}
\item \{$\forall x(Px \eor Qx)$, $\exists x\enot(Qx \eand Px)$\}
\item \{$\exists z(Nz \eand Ozz)$, $\forall x\forall y(Oxy \eif Oyx)$\}
\item \{$\enot \exists x \forall y Rxy$, $\forall x \exists y Rxy$\}
\end{earg}


\problempart
Construct models to show that the following arguments are invalid.
\begin{enumerate}[label=\arabic*), topsep=0pt, parsep=0pt, itemsep=3pt]
\item $\forall x(Ax \eif Bx) \sdtstile{}{} \exists x Bx$
\item $\{\forall x(Rx \eif Dx)$, $\forall x(Rx \eif Fx)\} \sdtstile{}{} \exists x(Dx \eand Fx)$
\item $\exists x(Px\eif Qx) \sdtstile{}{} \exists x Px$
\item $Na \eand Nb \eand Nc \sdtstile{}{}  \forall x Nx$
\item $\{Rde$, $\exists x Rxd\} \sdtstile{}{} Red$
\item $\{\exists x(Ex \eand Fx)$, $\exists x Fx \eif \exists x Gx\} \sdtstile{}{} \exists x(Ex \eand Gx)$
\item $\{\forall x Oxc$, $\forall x Ocx\} \sdtstile{}{} \forall x Oxx$
\item $\{\exists x(Jx \eand Kx)$, $\exists x \enot Kx$, $\exists x \enot Jx\} \sdtstile{}{} \exists x(\enot Jx \eand \enot Kx)$
\item $\{Lab \eif \forall x Lxb$, $\exists x Lxb\} \sdtstile{}{} Lbb$
\end{enumerate}




\problempart
%problem using identity, with solutions
\label{pr.IdentityModels}
\begin{earg}
\item\leftsolutions\ Show that $\{{\enot}Raa, \forall x (x=a \eor Rxa)\}$
is consistent.
%There are many possible answers. Here is one:
%\begin{partialmodel}
%UD & \{Harry, Sally\}\\
%\extension{R} &\{$<$Sally, Harry$>$\}\\
%\referent{a} & Harry
%\end{partialmodel}
\item\leftsolutions\ Show that $\{\forall x\forall y\forall z(x=y \eor y=z \eor x=z),
\exists x\exists y\ x\neq y\}$ is consistent.
%There are no predicates or constants, so we only need to give a UD.
%Any UD with 2 members will do.
\item\leftsolutions\ Show that $\{\forall x\forall y\ x=y, \exists x\ x \neq a\}$ is inconsistent.
%We need to show that it is impossible to construct a model in which these are both true. Suppose $\exists x\ x \neq a\$ is true in a model. There is something in the universe of discourse that is \emph{not} the referent of $a$. So there are at least two things in the universe of discourse: \referent{a} and this other thing. Call this other thing \script{b}---we know $a \neq \script{b}$. But if $a \neq \script{b}$, then $\forall x\forall y\ x=y$ is false. So the first sentence must be false if the second sentence is true is true. As such, there is no model in which they are both true. Therefore, they are inconsistent.
\item Show that $\exists x (x = h \eand x = i)$ is contingent.
\item Show that \{$\exists x\exists y(Zx \eand Zy \eand x=y)$, $\enot Zd$, $d=s$\} is consistent.
\item Show that $\forall x(Dx \eif \exists y Tyx)$ \therefore\ $\exists y \exists z\ y\neq z$ is invalid.
\end{earg}




%\problempart
%\label{pr.SemanticsEssay}
%\begin{earg}
%\item Many logic books define consistency and inconsistency in this way:
%`` A set $\{\script{A}_1,\script{A}_2,\script{A}_3,\cdots\}$ is inconsistent if and only if $\{\script{A}_1,\script{A}_2,\script{A}_3,\cdots\}\models(\script{B}\eand\enot\script{B})$ for some sentence \script{B}. A set is consistent if it is not inconsistent.''
%
%Does this definition lead to any different sets being consistent than the definition on  p.~\pageref{def.consistencySL}? Explain your answer.
%
%\item\leftsolutions\ Our definition of truth says that a sentence \script{A} is \define{true in} \model{M} if and only if some variable assignment satisfies \script{A} in $M$. Would it make any difference if we said instead that \script{A} is \define{true in} \model{M} if and only if \emph{every} variable assignment satisfies \script{A} in $M$? Explain your answer.
%\end{earg}

%\chapter{Proofs in Quantified Logic}
\markright{Chap \ref{chap:proofsinQL}: Proofs in QL}
\label{chap:proofsinQL}
\setlength{\parindent}{1em}

% ******************************************
%  *                       6.1 Rules for Quantifiers             *
%******************************************

\section{Rules for Quantifiers}

For proofs in QL, we use all of the basic rules of SL plus four new basic rules: both introduction and elimination rules for each of the quantifiers.

Since all of the derived rules of SL are derived from the basic rules, they will also hold in QL. We will add another derived rule, a replacement rule called quantifier negation.

\subsection{Universal elimination}

If you have $\forall x Ax$, it is legitimate to infer that anything is an $A$. You can infer $Aa$, $Ab$, $Az$, $Ad_3$--- in short, you can infer $A\script{c}$ for any constant \script{c}. This is the general form of the universal elimination rule ($\forall$E):

\begin{proof}
	\have[m]{a}{\forall \script{x}\script{A}}
	\have[\ ]{c}{\script{A}[\script{c}|\script{x}]} \Ae{a}
\end{proof}

$\script{A}[\script{c}|\script{x}]$ is a substitution instance of $\forall\script{x}\script{A}$. The symbols for a substitution instance are not symbols of QL, so you cannot write them in a proof. Instead, you write the subsituted sentence with the constant \script{c} replacing all occurrences of the variable \script{x} in \script{A}. For example:

\begin{proof}
	\hypo{a}{\forall x(Mx \eif Rxd)}
	\have{c}{Ma \eif Rad} \Ae{a}
	\have{d}{Md \eif Rdd} \Ae{a}
\end{proof}


\subsection{Existential introduction}

When is it legitimate to infer $\exists x Ax$? If you know that something is an $A$--- for instance, if you have $Aa$ available in the proof.

This is the existential introduction rule ($\exists$I):

\begin{proof}
	\have[m]{a}{\script{A}}
	\have[\ ]{c}{\exists \script{x}\script{A}[\script{x}||\script{c}]} \Ei{a}
\end{proof}

It is important to notice that $\script{A}[\script{x}||\script{c}]$ is not the same as a substitution instance. We write it with two bars to show that the variable \script{x} does not need to replace all occurrences of the constant \script{c}. You can decide which occurrences to replace and which to leave in place. For example:

\begin{proof}
	\hypo{a}{Ma \eif Rad}
	\have{b}{\exists x(Ma \eif Rax)} \Ei{a}
	\have{c}{\exists x(Mx \eif Rxd)} \Ei{a}
	\have{d}{\exists x(Mx \eif Rad)} \Ei{a}
	\have{e}{\exists y\exists x(Mx \eif Ryd)} \Ei{d}
	\have{f}{\exists z\exists y\exists x(Mx \eif Ryz)} \Ei{e}
\end{proof}


\subsection{Universal introduction}
A universal claim like $\forall x Px$ would be proven if {every} substitution instance of it had been proven, if every sentence $Pa$, $Pb$, $\ldots$ were available in a proof. Alas, there is no hope of proving \emph{every} substitution instance. That would require proving $Pa$, $Pb$, $\ldots$, $Pj_2$, $\ldots$, $Ps_7$, $\ldots$, and so on to infinity. There are infinitely many constants in QL, and so this process would never come to an end.

Consider a simple argument: $\forall x Mx$, \therefore\ $\forall y My$

It makes no difference to the meaning of the sentence whether we use the variable $x$ or the variable $y$, so this argument is obviously valid. Suppose we begin in this way:

\begin{proof}
	\hypo{x}{\forall x Mx} \by{want $\forall y My$}{}
	\have{a}{Ma} \Ae{x}
\end{proof}

We have derived $Ma$. Nothing stops us from using the same justification to derive $Mb$, $\ldots$, $Mj_2$, $\ldots$, $Ms_7$, $\ldots$, and so on until we run out of space or patience. We have effectively shown the way to prove $M\script{c}$ for any constant \script{c}. From this, $\forall x Mx$ follows.

\begin{proof}
	\hypo{x}{\forall x Mx}
	\have{a}{Ma} \Ae{x}
	\have{y}{\forall y My} \Ai{a}
\end{proof}

It is important here that $a$ was just some arbitrary constant. We had not made any special assumptions about it. If $Ma$ were a premise of the argument, then this would not show anything about \emph{all} $y$. For example:

\begin{proof}
	\hypo{x}{\forall x Rxa}
	\have{a}{Raa} \Ae{x}
	\have{y}{\forall y Ryy} \by{not allowed!}{}
\end{proof}


This is the schematic form of the universal introduction rule ($\forall$I):

\begin{proof}
	\have[m]{a}{\script{A}}
	\have[\ ]{c}{\forall \script{x}\script{A}[\script{x}|\script{c}]^\ast} \Ai{a}
\end{proof}
$^\ast$ \script{c} must not occur in any undischarged assumptions.

Note that we can do this for any constant that does not occur in an undischarged assumption and for any variable.

Note also that the constant may not occur in any \emph{undischarged} assumption, but it may occur as the assumption of a subproof that we have already closed. For example, we can prove $\forall z(Dz \eif Dz)$ without any premises.

\begin{proof}
	\open
		\hypo{f1}{Df}\by{want $Df$}{}
		\have{f2}{Df}\by{R}{f1}
	\close
	\have{ff}{Df \eif Df}\ci{f1-f2}
	\have{zz}{\forall z(Dz \eif Dz)}\Ai{ff}
\end{proof}


\subsection{Existential elimination}
A sentence with an existential quantifier tells us that there is \emph{some} member of the UD that satisfies a formula. For example, $\exists x Sx$ tells us (roughly) that there is at least one $S$. It does not tell us \emph{which} member of the UD satisfies $S$, however. We cannot immediately conclude $Sa$, $Sf_{23}$, or any other substitution instance of the sentence. What can we do?

Suppose that we knew both $\exists x Sx$ and $\forall x(Sx \eif Tx)$. We could reason in this way:
\begin{quote}
Since $\exists x Sx$, there is something that is an $S$. We do not know which constants refer to this thing, if any do, so call this thing $\Omega$. From $\forall x(Sx \eif Tx)$, it follows that if $\Omega$ is an $S$, then it is a $T$. Therefore $\Omega$ is a $T$.  Because $\Omega$ is a $T$, we know that $\exists x Tx$.
\end{quote}
In this paragraph, we introduced a name for the thing that is an $S$. We called it $\Omega$, so that we could reason about it and derive some consequences from there being an $S$. Since $\Omega$ is just a bogus name introduced for the purpose of the proof and not a genuine constant, we could not mention it in the conclusion. Yet we could derive a sentence that does not mention $\Omega$; namely, $\exists x Tx$. This sentence does follow from the two premises.

We want the existential elimination rule to work in a similar way. Yet since Greek letters like $\Omega$ are not symbols of QL, we cannot use them in formal proofs. Instead, we will use constants of QL which do not otherwise appear in the proof. A constant that is used to stand in for whatever it is that satisfies an existential claim is called a \define{proxy}. Reasoning with the proxy must all occur inside a subproof, and the proxy cannot be a constant that is doing work elsewhere in the proof.

This is the schematic form of the existential elimination rule ($\exists$E): 

\begin{proof}
	\have[m]{a}{\exists \script{x}\script{A}}
	\open	
		\hypo[n]{b}{\script{A}[\script{c}|\script{x}]^\ast}
		\have[p]{c}{\script{B}}
	\close
	\have[\ ]{d}{\script{B}} \Ee{a,b-c}
\end{proof}
$^\ast$ The constant \script{c} must not appear in $\exists\script{x}\script{A}$, in \script{B}, or in any undischarged assumption.

With this rule, we can give a formal proof that $\exists x Sx$ and $\forall x(Sx \eif Tx)$ together entail $\exists x Tx$. The structure of the proof is effectively the same as the English-language argument with which we began, except that the subproof uses the constant ``$a$'' rather than the bogus name $\Omega$.

\begin{proof}
	\hypo{es}{\exists x Sx}
	\hypo{ast}{\forall x(Sx \eif Tx)}\by{want $\exists x Tx$}{}
	\open
		\hypo{s}{Sa}
		\have{st}{Sa \eif Ta}\Ae{ast}
		\have{t}{Ta} \ce{s,st}
		\have{et1}{\exists x Tx}\Ei{t}
	\close
	\have{et2}{\exists x Tx}\Ee{es,s-et1}
\end{proof}

\subsection{Quantifier negation}

When translating from English to QL, we noted that $\enot\exists x\enot\script{A}$ is logically equivalent to $\forall x\script{A}$. In QL, they are provably equivalent. We can prove one half of the equivalence with a rather gruesome proof:

\begin{proof}
	\hypo{Aa}{\forall x Ax} \by{want $\enot\exists x\enot Ax$}{}
	\open
		\hypo{Ena}{\exists\enot Ax}\by{for reductio}{}
		\open
			\hypo{nc}{\enot Ac}\by{for $\exists$E}{}
			\open
				\hypo{Aa2}{\forall x Ax}\by{for reductio}{}
				\have{c2}{Ac}\Ae{Aa}
				\have{nc2}{\enot Ac}\by{R}{nc}
			\close
			\have{nAa}{\enot\forall x Ax}\ni{Aa2-nc2}
		\close
		\have{Aa3}{\forall x Ax}\by{R}{Aa}
		\have{nAa3}{\enot\forall x Ax}\Ee{nc-nAa}
	\close
	\have{nEna}{\enot\exists\enot Ax}\ni{Ena-nAa3}
\end{proof}

In order to show that the two sentences are genuinely equivalent, we need a second proof that assumes $\enot\exists x\enot\script{A}$ and derives $\forall x\script{A}$. We leave that proof as an exercise for the reader.

It will often be useful to translate between quantifiers by adding or subtracting negations in this way, so we add two derived rules for this purpose. These rules are called quantifier negation (QN):
\begin{center}
\begin{tabular}{rl}
$\enot\forall\script{x}\script{A} \Longleftrightarrow \exists\script{x}\enot\script{A}$\\
$\enot\exists\script{x}\script{A} \Longleftrightarrow \forall\script{x}\enot\script{A}$
& QN
\end{tabular}
\end{center}
Since QN is a replacement rule, it can be used on whole sentences or on subformulae.

%%%%%%%%%   		Practice problems for Section 6.1                %%%%

% rob: I put all the quantification problems from the original chapter 6 that don't involve identity or models in this section. 

\practiceproblems

\setlength{\parindent}{0pt}

\problempart
\label{pr.justifyQLproof}
Provide a justification (rule and line numbers) for each line of proof that requires one.

\begin{enumerate}[label=\arabic*)]
\begin{multicols}{2}
\begin{minipage}{\linewidth}
\item \textcolor{white}{.} \\  %$\vdash \exists x Mx \eor \forall x\enot Mx$
\vspace{-24pt}
\begin{proof}
	\open
		\hypo{p1}{\enot (\exists x Mx \eor \forall x\enot Mx)}
		\have{p2}{\enot \exists x Mx \eand \enot \forall x\enot Mx}{}
		\have{p3}{\enot \exists x Mx}{}
		\have{p4}{\forall x\enot Mx}{}
		\have{p5}{\enot \forall x\enot Mx}{}
	\close
\have{n}{\exists x Mx \eor \forall x\enot Mx} {}
\end{proof}
\end{minipage}

\pagebreak[4]
\item \textcolor{white}{.} \\ %$\{\forall x(\exists y)(Rxy \eor Ryx),\forall x\enot Rmx\}\vdash\exists xRxm$
\vspace{-16pt}
\begin{proof}
\hypo{p1}{\forall x\exists y(Rxy \eor Ryx)}
\hypo{p2}{\forall x\enot Rmx}
\have{3}{\exists y(Rmy \eor Rym)}{}
	\open
		\hypo{a1}{Rma \eor Ram}
		\have{a2}{\enot Rma}{}
		\have{a3}{Ram}{}
		\have{a4}{\exists x Rxm}{}
	\close
\have{n}{\exists x Rxm} {}
\end{proof}

\vspace{2ex}

\item \textcolor{white}{.} \\ %$\{\forall x(\exists yLxy \eif \forall zLzx), Lab\} \vdash \forall xLxx$
\vspace{-16pt}
\begin{proof} 
\hypo{1}{\forall x(\exists yLxy \eif \forall zLzx)}
\hypo{2}{Lab}
\have{3}{\exists y Lay \eif \forall zLza}{}
\have{4}{\exists y Lay} {}
\have{5}{\forall z Lza} {}
\have{6}{Lca}{}
\have{7}{\exists y Lcy \eif \forall zLzc}{}
\have{8}{\exists y Lcy}{}
\have{9}{\forall z Lzc}{}
\have{10}{Lcc}{}
\have{11}{\forall x Lxx}{}
\end{proof}



\item \textcolor{white}{.} \\ % $\{\forall x(Jx \eif Kx), \exists x\forall y Lxy, \forall x Jx\} \vdash \exists x(Kx \eand Lxx)$
\vspace{-16pt}
\begin{proof}
\hypo{a}{\forall x(Jx \eif Kx)}
\hypo{b}{\exists x\forall y Lxy}
\hypo{c}{\forall x Jx}
\have{d}{Ja}{}
\have{e}{Ja \eif Ka}{}
\have{f}{Ka}{}
\open
	\hypo{2}{\forall y Lay}
	\have{3}{Laa}{}
	\have{4}{Ka \eand Laa}{}
	\have{5}{\exists x(Kx \eand Lxx)}{}
\close
\have{j}{\exists x(Kx \eand Lxx)}{}
\end{proof}
\end{multicols}
\end{enumerate}

\problempart Without using the QN rule, prove $\enot\exists x\enot\script{A} \sststile{}{} \forall x\script{A}$

\problempart
\label{pr.someQLproofs}
Provide a proof of each claim.
\begin{earg}
\item $\sststile{}{}  \forall x Fx \eor \enot \forall x Fx$
\item $\{\forall x(Mx \eiff Nx), Ma\eand\exists x Rxa\}\sststile{}{}  \exists x Nx$
\item $\{\forall x(\enot Mx \eor Ljx), \forall x(Bx\eif Ljx), \forall x(Mx\eor Bx)\}\sststile{}{}  \forall xLjx$
\item $\forall x(Cx \eand Dt)\sststile{}{}  \forall xCx \eand Dt$
\item $\exists x(Cx \eor Dt)\sststile{}{}  \exists x Cx \eor Dt$
\end{earg}

\problempart
%Provide a proof of the argument about Billy on p.~\pageref{surgeon2}.

In the previous chapter (p.~\pageref{surgeon2}), we gave the following example

	\begin{quote}
	
	
	The hospital will only hire a skilled surgeon. All surgeons are greedy. Billy is a surgeon, but is not skilled. Therefore, Billy is greedy, but the hospital will not hire him.
	
	\begin{ekey}
	\item[UD:] people
	\item[Gx:] $x$ is greedy.
	\item[Hx:] The hospital will hire $x$.
	\item[Rx:] $x$ is a surgeon.
	\item[Kx:] $x$ is skilled.
	\item[b:] Billy
	\end{ekey}
	
	\begin{earg}
	\label{surgeon2}
	\item[] $\forall x\bigl[\enot (Rx \eand Kx) \eif \enot Hx\bigr]$
	\item[] $\forall x(Rx \eif Gx)$
	\item[] $Rb \eand \enot Kb$
	\item[\therefore] $Gb \eand \enot Hb$
	\end{earg}
	
	\end{quote}

Prove the symbolized argument. 

\problempart \label{pr.BarbaraEtc.proof1} \iflabelexists{chap:catstatements}{On page \pageref{table:full_twentyfour} you were introduced to the twenty-four valid Aristotelian syllogisms, and on page \pageref{venn_proofs} you were able to show 15 of these valid using Venn diagrams.  Now that we have translated them into QL (see page \pageref{pr.BarbaraEtc}) we can actually prove all of them valid. In this section, you will prove the unconditional forms. I have omitted Datisi and Ferio because their proofs are trivial variations on Darii and Ferison.}{On page \pageref{pr.BarbaraEtc} you translated the basic categorical syllogisms studied by Aristotle and his followers into QL. Now you need to provide derivations for some of them.}

\begin{enumerate}[label=\arabic*), topsep=0pt, parsep=0pt, itemsep=3pt]
\item \textbf{Barbara:} All $B$s are $C$s. All $A$s are $B$s.
	\therefore\  All $A$s are $C$s.
\item \textbf{Baroco:} All $C$s are $B$s. Some $A$ is not $B$.
	\therefore\  Some $A$ is not $C$.
\item \textbf{Bocardo:} Some $B$ is not $C$. All $B$s are $A$s.
	\therefore\  Some $A$ is not $C$.
\item\textbf{Celantes:} No $B$s are $C$s. All $A$s are $B$s.
	\therefore\  No $C$s are $A$s.
\item\textbf{Celarent:} No $B$s are $C$s. All $A$s are $B$s.
	\therefore\  No $A$s are $C$s.
\item\textbf{Campestres:} All $C$s are $B$s. No $A$s are $B$s.
	\therefore\  No $A$s are $C$s.
\item\textbf{Cesare:} No $C$s are $B$s. All $A$s are $B$s.
	\therefore\  No $A$s are $C$s.
\item\textbf{Dabitis:} All $B$s are $C$s. Some $A$ is $B$.
	\therefore\  Some $C$ is $A$.
\item\textbf{Darii:} All $B$s are $C$s. Some $A$ is $B$.
	\therefore\  Some $A$ is $C$.
\item\textbf{Disamis:} Some $B$ is $C$. All $B$s are $A$s.
	\therefore\  Some $A$ is $C$.
\item\textbf{Ferison:} No $B$s are $C$s. Some $B$ is $A$.
	\therefore\  Some $A$ is not $C$.
\item\textbf{Festino:} No $C$s are $B$s. Some $A$ is $B$.
	\therefore\  Some $A$ is not $C$.
\item\textbf{Frisesomorum:} Some $B$ is $C$. No $A$s are $B$s.
	\therefore\  Some $C$ is not $A$.
\end{enumerate}


\problempart
\label{pr.BarbaraEtc.proof2}
Now prove the conditionally valid syllogisms using QL. Symbolize each of the following and add the additional assumptions ``There is an $A$'' and ``There is a $B$.'' Then prove that the supplemented arguments forms are valid in QL. Calemos and Cesaro have been skipped because they are trivial variations of Camestros and Celaront. 

\begin{enumerate}[label=\arabic*), topsep=0pt, parsep=0pt, itemsep=3pt]

\item\textbf{Barbari:} All $B$s are $C$s. All $A$s are $B$s.
	\therefore\  Some $A$ is $C$.
\item\textbf{Celaront:} No $B$s are $C$s. All $A$s are $B$s.
	\therefore\  Some $A$ is not $C$.
\item\textbf{Camestros:} All $C$s are $B$s. No $A$s are $B$s.
	\therefore\  Some $A$ is not $C$.
\item\textbf{Darapti:} All $A$s are $B$s. All $A$s are $C$s.
	\therefore\  Some $B$ is $C$.
\item\textbf{Felapton:} No $B$s are $C$s. All $A$s are $B$s.
	\therefore\  Some $A$ is not $C$.
\item\textbf{Baralipton:} All $B$s are $C$s. All $A$s are $B$s.
	\therefore\  Some $C$ is $A$.
\item\textbf{Fapesmo:} All $B$s are $C$s. No $A$s are $B$s.
	\therefore\  Some $C$ is not $A$.
\end{enumerate}



\problempart
Provide a proof of each claim.
\begin{enumerate}[label=\arabic*), topsep=0pt, parsep=0pt, itemsep=3pt]
\item $\forall x \forall y Gxy \sststile{}{} \exists x Gxx$
\item $\forall x \forall y (Gxy \eif Gyx) \sststile{}{} \forall x\forall y (Gxy \eiff Gyx)$
\item $\{\forall x(Ax\eif Bx), \exists x Ax\} \sststile{}{} \exists x Bx$
\item $\{Na \eif \forall x(Mx \eiff Ma), Ma, \enot Mb\}\sststile{}{} \enot Na$
\item $\sststile{}{}\forall z (Pz \eor \enot Pz)$
\item $\sststile{}{}\forall x Rxx\eif \exists x \exists y Rxy$
\item $\sststile{}{}\forall y \exists x (Qy \eif Qx)$
\end{enumerate}

\problempart
Show that each pair of sentences is provably equivalent.
\begin{enumerate}[label=\arabic*), topsep=0pt, parsep=0pt, itemsep=3pt]
\item $\forall x (Ax\eif \enot Bx) \nsststile{}{} \hspace{.5em} \sststile{}{}\enot\exists x(Ax \eand Bx)$
\item $\forall x (\enot Ax\eif Bd) \nsststile{}{} \hspace{.5em} \sststile{}{} \forall x Ax \eor Bd$
\item $\exists x Px \eif Qc \nsststile{}{} \hspace{.5em} \sststile{}{}\forall x (Px \eif Qc)$
%\item $Rca \eiff \forall x Rxa$, $\forall x(Rca \eiff Rxa)$  rob: I'm embarassed to say I can't solve this one. 
\end{enumerate}



\problempart
Show that each of the following is provably inconsistent.
\begin{enumerate}[label=\arabic*), topsep=0pt, parsep=0pt, itemsep=3pt]
\item \{$Sa\eif Tm$, $Tm \eif Sa$, $Tm \eand \enot Sa$\}
\item \{$\enot\exists x \exists y Lxy$, $Laa$\}
\item \{$\forall x(Px \eif Qx)$, $\forall z(Pz \eif Rz)$, $\forall y Py$, $\enot Qa \eand \enot Rb$\}
\end{enumerate}




\problempart
\label{pr.likes}
Write a symbolization key for the following argument, translate it, and prove it:
\begin{quote}
There is someone who likes everyone who likes everyone that he likes. Therefore, there is someone who likes himself.
\end{quote}



%\problempart
%Look back at Part \ref{pr.QLarguments} on p.~\pageref{pr.QLarguments}. For each argument: If it is valid in QL, give a proof. If it is invalid, construct a model to show that it is invalid.


\problempart
\label{pr.QLequivornot}
For each of the following pairs of sentences: If they are logically equivalent in QL, give proofs to show this. If they are not, construct a model to show this.
\begin{enumerate}[label=\arabic*), topsep=0pt, parsep=0pt, itemsep=3pt]
\item $\forall x Px \eif Qc \nsststile{}{} \hspace{.5em} \sststile{}{}\forall x (Px \eif Qc)$
\item $\forall x Px \eand Qc\nsststile{}{} \hspace{.5em} \sststile{}{}\forall x (Px \eand Qc)$
\item $Qc \eor \exists x Qx\nsststile{}{} \hspace{.5em} \sststile{}{}\exists x (Qc \eor Qx)$
\item $\forall x\forall y \forall z Bxyz\nsststile{}{} \hspace{.5em} \sststile{}{}\forall x Bxxx$
\item $\forall x\forall y Dxy\nsststile{}{} \hspace{.5em} \sststile{}{}\forall y\forall x Dxy$
\item $\exists x\forall y Dxy\nsststile{}{} \hspace{.5em} \sststile{}{}\forall y\exists x Dxy$
\end{enumerate}

\problempart
\label{pr.QLvalidornot}
For each of the following arguments: If it is valid in QL, give a proof. If it is invalid, construct a model to show that it is invalid.
\begin{enumerate}[label=\arabic*), topsep=0pt, parsep=0pt, itemsep=3pt]
\item $\forall x\exists y Rxy \sststile{}{} \exists y\forall x Rxy$
\item $\exists y\forall x Rxy \sststile{}{} \forall x\exists y Rxy$
\item $\exists x(Px \eand \enot Qx) \sststile{}{} \forall x(Px \eif \enot Qx)$
\item $\{\forall x(Sx \eif Ta)$, $Sd\} \sststile{}{}Ta$
\item $\{\forall x(Ax\eif Bx)$, $\forall x(Bx \eif Cx)\} \sststile{}{} \forall x(Ax \eif Cx)$
\item $\{\exists x(Dx \eor Ex)$, $\forall x(Dx \eif Fx)\} \sststile{}{} \exists x(Dx \eand Fx)$
\item $\forall x\forall y(Rxy \eor Ryx)\sststile{}{} Rjj$
\item $\exists x\exists y(Rxy \eor Ryx)\sststile{}{}Rjj$
\item $\{\forall x Px \eif \forall x Qx$, $\exists x \enot Px\}\sststile{}{}\exists x \enot Qx$
\item $\{\exists x Mx \eif \exists x Nx$, $\enot \exists x Nx\}\sststile{}{} \forall x \enot Mx$
\end{enumerate}


%%%%%%%%%%%%%%%%%%%%%%%%%%%%%%%%                         6.2 Rules for Identity %%%%%%%%%%%%%%%%%%%%%%%%%%%%%%%%

\section{Rules for Identity}
The identity predicate is not part of QL, but we add it when we need to symbolize certain sentences. For proofs involving identity, we add two rules of proof.

Suppose you know that many things that are true of $a$ are also true of $b$. For example: $Aa\eand Ab$, $Ba\eand Bb$, $\enot Ca\eand\enot Cb$, $Da\eand Db$, $\enot Ea\eand\enot Eb$, and so on. This would not be enough to justify the conclusion $a=b$. (See p.~\pageref{model.nonidentity}.) In general, there are no sentences that do not already contain the identity predicate that could justify the conclusion $a=b$. This means that the identity introduction rule will not justify $a=b$ or any other identity claim containing two different constants.

However, it is always true that $a=a$. In general, no premises are required in order to conclude that something is identical to itself. So this will be the identity introduction rule, abbreviated {=}I:

\begin{proof}
	\have[\ \,\,\,]{x}{\script{c}=\script{c}} \by{=I}{}
\end{proof}

Notice that the {=}I rule does not require referring to any prior lines of the proof. For any constant \script{c}, you can write $\script{c}=\script{c}$ on any point with only the {=}I rule as justification.

If you have shown that $a=b$, then anything that is true of $a$ must also be true of $b$. For any sentence with $a$ in it, you can replace some or all of the occurrences of $a$ with $b$ and produce an equivalent sentence. For example, if you already know $Raa$, then you are justified in concluding $Rab$, $Rba$, $Rbb$. Recall that $\script{A}[\script{a}||\script{b}]$ is the sentence produced by replacing \script{a} in \script{A} with \script{b}. This is not the same as a substitution instance, because \script{b} may replace some or all occurrences of \script{a}. The identity elimination rule ({=}E) justifies replacing terms with other terms that are identical to it:
\begin{proof}
	\have[m]{e}{\script{a}=\script{b}}
	\have[n]{a}{\script{A}}
	\have[\ ]{ea1}{\script{A}[\script{a}||\script{b}]} \by{=E}{e,a}
	\have[\ ]{ea2}{\script{A}[\script{b}||\script{a}]} \by{=E}{e,a}
\end{proof}

%The basic rules for conjunction can be valuable in a proof even if there are no conjunctions in any of the assumptions; the basic rules for disjunction can be used even if there are no disjunctions in any assumptions; and similarly for the other basic rules. The rules for identity are different, in that there must be an identity claim in some assumption in order for the rules to do any work. Other than the trivial identity that we can introduce with the {=}I rule


%do not apply we can now prove that identity is \emph{transitive}: If $a=b$ and $b=c$, then $a=c$. The proof proceeds in this way:
%\begin{proof}
%	\open
%		\hypo{p}{a=b \eand b=c}\by{want $a=c$}{}
%		\have{ab}{a=b}\ae{p}
%		\have{bc}{b=c}\ae{p}
%		\have{ac}{a=c}\by{{=}E}{ab,bc}
%	\close
%	\have{conc}{(a=b \eand b=c)\eif a=c} \ci{p-ac}
%\end{proof}


%As an example, consider this argument:
%\begin{quote}
%There is only one button in my pocket. There is a blue button in my pocket. Therefore, there is no button in my pocket that is not blue.
%\end{quote}
%We begin by defining a symbolization key:
%\begin{ekey}
%\item{UD:} buttons in my pocket
%\item{Bx:} $x$ is blue.
%\end{ekey}
%\begin{proof}
%	\hypo{one}{\forall x\forall y\ x=y}
%	\hypo{eb}{\exists x Bx} \by{want $\enot\exists x \enot Bx$}{}
%	\open
%		\hypo{be1}{Be}
%		\have{ef1}{e=f}\Ae{one}
%		\have{bf1}{Bf}\by{{=}E}{ef1,be1}
%	\close
%	\have{bf}{Bf}\Ee{eb,be1-bf1}
%	\have{ab}{\forall x Bx}\Ai{bf}
%	\have{nnab}{\enot\enot\forall x Bx}\by{DN}{ab}
%	\have{nenb}{\enot\exists x\enot Bx}\by{QN}{nnab}
%\end{proof}

To see the rules in action, consider this proof:
\begin{proof}
	\hypo{one}{\forall x\forall y\ x=y}
	\hypo{eb}{\exists x Bx}
	\hypo{Abnc}{\forall x(Bx \eif \enot Cx)}
		\by{want $\enot\exists x Cx$}{}
	\open
		\hypo{be1}{Be}
		\have{ef1}{\forall y\ e=y}\Ae{one}
		\have{ef2}{e=f}\Ae{ef1}
		\have{bf1}{Bf}\by{{=}E}{ef2,be1}
		\have{bnc1}{Bf\eif\enot Cf}\Ae{Abnc}
		\have{ncf1}{\enot Cf}\ce{bnc1,bf1}
	\close
	\have{cf}{\enot Cf}\Ee{eb,be1-ncf1}
	\have{Anc}{\forall x \enot Cx}\Ai{cf}
	\have{nEc}{\enot\exists x Cx}\by{QN}{Anc}
\end{proof}

%\section*{Summary of definitions}
%\begin{itemize}
%\item A sentence \script{A} is a \define{theorem} if and only if $\vdash\script{A}$.
%
%\item Two sentences \script{A} and \script{B} are \define{provably equivalent} if and only if $\script{A}\vdash\script{B}$ and $\script{B}\vdash\script{A}$.
%
%\item $\{\script{A}_1,\script{A}_2,\ldots\}$ is \define{provably inconsistent} if and only if, for some sentence \script{B}, $\{\script{A}_1,\script{A}_2,\ldots\}\vdash(\script{B} \eand \enot \script{B})$.
%\end{itemize}

\practiceproblems


\problempart
\label{pr.identity}
Provide a proof of each claim.
\begin{enumerate}[label=\arabic*), topsep=0pt, parsep=0pt, itemsep=3pt] 
\item $\{Pa \eor Qb, Qb \eif b=c, \enot Pa\}\sststile{}{} Qc$
\item $\{m=n \eor n=o, An\}\sststile{}{} Am \eor Ao$
\item $\{\forall x x=m, Rma\}\sststile{}{} \exists x Rxx$
\item $\enot \exists x x \neq m \sststile{}{} \forall x\forall y (Px \eif Py)$
\item $\forall x\forall y(Rxy \eif x=y)\sststile{}{} Rab \eif Rba$
\item $\{\exists x Jx, \exists x \enot Jx\}\sststile{}{} \exists x \exists y x\neq y$
%\item $\{\forall x(x=n \eiff Mx), \forall x(Ox \eand \eor Mx)\}\sststile{}{} On$
\item $\{\exists x Dx, \forall x(x=p \eiff Dx)\}\sststile{}{} Dp$
\item $\{\exists x [(Kx \eand Bx) \eand \forall y(Ky \eif x=y)], Kd\}\sststile{}{} Bd$
\item $\sststile{}{} Pa \eif \forall x(Px \eor x \neq a)$
\end{enumerate}











%
%\part{Critical Thinking} \label{part:CT}	
%\include{tex/ch12-whatiscriticalthinking}
%\include{tex/ch13-substitutes}
%\include{tex/ch14-incompletearguments}
%\include{tex/ch15-emotions}
%\include{tex/ch16-generalizations}
%\include{tex/ch17-analogy}
%\include{tex/ch18-sources}
%\include{tex/ch19-maps}
%\include{tex/ch20-practicalarguments}

%\part{Inductive and Scientific Reasoning}  \label{part:inductive_scientific}
%\include{tex/ch21-whatareinductionandscientificreasoning}
%\include{tex/ch22-inductioninscience}
%\include{tex/ch23-mills-methods}
%\include{tex/ch24-causation-explanation}
%\include{tex/ch25-Analogy-in-Science}
%\include{tex/ch26-experimental-methods}
%\include{tex/ch27-Association-Diagrams-Cross-Tabulations}
%\include{tex/ch28-Explanation-Building}
%\include{tex/ch29-Problems-In-Induction}

\appendix
\iflabelexists{part:formal_logic}{\include{tex/z-app-notation}}{}
%\include{app-solutions}

%Bibstuff
%If the {part:CT} label is found, LaTeX will typeset separate bibliographies for sample passages and logical sources.

\iflabelexists{part:CT}{%text for CT version}

\defbibnote{sample}{\textit{ \large  This bibliography includes all sources except for those that were used as examples for logical analysis, either in the main text or problem sets}}

\printbibliography [keyword=samplepassage, title=Bibliography of Sample Passages, prenote=sample, heading=bibnumbered] %for separate bibs sample passages and general citations

\printbibliography [notkeyword=samplepassage, title=General Bibliography, heading=bibnumbered] %for separate bibs sample passages and general citations

}% End CT version
{\printbibliography[heading=bibnumbered]} %single bib for non-CT version


%%The way I’ve set this up now is that there is one bib for sample passages and one bib for everything else. This means that it would not be possible to put one entry in both bibliographies. (This might be needed for Aristotle.) To do that, you will need to define a separate logicsource category


\setglossarysection{chapter}
\printglossaries

\iflabelexists{part:formal_logic}{\include{tex/zz-quickreference}}{}
\thispagestyle{empty}



%\begin{center}
%\sf
%This book is dedicated to the logical connective \emph{tonk}.
%[A] man who perfectly understood a just syllogism, without believing that the conclusion follows from the premises, would be a greater monster than a man born without hands or feet.
%---Thomas Reid %EIP 6.5 p. 632
%\end{center}

%\parbox{2.5 in}{This book is motivated by the thought textbooks cost too much, especially for 


\parbox{3 in}{
{\sf About the authors:}\\


\textbf{P.D. Magnus} is an associate professor of philosophy in Albany, New York. His primary research is in the philosophy of science, concerned especially with the underdetermination of theory by data.\\

\textbf{J. Robert Loftis} is an associate professor of philosophy at Lorain County Community College in Elyria, Ohio. He received his Ph.D. in philosophy from Northwestern University. \\

\textbf{Cathal Woods} is Batten Associate Professor of Philosophy at Virginia Wesleyan University. He received his Ph.D. in philosophy from The Ohio State University.

}
\vfill

	







\end{document}
